  In order to express the behaviour of a vibrating string in suitable form for 
  the selection chart we are building up, it is useful to gather up some 
  earlier results and write them slightly differently. We are interested in a 
  monofilament string of circular cross-section of diameter $d$ and length $L$, 
  under tension $T$ and made of material with Young's modulus $E$ and density 
  $\rho$. As we found in section 5.4.3, using an argument based on Rayleigh's 
  principle, the $n$th natural frequency $\omega_n=2\pi f_n$ satisfies 

  $$\omega_n^2 \approx \dfrac{T}{m} \left[ \dfrac{n\pi}{L} \right] ^2 + 
  \dfrac{EI}{m} \left[ \dfrac{n\pi}{L} \right] ^4 \tag{1}$$ 

  where $m=\pi d^2 \rho /4$ is the mass per unit length, and the second moment 
  of area $I=\pi d^4 /64$. 

  The second term on the right-hand side of eq. (1) arises from the non-zero 
  bending stiffness of the string. For realistic musical strings, the bending 
  stiffness effect is relatively weak so that the fundamental frequency (i.e. 
  $n = 1$) is always well approximated by neglecting this term: 

  $$f_1^2 = \dfrac{\omega_1^2}{4 \pi^2} \approx \dfrac{T}{m}\left[ 
  \dfrac{1}{2L} \right] ^2 . \tag{2}$$ 

  We can turn this round to give an expression for $T$ in terms of the (known) 
  fundamental frequency $f_1$: 

  $$T \approx 4m(Lf_1)^2 = \pi \rho d^2 (Lf_1)^2 . \tag{3}$$ 

  Another quantity of interest is the wave impedance of the string: 

  $$Z_0 = \sqrt{Tm} \approx \pi d^2 \rho L f_1 /2 . \tag{4}$$ 

  Notice that, for a material of given density, both these quantities are 
  expressed in terms of $d$ and the combined parameter $\gamma=Lf_1$, so eqs. 
  (3) and (4) allow us to plot lines of constant tension and lines of constant 
  impedance in the $\gamma-d$ plane. 

  The parameter $\gamma$ is in any case a natural one to bring in, since if 
  bending stiffness is ignored, $\gamma$ remains constant as a given string on 
  a guitar, say, is fingered in different positions. Using eq. (2), $\gamma 
  \approx c/2$ where $c$ is the wave speed on the string. For material of a 
  given density, its value determines the stress: 

  $$\sigma = 4 \rho \gamma^2. \tag{5}$$ 

  Ultimate failure of the string under increasing tension is determined by a 
  threshold value of stress, so it follows that this failure condition will 
  appear as a vertical line in the $\gamma-d$ plane. 