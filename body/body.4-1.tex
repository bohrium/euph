  So far we have talked a lot about vibrating structures, but surely music is 
  about sound, rather than vibration as such? It is time to think about sound 
  waves --- how they are made, and how they travel through the air and interact 
  with the surrounding environment. Sound waves consist of fluctuations of 
  pressure in the air. Usually, the fluctuation is a very small fraction of the 
  ambient atmospheric pressure, and that means that, as with small-amplitude 
  mechanical vibrations, most relevant aspects of musical sound can be 
  understood using linear theory. 

  As explained in the next link, linearised sound waves can be modelled with a 
  differential equation which is already familiar from earlier sections, called 
  the wave equation. We have seen the one-dimensional version of this equation 
  for the ideal stretched string (see section 3.1.1), and the two-dimensional 
  version for a stretched membrane (see section 3.6.1). Sound waves obey the 
  three-dimensional version: but the one-dimensional wave equation also has a 
  simple physical interpretation for sound waves. A plane wave is one in which 
  the air pressure only varies along one direction, such as the $x$ coordinate 
  direction. The molecules of air oscillate back and forth parallel to the 
  $x$-axis, and the wave travels in the $x$ direction. Figure 1 shows an 
  animation of the result, for a sinusoidal waveform. 

\moobeginvid\begin{tabular}{ccc} \vidframe{ 0.30 }{ vids/vid-becdf2fc-00.png }&\vidframe{ 0.30 }{ vids/vid-becdf2fc-01.png }&\vidframe{ 0.30 }{ vids/vid-becdf2fc-02.png } \end{tabular}\caption{Figure 1.  Animation of a plane sound wave}\mooendvideo

  The wave equation shows that sound waves travel through air (or through 
  water, or any other medium) at a certain speed, determined by the density, 
  atmospheric pressure and thermodynamic properties. The speed of sound in air 
  is somewhat dependent on the temperature and humidity, but under normal 
  conditions it is around 340 m/s. A familiar consequence of this relatively 
  slow speed is the delay between seeing a lightning flash and hearing the 
  thunder: the light from the flash travels at the speed of light, so fast as 
  to be virtually instantaneous in the context of watching a thunderstorm. 

  The speed of sound tells us a very useful thing straight away: it relates the 
  frequency of a sound wave to its wavelength, the distance between successive 
  positions where the pressure fluctuation achieves a maximum value in a 
  travelling sinusoidal wave. The relation is very simple: the wavelength (in 
  metres) is given by the speed of sound (in m/s) divided by the frequency (in 
  Hz). As we saw earlier, the lowest frequency humans perceive as a musical 
  pitch is around 20 Hz, and this corresponds to a wavelength of 17 m. At the 
  opposite extreme, the highest frequency we can hear is around 20 kHz (and you 
  need to be quite young to hear that high). This corresponds to a wavelength 
  1000 times shorter, 17 mm. Wavelength will turn out to be important in 
  several different ways. The simplest will explain the fact, as we will see 
  shortly, that this range of lengths encompasses the range of sizes of musical 
  wind instruments: organ pipes, flutes, trombones and so on. Most instruments 
  are concentrated in the middle of the range, just as most musical notes are 
  intermediate between the extremes of pitch that we can perceive. 

  As well as the plane wave, there is another simple solution to the wave 
  equation which gives a useful component for thinking about more complicated 
  sound fields later on. If we write the equation in spherical polar 
  coordinates and then look for a solution that only depends on the radial 
  distance $r$, we can obtain a spherically symmetric wave field: details are 
  given in the next link. This solution can be used to describe a wave being 
  sent out by a pulsating sphere, as illustrated in Fig.\ 2. The pressure 
  varies sinusoidally with $r$, and the amplitude dies away proportional to 
  $1/r$. 

\moobeginvid\begin{tabular}{ccc} \vidframe{ 0.30 }{ vids/vid-4128ba98-00.png }&\vidframe{ 0.30 }{ vids/vid-4128ba98-01.png }&\vidframe{ 0.30 }{ vids/vid-4128ba98-02.png } \end{tabular}\caption{Figure 2.  Animation of a spherical sound wave radiated by a pulsating sphere.}\mooendvideo

  This simple example allows a first glimpse of a characteristic feature of 
  many acoustical problems, which often contributes to making acoustics 
  inherently more complicated than the structural vibration we have talked 
  about up to now. It is all to do with length scales. There are three 
  different length scales of interest here: the radius $a$ of the pulsating 
  sphere, the wavelength $\lambda$ of the sound, and the distance $r$ to the 
  observer. (The Greek letter $\lambda$ is pronounced ``lambda''.) We learn 
  something important if we express both $a$ and $r$ as multiples of the 
  wavelength, or equivalently if we introduce non-dimensional ratios 
  $a/\lambda$ and $r/\lambda$. 

  These ratios, between them, allow the qualitative behaviour of the wave field 
  to be classified into different regimes. The idea is similar to the use of 
  the Reynolds number to classify qualitatively different regimes of fluid 
  flow. The ratio $a/\lambda$ is called the Helmholtz number, or sometimes the 
  compactness ratio. It describes in wavelength terms the size of the object 
  making the sound: our pulsating sphere, or the body of a violin, for example. 
  When the Helmholtz number is small, meaning that the sound source is much 
  smaller than the wavelength, there is a useful approximate way to understand 
  sound radiation: we will meet it in section 4.3. When the Helmholtz number is 
  moderate, so that the size of the source is comparable with the wavelength, 
  things get more complicated because new phenomena will come into play: we 
  will see examples shortly. When the Helmholtz number is very large, the 
  source is very large compared to the wavelength, and a different type of 
  approximate analysis becomes possible, which we will also meet in section 
  4.3. 

  The second non-dimensional ratio doesn't have a standard name, but it 
  describes the distance of the observer in wavelength terms. For many sound 
  fields the behaviour is different when the observer (or measuring microphone) 
  is in the far field, where this ratio is large, compared to the behaviour in 
  the near field, where the ratio is small. This is the case in the simple 
  sphere example we have considered here: the details are given in the previous 
  link. For a musical application of the idea, think about a violin. The 
  player's ears are very close to the vibrating body of the instrument: they 
  are in the near field. A listener in the concert hall audience is likely to 
  be many wavelengths distant, though: they are in the far field. When people 
  talk about a violin sounding different ``under the ear'' from ``the sound in 
  the hall'', this far field/near field distinction is a key factor. 

  Now, the important thing about these two non-dimensional ratios is that they 
  allow us to distinguish broad types of behaviour that are different. It is 
  not the precise numerical values of these numbers that matters, so much as 
  the distinction between ``very small'', ``moderate'' and ``very large''. So 
  far I have defined these numbers in a particular way, based on the wavelength 
  $\lambda$. But I have cheated a bit. When the mathematics of the approximate 
  solutions is worked out, the natural quantities that arise are based on 
  something called the wavenumber, usually called $k$. This is defined as the 
  inverse of the wavelength, except that a factor $2 \pi$ comes in: $k=2 \pi / 
  \lambda$. So the most familiar form of the Helmholtz number for the sphere 
  problem is written $ka$: if you like, you can think of it as the ratio of the 
  circumference of the sphere to the wavelength, because the factor $2 \pi$ 
  cancels out. I will shortly show some particular examples, and when it comes 
  to labelling those I will usually use values of $ka$ or $kr$ for the specific 
  values used in the calculations. 

  In the remainder of this section, we will introduce some important 
  qualitative aspects of sound waves and their interaction with mechanical 
  objects: the ideas of impedance, intensity, wave interference, diffraction 
  and shadowing. First, we look at the energy carried by a travelling sound 
  wave. As explained in the next link, the intensity of a sound wave is a 
  vector which is the product of the pressure and the particle velocity. It 
  describes the rate and direction at which acoustic energy crosses a unit area 
  of space. For both the plane wave and the spherical wave discussed above, the 
  intensity is given by the expression $\frac{1}{2} \frac{p^2}{Z}$ where $p$ is 
  the amplitude of the sinusoidal pressure, and $Z$ is a constant called the 
  characteristic impedance. For the spherical wave from the pulsating sphere, 
  we saw that pressure dies away with the inverse of distance $r$, so the 
  intensity satisfies the inverse square law, dying away proportional to 
  $1/r^2$. This is exactly what you would expect: there is no dissipation of 
  energy, so the total power crossing a spherical surface at any distance $r$ 
  must be the same. The area of the surface grows with $r^2$, so the intensity 
  must indeed decay like $1/r^2$. 

  Perhaps the most characteristic property of waves of any kind, including 
  sound waves, is called interference, or sometimes phase cancellation. The 
  idea of wave interference was first explored by Thomas Young, at the very 
  start of the 19th century. He demonstrated the key concept using water waves 
  in a ripple tank, reputedly after watching the patterns of ripples on the 
  pond in the garden of Emmanuel College, Cambridge. The pond is still there, 
  and Fig.\ 3 shows an interference pattern on this same pond. Young went on to 
  apply the idea to interference of light waves. 

  \fig{figs/fig-7ba39957.png}{\caption{Figure 3. Wave interference pattern on 
  the pond of Emmanuel College, Cambridge. Two trains of waves, coming from the 
  lower right and the upper right, interfere to produce the ``quilted'' pattern 
  where they flow through each other.}} 

  Suppose we have two identical sound sources, sitting side by side and 
  producing sound at the same frequency (for example, a pair of loudspeakers of 
  a stereo system). Provided linear theory applies, the sound waves generated 
  by the two sources each behave as if the other source was not there, and the 
  combined sound that you would hear, or pick up with a microphone at a 
  particular position, is simply the sum of the two. Now, if our two sound 
  sources were to produce identical, synchronised, sound waves, the result 
  should simply be the same sound wave with twice the amplitude. However, if 
  the sources were in opposite phase, the sound waves would be ``equal and 
  opposite'', and they would cancel each other out. If the match of the two 
  fields was perfect, the combination of two sounds would be silence! It is 
  hard to achieve perfect cancellation like this in practice, but still this 
  effect has some very important consequences for music and musical instruments 
  (and in many other situations involving acoustics). For example, it is how 
  ``noise-cancelling headphones'' work, often used for listening to music in 
  noisy environments such as on an aeroplane. 

  We can investigate this effect quantitatively by combining two of the 
  spherical sound sources just discussed. Figures 4 and 5 show the pressure 
  fields resulting from two pulsating spheres in close proximity, vibrating 
  with the same amplitudes either in phase (Fig.\ 4) or in opposite phases 
  (Fig.\ 5). The Helmholtz number based on the separation of the centres of the 
  spheres is 1.3 for this case. (Specifically, this Helmholtz number is defined 
  as $ka$, where $a$ is the separation of the centres and $k$ is the wavenumber 
  of the sound wave, introduced earlier and defined in terms of the wavelength 
  $\lambda$ by $k=2 \pi/\lambda$.) Figure 4 shows a pattern which, apart from 
  some near-field details, is virtually indistinguishable from that of a single 
  pulsating sphere. Figure 5 is quite different. The pressure is consistently 
  lower because of cancellation effects, and along the horizontal mid-line the 
  cancellation is perfect and the pressure remains zero. Sound radiation is 
  strongest in the vertical directions, either upwards or downwards, along the 
  line of centres of the two spheres. This pattern is called a dipole field, 
  and we will come back to it in section 4.3. 

\moobeginvid\begin{tabular}{ccc} \vidframe{ 0.30 }{ vids/vid-54662cbc-00.png }&\vidframe{ 0.30 }{ vids/vid-54662cbc-01.png }&\vidframe{ 0.30 }{ vids/vid-54662cbc-02.png } \end{tabular}\caption{Figure 4. The pressure field produced by two pulsating spheres with the same amplitude and phase, separated by a distance corresponding to a Helmholtz number 1.3.}\mooendvideo

\moobeginvid\begin{tabular}{ccc} \vidframe{ 0.30 }{ vids/vid-05960ecc-00.png }&\vidframe{ 0.30 }{ vids/vid-05960ecc-01.png }&\vidframe{ 0.30 }{ vids/vid-05960ecc-02.png } \end{tabular}\caption{Figure 5. The pressure field produced by two pulsating spheres with the same amplitude but opposite phase, separated by a distance corresponding to a Helmholtz number 1.3. The colour scale matches that of Fig. 3: the pressures here are consistently lower because of cancellation effects.}\mooendvideo

  Figures 6 and 7 show a similar comparison with a larger Helmholtz number of 
  21. The two spheres are now several wavelengths apart, and a complicated 
  interference pattern is seen. Bright-coloured spots occur wherever the path 
  length difference from the two spheres results in waves arriving in phase 
  from the two places. The two figures show very similar patterns, and there is 
  no significant difference in the pressure levels in the two cases. The 
  interference pattern seen in these examples could be thought of as an 
  acoustical analogue of Thomas Young's patterns of water ripples (like the 
  ``quilted'' pattern visible in Fig.\ 3), and of his famous \tt{}``two-slits'' 
  experiment\rm{} in optics. Light waves obey the same wave equation as sound 
  waves, although the wavelengths of visible light are much shorter than the 
  typical wavelengths of sound so that the equivalent of the Helmholtz number 
  is usually very large for macroscopic problems in optics. 

\moobeginvid\begin{tabular}{ccc} \vidframe{ 0.30 }{ vids/vid-12800553-00.png }&\vidframe{ 0.30 }{ vids/vid-12800553-01.png }&\vidframe{ 0.30 }{ vids/vid-12800553-02.png } \end{tabular}\caption{Figure 6. The pressure field produced by two pulsating spheres with the same amplitude and phase, separated by a distance corresponding to a Helmholtz number 21.}\mooendvideo

\moobeginvid\begin{tabular}{ccc} \vidframe{ 0.30 }{ vids/vid-6d784eff-00.png }&\vidframe{ 0.30 }{ vids/vid-6d784eff-01.png }&\vidframe{ 0.30 }{ vids/vid-6d784eff-02.png } \end{tabular}\caption{Figure 7. The pressure field produced by two pulsating spheres with the same amplitude but the opposite phase, separated by a distance corresponding to a Helmholtz number 21. This time the pressure level is similar to that in Fig. 5. The interference pattern is also similar, but the individual high spots are in different places.}\mooendvideo

  The significance of very large Helmholtz number is revealed by the next 
  example. This concerns the phenomena of diffraction, wave scattering and the 
  formation of shadows. We are very familiar with the fact that ``light travels 
  in straight lines'', so that if there is an opaque object in the way then the 
  light can't reach the other side, and the result is a shadow. But with sound 
  waves, things are different. You can often hear something when it is round a 
  corner, for example a radio playing in the next room. But in truth, the 
  distinction is not so clear. Optics and acoustics obey the same laws, and the 
  only difference comes from the Helmholtz number. At very high Helmholtz 
  number, sound waves start to behave much more like the familiar behaviour of 
  light, and they can be analysed using ``ray theory''. 

  We will not go into much detail here, but the key effects can be illustrated 
  by a textbook example which happens to have a closed-form mathematical answer 
  (see section 8.1 of Morse and Ingard [1] for the details). The problem 
  concerns the interaction between a plane sound wave and an infinitely long 
  rigid cylinder, lying perpendicular to the direction the wave is travelling. 
  The wave diffracts around the cylinder, with details that depend on the 
  Helmholtz number $ka$, where $a$ is the radius of the cylinder and $k$ is the 
  wavenumber of the sound wave as before. 

  The difference between the original plane wave and the actual wavefield, 
  including the diffraction effects, is called the scattered field. Figure 8 
  shows plots of how this scattered field behaves for a range of values of the 
  Helmholtz number. Each case is a polar plot, to show how the intensity of the 
  scattered field varies with angle. The red symbol marks the origin in each 
  case, and the radial distance from there to the blue curve is proportional to 
  the far-field intensity of the scattered field in the corresponding 
  direction. The scale factor is the same for all the plots, so that Fig.\ 8 
  gives an accurate impression of how the strength of scattering varies with 
  Helmholtz number. 

  \fig{figs/fig-be40be4a.png}{\caption{Figure 8. Scattering of sound by a rigid 
  cylinder. A plane sound wave comes in from the left, and impinges on the 
  cylinder. The sound field is disturbed by interaction with the cylinder, and 
  the difference between the actual field and the original plane wave is called 
  the \textbf{scattered field}. The difference may be positive or negative. The 
  square of this scattered field is shown as a polar plot, for 5 different 
  cases of the Helmholtz number. In each case, the radial distance from the red 
  symbol to the curve gives the far-field intensity of the scattered field in 
  the corresponding direction. The scale is the same in all cases.}} 

  For the lowest value of Helmholtz number shown in Fig.\ 8, and indeed for any 
  value lower than this, the scattering is weak, and confined to directions on 
  the same side of the cylinder as the incident wave. The wavelength is long 
  compared to the size of the cylinder, and the sound wave flows round the 
  cylinder and carries on essentially unchanged. A little of the sound energy 
  is reflected back, with a directional pattern forming a circle in the polar 
  plot. 

  As the Helmholtz number increases, the strength of scattering increases: more 
  energy is reflected, and also we begin to see increasingly strong effects on 
  the ``downstream'' side of the cylinder. By the final example plotted here, 
  that downstream effect dominates: this is because an acoustic shadow has 
  formed behind the cylinder. The scattered field serves to reduce the 
  amplitude of the original plane wave; in other words, it is quieter behind 
  the cylinder. If Helmholtz number had been increased still further, the 
  downstream lobe describing the shadow would get longer and narrower. The 
  scattered pattern on the ``upstream'' side would converge to a shape that can 
  be analysed as a special case (see [1]). This case can be easily visualised 
  from the optical analogy. We are shining a torch from a long way away, at a 
  cylinder with a mirrored surface. A tight shadow forms behind the cylinder, 
  while light that falls on the shiny surface is reflected in a spread of 
  directions governed by the simple laws of geometric optics. 

  For the intermediate values of Helmholtz number, shapes are seen in the polar 
  plots which would be hard to guess. The general pattern revealed by this 
  example is typical of many acoustical problems, either of scattering and 
  shadowing or of sound radiation by vibrating structures like the body of a 
  stringed instrument. The behaviour at very low and very high Helmholtz number 
  has a relatively simple pattern, which is sometimes amenable to analysis, at 
  least as an approximation. But the intermediate range where the wavelength of 
  sound is of the same order as the size of the object is more complicated. 
  Directional patterns appear with a pattern of lobes which varies rapidly with 
  frequency. It is rare that behaviour in this regime can be calculated 
  mathematically: either numerical computation or direct measurement is needed. 



  \sectionreferences{}[1] Philip M. Morse and K. Uno Ingard; ``Theoretical 
  acoustics'', McGraw-Hill (1968) 