  This chapter picks up the story of plucked strings. It starts from the 
  question of whether you can recognise which instrument is being played, and, 
  if so, what features of the sound might be important for deciding. Section 
  7.1 includes a set of sound demonstrations, in which the same single note is 
  played on several different plucked-string or struck-string instruments. 
  There are many factors contributing to the distinctive sounds of these 
  instruments, and some of them are examined in later sections of this chapter. 

  First, the instruments are fitted with a wide variety of strings: metal and 
  nylon, plain and over-wound. Section 7.2 addresses the question of how to 
  choose strings for an instrument. The options and constraints on that choice 
  are set out in a systematic way, which gives strong clues about at least some 
  of the reasons that steel strings sound different from nylon ones, and even 
  why different choices of nylon string can sound different (for example the 
  difference between guitar and harp strings). 

  Another difference between the instruments contrasted in section 7.1 is 
  examined in Section 7.3. Some of them use more than one string for single 
  notes of an instrument, as is done in instruments as disparate as the piano, 
  the harpsichord, the lute and the 12-string guitar. The consequences of using 
  multiple strings (often called courses) are examined. Among other things, 
  this reveals something about the options available to a skilled piano tuner 
  to change the sound of a note by subtle adjustments. 

  Another possible reason for perceptible differences of sounds between the 
  instruments presented in section 7.1 is that in some cases nonlinear effects 
  begin to appear. Two particular cases are examined in section 7.4, relating 
  to the harp and to the lute. This serves to prepare the ground for the next 
  few chapters, in which nonlinear effects are examined in some detail. 

