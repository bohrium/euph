  For an ideal flexible string, obeying the wave equation as derived in section 
  3.1.1, the equations governing the motion in response to bowing at a single 
  point take a rather simple form. If the ends of the string are either rigid, 
  or terminated by dashpots, the reflection functions used in the 
  travelling-wave model derived in section 9.2.1 take the form of Dirac delta 
  functions. There is no ``corner-rounding''. For a rigid end, the delta 
  function has amplitude $-1$, whereas for a dashpot termination it has a 
  negative magnitude $\lambda$ with $|\lambda| < 1$. The reflection coefficient 
  $\lambda$ is determined by the impedance ratio of the string and the dashpot 
  as we showed in equation (5) of section 5.1.2: provided the impedance ratio 
  $Z_{0}/Z_{body} \ll 1$, $\lambda \approx -1 + 2Z_{0}/Z_{body}$. 

  Now an important simplification occurs if the bowing position $\beta$ is a 
  ratio $p/q$ of two integers. We can define a time interval $\Delta=T/(p+q)$ 
  where $T$ is the round-trip period for transverse waves on the string. 
  Reflections only arrive back at the bowed point at times that are multiples 
  of $\Delta$, so both the force and velocity at the bowed point can only 
  change at times $t=n\Delta$. In other words, provided the initial conditions 
  are consistent with this pattern, both the force and velocity waveforms 
  consist of steps of length $\Delta$ during which they stay constant, 
  separated by jumps. 

  We can define these constant levels to be $v_n=v(n \Delta)$ and $f_n=f(n 
  \Delta)$. The model development then follows the general treatment in section 
  9.2.1, but the convolution operations become very simple because both 
  reflection functions are delta functions. If for simplicity we assume that 
  both ends of the string have the same value of $\lambda$, then for example 
  $(r_1 * f)_n = \lambda f_{n-p}$ and $(r_2 * f)_n = \lambda f_{n-q}$. After a 
  little algebraic manipulation, the resulting equation can be cast into a 
  difference equation 

  \begin{equation*}v_n -- \lambda^2 v_{n-N} = \dfrac{1}{2Z_0} \left[f_n + 
  \lambda f_{n-p} + \lambda f_{n-q} + \lambda^2 f_{n-N} \right] 
  \tag{1}\end{equation*} 

  \noindent{}where $N=p+q$. As usual, for each successive value of $n$ this 
  equation is to be solved simultaneously with 

  \begin{equation*}f_n = F_b \mu(v_n) \tag{2}\end{equation*} 

  \noindent{}where $F_b$ is the bow force and $\mu(v)$ is the friction 
  coefficient. 

  The simplicity of this model allows various mathematical investigations to be 
  carried out relatively straightforwardly. Raman was able to solve the model 
  for simple cases of periodic solutions, working by hand in the pre-computer 
  era [1]. Friedlander was able to study the stability of periodic solutions 
  [2]. For the case without losses, $\lambda =-1$, he proved a striking result. 
  He first imposed the condition that the state of steady sliding is unstable, 
  based on the observation that it is virtually impossible to draw a bow across 
  a string without some vibration occurring. He showed that this condition is 
  enough to make all periodic solutions unstable! However, when $|\lambda| < 1$ 
  so that there is some energy dissipation, it becomes possible for periodic 
  solutions like Helmholtz motion to be stable while steady sliding is still 
  unstable [3]. 

  We can illustrate some simple periodic solutions. Suppose we look for 
  solutions with the natural period of the string, so that $v_{n+N}=v_n$ and 
  $f_{n+N}=f_n$ for all values of $n$. Equation (1) then becomes 

  \begin{equation*}2 Z_0 (1 -- \lambda^2) v_n = (1+ \lambda^2)f_n + \lambda 
  f_{n-p} + \lambda f_{n-q} \tag{3}\end{equation*} 

  \noindent{}except that when $n-p$ or $n-q$ go outside the range $1,\cdots,N$ 
  they are folded back into that range by adding or subtracting a suitable 
  multiple of $N$. The result is a set of $N$ simultaneous equations that can 
  be written in the form 

  \begin{equation*}2 Z_0 (1 -- \lambda^2) \mathbf{v} = M \mathbf{f} 
  \tag{4}\end{equation*} 

  \noindent{}where $\mathbf{v}$ and $\mathbf{f}$ are length-$N$ vectors 
  containing the values of $v_n$ and $f_n$, and the matrix $M$ is given by 

  \begin{equation*}M=\begin{bmatrix}1+\lambda^2 \& 0 \& \cdots \& \lambda \& 
  \cdots \& \lambda \& 0 \& \cdots \& 0\\0 \& 1+\lambda^2 \& 0 \& \cdots \& 
  \lambda \& \cdots \& \lambda \& 0 \& \cdots \\ \vdots \& \vdots \& \vdots \& 
  \vdots \& \vdots \& \vdots \& \vdots \& \vdots \& \vdots\\ \vdots \& \vdots 
  \& \vdots \& \vdots \& \vdots \& \vdots \& \vdots \& \vdots \& \vdots 
  \\\cdots \& 0 \& \lambda \& \cdots \& \lambda \& \cdots \& 0 \& 1+\lambda^2 
  \& 0\\0 \& \cdots \& 0 \& \lambda \& \cdots \& \lambda \& \cdots \& 0 \& 
  1+\lambda^2 \end{bmatrix} \tag{5}\end{equation*} 

  \noindent{}where the diagonal bands containing $\lambda$ are spaced $p$ and 
  $q$ places above and below the leading diagonal. 

  During an interval of sticking, we know that $v_n=v_b$, the bow speed, while 
  the corresponding $f_n$ is unknown. The situation is reversed during an 
  interval of slipping: now $v_n$ is unknown, but $f_n$ is determined by the 
  assumed friction curve. Under the simplest assumption in which the 
  coefficient of sliding friction is a constant value $\mu_d$, we have $f_n=F_b 
  \mu_d$ where $F_b$ is the bow force. So to find solutions to equation (4), we 
  decide which values of $n$ correspond to sticking, and which to slipping. Let 
  us seek solutions looking like Helmholtz motion: in that case, we can say 
  that for the first $q$ values we are sticking, while for the remaining $p$ 
  values we are slipping. Now we partition the vectors and the matrix in this 
  $q:p$ ratio: 

  \begin{equation*}\mathbf{v}=\begin{bmatrix} \mathbf{v_1}\\ 
  \mathbf{v_2}\end{bmatrix},~~~\mathbf{f}=\begin{bmatrix} \mathbf{f_1}\\ 
  \mathbf{f_2}\end{bmatrix}, M=\begin{bmatrix} A \& B \\ C \& D\end{bmatrix} 
  \tag{6}\end{equation*} 

  \noindent{}where $\mathbf{v_1}$ contains $q$ repeats of the value $v_b$, 
  $\mathbf{v_2}$ contains the unknown velocities, $\mathbf{f_1}$ contains the 
  unknown forces, and $\mathbf{f_2}$ contains $p$ repeats of the value $F_b 
  \mu_d$. 

  Expanding the partitioned problem out we obtain 

  \begin{equation*}2 Z_0 (1 -- \lambda^2) \mathbf{v_1}=A 
  \mathbf{f_1}+B\mathbf{f_2} \tag{7}\end{equation*} 

  \noindent{}and 

  \begin{equation*}2 Z_0 (1 -- \lambda^2) \mathbf{v_2}=C 
  \mathbf{f_1}+D\mathbf{f_2} . \tag{8}\end{equation*} 

  Thus the unknown values are given by 

  \begin{equation*}\mathbf{f_1}=A^{-1} \left[ 2 Z_0 (1 -- \lambda^2) 
  \mathbf{v_1} -- B\mathbf{f_2} \right] \tag{9}\end{equation*} 

  \noindent{}and then 

  \begin{equation*}\mathbf{v_2}=\dfrac{1}{2 Z_0 (1 -- \lambda^2)} \left[C 
  \mathbf{f_1}+D\mathbf{f_2} \right] . \tag{10}\end{equation*} 

  Figure 1 shows some computed examples. The first plot shows the force and 
  velocity waveforms for the case $p=1$, $q=12$. Cases like this, where 
  $\beta=1/(q+1)$ for some integer $q$, give the simplest possible examples. 
  Raman and Guettler both made extensive use of such cases in their work. In 
  order to show the effect of energy dissipation clearly, the rather 
  unrealistic value $\lambda=-0.97$ has been chosen. The force waveform (in 
  blue) shows the expected ``staircase'' structure, in an arch shape which is 
  familiar to us from the discussion of the minimum bow force in section 9.3.1. 
  The largest value of friction force occurs in the middle of the arch, and the 
  condition that this should not be big enough to cause a second slip at that 
  moment is what led Raman, and later Schelleng, to the criterion for minimum 
  bow force. 

  \fig{figs/fig-a0d7b4ec.png}{} 

  \fig{figs/fig-5a333df2.png}{} 

  \fig{figs/fig-6059b8c8.png}{} 

  \fig{figs/fig-2b4e2df7.png}{} 

  The other plots in Fig.\ 1 show cases with very similar values of $\beta$, 
  but involving larger numbers of steps in the staircase waveforms. The top 
  right plot shows $p=3$, $q=37$. There are now three steps in the slipping 
  interval, and the force waveform is a little more complicated, although still 
  showing the underlying arch shape clearly. The two plots in the bottom row 
  show values of $\beta$ on either side of the value 1/13. It can be seen that 
  the small changes in the slipping velocity waveform behave in opposite ways 
  in these two cases. 

  \sectionreferences{}[1] C. V. Raman: ``On the mechanical theory of the 
  vibrations of bowed strings and of musical instruments of the violin family, 
  with experimental verification of the results. Part I.'' Indian Association 
  for the Cultivation of Science Bulletin \textbf{15}, 1–158 (1918). 

  [2] F. G. Friedlander: ``On the oscillations of a bowed string'', Proceedings 
  of the Cambridge Philosophical Society, \textbf{49}, 516--530 (1953). 

  [3] J. Woodhouse: ``On the stability of bowed-string motion''; Acustica, 
  \textbf{80}, 58--72 (1994). 