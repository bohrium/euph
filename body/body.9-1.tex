  In a letter to Robert Hooke written in 1675, Isaac Newton famously wrote ``If 
  I have seen further it is by standing on the shoulders of giants.'' This 
  dictum is particularly apposite for the study of the motion of a bowed 
  string: the subject got under way with two of the giants of 19th and early 
  20th century science. The first of these giants is Hermann von Helmholtz 
  (1821—1894). Helmholtz made significant contributions to several scientific 
  fields ranging from physics to physiology: at different times in his academic 
  career he held professorships in both disciplines. 

  In 1862 Helmholtz wrote a book, later translated into English as “The 
  sensations of tone as a physiological basis for the theory of music” [1]. The 
  book ranges over what we would today regard as many different disciplines: 
  musicology, perception and psychoacoustics, the physiology of hearing, but 
  also mathematical theories of vibration and waves, and experimental work on 
  vibration and acoustics. In among all this, he describes an ingenious 
  experiment to observe the motion of a bowed violin string. This led him to 
  discover and describe mathematically the unexpected way that a bowed string 
  normally vibrates; something now known as “Helmholtz motion” in his honour. 
  The details appear in one of 19 Appendices to the book, which cover topics as 
  diverse as the constructional details for what we now call Helmholtz 
  resonators (see section 4.2.1), experiments on the production of vowel sounds 
  (see section 5.3), the mathematical theory of the motion of plucked strings 
  and struck piano strings, and a design of a keyboard instrument able to play 
  in “just intonation”. 

  Helmholtz’s experimental setup made use of his own improvement to the 
  “vibration microscope” first developed by Lissajous. This involved a lens 
  system, attached to the vibrating tine of an electrically-maintained tuning 
  fork. The violin string was blackened, and then a bright white spot of starch 
  was added at the point to be observed. The string was bowed, and arranged so 
  that the vibration was perpendicular to the direction of motion of the lens. 
  By synchronising the pitches of the bowed string and the tuning fork, and 
  then observing the path traced by the white dot when seen through the 
  vibrating lens, a closed loop called a “\tt{}Lissajous figure\rm{}” is seen. 
  A bit of careful thought can then reveal how the marked point on the string 
  must be vibrating. 

  What Helmholtz discovered is something really surprising. When you see a 
  bowed string vibrating, you see a fuzzy patch which is lens-shaped. You might 
  well imagine that the string simply vibrates from side to side with that 
  smooth arc-like shape. But Helmholtz found that the string is actually moving 
  in the counterintuitive way illustrated in the top animation of Fig.\ 2. At 
  any given moment, the string has a triangular shape: two straight portions 
  separated by the “Helmholtz corner”. This corner shuttles back and forth 
  between the bridge and the player’s finger. You can see a slow-motion movie 
  of Helmholtz motion on a real bowed string in \tt{}this YouTube video.\rm{} 

  If you look carefully in Fig.\ 2 at the point where the string crosses the 
  moving bow, you can see that all the time the corner is making the long 
  journey to the finger and back, the string is moving at the same speed as the 
  bow. In other words, during that part of each cycle of vibration there is 
  sticking friction between the bowhair and the string. But when the corner 
  travels to the bridge and back, the string is slipping across the bow hairs: 
  slipping friction. So this motion of the string is a stick-slip vibration, 
  and the frequency is governed by the time-keeping role of the travelling 
  corner. Since that corner travels at the normal wave speed on the string, 
  this explains why the frequency of a bowed note is the same as the frequency 
  of a plucked note on the same string. 

  We will see in later sections that many aspects of the description just given 
  are not exactly correct. There are small but important differences between 
  real string motion and the idealised form described by Helmholtz. The corner 
  is not really perfectly sharp; the bow does not really act at a single point 
  on the string because the ribbon of hair on a violin bow is a few millimetres 
  wide; the frequency of a bowed note is not always exactly the same as that of 
  the plucked string. But we will also see that Helmholtz motion gives an 
  excellent first approximation to what real bowed strings do. 

  The key to testing that claim lies in the lower panel of the animation. 
  Fortunately, modern electronic and computer-based measurement methods mean 
  that we can observe string motion in a less complicated way than with 
  Helmholtz’s vibration microscope. One way to do this is to embed a small 
  force-measuring sensor in the top of the violin bridge, at the point where 
  the string sits: the next link explains how that works. Such a sensor doesn’t 
  interfere (much) with the player’s ability to perform in a normal way, so it 
  is possible to capture waveforms of ``bridge force''. 

  This bridge force is a particularly useful thing to capture, because it 
  provides the excitation for vibration of the violin body. It gives the 
  clearest view of what the string is doing, while also allowing us to predict 
  the vibration and sound radiation of the instrument by making use of the kind 
  of frequency response functions we looked at in earlier chapters (see section 
  6.5 for some examples of such predictions). We will see many examples of 
  bridge force waveforms in the course of this chapter. The signature of 
  Helmholtz motion in such a measurement is a sawtooth waveform, as shown in 
  the lower panel of Fig.\ 2. The force ramps steadily upwards (or downwards if 
  the direction of bowing is reversed), then it jumps abruptly down (or up) at 
  the moment when the Helmholtz corner reflects from the bridge. 

  A measured example is shown in Fig.\ 3. The sawtooth shape is clearly 
  recognisable, but we can see some details that are different from the 
  idealised version. There are small wiggles during the ``ramp'' phase of each 
  cycle, and if you look very carefully you can see that these wiggles are not 
  exactly the same in every cycle, although they are very similar. We will find 
  out in section 9.2 what causes these wiggles, and also about other details of 
  the string motion, and what the player can do to influence them. 

  In the decades following Helmholtz’s book, there was a surprising amount of 
  published research on the vibration of violin strings. We get a clue about 
  why this apparently obscure topic was being studied, from a story about Lord 
  Rayleigh (1842—1919), who overlapped for much of his career with Helmholtz. 
  According to a biography written by his son [2], Rayleigh sometimes worried 
  that he might run out of physics problems of the kind he liked to study. It 
  seems extraordinary to us today that anyone in 1900 or so might have thought 
  that “physics was nearly all done”, because we know that just a few years 
  later the revolutionary developments of relativity and quantum mechanics 
  would open up new vistas of unimagined science. But, of course, they didn’t 
  know that at the time. 

  We can try to glimpse the mindset of a 19th century scientist. We get an 
  interesting hint from Rayleigh’s classic book “The theory of sound” [3]. 
  Volume 1, dealing with mechanical vibration, came out in 1877. Rayleigh 
  summarised the state of knowledge in his day, including a lot of his own 
  work. Mechanics was a high-profile science in those days, as is clear from 
  Rayleigh’s discussion and his footnote references. All the big names of 
  European science come into the discussion: Laplace, Lagrange, Euler, 
  Kirchhoff. But then Rayleigh produced a second edition of the book in 1894, 
  and the material he added is very revealing. As well as new material on 
  mechanics (such as shell vibration: recall section 3.2), there are several 
  new sections applying his analysis methods to the new-fangled electrical 
  circuits that were just starting to become important. Networks of capacitors 
  and inductances obey the same mathematical equations as networks of masses 
  and springs, so results from mechanics can be carried over directly: but the 
  theory was developed first in the mechanical context. This age saw the very 
  beginnings of electronics as a discipline. 

  So perhaps it is not so surprising that physics researchers shortly after 
  Helmholtz, in search of unsolved problems in mechanics, chose to investigate 
  the relatively obscure, but intriguing, question of the motion of a bowed 
  violin string. Some of the published work is experimental: ingenious rigs 
  were designed to allow strings to be bowed in a controlled and repeatable 
  way, and the string motion was observed using photographic methods which 
  allowed waveforms of string displacement to be captured for the first time. 
  These observations revealed that a bowed string can vibrate in a great 
  variety of ways, of which Helmholtz motion is just the simplest. Other 
  researchers followed Helmholtz in giving mathematical form to these various 
  observations. 

  These threads of research were pulled together, developed and given their 
  definitive form by our second giant, C. V. Raman (1888--1970). He would later 
  become India’s first Nobel prize winner, for discovering an optical 
  scattering effect leading to the important technique still known as 
  \tt{}Raman spectroscopy\rm{}. But earlier in his scientific career he studied 
  several problems involving musical instruments. We have already met Raman 
  back in section 3.6, because as well as his work on bowed strings he studied 
  the acoustics behind the characteristic sound of Indian drums. But the work 
  that interests us here is a “blockbuster” monograph, over 150 pages long, 
  that Raman published in 1918 [4]. It has the snappy title “On the mechanical 
  theory of the vibrations of bowed strings and of musical instruments of the 
  violin family, with experimental verification of the results: Part 1” (there 
  never was a “Part 2”). 

  In this work (and other papers published around the same date), Raman 
  advanced the subject to the limit of what was possible before the development 
  of electronic measurement equipment, and long before the age of computers. 
  First, he brought order to the bewildering array of bowed-string waveforms 
  that had been revealed by measurements. He used an ingenious argument, 
  explained in the next link, to suggest that the velocity of the string under 
  the bow should always alternate between two fixed values, with sudden jumps 
  between them. He was thus able to describe all the observed waveforms based 
  on travelling “corners” on the string. Helmholtz motion is the simplest of 
  the family, with a single corner, but there are many other possibilities with 
  multiple corners. The number of corners gave Raman the basis for a 
  classification scheme for what he called “higher types” of bowed-string 
  motion. This work of Raman’s was a very early example of detailed study of a 
  non-linear dynamical system of sufficient complexity to show multiple 
  solutions that are qualitatively different: probably the only such system 
  about which more was known at this early date was the motion of the planets. 

  We can see some examples of Raman’s “higher types”. They divide into two 
  families: one family involves more than one stick-slip alternation in each 
  cycle, while the other family is more like Helmholtz motion in that it 
  involves only a single slipping episode per cycle. The simplest member of the 
  first family will be very important to us in later sections. Figure 5 shows 
  an example: for obvious reasons, it is known as “double-slipping motion”. In 
  place of the Helmholtz sawtooth wave, we see a kind of double sawtooth shape 
  with two abrupt “flyback” episodes. There are two travelling corners, and 
  each generates a flyback jump when it hits the bridge during its round trip. 
  Every time a corner passes the bow, it triggers a transition between sticking 
  and slipping, so there are two slip episodes in every cycle. If you are 
  interested, you can see an animation of the string motion during this kind of 
  double-slipping motion in the previous link. 

  Figure 6 shows three examples from the second family of “Raman higher types”. 
  They each involve an underlying sawtooth shape, but superimposed on this they 
  have a regular pattern of large wiggles. Raman explained such waveforms in 
  terms of a large number of travelling corners (the exact number being related 
  to the number of the wiggles in each cycle). But most of these corners do NOT 
  trigger a transition between sticking and slipping at the bow, because the 
  pattern involves pairs of corners reaching the bridge simultaneously from 
  opposite sides in such a way that their effects cancel out. In each cycle 
  there are just two corners that pass the bow without being cancelled like 
  this, with the result that there is just a single episode of slipping in 
  every cycle. It turns out to be shorter than the corresponding slip for 
  Helmholtz motion, because the corner causing slipping to start is not the 
  same one that causes it to stop: it is a different corner, arriving at the 
  bow before the other one has had time to travel to the bridge and back. 

  This family was later rediscovered by Bo Lawergren in 1980 [5], using a 
  completely different style of analysis from Raman’s. Lawergren called this 
  kind of waveform “S-motion”, and we will use that convenient label. You can 
  see an animation of the string motion during an example of S-motion in the 
  previous link. We will find out more about S-motion, and see how these 
  particular measured bridge-force waveforms were obtained, in section 9.3. 

  That discussion will be just a small part of a larger and more complicated 
  story to be told in the remainder of this chapter. The study of bowed-string 
  motion over the last 50 years or so has given a kind of microcosm of 
  classical scientific method. Progressively more sophisticated measurements 
  have become possible, and every time a new approach has been applied to bowed 
  strings it has revealed shortcomings in physical understanding and called for 
  new developments in modelling. It is not so much that the old theories were 
  wrong, but when more searching questions are asked, inadequacies in the old 
  descriptions become obvious. The result is a series of layers of increasingly 
  detailed descriptions, accompanied inevitably by increasing complexity. 

  Within this rather complicated story, we can distinguish four different 
  agendas. They overlap, but they have different priorities and it is worth 
  being aware of them all. 

  \textbf{The physics agenda}. What are the main phenomena, and how can we 
  explain them, at least approximately? What ingredients must be included in a 
  satisfactory physical model of each different phenomenon? 

  \textbf{The nonlinear dynamics agenda}. This is a more mathematical outgrowth 
  of the physics agenda. Regarding the bowed string as an example of a 
  nonlinear dynamical system, what are the possible regimes of oscillation? 
  Where can each regime be found in the player’s parameter space, and what is 
  the bifurcation structure linking regimes? 

  \textbf{The computer music agenda}. To make any progress at all we will need 
  to use numerical models and simulations. This takes us near the territory of 
  the computer music specialists. They are interested in computer-based 
  resources for composers and performers wanting to make music. Some of the 
  methods they use are inspired by physics-based models of traditional musical 
  instruments, so-called “\tt{}physical modelling synthesis\rm{}”. But their 
  mindset is different from a physicist’s. They want to make interesting 
  sounds, in a way that can be controlled and manipulated for musical effect. 
  They may be inspired by physical models, but they are not bound by them: from 
  the perspective of physical models aimed at understanding how traditional 
  instruments work, they are perfectly free to “cheat” if that gives good 
  results. 

  \textbf{The musician’s agenda}. What can models of a bowed string tell us 
  about the practical concerns of a violinist? What is going on in the 
  different styles of specialised bow gesture, such as spiccato, sautillé or 
  martelé? What governs the “tonal palette” available to a player? Could the 
  answers to these questions help a student or their teacher in the long quest 
  to master an instrument? Then there are questions relevant to the instrument 
  maker: what might a player mean when they describe a particular instrument, 
  or string, or note, as being “easy to play” or “hard to play”? What could an 
  instrument maker or adjuster do to improve “playability”? 

  We can get a glimpse of how the musician’s agenda differs from the others by 
  looking at an example of actual musical performance. Figure 7 shows a few 
  seconds of music, played by violinist Keir GoGwilt on the G string of a 
  violin and recorded using a bridge-force sensor. You can hear it in Sound 1. 
  Surely the first thing that strikes the eye in Fig.\ 7 is that nothing is 
  ever steady. We can see that even more clearly in Fig.\ 8, which shows the 
  first half of the passage as a time-frequency spectrogram: the horizontal 
  axis shows frequency and the vertical axis shows time, running upwards in the 
  plot. The harmonics of each note show as a set of roughly vertical lines. 

  Reading upwards from the bottom in Fig.\ 8, the first note ends with a 
  frequency slide, progressively more obvious in the higher harmonics. The 
  slide ends on a higher note, which is faded out, to be followed by a fresh 
  transient on the same pitch. This next note, extending over the range 
  1.9--3.3 s in the plot, has vibrato, showing up as wiggliness in the harmonic 
  lines. The player has modulated this vibrato through the duration of the 
  note, increasing and then decreasing its amplitude. This note also shows, 
  especially in its earlier part, harmonics reaching up to roughly double the 
  frequency of the previous notes: we will find out in sections 9.2 and 9.3 how 
  the player might have achieved this effect. The remainder of the plot shows 
  four short notes and ends on a longer one. The frequency content is being 
  varied, and the pitch is continually being modulated in various subtle ways. 

  In just a few seconds of music, the violinist has made use of many different 
  aspects of bowed-string motion. Bow speed, force and position are being 
  modulated to vary the waveform details: we will look at how that might work 
  in section 9.3. Different note transitions and initial transients are used: 
  in section 9.5 we will dig a little into the details of these transients. A 
  player will not be consciously aware of all this underlying physics, but they 
  will be aware of the audible consequences and may spend a long time 
  practising details of the bowing and phrasing to get the passage to sound 
  right. Asked to play the same passage on two different violins, and to 
  comment on whether one is “easier to play” than the other, which of these 
  details will they be most aware of? There are no easy answers to that 
  question. The challenge to the scientist following the “physics agenda” is 
  formidable: to try to understand enough to be able to say something useful to 
  musicians and instrument makers. 

  The main contrast between the physics agenda and the musician’s agenda is 
  that the musician is always interested in the holistic impression of any 
  piece of musical performance, whereas the instinct of a physicist is to pull 
  a complicated problem apart, then try to understand each component using the 
  simplest model that captures it. This is a good strategy for making 
  scientific progress, but it can lead to a communication problem. You explain 
  one particular thing to a musician using a simplified model that emphasises 
  that particular thing, and they may react by saying “but that is hopelessly 
  over-simplified, you are missing all these other important things….” That is 
  a good objection if the other things are directly linked to the one you were 
  talking about: perhaps they really do have to be treated simultaneously in a 
  combined model. But there is always a price for a more inclusive and 
  complicated model: it is so easy to lose sight of the wood for the trees. We 
  should keep in mind a dictum paraphrased from something Einstein said in 
  1933: “Models should be made as simple as possible, but no simpler”. 

  The “nonlinear dynamics agenda” is somewhat more remote from the musician’s 
  agenda. The language of “regimes of oscillation”, and “bifurcation events” 
  where one regime transitions to another, makes an implicit assumption that we 
  are most interested in periodic motions, like the examples plotted in Figs.\ 
  3, 5 and 6. We will see in section 9.3 that transitions between regimes 
  (especially between Helmholtz motion and other, less desirable, regimes) are 
  indeed important. But the musical example highlights the fact that musicians 
  very rarely hold anything steady for long enough to qualify as a “periodic 
  waveform” to a mathematician. It is not at all clear how much of the 
  musician’s agenda can be addressed by that kind of investigation. 

  This distinction lies behind something you may have noticed: I didn’t include 
  sound examples of the bowed-string waveforms from Helmholtz motion or Raman’s 
  “higher types”. The reason is that any simple sound example, particularly 
  with steady periodic motion, never sounds anything like a real violin or 
  cello! Sounds 2 and 3 give some examples: they were both made by taking a 
  single cycle of a measured bridge force waveform and repeating it to make an 
  exactly periodic sound. Sound 2 is an example of Helmholtz motion and Sound 3 
  is double-slipping motion (the waveform from Fig.\ 5). 

  The single cycles underlying these sounds were both played on the same open G 
  string of a violin, but they come out sounding artificial and electronic. You 
  would probably not have recognised either as having anything to do with a 
  real bowed string. You can certainly hear a difference between the two 
  sounds, but you probably cannot translate this into any kind of quality 
  judgement about violin sound: the sound world is simply too far away from 
  real violin playing. Think all the way back to the upside-down faces in Fig.\ 
  2 of Chapter 1: our finely-tuned perceptual abilities rely on familiarity, 
  and it doesn’t take much in the way of novelty to throw them into confusion. 

  But surely you would agree that Sound 1 did immediately sound like a violin. 
  This tells us two things that are very important, and perhaps surprising. 
  First, the bridge force recording captures something of the essence of 
  “violinness”, which is missing from the periodic waveforms of Sounds 2 and 3. 
  Now remember the sound of a backwards piano, from Sound 12 of Section 7.1. 
  Simply playing that sound backwards was enough that it was not recognisable 
  as a piano. Our ability to recognise which instrument is being played relies 
  on the characteristic transient features that we have learned to associate 
  with each particular instrument. Those characteristic bowed-string transients 
  are present in Sound 1, but absent in Sounds 2 and 3. 

  But there is something surprising here. Bowed string motion will be roughly 
  the same on every violin, if they are fitted with the same strings and the 
  player performs the same bowing gestures. The differences in sound between 
  different violins, which can be associated with huge differences in price, 
  rely entirely on the additional colouration of the sound associated with the 
  frequency response of the violin body, as we have discussed in earlier 
  chapters. The bridge force signal that drives the body vibration hardly 
  involves any contribution from the particular violin body. Now listen to 
  Sound 1 again. Yes, it sounds like a violin: but maybe not a very nice one. 
  Perhaps the sound is rather bland, because it lacks the colouration coming 
  from the body vibration? In fact, it is what a violin would sound like if it 
  was designed like a hi-fi amplifier, with a flat frequency response that did 
  not contribute any additional colouration to the sound. 

  I have talked rather vaguely about “transients” and “colouration”, but can we 
  be more explicit? Well, there is one example that is easy to describe. Figure 
  8 has reminded us that violinists often use vibrato, modulating the 
  fundamental frequency of the note in a roughly sinusoidal way at a frequency 
  usually around 5~Hz. Naturally, all the harmonic frequencies are also 
  modulated, with the same percentage change as the fundamental. Putting this 
  information together with the things we already know about the frequency 
  response of a violin body, we see something interesting. It is illustrated 
  with a schematic animation in Fig.\ 9. 

  The blue curve shows the bridge admittance of a violin (this particular one 
  is a famous instrument, the “Jackson” Stradivarius). The red lines indicate 
  the frequencies of the first few harmonics of a particular note, and the 
  stars show where these cross the admittance curve. The extent of vibrato 
  shown here is about 1 semitone peak-to-peak, or about $\pm3\\%$ in frequency. 
  That level of vibrato is towards the upper end of what a violinist may use, 
  but it is a perfectly plausible level. Because I have used the bridge 
  admittance, this animation does not strictly show anything about the sound of 
  the violin, but it does indicate the variation through the vibrato cycle of 
  the energy absorbed from the string by the violin body. The animation does 
  not claim to be more than a schematic indication of what can happen, but it 
  is immediately clear that the amplitudes of the various harmonics could show 
  large variations during the vibrato cycle, each with a different range and 
  phase. 

  The resulting sound will be far more complicated than if vibrato had been a 
  simple frequency modulation of a fixed waveform. But that description applies 
  exactly to the bridge force: provided the player is producing Helmholtz 
  motion, the force waveform will always be a sawtooth, and the only effect of 
  vibrato is to modulate the period. However, the interaction of that force 
  waveform with the frequency response of the body produces something far more 
  interesting; not just in terms of physics, but also in terms of our 
  perception of the sound. People often use words like “richness” or “fullness” 
  when trying to describe the tonal effect of vibrato. 

  The description I have just given seems so obvious that a few years ago I and 
  some colleagues thought it would make a nice clear-cut problem for a 
  psychoacoustical investigation. We would synthesise vibrato sounds processed 
  through different frequency response functions, and get listeners to rate the 
  resulting sounds for qualities like “richness” [6]. That would open the way 
  for various investigations of how violins might differ in their 
  responsiveness to vibrato, and what an instrument maker might be able to do 
  to control or enhance the effect. 

  We were wrong to think this would be easy! The experiments we tried were 
  thwarted by the effect illustrated above with Sounds 1, 2 and 3. In order to 
  control the extent of vibrato in a very careful way, we were forced to used 
  synthesised sounds. Whatever we tried, our listeners responded to the sounds 
  by saying “they all sound horrible, not at all like a violin”. They did not 
  feel able to give musically-relevant judgements on qualities like “richness”. 
  I am still certain that the effect illustrated in Fig.\ 9 really is important 
  for the perceived tonal quality of violin notes played with vibrato, but it 
  will need a better experiment to prove that fact to the satisfaction of a 
  psychoacoustician. A challenge for the future. 

  So finally, here is a summary of the topics to be addressed in the following 
  sections. I will basically follow the physics agenda, but try not to lose 
  sight of the musician’s agenda. The first stage is to see what happened in 
  the next burst of scientific activity concerning bowed strings. This came 
  decades after Raman’s time, but many of the key issues are pre-figured in his 
  1918 monograph. So we aren’t by any means done with Raman yet: his name will 
  crop up several times during the next few sections. Section 9.2 will 
  investigate the first efforts to go beyond Helmholtz’s description, by 
  allowing for the fact that the “Helmholtz corner” will in reality be somewhat 
  rounded rather than being perfectly sharp. This has surprisingly profound 
  implications for what a player is able to do to vary the sound of a bowed 
  note. 

  The next step, in section 9.3, is to start to investigate the player’s 
  parameter space. The earliest investigation was restricted to steady string 
  motions, and led to a famous diagram first presented by John Schelleng. 
  Schelleng’s diagram shows what a player must do (and what they must avoid) in 
  order for Helmholtz motion to be at least possible. Schelleng’s theoretical 
  predictions will be compared to some experimental results, obtained by using 
  a robotic bowing machine to scan the Schelleng diagram systematically. 

  The first non-steady phenomenon to be investigated (by Raman) was the 
  infamous “wolf note”, especially prevalent in the cello. With the advent of 
  the first computer models of bowed-string motion, Raman’s account of the wolf 
  note could be tested. We will see some results in Section 9.4. 

  In Section 9.5 we look at initial transients associated with different bowing 
  gestures, and start to explore the concept of “playability”. Again, we will 
  use computer simulations to explore the validity of a diagrammatic 
  representation of bowed-string behaviour, this time due to Knut Guettler. 

  Efforts to compare the results of the computer models with quantitative 
  measurements on real bowed strings reveal serious problems. It immediately 
  becomes obvious that the models need to be improved. Several physical 
  phenomena were omitted from the early models in the interests of 
  simplification. In Section 9.6 we explore some of these: they are undoubtedly 
  part of the complicated jigsaw puzzle that must be solved in order to tackle 
  the musician's agenda. 

  But we will find that something is still wrong with the models. All these 
  issues were first explored based on the simple friction model that we have 
  already seen, in which the friction force is assumed to be a nonlinear 
  function of the instantaneous sliding speed. However, in the 1980s 
  experiments were done to test that assumption, and they revealed that it is 
  not an accurate physical model for rosin friction, especially when it comes 
  to predicting details of transients. In section 9.7 we will look at this 
  evidence, and start to explore alternative models for friction. We will see 
  some results that suggest that a better friction model gets us at least part 
  of the way towards agreement with measured bowed string transients. 



  \sectionreferences{}[1] H. von Helmholtz: On the sensations of tone as a 
  physiological basis for the theory of music (1862, reprinted by Dover, New 
  York 1954) 

  [2] R. J. Strutt: The Life of John William Strutt, 3rd Baron Rayleigh, Edward 
  Arnold (1924) 

  [3] J. W. S. Rayleigh:The Theory of Sound (1877, reprinted by Dover, New York 
  1945) 

  [4] C. V. Raman: ``On the mechanical theory of the vibrations of bowed 
  strings and of musical instruments of the violin family, with experimental 
  verification of the results. Part I.'' Indian Association for the Cultivation 
  of Science Bulletin \textbf{15}, 1–158 (1918). 

  [5] B. Lawergren: ``On the motion of bowed violin strings.'' Acustica 
  \textbf{44}, 194–206 (1980). 

  [6] ~ C. Fritz, J. Woodhouse, F. P. H. Cheng, I. Cross, A. F. Blackwell and 
  B. C. J. Moore: ``Perceptual studies of violin body damping and vibrato''. 
  Journal of the Acoustical Society of America \textbf{127} 513--524 (2010). 