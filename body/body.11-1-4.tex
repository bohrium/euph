  We will use Rayleigh's principle to find an expression for the effect on 
  natural frequencies of a small perturbation to the bore profile of a 
  cylindrical tube. This is a problem that Rayleigh himself addressed [1], and 
  I will follow his treatment closely. As in previous sections, we consider a 
  tube of length $L$ which is open at both ends. We now allow the 
  cross-sectional area to be 

  $$S(x) = S_0 +S'(x) \tag{1}$$ 

  with $S_0$ a constant, and $|S'| \ll S_0$. 

  In order to apply Rayleigh's principle, we need suitable expressions for the 
  potential and kinetic energies. It is simplest to formulate these in terms of 
  what Rayleigh described as ``the total transfer of fluid at time $t$ across 
  the section at $x$, reckoned from the equilibrium condition'', and following 
  his lead we denote this by $X(x,t)$. In terms of the particle displacement 
  $\xi(x,t)$ which we used earlier, 

  $$X=S \xi . \tag{2}$$ 

  The kinetic energy is straightforward: it is given by 

  $$T=\dfrac{1}{2} \rho_0 \int{S \dot{\xi}^2 dx} = \dfrac{1}{2} \rho_0 
  \int{\dfrac{1}{S} \dot{X}^2 dx}. \tag{3}$$ 

  For the potential energy, we start from an expression derived in section 
  4.1.3: 

  $$P=\dfrac{1}{2} \dfrac{c^2}{\rho_0} \int{\rho'^2 dV}=\dfrac{1}{2} 
  \dfrac{c^2}{\rho_0} \int{S\rho'^2 dx} \tag{4}$$ 

  in terms of the acoustic density fluctuation $\rho'$, where the first 
  integral is taken over the volume inside the tube, and the second integral 
  uses the fact that $dV=S dx$. 

  Now we need to express $\rho'$ in terms of $X$. Figure 1 shows a small 
  section of the tube between positions $x$ and $x+\delta x$. The particle 
  displacements are $\xi$ and $\xi + \delta \xi$, and the cross-sectional areas 
  are $S$ and $S+\delta S$. The initial volume of this fluid element is $V_0 
  \approx S \delta x$, and after taking the particle displacements into account 
  it increases by an amount 

  $$\delta V \approx (S+\delta S)(\xi + \delta \xi) -S \xi \approx S \delta \xi 
  + \xi \delta S =\delta(S \xi) . \tag{5}$$ 

  Now the mass of the fluid element remains constant, so the perturbed density 
  $\rho'$, which we are trying to find, satisfies 

  $$\dfrac{\rho'}{\rho_0}=-\dfrac{\delta V}{V_0} \tag{6}$$ 

  so that 

  $$\dfrac{S \delta x}{\rho_0}\rho' \approx -\delta(S \xi) . \tag{7}$$ 

  Dividing by $\delta x$ and taking the usual calculus limit gives 

  $$\rho'=-\dfrac{\rho_0}{S}\dfrac{\partial}{\partial x}(S \xi) \tag{8}$$ 

  so that our potential energy is 

  $$P=\dfrac{1}{2} \rho_0 c^2 \int{\dfrac{1}{S}\left[\dfrac{\partial}{\partial 
  x}(S \xi)\right]^2 dx}=\dfrac{1}{2} \rho_0 c^2 
  \int{\dfrac{1}{S}\left[\dfrac{\partial X}{\partial x}\right]^2 dx} . 
  \tag{9}$$ 

  Now we can apply Rayleigh's principle to our original problem of making a 
  small perturbation to the bore profile. We solved a similar problem back in 
  section 3.3.1, relating to the tuning of a marimba bar. The argument is the 
  same this time: we evaluate the Rayleigh quotient using the corrected energy 
  expressions (including the perturbed bore), but we use a mode shape from the 
  original, unperturbed problem. Errors arising from this incorrect mode shape 
  will only be of second order, so that the predicted natural frequency will be 
  correct to leading order in the small perturbation. 

  We are dealing with a cylindrical tube, open at both ends. The pressure mode 
  shapes are thus of the form $\sin n \pi x/L$ (as in Fig.\ 11 of section 4.2). 
  But for this formulation, we need the mode shapes expressed in terms of 
  particle displacements and the variable $X$. By virtue of equation (7) from 
  section 4.1.1, we know that the answer for the $n\mathrm{th}$ mode is 

  $$X_n=\cos \dfrac{n \pi x}{L} . \tag{10}$$ 

  Noting that 

  $$\dfrac{1}{S}=\dfrac{1}{S_0+S'} \approx 
  \dfrac{1}{S_0}\left[1-\dfrac{S'}{S_0}\right] \tag{11}$$ 

  by the binomial theorem, our Rayleigh quotient reads 

  $$\omega_n^2 \approx \dfrac{c^2 \int{(1-S'/S_0)(n \pi/L)^2 \sin^2(n \pi x/L) 
  dx}}{\int{(1-S'/S_0) \cos^2(n \pi x/L) dx}} \tag{12}$$ 

  using equations (3) and (9). This takes the form 

  $$\omega_n^2 \approx \dfrac{P_0 + \delta P}{T_0 + \delta T} \tag{13}$$ 

  with 

  $$P_0=c^2 \left(\dfrac{n \pi}{L}\right)^2 \int{ \sin^2 \dfrac{n \pi x}{L} 
  dx}=c^2 \left(\dfrac{n \pi}{L}\right)^2 \dfrac{L}{2} \tag{14}$$ 

  $$\delta P=-\dfrac{c^2}{S_0} \left(\dfrac{n \pi}{L}\right)^2 \int{S'(x) 
  \sin^2 \dfrac{n \pi x}{L} dx} \tag{15}$$ 

  $$T_0=\int{ \cos^2 \dfrac{n \pi x}{L} dx} =\dfrac{L}{2} \tag{16}$$ 

  and 

  $$\delta T=-\dfrac{1}{S_0} \int{S'(x) \cos^2 \dfrac{n \pi x}{L} dx} . 
  \tag{17}$$ 

  It follows that 

  $$\omega_n^2 \approx \dfrac{P_0}{T_0}\left[1 + \dfrac{\delta P}{P_0}\right] 
  \left[1 -- \dfrac{\delta T}{T_0}\right]\approx \dfrac{P_0}{T_0}\left[1 + 
  \dfrac{\delta P}{P_0} -- \dfrac{\delta T}{T_0}\right]$$ 

  $$=\left(\dfrac{c n \pi}{L}\right)^2 \left[1+\dfrac{2}{S_0L} \int{S'(x) 
  \left(\cos^2\dfrac{n \pi x}{L}-\sin^2 \dfrac{n \pi x}{L} \right) dx}\right]$$ 

  $$=\left(\dfrac{c n \pi}{L}\right)^2 \left[1+\dfrac{2}{S_0L} \int{S'(x) 
  \cos\dfrac{2n \pi x}{L} dx}\right] . \tag{18}$$ 

  Finally, Rayleigh expressed the result as an equivalent length correction to 
  the cylindrical tube to achieve the perturbed resonance frequency: the extra 
  length required is 

  $$\Delta L \approx -\dfrac{1}{S_0L} \int{S'(x) \cos\dfrac{2n \pi x}{L} dx} . 
  \tag{19}$$ 

  Apart from a multiplicative factor, this rather simple form is a Fourier 
  series coefficient of the bore perturbation $S'(x)$. 

  The frequency is raised by an enlargement of the bore near the ends, or near 
  any other nodal point of pressure (which is an antinodal point of particle 
  displacement or $X$). Conversely frequency is reduced by a bore enlargement 
  near an antinode of pressure (or a node of particle displacement). As 
  Rayleigh points out, the frequencies are rather insensitive to a slight 
  conicity of the tube, because that would give a linear form for $S'(x)$, and 
  such a linear form is orthogonal to the cosine function in the integral. 

  \sectionreferences{}[1] J. W. S. Rayleigh:The Theory of Sound (1877, 
  reprinted by Dover, New York 1945). This calculation appears in Volume 2, 
  section 265. 