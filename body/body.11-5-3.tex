  The brass simulations in section 11.5 used the simplest possible model to 
  represent the action of the player's lips; essentially the same as the model 
  used earlier for reed instruments except with a sign change to give the ``lip 
  reed'' the character of an opening reed rather than a closing reed. This 
  model gave qualitatively plausible predictions, and even achieved reasonable 
  quantitative agreement with a set of waveforms published by Boutin, Smith and 
  Wolfe [1]. However, in that paper Boutin et al. highlighted an aspect of the 
  observed lip vibration which is not included in the modelling so far. They 
  suggested that this extra effect might have important consequences, and the 
  purpose of this section is to present a preliminary investigation of that 
  suggestion. 

  The first ingredient is an additional component of the volume flow rate into 
  the instrument. We already have the flow rate associated with air flow 
  through the gap, but there is also a component generated directly by the 
  mechanical motion of the ``reed'', or in this case of the player's lips. By 
  analysing video recordings of a trombone player's lips (making use of a 
  transparent mouthpiece), Boutin et al. were able to estimate this ``sweeping 
  flow'' contribution. They found that the magnitude was significant, and also 
  found that the complicated motion of squashy lips resulted in a phase 
  difference between the ``opening'' and ``sweeping'' motions: they described 
  the result as ``swimming motion'' of the lips. Furthermore, when a 
  professional trombonist adjusted their embouchure in order to raise or lower 
  the pitch by ``lipping up'' or ``lipping down'', they seemed to be changing 
  this phase relationship. 

  Finally, the authors calculated that this phase difference makes a 
  significant contribution to the energy budget of the oscillation, which might 
  affect the blowing pressure threshold as well as the pitch of the played 
  note. These are the effects we now want to investigate, by extending the 
  simulation model to allow a simple representation of ``swimming motion'', 
  including the possibility of an adjustable phase shift. 

  Back in section 11.3.1, when we first described a model for flow through a 
  reed mouthpiece, we already mentioned the ``sweeping flow'' effect. Equation 
  (14) from that section gave the total volume flow rate into the instrument as 

  \begin{equation*}v=-w(H_0-y) \sqrt{2 |\Delta p|/\rho_0} \mathrm{~sign} 
  (\Delta p)+A_r \dot{y} \tag{1}\end{equation*} 

  \noindent{}where the first term describes the air flow through the gap, and 
  the second term describes the sweeping flow. This second term is proportional 
  to the tip velocity of the reed, $\dot{y}$, with a coefficient $A_r$ which is 
  an effective area of the vibrating reed. The other variables in this equation 
  are the pressure difference $\Delta p = p-p_m$, the density of air $\rho_0$, 
  the width $w$ of the lip opening (idealised as rectangular), and the initial 
  opening $H_0$. The time-varying pressure inside the mouthpiece is $p(t)$, 
  while $p_m$ is the player's mouth pressure, assumed constant for this simple 
  model. 

  This formulation of sweeping flow is quite natural for a woodwind reed 
  mouthpiece, but it feels rather artificial when describing lip vibration: if 
  you visualise the trombone being held horizontally as usual, the opening 
  displacement $y$ is predominantly in the vertical direction, but the sweeping 
  flow depends on predominantly horizontal motion of the lips. It is this 
  distinction between vertical lip opening and horizontal motion which allows 
  the possibility of a phase difference: the squashy nature of lips allows the 
  vertical and horizontal components of motion to be out of step. Boutin et al. 
  observed the two things separately by analysing high-speed video recordings 
  of fully-developed periodic motion of a professional trombonist's lips: we 
  will see their results shortly. 

  So the first step towards an enhanced model is to represent the horizontal 
  and vertical motions separately, with tip displacements $x$ and $y$ as 
  sketched in Fig.\ 1, measured from an equilibrium position determined by the 
  player's embouchure and mean flow rate. The sketch only shows the upper lip: 
  I will assume that the motion of the two lips is symmetrical, so that the 
  combined opening height is $2y+H_0$ where $H_0$ is the opening in the 
  equilibrium state, as in the reed case. An approximate revised version of 
  equation (1) is thus 

  \begin{equation*}v=-w(H_0+2y) \sqrt{2 |\Delta p|/\rho_0} \mathrm{~sign} 
  (\Delta p)+A_r \dot{x} \tag{2}\end{equation*} 

  \noindent{}where $A_r$ now represents a combined effective area of the two 
  vibrating lips. 

  \fig{figs/fig-e64ea4b7.png}{\caption{Figure 1. Sketch of the upper lip of a 
  brass player, showing the two components of motion}} 

  Now we introduce another factor suggested by Boutin et al. [1]. They point 
  out that this effective area is likely to vary slightly through the vibration 
  cycle of the lips. If that variation in area is somewhat out of phase with 
  the horizontal motion described by $x$, net work can be done by the steady 
  component of the pressure difference during a cycle of vibration, an effect 
  that might change the pressure threshold for producing a note on the 
  trombone. A simple idealisation of this effect is to use the vertical 
  displacement $y$ to modulate the area: we can express 

  \begin{equation*}A_r \approx w (L_{lip} -- \lambda y) \tag{3}\end{equation*} 

  \noindent{}where $L_{lip}$ is an effective combined length of the two 
  vibrating lip ``reeds'', and $\lambda$ is a dimensionless constant to be 
  investigated by numerical experimentation once we have a complete model. The 
  maximum plausible value of $\lambda$ would be 2, which would correspond to 
  the effective length $L_{lip}$ being reduced by the entire opening 
  displacement $2y$. 

  The dynamics of the ``reed'' is determined, in this simple idealisation, by 
  the horizontal displacement $x$: a suitable new version of the governing 
  equation from section 11.3.1 (equation (3) there) can be written 

  \begin{equation*}M_r \ddot{x} + C_r \dot{x} + K_r x=-\Delta p \dfrac{A_r}{w 
  L_{lip}}=-\Delta p \left(1-\dfrac{\lambda y}{L_{lip}}\right) . 
  \tag{4}\end{equation*} 

  The term involving $\lambda$ arises because the variables $M_r$, $C_r$ and 
  $K_r$ were defined per unit area, so we now need to use the equilibrium 
  values of those variables on the left-hand side of the equation, and then 
  adjust the right-hand side to take account of the (small) variation in 
  effective area. 

  In order to perform simulations using equations (2--4), we need a relation 
  between $x$ and $y$. The simplest approximation would be to assume 
  proportional motion, so that $x=\alpha y$ with some (positive) value of the 
  constant $\alpha$. This assumption does not give us any direct control over 
  the phase, but we can cheat to obtain a simple model for the purposes of this 
  preliminary investigation. The experimental data only tells us about the 
  final periodic part of the note, so we can concentrate on that in our 
  modelling. We can achieve a phase difference by calculating $y$ from a 
  delayed value of $x$ (the measurements of Boutin et al. [1] always show the 
  $x$ motion leading the $y$ motion). 

  For sinusoidal motion of the lips, this would correspond to the tip of each 
  ``lip reed'' moving along an elliptical path. The orientation and ellipticity 
  are determined by a combination of the delay and the value of the constant 
  $\alpha$. This elliptical path gives a simple representation of the observed 
  ``swimming motion''. One way to think about this approximate model is that we 
  are treating the combined lip motion as a mechanical system with a single 
  resonance, but with a mode shape which is complex so that the $x$ and $y$ 
  motions are not in the same phase. For any vibration involving a material as 
  highly damped as the flesh of lips, complex modes would in fact be expected. 

  Varying the delay will be (roughly) the same as varying the phase shift, so 
  this simple model can specify the relative phase of the opening motion and 
  the sweeping motion arbitrarily, and by adjusting the values of $L_{lip}$, 
  $\alpha$ and $\lambda$ we can also control the magnitudes of the various 
  contributing factors. The delay strategy does not really qualify as a 
  physical model of lip vibration, but it will allow us to make controlled 
  investigations over a range of amplitudes and phases. 

  We have three targets for this investigation. The first is to see how close 
  we can get to reproducing the three sets of measured waveforms from Boutin et 
  al. [1]. The second is to investigate the specific proposal that inclusion of 
  ``swimming motion'' with an appropriate phase shift might allow successful 
  simulations with a lower Q factor for the lip resonance. In section 11.5 we 
  showed results using Q factors in the range suggested by Doc et al. [2], with 
  values 7, 15 and 30. But even Q=7 seems implausibly high for a resonance of 
  soft flesh, suggesting that there may be something important missing from the 
  model --- perhaps swimming motion of the lips will be the answer? The third 
  effect concerns the pitch of the played note. The initial model from section 
  11.5 predicted that the player, by varying lip tension, could raise the pitch 
  of the note, but never lower it below the relevant resonance of the tube --- 
  see Fig.\ 18 of section 11.5. But in reality, a trombonist can ``lip down'' 
  just as easily as they can ``lip up'', so this again points to a deficiency 
  in the initial model. 

  With these aims in mind, we will concentrate our efforts on a small region of 
  the pressure --- lip resonance diagram, illustrated in Fig.\ 2. This version 
  of the diagram shows results for Q=7, and is colour-shaded to indicate the 
  length of transient following a cold start. The chosen region, outlined with 
  a green box, is concentrated around the ``nose'' of the feasible zone for the 
  lowest normal note of the trombone, based on the second tube resonance. The 
  three measured cases of Boutin et al. fall near the centre of this region, 
  and if swimming motion can have an influence that allows lower Q factors to 
  be used, we would expect to see the threshold pressure reduce — in other 
  words, the ``nose'' should move to the left. 

  \fig{figs/fig-62405eef.png}{\caption{Figure 2. A pressure --- lip resonance 
  diagram for a trombone for the same model as Fig. 22 of section 11.5, for the 
  case in which the Q factor of the lip resonance is 7. The colour shading 
  indicates the transient length before a periodic motion is established, and 
  the green box shows the region to be studied in detail in this section.}} 

  The chosen range has blowing pressure up to 2 kPa, and lip resonance in the 
  range 80--130 Hz. This region has been covered by a $10 \times 10$ grid of 
  points, and sets of simulations have been run. The grid has been repeated 
  with 5 different values of the delay: one case with no delay, and four cases 
  with progressively longer delays. The results will be shown in a plotting 
  format illustrated in Fig.\ 3. The top left plot is the case with no delay. 
  It corresponds to the original data as shown in Fig.\ 2, and forms our 
  reference comparison for all the other plots. After that, the delay increases 
  along the top row, and continues to increase on the second row. The constant 
  $\alpha$ is set (rather arbitrarily) to the value 2 so that the elliptical 
  path of the lips is angled at $45^\circ$. All simulations use a width $w$ of 
  the (assumed rectangular) lip opening of 12~mm (a value recommended by Doc et 
  al. [2]). The parameter $\lambda$ governing the area modulation effect is set 
  to zero for this initial case so that the effective area is then given by 

  \begin{equation*}A_r=w L_{lip} \tag{5}\end{equation*} 

  \noindent{}in terms of the ``effective reed length'' $L_{lip}$. The 
  simulations in Fig.\ 3 use the value $L_{lip}=1 \mathrm{~mm}$ so that the 
  sweeping flow term is very small, and the results for zero delay are almost 
  identical to those seen in Fig.\ 2. 

  \fig{figs/fig-60f95027.png}{\caption{Figure 3. Set of pressure --- lip 
  resonance diagrams, covering the range indicated by the green box in Fig. 1. 
  The Q factor of the lip resonance is 7, $L\_{lip}$ is 1~mm , $\alpha=2$ and 
  $\lambda=0$. Colour shading represents transient length following a cold 
  start transient. The top-left plot shows results with zero delay, while the 
  other 4 cases show progressive time lags between vertical motion $y$ and 
  horizontal motion $x$, as described in the main text.}} 

  Figures 4 and 5 show closely-related plots, with different parameter values. 
  Figure 3 had the area modulation effect switched off, and the sweeping flow 
  term rendered negligibly small, so that only the direct effects of the phase 
  delay between $y$ and $x$ were highlighted. In Fig.\ 4 a larger value is used 
  for $L_{lip}$, 8~mm. We will see shortly that this value gives an approximate 
  match to the magnitude of the sweeping flow as measured by Boutin et al. In 
  Fig.\ 5 the area modulation effect is turned on in addition, with the largest 
  credible value $\lambda =2$. 

  \fig{figs/fig-0ab6c20b.png}{\caption{Figure 4. Set of pressure --- lip 
  resonance diagrams for the same parameter values as Fig. 3, except that now 
  $L\_{lip}$ is 8~mm. The format and colour shading is the same as in Fig. 3.}} 

  \fig{figs/fig-025051ed.png}{\caption{Figure 5. Set of pressure --- lip 
  resonance diagrams for the same parameter values as Figs. 3 and 4, except 
  that now $L\_{lip}$ is 8~mm and $\lambda=2$. The format and colour shading is 
  the same as in Fig. 3.}} 

  Comparing Figs.\ 3, 4 and 5 proves very revealing. Figure 4 reveals a very 
  strong effect of the delay: as the delay is increased, the region of coloured 
  pixels grows and the ``nose'' of the region moves to the left. Figure 5 shows 
  that adding in the sweeping flow term has a small but clear effect in 
  reversing these changes: the coloured region is slightly smaller in every 
  case, and the nose retreats to the right. Finally, Fig.\ 5 shows that 
  switching on the area modulation effect has virtually no effect. The 
  differences between Figs.\ 4 and 5 are so small that you have to look very 
  carefully to see them. So we see a clear hierarchy among the three effects we 
  have added in to the original model: the phase difference between $x$ and $y$ 
  (and therefore between $x$ and pressure) has a very strong effect, the 
  sweeping flow term has a modest effect, and the area modulation has a 
  negligible effect. This conclusion has been confirmed by computed results for 
  other cases. 

  In order to see whether this model addresses our three targets, we need to 
  dig into the results in more detail. Figure 4 strongly suggests that we could 
  surely get away with a lower Q-factor for the lip resonance once we have 
  included a significant phase delay, and simulation experiments confirm this. 
  For the remainder of this section we will turn to results with a modestly 
  reduced value $Q=5$, but it seems likely that even lower values could be 
  tolerated within a model of this general type --- an investigation that is a 
  task for the future. 

  Figure 6 shows some results with this reduced value of Q, this time 
  colour-shaded to show the actual phase lag, in degrees, between $y$ and $x$. 
  The plot range has been restricted to $0-90^\circ$, because the measured 
  phase lags from Boutin et al. fall in this range: they found a lag by 
  $36^\circ$ for normal playing, rising to $61^\circ$ for lipping down, and 
  falling to $31^\circ$ for lipping up. 

  \fig{figs/fig-175bebde.png}{\caption{Figure 6. Set of pressure --- lip 
  resonance diagrams for a model with $Q=5$, $L\_{lip}=8$mm and $\lambda=0$. 
  Colour shading now indicates the phase lag of $y$ behind $x$, in degrees. The 
  coloured circles indicate cases that will be shown in Figs. 11--13.}} 

  To see whether that reported pattern is consistent with the model 
  predictions, Fig.\ 7 shows the same set of results colour-shaded to indicate 
  the playing frequency over a range of two semitones above and below the 
  nominal frequency. Careful inspection of this figure, in conjunction with 
  Fig.\ 6, is very encouraging. The shortest non-zero delay, shown in the 
  middle of the top row, gives a frequency pattern rather similar to the 
  original model: lots of yellow and white indicating the possibility of 
  lipping up to play sharp, but no darker reds indicating the possibility of 
  lipping down. 

  \fig{figs/fig-6687b27d.png}{\caption{Figure 7. Set of pressure --- lip 
  resonance diagrams for the same model as Fig. 6, with $Q=5$, $L\_{lip}=8$mm 
  and $\lambda=0$. Colour shading now indicates the deviation of the playing 
  frequency from the nominal frequency, expressed in cents and covering a range 
  of 2 semitones above and below nominal. The coloured circles indicate cases 
  that will be shown in Figs. 11--13.}} 

  But as the delay is increased, all the colours shift progressively towards 
  the red end. With the larger values of the phase lag, the model does indeed 
  predict a continuous pattern from lipping up, through the nominal pitch, to 
  lipping down. Comparing with the colours in Fig.\ 6, we can see that the 
  second step of increasing delay (top right plot) gives phases around the 
  values measured by Boutin et al. for normal playing or lipping up, while the 
  next step (bottom left plot) is about right for their lipped-down 
  measurement. 

  The simulation model is predicting a pattern of behaviour that matches, 
  qualitatively at least, two of our three targets. It remains to see if we can 
  find among these results credible matches to the measured waveforms for the 
  three cases studied by Boutin et al. To guide this search, we can use 
  versions of the pressure --- lip resonance diagram with a different choice of 
  colour shading. 

  Boutin et al. gave phasor diagrams, showing the relative phases of the 
  fundamental Fourier component of the various measured waveforms. Inspired by 
  this, we can determine the corresponding phases from each simulation that 
  yields a final periodic waveform (i.e. a ``note'' rather than ``silence''). 
  These can be compared with the measured values and combined to give a metric 
  of ``overall phase deviation'' from a particular choice among the three 
  measured cases. Specifically, I have normalised all the phases by the phase 
  of the opening area, then subtracted the measured value from the simulated 
  value for each of the pressure, volume flow rate and sweeping flow rate 
  waveforms, then added the absolute values of these three deviations together. 

  The results are plotted in Figs.\ 8, 9 and 10 corresponding to the three 
  cases from Boutin et al.: Fig.\ 8 shows ``lipping down'', Fig.\ 9 shows 
  normal playing, and Fig.\ 10 shows ``lipping up''. The best match of this 
  phase deviation metric is found where the colours are closest to white. 
  Selected points for the three cases are marked by circles, and the same three 
  points were also marked in Figs.\ 6 and 7. These selections were governed by 
  a combination of criteria. First, the blowing pressure in the experiments was 
  close to 1~kPa in each case, so I only considered cases with blowing pressure 
  near that value. Second, each case falls in a region of pale colour in the 
  corresponding plot of the phase deviation metric. But I also had an eye on 
  the correct phase difference in Fig.\ 6, and the expected frequency behaviour 
  as revealed by Fig.\ 7. 

  \fig{figs/fig-dd35ac53.png}{\caption{Figure 8. Set of pressure --- lip 
  resonance diagrams for the same model as Fig. 6, with $Q=5$, $L\_{lip}=8$mm 
  and $\lambda=0$. Colour shading now indicates a measure of the net deviation 
  of phase between the simulations and the measured waveforms of Boutin et al. 
  [1] for their ``lipped down'' case, the left-hand column of their Fig. 3. The 
  coloured circle indicates the case that will be shown in Fig. 11.}} 

  \fig{figs/fig-c0257793.png}{\caption{Figure 9. Set of pressure --- lip 
  resonance diagrams for the same model as Fig. 6, with $Q=5$, $L\_{lip}=8$mm 
  and $\lambda=0$. Colour shading now indicates a measure of the net deviation 
  of phase between the simulations and the measured waveforms of Boutin et al. 
  [1] for their ``normal'' case, the middle column of their Fig. 3. The 
  coloured circle indicates the case that will be shown in Fig. 12.}} 

  \fig{figs/fig-020fe328.png}{\caption{Figure 10. Set of pressure --- lip 
  resonance diagrams for the same model as Fig. 6, with $Q=5$, $L\_{lip}=8$mm 
  and $\lambda=0$. Colour shading now indicates a measure of the net deviation 
  of phase between the simulations and the measured waveforms of Boutin et al. 
  [1] for their ``lipped up'' case, the right-hand column of their Fig. 3. The 
  coloured circle indicates the case that will be shown in Fig. 13.}} 

  The comparison of waveforms of the three selected simulations with the 
  corresponding measurements is shown in Figs.\ 11, 12 and 13. The measured 
  waveforms are shown in dashed lines, the simulations in solid lines of the 
  corresponding colour. The top panel of each plot shows the pressure, the 
  middle panel shows the total volume flow rate (in blue) and the sweeping flow 
  contribution to that rate (in green). The bottom panel shows the open area of 
  the lips. 

  \fig{figs/fig-c85b71f0.png}{\caption{Figure 11. Simulated waveforms for the 
  case marked by a green circle in Fig. 8, compared with corresponding 
  waveforms measured by Boutin et al. [1] for their case in which the note was 
  ``lipped down''. All simulated waveforms are shown as solid lines, while 
  measured ones are shown as dashed lines in the corresponding colour. The top 
  plot shows the comparison for the pressure inside the mouthpiece. The middle 
  plot shows the total volume flow rate (blue) and the sweeping flow 
  contribution to that (green). The bottom plot shows the lip opening area.}} 

  \fig{figs/fig-340afc3f.png}{\caption{Figure 12. Simulated waveforms for the 
  case marked by a green circle in Fig. 9, compared with corresponding 
  waveforms measured by Boutin et al. [1] for their case in which the note was 
  played normally. All simulated waveforms are shown as solid lines, while 
  measured ones are shown as dashed lines in the corresponding colour. The top 
  plot shows the comparison for the pressure inside the mouthpiece. The middle 
  plot shows the total volume flow rate (blue) and the sweeping flow 
  contribution to that (green). The bottom plot shows the lip opening area.}} 

  \fig{figs/fig-ab950a30.png}{\caption{Figure 13. Simulated waveforms for the 
  case marked by a green circle in Fig. 10, compared with corresponding 
  waveforms measured by Boutin et al. [1] for their case in which the note was 
  ``lipped up''. All simulated waveforms are shown as solid lines, while 
  measured ones are shown as dashed lines in the corresponding colour. The top 
  plot shows the comparison for the pressure inside the mouthpiece. The middle 
  plot shows the total volume flow rate (blue) and the sweeping flow 
  contribution to that (green). The bottom plot shows the lip opening area.}} 

  All these comparisons are very encouraging. Every waveform is recognisably 
  close to the corresponding measurement, both in absolute level and in shape. 
  Furthermore, some trends in waveform shape between the three cases are 
  reproduced. In particular, look at the general shape of the blue waveforms in 
  the middle panels. There is a progressive change from trending upwards, 
  through more or less level, to trending strongly downwards as we move through 
  the three cases. These qualitative changes are well captured by the 
  simulations. 

  The results from this preliminary investigation of the effect of ``swimming 
  motion'' of a brass-player's lips are very encouraging: we have seen a 
  reasonable qualitative match for all three targets. The rather ad hoc delay 
  model has allowed us to probe the influence of separate ingredients of the 
  model, which a more complete and physical model might not have allowed. The 
  conclusion was that by far the most important effect to include in an 
  improved model of lip vibration was the potential for a phase difference 
  between the $x$ and $y$ motions to produce the ``swimming'' effect. By 
  comparison, the effects of the sweeping flow term in equation (2) and of 
  modulation of the effective lip area over which the pressure acts are very 
  minor. 

  However, there is no doubt that further development of the lip model would be 
  desirable to give a better physical basis. There are some suggestions already 
  in the literature, involving two or more degrees of freedom to represent 
  additional aspects of the lip motion: see for example the discussion in 
  section 6.4 of Campbell, Gilbert and Myers [3]. But we would need to be 
  cautious in exploring further down that line. The problem is that 
  progressively more complicated models introduce more and more parameters. 
  Without a genuinely physical model in the background to inform the choice of 
  values for these parameters, there are too many dimensions to explore 
  comfortably by a purely computational approach. Also, there are not very many 
  detailed measurements against which to benchmark the models: for example, 
  there is nothing as detailed as the experimental Guettler diagrams we were 
  able to use in chapter 9 when comparing different models for bowed-string 
  motion. We have already seen that a credible match to the only published 
  measurements can be obtained with the simple delay model. 

  \sectionreferences{}[1] Henri Boutin, John Smith and Joe Wolfe, “Trombone lip 
  mechanics with inertive and compliant loads (‘lipping up and down’)”, Journal 
  of the Acoustical Society of America 147, 4133—4144 (2020) 

  [2] J.-B. Doc, C. Vergez and J. Hannebicq, “Inverse problem to estimate lips 
  parameters values of outward-striking trumpet model for successive playing 
  registers”, Journal of the Acoustical Society of America 1\textbf{53}, 
  168—178 (2023). 

  [3] Murray Campbell, Joël Gilbert and Arnold Myers, “The science of brass 
  instruments”, ASA Press/Springer (2021) 