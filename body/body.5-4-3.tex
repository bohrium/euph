  To estimate the natural frequencies of a string with non-zero bending 
  stiffness, we can use Rayleigh's principle. In order to do that we need 
  expressions for the potential and kinetic energies of the stiff string. 
  Because kinetic energy is conventionally denoted $T$ in textbooks, for this 
  particular section we will denote the tension of the string by $P$ to avoid 
  confusion. 

  The kinetic energy is, as usual, the sum of ``$\frac{1}{2}m v^2$'' for all 
  small elements of the string, so that 

  $$T=\frac{1}{2}\int_0^L{m \left(\frac{\partial w}{\partial t}\right)^2} dx 
  \tag{1}$$ 

  for a string with mass per unit length $m$. 

  The potential energy is the sum of two terms, one representing the potential 
  energy of an ideal flexible string and the other being the term we have 
  already seen (in section 3.3.1) for a bending beam. To find the expression 
  for the string component, consider the small element of string sketched in 
  Fig.\ 1. This element started with length $\delta x$, but it stretches 
  slightly to a length $\delta l$ after the string has deflected. By 
  Pythagoras's theorem 

  $$\delta l = \sqrt{\delta x^2 + \delta w^2}=\delta x [1+(\delta w / \delta 
  x)^2]^{1/2}$$ 

  $$\approx \delta x \left[1+ \frac{1}{2}\left(\dfrac{\partial w}{\partial 
  x}\right)^2 \right]\tag{2}$$ 

  by the binomial theorem. Work is done during this stretch, against the 
  existing tension $P$ of the string. So the stored energy in this element is 
  approximately 

  $$\frac{P}{2}\left(\dfrac{\partial w}{\partial x}\right)^2 \delta x \tag{3}$$ 

  and the total potential energy of the string is obtained by integration along 
  the length. Adding in the term for the bending beam, the total potential 
  energy of the stiff string is 

  $$V=\frac{1}{2}\int_0^L{P \left(\frac{\partial w}{\partial x}\right)^2} dx + 
  \frac{1}{2}\int_0^L{E I \left(\frac{\partial^2 w}{\partial x^2}\right)^2} dx 
  . \tag{4}$$ 

  For a monofilament string with a circular cross-section of radius $r$ made of 
  material with Young's modulus $E$, the bending rigidity coefficient has the 
  value 

  $$EI=\dfrac{E \pi r^4}{4} \tag{5}$$ 

  but for a string with multi-layer wrapped construction the factors $E$ and 
  $I$ have no useful independent meanings, and the combined coefficient $EI$ is 
  best regarded as a single constant, to be fitted to measurements of the 
  behaviour of the string. 

  Now to use Rayleigh's principle to estimate frequencies, we need an 
  approximate expression for the mode shape; but the mode shapes are not much 
  affected by stiffness, so we can simply use the ideal shapes $u_n(x)=\sin(n 
  \pi x/L)$ for this purpose. We can then evaluate the Rayleigh quotient to 
  obtain 

  $$\omega_n^2 \approx \dfrac{\int_0^L{P \left(\frac{\partial u_n}{\partial 
  x}\right)^2} dx + \int_0^L{E I \left(\frac{\partial^2 u_n}{\partial 
  x^2}\right)^2} dx}{\int_0^L{m u_n^2} dx} \tag{6}$$ 

  so that 

  $$\omega_n^2 \approx \dfrac{P}{m} \left(\dfrac{n \pi}{L}\right)^2 + 
  \dfrac{EI}{m} \left(\dfrac{n \pi}{L}\right)^4 . \tag{7}$$ 

  For any realistic musical string, the effect of stiffness is relatively 
  small, and in particular the fundamental frequency is always well 
  approximated by neglecting the effect of stiffness: 

  $$\omega_1^2 \approx \dfrac{P \pi^2}{mL^2} . \tag{8}$$ 

  We can then approximate eq. (7) by 

  $$\omega_n^2 \approx n^2 \omega_1^2 \left[1 + 2 \alpha n^2 \right] \tag{9}$$ 

  so that 

  $$\omega_n \approx n \omega_1 \left[1 + \alpha n^2 \right] \tag{10}$$ 

  where 

  $$\alpha= \dfrac{EI \pi^2}{2P L^2} . \tag{11}$$ 