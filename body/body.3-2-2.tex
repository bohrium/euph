  The sound examples in this chapter were all computed by the same approach. 
  The system is characterised by the natural frequencies $\omega_n$ and 
  Q-factors $Q_n$ of the set of modes, up to some chosen cutoff frequency so 
  that we only have a finite number $N$ of modes to deal with. If mode $n$ has 
  amplitude $a_n$, the waveform we compute and save as a sound file is simply 

  $$f(t)=\sum_{n=1}^N~a_n \cos(\omega_n t) e^{-\omega_n t /(2 Q_n)} \tag{1}$$ 

  where $Q_n = 1/(2\zeta_n)$ in terms of the damping ratio $\zeta_n$ defined in 
  section 2.2.7. The amplitudes are chosen by a very simple strategy. First, 
  the ``instrument'' is assumed to be set into motion using a hammer that 
  behaves like the simple model presented in section 2.2.6. That model has a 
  variable that governs the ``hardness'' of the hammer: the duration of the 
  impact, given by a half-period of the vibration frequency $\Omega$ defined in 
  2.2.6. To represent this effect, $a_n$ includes a factor $\dfrac{\cos[\pi 
  \omega_n/(2 Q_n)]}{(\Omega^2-\omega^2)}$. 

  The second, rather ad hoc, factor included in the amplitudes is associated 
  with the fact that we want to make sounds that are reasonably familiar. But 
  the radiation of sound by vibrating structures is complicated (we will study 
  it in Chapter 4). For the present purpose, none of this complication need be 
  considered. Instead, the amplitude is simply scaled by a power of frequency, 
  with a power $\alpha$ chosen to give an acceptable sound. The chosen value is 
  $\alpha = -0.6$. So in total, 

  $$a_n=\dfrac{\cos[\pi \omega_n/(2 
  \Omega)]}{(\Omega^2-\omega^2)}~\omega_n^\alpha.\tag{2}$$ 

  There is a final factor that could have been included, but for the moment has 
  been ignored. If we wanted to investigate the effect of hitting an instrument 
  at different locations, we would use the result of eq. (11) from section 
  2.2.5 and include a factor $u_n(x)$, where $u_n$ is the nth mode shape and 
  $x$ represents the chosen hammer position. For this purpose the mode shapes 
  need to be normalised according to eq. (10) of section 2.2.5. 

  All modes are assigned the same Q-factor $Q_n$. For cases with a ``hard'' 
  hammer, the duration of the impact $d=\pi / \Omega$ has the value 0.1 ms, 
  while for cases with a ``soft'' hammer, $d=1$ ms. The value of $N$ is chosen 
  to ensure that enough modes are included to cover the frequency range over 
  which amplitudes are significant. 