

  Frequency response functions are the “workhorse” measurement for a lot of 
  musical acoustics research, whether they are used to characterise the 
  vibration and sound radiation from a violin body, or to identify the acoustic 
  signature of a clarinet or trumpet tube. We have already seen quite a few 
  frequency response functions, and made use of them for various purposes. In 
  this section some of this earlier material will be re-examined in the light 
  of the specific question of how to measure a frequency response that is 
  reliable and accurate. 

  When we first introduced the idea of frequency response, back in Chapter 2, 
  we thought about a “black box” which represented the behaviour of a linear 
  system with a single input and a single output. Figure 1 shows a repeat of 
  Fig.\ 2 from section 2.1 (the box is labelled “Drum” because we used a toy 
  drum as our example system in that first discussion). There were then two key 
  ideas. The first is the “sine wave in, sine wave out” property of any linear 
  system. The second is Fourier analysis, which allows us to express any input 
  or output signal as a combination of sine waves. Put together, these mean 
  that if we understand how a system responds to sine waves at every possible 
  frequency, we know everything there is to know about it (at least in 
  principle). 

  In earlier times, before the routine availability of computers, frequency 
  response was measured in exactly that way. Some kind of actuator was fixed to 
  the structure, and it was driven with a sine wave. The desired response was 
  measured, and the frequency of the sine wave was slowly swept through the 
  range of interest — for any musical problem, that range is usually limited to 
  the range of human hearing, roughly 20~Hz—20~kHz. At each frequency the ratio 
  of amplitudes of output/input was measured, and for a complete measurement 
  the phase shift between the two was also measured. The equipment was usually 
  large and expensive, and measurements were confined to laboratory settings. 
  The frequency response plot was usually drawn by a chart recorder, often 
  mechanically coupled by sprockets and a chain to the sine-wave oscillator. 

  The world has changed a lot since those days! Measurements are done with the 
  aid of a computer, and this changes every aspect of the procedure. We need 
  not use sine waves as the input signal: we can use any convenient input 
  signal which contains the range of frequencies we are interested in, and use 
  Fourier analysis (the FFT) in the computer to separate the different 
  frequency components. This opens many possibilities, which we will discuss 
  shortly. 

  The ubiquity of small but powerful computers means that measurements like 
  this have moved out of the laboratory. Some instrument makers are making 
  routine use of frequency response measurements in their workshops, to inform 
  their working processes and decisions. This fact opens another area for 
  discussion, because instrument makers are not necessarily highly trained in 
  science. They sometimes need a little guidance to perform and interpret the 
  measurements. I will try to provide something useful in the course of this 
  section. 

  \samsection{A. Actuators } 

  A.1 Shakers 

  Any measurement requires the instrument to be set into vibration somehow. For 
  a measurement done in the traditional way, with a swept sine wave, this means 
  attaching some kind of actuator capable of producing a continuous excitation 
  — such things are usually called “shakers”. The most common type works on the 
  same principle as a traditional loudspeaker, shown in Fig.\ 2 in a cutaway 
  view taken from \tt{}this Wikipedia page\rm{}. The electrical signal is fed 
  through a coil (2), which is close to a permanent magnet (1). Just as in an 
  electric motor, the interaction of the electric current and the magnetic 
  field produces a force that vibrates the cone (4), which then creates sound 
  waves in the surrounding air. The cone and coil are supported by bellows-like 
  structures (3) which keep things nicely lined up without interfering with 
  vibration in the desired direction. 

  A commercial vibration shaker, like the one that can be seen in Fig.\ 3, has 
  the same features, except that it does not have a cone. Instead, there is a 
  threaded boss attached to the coil, which can be used to connect the shaker 
  to the object you want to vibrate. Shakers like this come in many different 
  sizes: Fig.\ 4 shows a selection. This kind of shaker is fine for driving a 
  relatively heavy structure, like the undergraduate experiment seen in Fig.\ 
  3, but it is not immediately suitable for a lightweight object like a violin. 
  If something like this was attached to a violin bridge it would act like a 
  mute, changing the behaviour significantly by adding mass, and probably 
  adding damping as well. 

  However, moving-coil methodology was successfully used to measure frequency 
  response functions of violins in earlier times. To solve the problem of added 
  mass, some ingenuity was needed. One approach was to attach to the instrument 
  just the bare coil extracted from a loudspeaker. A magnet could then be 
  brought near, and by driving the coil electrically, a swept-sine test could 
  be made. This was the favoured test method of \tt{}Carleen Hutchins\rm{}, the 
  doyenne of scientific violin making in the United States in the second half 
  of the 20th century. The converse strategy has also been used: a very light 
  ceramic magnet can be glued to the violin bridge, and a coil brought close to 
  it to provide the force. 

  A.2 Impulse hammers 

  However, neither strategy is commonly used these days in the musical 
  instrument world. Moving-coil shakers have been almost entirely replaced by 
  the second major type of vibration actuator, the impulse hammer. The great 
  virtue of this kind of actuator is that once the hammer has bounced off the 
  structure, there is nothing remaining in contact. No added mass, no added 
  damping. 

  We have already seen an impulse hammer in action, back in section 5.1. As we 
  just saw with shakers, impulse hammers come in a variety of sizes: two are 
  shown in Fig.\ 5. Both these hammers can be seen to have a cable leading out 
  of them: they both have a force-measuring sensor close to the hammer tip, 
  allowing the precise waveform of force to be measured when a structure like a 
  violin is tapped. 

  Why might we want different sizes of hammer? We already know the answer to 
  that, from section 2.2.6 where we found the frequency spectrum of an 
  idealised hammer pulse. That calculation showed that the frequency bandwidth 
  of a hammer tap is inversely proportional to the time of contact during the 
  bounce. If we want a measurement with a wide bandwidth, we need a tap with 
  very short duration. That requires two things: the surfaces in contact during 
  the tap need to be sufficiently rigid, and the mass of the hammer needs to be 
  not too high. That is the reason that instrument measurements are usually 
  done with a very light hammer like the one seen at the bottom of Fig.\ 5. But 
  if you wanted to measure vibration of something like a railway bridge, you 
  would need a heavier hammer in order to put enough energy into the structure 
  to get clean results. 

  Of course, you could excite a violin body into vibration by tapping with 
  something more familiar (and cheaper), like a pencil. You can learn quite a 
  lot from such an informal hammer test: modal frequencies and damping factors, 
  for example. But what you can’t obtain that way is a calibrated frequency 
  response function that can be directly compared with measurements made by 
  someone else with a different experimental setup. For that, you need a 
  measurement of the input force as well as the vibration response. We will 
  come back to this question of what you can expect to measure using more and 
  less sophisticated equipment in subsection C. 

  A.3 Wire-break plucks 

  We have described the two main types of vibration actuator, but we haven’t 
  exhausted all the possibilities. There are at least three other methods that 
  have been used for measurements on violins and other stringed instruments, 
  and they deserve a brief mention. First is the wire-break pluck, which we met 
  briefly back in section 7.4. You take a length of very thin copper wire, loop 
  it round some part of the structure, and pull gently until it snaps. The 
  method is particularly useful for stringed instruments: it can be used for a 
  controlled string pluck, as we saw in section 7.4, and it can also be used to 
  excite body motion by looping the wire round a string right up against the 
  string notch in the bridge. 

  This method allows the position and direction of the pluck to be very well 
  controlled, and also gives a reliably repeatable amplitude between plucks 
  because the wire breaks at more or less the same force every time. It can 
  even be used, with care, to give a calibrated frequency response. The force 
  waveform when the wire breaks will be a very sharp step. The magnitude of the 
  step can be calibrated by hanging small weights from the wire to determine 
  the breaking load, and the frequency spectrum of the step is then known 
  mathematically so that even without measuring the input force, you can tell 
  your computer a good estimate of the input spectrum. Figure 6 shows an 
  example of this method being used to measure the frequency response of a 
  violin, compared with the impulse hammer method. The wire used in this test 
  had a breaking force of 1.5~N, corresponding to a breaking weight of 153~g. 
  For more details, and examples for a cello, see Zhang and Woodhouse [1]. 

  There is a snag with wire-break testing, though. You need to support the 
  instrument sufficiently firmly that it can withstand the pull on the wire, 
  before it breaks. However, we should note that if an instrument is being held 
  by a player in the usual way, that generally gives a sufficiently firm 
  support. So there is nothing intrinsically wrong or ``unmusical'' with 
  designing a support fixture for a measurement that is similarly firm. A bit 
  more will be said about this issue in section 10.4.2 (see subsection C.1). 

  A.4 Using real playing 

  The second approach is perhaps one you already thought of — why not use real 
  playing on a stringed instrument to excite the body vibration? This has the 
  obvious advantage that it is the “real thing”, with no complications or 
  doubts to do with artificial actuators. The disadvantage is that you cannot 
  ordinarily measure the input force at the instrument bridge. But if you 
  should happen to have a bridge equipped with the kind of force sensors we 
  have seen before (see section 9.1.1), then you can indeed use regular playing 
  to give a calibrated frequency response function. 

  An example is included in Fig.\ 6, using a one-octave glissando played on the 
  lowest string of a violin. For more detail, and examples for a cello, see 
  Zhang and Woodhouse [1]. Figure 6 demonstrates quite convincingly that the 
  hammer, wire-break and bowing methods are all capable of giving very similar 
  results: the differences between the three curves are hardly bigger than the 
  differences you find if you repeat the same measurement with the same 
  instrument, but on a different day. 

  However, the sharp-eyed may notice that the curves in Fig.\ 6 look a little 
  different from the violin bridge admittances we have seen previously: there 
  isn't much evidence for a ``bridge hill'' here. The reason is an important 
  one: the additional mass of the bridge-force sensors and their associated 
  cables have a muting effect on the violin. So the comparison of the three 
  excitation methods is entirely valid, but none of them gives a really good 
  representation of how this violin would behave with its normal bridge, 
  without the added sensors. This is the reason that the published version of 
  this work [1] concentrated on the cello: the heavier bridge of a cello makes 
  the added mass of the sensors much less significant. The moral is: always 
  play the instrument with all your instrumentation in place, to check how much 
  you might be changing the behaviour. 

  Without an instrumented bridge, you can still learn something useful about a 
  violin by playing it. One of the very earliest steps in quantifying violin 
  response was the “Saunders loudness test”, in which every semitone was played 
  on the instrument, trying in each case to get the loudest sound possible. A 
  sound level meter was used to record the level. Peaks in the loudness flag up 
  resonances. The early literature talked about the “air resonance” and the 
  “main wood resonance” based on such tests, but this terminology highlights an 
  interesting issue. What was then called the “main wood resonance” is now 
  understood to be a combination of the two modes called B1- and B1+ (see 
  section 5.3, figures 5c and 5d). The loudness test, moving in semitone steps, 
  was too crude to resolve the two peaks. 

  More recently, Oliver Rodgers investigated a large number of violins using a 
  technique based on regular playing. As in the violin and cello measurements 
  described above, a glissando was played on each string. This was analysed 
  with a variant of the spectrogram, which we have seen earlier. Although the 
  input force was not measured, and the glissando playing was never precisely 
  repeatable, he was able to obtain clear results which revealed strong 
  low-frequency resonances — much more clear than the Saunders loudness curves. 
  For some examples of his results by this approach, see [2]. 

  A.5 Loudspeakers 

  The final approach to excitation is to use sound waves from a loudspeaker, in 
  a similar way to making Chladni patterns (see section 10.3). An interesting 
  application of this approach to the frequency response of a violin was 
  developed by Gabriel Weinreich [3]. He was interested in measuring the 
  monopole and dipole components of sound radiation from a violin as a function 
  of frequency (see section 4.3.1). 

  He measured these by taking advantage of a general reciprocal theorem (which 
  we mentioned back in section 5.4). Instead of driving the violin with a force 
  at the bridge and measuring the sound at many points, then averaging over 
  directions to give the monopole component, he did the converse. A set of 
  loudspeakers surrounding the violin all played an identical signal, giving 
  (approximately) an omnidirectional sound field. He then measured the 
  mechanical response at the bridge of the violin, using a phonograph pickup. 
  The same set of loudspeakers could also be driven in ways that produced 
  dipole sound fields in all the possible orientations, and the measurement 
  repeated with each of those. 

  There is a much more obvious application for a loudspeaker as an actuator. I 
  have been concentrating up to now on stringed instruments, but if the 
  frequency response you would like to measure is of a wind instrument, then 
  some kind of loudspeaker is a natural choice. Figure 7 shows the input 
  impedance of a saxophone being measured: the figure is reproduced from Fig.\ 
  5 of section 8.5. The mouthpiece end of the tube is being driven by a 
  loudspeaker inside the canister. 

  \samsection{B. Sensors} 

  Now we have surveyed the main options for vibration actuators, we turn to the 
  corresponding range of sensors to register the result of excitation. Again, 
  the possibilities range from hi-tech and expensive to simple and home-made. 
  If we want to measure sound pressure, as in the wind instrument example we 
  have just seen, we are bound to use a microphone of some kind as our sensor 
  (although the actual details of the measurement seen in Fig.\ 7 need 
  ingenuity: see the next link). 

  For a stringed instrument or a percussion instrument, things are more 
  complicated. We might be interested in the sound radiated, and thus use a 
  microphone as our sensor. Such microphone measurements raise a number of 
  important but tricky issues, and I will defer a discussion until section D. 
  However, we might be more directly interested in the vibration of the 
  structure. For that, there is a wide choice of sensors. 

  B.1 Accelerometers 

  Many of the options measure relative motion between two points, and so they 
  require some kind of fixed reference position near the vibrating object. But 
  there are exceptions. The most widespread of these is the accelerometer: some 
  examples of accelerometers are shown in Fig.\ 8. You simply stick one of 
  these to your structure, and it gives an electrical voltage signal 
  proportional to the acceleration at the attachment point in a particular 
  direction, usually perpendicular to the surface. Some accelerometer units 
  actually include three separate sensors, aligned in perpendicular directions. 
  Called “triaxial accelerometers”, they give three output signals so that all 
  components of the acceleration are measured simultaneously. 

  Some accelerometers are very small indeed: they are made using silicon chip 
  fabrication technology, and they can be integrated on the chip with their 
  supporting electronics. Such devices are called MEMS accelerometers, standing 
  for “Micro-electro-mechanical systems”. They are all around you, although you 
  may not be aware of it. Many mobile phones include triaxial MEMS 
  accelerometers so that your phone can respond to motion (and can also sense 
  gravity, so that the phone knows which way up it is). All modern cars are 
  fitted with airbag safety systems, and these use MEMS accelerometers to 
  detect the sharp acceleration (or, usually, deceleration) which signals a 
  collision and so fires the airbag. There are many other types of MEMS device, 
  for example gas sensors used to detect pollution, but this is not the place 
  to explore them: see \tt{}this Wikipedia page\rm{} for more information. 

  All accelerometers work in a similar way, taking advantage of Newton’s law. 
  They contain a mass, attached to a force-measuring sensor. When the 
  accelerometer is subjected to an acceleration, the mass inside it must be 
  pushed by a force proportional to the acceleration, and this force is 
  measured and gives the output signal. Because they have a mass supported on 
  some kind of spring to provide the force sensor, all accelerometers have an 
  internal resonance frequency. This governs their useful frequency bandwidth: 
  once the frequency approaches the resonance frequency, the output is no 
  longer simply proportional to acceleration. 

  There are two main types of force sensor. Accelerometers like the ones seen 
  in Fig.\ 8, intended for high-frequency measurement, use a piece of 
  piezo-electric crystal. Accelerometers that are intended for very low 
  frequencies use some kind of mechanical spring with a strain gauge attached. 
  The key difference is that a strain gauge works down to the very lowest 
  frequencies, and even for static deformation. This means that such 
  accelerometers respond directly to the force of gravity. This provides a very 
  simple way to check the calibration of the device: turn it upside down, and 
  gravity reverses. The jump in output level will be twice the value of $g$, 
  the acceleration due to gravity: about $9.81\mathrm{~m/s}^2$ at sea level. 
  The down-side is that the strain-gauged sensing element will be much less 
  stiff than a piezo-electric crystal, so such accelerometers have a rather low 
  resonance frequency. 

  We commented earlier that if you use some kind of shaker as your vibration 
  actuator, you need to be careful about the mass this adds to your structure — 
  especially if the structure is very light in weight, like most musical 
  instrument bodies. Well, exactly the same consideration applies to sensors. 
  Convenient though accelerometers are, they always add some mass. For musical 
  instrument measurements, we usually need to use the very lightest of 
  accelerometers. The smallest one shown in Fig.\ 8 weighs a bit under 2~g, and 
  it is just about light enough --- but an even lighter one would be 
  preferable. The others in that picture are far too heavy: they are designed 
  for testing large engineering structures. Figure 9 shows one of these small 
  accelerometers attached to the soundboard of a harp. 

  B.2 Laser vibrometers 

  The best cure for the added-mass problem is, of course, to use a sensor that 
  adds no mass. This is the big virtue of a laser-Doppler vibrometer. The only 
  thing impinging on the structure is a laser beam, which adds no mass. You can 
  see one in use in Fig.\ 10 — in a domestic room, because this measurement 
  took place during Covid lockdown! The laser unit on its tripod is in the 
  lower left of the image. The red spot from the laser beam can be seen on the 
  bridge of a banjo. A miniature impulse hammer on a pendulum fixture is 
  striking the banjo bridge near the laser spot. Both signals are then 
  collected by the laptop visible on the right, to be processed into the 
  frequency response function. 

  Laser vibrometers have two disadvantages. First, they are very expensive so 
  they are usually only found in laboratory settings. Second, they are 
  surprisingly fiddly to use — much more difficult than an accelerometer. In 
  order to measure the Doppler shift from the vibrating structure, a 
  sufficiently strong reflection of the laser beam must be detected by the 
  vibrometer. This is usually helped by a small piece of retro-reflective tape, 
  of the kind used for safety stripes on a cyclist’s clothing. But even so, it 
  can be tricky to obtain a really high-quality signal. 

  B.3 Commercial pickups 

  These types of “laboratory-grade” sensors do not exhaust the possibilities. 
  One obvious thing for a stringed instrument is to use one of the many types 
  of commercial pickup, designed to allow a performer to add some amplification 
  to the sound of an acoustic instrument. For example, these may be 
  incorporated within or under the bridge, or they may be stuck to the 
  soundboard. They will all give an electrical output that responds in some way 
  to the vibration. The problem is that you usually don’t know exactly what the 
  output means: pickups are designed to sound good, not to give a 
  scientifically “clean” output signal. We will see what this means for the 
  purposes of acoustic measurement in the next subsection. 

  B.4 Magnets, coils and Faraday induction 

  Another class of sensors is related to the standard pickup of an electric 
  guitar. The principle of these is essentially like a moving-coil shaker, but 
  run in reverse. If you vibrate a magnet close to a coil, Faraday induction 
  will cause a voltage to appear in the coil. This can be picked up and 
  amplified. Alternatively, you can vibrate the coil while holding the magnet 
  still. Or, as in the guitar pickup, you keep both the magnet and the coil 
  still, while you vibrate a metal object like a string close to them both. 

  There are two sensing methods that work on this principle, which are 
  sometimes used for musical instrument testing. The first is to sense string 
  vibration by placing a concentrated magnet near the point you are interested 
  in, then use the string itself rather than a coil: any string with a metal 
  component will develop a voltage across its ends, proportional to the 
  velocity at the magnet position. The second option is a standard phonograph 
  pickup, which also works by Faraday induction. As we noted earlier, Weinreich 
  used a phonograph pickup to sense the violin body vibration in connection 
  with his reciprocal sound radiation measurement. 

  B.5 Sensors and frequency bias 

  There is an important consequence of using different types of sensor for a 
  measurement. All the sensors based on Faraday induction give a signal 
  proportional to velocity, as does a laser-Doppler vibrometer. However, an 
  accelerometer obvious gives a signal proportional to acceleration. The two 
  are related by the fact that acceleration is the rate of change of velocity, 
  or in mathematical terms it is the time derivative of it. Once we move to the 
  frequency domain to plot our frequency response function, this translates 
  into the fact that acceleration emphasises high frequencies and de-emphasises 
  low frequencies, relative to velocity. 

  Figure 11 demonstrates the effect. Both curves show the same measurement, 
  plotted as velocity per unit force in red, and as acceleration per unit force 
  in blue. The red curve is a bridge admittance, of a kind we have seen in many 
  earlier plots. The blue curve is called bridge accelerance. In this plot it 
  has been scaled so that the two curves agree in the middle range. It is 
  obvious that the accelerance falls to very low values at low frequency, but 
  rises to higher levels than the admittance at high frequency. 

  Indeed, we might have used a sensor that responded to displacement rather 
  than velocity or acceleration. Such things are sometimes used: an example 
  would be a capacitance pickup (working a bit like the touch screen of your 
  phone). In that case, the resulting frequency response would be called 
  receptance, and it would show the opposite trend, rising above the admittance 
  at low frequency and falling below it at high frequency. 

  Every measurement has a limited range of reliable output, governed by the 
  background noise which is always present at some level. So the result of the 
  distinction between receptance, admittance and accelerance is to alter the 
  effective bandwidth of the measurement. Choosing a sensor to emphasise the 
  frequency range you are most interested in is important when designing any 
  measurement like this. If you use an accelerometer you must resign yourself 
  to getting rather noisy and poor data at low frequency, but good results at 
  high frequency. If you chose a displacement sensor you would have the 
  opposite effect: excellent at low frequency, but very poor at high frequency. 

  \samsection{C. What do you want to get from your measurement?} 

  Now we know a bit about the main options for actuators and sensors, we can 
  think about the consequences for a measurement of frequency response. The key 
  question, and one you should always ask yourself before plunging in to a 
  measurement project, is “What exactly do I want to know, and how accurately 
  do I want to know it?” To focus this question, it is useful to remind 
  ourselves of the anatomy of a typical frequency response, as we discussed in 
  section 5.3. 

  Figure 12 shows the bridge admittance from Fig.\ 11, annotated with some key 
  details. At low frequency, we expect to see well-separated modal peaks. In 
  the case of this violin the four peaks indicated by arrows correspond to the 
  “signature modes” A0, CBR, B1- and B1+ which we showed earlier (see the 
  animations in Figs.\ 5(a)—(d) in section 5.3). As we go higher in frequency, 
  we expect the spacing of modal peaks to remain roughly constant, while the 
  bandwidth of the individual peaks gets wider (roughly in proportion to the 
  centre frequency). The result is that the peaks start to overlap, and by the 
  time we reach high frequencies there may be several modes with significant 
  amplitude at any given frequency. 

  It then ceases to be productive to ask questions about individual modes, but 
  there may be important response trends that we can still say something about. 
  In the case of a violin, the most important trend is probably the feature 
  indicated by the green dashed line in Fig.\ 12, the so-called “bridge hill”. 
  Underneath the small wiggles in the admittance function, there is a broad 
  “formant” feature. Another hill-like feature seen in many violins is 
  indicated by the grey dashed line: it is sometimes called the “transition 
  hill”. 

  So what might you want to know, if you are trying to measure the frequency 
  response of this violin? There is a kind of hierarchy of questions of 
  increasing difficulty, and how you choose to go about the measurement depends 
  critically on where you are in that hierarchy. Perhaps you just want to know 
  the frequencies of the signature mode peaks? Perhaps you would also like to 
  know the heights of those peaks? Perhaps you want this information purely for 
  your own use, to compare with other instruments you have made or tested; but 
  perhaps you want the information in a standardised form so that you can 
  compare with measurements made by a friend with a different test setup. 
  Perhaps you want to capture higher-frequency features like the bridge hill? 
  The most challenging option of all is if you want the frequency response 
  function in order to do some kind of clever processing with it, like modal 
  analysis (see section 10.5) or the synthesised guitar and banjo sounds we 
  heard earlier. 

  If you just want to know signature mode frequencies, the simplest possible 
  measurement is probably the best. You can tap the corner of the violin bridge 
  (or whatever it is you are measuring) with a pencil, and capture the 
  resulting clonk sound with a commercial pickup or a microphone, perhaps the 
  one in your phone. An FFT of that sound will show the peaks you want. With a 
  pencil tap, you do not of course know the exact waveform of the force, but 
  you do know that it is a short pulse of some kind. That is enough: the 
  frequency spectrum of any short pulse is bound to be rather smooth and 
  featureless at low frequency, so any peaks you see in the response must come 
  from the violin, not from the input tap. The bandwidth of those peaks should 
  also be reliably captured, if you are interested in the damping of your 
  signature modes. 

  If you want to know something about peak heights, things are immediately more 
  tricky. Heights are influenced by the strength of your tap, and if you are 
  using a microphone to capture the response they are also influenced by the 
  microphone position and distance, and also possibly by the acoustics of the 
  room you are in. If you only want the information for your own comparison 
  between instruments you could still get away with something quite simple, 
  though. Instead of tapping with a pencil you might swing a small pendulum of 
  some kind. You still aren’t measuring the input force, but with a pendulum it 
  is quite easy to make successive swings of the same height so that the force 
  pulse should be quite repeatable. You could still measure with a microphone, 
  provided you always use the same one and you place it somewhere reproducible, 
  preferably quite close to the instrument to minimise any influence of room 
  acoustics. 

  However, if you want peak heights that you can compare with someone else’s 
  measurement, there is no avoiding some system that measures the input force, 
  or tells you what that force must be in some other way. If you do not have 
  access to a force-measuring hammer, this is where a wire-break pluck can be 
  useful. If you have calibrated the breaking load of your wire, then the 
  frequency spectrum of the input force is predictable, and with a small amount 
  of effort with your computer you can use it to convert your output into a 
  true frequency response measurement. But then you still have the issue of 
  calibrating your microphone, or whatever sensor you are using. I will come 
  back to microphones and also to calibration in general, a little later in 
  this section. 

  If you want to know about features at higher frequency, such as the bridge 
  hill, then you need to ensure that your input force contains enough energy at 
  all the frequencies you are interested in. If it is a hammer tap of some 
  kind, the hammer needs to be light and its face needs to be quite hard so 
  that the bounce happens quickly, giving a short pulse and a high bandwidth. 
  Alternatively, the wire-break pluck is still a good option because thin 
  copper wire breaks extremely fast so that the force jump contains a lot of 
  high-frequency information. You also have to worry about the added mass of 
  any kind of sensor you may fix to the instrument: the influence of mass rises 
  inexorably with frequency (which is why a violin mute has its main effect at 
  high frequency). 

  Finally, if you want to measure something like a bridge admittance in a form 
  that can be used for further processing then it is hard to avoid the full 
  professional-grade procedure, with a force-measuring impulse hammer and 
  either a very small accelerometer or a laser-Doppler vibrometer. For this 
  final purpose, the wire-break pluck is not really good enough. The reason is 
  that processing for modal analysis, for example, needs an accurate value of 
  the phase of the frequency response, as well as the amplitude. It is not so 
  easy to get accurate phase with a wire pluck, and (for a related reason) it 
  is also not easy to do multiple takes and use averaging to improve the 
  quality of your frequency response measurement. Similarly, a commercial 
  pickup is not usually acceptable as a sensor, because it is hard to know 
  exactly what the signal means. Such pickups have often been designed with 
  features to improve the sound, by modifying the frequency-dependent amplitude 
  and/or phase in a way that the manufacturer doesn't tell you. 

  \samsection{D. The challenge of microphone measurements} 

  There are some tricky issues associated with microphone measurements, but 
  before we move on to those, there is something simple to be aware of. There 
  are many different kinds of microphone, and you need to make a sensible 
  choice for your measurements. The main differences concern the directional 
  sensitivity. Technical measurements are usually made using omnidirectional 
  microphones: the laboratory-grade version are usually called ``condenser 
  microphones'' , and conveniently the small, cheap electret microphones fall 
  into the same category --- these are good enough for most instrument 
  measurements. 

  The microphones you will see in use by singers on stage are usually a 
  different kind. These are directional, most commonly with a cardioid 
  (heart-shaped) directional pattern. These work on a different principle from 
  condenser microphones, and give colouration features that singers like. But 
  you do not want any complicated colouration in your measurement, nor do you 
  want the potentiall pitfall of directional behaviour. Stick to the 
  omnidirectional type! For more information about types of microphone, see \tt{}this Wikipedia page\rm{}. 

  If you want to measure a frequency response function to characterise a 
  stringed instrument, it seems very natural to measure the sound radiated by 
  the instrument, using a microphone. After all, it is surely the sound of the 
  instrument that is of most interest? However, there are interesting 
  subtleties and snags associated with microphone measurements, and it is 
  important to be aware of them. These have to do with physics, and even more 
  to do with psychoacoustics and perception. “Sound”, which seems so simple, is 
  actually a very slippery thing. 

  When you listen to an instrument being played in a normal room, you probably 
  have the impression that there is a characteristic “sound” of that particular 
  instrument which does not change in its essential character if you move your 
  head, or if you walk around in the room. However, as we will see shortly, 
  this is something of an illusion, created inside your head by the formidable 
  processing power of your hearing system. 

  Your visual system does something comparable. You probably have an impression 
  of being surrounded by a three-dimensional world full of visual details. 
  Actually, your eyes flit around the scene you are viewing, and only a small 
  region in the centre of your visual field has really high resolution. Your 
  brain somehow puts together a composite image from the rapidly varying 
  signals your eyes are supplying. 

  It is not even an “image” in the sense that you might stitch together camera 
  images in your computer to form a bigger panorama. Your visual system is 
  doing something far more clever: it is constantly interpreting the stream of 
  visual input in terms of recognisable components (people, clouds, trees and 
  so on). What your brain assembles is a kind of simulation model of the world 
  around you, made up from these components. This simulation model is the 
  ``scene'' that you are consciously aware of. Sometimes your brain's 
  interpretation system is fooled, and that produces what you would describe as 
  an optical illusion. 

  D.1 Effects of room acoustics 

  To start thinking about the corresponding auditory system, we can do a 
  measurement that seems intuitively sensible. If you are trying to choose a 
  new violin, you would probably play a few candidate instruments in a suitable 
  room — not too small, not too dead, but probably not a concert hall or a 
  church. So let us put a violin in a room like that, and measure the frequency 
  response from tapping the bridge with a small impulse hammer, and measuring 
  the sound with a microphone at a typical listener’s position. Figure 13 shows 
  two examples of such a measurement, with two different microphone positions 
  about 2~m away from the violin, and about 1~m apart. 

  Figure 13 also shows the bridge admittance of the same violin, with the three 
  “signature modes” A0, B1- and B1+ indicated: these are the three 
  low-frequency modes of a violin body that have the strongest sound radiation. 
  Well, all three do show up in the microphone measurements, but they do not 
  look like the simple peaks in the admittance plot. Instead, they look more 
  like the “formant” features we talked about before: broad peaks “dotted out” 
  by a lot of narrower peaks. Furthermore, the details of those narrower peaks 
  are different for the two microphone positions. 

  What is happening, as you have probably guessed, is that we are seeing 
  evidence of the acoustics of the room, as well as the acoustics of the 
  violin. The room acoustics is producing many, many extra peaks. Indeed, at 
  higher frequencies the extra peaks are so dense that they cannot be resolved 
  in this plot. Figure 14 shows a zoomed view of the same two curves, in a 
  frequency range around 2.5~kHz. 

  You might imagine that the peaks seen in Fig.\ 14 correspond to acoustic 
  modes of the room, but you would be quite wrong. The medium-sized domestic 
  room where this measurement was done is a bit bigger than the one that was 
  modelled in order to generate Fig.\ 17 of section 4.2, so we can deduce that 
  in this frequency range the room has more than 10,000 modes in every 100Hz 
  band! But if you count the small wiggles in the red or blue curves, you see 
  no more than about 15 in each 100~Hz band. 

  Each of these wiggles is a combined result of perhaps 1000 modes of the room, 
  which happen to conspire to produce a bigger sum total at the wiggle peak. 
  Between the peaks, we sometimes see very sharp dips: these are frequencies 
  where the 1000 room modes conspire to cancel at that particular microphone 
  position. The combination of rounded peaks and sharp dips is characteristic 
  of any system with high modal overlap, like our room. Notice that both peaks 
  and dips fall at entirely different frequencies in the red and blue curves: 
  another characteristic of high modal overlap. 

  D.2 Reducing the influence of the room 

  The conclusion seems to be that a measurement like the ones seen in Fig.\ 13 
  is not a really good way to see the characteristics of the violin, 
  independent of the room. There are various things we might do to improve 
  things — but we will see that those come with their own issues, and even a 
  hint of paradox. Two things we might try are to move the microphone closer to 
  the violin, and to choose a room which is more “dead” in order to do the 
  measurement. Both of those seem likely to reduce the relative influence of 
  the room acoustics. 

  Figure 14 illustrates the result of those two things. The red curve shows a 
  measurement of the same violin in the same room, but with the microphone 
  moved to be directly in front of the violin, at a distance corresponding 
  approximately to the length of the top plate of the instrument. The blue 
  curve shows a measurement with the same setup and microphone placement, but 
  in a deader room. This is still a domestic room, but one with carpets, 
  curtains and a lot of soft furnishing. 

  It is clear from the figure that both changes have had an effect. The 
  low-frequency peaks in both curves are much more recognisably related to the 
  bridge admittance plot, and the blue curve is significantly smoother than the 
  red curve. But already there is a snag we should consider. The microphone is 
  now sufficiently close to the violin that it is definitely not in the 
  acoustic “far field”. The sound measured in this position will not reflect 
  accurately the way this violin will send sound energy out into a large 
  concert hall. Of course, both ears of a violinist are at least this close to 
  the vibrating surface of the violin body, so the player also cannot judge how 
  the instrument will sound from a distant seat in the auditorium. 

  The ultimate extension of the idea of a “dead” room is an anechoic chamber, 
  like the one seen in the background of Fig.\ 7 above. This is a laboratory 
  facility specifically designed to have virtually no sound reflections from 
  floor, walls or ceiling so that a sound source will behave as if it is in 
  empty space, radiating sound which never comes back. It achieves this effect 
  by lining all the walls with deep wedges of foam, as you can see in the 
  picture. So the “scientific” answer to measuring sound radiated by a violin 
  is to place the microphone far enough away from the instrument in such an 
  anechoic chamber. 

  But this does not eliminate difficulties. As we saw back in section 4.3, a 
  violin (or any other vibrating object) will not send sound equally in all 
  directions except at the very lowest frequencies. As frequency goes up, the 
  instrument becomes increasingly directional — look back at the animation in 
  Fig.\ 9 of section 4.3, which showed the directivity pattern of a violin as a 
  function of frequency. So it is not good enough to have a single microphone 
  position in the anechoic chamber. Ideally, we need to surround the instrument 
  with microphones, in order to sample the sound field in all directions. This 
  indeed is how the animation from section 4.3 was generated: data was 
  collected by George Bissinger using 264 microphone positions in a 
  latitude-longitude distribution around the violin, as shown in Fig.\ 15 [4]. 

  But surely we have been led to a bit of a paradox? An anechoic chamber is the 
  worst possible place to perform music, or to listen to it. Players can’t hear 
  themselves clearly, nor the others in their ensemble. A listener is acutely 
  aware of the directional nature of sound from an instrument: if you walk 
  around a violinist performing in an anechoic chamber, the sound is very 
  different from different angles — the opposite of the common experience in a 
  normal performance space. So to obtain a “scientific” measurement of the 
  sound radiation behaviour of a violin, we need to test it in this least 
  musical of environments, the exact opposite of where you would choose to 
  judge the quality of the instrument. 

  This brings us back to the nature of human perception. When we listen to a 
  violin being played in a normal room, our brain apparently builds some kind 
  of stable “sound image” of that instrument, in a way that is not fully 
  understood. This allows you to walk around, quite unaware of the huge sound 
  changes from point to point suggested by Figs.\ 13 and 14. We already know 
  about one ingredient of the process, from section 6.2. We talked there about 
  the “precedence effect” and the idea of “echo fusion”. Our brains are 
  capable, at least to an extent, of recognising early echoes as being extra 
  copies of the direct sound. The information in these echoes is combined with 
  the information in the direct sound to build up a composite perception. Far 
  from confusing the sound, these early echoes can fill in missing information. 

  Of course, the early reflections in a typical room come from a range of 
  directions: from the walls, floor, ceiling, and perhaps furniture. This means 
  that these early echoes give the brain an opportunity to sample the 
  directional sound field of the violin from several directions, and then 
  combine these into a composite perception. This must be at least part of the 
  underlying explanation for the “stable sound image”, compared to the 
  situation in an anechoic chamber where the brain only has the direct sound 
  from a single direction to work with. 

  D.3 ...or embracing the effect of the room 

  Our computers and measurement systems are not as clever as our brains. We do 
  not yet know how to process measurements like the ones in Fig.\ 13 to put 
  together a corresponding “sound image” of that violin in that room. But, 
  naturally enough, people have tried to devise ways of doing measurements that 
  give at least some of the benefit of a reverberant listening space. Arthur 
  Benade advocated measuring a “room-averaged response” by deliberately moving 
  around in the room during the measurement sequence [5]. A somewhat similar 
  effect is sometimes achieved in noise measurements by using a microphone 
  suspended on a swinging pendulum. 

  Another variant was advocated by Erik Jansson. He made single-microphone 
  recordings of violin music being played in a reverberation chamber, which is 
  the opposite of an anechoic chamber: a kind of super shower cubicle, with 
  very hard walls and minimal sound absorption. He then calculated a long-time 
  averaged frequency spectrum from the recording [6]. The reverberation chamber 
  allowed the microphone to sample the directional radiation of the violin very 
  thoroughly, then the averaged spectrum gave a representation of the frequency 
  distribution of total sound energy of that particular violin, during normal 
  playing. 

  D.4 Internal microphone measurements 

  Finally, there is another way to use a small microphone to get some 
  information about an instrument body: put the microphone in through a 
  soundhole, and measure the internal sound in response to tapping on the 
  bridge. The justification for this apparently odd procedure relies on some 
  things we learned back in Chapter 4. At very low frequency the most important 
  source of radiated sound comes from the monopole component of the motion, 
  associated with volume change. 

  That volume change is a combination of the motion of the wooden parts of the 
  body and the motion of the invisible “Helmholtz piston(s)” in the 
  soundhole(s): one of them for an instrument like the guitar, two of them for 
  a violin. But we saw in section 4.2.2 that the same net volume change also 
  governs the pressure inside the body cavity, which will be approximately 
  uniform throughout the space when frequency is low enough that the wavelength 
  of sound is much longer than the body dimensions. So measuring the internal 
  pressure is a surrogate for the far-field monopole sound radiation. 

  This approach has been developed into a useful measurement procedure by Colin 
  Gough [7]. One big advantage of the approach is that it minimises the issue 
  of room acoustics, because the inside of the body cavity is reasonably well 
  insulated from external sound. Another advantage is for large instruments, 
  like the double bass. It is very hard to get far enough away to be in the far 
  field of an instrument that large, with a playing range extending to very low 
  frequency. But the internal sound measurement is actually easier on a bass 
  than on a violin, because the f-holes are bigger so it is easier to 
  manipulate the microphone into the correct position inside. 

  However, there are limitations to the usefulness of the method. The 
  justification just explained relies on very long wavelength, or very low 
  frequency. How low? From section 4.1 we know that the key condition is that 
  the “Helmholtz number” should be small. In section 4.3 we expressed that 
  approximately in terms of the circumference of the radiating body: the 
  wavelength of sound needs to be significantly longer than this circumference. 
  For a violin body, taking the circumference around the widest part of the 
  lower bout of the body leads to the condition that frequency must be lower 
  than about 680~Hz. That just about encompasses the three “signature modes” 
  A0, B1- and B1+, but the argument breaks down after that. 

  Something else happens at a frequency in the vicinity of B1- and B1+: the 
  first resonance of the internal air in the cavity, which is a standing wave 
  having approximately a half-wavelength in the length of the cavity. Gough’s 
  procedure limits the influence of that mode by placing the microphone on a 
  nodal line of this mode, and also on a nodal line of the first cross-wise 
  standing wave mode. 

  But higher in frequency, internal air modes come thick and fast. The inside 
  of a violin is just like a very small room, and the density of modal 
  frequencies will show the same rising trend that we saw in Fig.\ 17 of 
  section 4.2. As we showed in section 4.2.4, the typical spacing between 
  adjacent modes decreases in inverse proportion to the square of the 
  frequency. Somewhere in the low kHz range, the number of cavity modes inside 
  a violin will overtake the number of “wood” modes from the violin body 
  vibration. All these modes will create a trend in the internal sound 
  spectrum, which has very little to do with the external sound radiation of 
  the violin, and so is potentially misleading. The conclusion is that the 
  Gough approach is useful to study low-frequency signature modes, but it is 
  likely to become progressively less useful at higher frequencies. 

  \samsection{E. The housekeeping of frequency response measurement} 

  We have discussed a range of general issues to do with measuring frequency 
  response functions, and in the process we have found out some important 
  things, for example about the influence of room acoustics. But if you 
  actually want to do such measurements yourself, there are some additional 
  details that you need to be careful about. These are probably not of much 
  interest to the general reader, so I have tucked them away in the next link. 
  This particular side link does not have a lot of scary mathematics, though: 
  it simply has some warnings about potential traps, and hints on how to avoid 
  them. It is particularly aimed at instrument makers who may want to try some 
  acoustic measurements in their workshop, as an additional tool during the 
  process of learning to make better and better instruments. 



  \sectionreferences{}[1] A. Zhang and J. Woodhouse, “Reliability of the input 
  admittance of bowed-string instruments measured by the hammer method” ~ 
  Journal of the Acoustical Society of America \textbf{136}, 3371-3381, (2014). 

  [2] Oliver E. Rodgers and Pamela J. Anderson, “An engineering approach to the 
  violin making problem”, Catgut Acoustical Society Journal \textbf{4}, 2, 
  13—19 (2000). 

  [3] Gabriel Weinreich “Sound hole sum rule and the dipole moment of the 
  violin”, Journal of the Acoustical Society of America \textbf{77}, 710—718 
  (1985). 

  [4] George Bissinger and John Keiffer: “Radiation damping, efficiency and 
  directivity for violin normal modes below 4 kHz”, Acoustics Research Letters 
  Online \textbf{4} (2003), DOI 10.1121/1.1524623. 

  [5] Arthur H. Benade; “Fundamentals of Musical Acoustics”, Oxford University 
  Press (1976), reprinted by Dover (1990). 

  [6] E. V. Jansson “Long-time-average spectra applied to analysis of music, 
  Part III: A simple method for surveyable analysis of complex sound sources by 
  means of a reverberation chamber”, Acustica \textbf{34}, 275—280 (1976). 

  [7] Colin Gough, “Acoustic characterisation of string instruments by internal 
  cavity measurements”, Journal of the Acoustical Society of America 
  \textbf{150}, 1922—1933 (2021), DOI: 10.1121/10.0006205. 