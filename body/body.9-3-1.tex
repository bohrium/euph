  In this section we derive the expressions for minimum bow force and maximum 
  bow force used in Schelleng's diagram. These build on earlier work by Raman, 
  but for brevity I will call them ``Schelleng's'' limits. 

  The first step is to calculate the slipping speed $v_s$ during an ideal 
  Helmholtz motion. If the bow speed is $v_b$ and the bowing point is at 
  position $\beta L$ on a string with vibrating length $L$, we can calculate 
  $v_s$ easily from the condition that the integral of the string velocity over 
  a complete period must be zero, in order that the string has no net sideways 
  displacement. In an ideal Helmholtz motion with period $P$, the string is 
  sticking for time $(1-\beta) P$ and slipping for time $\beta P$ in every 
  cycle. So the condition is 

  $$v_b (1-\beta) P + v_s \beta P = 0 \tag{1}$$ 

  so that 

  $$v_s=-v_b \left(\dfrac{1-\beta}{\beta} \right)=-v_b \left(\dfrac{1}{\beta} 
  -1 \right) \tag{2}$$ 

  and the velocity jump from sticking to slipping is $v_b-v_s = v_b/\beta$. 

  \fig{figs/fig-0797953f.png}{Figure 1. Coefficient of friction against string 
  velocity, for a case with bow speed $v_b=0.1$ m/s, illustrating the condition 
  for the maximum bow force when the Helmholtz slip speed $v_s=1$ m/s.} 

  Schelleng's formula for the maximum bow force now follows immediately from 
  the graphical construction relating to frictional hysteresis, explained in 
  section 9.2. In order for Helmholtz motion to be possible under the 
  constraints of the hysteresis rule, the slip speed $v_s$ must be outside the 
  range of the hysteresis. The limiting case is illustrated in Fig.\ 1, for a 
  case with $v_b=0.1$ m/s and $v_s =1$ m/s. The sloping line passes through the 
  point corresponding to $v_s$, and if it were any further to the left, $v_s$ 
  would be inaccessible. We thus obtain the force criterion for the maximum bow 
  force $f_{max}$ by equating two slopes: 

  $$\dfrac{f_{max} (\mu_s -- \mu_d)}{v_v/\beta}=2 Z_0 \tag{3}$$ 

  so that 

  $$f_{max}=\dfrac{2Z_0 v_b}{\beta (\mu_s -- \mu_d)} \tag{4}$$ 

  where $Z_0$ is the characteristic impedance of the string, $\mu_s$ is the 
  maximum coefficient of friction during sticking (1.2 for the case plotted in 
  Fig.\ 1) and $\mu_d$ is the coefficient of sliding friction at the slip speed 
  $v_s$. 

  The condition for minimum bow force needs to take energy dissipation into 
  account. Schelleng modelled the body as a mechanical resistance (or 
  ``dashpot'') with impedance $R$, assumed to be much greater than the 
  impedance of the string, $Z_0 = \sqrt{Tm}$ where $T$ is the tension and $m$ 
  the mass per unit length. This dashpot is driven into motion by the force at 
  the bridge. For ideal Helmholtz motion, we know that this is a sawtooth wave. 
  From the geometry, we can deduce the details: for a note with fundamental 
  frequency $f_0$ it is a ramp with slope $Tv_b/\beta L$, interrupted by jumps 
  of magnitude $-Tv_b/\beta L f_0=-2v_b Z_0/\beta$ using the standard formula 

  $$f_0=\dfrac{1}{2L}~\sqrt{\dfrac{T}{m}} . \tag{5}$$ 

  The body dashpot will respond to this force with a velocity waveform of the 
  same shape, divided by the factor $R$. We can integrate this to give the 
  displacement at the bridge: one cycle can be written in the form 

  $$y=\dfrac{T v_b t^2}{2 \beta L R} + K, \mathrm{~~~~~}-\dfrac{1}{2f_0} < t < 
  \dfrac{1}{2f_0} \tag{6}$$ 

  where $K$ is a constant of integration. We have chosen an interval of time 
  which makes the displacement waveform symmetrical, by choosing $t=0$ to occur 
  in the middle of one ramp phase of the Helmholtz sawtooth. This ensures that 
  successive cycles of the displacement waveform join up neatly, with a 
  discontinuity of slope but no discontinuity of displacement. 

  This displacement at the bridge creates additional force at the bow. 
  Schelleng used a simple approximation at this stage, by assuming that the 
  bowed point is close to the bridge. The short section of string between 
  bridge and bow can then be treated quasi-statically, because it is 
  approximately straight for nearly all the time. It follows that the 
  perturbing force at the bow is given by 

  $$f_{pert} \approx \dfrac{Ty}{\beta L} = \dfrac{T^2 v_b t^2}{2 \beta^2 L^2 R} 
  + \dfrac{TK}{\beta L} . \tag{7}$$ 

  The constant of integration is determined by noting that the perturbing force 
  must be (approximately) zero during slipping because the total force is then 
  uniquely determined by the friction curve. Slipping occurs at the moment when 
  the displacement waveform has its slope discontinuity, corresponding to the 
  flyback of the sawtooth, so we set $f_{pert}=0$ at $t=1/2f_0$. The result is 

  $$\dfrac{TK}{\beta L} \approx -\dfrac{T^2 v_b}{8f_0^2 \beta^2 L^2 R}= 
  -\dfrac{Z_0^2 v_b}{2 \beta^2 R} . \tag{8} $$ 

  For the Helmholtz motion to be possible, the force $f_{pert}$ must evoke an 
  equal and opposite component of the friction force, which from equation (7) 
  will be a parabolic arch with peak value $-TK/\beta L$ at time $t=0$. So 
  finally, the condition for minimum bow force is that this value, added to the 
  steady friction force $f_b \mu_d$, just reaches the limiting value $f_b 
  \mu_s$. Thus 

  $$f_{min} \approx \dfrac{Z_0^2 v_b}{2 R \beta^2 (\mu_s-\mu_d)} . \tag{9}$$ 

  This is Schelleng's result. 

  Combining this with equation (4), we can deduce 

  $$\dfrac{f_{max}}{f_{min}}=4 \beta \dfrac{R}{Z_0} . \tag{10}$$ 

  It follows that, within these approximations, there is a value $\beta=Z_0/4R$ 
  below which the minimum bow force is bigger than the maximum bow force. In 
  other words, there should be a limit to how near the bridge the bow can be 
  placed, if Helmholtz motion is to be possible. 