  In the main discussion of reed instruments in section 11.3, we will look at 
  the clarinet and the soprano saxophone, and in both cases we will find that 
  there is an unexpected analogy with the behaviour of a bowed violin string. 
  If you have read Chapter 9 before reaching this point, these analogies will 
  make immediate sense to you. But if you have dipped in at this point without 
  knowing much about bowed strings, you will find it helpful to read a short, 
  non-technical summary of some key points: that is the purpose of this 
  section. In order to fully appreciate the analogies between bowed strings and 
  reed instruments like the clarinet or saxophone, you need to know the 
  material covered here — the effects we are interested in were studied first 
  in the bowed-string context. 

  The bowed-string story begins back in the 1860s with Hermann von Helmholtz, 
  who first observed and described the way that a violin string usually 
  vibrates. Figure 1 shows an animation of this “Helmholtz motion”, reproduced 
  from Fig.\ 2 of section 9.1. At any given moment, the string has a rather 
  unexpected triangular shape: two straight portions separated by the 
  “Helmholtz corner”. This corner shuttles back and forth between the bridge 
  and the player’s finger. 

\moobeginvid\begin{tabular}{ccc} \vidframe{ 0.30 }{ vids/vid-69279fd5-00.png }&\vidframe{ 0.30 }{ vids/vid-69279fd5-01.png }&\vidframe{ 0.30 }{ vids/vid-69279fd5-02.png } \end{tabular}\caption{Figure 1. The ``Helmholtz motion'' of a bowed string. The upper plot shows the string motion (with an exaggerated vertical scale), the lower plot shows the corresponding waveform of force exerted on the violin bridge, which excites body vibration and creates the sound of the instrument.}\mooendvideo

  If you watch the point where the string crosses the moving bow in the upper 
  animation of Fig.\ 1, you can see that all the time the corner is making the 
  long journey to the finger and back, the string is moving at the same speed 
  as the bow: the string is sticking to the bow throughout that time. But when 
  the corner travels to the bridge and back, the string is slipping across the 
  bow hairs. The motion of the string is a stick-slip vibration, and the 
  frequency is governed by the time-keeping role of the travelling corner. 
  Since that corner travels at the normal wave speed on the string, this 
  explains why the frequency of a bowed note is the same as the frequency of a 
  plucked note on the same string. 

  The lower plot in Fig.\ 1 shows the developing waveform of the force exerted 
  by the vibrating string on the bridge of the violin. This “bridge force” is 
  important for bowed-string studies because it is something we can easily 
  measure. However, for the purpose of understanding the analogy with reed 
  instruments a different waveform is more useful. This is the velocity 
  waveform of the string at the point where the bow interacts with it. Figure 2 
  shows a simulated example of this waveform, alongside the bridge-force 
  sawtooth wave. The stick-slip vibration results in a pulse-like waveform. 
  Most of the time the string velocity is equal to the bow speed, producing the 
  flat top part of the waveform. The short episodes of slipping result in 
  pulses of negative velocity. Pulse waveforms rather like this will appear 
  when we look at the waveform of pressure inside the mouthpiece of a clarinet 
  or saxophone. 

  \fig{figs/fig-fc61a10c.png}{Figure 2. A simulated example of Helmholtz motion 
  in a bowed string. The upper plot shows the bridge force, as in the lower 
  part of Fig. 1. The lower plot shows the waveform of string velocity at the 
  bowed point.} 

  However, something everyone knows about the violin is that it doesn’t always 
  produce the musical sound you wanted. A bowed string is capable of vibrating 
  in many other ways, and a beginner on the violin has to learn to control 
  their bow so as to create Helmholtz motion, rather than any of these other, 
  less desirable, types of string motion. In a famous study of the bowed string 
  in the early 20th century, the Indian physicist C. V. Raman (who later became 
  famous for work on spectroscopy) gave an ingenious argument that allowed 
  these other types of bowed-string motion to be described and classified. He 
  argued that they can all be described in terms of travelling “Helmholtz 
  corners”. The difference was that Helmholtz motion had a single corner, but 
  the undesirable types of motion had more than one corner. 

  The simplest example would have two corners, and lead to string motion with 
  two episodes of slipping per cycle, rather than a single episode. This kind 
  of “double-slipping motion” is often described by violinists (or critical 
  violin teachers) as “surface sound” or “not getting into the string 
  properly”. The corresponding string velocity waveform has two pulses in each 
  cycle. We will see in section 11.3 that an instrument like the saxophone can 
  exhibit a very similar type of pressure waveform. This is no coincidence: 
  Raman’s ingenious argument can be applied equally well to reed instruments. 

  Double-slipping motion played a crucial role in the next stage of our 
  exploration of bowed-string behaviour. In section 9.3 we introduced the 
  “Schelleng diagram”: a schematic version is shown in Fig.\ 3, reproduced from 
  Fig.\ 1 of section 9.3. When a violinist is trying to control the sound of a 
  single, steady note, they have three main parameters to think about: the bow 
  speed, the force with which they press the bow against the string, and the 
  position of the bow’s contact point on the string. Forget for the moment 
  about the bow speed, and think about the interaction between bow force 
  (sometimes called “bow pressure” by violinists who are a bit shaky about 
  physics) and bow position. 

  \fig{figs/fig-2522e306.png}{Figure 3. Schelleng's diagram, reproduced from 
  Fig. 1 of section 9.3. In order to sustain a steady Helmholtz motion, a 
  violinist must control their bow force and bow position so as to lie within 
  the wedge-shaped region between the two red lines.} 

  In the 1970s, John Schelleng pointed out that if we represent these two 
  parameters pictorially by a point in a plane, then you can only sustain 
  steady Helmholtz motion if that point lies within a wedge-shaped region of 
  the plane. For a given bow position, there is a range of allowed bow force. 
  Below a minimum bow force, you get double-slipping motion, “surface sound”. 
  Above a maximum bow force you get some kind of raucous “crunch”. But these 
  bow force limits vary with the bow position, leading to the wedge-shaped 
  region as sketched in Fig.\ 3. The Schelleng diagram neatly encapsulates an 
  important aspect of the “playability” of a violin: a player has to learn to 
  control the combination of bow force and bow position so as to stay within 
  the wedge. 

  Well, we will find somewhat analogous diagrams when we come to think about 
  the playability of wind instruments. They will not involve bow force and bow 
  position, of course, but wind players have their own control variables such 
  as blowing pressure. We will often be able to learn something useful by 
  looking at a plane defined by a pair of these control variables, and finding 
  that the player has to remain within some region of this plane to achieve a 
  particular note and tone quality. Indeed, we will often find that it is a 
  wedge-shaped region, reminiscent of the Schelleng diagram. 