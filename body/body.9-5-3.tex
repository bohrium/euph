  We did not need to think about torsional motion when we looked at plucked 
  strings, but in a bowed string torsion can play a significant role. The first 
  step is to derive the differential equation governing small-amplitude 
  torsional motion in a string (or indeed in any other system, such as the 
  shaft of an engine). As we have done several times before, we look at the 
  forces acting on a small element of the string, as sketched in Fig.\ 1. We 
  assume the string has external radius $a$, and has rotational displacement 
  $\theta(x,t)$ driven by an external torque $M(x,t)$ per unit length. We start 
  by thinking of a string made of a single material, with density $\rho$ and 
  shear modulus $G$. 

  \fig{figs/fig-cf883e3c.png}{\caption{Figure 1. An element of string 
  undergoing torsional vibration with angular displacement $\theta(x,t)$ and 
  subjected to external torque $M(x,t)$ per unit length.}} 

  At each end of the element, there is a torque associated with the local 
  deformation, which is resisted by the shear rigidity of the material. This 
  torque takes the form $GJ \dfrac{\partial \theta}{\partial x}$, a form that 
  is completely analogous to the force $T \dfrac{\partial w}{\partial x}$ 
  associated with transverse displacement $w(x,t)$ of a string under tension 
  $T$, as we saw in section 3.1.1. $J$ is the polar moment of area per unit 
  length, which for a uniform circular string is given by 

  \begin{equation*}J=\dfrac{\pi a^4}{2} . \tag{1}\end{equation*} 

  For our string element, we have this torque acting at both ends, while the 
  external torque contributes $M \delta x$. The rotational version of Newton's 
  law then gives 

  \begin{equation*}I \delta x \dfrac{\partial^2 \theta}{\partial t^2} = M 
  \delta x +GJ\left[ \left.\dfrac{\partial \theta}{\partial x} \right|_{x+ 
  \delta x} -\left.\dfrac{\partial \theta}{\partial x} \right|_{x} \right] 
  \tag{2}\end{equation*} 

  \noindent{}where $I$ is the polar moment of inertia per unit length, given by 
  $I=\rho J$ for a uniform string. 

  In the limit $\delta x \rightarrow 0$, this gives us the governing equation 

  \begin{equation*}I \dfrac{\partial^2 \theta}{\partial t^2} -- GJ 
  \dfrac{\partial^2 \theta}{\partial x^2}= M . \tag{3}\end{equation*} 

  For the uniform string, the factor $J$ cancels out, leaving 

  \begin{equation*}\rho \dfrac{\partial^2 \theta}{\partial t^2} -- G 
  \dfrac{\partial^2 \theta}{\partial x^2}= M . \tag{4}\end{equation*} 

  This is the one-dimensional wave equation, with exactly the same form as we 
  found for the ideal string (section 3.1.1) and for plane sound waves (section 
  4.1.1). The only difference between those three cases comes in the particular 
  constants: $G$ and $\rho$ in this case. The wave speed is given by 

  \begin{equation*}c_t=\sqrt{G/\rho}. \tag{5}\end{equation*} 

  This is the shear wave speed of the material, dependent on the material 
  properties but independent of the radius of the string. However, for a 
  typical musical string with a complicated construction involving multiple 
  layers of material, the parameter $GJ$ is best regarded as an effective 
  property of the string, determined by measurement or more detailed modelling. 
  The torsional wave speed is then $c_t=\sqrt{\dfrac{GJ}{I}}$. The polar moment 
  of inertia $I$ can be calculated straightforwardly if the thickness and 
  density of each layer is known since it is defined by 

  \begin{equation*}I= 2 \pi \int_0^a{r^3 \rho(r) dr} . \tag{6}\end{equation*} 

  Because torsional waves obey the same equation as transverse waves on an 
  ideal string, many results we have already seen relating to transverse 
  vibration have direct counterparts for torsion. One important example relates 
  to impedance. Consider the excitation of torsional motion by a force $fe^{i 
  \omega t}$ applied on the surface of the string. The corresponding moment is 
  $afe^{i \omega t}$. This will generate string rotation $\theta$, which in 
  turn produces surface displacement equal to $a \theta$. 

  The point force will generate outgoing torsional waves at the frequency 
  $\omega$, symmetrically in the two directions. We can write these in the form 

  \begin{equation*}\theta(x,t)= A e^{i \omega t \pm i k x} 
  \tag{7}\end{equation*} 

  \noindent{}where the $\pm$ signs apply to the two directions of travel, $A$ 
  is an amplitude factor, and $k=\omega /c_t$ is the wavenumber for torsional 
  waves. There will be a jump in $\dfrac{\partial \theta}{\partial x}$ at the 
  point where the force is applied, which we can take to be $x=0$. Moment 
  balance at that point then requires 

  \begin{equation*}fa=2 G J A i k \tag{8}\end{equation*} 

  \noindent{}so that the surface velocity is 

  \begin{equation*}v = i \omega a A = i \omega a \dfrac{fa}{2GJ i \omega 
  /c_t}=\dfrac{f a^2 c_t}{2GJ}=\dfrac{fa^2}{2 \sqrt{GJI}} . 
  \tag{9}\end{equation*} 

  The ratio of force to velocity is independent of frequency $\omega$: torsion 
  on an infinite string behaves like a dashpot, just as we saw for transverse 
  vibration of an ideal string. That ratio is 

  \begin{equation*}\dfrac{f}{v}=\dfrac{2 \sqrt{GJI}}{a^2} 
  \tag{10}\end{equation*} 

  \noindent{}and for the case of a uniform string this reduces to 

  \begin{equation*}\dfrac{f}{v}=\dfrac{2 
  J\sqrt{G\rho}}{a^2}=\dfrac{2}{a^2}~\dfrac{\pi a^4}{2} \sqrt{G \rho}=\pi a^2 
  \sqrt{G\rho} . \tag{11}\end{equation*} 

  The corresponding result for transverse motion was $2 Z_0$ in terms of the 
  string impedance $Z_0=\sqrt{T m}$ where $T$ was the tension and $m$ the mass 
  per unit length, so we see that the torsional quantity corresponding to $Z_0$ 
  is 

  \begin{equation*}Z_t=\dfrac{\sqrt{GJI}}{a^2}=\dfrac{\pi a^2}{2} \sqrt{G\rho} 
  \tag{12}\end{equation*} 

  \noindent{}where the first expression is for the general case, and the second 
  for the case of a uniform string. 

  On an infinite string, the tangential force $f$ would result in surface 
  motion from both transverse and torsional motion, and the combined motion 
  would be the sum of these. That means that such combined motion would be 
  governed by a parallel combination of the two impedances, with a combined 
  value $Z_{eff}$ satisfying 

  \begin{equation*}\dfrac{1}{Z_{eff}}=\dfrac{1}{Z_0}+\dfrac{1}{Z_t} . 
  \tag{13}\end{equation*} 

  The two impedances $Z_0$ and $Z_t$ play a role in the final topic to be 
  examined here. Back in section 9.2 we saw the origin of the pattern of 
  Schelleng ripples, sketched in Fig.\ 13 of that section. Each time the 
  rounded Helmholtz corner passed the bow, it generated perturbations on the 
  string, reflected back from the bowed point. These then travelled up and down 
  on the string, and when they encountered the sticking bow they were totally 
  reflected from it. 

  Once we include the possibility of torsional motion, that simple picture 
  changes. Now when a velocity perturbation reaches the sticking bow, it can 
  cause motion even though sticking is not interrupted, because the string can 
  roll on the bow. Rolling consists of equal and opposite components of surface 
  motion associated with transverse and torsional waves, so that the sum of the 
  two gives zero velocity relative to the bow. This is, of course, the 
  condition for sticking. 

  The situation is then as sketched in Fig.\ 2. The incident transverse wave, 
  shown in black, produces four separate outgoing waves: transmitted and 
  reflected transverse waves, shown in red, and transmitted and reflected 
  torsional waves, shown in blue. The torsional waves generally travel faster 
  than the transverse waves. For the case of an ideal flexible string without 
  bending stiffness, the reflection and transmission coefficients for all four 
  of these waves are determined completely by the ratio $Z_0/Z_t$. We needn't 
  go through the details: they can be found in reference [1], together with the 
  more complicated results when allowance is made for bending stiffness, and 
  also for compliant behaviour of the bow. 

  \fig{figs/fig-9027832a.png}{\caption{Figure 2. Diagram in the same format as 
  Fig. 13 of section 9.2. An incident transverse wave, such as a Schelleng 
  ripple, (in black) hits the bow during an interval of sticking. When only 
  transverse motion was allowed for, the wave was totally reflected. With the 
  possibility of torsional motion, and hence rolling on the sticking bow, it 
  now results in transmitted and reflected transverse waves (in red) and also 
  transmitted and reflected torsional waves (in blue).}} 

  \sectionreferences{}[1] R. Pitteroff and J. Woodhouse, “Mechanics of the 
  contact area between a violin bow and a string. Part I: reflection and 
  transmission behaviour”; Acta Acustica united with Acustica, \textbf{84}, 
  543—562 (1998). 