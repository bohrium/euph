  The technical definition of a linear system is as follows. Suppose we know 
  that an input $f_1(t)$ produces an output $g_1(t)$, while an input $f_2(t)$ 
  produces an output $g_2(t)$. Then it will always be the case that any linear 
  combination of the two inputs will produce the same linear combination of the 
  outputs. So for example input $2f_1-f_2$ will give the output $2g_1-g_2$. 
  Provided we also assume that the properties of the system itself do not vary 
  with time (although of course the input will vary with time), then this is 
  enough information to be able to prove the ``sine wave in, sine wave out'' 
  result described in the main text. I will give the proof at the end of this 
  section. 

  The underlying mathematical reason for the importance of sine waves is that 
  they are eigenfunctions of the process of differentiation: differentiating a 
  sine wave gives another sine wave, changed only in amplitude and phase. This 
  is related to the ``sine wave in, sine wave out'' property: if you imagine 
  that your system in the ``black box'' is described by a differential equation 
  of some kind, then if the equation is linear in the usual mathematical sense, 
  it consists of a linear combination of the function and its derivatives. It 
  follows that if the function is a sine wave, then the total output will still 
  be a sine wave since each separate term in the equation gives a sine wave, 
  and adding sine waves together (all at the same frequency) still leaves 
  another sine wave. 

  For many algebraic manipulations involving sine waves, it is easier to use a 
  complex-number representation. If you are not familiar with the idea of 
  complex numbers, break off here to look at the next link. 

  Now we take advantage of the famous result 

  \begin{equation*}e^{i \omega t}=\cos \omega t+i\sin \omega t . 
  \tag{1}\end{equation*} 

  This allows us to package the amplitude and phase of a sine wave into a 
  single complex number. From equation (1), $\cos\omega t=\mathrm{Re}(e^{i 
  \omega t})$. Now suppose we have a sine wave with amplitude $a$ and phase 
  $\phi$: 

  \begin{equation*}f(t)=a\cos(\omega t + \phi)=\mathrm{Re}(ae^{i \omega 
  t+i\phi})=\mathrm{Re}(Ae^{i \omega t}) \tag{2}\end{equation*} 

  \noindent{}where 

  \begin{equation*}A=ae^{i\phi}. \tag{3}\end{equation*} 

  As a shorthand, we usually do not bother to write the $\mathrm{Re}(...)$. We 
  just talk about a sine wave $Ae^{i \omega t}$ with frequency $\omega$ and 
  (complex) amplitude $A$. Just keep in the back of the mind that if the answer 
  to a calculation involves complex numbers, then what we really mean is that 
  the physical solution is the Real part of the complex answer. 

  Now we are ready to prove the ``sine wave in, sine wave out'' property. 
  Suppose our input $f(t)=e^{i \omega t}$, and suppose the corresponding output 
  is $g(t)$. Now think about a small time shift by $\delta$. The input is 

  \begin{equation*}x(t+\delta)=e^{i \omega (t+\delta)} = e^{i \omega t} e^{i 
  \omega \delta} = x(t) e^{i \omega \delta}. \tag{4}\end{equation*} 

  By the assumption of linearity, the output must follow the same pattern: 

  \begin{equation*}y(t+\delta)=y(t) e^{i \omega \delta} \tag{5}\end{equation*} 

  \noindent{}because the term $e^{i \omega \delta}$ is simply a constant 
  multiplier. But we know from Taylor's theorem that 

  \begin{equation*}y(t+\delta) \approx y(t) + \delta \dot{y}(t) 
  \tag{6}\end{equation*} 

  \noindent{}provided $\delta \ll 1$, where $\dot{y}$ is $dy/dt$. So we have 

  \begin{equation*}y(t) + \delta \dot{y}(t) \approx y(t) e^{i \omega \delta} 
  \approx y(t) (1 + i \omega \delta) \tag{7}\end{equation*} 

  \noindent{}leading to 

  \begin{equation*}\dot{y}(t) \approx i \omega y(t). \tag{8}\end{equation*} 

  In the limit $\delta \rightarrow 0$ this equation becomes exact. It is a 
  simple first-order differential equation for the output $y(t)$, which has the 
  general solution 

  \begin{equation*}y(t) = A e^{i \omega t} \tag{9}\end{equation*} 

  \noindent{}where $A$ is an arbitrary (complex) constant. This is the result 
  we want: the output must indeed be a sinusoidal wave at the same frequency 
  $\omega$ as the input, scaled by a factor $a$ and phase-shifted by an angle 
  $\phi$ where $A=a e^{i \phi}$. 