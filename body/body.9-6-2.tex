  It is useful to understand exactly how a temperature-based friction model can 
  be incorporated into the simulation approach described in section 9.2. This 
  sheds some light on the different behaviour of alternative choices of model. 
  First, a reminder of how the original friction-curve model is dealt with. At 
  each time step, the new values of friction force and string velocity are 
  found by the geometric construction illustrated in Fig.\ 1, which is a repeat 
  of Fig.\ 7 from section 9.2. 

  The history of the string motion allows us to compute the current value of 
  the variable we called $v_h$, and this determines the position of a sloping 
  straight line. We then have to find the intersection of that straight line 
  with the curve representing friction force as a function of velocity, 
  including the vertical segment which represents the range of possible values 
  of force during sticking. Three possible positions of the straight line are 
  shown, and two versions of the friction-velocity curve corresponding to 
  different values of the bow force. For the solid red friction curve, the blue 
  line labelled c indicates an ambiguity, which is resolved by a hysteresis 
  loop involving jumps of force and velocity, as illustrated in Fig.\ 2 (which 
  is another repeat of an earlier figure). 

  The original thermal friction model is based on the assumption that the the 
  friction force is proportional to the normal bow force, and that the 
  coefficient of friction is a function $\mu(T)$ of temperature $T$ alone. In 
  order to ensure that the friction force always opposes the direction of 
  sliding motion, the friction force $f$ must then take the form 

  $$f=-F_b \mu(T) \mathrm{sgn}(v) \tag{1}$$ 

  where $F_b$ is the bow force, and $\mathrm{sgn}(v)$ denotes the sign 
  function, equal to $\pm1$ depending on whether $v$ is positive or negative. 
  What this means for the simulation is that at each time step we first 
  calculate the temperature, and hence obtain $\mu$, then to deal with the 
  $\mathrm{sgn}(v)$ term we solve a version of the graphical construction 
  illustrated in Fig.\ 3. This function has a vertical portion representing the 
  state of sticking, but during sliding the function is simply constant, so 
  there can be no jumps and no hysteresis loop. 

  There is another possible temperature-based friction model we could have 
  explored. The measurements on the bulk properties of rosin shown in Fig.\ 8 
  of section 9.6 suggest that over most of the relevant temperature range rosin 
  behaves predominantly as a viscous liquid. That would suggest a friction law 
  taking the general form 

  $$f=-F_b \zeta(T) v . \tag{2}$$ 

  The function $\mathrm{sgn}(v)$ has been replaced by simple proportionality to 
  $v$, because a viscous shear force is proportional to shear rate. This factor 
  $v$ achieves the desired effect of reversing direction according to the sign 
  of the sliding velocity, but it has a very drastic effect on the predicted 
  behaviour. Figure 4 shows the graphical representation equivalent to Figs.\ 1 
  and 3. This time, there is no barrier whatever representing ``sticking'' 
  behaviour. Simulations based on such a model will regularly show ``forward 
  slipping'', something that is never (or virtually never) seen in 
  measurements. 

  A viscous-based model like this is clearly not, therefore, a good candidate 
  for the true friction law. The explanation for this apparent conflict with 
  the measured properties of rosin lies in the particular conditions of those 
  measurements. The rheometer tests used extremely small strains and strain 
  rates, in order to keep the behaviour in the linear range. But the bowed 
  string motion we are interested in involves intervals of gross sliding. This 
  will take the material well outside the range of the rheometer tests. 

  For an extension of the original thermal model to allow the possibility of an 
  initial jump in velocity (and hence in measured bridge force) we need to look 
  for something different. There is a very simple, if entirely ad hoc, approach 
  to that question which is motivated by Figs.\ 1 and 3. We can replace the 
  function $\mathrm{sgn}(v)$ in equation (1) by something like the function 
  plotted in Fig.\ 5. The middle one of the three blue lines shows that this 
  function can exhibit multiple intersections, and therefore sometimes predict 
  jumps and hysteresis. But note that the range of velocity plotted here is 
  much wider than in Fig.\ 1: the red curve here has a falling shape which is 
  much gentler than the original friction curve. 

  For this example I have used a hyperbolic shape to replace the flat portion 
  of the function $\mathrm{sgn}(v)$. It has a horizontal asymptote at the value 
  1, and it is then fully defined by two parameters: a vertical asymptote at a 
  speed $v_{as}$, and a value $k_{lim}$ as the sliding speed tends to zero. The 
  equation of the curve then takes the form 

  $$(f-1)(v-v_{as})=-v_{as} (k_{lim}-1) . \tag{3}$$ 

  The particular function plotted here has $v_{as}=0.8$ m/s and $k_{lim}=2$, 
  which are the parameter values used for the simulation examples shown in 
  section 9.6. 

  With this modified model, we can still calculate a form for the required 
  temperature-dependent friction coefficient by requiring that the simulated 
  results for steady sliding agree with the original measurements. The result, 
  when the term $\mathrm{sgn}(v)$ in equation (1) is replaced by the particular 
  function plotted in Fig.\ 5, is shown in Fig.\ 6: it is still broadly in line 
  with expectations based on the rheometer tests and glass transition range of 
  violin rosin. 