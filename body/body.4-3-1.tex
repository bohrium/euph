  For a more careful mathematical description of monopoles, dipoles and so on, 
  we first examine the pulsating sphere result from section 4.1.2 for the case 
  of very small Helmholtz number $\omega a/c$. The amplitude factor $A$ from 
  eq. (8) then takes the approximate form 

  $$A \approx -\omega^2 a' a^2 \rho_0 = i \omega a^2 v_a \rho_0 \tag{1}$$ 

  where $v_a$ is the velocity on the surface of the pulsating sphere. We can 
  write this in terms of the net volume flux $q=4 \pi a^2 v_a$, and thus obtain 
  an expression for the far-field pressure 

  $$p=\dfrac{i \omega \rho_0 q}{4 \pi r} e^{i \omega(t-r/c)} . \tag{2}$$ 

  Using the same approximation, we can obtain the radial velocity of air 
  particles, $v(r)$, using eq. (7) of section 4.1.2: 

  $$v \approx \dfrac{q}{4 \pi r} \left( \dfrac{1}{r} + \dfrac{i \omega}{c} 
  \right) e^{i \omega(t-r/c)} . \tag{3}$$ 

  For positions $r$ within the near field, the first term in the brackets 
  dominates, and the velocity decays like $1/r^2$. Further out, the second term 
  takes over and the far field decays like $1/r$. 

  The sphere radius $a$ no longer appears explicitly in either of eqs. (2) or 
  (3). There is a limiting case, in which a vanishingly small sphere produces a 
  volume flux $q$. This will not describe a physical system because the 
  pressure and velocity are singular as $r \rightarrow 0$, but it is a result 
  of great mathematical importance. Strictly, it is this limiting solution that 
  is called an acoustic monopole. 

  Equation (2) is the pressure field produced by a volume source described by a 
  Dirac delta function. Such solutions, in which the source term in a governing 
  equation is replaced by a delta function, are known as Green's functions. 
  They can be used to assemble the solution for a more general source 
  distribution by a process of linear superposition known as convolution. For 
  the case of sound pressure, this will lead to a result called the Rayleigh 
  integral: we will return to it in section 4.3.2. 

  For the present, we are interested in a different question. We want to obtain 
  a similar approximation for a pair of monopoles with opposite phase, to 
  obtain the idealised acoustic dipole. We can use a trick. Rather than 
  calculating the pressure at a single point due to two monopoles with flux $q$ 
  and $-q$ a short distance $h$ apart, we can calculate the difference of 
  pressure in the field from eq. (2), at two points that are a distance $h$ 
  apart. Suppose the line joining these two points is parallel to the $z$ axis. 
  In the limiting case of small $h$, the result can be expressed in terms of a 
  derivative of the pressure field: 

  $$p=-\dfrac{i \omega \rho_0 }{4 \pi} (hq) \dfrac{\partial}{\partial z} \left( 
  \dfrac{e^{i \omega(t-r/c)}}{r} \right) . \tag{4}$$ 

  Now a small movement $dz$ is related to a change $dr$ via the polar angle 
  $\theta$: $dr=dz \cos \theta$, so 

  $$p=-\dfrac{i \omega \rho_0}{4 \pi} (hq) \dfrac{\partial}{\partial r} \left( 
  \dfrac{e^{i \omega(t-r/c)}}{r} \right) \cos \theta \tag{5}$$ 

  $$=-\dfrac{c \rho_0 }{4 \pi} (hq) \dfrac{\omega^2}{c^2} \left( 1+ \dfrac{c}{i 
  \omega r}\right) \dfrac{e^{i \omega(t-r/c)}}{r} \cos \theta .\tag{6}$$ 

  For this case the pressure has a near-field term decaying like $1/r^2$, given 
  by the second term in the brackets, as well as a far-field term decaying like 
  $1/r$ given by the first term in the brackets. The source strength $q$ and 
  the separation $h$ only appear in the combination $qh$, which we can define 
  as the dipole moment $M$. 

  For this problem, because of the dependence on $\theta$, the particle 
  velocity is not purely radial. However, we learn something interesting if we 
  calculate the radial component of this velocity, by the same method used in 
  section 4.1.2. The result is 

  $$v_r = -- \dfrac{1}{i \omega \rho_0} \dfrac{\partial p}{\partial r} = 
  -\dfrac{M}{4 \pi} \left(\dfrac{\omega^2}{c^2 r} -\dfrac{2 i \omega}{c r^2} 
  -\dfrac{2}{r^3} \right) e^{i \omega(t-r/c)} \cos \theta . \tag{7}$$ 

  On a surface with $r=\mathrm{constant}$, the radial velocity simply varies 
  proportional to $\cos \theta$. But this is the radial velocity produced by 
  rigid motion of a sphere in the direction of the axis from which $\theta$ is 
  defined. So, as claimed in section 4.3, the ideal dipole field is exactly the 
  same as the field generated by rigid oscillation of a sphere. 

  The same derivative trick can be used to derive the corresponding expression 
  for an ideal quadrupole source: instead of the difference of two dipoles a 
  short distance apart, we can calculate the difference of pressure at two 
  nearby points in a single dipole field. We do not need to work through the 
  details, but one qualitative fact about the solution can be noted. So far, we 
  have seen that the pressure field of a monopole source decays like $1/r$ at 
  all distances, near field or far field. For a dipole, the far field still 
  decays like $1/r$ but the near field decays more rapidly, like $1/r^2$. For a 
  quadrupole, this trend continues: the far field still decays as $1/r$, but 
  the near field decays very rapidly, proportional to $1/r^3$. This is the 
  underlying explanation of the behaviour of sound from a tuning fork, 
  described in section 4.3. 