  By thinking about the balance of kinetic energy, we can get some useful 
  information about the bandwidth that can be excited when a bouncing hammer 
  hits a vibrating structure. It is convenient to think about the problem in 
  idealised form, as sketched in Fig.\ 1. A point mass $m$ hits a chosen 
  position on the system, approaching at speed $v_1$ and rebounding with speed 
  $v_2$. After the impact, the structure is vibrating with some velocity $v(t)$ 
  at the impacted point. 

  \fig{figs/fig-524f36e4.png}{\caption{Figure 1. Sketch of a mass approaching 
  and then bouncing off a system that can vibrate.}} 

  The contact force may be a single pulse or a cluster of multiple pulses, but 
  in either case once the rebound process is complete the total impulse 
  imparted to the structure must be given by the jump in momentum of the 
  bouncing mass: 

  $$I=m(v_1+v_2). \tag{1}$$ 

  We can write the force waveform as $I f(t)$ so that 

  $$\int{f(t) dt} = 1 \tag{2}$$ 

  where the integral is taken over the entire duration of the pulse(s). The 
  response velocity $v(t)$ is then given by the standard convolution integral 
  (see section 2.2.8) 

  $$v(t)=I \int_{-\infty}^t{g(t-\tau) f(\tau) d \tau} \tag{3}$$ 

  in terms of the impulse response 

  $$g(t) = \sum_n{\dfrac{\cos \omega_n t}{M_n}} \mathrm{~~~for~~} t\ge 0 
  \tag{4}$$ 

  where mode $n$ has resonance frequency $\omega_n$ and effective mass at the 
  impacted point $M_n$. This effective mass is related to the mode shape at the 
  impacted point, which we can write in a shorthand way as $u_n(x_0)$: then 

  $$M_n=\dfrac{1}{u_n^2(x_0)} \tag{5}$$ 

  where the mode shape is assumed to have been normalised in the usual way with 
  respect to the mass distribution of the structure: see section 2.2.5, where 
  the normalisation condition was stated in discrete form in equation (12). 
  Note that the influence of damping is ignored in equation (4), because an 
  impact event is too short for damping to have a significant influence. 

  Suppose first that the impact is an ideal, instantaneous one: then 

  $$f(t) = \delta(t), \tag{6}$$ 

  the Dirac delta function (see section 2.2.8). Equations (3) and (4) then 
  reduce to the simple form 

  $$v(t) = I \sum_n{\dfrac{\cos \omega_n t}{M_n}} . \tag{7}$$ 

  The expression $I \cos \omega_n t /M_n$ is thus the velocity of the effective 
  mass $M_n$ representing mode $n$, and since we know that the mass matrix is 
  always diagonalised when we use modal coordinates, we can write a simple 
  equation representing the kinetic energy of the system before and after 
  impact: 

  $$\frac{1}{2}m v_1^2 = \frac{1}{2}m v_2^2 +\frac{1}{2} \sum_n{M_n 
  \dfrac{I^2}{M_n^2}} \tag{8}$$ 

  where the right-hand side has been evaluated immediately after the impact, at 
  $t=0+$ so that all the cosine factors are simply equal to 1. Notice that 
  equation (8) ignores any actual loss of energy during the collision: the aim 
  here is to find an upper limit on the post-impact vibration. 

  Substituting from equation (1) and writing 

  $$v_2= \lambda v_1 \tag{9}$$ 

  we obtain 

  $$1- \lambda^2 = (1+\lambda)^2 \sum_n{\dfrac{m}{M_n}} . \tag{10}$$ 

  Cancelling a factor $(1+ \lambda)$ leaves 

  $$1-\lambda = (1+\lambda)K \tag{11}$$ 

  where 

  $$K=\sum_n{\dfrac{m}{M_n}} \tag{12}$$ 

  and so 

  $$\lambda = \dfrac{1-K}{1+K}. \tag{13}$$ 

  But now we hit a snag. The speed ratio $\lambda$ must be positive for a 
  rebound, and for that to happen equation (13) requires $K \le 1$. The 
  limiting case would have $K=1$, which is the condition for all the kinetic 
  energy from the original moving mass to turn into vibration in the structure, 
  leaving nothing for a rebound speed. But for a structure with an infinite 
  number of modes, $K \rightarrow \infty$ because nearly all modes have a value 
  of $M_n$ which is of the order of the total mass of the structure. 

  To see this, we can try a more realistic (but still idealised) case. We can 
  use the contact force waveform found in section 2.2.6 and reprised in section 
  12.1: a half-cycle of sine wave, which is the result of hitting a rigid 
  surface through a linear contact spring. The contact resonance frequency 
  $\Omega$ is now a variable, which we can use to describe impacts of different 
  durations. Specifically, we can write 

  $$f(t)=A \cos \Omega t \mathrm{~~~for~~} -T \le t \le T \tag{14}$$ 

  and zero outside that range, where the contact duration is $2T$. Since this 
  duration is a half-cycle at the frequency $\Omega$, we have 

  $$T=\dfrac{\pi}{2 \Omega}. \tag{15}$$ 

  The pulse is sketched in Fig.\ 2. The constant $A$ is determined by the 
  unit-area condition (2), with the result 

  $$A=\dfrac{\pi}{4T}. \tag{16}$$ 

  \fig{figs/fig-3d62b640.png}{\caption{Figure 2. The idealised force pulse when 
  a mass bounces off a rigid surface, mediated by a linear contact spring.}} 

  Now we can evaluate the convolution integral (3) with the impulse response 
  (4). After a little algebraic manipulation, the result for the velocity 
  $v(t)$ at time $t=T$ immediately after the pulse finishes is 

  $$v(T)=\dfrac{I}{2} \sum_n{\dfrac{1}{M_n} \left\lbrace 
  \dfrac{\Omega^2}{\Omega^2 -- \omega_n^2} (1 + \cos \omega_n \pi/\Omega) 
  \right\rbrace} . \tag{17}$$ 

  The effective velocity of the modal mass $M_n$ just after the pulse finishes 
  is thus 

  $$v_n=\dfrac{I}{M_n} \dfrac{\Omega^2}{\Omega^2 -- \omega_n^2} \dfrac{(1 + 
  \cos \omega_n \pi/\Omega)}{2} \tag{18}$$ 

  and the kinetic energy of the vibrating structure is $(1/2)\sum_n{M_n 
  v_n^2}$. Substituting this into the equivalent of equation (8), we can obtain 
  an equation identical to equation (11) if we replace $K$ by 

  $$K^\prime (\Omega) = \sum_n{\dfrac{m}{M_n} \left\lbrace 
  \dfrac{\Omega^2}{\Omega^2 -- \omega_n^2} \dfrac{(1 + \cos \omega_n 
  \pi/\Omega)}{2} \right\rbrace^2} . \tag{19}$$ 

  It is reassuring to see that we recover the delta-function version (12) in 
  the limit $T \rightarrow 0$ so that $\Omega \rightarrow\infty$, as we would 
  expect. But with a finite value of $\Omega$ the problem of $K$ being infinite 
  has gone away. Once the modal frequency $\omega_n$ grows much bigger than 
  $\Omega$, the term inside the curly brackets dies away: the finite-length 
  impulse has the effect of a low-pass filter on the modal sum. 

  We can see the effect most clearly by an example. Figure 3 shows the function 
  $K^\prime(\Omega)$ for the particular case of a rectangular plate of total 
  mass 200~g and lowest resonance frequency 100~Hz being impacted by a 
  ``hammer'' with mass 5~g. The plate is the same one used for some of the 
  examples in section 12.1, with details given in section 12.1.1. The function 
  $K^\prime$ satisfies $K^\prime(0)=0$, and we have already noted that it tends 
  to infinity as $\Omega \rightarrow \infty$, so there is always at least one 
  value of $\Omega$ for which $K^\prime=1$. In practice, as in the example in 
  Fig.\ 3, there usually seems to be just one such value, which represents the 
  shortest possible impact (of the chosen form (14)) for which a rebound might 
  occur without multiple contacts. It is also the condition for maximum energy 
  transfer from the moving mass to the vibrating plate. 

  The interpretation is that if you want to hit this plate with a hammer of 
  this particular mass, you should equip it with a sufficiently soft tip that 
  the contact resonance is below the threshold value (around 3.5~kHz) if you 
  want even a fighting chance of tapping with a single impact. In reality this 
  condition gives an upper limit, but it should give a guide to the order of 
  magnitude and the trend. In real impacts it is of course possible that 
  multiple contacts could occur at significantly lower values of the contact 
  stiffness --- some computed examples are shown in section 12.1, and we will 
  see others in later sections of this chapter. 

  In Fig.\ 4, the threshold value of contact resonance frequency is plotted for 
  a wide range of hammer masses. The same plate is being tapped in all cases. 
  Figure 5 shows the same data in a different form, plotting the threshold 
  contact time as a function of hammer mass. 

  \fig{figs/fig-0898e2c2.png}{\caption{Figure 4. The threshold contact 
  resonance frequency for which $K^\prime = 1$, as a function of hammer mass, 
  when striking the same plate as used in Fig. 3. Logarithmic scales are used 
  on both axes.}} 

  \fig{figs/fig-442ccb6e.png}{\caption{Figure 5. The threshold contact duration 
  in ms for which $K^\prime = 1$ expressed as a function of hammer mass, using 
  the same data as for Fig. 4. Logarithmic scales are used on both axes.}} 

  Finally, we should note that this approximate calculation can be easily 
  extended to include the effect of the vibration modes of the beater, or 
  drumstick. The same force that drives the structure also acts (in the 
  opposite direction) on the beater. In a simple model, such as the pinned-free 
  beam model that is used to represent a drumstick in section 12.1, the mass we 
  have so far considered corresponds to a rigid-body mode of the beater. If the 
  beater also has modes with non-zero frequency, they will acquire kinetic 
  energy from the impact in exactly the same way as discussed above for the 
  modes of the structure. So all we need to do to extend the analysis to this 
  case is include these extra modes of the beater, suitably mass-normalised, 
  into the expressions for $K(\Omega)$ or $K^\prime (\Omega)$ in equations (12) 
  and (19). 