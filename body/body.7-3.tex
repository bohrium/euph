

  One obvious difference among the instruments presented in section 7.1 is that 
  some instruments use a single string for each note (guitar, harp, violin, 
  banjo) while others used pairs (lute, mandolin) or even triples (piano). In 
  this section we tease out some of the consequences of using multiple strings. 
  These may go some way to explain what gives each instrument its 
  characteristic and recognisable sound. This account is inspired by a classic 
  study of coupled piano strings by Gabriel Weinreich [1]. 

  Suppose we have a pair of nominally identical strings, attached to the 
  soundboard of an instrument at essentially the same point. Our first guess 
  about what will happen is very simple. We imagine that the two strings are 
  identical in every way, including being tuned to perfect unison. We also 
  suppose that vibration of the soundboard is in one direction only, 
  perpendicular to its surface. Under those circumstances, we can predict the 
  behaviour without needing any detailed calculations. 

  This idealised version of the system has a symmetry: imagine a mirror between 
  the two strings, oriented normal to the soundboard. As we saw right back in 
  Chapter 2, if a system is symmetrical like this, every mode must be either 
  symmetric or antisymmetric in that mirror. So every string overtone splits 
  into a pair: one is symmetric, with the two strings vibrating in unison; the 
  other is antisymmetric, with one string going up when the other goes down. 
  Figure 1 shows a sketch, based on the lowest string mode. The symmetric mode 
  exerts force on the soundboard, which will vibrate in response. But the 
  antisymmetric mode exerts precisely zero net force, because the forces from 
  the two strings always cancel out. 

  \fig{figs/fig-6d8734b5.png}{\caption{Figure 1. Idealised model of two coupled 
  strings, tuned identically. The upper plot shows symmetrical motion, the 
  lower plot shows antisymmetric motion with the two strings vibrating in 
  opposite directions.}} 

  There are three results of this. First, the antisymmetric mode will make 
  virtually no sound, because it does not excite vibration in the soundboard. 
  Second, the symmetric and antisymmetric modes will have slightly different 
  frequencies: the antisymmetric mode will be at the frequency of a string with 
  a rigid termination at the bridge, but the symmetric mode will have a 
  slightly different “effective length” because the bridge is moving a little. 
  Third, the two modes will have different damping factors, and thus different 
  decay rates. The symmetric mode is losing some energy to the soundboard, 
  while the antisymmetric mode is not. 

  The net result of all this is that the system behaves for all practical 
  purposes as if it had a single string, with twice the mass, twice the tension 
  and twice the impedance of the two separate strings. If that was all we could 
  achieve, it might have been simpler to use a single, thicker string. But, in 
  reality, the behaviour is more complicated and the story is more interesting. 
  Something always interferes with the exact symmetry, and this can have 
  profound consequences. The most obvious possibility from the perspective of a 
  player or a piano tuner is that the two strings might have very slightly 
  different tensions, and that is the case we will examine first. But the 
  soundboard vibration may also violate the symmetry condition: the point on 
  the bridge where the strings are attached may move in a three-dimensional way 
  rather than purely in the normal direction. We will return to that case a 
  little later. 

  For a first view of what can happen, we will look at a simple model inspired 
  by the sketch in Fig.\ 1. We will allow each string to have only a single 
  “mode”, with a half-wave sinusoidal shape like the sketch. Both strings are 
  rigidly anchored at one end, but at the other end they share an attachment to 
  the “soundboard”, which for this purpose we will model as a single 
  mass-spring-dashpot system, representing a single body mode. So our model has 
  three degrees of freedom, describing the “body” motion at the bridge, and the 
  amplitudes of motion of the two strings. We will calculate the modes of this 
  coupled system; but the calculation will be a little more complicated than 
  the modal problems we thought about back in chapter 2, because this time we 
  will be particularly interested in the damping behaviour. The details of the 
  model and the method used to compute the result are described in the next 
  link. 

  We can illustrate the range of behaviour through a case study, based 
  approximately on the second course of a 12-string guitar. We will take a pair 
  of steel strings with diameter 0.45~mm and length 650~mm, identical except 
  for having slightly different tensions. They will both be tuned to 
  frequencies near 250~Hz, but the two strings will be mistuned by a small 
  amount, which we can characterise in cents --- remember from section 2.3 that 
  one cent is a hundredth of a semitone. Each string resonance, in isolation, 
  is given a Q-factor of 3000. 

  The single body mode is chosen to match, approximately, one of the strong 
  low-frequency modes of a guitar body. It will be given an effective mass of 
  200~g, and the stiffness will be adjusted to move the resonance frequency 
  through a range around the string frequencies. Figure 2 shows the “bridge 
  admittance” of this single-mode guitar body, for the 6 cases we will study. 
  The resonance has a Q-factor of 50 in each case. The string tuning is 
  indicated by the vertical dashed line. So you can see that the first curve, 
  labelled (a), has the body resonance placed well below the string tuning. In 
  the second curve (b) it is closer, and in the third case (c) the body 
  resonance is right on top of the string tuning. Cases (d,e,f) have the body 
  resonance progressively higher than the string tuning. 

  \fig{figs/fig-4bb0b299.png}{\caption{Figure 2. The bridge admittance of the 
  single-mode ``guitar body'' used in the study of string mistuning. In all 
  cases the effective mass is 200 g and the Q-factor is 50. The stiffness has 
  been adjusted to 6 different values, to produce the 6 cases plotted. The 
  labels `a'--`f' match the labelling in Figs. 3--5. The dashed vertical line 
  marks the nominal tuning of the string investigated in those figures.}} 

  For each of the 6 cases, the degree of mistuning between the two strings will 
  be varied through a range of $\pm8$ cents. The system has three degrees of 
  freedom, so it has three modes. They will all involve motion of the strings 
  and the body resonance, but the modes can generally be divided into two 
  ``string modes'' and one ``body mode'' based on how the energy is shared 
  between the degrees of freedom: we did something similar for the banjo model 
  back in section 5.5. The frequencies, Q-factors and mode shapes have been 
  computed at each stage, and the results for the two ``string modes'' are 
  plotted in Figs.\ 3-5. First, Fig.\ 3 shows how the frequencies vary with the 
  degree of mistuning. The 6 sub-plots (a--f) correspond to the 6 bridge 
  admittances from Fig.\ 2, in order of increasing resonance frequency and 
  labelled with matching letters. 

  \fig{figs/fig-1cd0f434.png}{\caption{Figure 3. Frequencies of the two 
  ``string modes'' as mistuning is varied over a small range, corresponding to 
  the 6 different placements of the single body mode shown in Fig. 2.}} 

  Concentrate in the first instance on sub-plot (a). This shows behaviour often 
  called ``curve veering'' or ``avoided crossing'', for reasons that are 
  obvious in the plot. If you looked at the plot without your spectacles on, so 
  that it was blurred, you might think that it shows two straight lines which 
  cross in the middle of the plot --- pretty much what you might have expected, 
  since we have varied the tuning of one string while leaving the other one 
  fixed. But look more carefully: the lines do not cross, but instead they veer 
  apart at the last moment. 

  However hard you try, it is not possible to adjust the tuning of this pair of 
  strings to produce exact unison, or in other words a state with no beats 
  present in the sound. The coupling of the two strings through the bridge 
  means that there is a minimum separation of the two frequencies. Really, we 
  should have been expecting that from the earlier discussion. The minimum 
  separation occurs when the strings are tuned with exactly the same tension, 
  and the result is the symmetrical case we looked at earlier. The modes are as 
  sketched in Fig.\ 1, and we already noted that the two frequencies will be 
  different because one involves motion of the bridge, while the other does 
  not. 

  Now look at the other cases in Fig.\ 3. They all show a similar ``veering'' 
  shape, but as you go through the cases the closest approach of the two curves 
  changes. In cases (b) and (c) the two curves get progressively further apart, 
  and then they come closer together again once you get to the second row of 
  plots. We already know from Fig.\ 2 what is special about case (c): this is 
  the case where the body resonance falls very close to the string tuning. 
  Figure 3(c) shows the effect of this nearby body resonance: the two curves 
  stay well apart throughout the range of mistuning. Again, this conclusion is 
  not a surprise: at the point of identical tensions in the two strings the 
  body motion is particularly large because we are near a resonance frequency, 
  so the frequency of the symmetric mode is shifted a long way away from that 
  of the antisymmetric mode. In cases (d--f) the body resonance is above the 
  string tuning, and progressively further away from it. What we see in in 
  Fig.\ 3 is that the closest approach of the two curves gets nearer and 
  nearer. 

  Figure 4 shows the corresponding behaviour of the modal Q-factors for the 6 
  cases, and Fig.\ 5 shows the ratio of amplitudes of motion in the two 
  strings. In both cases the colours of the curves match the ones in Fig.\ 3. 
  Looking at the patterns in Fig.\ 4, it can be seen that the Q-factors vary 
  sensitively with mistuning, especially in the range where the frequencies 
  were actively veering in Fig.\ 3. In the middle of each plot, where the 
  mistuning is zero, the two Q-factors reach their maximum and minimum values. 
  The minimum value varies significantly across the 6 cases, with the most 
  extreme behaviour in case (c), where the body resonance falls close to the 
  string tuning. This is the only case where the Q-factor of a ``string mode'' 
  comes close to the Q-factor of the ``body mode'', indicated by the dotted 
  line. In all cases the red and blue curves come close together once the 
  mistuning is relatively large, whether it is positive or negative. 

  \fig{figs/fig-312a7597.png}{\caption{Figure 4. Modal Q-factors plotted 
  against mistuning, for the same 6 cases as in Figs. 2 and 3. Colours 
  correspond to Fig. 3. The black dotted line indicates the Q-factor of the 
  ``body mode''. (Recall that Q-factor is the inverse of the loss factor we 
  have talked about in some other sections: high Q means low damping, and slow 
  decay; low Q conversely means high damping and rapid decay.)}} 

  Notice that in every panel of Fig.\ 4, the blue curve reaches a maximum value 
  of 3000 at the point where the mistuning is zero. This is the case we talked 
  about earlier: the two strings are identical, and the symmetry argument means 
  that we know what the mode shapes must be. One is symmetric, the other 
  antisymmetric. The antisymmetric case must have a Q-factor of 3000, because 
  the strings cannot lose energy to the body mode so the mode exhibits the 
  value we have assigned to the strings in isolation. On the other hand, the 
  symmetric mode can lose a lot of energy to the body, leading to a low 
  Q-value. Looking at the corresponding points in Fig.\ 5, this identification 
  of the modes is confirmed. At that tuned point, the blue curve has the value 
  $-1$ in every case, while the red curve has value $+1$: equal amplitudes, 
  either with the same sign (red curves) or opposite signs (blue curves). 

  \fig{figs/fig-98bec0e4.png}{\caption{Figure 5. Modal amplitude ratios between 
  the two strings, plotted against mistuning, for the same 6 cases as in Figs. 
  2 and 3. Colours correspond to Figs. 3 and 4.}} 

  

  But when the mistuning is non-zero, the mode shapes are no longer exactly 
  ``symmetric'' or ``antisymmetric''. What we can see in Fig.\ 5 is that when 
  the mistuning is large, the ratio of amplitudes of the two strings gets 
  either much bigger than unity, or else much less than unity, as can be seen 
  in Fig.\ 5. This has a simple interpretation. If the strings are sufficiently 
  out of tune, each mode simply has one string vibrating strongly while the 
  other hardly moves: exactly what you would expect without taking into account 
  coupling between the strings through the body. 

  However, one feature persists throughout the plots: the blue curves always 
  describe a case in which the two amplitudes have opposite signs, while the 
  red curves always correspond to the same signs. You can see this directly in 
  Fig.\ 5: the blue lines are always below the x-axis, while the red curves are 
  always above it. 

  Looking back at Fig.\ 3, we can see that the red and blue curves swap places 
  once the body resonance is above the string tuning. What is happening is that 
  the body influence switches from ``mass-like'' to ``stiffness-like'' 
  depending whether the string frequency is above or below the body resonance 
  (the reason for that was explained back in section 2.2.7). This determines 
  which of the two string modes has the lower frequency. 

  Having seen the behaviour of the individual modes, we can put these together 
  to make a kind of “pluck response” of our one-mode coupled strings. This will 
  consist of a combination of the three modal responses, in a mixture governed 
  by the assumed initial conditions: technical details are described in the 
  previous link. There are two extreme cases that are illuminating to look at: 
  either the two strings are initially excited identically, or just one string 
  is driven. This is an idealisation of the two choices for playing a note on 
  the piano: ordinarily the piano hammer strikes the pair or triplet of strings 
  simultaneously, but if the una corda pedal is pressed the hammer is shifted 
  sideways so that it only strikes one or two strings. 

  The output variable we are interested in is the motion of the “soundboard”, 
  represented here by the single body mode. This motion corresponds, in our 
  crude model, to the source of sound from the vibrating soundboard. To bring 
  out the most interesting aspect of the behaviour, the results will be plotted 
  in the form of an envelope. We smooth over the individual cycles of the 
  vibration to show how the amplitude varies in time, then plot it on a decibel 
  scale. We can start with the most extreme case, where the body mode nearly 
  matches the string tuning: this is case (c) in Figs.\ 2--5, the case that 
  gave the biggest effects. 

  Sub-plot (a) of Fig.\ 6 shows the envelope curves for the two excitation 
  cases, when the strings are perfectly tuned in unison. The red curve shows 
  the result from driving both strings, while the blue curve shows the una 
  corda response. The main thing we see here is a pair of lines, sloping 
  steeply downwards, parallel to one another. This is the case of perfect 
  symmetry, in which one string mode is antisymmetric so that it does not 
  excite the soundboard at all. We see the rapid decay associated with the 
  symmetric string mode, which has a very low Q-factor as seen in the red curve 
  of Fig.\ 4(c). At the start of the note, we also see some wiggles in the 
  envelope: this is the result of transient excitation of the body mode. The 
  only difference between the red and blue curves in this case is a vertical 
  shift by 6~dB, arising simply because when both strings are initially driven 
  the amplitude of the symmetric mode is twice as high as when only one string 
  is driven. 

  \fig{figs/fig-b1555ec6.png}{\caption{Figure 6. Envelope plots for the 
  transient response of the string model, for the case corresponding to 
  sub-plot (c) in Figs. 3, 4 and 5. Red curves show the result of driving the 
  two strings simultaneously, blue curves show the response to driving only one 
  of the strings. The sub-plots correspond to different levels of string 
  mistuning: (a) 0; (b) 1 cent; (c) 3 cents; (d) 5 cents.}} 

  Moving to Fig.\ 6(b) we see something more complicated. The two strings are 
  now mistuned by just 1 cent, breaking the symmetry so that both string modes 
  are now able to excite the soundboard. Note that 1~cent is a very small 
  difference of tuning --- surely near the limit of what guitarists or piano 
  tuners are likely to be able to achieve. Both the curves in this plot show a 
  characteristic double decay: a fast initial decay, giving way to a much 
  slower decay in the tail of the note. We are seeing the effect of the very 
  different Q-factors of the two string modes, shown in Fig.\ 4(c). Provided 
  both modes are driven to some degree, a decay pattern like this is 
  inevitable: the fast-decaying mode, however loud it may be initially, is 
  bound to decay below the level of the other mode after a time, leaving that 
  slower-decaying mode to dominate the sound. 

  In piano terms, the second, slower-decaying, phase is sometimes called the 
  ``aftersound''. The balance between the early sound and the aftersound is 
  quite different in the red and blue curves of Fig.\ 6(b). Although the string 
  modes are no longer exactly symmetric and antisymmetric, with this small 
  degree of mistuning the two mode shapes have not changed a great deal. The 
  result is that striking both strings together excites mainly the mode with 
  the strings in the same phase, which is the one with the fast decay. It takes 
  some time for this initially louder mode to decay enough to reveal its 
  slower-decaying companion. 

  But in the una corda case both modes are excited to similar amplitudes. This 
  means that the aftersound takes over sooner, and is a bigger component of the 
  overall sound. As Weinreich pointed out [1], this is probably the main 
  function of the una corda pedal on a piano. Although it is often called the 
  ``soft pedal'', the difference of loudness with and without this pedal is not 
  very great. More important, probably, is the fact that the sound is less 
  percussive and dominated by the aftersound: more ``lyrical'', perhaps. 

  As we increase the mistuning, in Fig.\ 6(c,d), the pattern remains 
  recognisably similar but the details change. The decay rate of the aftersound 
  component gets faster, and the difference in level between the blue and red 
  curves is somewhat reduced. At least for this special case, in this very 
  simplified model, we get a strong hint that a skilled piano tuner might be 
  able to exercise considerable influence over the sound by controlling very 
  small levels of mistuning between the strings. 

  Figure 7 shows a similar set of envelope curves to Fig.\ 6, for the case 
  corresponding to case (d) of Figs.\ 2-5. Case (a), with the strings in 
  perfect unison, looks quite similar to the previous case, except that the 
  slope of the parallel lines is less steep because the Q-factor of the 
  symmetric mode does not fall as low this time. The other sub-plots show 
  rather more complicated patterns. Case (b) shows a trace of the double decay 
  profile in the blue curve (it starts below the red curve and ends above it), 
  but all the curves show strong wiggles. These correspond to beats between the 
  frequencies of the two ``string modes''. These frequencies are not exactly 
  the same as the nominal tuned frequencies of the two strings in isolation, 
  but they are less affected by the coupling in this case because the bridge 
  admittance at the string frequency is lower by almost 20~dB: compare the 
  heights of the yellow curve and the purple curve where they cross the 
  vertical dashed line. 

  \fig{figs/fig-3e57f4f5.png}{\caption{Figure 7. Envelope plots for the 
  transient response of the string model, for the case corresponding to 
  sub-plot (d) in Figs. 3, 4 and 5. Red curves show the result of driving the 
  two strings simultaneously, blue curves show the response to driving only one 
  of the strings. The sub-plots correspond to different levels of string 
  mistuning: (a) 0; (b) 1 cent; (c) 3 cents; (d) 5 cents.}} 

  Of course, we would like to know what this all sounds like. It is not very 
  interesting to listen to the sounds from the simple model with just a single 
  mode in each string: all you hear is a variety of modulated sine waves, not 
  sounding at all like a plucked string of any kind, and therefore not suitable 
  for making interesting musical judgements. But we can take the model of two 
  strings coupled via a soundboard, and use the synthesis approach described 
  back in section 5.4 to create versions of the sound of the open second course 
  of a 12-string guitar, which motivated the model we have been looking at.~ 
  Sound 1 illustrates the “una corda” case, where only one of the strings is 
  plucked.~ Sound 2 is the “regular piano” case (or perhaps it is better 
  thought of as a ``regular harpsichord'' case since we are modelling plucked 
  strings rather than hammered strings) where both strings are plucked equally 
  and simultaneously. In both cases you will hear three notes, corresponding to 
  perfect unison, 1 cent mistuning, and 5 cents mistuning. The sounds are 
  encouragingly familiar from the sounds made by a 12-string guitar: 
  realistically, 1~cent is as accurate as any guitarist is likely to tune their 
  strings, but 5~cents represents a serious degree of mistuning. 

  \aud{auds/aud-4a8ce55c-plot.png}{\caption{Sound 1. Synthesised pluck sounds, 
  for two steel strings coupled through a guitar body (using a measured bridge 
  admittance). The first note has the strings tuned in perfect unison, the 
  second has 1~cent of mistuning, the third has 5~cents mistuning. These 
  examples illustrate the ``una corda'' case, in which only one of the two 
  strings is plucked.}} 

  \aud{auds/aud-b582e4b1-plot.png}{\caption{Sound 2. Synthesised guitar-like 
  sounds as in Sound 1, except that now both strings are plucked equally and 
  simultaneously.}} 

  These two cases may be natural for discussing the piano, but they are rather 
  artificial in the context of a guitar. But we still learn something 
  interesting from them, because they give an indication of the extent to which 
  a player may be able to vary the sound by subtle details of performance 
  technique. It is hard to guess exactly how a player will pluck a course of 
  two strings. They probably excite both strings to some extent, but not 
  exactly equally and not exactly simultaneously. The cases demonstrated here, 
  which sound strikingly different from each other, give extremes from the 
  range of possibilities. Most normal notes probably lie somewhere in between 
  the two. 

  There is one circumstance under which a note on a guitar, banjo or other 
  stringed instrument can show the behaviour we have been looking at, even when 
  it has only single strings. This is the case where sympathetic resonance 
  occurs with one of the other strings of the instrument. A simple example 
  would be to finger a note on a lower string at the same pitch as one of the 
  open higher strings. You then have two strings, coupled at the bridge, tuned 
  to the same nominal pitch: exactly the model we have been studying. But they 
  are not a course, where both strings would be played together: you play one 
  string, and the other will respond just like the una corda case, as just 
  discussed. 

  However, this is not the end of the story. Multiple strings are not the only 
  way to get sound envelopes showing a double decay. There is another aspect of 
  string vibration, which we have been playing down: each string mode can occur 
  in two polarisations. In other words, the string could vibrate in two 
  different planes. If both ends of the string were rigidly anchored in a 
  symmetrical way, these two modes would occur at identical frequencies. What 
  would happen then is similar to something we saw earlier, when we looked at 
  drums. As was shown in section 2.2.4, linear combinations of this pair of 
  string modes can be found, which would correspond to vibration in any chosen 
  plane. 

  But when the string is attached to a real instrument, things get more 
  complicated. Motion at the bridge, and to a more limited extent at the other 
  end of the string at the nut or the fret, leads us to expect that vibration 
  in the plane perpendicular to the soundboard will be different from vibration 
  parallel to the soundboard of the instrument. Instead of the plane of 
  vibration being arbitrary, the combined system of string and instrument body 
  will choose two particular orientations, resulting in a pair of modes with 
  slightly different frequencies. 

  At first sight it seems obvious that these orientations will be parallel and 
  perpendicular to the soundboard. Well, that is sometimes true, especially at 
  low frequencies, but it is not always true. The details depend on the 
  particular instrument. Think about the guitar, as a typical example. The 
  string positions on the bridge saddle of a guitar stand a few millimetres 
  above the soundboard. As the soundboard vibrates, the bridge participates in 
  that motion. But that does not necessarily mean that the top of the bridge 
  moves purely normal to the soundboard surface. Bending modes of the 
  soundboard can also involve rotation of the bridge, like the sketch in Fig.\ 
  8 (showing a cross-section through the guitar body in the plane of the 
  bridge). The strings experience more complicated three-dimensional motion, 
  slightly different for each one: motion at the string notches associated with 
  this mode is not purely normal to the surface. The sketch also tells us 
  something else: through the rotational component of the bridge motion, string 
  vibration in the plane parallel to the soundboard can excite soundboard 
  vibration, and thus create sound radiation. 

  \fig{figs/fig-63a3d7da.png}{\caption{Figure 8. Sketch of a guitar body and 
  bridge, undergoing a vibration mode of the soundboard. The six string 
  positions on the bridge are indicated by black dots. The combination of 
  displacement and rotation means that each string moves in a different way 
  during this vibration}} 

  Pulling these features together, we find something interesting. Each “string 
  mode” is in fact a pair of modes, with slightly different frequencies. They 
  will also have different damping factors, because the rate of energy loss to 
  the soundboard will be different. Both modes can create at least some sound 
  radiation. When the player plucks the string, they will excite some mixture 
  of the two modes. The result will be a “double decay”, for reasons that are 
  very similar to the earlier discussion about the modes resulting from a 
  coupled pair of strings. The mode that couples better to the soundboard will 
  tend to make more sound, but it will decay faster. Sooner or later, it will 
  have decayed below the level of the other mode, with its slower decay. 

  To complete the story of double decays, we can ask whether all instruments 
  show the effect, or whether some are more prone to it than others. The 
  physics we have just described is always present, of course. Every stringed 
  instrument can exhibit effects of the two polarisations of motion, and any 
  instrument with double or triple strings (or sympathetic strings) will have 
  coupled mode behaviour similar to what has been seen here. But that does not 
  necessarily mean that a double decay will occur at a level to be perceptible 
  in the sound of the instrument. 

  We can use information we have already developed to give a very simple 
  criterion for the effect to be significant. The crucial factor, for both 
  physical mechanisms of generating double decays, is that the decay rates 
  associated with the ``early sound'' and the ``aftersound'' must be 
  significantly different. If they are not, the envelope will not be able to 
  show the characteristic transition that we saw in Fig.\ 6(b,c,d). To get 
  significantly different loss factors, either for the in-phase mode of 
  multiple strings or the favoured polarisation of a single string, the energy 
  loss into the body must be significant in comparison with the internal loss 
  factor of the string. 

  A double-decay envelope is mainly dominated by the fundamental, or the very 
  low overtones of the string. This is the frequency range where the string’s 
  intrinsic damping is dominated by air viscosity effects: we gave an 
  expression for the associated loss factor $\eta_{air}$ in section 5.4.5. 
  (Remember $\eta$ is Greek letter ``eta''.) This is to be compared to the loss 
  factor $\eta_{body}$ associated with energy loss to the body: in section 
  5.1.2 we derived an expression for that in terms of the bridge admittance. So 
  we can simply plot $\eta_{air}$ and $\eta_{body}$ on top of each other, as 
  functions of frequency, for some representative instruments. That will tell 
  us immediately whether the double decay effect is likely to be important for 
  that instrument. 

  The first example, shown in Fig.\ 9, is for a steel-string guitar. The green 
  line shows $\eta_{air}$, while the red lines show $\eta_{body}$. The solid 
  curve is for a single string, the dashed curve is for a double string like 
  the example discussed above: the loss factor is doubled (i.e. raised by 6~dB 
  in this plot), because the combined impedance is doubled with a pair of 
  strings as in a 12-string guitar. The figure reveals that the red curves only 
  rise convincingly above the green curve around strong resonance peaks. That 
  is exactly what we found in the discussion of Figs.\ 2--7: there was a strong 
  effect in the case where the string tuning was close to the single body 
  resonance, but only much weaker effects otherwise. 

  \fig{figs/fig-44c5b7fa.png}{\caption{Figure 9. Comparison of loss factors for 
  a steel-string guitar, based on the second string. The solid green line shows 
  the loss factor due to air damping, $\eta\_{air}$. The solid red line shows 
  the loss factor due to energy flow into the body at the bridge, 
  $\eta\_{body}$. The dashed red line shows the effect of turning the guitar 
  into a 12-string instrument, to match the example studied earlier.}} 

  This is to be contrasted with the situation revealed in Fig.\ 10, for a 
  piano. This time the solid curve shows the effect of the actual triple string 
  group, while the dashed line shows what would have happened if the piano had 
  only single strings. This time the loss factor is raised by a factor 3 with 
  three strings. The solid red line lies above the green line over the entire 
  frequency range, often by a substantial factor of about 20~dB. That is 
  precisely the condition for the double decay phenomenon to occur strongly, 
  for all notes rather than just for ones that happen to fall near a soundboard 
  resonance. 

  \fig{figs/fig-c6e4e9f8.png}{\caption{Figure 10. Loss factor comparison in the 
  same format as Fig. 9, for a piano (the baby grand shown in Fig. 1 of section 
  7.1), based on plain wire string triples in the mid-range of the instrument. 
  The solid red line shows the result with three strings; the dashed red line 
  shows how it would change with only one of the same strings.}} 

  In the piano, the excitation direction of the strings is controlled in a very 
  standardised way by the piano mechanism. This means that the multiple-string 
  effect can operate strongly, including the effect of the una corda pedal, but 
  the effect of polarisations is probably rather minor because the strings are 
  excited predominantly in the vertical plane. Also, there is no equivalent of 
  the discretion a guitarist has over the exact orientation of the initial 
  string motion. 

  The final example, in Fig.\ 11, shows the corresponding plot for a banjo. The 
  banjo has only single strings, but we have already seen in section 5.5 that 
  they are coupled to the ``soundboard'' (which is of course a membrane in the 
  banjo) far more strongly than occurs in a guitar. The result in Fig.\ 11 is 
  that the red curve lies well above the green curve over most of the frequency 
  range. So we can expect banjo notes to exhibit a strong double decay 
  envelope: not in this case because of multiple strings, but arising from the 
  two string polarisations. This has indeed been observed experimentally, by 
  Stephey and Moore [2]. 

  \fig{figs/fig-db3185a5.png}{\caption{Figure 11. Loss factor comparison in the 
  same format as Fig. 9, for a banjo, based on the thickest plain steel string 
  (the 3rd string).}} 



  \sectionreferences{}[1] Gabriel Weinreich ``Coupled piano strings''; Journal 
  of the Acoustical Society of America \textbf{62}, 1474--1484 (1977). 

  [2] L. A. Stephey and T. R. Moore, “Experimental investigation of an American 
  five-string banjo,” Journal of the Acoustical Society of America 
  \textbf{124}, 3276--3283 (2008). 