  This chapter is the first of a series of ``Underpinnings'' chapters, which 
  are designed to supply some background information about the various areas of 
  science that are relevant to musical instruments. This first one deals with 
  vibration: later ones will deal with acoustics, hearing and psychoacoustics, 
  nonlinearity, and measurements/experiments. 

  This chapter introduces some key ideas about linear systems: the idea of an 
  input-output system, Fourier analysis which allows us to assemble any kind of 
  sound out of sine waves, and the consequent central idea of a frequency 
  response function, which encapsulates everything we need to know about how a 
  particular linear system behaves. The chapter will also introduce the idea of 
  vibration modes: the particular patterns of vibration responsible for 
  resonances of any structure. We will see that there is a strong connection 
  between modes and the frequency response function. 

  Section 2.3 gives a digression about the relation between frequency, a 
  measurable physical quantity, and pitch, a musically-relevant perception. The 
  section will also give a brief introduction to musical scales and intervals. 

  Section 2.4 gives examples of different ways to measure and visualise 
  vibration, all of which will prove important in the story to be told in later 
  chapters. Examples of measured frequency response functions will be shown, 
  for a simple vibrating system: a toy drum. The peaks in these frequency 
  response functions correspond to the vibration modes, and some examples of 
  the associated modal vibration patterns are shown. Finally, examples of a 
  different kind of analysis are shown: the spectrogram is a way to visualise a 
  waveform of sound or vibration which is somewhat similar to the way our 
  hearing system works. 

