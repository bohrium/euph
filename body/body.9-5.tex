

  So far in this chapter we have been following the “physicist’s agenda”, using 
  simplified models to look at one phenomenon at a time to build up a picture. 
  But what about the “musician’s agenda”? How much have we learned that will be 
  interesting to a player of the violin or cello? If we are honest, the answer 
  is ``only a rather limited amount''. Schelleng’s diagram is relevant, 
  particularly to beginners. Wolf notes can be of direct interest to all 
  players, but they are somewhat of a niche interest since they only affect 
  particular notes. 

  But what players really care about, and spend a lot of time on, relates to 
  transients. Hours of practice are devoted to mastering different bowing 
  techniques, each associated with a particular way to start a note or to 
  transition between notes. If an instrument had some subtle feature which made 
  a particular tricky bowing gesture just a little easier or more reliable, it 
  seems a good guess that a player would describe that instrument as being 
  “easier to play”. This possibility opens an interesting avenue of study, 
  using computer simulation models of the kind we have been talking about. 

  The earliest computer models date from the 1970s. A decade later, the 
  increasing power of computers allowed these early bowed-string simulation 
  models to be used to start exploring transient effects systematically [1,2]. 
  How quickly, if at all, is the Helmholtz motion established after a given bow 
  gesture? How does the transient length vary if you change parameters in the 
  model that are relevant to a player, or an instrument maker? This early work 
  established some useful methodology, but we needn’t look at the detailed 
  results because the studies suffered from major flaws that only became 
  apparent later: we will meet them in the course of this section and the next. 

  The people we have mentioned so far, like Raman, Cremer and Schelleng, have 
  all been scientists or engineers with an interest in music. But the hero of 
  this section was a musician who developed a strong interest in science. Knut 
  Guettler was a virtuoso player and teacher of the double bass. He got 
  interested in whether theoretical models and computer simulations could tell 
  him things that would be useful in his teaching. Double bass players have a 
  particular problem: some of the notes they play have such low frequencies 
  that they can’t afford a bowing transient that takes 10 or 20 period lengths 
  to settle into Helmholtz motion: a short note may be over by then! So 
  Guettler set himself the task of understanding what kind of bow gesture a 
  player needed to perform in order to get a “perfect start” in which the 
  Helmholtz sawtooth waveform was established right from the first slip of the 
  string over the bowhair. 

  \fig{figs/fig-cce1df25.png}{Figure 1. Knut Guettler. Image copyright Anders 
  Askenfelt, reproduced by permission.} 

  Guettler's initial step was to point out the first of the major flaws in the 
  early computer studies: the transients used in those studies were all 
  physically impossible! In the computer, it is easy to simulate “switch-on” 
  transients in which the bow speed or force suddenly changes. But any physical 
  transient cannot have jumps in either quantity: it must start with either the 
  bow force or the speed (or both) equal to zero. If the bow is already in 
  contact with the string with non-zero normal force, the speed must start from 
  zero. On the other hand, if the bow is already moving when the bow makes 
  contact with the string, as in a string-crossing gesture, then the normal 
  force must build up from zero. 

  This realisation led Guettler to study a more realistic family of transient 
  gestures. The bow starts in contact with the string, and the force is held 
  constant while the bow is accelerated from rest with a chosen value of 
  acceleration. Guettler then used the simplest available model of bowed string 
  motion to pursue his agenda of finding the conditions under which a “perfect 
  start” was possible from one of these constant-acceleration gestures. He 
  assumed an ideal “textbook” string, terminated in mechanical resistances or 
  “dashpots”. This simple model is yet another thing that goes back to Raman, 
  and it is essentially the same model that Schelleng used in his discussion of 
  bow force limits for steady Helmholtz motion. Some mathematical details of 
  this “Raman model” are given in the next link. 

  A virtue of this simple model is that it allowed Guettler to follow how the 
  string motion develops during the early stages of a transient. The string is 
  initially sticking to the bow, and it is pulled to one side as the bow moves. 
  The first interesting thing to happen is that, sooner or later, the string 
  releases from the bow. As sketched in Fig.\ 2, this release will give a 
  similar effect to plucking the string. As we saw back in section 5.4, a pair 
  of corners will then be created, travelling away from the bow symmetrically 
  in both directions. One of these (the one travelling towards the bridge) 
  looks very much like the Helmholtz corner we want, but the other one has the 
  wrong sign. As Schelleng first pointed out [3], in order for a perfect start 
  to occur, the “good” corner must survive and develop into the Helmholtz 
  corner while the other one needs to disappear. 

  \fig{figs/fig-d5578037.png}{Figure 2. Sketch of string displacement before 
  the first slip at the bow (solid line), and at two times shortly after that 
  (dashed lines). A pair of corners travel away from the bow, indicated by the 
  red arrows.} 

  In a tour de force of analysis, Guettler was able to track the behaviour 
  through the first few period-lengths after the first release, and he 
  identified four things that might go wrong with the desired sequence of 
  events [4]. For each of the four, he was able to find a criterion that would 
  decide success or failure. The details of the calculation are quite messy, 
  and we need not go into them, but we can show the key results in graphical 
  form. All four criteria take the form of a critical value of the ratio of bow 
  force to bow acceleration. For two of them, the ratio must be bigger than the 
  critical value, while for the other two it must be less than the critical 
  value. As an aside, Galluzzo [5] has generalised Guettler's result with an 
  argument that suggests that any threshold relating to bowed-string transients 
  within the Raman model would be expected to take a similar form, as a 
  critical ratio of force to acceleration. 

  There is a simple way to represent the result, illustrated in Fig.\ 3. Each 
  criterion corresponds to a straight line in the acceleration—force plane. The 
  slopes of these lines are determined by the four critical values. So there 
  are four radial lines in that plane, and for a perfect start we need to be 
  above two of these lines (shown in blue), and below the other two (shown in 
  red). Unless the criteria are inconsistent, the result is a wedge-shaped 
  region in the plane (shaded yellow here) within which a perfect start might 
  be possible. Such plots are now known, naturally enough, as ``Guettler 
  diagrams''. This particular example is computed using the frequency and 
  typical impedance of a violin G string (196~Hz and 0.363~Ns/m respectively), 
  and the string is assumed to be undamped. The assumed friction coefficients 
  are taken from the measurement shown in Fig.\ 6 of section 9.2. The chosen 
  bow position has $\beta=0.13$ (recall that this parameter “beta” specifies 
  the position of the bowed point as a fraction of the string length). 

  \fig{figs/fig-2d72d9ac.png}{Figure 3. Example of Guettler's criteria for a 
  perfect start. For a given bow acceleration, the force must lie above the two 
  blue lines, and below the two red lines: in other words, it must lie in the 
  shaded wedge-shaped region.} 

  All four of Guettler's boundary lines move in a rather complicated way when 
  the bow position $\beta$ changes. An example of the variation of the slopes 
  of the four lines is plotted in Fig.\ 4, using the same line colours and 
  types as in Fig.\ 3. Logarithmic scales have been used for both axes here, to 
  highlight an intriguing parallel with the Schelleng diagram. The Schelleng 
  diagram shows that for a given bow speed, there are limits on the bow force 
  in order for Helmholtz motion to be possible. If $\beta$ is decreased, both 
  limits increase, and they get closer together and eventually meet. The new 
  diagram says that for a given bow acceleration, there are limits on the bow 
  force in order for a perfect start to be possible. These limits, too, 
  increase as $\beta$ decreases, and get closer together and eventually meet. 
  The pattern is more complicated than the Schelleng diagram, because there are 
  two criteria for each of the upper and lower limits (shown in solid and 
  dashed lines), and the lines cross so that all four play a role in 
  determining the allowed region for some values of $\beta$. 

  \fig{figs/fig-d31900ac.png}{Figure 4. The variation with bow position $\beta$ 
  of the slopes of the four lines from Fig. 3, using the same line colour 
  convention. The curves are all taken from equations given in Guettler's study 
  [4]. The two solid lines represent Guettler's equation (8b), the dashed red 
  line is for his equation (10b) and the blue dashed line is for his equation 
  (12). The region shaded in yellow is where a perfect start might be possible. 
  The example from Fig. 3 had $\beta=0.13$, so that the allowed region was 
  between the dashed blue line and the solid red line.} 

  Just as we did with Schelleng's diagram, we can compare Guettler's 
  predictions with measurements using the Galluzzo bowing machine described in 
  section 9.3.2. For each value of $\beta$, the machine bowed the open D string 
  of a cello 400 times, in a $20\times 20$ grid in the Guettler plane. The 
  bridge force from each note was recorded, and analysed using an automated 
  procedure that attempted to find the length of transient before Helmholtz 
  motion was established (if it ever was established, of course). This 
  automated analysis is fallible, there is no doubt about that, but the same 
  routine has been used in all cases (and will be used again when we come to 
  compare with simulated results) so the comparison between cases should be 
  fair. 

  Figure 5 shows the results, for 6 particular values of $\beta$. It is 
  immediately clear that the results give at least qualitative support for 
  Guettler's predictions. There are not very many perfect starts (which appear 
  as white pixels), but the successful transients (shown in colours other than 
  black) are confined in each case to a vaguely wedge-shaped region. As $\beta$ 
  increases, the boundaries rotate downwards and the wedge tends to get broader 
  (although in the bottom left-hand plot it seems to have got narrower again). 
  For the smallest value of $\beta$ shown here, there are very few coloured 
  pixels. For even smaller values of $\beta$ the results were not worth 
  showing, because they show virtually no coloured pixels at all. The left-hand 
  plot in the bottom row shows a lot of black pixels within the wedge region. 
  The main reason for this is something we saw earlier: with this particular 
  value of $\beta$ the string often chooses to vibrate with S-motion rather 
  than Helmholtz motion. 

  We can see and hear a few sample waveforms. They are all drawn from one 
  particular column of the right-hand Guettler diagram in the middle row of 
  Fig.\ 5, corresponding to $\beta=0.0899$. Figure 6 shows three waveforms, 
  corresponding to pixels 2, 9 and 17 of the 9th column of that figure, 
  counting everything from the bottom left-hand corner. The value of bow 
  acceleration is 1.39~m/s$^2$, and the three force values are 0.55~N, 1.58~N 
  and 2.76~N respectively. At the top, plotted in black, is the waveform for 
  the highest force. It appears as a black pixel in Fig.\ 5, but it looks as if 
  it is close to settling into the Helmholtz sawtooth by the end of the time 
  shown here. You can hear it in Sound 1: it has a rather ``scratchy'' sound. 
  In the middle, plotted in red, is a ``perfect start'' which you can hear in 
  Sound 2. At the bottom, plotted in blue, is a case with a slow transient, 
  settling into double-slipping (``surface sound''). You can hear it in Sound 
  3. All the sounds are very short, only 1/4~s for each one. You should be able 
  to hear noticeable differences between all three, although the quality 
  difference between the second and third sounds may not come across very 
  clearly. Bear in mind that these are not the actual sounds made by the cello 
  when the body vibrates and radiates: they are just waveforms of force acting 
  at the bridge. These waveforms are recognisable as a bowed-string sounds, but 
  they lack the extra interest generated by the body response. 

  \fig{figs/fig-2d05e3c0.png}{Figure 6. Measured waveforms for three transients 
  from the Guettler diagram corresponding to $\beta=0.0899$ in Fig. 4. They are 
  all drawn from the 9th column, with acceleration 1.39~m/s$^2$. They are laid 
  out in the same sense as in Fig. 4: counting from the bottom, they correspond 
  to pixels 2, 9 and 17.} 

  To harness the full potential of theoretical bowed-string models for 
  exploring issues of playability, qualitative agreement with measurements is 
  not enough. We would like to use computer models to find out how transient 
  lengths change as a result of changing parameters relevant to players and 
  instrument makers. Well, in order for that agenda to be possible, the model 
  must be sufficiently complete and realistic that it contains all those 
  parameters. It must also be reliable enough to capture the influence of 
  changing them. A first step would be to demonstrate quantitative agreement 
  with measurements. As we will see, this proves to be a tall order. 

  What might we need to include in such a model? We have already met several 
  factors that proved to be significant when looking at plucked strings, and it 
  seems a fair bet that those will all be relevant to bowed strings too: the 
  effect of the string’s bending stiffness and its intrinsic damping; the 
  effect on frequencies and damping of coupling to the instrument body; the 
  influence of the second polarisation of string motion, perpendicular to the 
  bowing direction. These factors can indeed all be included [6]: the next link 
  gives some technical details. 

  In addition, there is a new factor that was not relevant to plucked strings. 
  Transverse string vibration is not the only thing to be driven by the force 
  that a violin bow applies to the string. That force is applied tangentially 
  to the surface of the string, so it can also excite torsional motion of the 
  string, as sketched in Fig.\ 7 and explained in more detail in the next link. 
  Such torsional string motion probably isn’t directly responsible for a lot of 
  sound from the instrument, but it can still be very important because of the 
  way it interacts with transverse motion. It can be incorporated in the 
  simulation model with no difficulty, although at some cost in complication 
  [7]. 

  \fig{figs/fig-0c215e2c.png}{Figure 7. Sketch to show how the friction force 
  from the bow (red arrow) can produce both transverse and torsional motion of 
  the string (blue arrows).} 

  To see an important example of this interaction, think what happens when the 
  string is sticking to the bow. If we only consider transverse motion, as in 
  the discussion up to now, that means that the string just under the bow must 
  be moving at the speed of the bow. But when torsional motion is also allowed, 
  the string can roll on the sticking bow. This means, for example, that 
  Schelleng ripples which were previously “trapped” on the finger side of the 
  bow (see Fig.\ 13 of section 9.2) can now “leak” past the sticking bow and 
  show up in the bridge force: see the previous link for more detail. 

  There is just one context where torsional string vibration is directly 
  relevant to a violinist. Some violins are prone to a phenomenon called the 
  ``E string whistle''. Occasionally, when playing the open E string, a 
  high-pitched sound is obtained at a pitch unrelated to the expected note at 
  660~Hz: it occurs at a frequency more like 5~kHz. This is the expected 
  fundamental frequency of the first torsional mode of an E string, and Bruce 
  Stough [8] has convincingly demonstrated that the whistle is indeed caused by 
  torsional motion. Most violin strings have rather high damping of torsional 
  modes, and the player's finger will contribute further damping in a stopped 
  note. But the E string of a violin is usually a steel monofilament, with no 
  over-wrapping. So for an open E string, neither of these factors is present, 
  and torsional modes can have very low damping. This allows torsional 
  vibration to be directly excited by bowing. Some string manufacturers now 
  offer E strings with a layer of over-wrapping, specifically to add torsional 
  damping in order to suppress the whistle. 

  There is one more layer of potential complications in a realistic simulation 
  model, associated with the details of a conventional violin or cello bow. The 
  ribbon of bow hair has a finite width in contact with the string, rather than 
  the single-point contact we have been assuming. Both the hair and the stick 
  of a bow have vibration behaviour of their own, and those might influence the 
  behaviour of the bowed string. But we can duck all those issues for the 
  moment: the Galluzzo experiment deliberately used a rosin-coated rod in place 
  of a conventional bow, and we will first try to match those results. In 
  section 9.7 we will have a careful look at how things might change with a 
  real bow. 

  For the particular cello string used in the Galluzzo measurements, we have a 
  fairly complete set of measured properties relating to its transverse and 
  torsional vibration behaviour. This means that we can formulate a computer 
  simulation based on reliable, calibrated values of all the relevant 
  parameters [5,6]. We can also incorporate a reasonably realistic model of the 
  cello body, by extending the approach used for the wolf note in section 9.4 
  to include more body resonances. 

  Putting all this together, we can assemble a simulation model that includes a 
  good representation of the behaviour of the cello and its string. We can 
  “bow” this simulated string with the same friction curve we have used before, 
  derived from the steady-sliding measurements and shown in Fig.\ 6 of section 
  9.2. We can then run a set of Guettler transients, with the same set of bow 
  forces and accelerations as the measured set we saw in Fig.\ 5. We can 
  process the result using the same automatic classification routine that we 
  used for the experiments. 

  A typical result, compared with the corresponding measurement, is shown in 
  Fig.\ 8, and it is very disappointing! No aspect of the simulated pattern 
  gives a convincing match to the measurement. This simulated string would be 
  far harder to play than the real cello: there are fewer coloured pixels, and 
  many of them are scattered around in a speckly pattern rather than coalescing 
  to give broad areas of bright yellow such as we see in the measurement. The 
  only continuous patch of pale-coloured pixels is a narrow wedge running 
  diagonally across the plot. This wedge is indeed reminiscent of Guettler's 
  prediction, but we see no trace of a directly corresponding feature in the 
  measurement. We will explore some of the reasons for this in the subsequent 
  discussion. 

  To see a little more detail of how the simulation model behaves, Fig.\ 9 
  shows the results of some model variations. The top row shows results without 
  including the effect of the string's bending stiffness, while the bottom row 
  has it included. The left-hand column omits the effect of torsional motion in 
  the string, while the right-hand column includes it. So the top left plot is 
  without bending stiffness or torsion, while the bottom right plot includes 
  both, and is the case shown in Fig.\ 8. 

  To glimpse what lies behind these results, Fig.\ 10 shows a set of simulated 
  waveforms for one particular pixel of these Guettler plots. Counting from the 
  bottom left-hand corner, it is the 9th pixel in the horizontal direction 
  corresponding to acceleration 1.39~m/s$^2$, and the 6th pixel in the vertical 
  direction corresponding to force 1.14~N. This is a bright yellow pixel in the 
  bottom right-hand Guettler plot of Fig.\ 9 and a slightly darker colour in 
  the upper right-hand plot. These correspond to the waveforms plotted in black 
  and red respectively. It appears as a black pixel in the two left-hand 
  diagrams, and we can see why in the two waveforms plotted in blue and green. 
  The blue curve, corresponding to the case with neither torsion nor bending 
  stiffness, settles into regular double-slipping motion rather than Helmholtz 
  motion. The green curve, with bending stiffness but no torsion, does 
  something similar, but the waveform shows more persistent irregularity from 
  cycle to cycle. 

  \fig{figs/fig-318d849a.png}{Figure 10. Simulated waveforms corresponding to 
  pixel (9,6) of the four Guettler diagrams shown in Fig. 9. This pixel has bow 
  acceleration 1.39~m/s$^2$ and bow force 1.14~N. The blue curve omits both 
  torsion and bending stiffness. The red curve has torsion but no bending 
  stiffness. The green curve has bending stiffness but no torsion. The black 
  curve has both effects included, so that it should be the most realistic of 
  the four.} 

  Figure 9 tells us several interesting things. First, none of the plots look 
  anything like the measured case. They all look too “speckly” for comfort: 
  even for cases where a particular transient gave a satisfactory transient 
  leading quickly to Helmholtz motion, there are often neighbouring pixels that 
  are either black, or at least dark red. The interpretation is that if the 
  player tried to do the same gesture twice, they are likely to get very 
  different results because of inevitable tiny differences between the two 
  gestures. 

  This “twitchy” behaviour should sound familiar, if you remember the 
  discussion of chaotic systems in Chapter 8. It looks very much as if all four 
  cases explored in Fig.\ 9 are exhibiting “sensitive dependence on initial 
  conditions”, one of the hallmarks of a chaotic system. The measured Guettler 
  diagram also has some evidence of “twitchiness”, with occasional black pixels 
  in the middle of the bright yellow patch, but the effect seems far less 
  extreme. 

  Indeed, sensitive dependence in the measured responses can be demonstrated 
  directly. Paul Galluzzo made repeat measurements of the Guettler plane under 
  nominally identical conditions, and 9 examples of these are illustrated in 
  Fig.\ 11. Although the qualitative picture remains the same, individual 
  pixels change between takes. 

  \fig{figs/fig-cffd468d.png}{Figure 11. Nine repeat scans of the Guettler plan 
  using the Galluzzo experimental rig, illustrating a degree of sensitivity in 
  the response of the real bowed string to small variations. Image copyright 
  Paul Galluzzo, reproduced by permission.} 

  We can hazard a guess about where the twitchiness is coming from in the 
  simulated results. When we looked at the phenomenon of chaos in section 8.4, 
  we found that sensitive dependence was intimately tied up with the presence 
  of saddle points in the phase space: two trajectories that are initially very 
  close together can approach a saddle point, and be sprayed apart so that they 
  end up in very different parts of the phase space. Well, our bowed-string 
  model contains a mechanism that could do a similar job, separating two 
  initially similar transients so that they diverge. 

  Figure 12 shows a copy of Fig.\ 8 from section 9.2, and it reminds us of the 
  abrupt jumps that are inevitably generated by using this friction curve. We 
  can imagine two transients which start out similar, and then come close to 
  one of the critical points at which a jump is triggered. If one transient 
  falls just short of the critical point but the other one passes it, one will 
  have a jump and the other will not, and their subsequent development might be 
  quite different. This idea points us towards the second major flaw in the 
  early computer models: in the next section we will have a critical look at 
  this “friction curve model” and discover that the actual frictional behaviour 
  of a rosin-coated violin bow does not follow any single friction--velocity 
  curve: it is all much more complicated! 

  \fig{figs/fig-dcb541e6.png}{Figure 12. A copy of Fig. 8 from section 9.2, 
  reminding us that this friction model involves sudden jumps in the force and 
  velocity predictions. This may be the origin of the ``twitchiness'' seen in 
  the simulation results of Fig. 8.} 

  Before that, we should note something else about Fig.\ 9. At least for these 
  particular cases, we see that both bending stiffness and string torsion have 
  a significant effect on the results: including torsion makes things better 
  (more coloured pixels), while including bending stiffness makes things worse. 
  The extent of these differences is perhaps another manifestation of “chaotic 
  twitchiness”: as well as sensitive dependence on details of the bowing 
  transient, we are also seeing sensitive dependence on changes to modelling 
  details, by including or excluding various effects. 



  \sectionreferences{}[1] J. Woodhouse; ``On the playability of violins, Part 
  II minimum bow force and transients''; Acustica \textbf{78}, 137--153 (1993). 

  [2] R. T. Schumacher and J. Woodhouse, ``The transient behaviour of models of 
  bowed-string motion''; Chaos \textbf{3}, 509--523 (1995). 

  [3] J. C. Schelleng; ``The bowed string and the player'', Journal of the 
  Acoustical Society of America \textbf{53, }26–41 (1973). 

  [4] Knut Guettler, ``On the creation of the Helmholtz motion in bowed 
  strings''; Acta Acustica united with Acustica, \textbf{88}, 970--985 (2002). 

  [5] Paul M. Galluzzo; On the playability of stringed instruments, \tt{}PhD 
  Dissertation\rm{}, University of Cambridge (2003). See section 6.1. 

  [6] Hossein Mansour, Jim Woodhouse and Gary P. Scavone, ``Enhanced wave-based 
  modelling of musical strings, Part 1 Plucked strings''; Acta Acustica united 
  with Acustica, \textbf{102}, 1082--1093 (2016). 

  [7] Hossein Mansour, Jim Woodhouse and Gary P. Scavone, ``Enhanced wave-based 
  modelling of musical strings, Part 2 Bowed strings''; Acta Acustica united 
  with Acustica, \textbf{102}, 1094--1107 (2016). 

  [8] Bruce Stough, “E string whistles”; Catgut Acoustical Society Journal 
  (Series II), \textbf{3}, 7, 28—33 (1999). 