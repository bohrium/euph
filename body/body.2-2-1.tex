  Mathematically, the Fourier series works like this. Suppose the periodic 
  function you start with is $f(t)$, repeating with a period $T$. Then it can 
  be expressed by 

  \begin{equation*}f(t)=a_0+\sum_{n=1}^\infty \left( a_n \cos \dfrac{2n\pi 
  t}{T} + b_n \sin \dfrac{2n\pi t}{T} \right) \tag{1}\end{equation*} 

  \noindent{}where 

  \begin{equation*}a_0=\dfrac{1}{T}\int_0^T f(t) dt, a_n=\dfrac{2}{T}\int_0^T 
  f(t)\cos \dfrac{2n\pi t}{T} dt \tag{2}\end{equation*} 

  \noindent{}and 

  \begin{equation*}b_n=\dfrac{2}{T}\int_0^T f(t)\sin \dfrac{2n\pi t}{T} dt. 
  \tag{3}\end{equation*} 

  Alternatively, there is a neater form in terms of complex numbers: 

  \begin{equation*}f(t)=\sum_{n=-\infty}^\infty c_n e^{2in\pi t/T} 
  \tag{4}\end{equation*} 

  \noindent{}where 

  \begin{equation*}c_n=\dfrac{2}{T}\int_0^T f(t)e^{-2in\pi t/T} dt. 
  \tag{5}\end{equation*} 

  The two versions of the Fourier series are related in a very simple way, via 
  the real and imaginary parts of the complex numbers $c_n$: 

  \begin{equation*}a_n=\Re (c_n), \mathrm{~~~~}b_n=\Im (c_n) 
  \tag{6}\end{equation*} 

  We have written all the integrals in equations (2), (3) and (5) over the 
  range from $t=0$ to $t=T$, but we could equally well have chosen any other 
  complete range of length $T$, since the function $f(t)$ is periodic. 

  If the function $f$ is not periodic, like the hammer pulse discussed in the 
  main text, then instead of a sum over discrete harmonic frequency components 
  we need an integral: 

  \begin{equation*}f(t)=\int_{-\infty}^\infty F(\omega) e^{i \omega t} dt 
  \tag{7}\end{equation*} 

  \noindent{}where 

  \begin{equation*}F(\omega)=\dfrac{1}{2\pi}\int_{-\infty}^\infty f(t) e^{-i 
  \omega t} d\omega. \tag{8}\end{equation*} 

  $F(\omega)$ is called the Fourier transform of $f(t)$. 

  You should note that not everyone chooses to put the factor $2 \pi$ in 
  exactly the same place as it appears in equation (8). The factor always 
  appears somewhere, in either or both of equations (7) and (8), but there are 
  several possible ways to do it. These are all self-consistent, but lead to 
  slightly different definitions of the transform. I will always use the 
  convention shown here, but you may find alternative usage in textbooks or 
  other web sites. 

  The FFT algorithm gives a way to calculate a Fourier series for a sampled 
  function, where we only know the values of the function at times $t_n=nh$ 
  where $h$ is the chosen time step, the inverse of the sampling rate. So for 
  example a typical audio file with a sampling rate of 44.1 kHz has an interval 
  $h = 1/44100$ s = 22.7 $\mu$s. Although it is called the Fast Fourier 
  Transform, this is misleading: a computer can only ever handle a finite 
  number of sampled values, and what the FFT gives you is the set of (complex) 
  Fourier series coefficients based on harmonics of the frequency which is the 
  inverse of the total length of the sampled segment in seconds. 

  An important result, the \tt{}Nyquist-Shannon sampling theorem\rm{}, tells us 
  that the FFT contains frequency components only up to the Nyquist frequency, 
  which is half the sampling frequency (e.g. 22.05 kHz for the audio file). If 
  the original signal contained frequency content higher than this frequency it 
  will be represented in a misleading way in the FFT result, a phenomenon known 
  as aliasing. We will have more to say about aliasing, with an audio demo, in 
  section 10.4. 

  The example of a Fourier series plotted in Fig.\ 1 of section 2.2 is for the 
  sawtooth wave. This has been chosen deliberately, because we will be very 
  interested in the sawtooth wave when we come to examine the motion of a bowed 
  violin string in Chapter 9. So as an example of the calculation, we will 
  derive the explicit form of this Fourier series. If we want a sawtooth wave 
  with period $T$ and peak amplitude of 1, we can define 

  \begin{equation*}f(t) = \dfrac{2t}{T} \tag{9}\end{equation*} 

  \noindent{}within the range $-T/2 \le t \le T/2$, and then repeating 
  periodically. 

  It is simplest to calculate the complex form of the series: 

  \begin{equation*}c_n=\dfrac{4}{T^2} \int_{-T/2}^{T/2}{t e^{-2in \pi t/T} dt} 
  \tag{10}\end{equation*} 

  \noindent{}where we have chosen the integration range to match the way the 
  function was defined. 

  So, using integration by parts, 

  \begin{equation*}\dfrac{T^2 c_n}{4} = \left[ -t \dfrac{e^{-2in \pi 
  t/T}}{(-2in \pi /T)} \right]_{-T/2}^{T/2} -- \dfrac{T}{2in \pi} 
  \int_{-T/2}^{T/2}{e^{-2in \pi t/T} dt} \tag{11}\end{equation*} 

  \begin{equation*}=\dfrac{T}{2in \pi} \left\lbrace \dfrac{T}{2} e^{-in \pi} + 
  \dfrac{T}{2} e^{in \pi} \right\rbrace = \dfrac{T^2 (-1)^n}{2in \pi} 
  \tag{12}\end{equation*} 

  \noindent{}because the integral in equation (11) is zero. So finally 

  \begin{equation*}c_n = \dfrac{2(-1)^n}{in \pi} . \tag{13}\end{equation*} 