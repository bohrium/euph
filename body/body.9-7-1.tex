  The concept of Centre of Percussion is relevant to many problems in 
  mechanics, not only to violin bows. One very clear example is that it defines 
  the ``sweet spot'' of a baseball bat or cricket bat. Figure 1 shows a sketch 
  of the situation we are interested in, illustrated with a baseball bat. The 
  bat is pivoted about a point P, and is free to swing in a plane, displaced by 
  an angle $\theta$. The centre of mass G lies a distance $L_{\mathrm{G}}$ from 
  P. The bat has total mass $M$, and its moment of inertia about G is 
  $I_{\mathrm{G}}$. The moment of inertia about P can then be found by the 
  parallel axis theorem: 

  $$I_{\mathrm{P}}=I_{\mathrm{G}}+ML_{\mathrm{G}}^2. \tag{1}$$ 

  Initially, the bat is hanging vertically at rest. A horizontal impulse $F$ is 
  then applied at a point a distance $x$ from the pivot P. There will in 
  general be an impulsive reaction at the pivot, $R$. We are interested in the 
  angular velocity $\dot{\theta}$ immediately after the impulse: call this 
  $\Omega$. 

  The equation ``impulse equals change in momentum'' applied to horizontal 
  motion gives 

  $$F+R=M L_{\mathrm{G}} \Omega \tag{2}$$ 

  since the horizontal velocity of G is $L_{\mathrm{G}} \Omega$ immediately 
  after the impulse. We can also take moments about P, and write down the 
  angular version of the same general equation: 

  $$Fx=I_{\mathrm{P}} \Omega . \tag{3}$$ 

  Eliminating $\Omega$ between these two equations gives 

  $$ F+R=\frac{ML_{\mathrm{G}}Fx}{I_{\mathrm{P}}} . \tag{4}$$ 

  Now the centre of percussion is defined to be the position $x$ which gives 
  $R=0$. From equation (4), this position is given by 

  
  $$x=\frac{I_{\mathrm{P}}F}{ML_{\mathrm{G}}F}=\frac{I_{\mathrm{P}}}{ML_{\mathrm{G}}}. 
  \tag{5}$$ 

  Using equation (1), we can deduce that 

  $$x=\frac{I_{\mathrm{G}}+ML_{\mathrm{G}}^2}{ML_{\mathrm{G}}} = 
  L_{\mathrm{G}}+\frac{I_{\mathrm{G}}}{ML_{\mathrm{G}}}. \tag{6}$$ 

  Since the final term in this equation is always positive, we see that the 
  centre of percussion is always further from the pivot than G. 

  For the particular case of a uniform rod of length L, pivoted at one end, we 
  have $L_{\mathrm{G}}=L/2$ and $I_{\mathrm{G}}=ML^2/12$, so that equation (6) 
  gives $x=2L/3$. This tells us roughly where to expect to find the centre of 
  percussion for a violin bow, since the bow stick is not so very different 
  from a uniform rod. With the pivot point at the frog, the extra mass of the 
  frog makes relatively little difference. 

  If you want to find the centre of percussion experimentally, a direct 
  determination would involve a force sensor to measure the reaction $R$. But 
  there is an easier way, requiring no special equipment (and thus suitable for 
  a bow-maker to use in their workshop). Returning to Fig.\ 1, we can easily 
  write down the governing equation for free swinging of the bat or bow about 
  P: 

  $$I_{\mathrm{P}} \ddot{\theta} = -MgL_{\mathrm{G}} \sin\theta \tag{7}$$ 

  where $g$ is the acceleration due to gravity. It follows that 

  $$\ddot{\theta} =-\frac{g}{x} \sin \theta \tag{8}$$ 

  from equation (5). Now compare this with the corresponding equation for the 
  swinging of a simple pendulum of length $L$, which we saw in section 8.2.1: 

  $$\ddot{\theta} =-\frac{g}{L} \sin \theta. \tag{9}$$ 

  So $x$ is the length of a simple pendulum which swings in exact synchrony 
  with the object we are interested in. The measurement can thus be done in the 
  way shown in Fig.\ 2. The bow is hung from the same pivot rod as a simple 
  pendulum made from a heavy object (a steel nut in the picture) and a length 
  of thread. You set the two things swinging and adjust the length of the 
  thread until they swing together. The only pitfall to watch out for is that 
  the vibration described by equations (8,9) is nonlinear: the swing period 
  will get longer when the amplitude is bigger. So the two amplitudes should be 
  kept the same. 