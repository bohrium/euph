  This chapter gives an overview of nonlinear phenomena in acoustics and 
  vibration, concentrating on aspects relevant to musical instruments. This is 
  a large and complicated subject, and there is no way that this short account 
  can do full justice to it. The intention is to introduce and illustrate some 
  key concepts, especially ones that have direct relevance to musical 
  applications. 

  Sections 8.1 and 8.2 describe some common sources of nonlinearity, and 
  introduce the important distinction between “smooth” and “non-smooth” 
  nonlinearities. This makes a difference to what kind of analysis methods can 
  be brought to bear. An important technique that relies on smoothness is the 
  method of “harmonic balance”, introduced here through a simple example. 

  Section 8.3 introduces the idea of “phase space”, a powerful tool used by the 
  mathematicians to give a geometrical interpretation of nonlinear phenomena. 
  This leads on in section 8.4 to an introduction to the idea of “chaos”. Many 
  nonlinear systems, even ones that look beguiling simple, can exhibit motion 
  that, although fully determined by a mathematical equation, nevertheless 
  strikes an observer as having an unpredictable nature. A particular 
  manifestation of chaotic response is “sensitive dependence”: starting the 
  system off with two very similar initial conditions (but not completely 
  identical ones) produces two versions of the response which after a while 
  become completely different. 

  Section 8.5 brings us to a topic more obviously relevant to musical 
  instruments. Any instrument capable in principle of producing a sustained 
  note (rather than a transient note as on a guitar, piano or xylophone) must 
  involve nonlinearity, to compensate the energy dissipation that will occur in 
  all physical systems. So all the wind instruments and bowed-string 
  instruments rely for their very existence on nonlinearity. Some examples of 
  this kind of “self-excited vibration” are presented, as a taster for much 
  fuller discussion to come in later chapters in which bowed strings and wind 
  instruments are explored in detail. 

