  There is a simple way to see why the natural frequencies of a rectangular 
  plate with simply-supported edges give rise to a modal density that is 
  constant on average. Equation (6) of section 3.2.3 gave natural frequencies 

  \begin{equation*}\omega_{nm} = \sqrt{\frac{EK}{\rho h}} \left[\frac{n^2 
  \pi^2}{a^2}+\frac{m^2 \pi^2}{b^2} \right] \tag{1}\end{equation*} 

  \noindent{}which we can write in the form 

  \begin{equation*}\omega_{nm} = \sqrt{\frac{EK}{\rho h}} \left[k_x^2+k_y^2 
  \right] \tag{2}\end{equation*} 

  \noindent{}where $k_x$ and $k_y$ are the components of wavenumber in the $x$ 
  and $y$ directions respectively. The solutions allow $n$ and $m$ to take any 
  integer values 1,2,3..., so if we plot these in the wavenumber plane we 
  obtain a regular grid, sketched in Fig.\ 1. 

  \fig{figs/fig-3d9226ce.png}{\caption{Figure 1. Modes of a rectangular plate 
  of dimensions $0.6 \times 0.4$ m, plotted in wavenumber space (red stars). 
  The blue curve encloses the set of modes with natural frequencies below a 
  chosen frequency.}} 

  If we now ask how many modes have natural frequencies below some fixed value 
  $\Omega$, we have to count the number of points on this wavenumber grid lying 
  inside a quarter-circle with radius $r$ satisfying 

  \begin{equation*}\sqrt{\frac{EK}{\rho h}} r^2 = \Omega \tag{3}\end{equation*} 

  \noindent{}such as the one plotted in blue in Fig.\ 1. The points are 
  uniformly distributed over the plane, so this mode count is proportional to 
  the area inside the blue curve. But the radius $r$ is proportional to 
  $\sqrt{\Omega}$, and of course the area is proportional to $r^2$, so we 
  deduce that the count of natural frequencies below $\Omega$ is proportional 
  to $\Omega$. This is another way of saying that the average modal density is 
  a constant, independent of $\Omega$. 

  To obtain an explicit formula, we first note that the area of one ``box'' of 
  the wavenumber grid is $(\pi/a) \times (\pi/b)$. The area of the 
  quarter-circle is 

  \begin{equation*}A_c = \dfrac{\pi r^2}{4} = \dfrac{\pi}{4} \Omega 
  \sqrt{\dfrac{\rho h}{EK}} \tag{4}\end{equation*} 

  \noindent{}so using the value of $K$ from eq. (4) of section 3.2.3 we can 
  deduce that the number of ``boxes'' needed to fill this area, and hence the 
  mode count $N(\Omega)$, is given by 

  \begin{equation*}N \approx \dfrac{ab}{\pi^2} \dfrac{\pi}{4} \Omega 
  \sqrt{\dfrac{12 \rho (1-\nu^2)}{Eh^2}} = \Omega \dfrac{ab}{2 \pi h} 
  \sqrt{\dfrac{3 \rho (1-\nu^2)}{E}} . \tag{5}\end{equation*} 

  Thus the modal density $n$ is 

  \begin{equation*}n=\dfrac{dN}{d \Omega} = \dfrac{ab}{2 \pi h} \sqrt{\dfrac{3 
  \rho (1-\nu^2)}{E}} . \tag{6}\end{equation*} 

  This argument has been based on the special case of the rectangular plate, 
  with simply-supported edges. If the shape and/or the boundary conditions are 
  changed in some continuous way, the points in the wavenumber plot will move 
  around continuously, but they never appear or disappear. So the average 
  density of points in the plane remains fairly uniform, and the argument given 
  above still applies. Of course, the density of points and hence the numerical 
  value of the modal density may change, for example if the plate is made 
  smaller, larger or thicker. But it will remain the case that the average 
  modal density will remain roughly constant. 