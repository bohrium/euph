  In order to carry out frequency-domain synthesis of plucked strings as 
  described in section 5.4, we need two frequency response functions related to 
  the string: the drive-point admittance (or in fact its inverse, the 
  impedance) at the end of the string, and also the transfer function linking 
  motion at the end to motion at the required plucking point. 

  These need to incorporate the effects of bending stiffness and damping, 
  derived in sections 5.4.4 and 5.4.5. The same trick can be used for both 
  functions. We first solve the problem for the ideal string without damping or 
  stiffness: there are analytic formulae for both functions. We then expand the 
  resulting expressions in partial fractions, interpret the terms as 
  corresponding to the modes of the string, then adjust the complex pole 
  frequencies to allow for stiffness and damping, without changing anything 
  else. 

  Suppose the string is fixed at position $x=0$, and that a harmonic 
  displacement $y e^{i \omega t}$ is imposed at $x=L$. To satisfy the fixed 
  boundary condition, the string displacement must take the form 

  $$w=k \sin \frac{\omega x}{c} \tag{1}$$ 

  where $k$ is a constant, and $c=\sqrt{P/m}$ is the wave speed on the string. 
  Imposing the other boundary condition fixes the value of $k$, so that 

  $$w=y \dfrac{ \sin (\omega x/c)}{ \sin (\omega L/c)} \tag{2}$$ 

  This gives the first of our two frequency response functions, $w/y$. 

  Now if the end motion is caused by a force $f e^{i \omega t}$ applied to the 
  string, force balance requires 

  $$f=P \left. \dfrac{\partial w}{\partial x} \right|_{x=L}=\dfrac{Py\omega 
  \cos(\omega L/c)}{c \sin (\omega L/c)} \tag{3}$$ 

  so the required end impedance is 

  $$Z(\omega)=\frac{f}{i \omega y}=\frac{P}{ic} \cot \frac{\omega L}{c}=-i 
  Z_{string} \cot \frac{\omega L}{c} \tag{4}$$ 

  in terms of the string's characteristic impedance $Z_{string}=\sqrt{Pm}$. 

  The impedance $Z$ has poles (resonances) where 

  $$\frac{\omega L}{c} = n \pi, \mathrm{~~~for~~~}n=0, \pm 1,\pm 2,\pm 3... 
  \tag{5}$$ 

  We recognise these pole frequencies as the natural frequencies of the string, 
  $\omega_n$. The function $Z(\omega)$ is obviously periodic, so that the 
  coefficients, called residues, of these poles must all have the same value. 
  This value can be found by looking at the behaviour near $\omega=0$, where 

  $$\frac{P}{ic}\cot \frac{\omega L}{c} \approx \frac{P}{ic} \times 
  \frac{c}{\omega L}=\frac{-iP}{\omega L}. \tag{6}$$ 

  We can deduce that 

  $$Z(\omega) = -\frac{iP}{L} \sum_{n=-\infty}^{\infty}{\dfrac{1}{\omega -- n 
  \pi c/L}} . \tag{7}$$ 

  We can add small damping with loss factor $\eta_j$ for the $j$th mode, and 
  regroup the terms to look more like the familiar modal summation: 

  $$Z(\omega) = -\frac{iP}{L} \left[ \frac{1}{\omega} + 
  \sum_{j=0}^{\infty}{\left\lbrace \dfrac{1}{\omega -- \omega_j(1+i \eta_j 
  /2)}+\dfrac{1}{\omega + \omega_j(1-i \eta_j /2)}\right\rbrace} \right] $$ 

  $$=-\frac{iP}{L} \left[ \frac{1}{\omega} + \sum_{j=1}^{\infty}{ \dfrac{2 
  \omega -- i \omega_j \eta_j}{\omega^2 -- i\omega \omega_j \eta_j -- 
  \omega_j^2}} \right] \tag{8}$$ 

  where we can now substitute for $\omega_j$ the approximate formula including 
  the effect of bending stiffness, from eq. (10) of section 5.4.3. 

  A similar procedure can be applied to the other frequency response function 
  of interest, $w/y$ from eq. (2), to provide an approximation to the transfer 
  function from the end of the string to the plucking point $x=a$, including 
  the effects of damping and bending stiffness. The result is 

  $$\frac{w}{y} \approx \frac{a}{L} + \frac{c}{L}\sum_{j=1}^{\infty}{ (-1)^j 
  \dfrac{2 \omega \sin j \pi a/L}{\omega^2 -- i\omega \omega_j \eta_j -- 
  \omega_j^2}} . \tag{9}$$ 