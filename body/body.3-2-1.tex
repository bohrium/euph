  Bending vibration of a slender beam can be understood via the simplest 
  approximation, Euler-Bernoulli beam theory. Suppose the beam is made of 
  material with Young's modulus $E$ and density $\rho$, and that it has a 
  uniform cross-section with area $A$. We are interested in free vibration with 
  small transverse displacement $w(x,t)$. Consider the forces acting on a small 
  element of the beam, as sketched in Fig.\ 1. 

  If you imagine a cut across the cross-section of the beam, there will be an 
  equal and opposite pair for forces acting on the two faces of the cut, and 
  also an equal and opposite pair of moments. The force is called the shear 
  force, $S(x,t)$, and the moment is called the bending moment, $M(x,t)$. This 
  force and moment act on both ends of the small element, at positions $x$ and 
  $x + \delta x$. Within this simplest theoretical framework, the rotational 
  inertia of the element is ignored so that the element is approximately in 
  overall moment balance. This requires 

  $$S(x,t) \delta x \approx M(x+ \delta x,t) -- M(x,t). \tag{1}$$ 

  In the limit as $\delta x \rightarrow 0$ this implies 

  $$S \approx \dfrac{\partial M}{\partial x} . \tag{2}$$ 

  Now apply Newton's law to the transverse motion of the element: 

  $$\rho A ~\delta x \dfrac{\partial^2 w}{\partial t^2} \approx S(x,t) -- 
  S(x+\delta x,t) \tag{3}$$ 

  so that as $\delta x \rightarrow 0$, 

  $$\rho A \dfrac{\partial^2 w}{\partial t^2} = -\dfrac{\partial S}{\partial 
  x}. \tag{4}$$ 

  For linearised bending theory, the bending moment $M$ is proportional to the 
  curvature of the beam: 

  $$M(x,t) = EI\dfrac{\partial^2 w}{\partial x^2} . \tag{5}$$ 

  The constant of proportionality is called the bending rigidity, and is the 
  product of the Young's modulus $E$ with a quantity $I$ which is called the 
  second moment of area of the cross-section. For a rectangular section, 
  appropriate for our xylophone bar, there is a simple formula in terms of the 
  width $b$ and the thickness $h$: 

  $$I=\dfrac{1}{12} bh^3. \tag{6}$$ 

  Now combining eqs. (2), (4) and (5) we find the equation of motion for free 
  vibration of the beam: 

  $$m \dfrac{\partial^2 w}{\partial t^2}+EI \dfrac{\partial^4 w}{\partial 
  x^4}=0 . \tag{7}$$ 

  Because this is a fourth-order differential equation in $x$, we will need two 
  boundary conditions at each end of the beam. We are interested in the case 
  with free ends, so we obtain boundary conditions from the fact that both the 
  bending moment and the shear force must be zero: this requires 

  $$\dfrac{\partial^2 w}{\partial x^2}=0 \mathrm{~~and~~} \dfrac{\partial^3 
  w}{\partial x^3}=0~~\mathrm{(free~boundary).} \tag{8}$$ 

  Now we are ready to find the modes of a free-free beam, the case we are 
  interested in with both ends free. As usual, we start by looking for 
  solutions of the form 

  $$w(x,t) = u(x) e^{i \omega t} \tag{9}$$ 

  so that eq. (7) requires 

  $$EI\dfrac{d^4 u}{d x^4}= \rho A \omega^2 u. \tag{10}$$ 

  The general solution to this equation can be written 

  $$u=C_1 e^{ikx} + C_2 e^{-ikx} + C_3 e^{kx} + C_4 e^{-kx} \tag{11}$$ 

  where $C_1 -- C_4$ are arbitrary constants and the wavenumber $k$ satisfies 

  $$k^4 = \dfrac{\rho A}{EI}\omega^2 . \tag{12}$$ 

  The terms in eq. (11) can be described physically. The terms $e^{i \omega t} 
  e^{\pm i k x}$ describe sinusoidal waves which travel along the beam, to the 
  left or to the right, at a speed $\omega/k$. These are directly similar to 
  the waves found on the stretched string in section 2.1.1. The terms $e^{i 
  \omega t} e^{\pm k x}$ describe something new. These are disturbances on the 
  beam which do not travel, but simply decay exponentially in one direction or 
  the other along the beam. These are known as evanescent waves or near fields. 

  Equally well, instead of (11) we could write the general solution in the 
  equivalent form 

  $$u=K_1 \cos kx + K_2 \sin kx + K_3 \cosh kx + K_4 \sinh kx \tag{13}$$ 

  with a different set of constants $K_1 -- K_4$. This form turns out to be 
  more convenient for calculating the mode shapes. Applying the two boundary 
  conditions (8) at $x=0$ leads immediately to $K_1=K_3$ and $K_2=K_4$. 
  Enforcing the same two conditions at $x=L$ then requires 

  $$K_1 (-\cos kL +\cosh kL) +K_2 (-\sin kL +\sinh kL)=0 \tag{14}$$ 

  and 

  $$K_1 (\sin kL +\sinh kL) + K_2(-\cos kL +\cosh kL)=0. \tag{15}$$ 

  These two simultaneous equations for $K_1$ and $K_2$ both have zero on the 
  right-hand side, so they can only have non-zero solutions if the determinant 
  of the $2 \times 2$ matrix of coefficients is zero. This requires 

  $$(\cosh kL -\cos kL)^2 = (\sinh kL + \sin kL)(\sinh kL -- \sin kL) 
  \tag{16}$$ 

  which simplifies after a little algebra to the requirement 

  $$\cos kL \cosh kL =1. \tag{17}$$ 

  The roots of this equation give the only possible values of $k$, which then 
  lead to corresponding values of $\omega$ via eq. (12). These are the natural 
  frequencies. Having found the value of $k$ for a given mode, the mode shape 
  can be found by substituting back in the equations for the four constants 
  $K_1 -- K_4$ from the earlier calculation. 

  We cannot solve eq. (17) exactly to obtain the natural frequencies. However, 
  we can easily obtain an approximate answer using a graphical approach. 
  Rearrange eq. (17) in the form 

  $$\cos kL =1/\cosh kL. \tag{18}$$ 

  We can easily sketch the left- and right-hand side functions; intersections 
  of the two plots will give the roots. Figure 2 shows the result. As the value 
  of $kL$ increases, $\cosh kL$ increases rapidly so the intersections get 
  closer and closer to the zeros of $\cos kL$. This gives an approximate 
  expression for $n$th intersection: 

  $$kL \approx (n +1/2) \pi \tag{19}$$ 

  and the corresponding natural frequency is 

  $$\omega_n \approx \left[ \dfrac{(n+1/2)\pi)}{L}\right]^2 
  \sqrt{\dfrac{EI}{\rho A}} = \left[ \dfrac{(n+1/2)\pi)}{L}\right]^2 
  \sqrt{\dfrac{Eh^2}{12 \rho}} \tag{20}$$ 

  where the final expression is for the rectangular section. 