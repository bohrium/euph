  Consider vibration of a plane surface $S$, radiating sound into the 
  half-space on one side of the plane. The geometry is sketched in Fig.\ 1. 
  Choose an origin in the plane, and use a system of spherical polar 
  coordinates based on this point. The observation point is a distance $r$ 
  away, at angle $\theta$. A small element in the plane lies between two radii 
  $\xi$ and $\xi+\delta \xi$, and azimuthal angles $\phi$ and $\phi + \delta 
  \phi$, so it has area $\delta S= \delta \xi \cdot \xi \delta \phi$. Denote 
  its distance from the observation point by $R$. Using eq. (2) from section 
  4.3.1, we can write down the pressure produced at the observation point when 
  this element of the surface vibrates with a velocity $v e^{i \omega t}$: 

  \begin{equation*}\delta p = \dfrac{i \omega \rho_0 v \delta S}{2 \pi} e^{i 
  \omega t} \dfrac{e^{-i k R}}{R} \tag{1}\end{equation*} 

  \noindent{}where $k=\omega /c$ as usual, and the factor 2 rather than 4 in 
  the denominator arises because the volume flux is spreading into a half-space 
  rather than the full space. Integrating over all positions in the plane gives 
  Rayleigh's formula for the total pressure: 

  \begin{equation*}p = \dfrac{i \omega \rho_0}{2 \pi} e^{i \omega t} 
  \int_S{\dfrac{v e^{-i k R}}{R} dS} . \tag{2}\end{equation*} 

  \fig{figs/fig-a48d8fb8.png}{\caption{Figure 1. Configuration sketch for the 
  Rayleigh integral and the baffled piston calculation.}} 

  Now we apply this result to the special case of a rigid circular piston of 
  radius $a$ centred around the origin, vibrating with velocity amplitude $v$ 
  within an infinite rigid baffle. If we only seek the far-field result with $r 
  \gg a$, we can replace the $R$ in the denominator by the constant value $r$ 
  because the value of $R$ will vary little over the region of integration. 
  However, it is essential to keep the factor $R$ in the term $e^{-ikR}$ 
  because this phase factor is rapidly varying, and it governs the extent of 
  cancellation between the different elements making up the piston. 

  In terms of the Cartesian axes shown in Fig.\ 1, the observation point has 
  coordinates $(r \sin \theta, 0, r \cos \theta)$ while the element is at 
  position $(\xi \cos \phi, \xi \sin \phi, 0)$. Thus 

  \begin{equation*}R^2=(r \sin \theta -- \xi \cos \phi)^2 + \xi^2 \sin^2 \phi + 
  r^2 \cos^2 \theta \end{equation*} 

  \begin{equation*}= r^2 -- 2 r \xi \sin \theta \cos \phi +\xi^2 . 
  \tag{3}\end{equation*} 

  If we neglect the term $\xi^2$ because $r \gg \xi$, we have 

  \begin{equation*}R = r \left[ 1 -- \dfrac{2\xi}{r} \sin \theta \cos \phi 
  \right]^{1/2}\end{equation*} 

  \begin{equation*}\approx r \left[ 1 -- \dfrac{\xi}{r} \sin \theta \cos \phi 
  \right]=r-\xi \sin \theta \cos \phi \tag{4}\end{equation*} 

  \noindent{}using the binomial theorem. 

  Using this in eq. (2) gives 

  \begin{equation*}p\approx \dfrac{i \omega \rho_0 v}{2 \pi r} e^{i \omega t -- 
  ikr} \int_0^a{\xi \left[ \int_0^{2 \pi}{\exp(-ik\xi \sin \theta \cos \phi) d 
  \phi} \right] d\xi} \tag{5}\end{equation*} 

  \begin{equation*}=\dfrac{i \omega \rho_0 v}{2 \pi r} e^{i \omega t -- ikr} 
  \int_0^a{J_0(ky \sin \theta) \xi d\xi} \tag{6}\end{equation*} 

  \begin{equation*}= i \rho_0 c k a^2 v e^{i \omega t} \left[ \dfrac{J_1(ka 
  \sin \theta)}{ka \sin \theta} \right] \dfrac{e^{-ikr}}{r} 
  \tag{7}\end{equation*} 

  \noindent{}where $J_0$ and $J_1$ are Bessel functions, and we have made use 
  of two standard identities for Bessel functions. The term in square brackets 
  in eq. (7) describes the directional dependence of the far-field sound, 
  plotted in Fig.\ 10 of section 4.3. 