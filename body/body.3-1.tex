  Armed with an understanding of vibration modes and natural frequencies, there 
  is one interesting musical question we can tackle straight away. Most small 
  children get their first glimmerings of practical music-making by banging on 
  saucepan lids and tables, perhaps graduating after a bit to a toy xylophone. 
  (Strictly, these toys are glockenspiels rather than xylophones because they 
  have metal rather than wooden bars for the notes, but I will continue to call 
  them toy xylophones for the sake of familiarity.) At least in principle, a 
  child can play tunes on the ``xylophone'', but not really on the saucepan 
  lids. Why is that? In this chapter we will explore what is needed to create a 
  tuned percussion instrument, with examples from around the world. This will 
  reveal a story of ingenuity by people from many different cultures, which in 
  itself illustrates how widespread and deep is the human commitment to 
  music-making. 

  There is one type of vibrating object that everyone agrees you can use to 
  play music: a stretched string, as on a guitar or piano. When you pluck or 
  strike a stretched string, you always hear a sound with a definite musical 
  pitch. The pitch can be changed by varying the length, tension or weight of 
  the string, all familiar to musicians. What is different about strings, 
  compared to the sound of banging a saucepan or a table? 

  The simplest mathematical description of the vibration of a stretched string, 
  described in the next link, reveals a very special pattern in the set of 
  natural frequencies. Once the lowest (or fundamental) frequency has been 
  fixed by choosing the weight, tension and length of the string, then all the 
  other frequencies are whole-number multiples: if the first is $f_1$, then the 
  second is $f_2=2f_1$, the third $f_3=3f_1$ and the $n$th is $f_n=nf_1$. This 
  simple numerical pattern was already encountered in section 2.2: it is called 
  a harmonic series. The idealised stretched string has natural frequencies 
  that are harmonics, quite different from the irregularly-spaced natural 
  frequencies of the drum described in section 2.2. 

  Just as we saw for the drum, each natural frequency of the vibrating string 
  is associated with a particular mode shape. The first few of these shapes are 
  shown in Fig.\ 1: they consist simply of sine waves, fitted into the length 
  of the string in all the ways that allow the ends of the string to be fixed. 
  The simplest way, with the longest wavelength and the lowest frequency, has 
  one half-wavelength trapped in the length of the string. The next mode has 
  two half-wavelengths, and then three, and so on. 

  All modes except the lowest have one or more points where the string does not 
  move: these are called nodal points or just nodes. These will be important 
  later. If you want to support a structure in a way that doesn't interfere 
  with the vibration of a particular mode shape, these nodal points are the 
  place to use. For the particular case of a vibrating string, the nodal points 
  have another significance that is familiar to musicians. If you touch a 
  string lightly with a finger at the exact mid-point, then pluck or bow the 
  string somewhere else, your finger will rapidly damp out any motion which 
  does not have a node at the mid-point. You are left with the second mode, 
  fourth mode and so on. This is, of course, what a violinist or guitarist 
  calls ``playing a harmonic'', in this case at the octave. Touching the string 
  at 1/3 or 1/4 of its length will produce higher ``harmonics''. 

  Many aspects of this simple model of string vibration are not quite right if 
  examined carefully, and these details are very important for some musical 
  questions. For example, a musician's ``harmonics'' are not truly harmonics on 
  a real string, although they would be on our idealised model string. We will 
  return to this in section 5.4, with some details in section 5.4.3. But for 
  now, the simple model gives a strong hint about the most important thing we 
  need to know for investigating tuned percussion instruments. When a structure 
  has some or all of its natural frequencies in something close to a harmonic 
  series, any sound you can make on it is likely to be perceived as having a 
  definite musical pitch. Most structures, like the toy drum, do not have this 
  special property and they do not give rise to a definite pitch. 

  The art and science of creating a tuned percussion instrument, starting with 
  a structure that does not naturally produce harmonics, involves playing 
  around in some way with the design details in order to coax at least a few of 
  the lower natural frequencies into approximate harmonic relations. The more 
  modes that can be acoustically engineered like this, and the more accurate 
  the harmonic relations, the more clear will be the musical pitch. That, at 
  least, is my claim. Of course you should not believe it without some 
  evidence. 

  We will look at tuned percussion instruments of several different types, and 
  each case will be accompanied by sound files. By the time you have listened 
  to these you should become convinced that there is some truth in the claim 
  about the benefits of harmonically-related natural frequencies. That is not 
  to say that this is the only important issue for the sound quality of these 
  instruments, but it is definitely a big part of the story. 

  However, as we explore various sound examples we will rather rapidly find 
  ourselves amid the ``smoke and mirrors'' of perception. In case you think 
  that pitch is a simple thing, a famous auditory illusion serves to 
  demonstrate that the perception of pitch can be rather slippery. Listen to 
  the sound example below. You should hear a rising scale, with each note 
  seeming to be clearly higher than the one before. But after a bit you realise 
  that the pattern is somehow circular and repetitive. This particular illusion 
  takes advantage of the fact that our pitch perception is particularly shaky 
  when it comes to deciding which octave a given note is in: the pattern 
  ``sneaks'' down an octave, somewhere during the ever-rising pattern. Try 
  pausing the demo after each note and humming it: that will force you to pay 
  attention to the octave. This demo is a scale based on so-called ``Shepard 
  tones'': \tt{}look them up in Wikipedia\rm{} if you want to know how it is 
  done. 

