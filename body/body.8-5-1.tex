  Van der Pol's equation [1] for the displacement $x(t)$ of a nonlinear 
  oscillator is 

  $$\ddot{x}-\mu (1-x^2) \dot{x} + x =0 \tag{1}$$ 

  for free motion. The term scaled by the constant coefficient $\mu$ describes 
  a kind of nonlinear damping effect: it is intuitively clear that if $\mu > 
  0$, that term describes a kind of negative damping if $x < 1$, but that it 
  switches to positive damping when $x > 1$. The full behaviour is more 
  complicated than that simple description suggests, because of the nonlinear 
  nature of the term $x^2 \dot{x}$, but the argument conveys the correct 
  flavour insofar as trajectories starting with small values of $x$ tend to 
  spiral outwards (because of the negative damping) while trajectories starting 
  with large $x$ tend to spiral inwards. 

  For small values of $x$, the linearised version of the equation is simply 

  $$\ddot{x}-\mu \dot{x} + x \approx 0 \tag{2}$$ 

  which describes a linear oscillator with damping coefficient $-\mu$. The only 
  singular point of eq. (1) is at $x=0$, $\dot{x}=0$, and we can deduce 
  immediately from the linearised equation (2) that this is a centre if 
  $\mu=0$, a stable spiral if $ \mu < 0$ and an unstable spiral if $\mu > 0$. 

  You can find more information about the Van der Pol equation from the 
  \tt{}Wikipedia page\rm{}. One of the things revealed there, and expanded in 
  other pages linked from there, is that the equation played a role in the 
  early developments leading to chaos theory. Electrical engineers testing 
  circuits of the kind Van der Pol was originally interested noticed that they 
  did interesting things when driven with an external sinusoidal signal. This 
  would appear in eq. (1) as a forcing term $A \cos \omega t$ on the right-hand 
  side, for forcing with amplitude $A$ at frequency $\omega$. 

  When driven close to the frequency of the limit cycle, they found that the 
  oscillator was sometimes entrained: the oscillation frequency was ``pulled'' 
  to match the drive frequency. Something a bit similar can happen if two 
  pendulum clocks or two metronomes are sitting on the the same flexible table, 
  or hanging from the same non-rigid wall: the two clocks or metronomes may, 
  after a while, synchronise with each other if their separate frequencies were 
  already fairly close. This phenomenon was first reported by the 17th-century 
  Dutch astronomer and scientist \tt{}Christiaan Huygens\rm{}, and it is often 
  know by the term ``Huygens' clocks''. 

  Coming back to the electrical engineers, they noticed that for some ranges of 
  parameter values there was a kind of background noise accompanying the 
  periodic sound of the limit cycle. Initially they blamed flaws in the 
  equipment, but in 1945 the British mathematicians Mary Cartwright and J. E. 
  Littlewood [2] were able to show that the effect was predicted by Van der 
  Pol's equation. It was an early example of ``deterministic chaos''. 

  \sectionreferences{}[1] Van der Pol, B., ``On relaxation-oscillations'', \tt{}The London, 
  Edinburgh and Dublin Philosophical Magazine \& Journal of Science\rm{}, 
  \textbf{2}(7), 978–992 (1926). 

  [2] \tt{}Mary 
  Cartwright\rm{} and \tt{}J. E. Littlewood\rm{} 
  (1945) ``On Non-linear Differential Equations of the Second Order'', \tt{}Journal 
  of the London Mathematical Society\rm{} 20: 180 \tt{}doi\rm{}:\tt{}10.1112/jlms/s1-20.3.180\rm{} 