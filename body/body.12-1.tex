  \textbf{A. Why do things bounce?} 

  If you have been following the pattern of chapters so far, you might be 
  expecting an “underpinnings” chapter next. Well, the next topic doesn’t need 
  a whole chapter of underpinning, but this first section does indeed give some 
  background information for the musical applications in the ensuing sections. 
  The questions of interest all involve collisions, bouncing and buzzing. 

  We have already said a little about the mechanics of bouncing, when we 
  discussed impact hammers for frequency response measurement, back in section 
  2.2.6. In order to get a first impression of how hammers behave we used a 
  very crude approximation: we allowed for the mass of the hammer and an 
  effective stiffness acting between this mass and the structure being tapped, 
  but we treated that structure as being rigid. We did not allow for the fact 
  that the structure would vibrate in response to the tap — which is of course 
  the point of tapping in the first place, whether we are thinking of a 
  vibration measurement or a percussionist hitting a drum or a marimba bar. In 
  the course of this section we will rectify this omission. 

  A particularly simple kind of collision involving bouncing is illustrated in 
  Fig.\ 1. This shows two identical steel balls, suspended from a frame so that 
  they can only move along a circular arc. We start with one ball stationary, 
  and we release the other from a height. They collide, and the result is that 
  the moving ball stops dead, while the stationary ball moves off with the same 
  speed that the moving ball had before the collision. The second ball swings 
  out and then returns, and the process repeats in the opposite direction. 

\moobeginvid\begin{tabular}{ccc} \vidframe{ 0.30 }{ vids/vid-08324ccb-00.png }&\vidframe{ 0.30 }{ vids/vid-08324ccb-01.png }&\vidframe{ 0.30 }{ vids/vid-08324ccb-02.png } \end{tabular}\caption{Figure 1. Two identical steel balls, bouncing in the toy known as ``Newton's cradle''}\mooendvideo

  A variant of this example is shown in Fig.\ 2. This shows the same toy as in 
  Fig.\ 1, but now there are five identical balls. We drop a ball at one end, 
  and the ball at the opposite end flies off — but apparently without the balls 
  in between moving at all. There is a simple way to see why this behaviour 
  could have been anticipated. Suppose there was a very small space between 
  each ball and the next one. When you drop the first ball, the first thing 
  that happens is exactly the same as in Fig.\ 1: the first ball stops dead, 
  and the second one moves off at the original speed. But in a very short time 
  this ball will hit the third ball. The second ball, in its turn, will stop 
  dead while the third ball moves off. This process would be repeated all the 
  way along the chain, however many balls we had, until the last ball is 
  launched. This one doesn’t have another one to hit, so it flies off, swings 
  upwards, returns, and repeats the process in reverse. 

\moobeginvid\begin{tabular}{ccc} \vidframe{ 0.30 }{ vids/vid-8ea64739-00.png }&\vidframe{ 0.30 }{ vids/vid-8ea64739-01.png }&\vidframe{ 0.30 }{ vids/vid-8ea64739-02.png } \end{tabular}\caption{Figure 2. The more usual behaviour of Newton's cradle, with a row of identical balls.}\mooendvideo

  One way to think about the result of the first impact seen in Fig.\ 1 is that 
  we have used a mass (the right-hand ball) to strike a pendulum. This caused 
  the pendulum to vibrate, in the rather stately way that a swinging pendulum 
  does. It has only a single vibration frequency, and we saw a half-cycle of 
  that vibration before the ball returned, and a second impact occurred with 
  the original mass. But if we had moved the first mass, our “striker”, smartly 
  out of the way, the pendulum would have continued to swing. This is a simple 
  model for what happens when a percussionist hits a marimba bar, or an 
  acoustician hits a violin bridge with a small impulse hammer in order to 
  measure its vibration response. In both cases, multiple impacts are usually 
  not wanted (for reasons we will explore shortly). You want the percussion 
  beater or the impulse hammer to rebound out of the way, leaving the structure 
  free to vibrate. 

  But there may also be a conflicting requirement. If you want to make the 
  loudest sound on the marimba, you would like to put as much energy as 
  possible into the vibration. You supply a certain amount of kinetic energy in 
  the moving beater, just before it strikes. The most you could possibly hope 
  would be for all that energy to go into vibration of the struck object. This 
  is exactly the situation we saw in Fig.\ 1. The first ball transferred all 
  its kinetic energy to the second ball, the swinging pendulum. The first ball 
  did not rebound so as to get out of the way, and a second impact followed 
  shortly afterwards. If something similar happened to the marimba player, they 
  might describe the result as a “buzz” or as “chattering” of the beater. For 
  practical purposes, we perhaps need a compromise: transfer a reasonable 
  amount of kinetic energy, but allow a clean bounce. 

  There is a third important factor: the frequency spectrum of the force 
  applied by the bouncing beater, which will determine the brightness or 
  mellowness of the resulting sound. We will see shortly that there is an 
  interesting interaction between the three factors, linked to the design of 
  the beater and also to the vibration characteristics of the object you are 
  tapping. This interaction plays out somewhat differently in different 
  applications: we will find rather different requirements for an impact hammer 
  for an acoustic measurement, the choice of beaters or drumsticks for a 
  percussionist, and the design of a suitable clapper for a church bell. 

  First, a reminder of the simple calculation of the behaviour of a bouncing 
  hammer from section 2.2.6. The hammer is only in contact with the structure 
  for a very short time, but during that time we know that a force must act to 
  prevent the hammer-head from penetrating into the structure. The simplest 
  idealisation of that force is to imagine a very stiff spring separating the 
  two components. In the case of our bouncing steel balls, for example, this 
  contact spring force is provided by small deformations of the steel in a tiny 
  region around the contact, as indicated in a sketch in Fig.\ 3. For the 
  purposes of a simple model we can replace this by a contact spring joining 
  two rigid balls, as sketched in Fig.\ 4. 

  \fig{figs/fig-ab4a4929.png}{\caption{Figure 3. Close-up of the region near 
  the contact point, when the steel balls from Fig. 1 collide. Contact 
  stiffness is provided by small deformations of the steel in the region shaded 
  red.}} 

  \fig{figs/fig-5e7d999f.png}{\caption{Figure 4. The two colliding spheres from 
  Fig. 3, with the deforming region replaced by a contact spring.}} 

  During contact, we now have the two rigid masses linked by a spring. This 
  combination will produce a resonance frequency, as usual — it is called a 
  contact resonance. At the moment when contact begins, the balls are moving 
  towards one another, so while they remain in contact the spring force will 
  follow a sinusoidal waveform at the resonance frequency. If the balls were 
  sticky so that they remained together thereafter, the result would be the 
  dashed curve in Fig.\ 5. But in the absence of stickiness, the force can only 
  be compressive. Once the model calls for a tensile force, we know that the 
  balls will in fact separate. So (within this simple model) the contact force 
  during a single impact should follow a half-cycle of the sinusoidal wave, as 
  shaded in red in Fig.\ 5. 

  \fig{figs/fig-26e78589.png}{\caption{Figure 5. The force between two 
  colliding objects, allowing for their mass and for a linear contact spring as 
  in Fig. 4. The force would follow a half-cycle of a sinusoidal waveform at 
  the frequency determined by the ratio of contact stiffness to mass, until 
  bouncing occurred when that sinusoidal force would need to go negative. 
  (Figure reproduced from Fig. 1 of section 2.2.6.)}} 

  We can find the corresponding spectrum of the contact force by calculating 
  the Fourier transform of this half-cycle of sine wave: the details were given 
  in section 2.2.6. Two typical examples are shown in Fig.\ 6, reproduced from 
  Fig.\ 3 of that section. The plot shows results for two different values of 
  the contact resonance frequency. For the lower frequency (red curve), the 
  amplitude of the force spectrum is only high up to about 700~Hz, while for 
  the higher frequency (black curve) it extends up to around 4~kHz. In both 
  cases this effective bandwidth is a bit more than twice the contact resonance 
  frequency, something to bear in mind for later in this section. 

  \fig{figs/fig-a14bbc70.png}{\caption{Figure 6. The spectrum of the hammer 
  force from Fig. 5, for two different contact resonance frequencies: 300~Hz in 
  red; 1500~Hz in black. The vertical scale is logarithmic, expressed in 
  decibels (dB): each interval of 20~dB means the force has changed by a factor 
  of 10. (Figure reproduced from Fig. 3 of section 2.2.6.)}} 

  The resonance frequency is determined, as usual, by the ratio of the contact 
  stiffness to the mass. So a low stiffness or a high mass tends to give a low 
  resonance frequency, and correspondingly low bandwidth. A low mass or a high 
  stiffness gives the opposite result, and a high bandwidth. In other words, a 
  heavy, soft beater gives a more gentle, mellow tone from a struck object, 
  while a light, hard one gives a crisper, brighter sound: exactly as every 
  percussionist knows. 

  \textbf{B. The missing ingredients} 

  So far, so good: the simple model gives us some plausible and useful 
  information. However, there are four separate reasons why it is incomplete, 
  and therefore potentially unrealistic. We will outline all four issues, and 
  then explore the consequences using computer simulations of a model problem 
  which will allow us to add them in one at a time. 

  For the first missing ingredient, look back for a moment at Fig.\ 1. Without 
  writing down any equations, we can glimpse an important aspect of the 
  underlying physics of collision. Because the two balls are identical and the 
  collision is exactly aligned along the line joining the centres of the balls, 
  everything is symmetrical. After the first collision the ball that was 
  initially stationary moves off with (almost exactly) the same speed that the 
  moving ball previously had, and this tells us two things: the total kinetic 
  energy stays the same, and the total momentum stays the same. (Both kinetic 
  energy and momentum are calculated from the mass and the speed of the balls. 
  Since the balls have the same mass and the speed stays the same, both 
  quantities must be the same before and after the collision.) 

  However, I have glossed over something. Momentum really is conserved in a 
  collision, but kinetic energy is not quite conserved: some energy will always 
  be lost. If nothing else, the audible “click” of the bouncing balls means 
  that a small amount of energy has been carried away in the form of sound 
  waves. With our steel balls the loss is very small, but if the balls had been 
  made of something like wood a higher proportion of the energy might be lost, 
  mainly converted into heat associated with the contact deformation sketched 
  in Fig.\ 3. The ratio of kinetic energies before and after a collision can be 
  used to define a coefficient of restitution which can then be incorporated 
  into a simulation model: some details of this and other aspects of the 
  mathematical modelling of more realistic collisions are given in the next 
  link. 

  \textbf{C. Simulation results for hitting a rigid surface} 

  To explore the influence of these various factors we will look at a simple 
  model system, vaguely resembling a percussion instrument like a cymbal being 
  hit with a drumstick. The “instrument” is a rectangular thin plate, hinged 
  round its boundary, and the “drumstick” is a hinged-free bending beam which 
  hits the plate (through a contact spring) with its free end. The system is 
  sketched in Fig.\ 7, and some details of how the model has been implemented 
  in the computer were given in the previous link. This model is not meant to 
  be an accurate representation of a real musical instrument, or of a drumstick 
  and the way a drummer really holds it — we are looking for qualitative 
  insights by varying the model parameters to get an impression of, for 
  example, the circumstances that will lead to multiple impacts between the 
  plate and the drumstick. 

  \fig{figs/fig-49e67006.png}{\caption{Figure 7. The idealised system to be 
  used to investigate bouncing. A thin rectangular plate (the ``instrument'', 
  shown in blue) is struck by a beam (the ``drumstick'', shown in dark red), 
  hinged at the far end. During contact, a spring (in bright red) provides the 
  contact force.}} 

  The relevant vibration behaviour of the plate and the drumstick will be 
  adjusted by specifying the total mass and the lowest vibration resonance 
  frequency of each. Other details, such as the assumed aspect ratio of the 
  plate, the striking position on the plate, and the damping factors for both 
  plate and drumstick will be kept fixed (if you are curious, they were 
  specified in the previous link). 

  The first step is to verify that the model reproduces the behaviour sketched 
  in Fig.\ 5 when the plate is effectively rigid, and the lowest resonance 
  frequency of the stick is so high that it doesn’t affect the behaviour. 
  Specifically, we make the plate mass 500 metric tons, and the first stick 
  resonance 10~kHz. We also assume a linear contact spring, as in the 
  theoretical calculation. The red curve in Fig.\ 8 then shows the computed 
  waveform of contact force for a particular choice of the stick mass and 
  contact stiffness. It does indeed show the expected shape, a half-cycle of a 
  sine wave before a clean bounce occurs and contact is lost. 

  \fig{figs/fig-1675b962.png}{\caption{Figure 8. The computed force pulse when 
  both stick and plate are effectively rigid, computed for a linear contact 
  spring (red curve) and for a Hertzian spring (blue curve). The mass of the 
  drumstick was 0.1~kg, the linear spring stiffness was $k\_c =88.6~~ 
  \mathrm{kN/m}$, and the stiffness of the Hertzian spring was chosen to give 
  approximately the same contact time (the value was $k\_H = 695 ~ 
  \mathrm{kN/m}^{3/2}$).}} 

  The blue curve in Fig.\ 8 shows what happens if we replace the linear contact 
  spring with a nonlinear Hertzian spring, with a stiffness chosen to give more 
  or less the same contact time. There is still a single, symmetrical pulse of 
  force, but the shape is subtly different. At the moment of first contact, the 
  force builds up less abruptly than with the linear spring. But the force 
  grows progressively more steeply a little later in the pulse, a direct result 
  of the “hardening spring” behaviour. 

  We can see a different comparison between these two force pulses if we take 
  the FFT and look at their frequency spectra: the results are shown in Fig.\ 
  9, using the same plot colours as Fig.\ 8. The spectra are shown on a 
  logarithmic (dB) scale, normalised in the same way as Fig.\ 6 so that the 
  value tends to zero at very low frequency. For both pulses, the spectrum 
  looks generically similar to Fig.\ 6: high values restricted to a relatively 
  narrow band of low frequencies, then a pattern of sharp dips and secondary 
  peaks with decreasing height. We can see that the Hertzian spring (blue 
  curve) gives a slightly wider bandwidth at low frequency, and that the 
  frequencies of the secondary peaks are all a little higher, while the peak 
  heights decrease more rapidly (a consequence of the more gentle start and end 
  of the pulse). 

  \fig{figs/fig-71ad8be4.png}{\caption{Figure 9. Frequency spectra of the two 
  force pulses shown in Fig. 8, shown with the same plot colours. The spectrum 
  magnitude is shown on a dB scale normalised to the value zero at low 
  frequency, as in Fig. 6}} 

  It is useful to see a broader view of how things change when the drumstick 
  properties are varied. If we take a range of values of the mass and contact 
  stiffness, we can compute a grid of cases and then show the results in 
  graphical form rather similar to the “playability plots” we used earlier when 
  talking about bowed string and wind instruments. Figure 10 shows the contact 
  time, for a range of stick masses up to 200~g, and for a wide range of linear 
  contact spring stiffnesses. (Because of the wide range, varying over a factor 
  of 100, a logarithmic scale has been used on the vertical axis.) The colours 
  show a curving pattern, which for this preliminary case simply follows the 
  contour lines of the contact resonance frequency. 

  \fig{figs/fig-4398e773.png}{\caption{Figure 10. Contact time, for drumsticks 
  of a range of masses and tip stiffnesses impacting on an effectively rigid 
  surface. A linear contact spring is assumed here, with a stiffness varying 
  from 10~kN/m to 1000~kN/m on a logarithmic scale. The green circle marks the 
  case shown in red in Figs. 8 and 9.}} 

  Figure 11 gives an indication of how the force spectrum varies over the same 
  grid of simulated cases. The colours here show the spectral centroid, which 
  gives a simple guide to the frequency content of each force waveform. The 
  curves mark out the same contour lines of contact resonance frequency as in 
  Fig.\ 10. Comparing Figs.\ 10 and 11, we see the expected behaviour: a light 
  stick with a stiff tip gives a short contact time and a high spectral 
  centroid implying a bright sound, while a heavy stick with a soft tip gives a 
  long contact time, a lower spectral centroid, and a more mellow or muffled 
  sound. However, at this stage there is no ``sound'', because we are hitting 
  an essentially rigid surface. Shortly, we will relax this assumption and hear 
  some sound examples. 

  \fig{figs/fig-0d60b5cb.png}{\caption{Figure 11. Spectral centroid of the 
  force pulse waveform, for the same grid of simulations as in Fig. 12.}} 

  Figures 12 and 13 show corresponding plots with a Hertzian contact spring. 
  This time we do not have a simple mathematical result for how the contact 
  force behaves, but in fact both plots look broadly similar to Figs.\ 10 and 
  11 for the linear spring. The same trends are followed, and you have to look 
  rather carefully to spot the subtle differences in the shapes of the contours 
  mapped out by the colour shading. Provisionally, then, the nature of the 
  contact spring doesn’t make a huge difference to the predicted behaviour. 

  \fig{figs/fig-81e13e16.png}{\caption{Figure 12. Contact time for a similar 
  grid of simulations to Fig. 10 except that the linear contact spring has been 
  replaced by a Hertzian spring. The green circle marks the case shown in blue 
  in Figs. 8 and 9.}} 

  \fig{figs/fig-f150fa4b.png}{\caption{Figure 13. Spectral centroid of the 
  force pulse for the same grid of simulations as in Fig. 12.}} 

  \textbf{D. Simulations with flexible plates} 

  If make our plate or our drumstick (or both) have more realistic dynamic 
  behaviour, things immediately get more complicated. As a first step we will 
  change the plate behaviour, leaving the drumstick still essentially rigid. 
  Before showing detailed simulation results for this case, though, it is 
  useful to introduce an approximate argument which tells us something 
  important about what we can expect to see. 

  The basis of the argument is something we mentioned at the start of this 
  section, when we talked about the transfer of kinetic energy from a moving 
  “beater” to vibration of the structure which is struck. Every vibration mode 
  that is excited takes some of the kinetic energy that the drummer put into 
  the moving drumstick. Perhaps this makes it intuitively plausible that there 
  might be a limit to how many modes can be strongly excited before the supply 
  of kinetic energy is “used up”. 

  This idea proves to be correct — the details are explained in the next link. 
  The argument makes use of the force waveform shown in Fig.\ 5. We have 
  already seen (see Fig.\ 6) that the frequency bandwidth associated with a 
  pulse like this depends on the contact duration, or equivalently on the 
  contact resonance frequency. When a force pulse like this is applied to a 
  given structure, it is straightforward to calculate how strongly each mode 
  will be excited. 

  The total kinetic energy of the vibrating structure can then be calculated. 
  It depends on the contact resonance frequency, and there is a threshold value 
  of the contact resonance frequency above which the vibrating structure would 
  require more kinetic energy than is available. This puts a limit on the 
  shortest contact time that is possible if the drumstick is to rebound. At the 
  threshold the incoming drumstick would be stopped dead by the impact (just as 
  we saw with the steel balls in Figs.\ 1 and 2). A second impact is almost 
  certain to follow, when the vibrating structure returns and hits the 
  stationary drumstick. 

  We will illustrate with results for two different vibrating systems. First, 
  we look at an example of the vibrating plate described earlier (see Fig.\ 7), 
  chosen to have total mass 200~g and a lowest resonance frequency 100~Hz. This 
  plate will be impacted at the four different points shown in Fig.\ 14, and 
  the resulting threshold values of the contact resonance frequency are shown 
  as a function of the “hammer” mass in Fig.\ 15. Because the mass and the 
  threshold frequency cover a very wide range, logarithmic scales have been 
  used on both axes. 

  \fig{figs/fig-8753f68a.png}{\caption{Figure 14. Sketch of the vibrating plate 
  used to illustrate the approximate calculation of a threshold contact 
  resonance frequency. The plate is rectangular, with aspect ratio 1.55:1, 
  total mass 200~g and a lowest resonance frequency 100~Hz. It has ``hinged'' 
  boundaries all around its edge: the plate cannot move at the edge, but near 
  the edge it is free to rotate about the edge line. The four tapping positions 
  to be shown in Fig. 15 are indicated by stars, coloured to match the curves 
  in Fig. 15. They all lie on a line, and the proportional distances from the 
  left-hand edge are respectively 0.05, 0.2, 0.3 and 0.5.}} 

  \fig{figs/fig-043167b6.png}{\caption{Figure 15. Threshold contact resonance 
  frequency for the vibrating plate, as a function of the mass of the 
  ``hammer'' hitting the plate. The four curves correspond to the four 
  excitation points shown in Fig. 14: from left to right the curves are 
  coloured red, blue, green and black matching the stars in Fig. 14.}} 

  All four curves show a falling trend — a heavier hammer means that the 
  threshold frequency is lower. Three of the tapping positions give rather 
  similar results (the blue, green and black curves), but the position close to 
  the edge of the plate (the red curve) shows a significantly higher frequency 
  over the entire range. In other words, it is easier to excite a wide 
  bandwidth of response in this plate by tapping close to an edge. It is 
  important to recall that this plate has fixed edges all the way round, so 
  that the plate “feels” relatively rigid for a tapping position near the edge. 

  This behaviour contrasts with our second example, a vibrating beam with free 
  ends, like a marimba bar. The system, sketched in Fig.\ 16, is given the same 
  total mass and the same lowest resonance frequency as the plate. Again, it is 
  tapped at four different positions, giving the threshold results plotted in 
  Fig.\ 17. All four curves show a falling trend again, but the shape and 
  position of the curves is different from the ones in Fig.\ 15: changing the 
  system you tap can make a big difference to the bandwidth you can excite with 
  a given hammer. The main reason behind this contrast is the fact that the 
  resonances of a plate come thick and fast after the first one, whereas the 
  resonances of a bending beam are spaced progressively wider apart (see 
  section 4.2.4 for some detailed analysis of this difference of modal 
  density). 

  \fig{figs/fig-ff34f8d3.png}{\caption{Figure 16. Sketch of a vibrating beam 
  used to illustrate the approximate calculation of a threshold contact 
  resonance frequency. The beam has total mass 200~g and a lowest resonance 
  frequency 100~Hz. It has free boundaries, like a marimba bar. The four 
  tapping positions to be shown in Fig. 17 are indicated by arrows, coloured to 
  match the curves in Fig. 17. The proportional distances from the left-hand 
  edge are respectively 0, 0.03, 0.1 and 0.4.}} 

  \fig{figs/fig-3f2c6b00.png}{\caption{Figure 17. Threshold contact resonance 
  frequency for the vibrating beam, as a function of the mass of the ``hammer'' 
  hitting the plate. The four curves correspond to the four excitation points 
  shown in Fig. 16: from left to right the curves are coloured red, blue, green 
  and black matching the arrows in Fig. 16.}} 

  The other contrast with Fig.\ 15 lies in the position of the red curve 
  relative to the others. Again this red curve is associated with tapping near 
  the edge of the structure — in fact, right at the edge in this case. For the 
  beam, the red curve lies below the others rather than above them as it did 
  for the plate. The main reason this time is not to be sought in the 
  difference between plates and beams, though. Instead, the important 
  difference is in the edge conditions. The plate was fixed at the edge, while 
  the beam is free at its ends. So now the red curve is the result of tapping 
  at a position where the structure “feels” most floppy. This makes it harder 
  to avoid multiple impacts. 

  \fig{figs/fig-91d17972.png}{\caption{Figure 18. Map of contact time in the 
  same format as Fig. 10, for a case in which the plate mass is 2~kg. The 
  contact spring is linear, and all other parameters are the same as for Fig. 
  10.}} 

  Now we are ready to see some simulation results for the plate system. If the 
  total mass of the plate is set to 2~kg, with a first resonance frequency at 
  100~Hz, the equivalent of Fig.\ 10 now looks like the plot in Fig.\ 18. 
  Virtually nothing has changed: this plate is still heavy enough that the 
  bouncing process is not affected significantly. But if we reduce the mass 
  further to 200~g as in the case studied in Figs.\ 14 and 15, the result is 
  shown in Fig.\ 19: this time something looks obviously different, especially 
  in the upper right-hand region of the plot. To see what has happened, Fig.\ 
  20 shows a map of the same grid of simulations, coloured black where there 
  was a single, clean bounce and in colours where a multiple impact of some 
  kind occurred. These multiple impacts occur throughout the region that looked 
  different in Fig.\ 19. 

  \fig{figs/fig-d7a8597e.png}{\caption{Figure 19. Map of contact time in the 
  same format as Figs. 10 and 18, for a case in which the plate mass is 200~g. 
  The contact spring is linear, and all other parameters are the same as for 
  Fig. 10. The green circle marks the case shown in detail in Fig. 21, and the 
  black circle marks the case shown in Fig. 22.}} 

  \fig{figs/fig-beff6899.png}{\caption{Figure 20. Map showing single impacts 
  (black) and multiple impacts (colour) for the same set of simulations shown 
  in Fig. 19. The colour scale indicated the ratio of the contact stiffness to 
  the threshold value calculated by the approximate argument: the threshold 
  given by the approximate argument is marked by the boundary of the region of 
  white pixels. The green and black circles are as in Fig. 19, while the blue 
  circles mark the remaining cases illustrated with sound examples.}} 

  The colour scale in Fig.\ 20 indicates the ratio of the contact stiffness to 
  the threshold value calculated by the approximate argument, as in Fig.\ 15. 
  Where the pixels become white, the approximate criterion has been reached or 
  exceeded. We can see in the plot that there is a substantial range of contact 
  stiffness for which the simulations predict multiple impacts of some kind 
  although the limiting stiffness from the approximate argument has not been 
  reached. However, the general shape of the contours of these intermediate 
  colours follows the trend marked by the edge of the patch of white pixels, so 
  the approximate argument does give a useful indication of the pattern. 

  Some examples of the shape of the contact force waveforms for different plate 
  masses are shown in Figs.\ 21 and 22. The red curve in Fig.\ 21 is the same 
  as the red curve in Fig.\ 8, showing the half-sine shape when the impacted 
  plate was effectively rigid. The blue curve shows the result for the same 
  case with the plate mass 2~kg as in Fig.\ 18, and the green curve shows the 
  result for the 200~g plate as in Fig.\ 19. The green circle in Figs.\ 19 and 
  20 marks the pixel corresponding to these three waveforms. The blue curve has 
  slightly smaller magnitude than the red curve, because of energy transfer to 
  the plate, but otherwise the pulse shape looks very similar. Comparing with 
  Fig.\ 20, this pixel lies just above the region where single impacts occurred 
  for the 200~g plate, and the green curve shows the kind of behaviour we might 
  have guessed: a double-humped shape, but not quite losing contact in between 
  to give multiple impacts. 

  \fig{figs/fig-f7d4f144.png}{\caption{Figure 21. The pulses of contact force 
  for parameter values corresponding to the green circle in Fig. 15, for the 
  three cases so far analysed: the red curve shows the rigid plate as in Fig. 
  10, the blue curve shows the 2~kg plate from Fig. 14, and the green curve 
  shows the 0.5~kg plate from Fig. 15.}} 

  Figure 22 shows corresponding results for the pixel marked by a black circle 
  in Figs.\ 19 and 20. The red and blue pulses in this plot show a shorter 
  contact time than the corresponding ones in Fig.\ 21, as expected with the 
  higher contact stiffness. Looking at Fig.\ 20, we see that this pixel lies in 
  the white region for the 200~g plate, where the criterion based on kinetic 
  energy has been exceeded. The plot confirms this. The green curve shows a 
  rather ragged waveform, with three separate impacts. There is a significant 
  delay before the third impact: the drumstick had to wait for the vibrating 
  plate to come back and hit it before it was thrown off, away from any further 
  impacts. 

  \fig{figs/fig-6d330bb4.png}{\caption{Figure 22. The pulses of contact force 
  for parameter values corresponding to the black circle in Fig. 19, for the 
  three cases: the red curve shows the rigid plate as in Fig. 10, the blue 
  curve shows the 2~kg plate from Fig. 18, and the green curve shows the 200~g 
  plate from Fig. 19. The green curves shows obvious multiple impacts.}} 

  Figures 23 and 24 show the corresponding frequency spectra to Figs.\ 21 and 
  22 respectively, plotted in the matching colours. In both cases we see that 
  as the impacted plate gets lighter, the dips in the spectrum get shallower. 
  For cases in which there is only a single impact, the relatively subtle 
  modification to the pulse shape from impacting a vibrating plate has resulted 
  in a smoother spectrum. But the green curve in Fig.\ 24 shows more ragged 
  behaviour, which is perhaps not too surprising. We will come back to these 
  observations in a bit — they have implications for measurements using an 
  impulse hammer. 

  \fig{figs/fig-15df465b.png}{\caption{Figure 23. Frequency spectra for the 
  three pulses shown in Fig. 21, plotted in the corresponding colours.}} 

  \fig{figs/fig-5aef3b17.png}{\caption{Figure 24. Frequency spectra for the 
  three pulses shown in Fig. 22, plotted in the corresponding colours.}} 

  You may be wondering what these simulated plate impacts sound like. Some 
  examples are given in Sounds 1—5, corresponding to the five cases marked by 
  circles in Fig.\ 20 for the 200~g plate. In each case, what you are listening 
  to is the waveform of plate velocity at the struck point. Sound 1 is the 
  datum case, marked by the green circle and lying in the middle of the 
  diagram. Sounds 2 and 3 illustrate what happens if we keep the same contact 
  stiffness but vary the hammer mass: Sound 2 goes with the left-hand blue 
  circle, and Sound 3 with the right-hand one. Sounds 4 and 5 give the 
  corresponding comparison for keeping the mass the same but varying the 
  contact stiffness. Sound 4 corresponds to the blue circle at the bottom of 
  the diagram, Sound 5 to the black circle at the top. 

\audio{}

\audio{}

\audio{}

\audio{}

\audio{}

  All five examples give a recognisable impression of the kind of sounds you 
  might make by hitting a metal plate with a stick of some sort. They are quite 
  “unmusical” sounds, because this plate does not represent a carefully tuned 
  percussion instrument — there are no harmonic relationships between the 
  resonance frequencies (look back at Chapter 2 for a reminder of why that is 
  important). Compared to the datum case in Sound 1, Sound 2 has a shorter 
  contact time because of the lower mass, while Sound 3 has a heavier mass and 
  a longer contact time. The effect of those changes on the degree of 
  “brightness” is clear. Sounds 4 and 5 give a somewhat similar contrast in 
  brightness, this time caused by a change in contact stiffness rather than 
  mass. 

  Sound 5 stands out as being a bit “rough”. Recall that this case falls in the 
  white region of Fig.\ 20, where the approximate criterion for double bouncing 
  has been exceeded. We saw the resulting force waveform in the green curve in 
  Fig.\ 22, and now we hear the consequence of those multiple impacts for the 
  sound. It is easy to believe that under some circumstances a percussionist 
  would not want this kind of rough sound. To avoid that, Fig.\ 20 shows that 
  they need to choose a lighter stick or one with a softer head: starting from 
  the black circle you can escape the white region either by moving left or by 
  moving down. 

  \textbf{E. Simulations with a flexible drumstick} 

  Now we can add the final ingredient to the model, by allowing the “drumstick” 
  to have vibration resonances in the frequency range of interest. We want to 
  plot an image similar to the earlier ones, so we still want the drumstick 
  mass to be a variable. I have chosen a simple approach: the change in mass is 
  achieved by choosing different diameters of circular rod, assuming that the 
  drumstick is always made of the same material. The theory of bending beams 
  (see section 3.2.1) then tells us that the resonance frequencies of the beam 
  scale proportional to the square root of the mass (so thinner, lighter beams 
  have lower resonance frequencies, as you would expect). For a specific model 
  that might be in the right ballpark for a normal drumstick, I have chosen the 
  lowest frequency to be 1~kHz when the mass of the beam is 30~g, and then used 
  the scaling relation to calculate the frequency for other masses. 

  Running a set of simulations over the same range as earlier figures gives the 
  plot shown in Fig.\ 25. The results look very similar to Fig.\ 20 over most 
  of the plane, but at the left-hand side something new has appeared: instead 
  of black pixels down the left-hand side, we see a lot of red. This red colour 
  is a bit misleading, though. I have calculated the threshold value of contact 
  stiffness exactly as before, using the approximate argument based on the 
  vibration modes of the plate. But really we should include the modes of the 
  drumstick as well in this calculation — the previous link explains how to do 
  this. When that is done, the result is shown in Fig.\ 26. An additional patch 
  of white pixels has appeared in the top left corner, and the pattern of the 
  other colours now tracks the white region in very much the same way as found 
  earlier. 

  \fig{figs/fig-843bb089.png}{\caption{Figure 25. Plot in the format of Fig. 
  20, using the same plate as in that figure but now allowing for a flexible 
  drumstick with a lowest resonance frequency at 1~kHz when the mass is 30~g. 
  For other masses, the lowest frequency is scaled to represent the fact that 
  the material of the stick remains the same, but the diameter has been changed 
  to vary the mass. The threshold value of contact stiffness based on the 
  approximate argument has been calculated using only the modes of the plate, 
  as in Fig. 20.}} 

  \fig{figs/fig-d8929dac.png}{\caption{Figure 26. Plot of the same data as in 
  Fig. 25, except that now the modes of the ``drumstick'' have been included in 
  the calculation of the threshold value of contact stiffness. The green circle 
  marks the case shown in Figs. 27 and 28, and in Sound 6. The black circle 
  marks the case shown in Figs. 29 and 30, and in Sound 8.}} 

  We deduce that the vibration of the drumstick makes multiple contacts more 
  likely, especially if the stick is quite light. To see the consequences, two 
  cases have been chosen to show in detail. They are marked with circles in 
  Fig.\ 26: both involve the “nominal” 30~g drumstick, with a first resonance 
  at 1~kHz. One is not far above the black pixels, the other is up in the white 
  region. For the first of these, marked by the green circle, the waveform of 
  contact force is shown in red in Fig.\ 27. The plot also shows the 
  corresponding case from Fig.\ 20, with the rigid drumstick. The red waveform 
  shows a clear double peak, although it does not lose contact in between so it 
  is not strictly a double bounce. 

  \fig{figs/fig-bf8275d3.png}{\caption{Figure 27. The force waveform 
  corresponding to the case marked with a green circle in Fig. 26 (red curve, 
  flexible drumstick), and the corresponding case from Fig. 20 (blue curve, 
  rigid drumstick).}} 

  Figure 28 shows the frequency spectra of these two force pulses, and it also 
  shows the spectrum of the resulting plate velocity in the two cases. The 
  colours match Fig.\ 27: both spectra for the flexible drumstick are shown in 
  red, and both for the rigid drumstick are in blue. The smooth curves show the 
  force spectrum, and the obviously jagged curves show the plate velocity, with 
  peaks at all the resonance frequencies. 

  \fig{figs/fig-33ac46b4.png}{\caption{Figure 28. Frequency spectra 
  corresponding to the case shown in Fig. 27 and marked by a green circle in 
  Fig. 26. Red curves correspond to the case with a flexible drumstick (Fig. 
  26), black curves to the case with a rigid drumstick (Fig. 20). For each, the 
  smooth curve shows the spectrum of the contact force, while the jagged curve 
  shows the spectrum of the plate velocity at the contact point.}} 

  This comparison of spectra tells an interesting story. Look first at the two 
  smooth curves. Near the drumstick resonance at 1~kHz, the red curve has a 
  significant dip — it falls over 15~dB below the blue curve. This difference 
  is manifested directly in the spectra of plate velocity: if you look 
  carefully, you can see that the peak heights in the red curve fall well below 
  those of the blue curve in this frequency range. What has presumably happened 
  is that the first resonance of the flexible drumstick has had a strong 
  influence on the spacing of the double hump in Fig.\ 27. Even though the 
  total contact duration is only about 1~ms, this is enough for the stick 
  resonance to make itself felt in the force spectrum. 

  Figures 29 and 30 show the same comparisons for the case marked with a black 
  circle in Fig.\ 26. Both force waveforms are now very jagged, as we would 
  expect since we are up in the white region. At first glance the rigid stick 
  (in blue) shows more drastic effects, with several separate contacts. The 
  flexible stick (in red) has only a single contact, but it lasts longer than 
  the main pulse in the blue waveform and has a more complicated shape. When we 
  look at the comparison of spectra in Fig.\ 30, we see an important 
  consequence of this complicated shape. The force spectrum for the flexible 
  stick (smooth curve in red) shows a dip around the resonance frequency at 
  1~kHz, very similar to the case shown in Fig.\ 27. 

  \fig{figs/fig-4fd3371b.png}{\caption{Figure 29. The force waveform 
  corresponding to the case marked with a black circle in Fig. 26 (red curve, 
  flexible drumstick), and the corresponding case from Fig. 20 (blue curve, 
  rigid drumstick).}} 

  \fig{figs/fig-b4e3759a.png}{\caption{Figure 30. Frequency spectra 
  corresponding to the case shown in Fig. 29 and marked by a black circle in 
  Fig. 26. Red curves correspond to the case with a flexible drumstick (Fig. 
  26), black curves to the case with a rigid drumstick (Fig. 20). For each, the 
  smooth curve shows the spectrum of the contact force, while the jagged curve 
  shows the spectrum of the plate velocity at the contact point.}} 

  The conclusion is that the first resonance of the drumstick has a 
  surprisingly consistent effect on the two cases. The obvious next question 
  is: “can you hear it?” Sounds 6 and 7 allow you to listen to the two plate 
  velocity waveforms for the case shown in Figs 27 and 28. Sound 6 is for the 
  flexible drumstick, Sound 7 the comparison for the rigid one. Sounds 8 and 9 
  give the same comparison for the cases shown in Figs.\ 29 and 30. You may 
  need to ensure good audio reproduction to hear this effect clearly 
  (headphones may work best), but I think you will agree that in both cases you 
  can hear a difference of sound between the flexible and rigid drumsticks. 
  Furthermore, it is a rather similar kind of difference in both cases. 

\audio{}

\audio{}

\audio{}

\audio{}

  It seems a good guess that you are hearing directly the effect of reducing 
  the sound from the vibrating plate near the lowest resonance of the flexible 
  drumstick. Of course, other details may also influence the sound (such as the 
  other resonances of the stick). But tentatively, this simple example lends 
  support to the claim of percussionists that different sticks give different 
  sounds. Now, we must not over-interpret this example. Drummers do not hold 
  their sticks at the end like our model, and the “drumstick” model is very 
  crude compared to the design of real drumsticks (which usually taper and have 
  a distinct “head”). These factors will influence the resonance frequencies 
  and the effective masses of those resonances as felt at the striking point. 
  But the idea that one, and possibly more, of the stick resonances can 
  sometimes show up in the sound as audible dips in the spectrum surely 
  deserves to be investigated in more detail. 

  There is very little scientific literature about the vibration of real 
  drumsticks: the only measurements seem to be contained in a Masters thesis by 
  Andreas Wagner [1]. Figure 31 reproduces some of his results: measured 
  resonance frequencies and mode shapes for two commercial drumsticks. Each 
  stick was supported by a soft clamp at the balance point or fulcrum point, 
  which is a position that drummers are often recommended to hold a stick for 
  playing involving fast bouncing. The drummer’s fingers will no doubt add 
  significant damping to any vibration mode which is moving at this point, but 
  intriguingly both sticks show a mode near 900~Hz with a rather extended 
  near-nodal region around the fulcrum. This mode might be a strong candidate 
  for influencing the timing of multiple bounces, and thus affecting the sound 
  of the struck object. 

  \fig{figs/fig-0b413d46.png}{\caption{Figure 31. Measured vibration modes of 
  two drumsticks, taken from Wagner [1] and reproduced by permission. The 
  measurements were made with each stick supported by a soft clamp at its 
  fulcrum point, which is the recommended place for a drummer to hold a stick}} 

  \textbf{F. Coda: implications for impact hammer testing} 

  So far, the discussion of bouncing has mainly been in the context of the 
  sound of a struck percussion instrument. However, the modelling and analysis 
  are also relevant to something else we have met in earlier chapters: the use 
  of an impact hammer to excite a structure, not in order to make a particular 
  noise on it but as part of the process of measuring something like the input 
  admittance. There are subtleties of detail (some of these have been discussed 
  in sections 5.1.1, 10.4 and 10.4.2), but in essence such a measurement goes 
  like this. The structure is tapped, and the force waveform and the 
  structure’s response are collected into a computer. Both are converted to 
  frequency spectra by using the FFT, then finally the required frequency 
  response is obtained by dividing the response spectrum, frequency by 
  frequency, by the force spectrum. 

  The useful frequency range of such a measurement depends on the force 
  spectrum. We need to divide by that spectrum, so we certainly can’t tolerate 
  frequencies at which it is zero. But there is a more insidious problem: all 
  measurements contain some noise, so in practice we can only use frequencies 
  for which the force spectrum rises significantly above the “floor” determined 
  by that noise. 

  If the impact gives a single, clean pulse rather like the idealised version 
  from Fig.\ 5, the force spectrum will look like a somewhat noisy version of 
  the examples in Fig.\ 6. Everything should work for frequencies within the 
  first “hump” of the force spectrum, but we won’t be able to get useful 
  information as we approach the frequency of the first zero in the spectrum. 
  We might or might not get some useful information from frequencies lying near 
  the subsequent peaks in the spectrum, but the level is low compared to the 
  hump at low frequency, and at best the measurements will be noisy and rather 
  unsatisfactory. The conclusion from this, which we have already seen earlier, 
  is that if you want to measure frequency response reliably up to high 
  frequency, you need to achieve a very short hammer pulse: this needs a light 
  hammer, and a high contact stiffness. 

  The discussion in this section now reveals a snag. Hammers with high contact 
  stiffness are liable to give multiple bounces rather than a single contact, 
  especially if the hammer itself has a resonance within the frequency range 
  you are trying to cover. Does this matter? The force spectra in the red 
  curves of Figs.\ 28 and 30 suggest a potential problem. Both show a strong 
  dip around the frequency of the “hammer” resonance, determined by the spacing 
  of the “double hump” in a force waveform like the red curve in Fig.\ 27. This 
  might be a significantly longer time than the duration of a single pulse, so 
  the dip in the force spectrum will occur at a lower frequency than we were 
  expecting and the useful frequency range for the measurement will be reduced. 

  The extreme case arises when a “double bounce” from the hammer produces two 
  pulses of similar height, like the idealised example plotted in Fig.\ 32. The 
  corresponding frequency spectrum is shown in Fig.\ 33, and we can see that it 
  has very deep troughs indeed. Whatever the noise floor of the measurement 
  equipment might be, a dip like this is guaranteed to fall below it and thus 
  limit the usable bandwidth of the measurement. The less symmetrical double 
  hump in the red waveform of Fig.\ 28 gave a far shallower dip, perhaps 
  shallow enough that it still lies above the noise floor. The multiple 
  contacts shown in the green curve in Fig.\ 22 give rise to the force spectrum 
  shown in Fig.\ 24, which has a lot of small dips --- these will not 
  necessarily be deep enough to cause big problems with a frequency response 
  measurement. 

  \fig{figs/fig-5a3089b9.png}{\caption{Figure 32. Idealised example of a 
  symmetrical double strike from an impulse hammer}} 

  \fig{figs/fig-191b2f88.png}{\caption{Figure 33. The frequency spectrum of the 
  idealised force shown in Fig. 32}} 

  The conclusion is that double bounces from an impact hammer might or might 
  not be tolerated in a measurement, depending on the noise floor of the 
  measuring equipment together with the details of the structure being hit, the 
  vibration behaviour of the hammer, and the consequent symmetry or asymmetry 
  of the force waveform. Any trough in the force spectrum will result in some 
  reduction of the signal-to-noise ratio of the frequency response measurement, 
  but if the effect is not too extreme you may be able to compensate by 
  repeating the measurement more times and getting the advantages of averaging. 
  Reducing the noise level in the measurement equipment always helps, but it is 
  always worth keeping an eye on the force waveform and its spectrum. If you 
  see a double bounce with rather similar pulses, you should expect problems 
  and take steps to change something. 

  \sectionreferences{}[1] Andreas Wagner, “Analysis of drumbeats — interaction 
  between drummer, drumstick and instrument”, \tt{}MSc dissertation\rm{}, Dept. 
  of Speech Music and Hearing, KTH Royal Institute of Technology, Stockholm 
  (2006). 