  So far, we have treated damping in a rather cavalier, way, simply by 
  analysing undamped systems, then finding a suitable damping factor or 
  Q-factor to assign to each individual mode. For most purposes this is all 
  that is required. However, occasionally it is necessary dig a little deeper, 
  and the problem of coupling two strings via a soundboard mode is one of those 
  occasions. The underlying issue is that, taken by themselves, the string 
  modes and the body mode will have very different Q-factors: especially for 
  metal strings, the strings will have far higher Q-factors (or far lower 
  damping) than the body mode. When we want to analyse how things change when 
  we couple these together, this difference of damping levels will play a 
  significant role in the behaviour. We will first develop the basic theory, 
  then apply it to the coupled-string problem. 

  There is no completely general mathematical model of damping. This is in 
  strong contrast to the behaviour of mass and stiffness for small vibration of 
  any system. As we showed back in section 2.2.5, there are very strong reasons 
  to expect any undamped system to be described by a mass matrix and a 
  stiffness matrix, capturing the dominant behaviour of the potential energy 
  and the kinetic energy of the system, respectively. However, a good 
  indication of the behaviour to be expected from damped systems can be gained 
  from a particular damping model. We showed in section 2.2.5 that undamped 
  vibration of a system with $N$ degrees of freedom can be represented by the 
  equation 

  $$M \ddot{\mathbf{q}} + K \mathbf{q}= 0, \tag{1}$$ 

  where $\mathbf{q}$ is a vector of generalised coordinates, $M$ is the mass 
  matrix and $K$ is the stiffness matrix. Both matrices have dimensions $N 
  \times N$. Physically, these equations can be interpreted as representing $N$ 
  coupled harmonic oscillators. It seems very natural to extend this to damped 
  systems by adding a new term by analogy with the expression for a damped 
  harmonic oscillator derived in section 2.2.7, to represent $N$ coupled, 
  damped oscillators: 

  $$M \ddot{\mathbf{q}} + C \dot{\mathbf{q}} + K \mathbf{q}= 0, \tag{2}$$ 

  where $C$ is the damping matrix or dissipation matrix. Damping governed in 
  this way by the generalised velocities $\dot{\mathbf{q}}$ is called viscous 
  damping. 

  Now recall from section 2.2.5 that the undamped equation (1) can be derived 
  from Lagrange's equations, and that the mass and stiffness matrices can be 
  assumed symmetric because they arise from the quadratic expressions for the 
  kinetic and potential energy respectively. A similar argument can be used to 
  deduce that the damping matrix can also be assumed symmetric. Multiplying eq. 
  (2) by $\dot{\mathbf{q}}^t$ (where $t$ denotes the transpose of the vector) 
  gives 

  $$\dot{\mathbf{q}}^t M \ddot{\mathbf{q}} + \dot{\mathbf{q}}^t C 
  \dot{\mathbf{q}} + \dot{\mathbf{q}}^t K \mathbf{q}= 0$$ 

  $$\therefore \dot{\mathbf{q}}^t C \dot{\mathbf{q}} = -\dfrac{d}{dt}\left[ 
  \dfrac{1}{2}\dot{\mathbf{q}}^t M \dot{\mathbf{q}} + \dfrac{1}{2}\mathbf{q}^t 
  C \dot{\mathbf{q}}\right] = -\dfrac{d}{dt}\left[ \mathrm{total~energy} 
  \right] \tag{3}$$ 

  making use of the fact that both $K$ and $M$ are symmetric. This makes it 
  clear that the function 

  $$F= \dfrac{1}{2} \dot{\mathbf{q}}^t C \dot{\mathbf{q}} \tag{4}$$ 

  known as the Rayleigh dissipation function, is half the rate of dissipation 
  of energy. This is another quadratic expression, in which we can choose the 
  entries in the matrix to be symmetric. Working back via Lagrange's equations, 
  with a generalised force 

  $$Q_j=\dfrac{\partial F}{\partial \dot{q_j}} \tag{5}$$ 

  representing the damping force, we arrive back at eq. (2). 

  For the undamped system (1) we would find modes by calculating the 
  eigenvalues and eigenvectors from 

  $$K \mathbf{u}^{(n)}=\omega_n^2 M \mathbf{u}^{(n)} . \tag{6}$$ 

  When we change to modal variables, or ``normal coordinates'', both $M$ and 
  $K$ are diagonalised. It is not possible to do this for the system of eq. (2) 
  — it is not in general possible to find a change of variables which 
  diagonalises three matrices simultaneously. There are special cases for which 
  it is possible, for example if 

  $$C= \alpha M + \beta K \tag{7}$$ 

  where $\alpha, \beta$ are constants (so-called ``Rayleigh damping''). This 
  special form is often assumed for convenience, e.g. in Finite Element 
  packages, but there is usually no physical justification whatever for this 
  assumption. If the damping in a system is modelled by a physically sensible 
  matrix $C$, standard modal analysis will usually not work. We need a 
  different mathematical formalism to find what has become of the modes. 

  A useful approach, which is also widely used to analyse systems in control 
  theory, is to reformulate the governing equations as a set of $2N$ 
  first-order equations, rather than the $N$ second-order equations we have at 
  present. This change is simply an algebraic trick, with no physical 
  significance. Define a new vector 

  $$\mathbf{y} = \begin{bmatrix}\mathbf{q}\\ 
  \dot{\mathbf{q}}\end{bmatrix},~~~\dot{\mathbf{y}} = 
  \begin{bmatrix}\dot{\mathbf{q}}\\\ddot{\mathbf{q}}\end{bmatrix} . \tag{8}$$ 

  Then 

  $$\dot{\mathbf{y}} = \begin{bmatrix}0 \& I\\ -M^{-1}K \& 
  -M^{-1}C\end{bmatrix} \begin{bmatrix}\mathbf{q}\\ 
  \dot{\mathbf{q}}\end{bmatrix} = A \mathbf{y} \tag{9}$$ 

  where $0$ denotes the $N \times N$ matrix of zeros, $I$ denotes the $N \times 
  N$ identity matrix, and the final matrix $A$ is a $2N \times 2N$ matrix. 

  The top half of this set of equations simply says 
  $\dot{\mathbf{q}}=\dot{\mathbf{q}}$, while the bottom half says 

  $$\ddot{\mathbf{q}} = -M^{-1} K \mathbf{q} -- M^{-1} C \dot{\mathbf{q}} 
  \tag{10}$$ 

  which is a rearranged version of eq. (2). 

  We can now look for ``modal'' solutions of eq. (9): try 

  $$\mathbf{y} = \mathbf{u} e^{\lambda t} . \tag{11}$$ 

  Then we require 

  $$A \mathbf{u} = \lambda \mathbf{u} \tag{12}$$ 

  which is a standard matrix eigenvalue-eigenvector problem. It is difficult to 
  solve by hand for problems of any realistic size, but it is very easy to 
  compute answers. 

  Note that the matrix $A$ is not symmetric, so the eigenvalues $\lambda$ will 
  in general be complex. From eq. (11), the imaginary part of $\lambda$ will 
  give the frequency. The real part will be negative, equal to minus the decay 
  rate. Also, we no longer expect the useful result that eigenvectors are 
  orthogonal. There are, however, equivalent (but more complicated) relations 
  between the eigenvectors, which allow things like step response and impulse 
  response to be calculated in a similar way that we did earlier. We will not 
  go into the full details here, we will simply use the results: see for 
  example Chapter 8 of Newland [1]. 

  We can now apply the approach to the coupled-strings problem. We have two 
  strings of length $L$, with tension $T_j$, mass per unit length $m_j$ and 
  Q-factor $Q_j$, where $j=1,2$. At position $x=0$ both strings are rigidly 
  anchored, while at $x=L$ they are both attached to a mass $m$, which is 
  itself attached to a fixed base through a parallel combination of a spring of 
  stiffness $k$ and a dashpot of strength $c$. We will only consider the first 
  resonance of each string, considered in isolation. So suppose that the 
  displacement of string $j$ is given approximately by 

  $$w_j(x,t) = b_j(t) \sin \dfrac{\pi x}{L} + a(t)\dfrac{x}{L} \tag{13}$$ 

  where $a(t)$ is the displacement of the oscillator representing the body 
  mode. 

  We now calculate the potential and kinetic energies of the system, and deduce 
  the mass and stiffness matrices for the three degree-of-freedom system 
  parameterised by the vector $\mathbf{q} = [a~b_1~b_2]^t$. The potential 
  energy is given by 

  $$V = \frac{1}{2} k a^2 +\sum_j{\dfrac{1}{2}T_j \int_0^L{\left( 
  \dfrac{\partial w_j}{\partial x} \right)^2 dx}} $$ 

  $$= \frac{1}{2} k a^2 +\sum_j{\dfrac{1}{2}T_j \int_0^L{\left( \dfrac{b_j 
  \pi}{L} \cos \dfrac{\pi x}{L}\right)^2 dx}} $$ 

  $$= \frac{1}{2} k a^2 +\sum_j{\dfrac{\pi^2 T_j b_j^2}{4L}} \tag{14}$$ 

  so that the stiffness matrix is 

  $$K =\begin{bmatrix}k \& 0 \& 0\\ 0 \& \frac{\pi^2 T_1}{2L} \& 0 \\ 0 \& 0 \& 
  \frac{\pi^2 T_12}{2L}\end{bmatrix} . \tag{15}$$ 

  The kinetic energy is given in a similar way, by 

  $$T = \frac{1}{2} m \dot{a}^2 +\sum_j{\dfrac{1}{2}m_j \int_0^L{\dot{w}_j^2 
  dx}} $$ 

  $$= \frac{1}{2} m \dot{a}^2 +\sum_j{\dfrac{1}{2}m_j \int_0^L{\left[\dot{b}_j 
  \sin \dfrac{\pi x}{L} + \dfrac{\dot{a}x}{L}\right]^2 dx}} $$ 

  $$= \frac{1}{2} m \dot{a}^2 +\sum_j{\dfrac{1}{2}m_j \left\lbrace \dot{b}_j^2 
  \dfrac{L}{2} + \dfrac{\dot{a}^2}{L^2} \dfrac{L^3}{3} + \dfrac{2 \dot{a} 
  \dot{b}_j}{L} \int_0^L{x \sin \dfrac{\pi x}{L} dx}\right\rbrace }$$ 

  $$= \frac{1}{2} m \dot{a}^2 +\sum_j{\dfrac{1}{2}m_j \left\lbrace \dfrac{L}{2} 
  \dot{b}_j^2 + \dfrac{L}{3} \dot{a}^2 + \dfrac{2 \dot{a} \dot{b}_j}{L} 
  \dfrac{L^2}{\pi}\right\rbrace } \tag{16}$$ 

  after using integration by parts on the final term. It follows that the mass 
  matrix is 

  $$M =\begin{bmatrix}m+L(m_1+m_2)/3 \& m_1 L /\pi \& m_2 L/\pi\\ m_1 L /\pi \& 
  m_1 L /2 \& 0 \\ m_2 L /\pi \& 0 \& m_2 L /2\end{bmatrix} . \tag{17}$$ 

  Finally, we need the damping matrix. The rate of energy dissipation in the 
  body dashpot is $c \dot{a}^2$, so 

  $$C = \begin{bmatrix}c \& 0 \& 0 \\ 0 \& c_1 \& 0 \\ 0 \& 0 \& c_2 
  \end{bmatrix} \tag{18}$$ 

  where the equivalent dashpot strengths of the two string modes are given by 

  $$c_j = \dfrac{\pi \sqrt{T_j m_j}}{2 Q_j} . \tag{19}$$ 

  We can then assemble the $6 \times 6$ matrix $A$ from eq. (9), and compute 
  its eigenvalues and eigenvectors. There, will of course, be 6 of them: but 
  surely we are only expecting to see 3 modes for this system? The answer to 
  that puzzle is simple. Each mode appears twice, related to each other by 
  being complex conjugates. The matrix $A$ is real, so we can deduce from eq. 
  (12) that if $\mathbf{u}$ is an eigenvector with eigenvalue $\lambda$ then 
  $\mathbf{u}^*$ must also be an eigenvector with eigenvalue $\lambda^*$. 

  Finally, we can calculate the ``pluck response'' using the formula from 
  Newland [1]. For the una corda case where only one string is initially 
  excited, we apply a step function of generalised force to $b_1$ and calculate 
  the transient response $a(t)$. To obtain the case where both strings are 
  excited equally, we first calculate the corresponding response when a step 
  function of generalised force is applied to $b_2$, then add the result to the 
  previous case. We are still dealing with a linear system, so we can use 
  superposition to obtain the combined response. 

  \sectionreferences{}[1] David E. Newland; ``Mechanical Vibration Analysis and 
  Computation'', Longman Scientific and Technical, Harlow (1989). 