  A tuned-mass damper, or TMD, is a device often used to add damping to a 
  structural resonance that is causing problems. There are many types of 
  application. In civil engineering TMDs are used on lightweight bridges and on 
  tall buildings, especially ones in earthquake zones. The discussion in 
  section 9.4 described the application in archery bows. And then there is the 
  application we are particularly interested in here, as a wolf note 
  ``eliminator'' for cellos and other bowed-string instruments. 

  For all these applications, we can get a good idea of how a TMD works and how 
  it might best be designed from a very simple model. We can represent the 
  underlying structural resonance by a mass-spring-dashpot oscillator with 
  effective mass $m_1$, stiffness $k_1$ and dashpot strength $c_1$. Its loss 
  factor (see section 2.2.7) is then $\eta_1=c_1/\sqrt{k_1 m_1}$. 

  The effective mass $m_1$ is not a fixed number for a given mode of the cello 
  body (or whatever the underlying structure may be). Instead, it depends on 
  the mode shape at the position we choose to fit our TMD. From section 2.2.5 
  we know that if the normalised modal amplitude at this point is $u_n$, then 
  $m_1=1/u_n^2$. So for a location close to a node line of the mode $u_n$, the 
  effective mass would become very large. But we would not want to fit a TMD 
  close to a node line: it will work best when fitted at a position where the 
  relevant mode has significant amplitude. The actual value of the effective 
  mass then depends on the mass of the structure, via the normalisation 
  condition (equation (12) of section 2.2.5). For the application to a cello 
  wolf note, as a guide we would expect this to be somewhere between a quarter 
  and a half of the mass of the top plate. 

  The TMD then consists of a second mass-spring-dashpot oscillator, sitting on 
  top of the first one, as shown schematically in Fig.\ 1. The TMD oscillator 
  has effective mass $m_2$, stiffness $k_2$ and dashpot strength $c_2$, and its 
  loss factor if the mass $m_1$ was held fixed would be $\eta_2=c_2/\sqrt{k_2 
  m_2}$. 

  \fig{figs/fig-44b70943.png}{\caption{Figure 1. Sketch of a 
  two-degree-of-freedom system representing a structural mode and a TMD.}} 

  To see how the coupled system behaves, we can simply write down Newton's law 
  for the two masses, to deduce 

  $$m_1 \ddot{x_1} = -- k_1 x_1 -- k_2 (x_1 -x_2) \tag{1}$$ 

  and 

  $$m_2 \ddot{x_2} = -- k_2 (x_2 -x_1) \tag{2}$$ 

  where $x_1$ and $x_2$ are the displacements of the two masses away from their 
  equilibrium positions. We can deduce the mass, stiffness and damping 
  matrices: 

  $$M=\begin{bmatrix}m_1 \& 0\\ 0 \& 
  m_2\end{bmatrix},~~K=\begin{bmatrix}k_1+k_2 \& -k_2\\ -k_2 \& 
  k_2\end{bmatrix},~~C=\begin{bmatrix}c_1 \& 0\\ 0 \& c_2\end{bmatrix} . 
  \tag{3}$$ 

  The case of interest arises when the TMD is tuned to the same natural 
  frequency as the original resonance, so we can choose to set 

  $$m_2=\epsilon m_1,~~~k_2=\epsilon k_1 \tag{4}$$ 

  where the mass ratio $\epsilon$ is expected to be a small number. The 
  absolute values of $m_1$ and $k_1$ will not influence the behaviour except to 
  set the specific value of the resonance frequency, and act as an overall 
  scale factor in any frequency response functions. For maximum simplicity, we 
  can treat the non-dimensionalised case with $m_1=1,~~k_1=1$ so that when we 
  calculate the coupled natural frequencies they will be normalised by the 
  original resonance frequency. 

  The mass and stiffness matrices then take the form 

  $$M=\begin{bmatrix}1 \& 0\\ 0 \& 
  \epsilon\end{bmatrix},~~K=\begin{bmatrix}1+\epsilon \& -\epsilon\\ -\epsilon 
  \& \epsilon\end{bmatrix} . \tag{5}$$ 

  We can now calculate the undamped natural frequencies and mode shapes in the 
  usual way (see equations (8) and (9) of section 2.2.5). The (normalised) 
  natural frequencies satisfy 

  $$\begin{vmatrix}1-\epsilon-\omega^2 \& -\epsilon \\ -\epsilon \& 
  \epsilon-\epsilon \omega^2\end{vmatrix} =0 \tag{6}$$ 

  which simplifies to the quadratic equation 

  $$\omega^4 -- (2-\epsilon) \omega^2 + (1-2\epsilon) = 0 \tag{7}$$ 

  so that 

  $$\omega^2 = 1 -\dfrac{\epsilon}{2} \pm \sqrt{\epsilon -- \epsilon^2 /4} 
  \approx 1 \pm \sqrt{\epsilon} \tag{8}$$ 

  where the final approximation uses the fact that $\epsilon$ is small. 

  Now we can deduce the ratio of modal amplitudes from equation (2): 

  $$\dfrac{x_2}{x_1} = \dfrac{1}{1-\omega^2} \approx \pm 
  \dfrac{1}{\sqrt{\epsilon}} . \tag{9}$$ 

  A physical interpretation of equation (9) is that the two masses have 
  approximately equal kinetic energy in both modes. 

  Equation (8) tells us that there are two natural frequencies, one on either 
  side of the original resonance. The bigger the mass ratio $\epsilon$, the 
  further apart these two frequencies are split. Equation (9) tells us the 
  corresponding mode shapes. Both of them have large motion of the TMD mass, 
  because of the factor $1/\sqrt{\epsilon}$. The mode at lower frequency has 
  the two masses vibrating in phase, the one at higher frequency has them in 
  antiphase (just as we would have expected from the requirement of 
  orthogonality of these two modes). Figure 2 shows an animation of a typical 
  case of these two modes, for the value $\epsilon = 0.03$ so that the TMD mass 
  is 3\% of the original resonator mass. 

\moobeginvid\begin{tabular}{ccc} \vidframe{ 0.30 }{ vids/vid-9b574e0f-00.png }&\vidframe{ 0.30 }{ vids/vid-9b574e0f-01.png }&\vidframe{ 0.30 }{ vids/vid-9b574e0f-02.png } \end{tabular}\caption{Figure 2. Animation of the two modes of a typical TMD system, without damping. For convenience of the animation, they are shown here as having the same frequency, but in reality the two frequencies will be slightly different.}\mooendvideo

  Finally, we want to include the effects of damping, and see what influence 
  the TMD has on the frequency response for driving on the mass $m_1$. We can 
  write equations (1) and (2) in the frequency domain, for the case of harmonic 
  response driven by forces $F_1 e^{i \omega t}$ on mass $m_1$ and $F_2 e^{i 
  \omega t}$ on mass $m_2$. If $x_1 =X_1 e^{i \omega t}$ and $x_2 =X_2 e^{i 
  \omega t}$, then 

  $$[-\omega^2 M + i \omega C +K] \textbf{X} = \textbf{F} \tag{10}$$ 

  where $\textbf{X}$ is the vector $[X_1~~X_2]^t$ and $\textbf{F}$ is the 
  vector $[F_1~~F_2]^t$. The term in square brackets $[...]$ in equation (10) 
  is called the dynamic stiffness matrix $D$. The simplest way to compute the 
  frequency response we want is to invert it, because 

  $$\textbf{X} = D^{-1} \textbf{F} \tag{11}$$ 

  so that the (1,1) element of $D^{-1}$ is the frequency response we want: the 
  displacement response of mass $m_1$ to a force $F_1$. 

  Figure 3 shows some examples. The blue curve shows the original response of 
  the resonance without the added TMD, assuming a value $\eta_1 = 0.01$. The 
  red curves show the result of adding two alternative TMDs, both with the 
  value $\eta_2 = 0.15$. The dashed curve has $\epsilon = 0.01$, while the 
  solid red curve has $\epsilon = 0.03$ (the case shown in Fig.\ 2). In the 
  solid curve, the two separate peaks are very clear. In the dashed curve, with 
  weaker coupling, the frequencies are close enough together that the peaks 
  overlap to give a single, slightly distorted, peak. 

  \fig{figs/fig-dd57a23e.png}{\caption{Figure 3. Driving-point frequency 
  responses on the mass $m\_1$ of the system shown in Fig. 1. The blue curve 
  shows the original system without the TMD. The two red curves show the effect 
  of TMDs with two different mass ratios $\epsilon$.}} 

  This simple analysis gives a good impression of what you would need to 
  consider when designing a practical TMD. There is an obvious advantage to 
  making $\epsilon$ as small as possible. On paper, you can achieve a good 
  outcome with any value of $\epsilon$, however small. But there is a snag: the 
  smaller $\epsilon$ becomes, the larger the amplitude of vibration required of 
  the TMD mass. There are always practical limitations on this amplitude, for 
  two different reasons. One relates to constraints on available space: there 
  needs to be room for this large amplitude of vibration without hitting 
  something. But there is another reason. We have done this calculation 
  assuming linear behaviour of the springs and dashpots. But if the TMD moves 
  at very large amplitude, nonlinear effects are likely to become apparent. 
  Most springs, of whatever kind, have some limit on how far they can be 
  stretched or compressed. The combined effect of these two things is to impose 
  a minimum practical limit on $\epsilon$. 

  The examples shown in Fig.\ 3 use deliberately extreme values, to illustrate 
  the behaviour clearly. The height of the peak is reduced by some 20 dB in 
  both cases. This is a far bigger effect than you would need to suppress a 
  wolf note, which is fortunate because it would be hard to achieve such a 
  large effect with something you could realistically fit to a cello. In 
  addition to the two constraints just described, there is another issue 
  concerning the assumed damping. We have assumed a rather low value of the 
  cello loss factor $\eta_1$, and a very high value of the TMD loss factor 
  $\eta_2$. In practice these two loss factors would not be so different, and 
  the effect of the TMD on the peak height would be reduced. 

  Finally, we should note that the smaller the value of $\epsilon$, the more 
  careful we need to be to tune the TMD frequency to match the wolf frequency. 
  When the mass ratio is very tiny, the strength of coupling between the two 
  oscillators is very weak. As the tuning is varied, the two coupled 
  frequencies will go through a ``veering'' pattern (like Fig.\ 3 of section 
  7.3). In order for the TMD to work, the tuning has to be accurate enough to 
  achieve a state well within the veering region, and this region will be 
  narrow when the coupling is weak. 