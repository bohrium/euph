

  There are two aspects of acoustics that will be of direct interest to us to 
  understand the behaviour of musical instruments. Wind instruments of all 
  kinds rely on acoustical resonators, and acoustic resonances also play a role 
  in understanding the behaviour of stringed instrument bodies and of room 
  acoustics. This section will give an initial overview of all these examples. 
  Then in the next section we will address the other important aspect of 
  acoustics : the radiation of sound by a vibrating structure such as a violin 
  body. 

  When we looked at mechanical vibration in Chapter 2, we started with the 
  mass-spring system, with a single degree of freedom, then went on to look at 
  multi-modal systems like stretched strings and bending beams. We follow a 
  similar sequence here. There is an acoustical equivalent of the mass-spring 
  oscillator, called a Helmholtz resonator. 

  \textbf{A: The Helmholtz resonator} 

  This simple resonating system is responsible for the popping noise when you 
  pull out a cork or flick a thumb out of the top of a bottle. All 
  “bottle-like” vessels have a low-frequency resonance in which a “plug” of air 
  in the neck behaves like an invisible piston, and can oscillate on a “spring” 
  resulting from compression of the air in the enclosed volume inside the 
  bottle. This ``spring'' is the force you feel if you try to operate a bicycle 
  pump while you have your thumb over the end. Figure 1 shows a sketch, and the 
  next link gives a derivation of the resulting formula for the resonant 
  frequency. 

  Before the days of electronics, a set of tuned Helmholtz resonators could be 
  used as a “spectrum analyser”: an example is shown in Fig.\ 2. Helmholtz 
  resonance also accounts for the effect that children often encounter when 
  visiting the seaside: they hold a sea-shell to one ear, and are told ``You 
  can hear the sound of the sea inside the shell''. Both these examples work in 
  the same way: the resonator amplifies ambient sound in a frequency range 
  close to the resonant frequency, as demonstrated explicitly in the previous 
  link. If you place an ear close enough, you hear this amplification effect. 
  The resonators shown in Fig.\ 2 are provided with a nipple that is placed in 
  the ear canal for this purpose. 

  The most direct musical application of a Helmholtz resonator is an ancient 
  instrument called the ocarina. Figure 3 shows an example. A whistle-like 
  mouthpiece leads to a chamber with a number of finger holes. Unlike other 
  wind instruments with finger holes, it doesn't matter which particular holes 
  you cover. The note you obtain from the ocarina depends only on the total 
  area of open holes: it is a Helmholtz resonator and the frequency is 
  determined by the chamber volume and the combined area and ``neck length'' of 
  the open holes, according to the formula given in the previous link. This 
  means that ocarinas can be made in many different shapes, but all work in the 
  same way and all sound rather similar. 

  A more mainstream application of Helmholtz resonance in musical instruments 
  is relevant to many stringed instruments. For definiteness, we will use the 
  guitar as an example. The body of an acoustic guitar like the one seen in 
  Fig.\ 4 is a thin-walled box, usually made of wood, and there is an opening 
  in the top plate, usually called the soundhole. It is sometimes thought that 
  this hole is there ``to let the sound out'', but that is quite misleading. 
  Most of the sound is created by vibration of the wooden plates, but the hole 
  serves to make a significant enhancement to the radiation of sound at lower 
  frequencies in the instrument's range by adding a Helmholtz-like resonance. 
  ``Helmholtz-like'', because the box is not a rigid enclosure. It is very 
  important to allow for the flexibility of the plates forming the walls of the 
  box. 

  We can get a good impression of how this works using a simple model which 
  takes account of only one vibration mode of the box, the lowest mode of the 
  top plate. We will come to a more complete description of how the guitar 
  works in section 5.3, including some images of mode shapes, but for now we 
  simply need to know that this lowest mode behaves just as you would expect. 
  The main area of the plate, carrying the bridge in the middle, bulges in and 
  out roughly in the manner sketched in Fig.\ 5. 

  This plate vibration will interact with the internal air pressure and the 
  Helmholtz resonance. The vibrating plate will cause changes in internal air 
  pressure, and in turn that pressure exerts a force on the plate. But the same 
  is still true of the invisible ``Helmholtz piston'' in the soundhole. The 
  combined system can be represented in a different form, shown in Fig.\ 6. The 
  mode of the top plate is represented by a mass and a spring, and also a 
  piston connecting it to the internal air. The enclosure is otherwise rigid, 
  with a ``neck'' in which the Helmholtz piston moves. It is no coincidence 
  that this figure resembles a loudspeaker in an enclosure: the same modelling 
  is used in the design of ducted (or bass reflex) loudspeakers. 

  The detailed analysis of this system is given in the next link, but we can 
  learn the most important things about the result without mathematical detail 
  by doing one more stage of abstraction. The system in Fig.\ 6 can be 
  represented by an equivalent mass-spring system, shown in Fig.\ 7. The 
  left-hand mass and spring represent the mode of the plate (or of the 
  loudspeaker cone). The right-hand mass represents the Helmholtz piston. The 
  spring connecting the two masses represents the springiness of the internal 
  air: the net volume, and hence the internal pressure, depends on the relative 
  motion of the two pistons. For example, if the Helmholtz piston moved inwards 
  at the same time as the plate moved outwards, with the correct ratio of 
  amplitudes, the volume and internal pressure would not change. 

  The two-mass system of Fig.\ 7 will have two vibration modes, as explored in 
  some generality in Section 2.2.5. The lower-frequency one of these modes will 
  have the two masses moving in the same phase, while the higher frequency will 
  have them moving in opposite phases. Both modes will involve some stretching 
  and compression of the right-hand spring, representing changes of internal 
  air pressure. We will see in section 4.3 that changes in internal pressure, 
  associate with changes in net volume, give a good indication of the strength 
  of external sound radiation by each mode. 

  Figure 8 shows a schematic representation of the two modes, in a way that 
  relates directly to Fig.\ 6. It is based on parameter values appropriate to a 
  typical guitar body (see previous link for details). In order to illustrate 
  the relative volume displacements of the two pistons, which is the important 
  factor determining the sound radiation, the two pistons are shown with the 
  same sizes. In the real guitar, the Helmholtz piston has about 1/10 the area 
  of the effective plate piston , so that its relative displacement is in fact 
  10 times bigger to compensate. 

  Both modes involve some motion of the mass representing the top plate motion. 
  This is important: the vibrating string is attached to the plate, and the 
  excitation of each mode depends on plate motion. For an ideal Helmholtz 
  resonator with rigid cavity walls, the resonance could not be excited by a 
  vibrating string. 

  So in summary the combination of a flexible top plate with a cavity 
  containing a soundhole is that the guitar has two resonances which are good 
  radiators of sound, and which can both be driven efficiently by the vibrating 
  strings. The details are, to a large extent, under the control of the guitar 
  maker. The frequencies of the two resonances can be controlled: the thickness 
  and bracing details of the top plate determine where the lowest plate 
  resonance would fall in the absence of interaction with the air, while the 
  volume of the box and the size of the soundhole determine where the Helmholtz 
  resonance would fall in the absence of plate motion. A trick sometimes used 
  by guitar makers when they want to reduce the Helmholtz frequency is to add a 
  ``tornavoz'': a cylindrical collar fitted inside the box, extending the 
  soundhole into a longer neck and thus increasing the mass of the ``Helmholtz 
  piston''. An example is shown in Fig.\ 9. 

  \textbf{B: Pipe resonances} 

  Most wind instruments rely on resonances in pipes of one kind or another. 
  They have a variety of internal shapes, or bore profiles. We look first at 
  two simple examples, then take a preliminary look at how to deal with more 
  complicated profiles such as those found in brass instruments. 

  The first example makes use of the simplest sound field that we met in 
  section 4.1, the plane wave. As Fig.\ 10 indicates, a straight-walled pipe 
  can be superimposed on a plane wave, aligned with the direction of 
  propagation, and since the particle motion is always parallel to the pipe 
  walls, the wave inside the pipe can behave exactly the same way as it did in 
  empty space. So plane waves can propagate along a straight pipe at the speed 
  of sound: a practical application of this is the speaking tube, still 
  sometimes used as a way to communicate which doesn't rely on electricity. 

  Now we want to think about pipes of finite length, to understand their mode 
  shapes and resonant frequencies. First, we need to think about possible 
  boundary conditions at the end of a pipe. One possibility is an open end: the 
  pipe is simply cut off. The sound wave inside the pipe is then exposed to the 
  outside world. Some sound will escape from the end and radiate away, but to a 
  good first approximation we can say that at the end of the pipe (or, at 
  least, somewhere near the end of the pipe: see the discussion of end 
  corrections in section 4.2.1) the pressure simply becomes the steady 
  atmospheric pressure. In terms of the acoustic pressure, that means the open 
  end of the pipe must be a node. 

  Another simple boundary condition would occur if one end of the pipe was 
  blocked. At a closed end like this, the thing we can say straight away is 
  that the oscillating air particles cannot pass through the blocked end: we 
  must have a nodal point of particle displacement. We can deduce what this 
  means for pressure by referring back to eq. (2) of section 4.1.1: pressure is 
  proportional to the spatial derivative of displacement. Once we assume 
  sinusoidal time dependence in order to find modes of the pipe, then the 
  spatial variation must also be sinusoidal. It follows that if the pressure 
  variation is like $\sin kx$, the displacement must be like $\cos kx$, and 
  vice versa. So a nodal point of displacement means an antinode of pressure. 

  We now have enough information to find mode shapes and natural frequencies. 
  For a pipe open at both ends, the pressure must be sinusoidal with nodes at 
  both ends. It follows that the mode shapes are exactly the same as the ones 
  we found for a vibrating string in section 3.1.1. The first few are 
  illustrated in Fig.\ 11. The natural frequencies also obey the same formula 
  as for the string: the $n$th frequency is 

  $$f_n=\frac{nc}{2L} \tag{1}$$ 

  for a pipe of length $L$, where $c$ is the speed of sound. 

  For a pipe that is open at one end but closed at the other, the corresponding 
  mode shapes have to be as shown in Fig.\ 12. The lowest mode has a 
  quarter-wavelength trapped in the length, rather than a half-wavelength as 
  for the open-open pipe. The next mode has 3 quarter-wavelengths, the next 5 
  and so on. The corresponding frequency for the $n$th mode is 

  $$ f_n=\frac{(2n-1)c}{4L} . \tag{2}$$ 

  It follows that the pattern of natural frequencies is different from the 
  open-open case. The frequency ratios are 1:3:5:7, rather than 1:2:3:4. In 
  other words, they are alternate terms of the harmonic series, rather then 
  every term. Furthermore, the fundamental frequency of the closed-open pipe is 
  an octave lower than that of the open-open pipe, for a tube of the same 
  length. This explains a familiar effect in instruments. To a first 
  approximation, a flute is a straight open-open tube, and a clarinet is a 
  straight closed-open tube. The two instruments have roughly the same length, 
  but the clarinet plays an octave lower for a given fingering. It overblows at 
  the twelfth (a frequency ratio of 3), whereas the flute overblows at the 
  octave. 

  Some instruments, like the oboe and the saxophone, are better approximated as 
  conical tubes rather than straight tubes. The ideal straight-sided conical 
  tube is another case that we can understand easily, based on something we 
  already know: this time, it is a spherical wave field such as that produced 
  by a pulsating sphere as in section 4.1.2. As Fig.\ 13 indicates, a 
  straight-walled conical pipe can be superimposed on such a spreading 
  spherical wave, aligned with the direction of propagation and with its point 
  placed at the origin. As happened for the plane wave and the straight pipe, 
  the particle motion is always parallel to the pipe walls, so the wave inside 
  the pipe can behave exactly the same way as it did in empty space. 

  We need to think about boundary conditions again. The theory for the 
  spherical wave showed that the combination $xp$ satisfies the one-dimensional 
  wave equation, rather than the pressure $p$ alone as in the case of the 
  straight pipe. Here, $x$ is the distance along the cone from the point. At an 
  open end of the cone, we will have a node of pressure as before, and hence a 
  node of $xp$. But as we approach the pointed end of the cone, $x \rightarrow 
  0$, and so $xp(x) \rightarrow 0$. The result is that the combination $xp$ has 
  sinusoidal variation, with zeros at both ends --- exactly the same conditions 
  as the open-open straight pipe. The mode shapes for pressure are thus 

  $$p_n(x)=\dfrac{\sin n \pi x/L}{x} \tag{3}$$ 

  with corresponding natural frequencies that are exactly the same as the 
  open-open straight pipe: 

  $$f_n=\frac{nc}{2L} . \tag{4}$$ 

  The first few of these pressure mode shapes are plotted in Fig.\ 14. At first 
  glance they look similar to the closed-open modes of Fig.\ 12, but in fact 
  they are significantly different: the wavelength, visible by the positions of 
  the nodal points, matches the open-open case, not the closed-open case of the 
  straight tube. 

  We now turn to brass instruments, and we immediately encounter an apparent 
  paradox. Something like a trumpet is clearly an open-closed tube: the 
  mouthpiece end is closed by the player's lips. The tube does not taper down 
  in a conical way at the mouthpiece end: most brass instruments have a long 
  section of straight tube before the flaring bell, as is obvious if you think 
  about the slide of a trombone. So we might expect the instrument to have 
  resonances at alternate harmonics (or, at least, approximate harmonics). But, 
  as is familiar from the kinds of tune that can be played on a bugle or a 
  post-horn, the instruments in fact seem able to play the complete harmonic 
  series, not just the odd terms. 

  The resolution of this question is that the bore profile of a typical brass 
  instrument is not well approximated by any of the simple shapes we have 
  looked at up to now. We need to explore the underlying theory of horns with 
  varying cross-section to understand the ingenious trick that is used by 
  makers of brass instruments. If the cross-sectional area of the bore varies 
  slowly and smoothly with distance, then to a good first approximation the 
  pressure obeys a modified version of the wave equation called the Webster 
  horn equation. The details are given in the next link. 

  This equation doesn't have easy mathematical solutions for realistic bore 
  profiles, so to see roughly what happens we will resort to the computer. 
  Figure 15 shows the first few pressure modes, computed from the Webster 
  equation for a bore profile which has the right kind of features for a 
  realistic brass instrument. It has a straight tube with a closed end, leading 
  into a section which flares: gradually at first, then more abruptly as the 
  bell is approached. 

  The sequence of mode shapes is recognisably related to the closed-open modes 
  for a straight tube, seen in Fig.\ 12, but the shapes have been ``squashed 
  in'' towards the mouthpiece end. Most conspicuously, the fundamental mode is 
  mainly confined to the left-hand half of the tube, dying away to low levels 
  long before the bell is reached. Something similar happens for the second 
  mode, but it reaches closer to the bell before it fades away. The sequence 
  continues with the higher modes: each successive mode looks a bit more like 
  the corresponding shape in Fig.\ 12. 

  The previous link explains what is happening. It involves behaviour similar 
  to something we have met back in section 3.5, in connection with the 
  ``musical saw''. For a given frequency, and hence a given wavelength, there 
  is a point in the flaring bore beyond which it is no longer possible for a 
  travelling wave to propagate. It switches over to an evanescent wave, with 
  exponential decay. Just as happened in the musical saw, most of the energy in 
  the travelling wave is reflected back down the tube from this critical point, 
  which can be calculated easily from the bore profile. For each mode, this 
  point is marked in Fig.\ 15 by a vertical black line. It can be seen that 
  these lines match quite well to the point where the sinusoidal behaviour of 
  mode amplitude switches over to a decaying shape. 

  The result is that the apparent length of the tube is shorter than the real 
  length: by a large amount for the fundamental mode, and by a decreasing 
  amount for the successive higher modes. This naturally changes the natural 
  frequencies: a shorter tube will always give a higher frequency. The pattern 
  that emerges is most clear if we scale the natural frequencies by taking the 
  second mode as a reference, and calling that frequency ``2''. In these terms, 
  the frequencies of the 5 modes seen in Fig.\ 15 are 0.75, 2, 2.97, 4.07 and 
  5.07. Modes 2, 3, 4 and 5 now have frequencies reasonably close to the 
  harmonic relations 2:3:4:5. The fundamental mode, though, does not fit into 
  this approximate harmonic series. This example, despite being based on a very 
  crude model, gives a good idea of what happens in real brass instruments: at 
  some time in the past, instrument makers have hit on a type of bore profile 
  that allows the instrument to have resonances which fall in a good 
  approximation to the complete harmonic series, apart from the fundamental 
  frequency which is always too low. 

  \textbf{C: Acoustic cavities and room acoustics} 

  There is a final class of acoustical resonators that deserves a brief 
  discussion: resonances inside cavities, which could range from a concert hall 
  to the inside of a violin body. The simplest example is a rectangular space 
  with hard walls. For this case we already know enough to deduce the mode 
  shapes and natural frequencies. Suppose we have a box with side lengths $A 
  \times B \times C$ along the coordinate directions $x,y,z$. We could start by 
  looking for modes that only depend on $x$: these would involve plane waves in 
  the $x$ direction, and the cavity would simply behave like a tube which is 
  closed at both ends. There has to be a pressure antinode at both ends, so the 
  modes are similar to the open-open modes of Fig.\ 11, with the same natural 
  frequencies, but they involve cosines rather than sines. The first few shapes 
  are shown in Fig.\ 16. 

  But obviously we could get the same set of shapes from plane waves in the $y$ 
  direction, or in the $z$ direction. More than that, we can find a mode that 
  has any combination of these $x$, $y$ and $z$ shapes simultaneously: the most 
  general pressure mode takes the form 

  $$p_{qrs}=\cos \frac{q \pi x}{A} \cos \frac{r \pi y}{B} \cos \frac{s \pi 
  z}{C} \tag{5}$$ 

  where each of $q$, $r$ and $s$ can take any integer value 0,1,2,3,... Notice 
  the inclusion of 0 here: this option is necessary to allow plane-wave modes. 
  For example, the case we just described has $r=s=0$ for a plane wave in the 
  $x$ direction. The modes plotted in Fig.\ 16 have $q=1,2,3,4,5$. 

  The frequencies corresponding to the modes in eq. (5) can be deduced 
  immediately from the wave equation: 

  $$f_{qrs}=\frac{c}{2 \pi} \sqrt{\left[ \left( \dfrac{q \pi}{A} \right)^2 + 
  \left( \dfrac{r \pi}{B} \right)^2 + \left( \dfrac{s \pi}{C} \right)^2 \right] 
  } . \tag{6}$$ 

  To see the most important thing that this equation tells us, it is useful to 
  compute an example. Consider a rectangular space $5 \times 4 \times 3$ m in 
  size: a typical domestic room. The fundamental frequency is 34 Hz: this is a 
  plane standing wave in the 5 m direction, exactly like the top plot in Fig.\ 
  16. But the important behaviour becomes clear if we calculate a lot of the 
  natural frequencies of this room and plot a histogram of them. Choosing 100 
  Hz bins, we get the result seen in Fig.\ 17. The histogram shows that the 
  room has an enormous of modes: the number per 100 Hz band grows very rapidly, 
  and by 5 kHz there are about 50,000 modes within each 100 Hz band! This modal 
  density function grows in proportion to the square of frequency, and this 
  trend applies to any acoustic volume: the next link explains why. 

  All the modes in our room will have some energy dissipation, and hence every 
  resonant peak will have a finite half-power bandwidth determined by the 
  damping level (see section 2.2.7). The high modal density seen in Fig.\ 17 
  then tells us that, except at the very lowest frequencies, the modal overlap 
  factor will be large: this is defined as the ratio of the half-power 
  bandwidth to the typical spacing between adjacent modes. This has profound 
  implications for the acoustical behaviour of the room. The response at any 
  given frequency will involve a very large number of modes. The conclusion is 
  that we probably won't learn very much if we try to understand the acoustics 
  of our room by looking for individual modes across the audible frequency 
  range: room acoustics is a statistical science. We will illustrate some of 
  the behaviour with measurements involving room acoustics, in section 10.4 
  (see subsection D). 

