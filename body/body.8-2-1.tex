  A familiar example of an oscillator showing weakening-spring behaviour is a 
  pendulum, like the sketch in Fig.\ 1. A mass $m$ is suspended from a rigid, 
  massless rod of length $L$. The rod is pivoted at the other end, with no 
  friction. 

  \fig{figs/fig-f607a5ff.png}{Figure 1. Sketch of a simple pendulum} 

  The only external force acting on the mass is gravity, so taking moments 
  about the pivot we obtain 

  $$m L^2 \ddot{\theta} = -mgL \sin \theta .$$ 

  We can rewrite this to look like the equation of an equivalent mass-spring 
  oscillator: 

  $$m \ddot{\theta} + \dfrac{mg}{L} \sin \theta =0 ,$$ 

  so that the equivalent ``nonlinear spring'' has the force-``displacement'' 
  relation $(mg/L) \sin \theta$. 

  This spring characteristic is plotted in Fig.\ 2, in a form that can be 
  compared with Fig.\ 2 of Section 8.2 for hardening and softening springs. The 
  plot only shows values $-\pi \le \theta \le \pi$ because we are interested in 
  cases where the pendulum oscillates backwards and forwards, rather than 
  rotating all the way round. We will be interested in the rotating case later, 
  though: we return to it in Section 8.3. 

  \fig{figs/fig-dfd5d512.png}{Figure 2. Equivalent spring behaviour of the 
  pendulum} 

  It is clear that the pendulum shows softening behaviour. We can link the 
  behaviour explicitly to the earlier plot and to Duffing's equation (see 
  section 8.2.2) by using the power-series expansion of $\sin \theta$ for small 
  angles: 

  $$\sin \theta \approx \theta -- \dfrac{\theta^3}{3!} .$$ 

  Our equivalent spring follows Hooke's law for very small displacements 
  $\theta$, and then the first departure from linear behaviour takes the form 
  of a cubic term with a negative coefficient. For larger angles $\theta$, 
  Fig.\ 2 shows that the spring force reaches its maximum value at $\theta = 
  \pi/2$ when the pendulum is horizontal, and then it falls all the way to zero 
  as the angle approaches $\pi$, with the pendulum vertically upwards. 