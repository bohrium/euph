

  We turn to the final family of wind instruments, the air-jet instruments like 
  the flute, recorder and flue organ pipe. Looking back over previous sections 
  of this chapter, the logic behind the organisation should now become clear: 
  we started with the best-understood instruments, and we have progressively 
  moved to the ones where physical understanding is more challenging, and 
  therefore more rudimentary. 

  For reed instruments like the clarinet, we were able to construct a fairly 
  simple physical model with quantitative experimental support for all aspects. 
  The pipe acoustics was anchored in measured input admittance which could be 
  interpreted via simple theory, while the nonlinear characteristics of the 
  reed mouthpiece were based on uncontroversial theory with some direct 
  measurements in support. The result was a simulation model with some claims 
  to realism. This could be used to explore questions about how the player 
  might control the instrument to maximise the range of musical options, and 
  also how the instrument maker might help by “playability” improvements. 

  Moving on to the brass instruments, our understanding of tube acoustics was 
  still supported by direct measurements of input impedance, but the model for 
  the lip dynamics was less convincing. We started with essentially the same 
  model as for the reeds, except for a reversal of sign to reflect the 
  opening-reed behaviour of lip vibration as opposed to the closing-reed 
  character of the reed woodwinds. This super-simple model, with just a single 
  degree of freedom, can only be expected to give a rather crude representation 
  of the vibration of the squashy flesh of lips. We then explored a rather ad 
  hoc extension of the model in order to incorporate at least some aspects of 
  the vibration of real lips. The resulting simulation models gave 
  qualitatively plausible results, correctly reflecting many aspects of the 
  behaviour of this family of instruments. 

  When we moved to the free reed instruments, things got more complicated. This 
  time, the behaviour of the reeds themselves was reasonably uncontroversial, 
  but it was far from clear what additional physics needed to be included in 
  order to model the excitation mechanism with any claim to realism. For some 
  problems, with strong acoustical feedback from a well-characterised system, 
  it was easy to complete a model in the same style as the original clarinet 
  model and obtain quite convincing agreement with experiments. 

  But in other cases we met snags of two different kinds. First, no impedance 
  measurements appear to have been made on instruments like the concertina, let 
  alone on the harmonica for which the player’s vocal tract is a significant 
  part of the system. Second, we saw strong hints that acoustical feedback is 
  not always the predominant excitation mechanism, suggesting that something 
  more complicated is needed. Initial modelling efforts have been made, based 
  on idealised analysis of the fluid flow in the immediate vicinity of a 
  vibrating reed, but the story is by no means complete. 

  Now we turn to the air-jet instruments, and we are squarely faced with 
  problems involving non-trivial fluid dynamics. These instruments have no 
  moving parts analogous to reeds or brass-player’s lips. The behaviour of the 
  instrument is determined by the interaction of the internal acoustics of the 
  instrument tube with air-flow from a mouthpiece slot or from a flute-player’s 
  lips. The tube acoustics can be characterised by measured input impedance (or 
  its inverse, input admittance: as we saw back in section 11.1, an air-jet 
  instrument is expected to play notes determined by peaks of admittance, or 
  antiresonances of impedance). But the fluid dynamics governing how an air jet 
  interacts with the solid part of a mouthpiece and with the internal 
  acoustical field is far more complicated than anything we have grappled with 
  in previous sections. 

  \textbf{A. Introducing the fluid dynamics of air-jet instruments} 

  We can get a first idea of how flute-like instruments work, and 
  simultaneously gain some insight into the complications, by looking at 
  examples of flow visualisation. Figure 1 shows a kind of rectangular organ 
  pipe, or one-note recorder. It has a transparent section in the pipe wall to 
  allow schlieren imaging (look back at section 10.6 for an account of how the 
  schlieren system works). Air is injected from the bottom, passes through a 
  slot, and the emerging jet then interacts with a sharp edge. It is this 
  interaction between the air jet and the edge (known as the ``labium'' in the 
  terminology of recorders) which is the key to understanding how such 
  instruments work. For the purposes of the flow visualisation, some carbon 
  dioxide is also injected to give a density contrast with the air and allow 
  the imaging process to work. 

  \fig{figs/fig-01c101ac.png}{Figure 1. The experimental setup used to obtain 
  the images shown in Fig. 2. The ``recorder'' is vertical, with the flow 
  entering from the bottom. One of the lenses of the schlieren system can be 
  seen on the left. Image copyright Avraham Hirschberg, reproduced by 
  permission.} 

  Figure 2 shows a sequence of images spread through a single cycle of periodic 
  oscillation, when the “recorder” is playing a steady note. Note the 
  orientation: the resonating pipe lies below the labium, in the lower part of 
  each figure --- exactly as in the normal configuration for playing a 
  recorder. To interpret these images, we will need to invoke some of the 
  fluid-dynamical concepts outlined in section 11.2: we can see both laminar 
  and turbulent flow, we can see flow separation at the sharp edges of the 
  labium and of the slot, we can see vortices being formed. 

  For this particular geometry and blowing pressure, the jet emerging from the 
  slot (at the left of each image) remains laminar long enough to reach the 
  sharp edge. But just a little further downstream, you can see that the jet 
  becomes turbulent. Looking more carefully at the sequence of images, during 
  this cycle of oscillation the jet “switches” from one side of the labium to 
  the other. Look particularly at the fourth image in the sequence (right-hand 
  column, second from the top). The jet is still lying predominantly above the 
  labium (in other words, outside the pipe), but in the gap between the slot 
  and the sharp edge (the “mouth” of the recorder) you can see that the jet is 
  bending downwards. By the next picture in the sequence, it has switched to 
  the lower side (but you can still see the residual turbulence left behind on 
  the upper side). The final image in the sequence shows the opposite 
  transition: the jet has bent upwards, and is in the process of switching back 
  to the upper side. 

  This switching process involves two physical phenomena. The one we can see 
  some evidence of in the images is instability of the jet: when the jet is 
  perturbed, a disturbance travels along the jet and it grows as it travels. 
  What is causing the perturbation is the second phenomenon, not directly 
  visible in these images. The “recorder” is playing a note with a frequency 
  close to the lowest resonance of the tube, and the tube is open at the mouth, 
  so there is an acoustic air flow in and out through the mouth. This 
  acoustical flow interacts with the jet, triggering a disturbance in the jet 
  near the slot. The disturbance grows as it is carried along by the jet, and 
  ultimately causes switching from one side of the labium to the other. 

  Each time the jet switches past the sharp edge, it sheds a vortex. We can see 
  vortices even more clearly in Fig.\ 3, which shows a similar sequence of 
  schlieren images, but this time they show the very early stages of a 
  transient after the air supply has been abruptly turned on. The first two 
  images in this sequence show a very clear mushroom-shaped vortex being 
  generated from the sharp edges of the slot on the left-hand side. When this 
  vortex reaches the labium, it interacts to generate more complicated vortex 
  structures. 

  The images in Fig.\ 2 illustrate a regime of vibration which is generally 
  accepted to form the basis for “normal” playing on a recorder or a transverse 
  flute. However, this is not true for all instruments, or for all playing 
  techniques. In Fig.\ 2 we see a relatively short, relatively thin jet which 
  is laminar throughout the region of most interest. This thin jet tends to 
  move as a whole, in a sinuous motion. But sometimes we might have a laminar 
  jet which is wider, in which case a different-looking fluid-mechanical 
  phenomenon occurs, illustrated in Fig.\ 4 (reproduced from section 11.2). The 
  two separate shear layers on each side of the wide jet show a classic 
  instability pattern in which they generate vortices on alternate sides before 
  the jet reaches the labium. 

  \fig{figs/fig-20eca420.png}{Figure 4. A copy of Fig. 8 from section 11.2, 
  showing a schlieren image of a wide air-jet interacting with a sharp edge in 
  the upper right of the picture. It shows unstable vortex growth of the shear 
  layers on either side of the jet. Images created by Bram Wijnands, Sylvie 
  Dequand and Avraham Hirschberg, and reproduced by permission.} 

  If the Reynolds number of the air-jet is higher, either because the flow is 
  faster or the jet is longer, the jet becomes turbulent before it reaches the 
  labium. Flue organ pipes are commonly designed like this, for example. The 
  jet may still be able to interact in a way that gives a periodic movement of 
  the jet centre-line, but the sound of this periodic oscillation will be 
  accompanied by a significant component of “wind noise” generated by the 
  interaction of the turbulence with the edge. The result is a more “breathy” 
  sound quality, desirable in some instruments (such as the Japanese 
  shakuhachi). 

  Jets that show different combinations of short or long, narrow or wide, 
  laminar or turbulent would require different theoretical modelling. Only for 
  a few of these combinations have credible models been developed that can 
  allow simulations of the kind we have shown for the other wind instruments. 
  We will confine our attention to just one of these cases, the one illustrated 
  in Fig.\ 2. For this type of problem, a procedure known as the “jet-drive 
  model” has been found to give reasonably good results. The approach was 
  pioneered by John Coltman back in the 1960s [1,2]. This model has three main 
  ingredients: the time delay and exponential growth associated with a 
  travelling, growing perturbation on the jet; a simplified form of how the 
  acoustical air-flow from the resonant tube perturbs the jet; and a dipole 
  sound source from the interaction of the switching jet with the labium, which 
  generates acoustical pressure that completes the feedback loop. Some details 
  of this model are given in the next link, based on the descriptions given by 
  Auvray, Ernoult, Fabre and Lagrée [3], and by Fabre in chapter 10 of Chaigne 
  and Kergomard [4]. 

  There is only one other case for which a simple model suitable for simulation 
  has been developed and validated against experiments. This is the case 
  illustrated in Fig.\ 4, for which a “discrete vortex model” has been 
  presented in the same two references [3,4], building on pioneering work by 
  Holger, Wilson and Beavers in the 1970s [5]. This oscillation regime has some 
  relevance to musical performance, but it seems to be rather marginal compared 
  with the jet-drive model, which relates to the most common playing regime on 
  the recorder or transverse flute. So we will concentrate on the jet-drive 
  model, keeping in mind that it involves major simplifications to the messy 
  fluid mechanics seen in Fig.\ 2 — and it doesn’t even attempt to describe 
  transient behaviour like Fig.\ 3, since the model assumes that the jet is 
  ``already there'' when the note starts. As a result, we can hope to predict 
  some qualitative patterns of behaviour but we should not expect quantitative 
  transient predictions relating to the subtle things players do to start notes 
  in particular ways. 

  \textbf{B. The influence of cork position in a transverse flute} 

  But before we plunge into simulation results, we need to say a few words 
  about the easy part of the problem, the linear acoustics of the tube. For all 
  these instruments, whether it is a flute, recorder or flue organ pipe, we 
  have a more-or-less cylindrical tube, open at both ends (or stopped at the 
  far end, in the case of a stopped organ pipe). For the recorder or the organ 
  pipe, we already know more or less how the input admittance will behave: it 
  will resemble the example shown back in section 11.1, in Fig.\ 14 there. 

  But the transverse flute brings in another ingredient, which has a 
  consequence similar in some ways to the mouthpiece of a brass instrument, 
  discussed in section 11.6. As shown schematically in Fig.\ 5, the embouchure 
  hole of a flute is not right at the end of the tube, like the mouth of a 
  recorder. Instead, there is a short length of closed tube extending in the 
  opposite direction to the main playing length, with the fingerholes. The 
  effective acoustical length of this closed tube is determined by the position 
  of a tight-fitting cork, which can be adjusted. 

  \fig{figs/fig-069b63f4.png}{Figure 5. Schematic diagram of a transverse 
  flute, showing the closed tube to the left of the embouchure hole and the 
  adjustable cork that determines the acoustical length of that tube. The 
  embouchure hole is shown as a ``chimney'': its effective length is a 
  combination of the actual thickness of material, plus end corrections at both 
  ends.} 

  The influence of this closed tube is significant. The black dashed curve in 
  Fig.\ 6 shows the measured input admittance of a modern Boehm flute for the 
  note $C_4$, 262~Hz, with all but the last fingerhole closed. The red curve 
  shows the predicted input admittance for an ideal double-open tube, with a 
  suitable length corresponding to the distance from embouchure hole to tube 
  end. We will see in a moment that when we correct for the extra closed tube, 
  this length of plain tube will produce more or less the same set of resonance 
  peaks as the measurement, although at this stage they are all a little too 
  high. But the most striking disparity between the two curves concerns the 
  peak heights rather than their frequencies. It can be seen that the first few 
  peak heights are rather similar to the measurement, but the measured 
  admittance dies away rapidly in amplitude while the theoretical curve carries 
  on with strong peaks. (The measurement only extended up to 4~kHz.) 

  Figure 7 shows the effect of adding the extra closed tube to the calculation, 
  as explained in the next link. The red curve from Fig.\ 6 is included as a 
  dotted line, for comparison. Using a carefully-selected cork position (in 
  other words, a length for the closed tube), the result is an excellent match 
  to the measurement. The peak frequencies and amplitudes are well captured for 
  the first nine or so resonances. Above that, we start to see small 
  deviations. The inclusion of the closed tube has clearly made a crucial 
  difference to the frequency response, but the simple model has captured this 
  difference very well. 

  \fig{figs/fig-73841a55.png}{Figure 7. The measured admittance from Fig. 6 
  (black dashed curve) compared to predictions using a straight tube (red 
  dotted curve, as in Fig. 6) and from a model including the effect of the 
  closed tube with a suitable cork position (red curve). In the model, the cork 
  is placed 18~mm from the embouchure hole.} 

  Figure 8 shows what the simple model predicts if we move the cork, to change 
  the length of the closed tube. The red curve is the same as in Fig.\ 7, while 
  the blue and green curves (displaced vertically in the plot for clarity) show 
  the effect of a shorter and a longer tube, respectively. The differences here 
  are far larger than a player or flute-maker would normally make, to show the 
  effect clearly. The admittance peaks fade away to zero at a frequency that 
  depends on the cork position: making the closed tube shorter raises the 
  frequency, making it longer reduces the frequency. 

  \fig{figs/fig-eab70349.png}{Figure 8. The effect on input admittance of 
  varying the cork position. The red and black curves are the same as in Fig. 
  7. The blue curve is the prediction for a cork position 10~mm behind the 
  embouchure hole, the green curve for a distance 25~mm. The blue and green 
  curves have been displaced vertically by 40~dB for clarity.} 

  The effect was discussed by Benade [6], and we can illustrate his explanation 
  with some computed examples of the pressure distribution inside the tube of 
  the flute. Figure 9 shows the theoretical admittance from Figs.\ 7 and 8, and 
  a selection of the resonance peaks have been marked. At each of these 
  frequencies, making use of the analysis described in the previous link, the 
  pressure distribution along the tube has been plotted in Fig.\ 10. The main 
  portion of the flute, from embouchure hole to the open end, is indicated by 
  blue curves, while the short portion inside the closed tube is plotted in 
  red. All the curves have been scaled to the same peak amplitude. 

  \fig{figs/fig-22ce438d.png}{Figure 9. The predicted admittance as in Fig. 7, 
  annotated to mark the peaks that are examined in Fig. 10.} 

  \fig{figs/fig-8d6a8434.png}{Figure 10. The computed pattern of pressure along 
  the tube, at the set of frequencies marked by green circles in Fig. 9. The 
  first four tube resonances are shown, then a group of three in the run-up to 
  the ``dead zone'' around 4.5~kHz, and finally one peak above that dead zone. 
  In each case, the pressure is plotted in red in the closed tube and in blue 
  for the rest of the tube. Each shape has been normalised to the same 
  amplitude.} 

  The first four curves, at the top in Fig.\ 10, correspond to the four lowest 
  resonances of the tube. The pressure distributions show more or less what we 
  expect: successively one, two, three and four half-wavelengths fitted in to 
  the length of the tube. At the open end (at the right) the simple model 
  imposes a nodal point of pressure. But this is not quite true at the 
  embouchure hole. The hole is smaller than the bore of the tube, and there is 
  a small but non-zero mass of air in the short “chimney” sketched in Fig.\ 5. 
  The combined effects of these two things is that the pressure inside the tube 
  does not fall to zero at the hole. 

  The red portions of the curves, inside the closed tube, show different 
  behaviour. At the closed end the pressure variation must have a horizontal 
  tangent, corresponding to the fact that the volume flow rate must be zero 
  there. The pressure rises towards the closed end, but for these modes at low 
  frequency the rise is very slight, barely visible in the figure. This is as 
  we expect: at frequencies such that the length of the closed tube is very 
  short compared to the wavelength of sound, the pressure must be more or less 
  constant. (We used the same approximation when we talked about the Helmholtz 
  resonator back in section 4.2.1, and we used it again when discussing the 
  effect of the mouthpiece volume in conical woodwind instrument like the 
  saxophone, in section 11.3, part F.) 

  The next group of three plots correspond to three resonance peaks in Fig.\ 9 
  which are approaching the “dead zone” where the admittance shows no peaks. 
  The final plot in Fig.\ 10 corresponds to a peak beyond this dead zone. For 
  all four of these frequencies, Figure 10 shows rather similar behaviour. The 
  red portion of each curve joins rather smoothly on to the blue portion, and 
  the combined pressure variation looks very much like what we would see if the 
  embouchure hole was blocked, so that we had a closed-open tube with no 
  side-hole. The shapes look sinusoidal, with an antinode at the closed end and 
  a node at the open end. 

  Now look closely at the pattern of the joins between the red and blue curves 
  — you may find it helpful to look at Fig.\ 11, which shows a zoomed view of 
  Fig.\ 10 concentrating near the region of the join at the embouchure hole. 
  For the lower four plots, in the vicinity of the dead zone, we see a 
  systematic pattern of behaviour. If you look at the position of the first 
  nodal point of pressure, you will see that it moves closer and closer to the 
  red-blue join as the dead zone is approached. Then in the lowest plot, above 
  the dead zone, the nodal point has moved into the red portion of the curve. 
  Now we can see that the centre of the dead zone is determined by the 
  condition that there is exactly a quarter-wavelength fitted into the closed 
  tube. The embouchure hole then falls exactly at a nodal point of pressure, so 
  at that frequency it is not possible to excite the internal tube resonance 
  from the embouchure hole. That is the reason the peaks heights in the 
  admittance fade away to zero at this frequency, then grow again once the 
  wavelength is shorter. 

  \fig{figs/fig-7f8d5b5f.png}{Figure 11. A zoomed view of Fig. 10, showing just 
  the region of the tube near the embouchure hole so that the join of the red 
  and blue regions can be seen more clearly.} 

  \textbf{C. Simulation results} 

  We are ready to combine the jet-drive model set out in section 11.8.1 with 
  measured input admittance, to run some simulations. We need to choose an 
  input admittance: the three cases to be used here are shown in Fig.\ 12. The 
  lowest plot, shown in a black dashed line, is the flute measurement we have 
  already seen in Figs.\ 6, 7 and 8. But we will not start with that, for 
  reasons that will emerge shortly. The top curve, shown in red, is a 
  three-mode approximation to the measured admittance of a recorder, as used in 
  reference [1]. In order to check the simulation code, we will start with a 
  run matching that reference --- more details of this comparison, including 
  the parameter values used in the simulations, are given in the first link, 
  section 11.8.1. After that we will look at a comparable note on a modern 
  transverse flute, shown in the blue curve in Fig.\ 12. 

  In reference [3], the results for playing frequency and amplitude were 
  extracted from single long simulations in which the jet speed was slowly 
  ramped up or down. Instead, the results shown here use a separate transient 
  for each of a set of steady values of the jet speed, in steps of 0.1 m/s. We 
  do not expect this model to give accurate results for initial transients, for 
  reasons explained above, so we are mainly interested in the possible pattern 
  of steady notes that can be produced on this ``recorder''. To focus on this 
  question, each transient was “seeded” with a small pressure fluctuation in 
  the lowest mode, as was done by Auvray et al. [1], and also as we sometimes 
  did in earlier sections on reed and brass instruments. 

  Figure 13 shows a first set of results. The playing frequency has been 
  normalised by the nominal fundamental frequency to give what we earlier 
  called the “correlation harmonic number”. For in-tune notes, this should take 
  values close to 1, 2, 3 etc. This quantity is plotted against the jet speed, 
  and the resulting plot is surprisingly complicated. At least as predicted by 
  this crude model, the frequency jumps all over the place, especially with 
  very slow jet speeds. The “normal” playing regime for this note corresponds 
  to the flattish portion of the plot for jet speed in the range 16—32 m/s. The 
  line can be seen to slope slightly uphill: the frequency of the note is close 
  to the nominal fundamental, but it increases slightly as the jet speed 
  increases. At higher speed we see another flattish portion with the 
  normalised frequency roughly equal to 2: the “recorder” has overblown to the 
  octave. This line also slopes gently upwards. 

  \fig{figs/fig-6e11b7dd.png}{Figure 13. Playing frequency deduced from 
  simulations using a set of jet speeds in steps of 0.1~m/s, using the input 
  admittance shown in the red curve of Fig. 12. Each frequency has been 
  normalised by the nominal fundamental frequency, and plotted against the jet 
  speed.} 

  The curious cascades of frequencies at low jet speeds are related to an 
  advanced flute-playing technique known as “Aeolian tones”. To see what is 
  going on in the model to produce these tones, it is useful to plot the data 
  in a different way. Instead of using the jet speed on the horizontal axis, we 
  can use the delay time for disturbances on the jet, as they travel the 
  distance from the slot to the labium. This delay time is simply determined by 
  the jet speed and the distance to the labium: in recorder terminology, this 
  is the width of the mouth. The disturbances do not travel exactly at the jet 
  speed, but within this simplified model they travel at 40\% of that speed. 

  The resulting plot is shown in Fig.\ 14. It is important to note that by 
  using delay rather than jet speed on the axis, we have reversed the order of 
  the points: the left-most clusters of points here correspond to results on 
  the right-hand side of Fig.\ 13, and vice versa. The delay time has been 
  expressed as a fraction of the period length of each note, and it can 
  immediately be seen that the plotted points cluster around particular values 
  of this normalised delay: there are columns of clustered points near the 
  values $\frac{1}{4}$, $1 \frac{1}{4}$, $2 \frac{1}{4}$etc. 

  \fig{figs/fig-825650fd.png}{Figure 14. The same frequency data as plotted in 
  Fig. 13, but now plotted against the delay time for perturbations of the jet 
  as they travel from slot to labium. Note that this reverses the direction of 
  the axis, so that the ``normal'' notes which appeared towards the right-hand 
  side of Fig. 13 now appear clustered on the left. The delay time has been 
  normalised as a fraction of the period of each separate note, and the 
  ``normal'' fundamental and octave notes both appear with this normalised 
  delay approximately equal to 1/4.} 

  The explanation of this pattern is given in the earlier side link, section 
  11.8.1. For any note that can be played at a frequency close to a peak of 
  admittance, the normalised delay needs to be in the vicinity of 
  $n+\frac{1}{4}$, where $n$ could take the values 0, 1, 2,… The normal playing 
  regime on a flute or recorder corresponds to $n=0$, while the Aeolian tones 
  correspond to higher values of $n$. What is happening is illustrated in 
  sketch form in Fig.\ 15: the disturbance on the jet needs to be in a 
  particular phase when it reaches the labium, and this can only be achieved by 
  fitting certain numbers of quarter-wavelengths into the width of the mouth. 
  With a fast jet speed, only the longest wavelength option can be achieved, 
  but with a slow jet it is possible to fit extra wavelengths in. It is then 
  clear that the four columns of points evident in Fig.\ 14 correspond to the 
  values $n=0,1,2,3$. The ``normal'' fundamental and octave notes both appear 
  in the $n=0$ column. 

  \fig{figs/fig-e8dff14c.png}{Figure 15. Sketches of the jet configuration 
  corresponding to different allowed values of the delay} 

  Aeolian tones like this can really occur on flutes or recorders, but the 
  details of them predicted by this model should be taken with a generous pinch 
  of salt. In reality a jet with these extra wiggles will tend to break up into 
  separate vortices, which is the case addressed by the discrete-vortex model, 
  which we will not go into here. According to \tt{}this web site\rm{} 
  (addressing non-standard flute performance techniques) “Aeolian sounds are 
  colored air sounds ~with no normal flute tone. The~air stream across the 
  embouchure produces an airy pitch resonance in the~first octave~only, low B- 
  middle D\#”. We can perhaps attribute the ``airy'' sound to the fact that the 
  jet is more likely to become turbulent under these conditions, generating 
  ``wind noise'' when it interacts with the labium. The statement that the 
  effect is only found in the lowest octave agrees with the experimental 
  findings of Auvray et al. [3]: they were able to obtain an Aeolian tone with 
  their artificially-blown model recorder corresponding to the fundamental 
  pitch with $n=1$, but not for higher overtones or higher values of $n$. So it 
  appears that the jet-drive model is considerably over-predicting the 
  occurrence of Aeolian tones. 

  The next step is to see what changes if we use input admittance measured from 
  a transverse flute. The blue curve in Fig.\ 12 shows a measurement from 
  \tt{}Joe Wolfe’s website\rm{}, for a modern Boehm flute with B foot, fingered 
  in a manner that he says flautists use for both $C_5$ and $C_6$, adjusting 
  the speed, length and shape of the air jet to choose between these two notes. 
  On a recorder, the only one of these available for a player to vary is the 
  jet speed. The transverse flute allows extra degrees of freedom, as the 
  player can shape the lip opening and vary the distance from the edge as well 
  as adjusting flow speed. 

  Figures 16 and 17 show equivalent plots to Figs.\ 13 and 14, keeping the same 
  parameter values as those runs so that this could be thought of as a 
  transverse flute played using a recorder mouthpiece. It is immediately clear 
  that the results are very similar. In this case, the inclusion of extra modes 
  in the admittance, and the small differences of peak frequencies, heights and 
  bandwidths, all seem to make rather little difference to the outcome. 

  \fig{figs/fig-cd2f4338.png}{Figure 16. Normalised playing frequency plotted 
  against jet speed in the same format as Fig. 13, computed using the 
  admittance shown as the blue curve in Fig. 12. The green and blue circles 
  mark cases for which waveforms are plotted in Figs. 18 and 19 respectively.} 

  \fig{figs/fig-b448c04f.png}{Figure 17. Normalised playing frequency plotted 
  against normalised jet delay in the same format as Fig. 14. The frequency 
  data is the same as in Fig. 16, computed using the admittance shown as the 
  blue curve in Fig. 12. The green and blue circles mark cases for which 
  waveforms are plotted in Figs. 18 and 19 respectively.} 

  The circles in these two plots pick out a case playing the fundamental (i.e. 
  $C_5$) and a case playing the octave ($C_6$), The detailed waveforms for 
  these two cases are shown in Figs.\ 18 and 19. In each group, the top plot 
  shows the acoustic pressure inside the mouthpiece and the middle plot shows 
  the acoustic volume flow rate through the mouth (or embouchure hole). The 
  bottom plot shows an indication of the jet position: a value approaching $-1$ 
  indicates that the jet is fully above the labium and outside the tube, while 
  a value approaching $+1$ indicates the converse with jet fully below the 
  labium. (Specifically, in terms of the model set out in section 11.8.1, this 
  plot shows the combination $\tanh \left( \dfrac{\eta(w,t)-y_0}{b} \right)$ 
  that appears in equation (6).) 

  \fig{figs/fig-2ad44b5a.png}{Figure 18. Waveforms from the simulation run 
  marked with green circles in Figs. 16 and 17. The top plot shows the acoustic 
  pressure just inside the mouthpiece, the middle plot shows the acoustic 
  volume flow rate inwards through the embouchure hole, and the bottom plot 
  indicates the jet position as described in the main text.} 

  \fig{figs/fig-f72f505b.png}{Figure 19. Waveforms plotted in the same format 
  as Fig. 18, for the simulation run marked with blue circles in Figs. 16 and 
  17.} 

  Apart from the octave difference in frequency, the waveforms are recognisably 
  similar. The jet is switching in and out the tube once every cycle, and the 
  acoustic volume flow rate is broadly following the jet position. The pressure 
  waveforms show a pair of main peaks in every cycle, one up and the other 
  down. To estimate the sound radiated from the embouchure hole by these 
  waveforms, we can note that the hole is very small compared to the 
  wavelength, so (from the analysis in section 11.7.1) the radiated sound 
  pressure will be proportional to the rate of change of the volume flow rate 
  through the hole. The resulting sound of the two simulations from which 
  Figs.\ 18 and 19 were extracted can be heard in Sound 1. Do these sound 
  recognisably like a flute? Not really, it must be admitted. There are two 
  main reasons. First, the main body of each sound is accurately periodic, with 
  none of the “wind noise” that accompanies a real flute note, and none of the 
  subtle modulation (such as vibrato) that a human player would add. Second, 
  although each note here has a starting transient, it is not one that we have 
  any great confidence in because of the limitations of the model. 
  Nevertheless, perhaps these sounds capture something a bit flute-like? 

  Now, finally, we can look at some results based on the input admittance for 
  the Boehm flute fingered for the note $C_4$, the black dashed curve in Fig.\ 
  12 and the one we used earlier when we were looking at the influence of the 
  cork position. Using the same parameter values as the previous simulations, 
  this admittance leads to the behaviour shown in Figs.\ 20 and 21 in the same 
  format as earlier plots. The general pattern is recognisable based on the 
  earlier discussion, but the model is predicting that the fundamental tone is 
  never produced. The clusters of points for Aeolian tones and for “normal” 
  tones show normalised frequencies close to 2, 3, 4 up to 7, but never 1. As 
  is well known, beginners on the flute often struggle to achieve good tone on 
  the lowest notes on the instrument, and to avoid overblowing to the octave, 
  but it is not supposed to be completely impossible! 

  \fig{figs/fig-24910bac.png}{Figure 20. Normalised frequency as a function of 
  jet speed, in the same format as Figs. 13 and 16, for simulations based on 
  the input admittance shown as a dashed black line in Fig. 12.} 

  \fig{figs/fig-0cfe37c7.png}{Figure 21. The frequency data as in Fig. 20, 
  plotted as a function of jet delay in the same format as Figs. 14 and 17.} 

  We can perhaps guess what the problem is. We can see whereabouts in Fig.\ 20 
  we would expect the “normal” playing regime at the fundamental to sit, but in 
  this range of jet speed the model is choosing instead to play Aeolian tones 
  at high frequency. But we already know that this jet-drive model seriously 
  over-estimates the prevalence of Aeolian tones. We can test this idea with a 
  simple numerical experiment. We can take the same input admittance, but only 
  use the first 4 resonances and ignore the higher ones. Without those higher 
  resonances, it should not be possible for the associated Aeolian tones to be 
  produced. 

  Figures 22 and 23 show the result, keeping all other details the same as in 
  Figs.\ 20 and 21. By suppressing the possibility of Aeolian tones higher than 
  the 4th harmonic, we have opened a gap in the range of jet speed previously 
  filled with Aeolian tones. Figure 22 shows that this gap is populated in part 
  by extending the range when the octave is played (normalised frequency around 
  2), but it also now includes a short interval in which the fundamental is 
  played (normalised frequency around 1). Figure 23 confirms that this short 
  interval is indeed associated with values of the delay around a 
  quarter-period, in other words with $n=0$. This model is, of course, 
  artificial: the higher resonances really are there in the admittance. But the 
  experiment is enough to confirm that by suppressing some Aeolian tones that 
  are never in practice heard, it becomes possible for the model to produce 
  this fundamental note. 

  \fig{figs/fig-807411b2.png}{Figure 22. Normalised playing frequency as a 
  function of jet speed, exactly as in Fig. 20 except that only the first 4 
  resonances of the tube are included in the calculations.} 

  \fig{figs/fig-00621567.png}{Figure 23. The normalised frequency data from 
  Fig. 22, plotted as a function of jet delay in the same format as earlier 
  plots.} 

  We have seen the crucial role of the delay, the time taken for disturbances 
  to travel along the jet from slot (or lips) to labium. In a recorder, the 
  geometry is fixed by the instrument maker, and the only way the player can 
  influence this delay is by changing the jet speed. But in a transverse flute, 
  the player has extra degrees of freedom. For the low note we were considering 
  in the last example, rather a long delay is needed for normal tone. In the 
  simulations, the distance has been kept fixed, and the appropriate delay was 
  achieved in a narrow range of rather slow jet speeds. 

  On a transverse flute, the same delay could be achieved with a faster jet and 
  a correspondingly longer distance: the player can roll the embouchure hole 
  away from the lips, while blowing harder. But this combination of changes 
  will also increase the Reynolds Number, and make it more likely that the jet 
  will become turbulent. Possibly the player can counteract this tendency, to 
  an extent, by careful shaping of the lips to tailor the details of the flow 
  leaving the mouth. The act of rolling the tube so the embouchure hole is less 
  covered will also affect the intonation of the note, by changing the end 
  correction at the hole. All this gives a hint of the subtleties that a 
  flautist must learn to master. But to represent such effects in a simulation 
  lies beyond the reach of the very simple model we are using here. Indeed, it 
  is far from clear that any simple model could be constructed that would 
  capture such effects accurately. This challenging task is a topic for future 
  research. 



  \sectionreferences{}[1] John W. Coltman, “Sounding mechanism of the flute and 
  organ pipe,” Journal of the Acoustical Society of America \textbf{44}, 
  983–992 (1968). 

  [2] John W. Coltman, “Jet drive mechanisms in edge tones and organ pipes,” 
  Journal of the Acoustical Society of America \textbf{60}, 725–733 (1976). 

  [3] Roman Auvray, Augustin Ernoult, Benoît Fabre and Pierre-Yves Lagrée, 
  “Time-domain simulation of flute-like instruments: Comparison of jet-drive 
  and discrete-vortex models”, Journal of the Acoustical Society of America 
  \textbf{136}, 389—400 (2014) 

  [4] Antoine Chaigne and Jean Kergomard; “Acoustics of musical instruments”, 
  Springer/ASA press (2013) 

  [5] D.K. Holger, T. A. Wilson and G. S. Beavers: “Fluid mechanics of the 
  edgetone'', Journal of the Acoustical Society of America \textbf{62}, 
  1116—1128 (1977). 

  [6] Arthur H. Benade; “Fundamentals of Musical Acoustics”, Oxford University 
  Press (1976), reprinted by Dover (1990). See chapter 22. 