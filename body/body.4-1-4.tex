  Whenever we need to cover technical material relating to fluid dynamics, we 
  need the language and machinery of vector calculus, so it is useful to give 
  an overview of the subject here. There will be no proofs or derivations in 
  this section, but the key concepts will be introduced and I will attempt to 
  explain how they all fit together. Vector calculus provides the essential 
  mathematical toolkit for analysing many problems in three dimensions: not 
  only fluid dynamics, but also areas ranging from electromagnetism to computer 
  graphics. 

  First, a reminder of what ordinary calculus is about. Newton invented 
  calculus in order to solve problems involving gravity and the motion of 
  projectiles or the solar system. Without the concepts of calculus we can't 
  even begin to apply modern scientific ideas to the simplest problems in 
  dynamics: Newton's governing equations are differential equations. So what 
  does ordinary calculus consist of? There are three things.... 

  First, differentiation: things like 

  \begin{equation*}\dfrac{df}{dx}\end{equation*} 

  \noindent{}which you may interpret as the slope of a graph of the function 
  $f(x)$. 

  Second, integration: things like 

  \begin{equation*}\int_a^b{f(x) dx}\end{equation*} 

  \noindent{}which you may interpret as the area under the graph of $f(x)$ 
  between the points $x=a$ and $x=b$. 

  The third thing is perhaps the one you did not immediately think of. There is 
  a very important theorem which connects the first two things together: 

  \begin{equation*}\int_a^b{\dfrac{df}{dx} dx} = [f(x)]_a^b . 
  \tag{1}\end{equation*} 

  In words, the integral of a derivative gives you the original function, 
  evaluated on the boundaries of the region of interest. 

  Vector calculus involves versions of the same three things, albeit made more 
  complicated by working in three dimensions instead of one. 

  \textbf{A. Vector differentiation: grad, div and curl} 

  There are three types of vector derivative. The simplest of them is called 
  the gradient. Suppose we have a scalar field variable, like temperature 
  $T(\underline{r})$, where $\underline{r}$ is the position vector with 
  components $(x,y,z)$ with respect to Cartesian axes. Then the vector 

  \begin{equation*}\mathrm{grad~}T = \nabla T = \left[\dfrac{\partial 
  T}{\partial x},\dfrac{\partial T}{\partial y},\dfrac{\partial T}{\partial 
  z}\right] \tag{2}\end{equation*} 

  \noindent{}is called the gradient of $T$. This vector has a simple physical 
  interpretation. Its direction is towards the steepest rate of increase of 
  temperature, and its magnitude is the rate of increase. One interpretation of 
  this equation is that we have a vector operator 

  \begin{equation*}\nabla = \left[\dfrac{\partial }{\partial x},\dfrac{\partial 
  }{\partial y},\dfrac{\partial }{\partial z}\right] \tag{3}\end{equation*} 

  \noindent{}acting on the scalar field $T$ to produce the vector gradient. 
  This operator is usually called ``del'', or sometimes ``nabla''. 

  Now suppose we have a vector field variable instead of a scalar field: for 
  example the velocity field $\underline{u}(\underline{r})$ in a fluid flow. In 
  what sensible ways might we combine a vector operator with a vector function 
  like this? There are only two, which correspond to the two ways to form a 
  kind of product from two vectors. These are the scalar product (or ``dot 
  product''), and the vector product (or ``cross product''). If we have two 
  vectors $\underline{a}$ and $\underline{b}$, with components $(a_1,a_2,a_3)$ 
  and $(b_1,b_2,b_3)$, the scalar product is defined by 

  \begin{equation*}\underline{a}\cdot \underline{b} = a_1b_1+a_2b_2+a_3b_3 
  \tag{4}\end{equation*} 

  \noindent{}and the vector product is defined by a $3 \times 3$ determinant: 

  \begin{equation*}\underline{a} \times \underline{b} = \begin{vmatrix} 
  \underline{i}\&\underline{j}\&\underline{k}\\a_1\&a_2\&a_3\\b_1\&b_2\&b_3 
  \end{vmatrix}\end{equation*} 

  
  \begin{equation*}=(a_2b_3-a_3b_2)\underline{i}+(a_3b_1-a_1b_3)\underline{j}+(a_1b_2-a_2b_1)\underline{j} 
  \tag{5}\end{equation*} 

  \noindent{}in terms of the unit vectors 
  $\underline{i},\underline{j},\underline{k}$ in the $x$, $y$ and $z$ 
  directions respectively. 

  So we define two vector derivatives of our vector field $\underline{u}$ in 
  the same way. First is the divergence: 

  \begin{equation*}\mathrm{div~} \underline{u} = \nabla \cdot \underline{u} = 
  \dfrac{\partial u_1}{\partial x}+\dfrac{\partial u_2}{\partial 
  y}+\dfrac{\partial u_3}{\partial z} \tag{6}\end{equation*} 

  \noindent{}where the components of $\underline{u}$ are $(u_1,u_2,u_3)$. The 
  second one is the curl: 

  \begin{equation*}\mathrm{curl~} \underline{u} = \nabla \times \underline{u} 
  =\begin{vmatrix} \underline{i}\&\underline{j}\&\underline{k}\\ 
  \frac{\partial}{\partial x}\&\frac{\partial}{\partial 
  y}\&\frac{\partial}{\partial z}\\u_1\&u_2\&u_3 \end{vmatrix} . 
  \tag{7}\end{equation*} 

  Fluid dynamics provides good examples of physical interpretation of 
  divergence and curl. If you inject a small blob of dye into a flowing fluid 
  with velocity field $\underline{u}$, $\nabla \cdot \underline{u}$ determines 
  the rate of change of the volume of the blob as it moves with the flow, while 
  $\nabla \times \underline{u}$ (called ``vorticity'') determines the rate and 
  axis of rotation of the blob. We will say more about this later, in section 
  11.2.1. 

  \textbf{B. Three-dimensional integration: volume, surface and line integrals} 

  Integrals also become rather more complicated for 3D problems: again, there 
  are three types that crop up naturally in different applications. First is 
  the volume integral. What is the total mass of a car? Various materials of 
  different densities are distributed within a defined volume. If the variation 
  of density is $\rho(\underline{r})$, the total mass is 

  \begin{equation*}\int{\int{\int_V{\rho(x,y,z) dx}dy}dz} = 
  \int{\int{\int_V{\rho(\underline{r})}}dV} \tag{8}\end{equation*} 

  \noindent{}since $\rho dV$ is the mass inside a small volume $dV$, and to 
  find the total mass you simply need to add them all up, so the integration 
  needs to encompass the entire volume $V$ of the car. 

  Our second type of integral is a surface integral. Again, we can introduce 
  the idea through an example. Water flows through a pipe, with some velocity 
  profile $\underline{u}(\underline{r})$. What is the total flow rate in litres 
  per second? We could choose any surface which spans across the pipe (it need 
  not be a flat surface), and find the flow rate across that surface: Fig.\ 1 
  shows a sketch. 

  \fig{figs/fig-835f27db.png}{\caption{Figure 1. Sketch of flow through a pipe, 
  being integrated over a surface spanning the pipe cross-section. The lower 
  diagram shows a small element of area with the vector area indicated by the 
  red arrow.}} 

  Flow rate through a small element of surface is given by the local velocity 
  multiplied by the projected area parallel to the flow direction. We can 
  define the vector area $d\underline{A}$ as a vector with magnitude equal to 
  the area of a small element of the surface, and direction normal to the 
  surface: then the volume flow rate through this element of area is 
  $\underline{u} \cdot d\underline{A}$, and the total flow rate is the integral 
  of this quantity over the surface: 

  \begin{equation*}\mathrm{Flow~rate} = 
  \int{\int_{\mathrm{surface}}{\underline{u} \cdot d\underline{A}}} 
  \tag{9}\end{equation*} 

  The third type of integral is a line integral. This one is most easily 
  introduced through an electrical example. An electrically charged particle 
  (with charge $q$) is pushed along a specified curve, through an electric 
  field $\underline{E}(\underline{r})$, as sketched in Fig.\ 2. What is the 
  work done? 

  \fig{figs/fig-8ea9d091.png}{\caption{Figure 2. Sketch of the path of a 
  charged particle through an electrical field}} 

  The force on the particle is $q\underline{E}$, so the work done as it moves 
  along a small piece of the curve described by a line element $d\underline{l}$ 
  is $q\underline{E} \cdot d\underline{l}$. The total work done is then given 
  by an integral along the curve: 

  \begin{equation*}W=\int_\mathrm{curve}{q\underline{E} \cdot d\underline{l}} 
  \tag{10}\end{equation*} 

  Notice that the vector line element is defined to point tangentially to the 
  curve, whereas the vector area element was defined to point normal to the 
  surface. 

  \textbf{C. The connection: divergence theorem and Stokes' theorem} 

  Finally, there are two theorems analogous to the ``integral of a derivative'' 
  result. In the simplest form as I will state them here, they connect two of 
  our three kinds of vector derivative with all three kinds of integral. The 
  first of these is called the divergence theorem, or sometimes Gauss's 
  theorem. The context of this one is a volume $V$ with a closed surface $S$ 
  defining its outer boundary, as sketched in Fig.\ 3 (think of something like 
  a potato). 

  \fig{figs/fig-0aa08b9a.png}{\caption{Figure 3. Sketch of a closed surface $S$ 
  enclosing a volume $V$}} 

  Then, for any vector field $\underline{u}(\underline{r})$, the theorem states 
  that 

  \begin{equation*}\int{\int{\int_V{\nabla \cdot \underline{u} \mathrm{~} 
  dV}}}=\int{\int_S{\underline{u} \cdot d\underline{A}}} . 
  \tag{11}\end{equation*} 

  This looks far more complicated than our original theorem (1) for ordinary 
  calculus, but if we describe it in words we can see a parallel. Equation (11) 
  says that a triple integral of some kind of derivative (a divergence in this 
  case) is equal to a double integral of the function ($\underline{u}$), 
  evaluated on the boundary of the region of interest (the surface $S$). The 
  surface integral on the right-hand side of the equation is indeed the natural 
  way to ``evaluate on the boundary'' in the context of this three-dimensional 
  problem. So one integral has been ``cancelled'' by a derivative 

  The second theorem is called Stokes' theorem, and the context of this one is 
  a section of surface $S$ (which may be curved), with a closed curve $C$ 
  running round the edge, as sketched in Fig.\ 4 (think of something like a 
  frisbee, for example). Then the theorem states 

  \begin{equation*}\int{\int_S{\nabla \times \underline{u} \cdot 
  d\underline{A}}} = \oint_C{\underline{u} \cdot d\underline{l}} 
  \tag{12}\end{equation*} 

  \noindent{}where the integral symbol on the right-hand side means that the 
  integral is to be taken around the closed loop. 

  \fig{figs/fig-270b1ba9.png}{\caption{Figure 4. Sketch of a curve $C$ (in red) 
  forming the edge of a section of curved surface $S$}} 

  Again we can see an analogy with equation (1). This time we have a double 
  integral of a derivative of some kind (a curl in this case), being equal to 
  the function evaluated on the boundary of the region, which in this case is 
  naturally represented by a line integral round the boundary curve. So again, 
  one integral has been ``cancelled out'' by a derivative. 