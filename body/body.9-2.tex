

  There is a problem with Helmholtz motion, in the idealised form we described 
  in the previous section. There is apparently nothing a player can do to vary 
  the details of the sawtooth waveform, apart from adjusting the amplitude. But 
  surely violinists can make tonal variations by changing their bowing, or by 
  playing the same note on different strings? That intuition is perfectly 
  correct, as can readily be verified by measurement. Figure 1 shows two 
  examples of the Helmholtz sawtooth waveform: the red curve is measured on the 
  open E string of a violin, while the blue curve shows the same note played 
  high on the G string. 

  Perhaps the difference between these two waveforms looks rather subtle and 
  unimportant? By no means: look at Fig.\ 2, which shows the frequency spectra 
  of these two waveforms. The blue curve, for the G string, shows only 6 
  harmonics with significant amplitudes, confined to the frequency range below 
  about 5~kHz. But the red curve, for the E string, shows strong peaks 
  extending way beyond the range of human hearing, up to some 50~kHz! It is 
  hardly surprising that we hear a strong difference of “brightness” between 
  the sounds of these two notes. Indeed, your dog or your cat might complain 
  about the sound of the E-string note. 

  We can see something else interesting in Fig.\ 2. Notice that the first few 
  harmonics show peaks of essentially the same height in both red and blue 
  curves. These heights fall with frequency in a smooth and regular pattern. 
  This pattern is dictated by the Fourier series of an ideal sawtooth wave, for 
  which the $n$th harmonic has an amplitude proportional to $1/n$ (we proved 
  that in section 2.2.1). The blue curve starts to fall away from this pattern 
  by the 5th harmonic, but the red curve continues to follow it for a lot 
  longer. Above about 10~kHz the pattern of peaks looks different, for reasons 
  we will come back to in section 9.3. We showed the progression of the ideal 
  sawtooth Fourier series in Fig.\ 1 of section 2.2, but it is worth repeating 
  it here: see Fig.\ 3. 

  This plot reminds us that the sharpness of the “flyback” portion of the 
  sawtooth depends upon how many harmonics are included. This is the clue to 
  understanding the difference between the two waveforms in Fig.\ 1: the 
  Helmholtz corner is rather rounded in the note on the G string, but much 
  sharper in the note on the E string. This insight, and its consequences for a 
  violinist, were first explored by Lothar Cremer in the late 1960s. Cremer 
  (1905--1990) was a leading acoustician of the mid-20th century, famous 
  particularly for his work in architectural acoustics (he was the acoustical 
  consultant for the Berliner Philharmonie concert hall). He was also a keen 
  amateur viola player, and had a lifelong interest in the acoustics of the 
  violin and its relatives. Figure 4 shows him in 1974, talking to another 
  luminary of the violin acoustics world, Carleen Hutchins. 

  Cremer thought about what happens to the Helmholtz corner during a cycle of 
  Helmholtz motion. Starting from the moment when the corner leaves the bow 
  heading towards the player’s finger, the effects of propagation along the 
  string, reflection from the finger, and propagation back again all tend to 
  make an initially sharp corner more rounded. Particularly for a stopped note, 
  the mechanical properties of the player’s fingertip will result in 
  significant energy loss during wave reflection: the difference in sound 
  between an open string and a stopped note when played pizzicato demonstrates 
  this effect very directly. 

  The returning Helmholtz corner then passes the bow, and triggers a transition 
  from sticking to slipping friction. If we ignore for the moment the finite 
  width of the ribbon of bow-hair, this transition must happen at a definite 
  moment: the string sticks to the hair until limiting friction is reached, 
  then releases rather abruptly. That abruptness has the effect of sharpening 
  up the rounded corner. The corner then travels to the bridge and back, 
  becoming somewhat more rounded, then passes the bow again triggering the 
  opposite frictional transition and getting sharpened up. We will shortly 
  illustrate this process in graphical form, once we have covered a bit more 
  background material. 

  Competing effects of corner-rounding and corner-sharpening happen during each 
  cycle, and the shape of the periodic waveform is determined by a balance of 
  the two effects. The rounding effects depend on the physical properties of 
  the string and the details of its terminations. For a thin, flexible, 
  lightly-damped string like a violin E string these rounding effects are 
  rather weak, especially for an open string without the influence of a finger. 
  But for a thicker string like the G string, the bending stiffness is higher 
  and the string damping is higher, both contributing to extra rounding. If the 
  string is also stopped by a finger, the rounding effect is even stronger. 
  These are the reasons for the contrast seen in Figs.\ 1 and 2. 

  The corner-sharpening effect is determined by the frictional interaction 
  between the string and the rosin-coated bow hair. It is crucially dependent 
  on how hard the player presses down with the bow, so-called “bow force”: 
  pressing harder makes the effect stronger. Thus higher bow force shifts the 
  balance in favour of corner-sharpening, and this provides the player with a 
  way to vary the sound. We will see some computer-simulated examples of this 
  effect shortly, and more detailed results on the effect in the next section. 

  Cremer used simple analytical calculations to explore the balance of 
  corner-rounding and corner-sharpening, but he was a scientist from the 
  pre-computer age. Reading his paper [1], it became apparent to scientists of 
  the next generation that he had in fact described the essential stages of a 
  time-stepping numerical simulation of the bowed string. The result was the 
  development of the first “digital waveguide” simulations of a bowed string. 
  The method is closely related to the description we gave in section 8.5, of a 
  model for self-excited oscillation of a simplified clarinet. Just like the 
  clarinet model, it involves two variables linked by a feedback loop, as 
  sketched in Fig.\ 5. This time the variables are the sliding speed $v(t)$ of 
  the string at the bowed point, and the friction force $f(t)$ acting at that 
  point. 

  Simulations based on this model started to be performed in the late 1970s, 
  allowing Cremer’s mechanism for the influence of bow force on tone quality to 
  be investigated in quantitative detail. However, the simulation model does 
  much more than that because it opens the door to the investigation of 
  transient motions of the string: these include initial transients from a 
  given bow gesture, transitions between oscillation regimes when bifurcation 
  events occur, and the curious transient interaction that sometimes occurs 
  between bowed-string vibration and the motion of the instrument body, known 
  as a “wolf note”. We will look at all these topics in subsequent sections. 

  This early model was based on the same model of friction that we have already 
  met: what we can call the ``friction curve model'' because it assumes that 
  the friction force during sliding is determined by a nonlinear function of 
  the sliding speed. Figure 6 shows the results of some measurements of 
  friction force during steady sliding of a surface coated with violin rosin, 
  together with a curve fit to those measurements that can be used in the 
  computer simulation model. The plot also shows the vertical portion where the 
  string and the bow are sticking, and the mirror-image portion that would 
  apply if the string were to slip in the opposite direction across the bow. 
  This red curve plays the same role in the model as the nonlinear mouthpiece 
  characteristic in the clarinet model, as shown in Fig.\ 7 of section 8.5. We 
  will assume the Amontons-Coulomb ``law'' that the friction force is 
  proportional to the bow force. The plot in Fig.\ 6 takes advantage of that 
  assumption by showing the result in the form of the coefficient of friction, 
  the friction force divided by the bow force. Coefficient of friction is 
  normally described using the Greek letter $\mu$ (``mu''). 

  The details of the bowed-string model are described in the next link. Just as 
  in the clarinet model, the new values of $v(t)$ and $f(t)$ at each time step 
  are found where the friction-force curve intersects with a straight line. The 
  curve for friction force is obtained by taking the curve from Fig.\ 6, and 
  scaling it by the bow force. Figure 7 shows the result, for two different 
  values of the bow force: the solid line has a higher bow force while the 
  dashed line has a lower one. The slope of the straight line is determined by 
  the properties of the string. The horizontal position of the line is 
  determined by the sum total of reflected waves arriving back at the bowed 
  point from the two sides of the string, which can be computed using a pair of 
  “reflection functions” like Fig.\ 10 of section 8.5: one for the length of 
  string between bow and bridge, the other for the length between bow and 
  finger. 

  Figure 7 shows three possible positions of this line. All three positions 
  correspond to slipping friction with the lower value of bow force (the dashed 
  curve). The position marked `a' corresponds to slipping friction for the 
  higher bow force (solid curve), but the position `b' indicates sticking 
  friction, and we can see that the vertical portion of the friction curve 
  causes no difficulty: there is a perfectly well-defined intersection with the 
  sloping line. The position `c' is somewhat problematic for the higher bow 
  force: there are three intersections with the friction curve. However, for 
  the lower bow force this ambiguity cannot arise, because the slope of the 
  straight line is lower than the maximum slope of the friction curve. When the 
  ambiguity arises, how do we know which of the three intersections to choose? 

  The answer is the same as we found when facing a similar problem with 
  Duffing's equation, back in section 8.2 and illustrated in Fig.\ 4 there. The 
  string chooses to stay as long as it can on the branch of the curve it is 
  already following, until it is forced to make a jump. The result is a 
  hysteresis loop, illustrated in Fig.\ 8 for the case when the bow force is 
  the same as the higher value in Fig.\ 7. If the string is slipping, and the 
  line enters the ambiguous range from the left, it continues to slip until the 
  sloping line reaches the right-hand position in the figure, whereupon the 
  system jumps (following the dark green arrow) to the sticking branch. Once it 
  is sticking, it continues on that branch until such time as the sloping line 
  reaches the left-hand position in the figure, whereupon it is forced to make 
  a jump to the slipping branch following the longer dark green arrow. 

  To see the significance of this hysteresis behaviour, we can look at a first 
  example of computer simulation. We will choose a deliberately simplified 
  model, ignoring many aspects of the behaviour of real strings, in order to 
  bring out one particular phenomenon in the clearest possible way. (Remember 
  the “physics agenda”? This is a good example. We will gradually add in the 
  complicating factors ignored here, in the course of subsequent sections.) We 
  will assume for the moment that the two reflection functions for the two 
  sections of the string are the same, with a simple symmetrical shape like the 
  one used in the clarinet model (see Fig.\ 10 of section 8.5). 

  We will start the system off with an ideal, sharp Helmholtz corner, and then 
  track it through the first few stages of corner-rounding and 
  corner-sharpening. These are shown in Fig.\ 9, but you will have to look 
  backwards and forwards between the two halves of this plot in order to follow 
  the story. The dashed line in the left-hand plot shows the original ideal 
  Helmholtz corner. We are plotting waveforms of string velocity, so this 
  appears as an abrupt jump downwards from the bow speed (assumed here to be 
  0.1~m/s) to the slipping speed (assumed here to be $-1$~m/s). The jump is 
  downwards, so this plot corresponds to the moment of release, when the string 
  starts to slip. 

  The corner then travels to the violin bridge and back, and it returns rounded 
  and inverted. This new shape appears in the right-hand plot, as the blue 
  curve. Now it interacts with the friction force at the bow. We have chosen a 
  bow force high enough that the hysteresis effect of Fig.\ 8 operates. The 
  string is initially slipping, so we are on the lower branch of the hysteresis 
  cycle. The string follows round the curve until it reaches the point where it 
  must make a relatively small jump across to the sticking part of the curve: 
  this is the moment of capture of the string by the bow. The result is the 
  waveform shown in green in the right-hand panel of Fig.\ 9. You can see that 
  this follows the blue curve quite closely for a while, but then rises above 
  it and commences sticking with a sharp corner in the waveform. This is 
  “corner-sharpening” in action. 

  The new shape of Helmholtz corner then travels down to the player’s finger 
  and back, which results in more rounding and another inversion. It arrives at 
  the bow as the blue curve in the left-hand panel of Fig.\ 9. Now it interacts 
  again with the friction force, and this time we are following the upper part 
  of the hysteresis cycle. The result is the green curve in this left-hand 
  panel: the string continues to stick for a long time, then makes a big jump 
  down to rejoin the blue curve: this is the moment of release of the string by 
  the bow. 

  That is the end of one complete cycle, and the process then repeats in the 
  same sequence. The Helmholtz corner travels to the bridge and back, then 
  interacts with friction again, to produce the red curve in the right-hand 
  panel. Then to the finger and back and another frictional interaction, to 
  produce the red curve in the left-hand panel. These two red curves look very 
  similar to the two corresponding green curves, except for one thing: there is 
  a time shift. The effect of the hysteresis cycle is that the round-trip time 
  is a little longer than we expected. 

  The period of the motion is systematically lengthened relative to the natural 
  period of the free string. In musical terms, hysteresis means that the note 
  plays flat. The extent of hysteresis increases when bow force increases, as 
  Fig.\ 7 showed. The result is that if bow force is slowly increased after a 
  Helmholtz motion has been established, the note may play progressively flat. 
  This is an effect that can be readily demonstrated on a real bowed 
  instrument. The scope for flattening is determined by how rounded the 
  Helmholtz corner is: each stick-slip transition must occur somewhere within 
  the range of the rounded corner, and so the corner width puts a limit on the 
  possible extent of flattening. As a result, the flattening effect occurs most 
  strongly when playing in a high position on a thicker string, as in the blue 
  curve in Fig.\ 1. From the perspective of a player, the effect may feel more 
  like pitch instability rather than flattening as such: violinists are used to 
  fine-tuning the pitch of each note, but when the flattening effect is strong 
  the pitch may ``waver'' in response to changes in bow force, for example if 
  the bow is bouncing slightly on the string. 

  Figure 10 shows the hysteresis cycle during the simulation shown in Fig.\ 9. 
  This plot shows the regions of the friction curve traversed during the 
  capture process in red, and during the release process in blue. It should be 
  clear how this plot relates to Fig.\ 8. 

  Figures 11 and 12 show plots corresponding to Fig.\ 9 and 10, but using a 
  much lower bow force. It is low enough that hysteresis does not occur, as can 
  be seen directly in Fig.\ 12. Comparing Figs.\ 9 and 11, we can see several 
  important differences. This time, the two red curves do not show a time delay 
  compared to the two green curves. Without hysteresis, the pitch-flattening 
  effect does not occur. However, this time the shapes of the red and green 
  curves are quite different from each other. The effects of corner-sharpening 
  due to the frictional interaction are now much weaker, so, exactly as 
  Cremer’s original argument told us, the effects of corner-rounding become 
  more significant. We have only followed the Helmholtz corner for two cycles, 
  starting from an ideal sharp jump. That is not enough time for the effects of 
  corner-rounding to take full effect: the Helmholtz corner would continue to 
  get more and more rounded for a few more cycles, but we haven’t included them 
  in this plot. 

  There is another consequence of rounding of the Helmholtz corner, which we 
  haven’t mentioned yet. With a rounded corner, each velocity transition from 
  sticking to slipping, or the reverse, takes a finite time, as illustrated in 
  Figs.\ 9 and 11. During that time, if the friction-velocity characteristic 
  curve is still being followed, the force cannot be constant as in Raman's 
  original argument. Instead, there will be a pulse of extra friction force, 
  and this will excite some additional motion of the string. There will be two 
  of these pulses per cycle, generated by the processes of capture and release. 
  John Schelleng (who we will meet in the next section) called the resulting 
  features in the waveform of friction force “rabbit ears”. 

  These force pulses will send outgoing waves along the string in both 
  directions. The wave travelling in the same direction as the Helmholtz corner 
  creates the effect we have already discussed: this is precisely the origin of 
  the corner-sharpening effect we have already seen. But the other outgoing 
  wave will travel off in the opposite direction. Figure 13 might help to 
  visualise what then happens. This diagram shows time running horizontally, 
  and distance along the string running vertically. 

  The Helmholtz corner, bouncing back and forth along the string, traces out 
  the zig-zag line shown in red. The position of the bow is indicated by the 
  horizontal line. This is shown solid during the sticking intervals, and 
  dotted during the slipping intervals. This change of line type is meant to 
  indicate an important effect. While the string is sticking to the bow, if any 
  perturbing velocity hits the bow it will be reflected: the sticking bow acts 
  like a fixed end of the string. But during slipping, the bow is almost 
  transparent to any such additional perturbation (this is a consequence of the 
  friction curve being almost flat near the slipping velocity). 

  The blue lines indicate what happens to the additional waves generated by 
  Schelleng’s “rabbit ears”. Each force pulse produces a corresponding pulse of 
  velocity, which travels along the string. The solid blue zig-zag shows the 
  first three bounces of one of these pulses, generated at a slip-to-stick 
  transition (where the dotted line turns into a solid line). It is confined 
  between the bow and the bridge, because it hits the bow during sticking and 
  so is reflected. The dashed blue zig-zag shows the corresponding set of three 
  bounces of a velocity pulse generated at a stick-to-slip transition. This 
  time it is trapped between the bow and the player’s finger. 

  The solid blue zig-zag shows extra activity accompanying the Helmholtz 
  motion, and because it is on the bridge side of the bow it will show up in a 
  measurement of bridge force. Look back at Fig.\ 1: this effect accounts for 
  the fact that neither waveform is a perfect sawtooth. The path in Fig.\ 13 
  shows that pulses at the bridge are expected, with a spacing determined by 
  the position of the bow on the string: bowing closer to the bridge makes the 
  spacing tighter. The result was described as ``ripples'' by Schelleng, and 
  ``secondary waves'' by Cremer. The pattern is particularly clear in the red 
  curve of Fig.\ 1: it shows up as a series of peaks following the flyback of 
  the sawtooth, with a magnitude that steadily decreases because the secondary 
  waves are affected by the same corner-rounding effects we have already 
  discussed, but they do not have any compensating corner-sharpening mechanism 
  because they always hit the bow during an episode of sticking. 

  To see this effect in action, we can again use the simulation model. Figure 
  14 shows the resulting bridge force for three cases with different values of 
  bow force. The red curve has the lowest force, the black one the highest. The 
  pattern of ``Schelleng ripples'' is clear: as expected it gets more marked as 
  the bow force increases. The spacing of the ripples is governed by the 
  assumed position of the bow on the string. That position is usually described 
  by a parameter $\beta$ (``beta''), which is the bow-bridge distance as a 
  fraction of the total vibrating length of the string. From Fig.\ 13, we 
  expect to see roughly $1/\beta$ ripples in each cycle. For these particular 
  simulations, the value of $\beta$ was 0.083, so $1/\beta \approx 12$. You can 
  indeed count about 12 ripples in each period. Some other key parameter values 
  for these simulations are as follows. The string corresponds to the open D 
  string of a cello, with fundamental frequency 147 Hz and string impedance 
  $Z=0.55$ Ns/m. It is bowed at a speed of 5 cm/s, with the three values of bow 
  force 0.03 N, 0.1 N and 0.6 N. 

  Figure 15 shows the corresponding waveforms of string velocity, which can be 
  compared with the shapes seen in Figs.\ 9 and 11. The red curve, with the 
  lowest bow force, shows a very rounded shape, and as the force increases the 
  velocity pulse gets progressively more square. If you look carefully at the 
  black curve, you can see that the shape of the pulse is not quite 
  symmetrical, as a result of the hysteresis effect. In fact, this black curve 
  shows a bow force that is close to the limit, beyond which Helmholtz motion 
  ceases to be possible. This gives us a neat link to the next section, where 
  the limits on bow force will be discussed in some detail. 



  \sectionreferences{}[1] Lothar Cremer; ``Influence of bow pressure on 
  self-excited vibrations of stringed instruments'', Acustica, \textbf{30}, 
  119--136 (1974). 