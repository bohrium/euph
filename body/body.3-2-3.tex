  The governing equation for a vibrating plate can be derived by a similar 
  method to that used in section 3.2.1 for the beam: consider a small element, 
  look at the shear forces, bending moments and twisting moments on all the 
  edges, then take a limit as the element size tends to zero. We will not go 
  through that in detail here: instead we will give a plausibility argument 
  which leads to the correct equation, by analogy with the bending beam. Recall 
  from section 3.2.1 that the governing equation for beam vibration was 

  \begin{equation*}\rho A \dfrac{\partial^2 w}{\partial t^2}+EI 
  \dfrac{\partial^4 w}{\partial x^4}=0. \tag{1}\end{equation*} 

  \noindent{}for displacement $w(x,t)$. For the plate, we will have 
  displacement $w(x,y,t)$. What should we expect to be the equivalent of the 
  fourth derivative in $x$? A natural candidate is found by replacing 
  $\frac{\partial^2}{\partial x^2}$ with $\frac{\partial^2}{\partial 
  x^2}+\frac{\partial^2}{\partial y^2}$, an operator called the Laplacian that 
  is known to be invariant under change of coordinate system. 

  This leads us to expect an equation of the form 

  \begin{equation*}\rho h \dfrac{\partial^2 w}{\partial t^2}+EK 
  \left[\frac{\partial^2}{\partial x^2}+\frac{\partial^2}{\partial y^2} \right] 
  \left[\dfrac{\partial^2 w}{\partial x^2}+\dfrac{\partial^2 w}{\partial 
  y^2}\right]=0\tag{2}\end{equation*} 

  \noindent{}so that 

  \begin{equation*}\rho h \dfrac{\partial^2 w}{\partial t^2}+EK 
  \left[\frac{\partial^4 w}{\partial x^4}+2\frac{\partial^4 w}{\partial x^2 
  \partial y^2} +\frac{\partial^4 w}{\partial y^4} 
  \right]=0\tag{3}\end{equation*} 

  \noindent{}for a plate of thickness $h$, with some constant $K$. $E$ is 
  Young's modulus and $\rho$ is density as before: we are assuming for the 
  moment that our plate is made of an isotropic material like a metal, with the 
  same properties in all directions. Explicit calculation reveals that 

  \begin{equation*}K=\frac{h^3}{12(1-\nu^2)} \tag{4}\end{equation*} 

  \noindent{}where $\nu$ is Poisson's ratio. This Poisson's ratio factor arises 
  because of an assumption that the plate is in a state of plane stress, 
  because it is thin enough for any through-thickness stresses to relax to 
  zero. 

  We are interested in modes, so as usual we assume $w(x,y,t)=u(x,y) e^{i 
  \omega t}$. By far the simplest special case arises if we assume a 
  rectangular plate of dimensions $a \times b$, with simply-supported boundary 
  conditions on all four edges. This requires zero displacement, and zero 
  bending moment about each edge line. It is easily shown that eq. (3) and 
  these boundary conditions are all satisfied by the shapes 

  \begin{equation*}u_{nm}=\sin(n \pi x/a) \sin(m \pi y/b) 
  \tag{5}\end{equation*} 

  \noindent{}where $n$ and $m$ can take any integer values 1,2,3,... The 
  corresponding natural frequencies are given by 

  \begin{equation*}\omega_{nm} = \sqrt{\frac{EK}{\rho h}} \left[\frac{n^2 
  \pi^2}{a^2}+\frac{m^2 \pi^2}{b^2} \right] .\tag{6}\end{equation*} 

  Later, especially when we look at the soundboards of stringed instruments, we 
  will be interested in the vibration of wooden plates. Soundboards are usually 
  made of high-quality softwood: typically varieties of spruce or cedar. Such 
  materials are by no means isotropic: the Young's modulus along the grain of 
  the wood can be at least an order of magnitude greater than the Young's 
  modulus in the radial direction of the growing tree. That case is governed by 
  a similar equation, but instead of a single constant $E$ we will need 
  different constants for each of the three terms in the square bracket in eq. 
  (3). A fourth elastic constant may also enter the problem if we have more 
  complicated boundary conditions than the ones assumed here. 