  We saw a preliminary account of the excitation mechanism in a clarinet, back 
  in sections 8.5 and 8.5.3. That description was qualitatively correct, but no 
  attempt was made at that stage to find a physically-correct form for the 
  nonlinear valve characteristic of the mouthpiece and reed. Now we need to 
  remedy that omission, to obtain a simulation model with plausible functional 
  form and parameters that have a direct physical interpretation. 

  The situation is sketched in Fig.\ 1. The player provides a steady mouth 
  pressure $p_m$, while the pressure $p(t)$ just inside the mouthpiece is 
  varying in time: its evolution is the thing we are trying to simulate. 
  Normally, $p_m > p$ so that there is a pressure difference across the narrow 
  opening between reed and lay, through which air flows with a volume rate 
  $v(t)$. For the simplest model, we can picture this narrow opening as being a 
  rectangular slot of width $w$ and height $H$, so that the cross-sectional 
  area is $wH$. 

  \fig{figs/fig-83220222.png}{Figure 1. Sketch of the mouthpiece and reed.} 

  As the reed vibrates, $H$ will vary. It is convenient to write it as 
  $H=H_0-y(t)$ where $H_0$ is the initial gap set by the player, and $y(t)$ is 
  the inward displacement of the vibrating reed. But, as with other parameters 
  in this model, $H$ and $y$ should be regarded as effective values, not 
  necessarily exactly equal to distances you could directly measure. Some air 
  will flow through the tapering gaps on the sides of the reed, for example, 
  and this must be factored in to the effective area. Also, there is a fluid 
  dynamical phenomenon called the \tt{}vena contracta\rm{} that can make the 
  effective thickness of the air jet a bit less than the height of the opening. 

  Under the normal conditions of clarinet playing, it is a reasonable 
  approximation to assume that the simplest form of Bernoulli's law (see 
  section 11.2.1) applies to the air jet. This assumption gives us a simple 
  formula for the flow speed, and hence for the volume flow rate into the 
  mouthpiece: 

  $$v=-w(H_0-y) \sqrt{2 |\Delta p|/\rho_0} \mathrm{~sign} (\Delta p) \tag{1}$$ 

  where the pressure difference 

  $$\Delta p = p-p_m \tag{2}$$ 

  and the term $\mathrm{sign}(\Delta p)$ serves to guarantee that the flow 
  correctly reverses if the pressure difference should reverse. It might seem a 
  little perverse to define the pressure difference this way, so that it is 
  usually negative. But this form matches the way earlier plots were presented, 
  for example in section 8.5, and it will be convenient for the later 
  development of the simulation model. Note that in this derivation we have 
  implicitly assumed that the flow speed $v$ is very small compared to the 
  speed of sound. This is usually the case in normal woodwind playing, but in 
  some extreme cases the local flow speed can get very high, and the 
  approximation becomes questionable. 

  Now we need to think about the motion of the reed. The reed is essentially a 
  cantilever beam, set into vibration by the time-varying pressure difference 
  between the two faces. The reed will have many vibration resonances, of 
  course, but it is sufficient for our purpose to include just the first of 
  these. In that case, the reed displacement $y$ will behave like a damped 
  harmonic oscillator, with a governing equation of the form 

  $$M_r \ddot{y} + C_r \dot{y} + K_r y=-\Delta p \tag{3}$$ 

  where $M_r$ is the (effective) mass per unit area of the reed, $K_r$ is the 
  corresponding effective stiffness per unit area, and $C_r$ is a damping 
  coefficient per unit area. The resonance frequency $\omega_r$ will satisfy 

  $$\omega_r^2=K_r/M_r \tag{4}$$ 

  as usual. 

  This resonance frequency for a typical clarinet reed and embouchure has been 
  measured, and it is in the vicinity of 3~kHz. This means that playing 
  frequencies of clarinet notes are almost invariably far lower than the reed 
  resonance. In that case, the stiffness term of equation (3) will dominate, 
  and we obtain the approximate result 

  $$K_r y \approx -\Delta p . \tag{5}$$ 

  Eliminating $y$ in equation (1) then gives 

  $$v \approx -w\left(H_0+ \dfrac{\Delta p}{K_r}\right) \sqrt{2 \dfrac{|\Delta 
  p|}{\rho_0}} \mathrm{~sign} (\Delta p) . \tag{6}$$ 

  It is useful to introduce the closure pressure 

  $$p_c=K_r H_0 , \tag{7}$$ 

  which is the minimum mouth pressure needed to close the reed completely 
  against the lay, when the internal pressure $p=0$. We can then rewrite 
  equation (6) in the form 

  $$v \approx -u_A\left(1+ \dfrac{\Delta p}{p_c}\right) \sqrt{\dfrac{|\Delta 
  p|}{p_c}} \mathrm{~sign} (\Delta p) \tag{8}$$ 

  where the constant 

  $$u_A = w H_0\sqrt{\dfrac{2 p_c}{\rho_0}} =w H_0^{3/2}\sqrt{\dfrac{2 
  K_r}{\rho_0}} . \tag{9}$$ 

  Equation (8) is the nonlinear valve characteristic we are looking for. It is 
  valid while $\Delta p > -p_c$, but at that point the reed closes. Once 
  $\Delta p$ is more negative than this threshold, we have $v=0$. 

  The resulting nonlinear characteristic is plotted in the red curve of Fig.\ 
  2. The parameter values here are typical of a clarinet, according to the 
  measurements of Dalmont and Frappé [1]: width $w$=15~mm, initial reed gap 
  $H_0$=0.6~mm and closure pressure $p_c$=7~kPa. The corresponding reed 
  stiffness is $K_r$=12.7~kPa/mm. Dalmont and Frappé also show direct 
  measurements of the steady flow characteristics of a clarinet reed and 
  mouthpiece, demonstrating excellent agreement with the functional form of 
  equation (8) (despite the approximations we have used in order to derive this 
  equation). 

  \fig{figs/fig-f9e4a15a.png}{Figure 2. An example of the nonlinear valve 
  characteristic of equation (9)} 

  The dashed blue line in Fig.\ 2 shows a typical example of the sloping 
  straight line whose intersection with the red curve forms a key part of the 
  simulation model (see section 8.5.3 for the details). The slope of this line 
  is $1/Z_t$ where $Z_t=\rho_0 c/S$ is the characteristic impedance of the 
  tube. The cross-sectional area $S$ used here is based on the typical bore 
  diameter of a clarinet tube, 15~mm. Notice that this sloping line is 
  significantly steeper than the maximum slope of the red curve. We can deduce 
  that for this model clarinet there will be no equivalent of the ``Friedlander 
  ambiguity'' that we found for a bowed string, in which the corresponding 
  sloping line could sometimes intersect the nonlinear friction curve at three 
  points rather than one, leading to the ``flattening effect'' (see section 
  9.2). 

  For $\Delta p > -p_c$ we can rewrite equation (8) as 

  $$v \approx u_A\left(1+ \dfrac{p-p_m}{p_c}\right) \sqrt{\dfrac{p_m-p}{p_c}} 
  .\tag{10}$$ 

  Differentiating this equation gives 

  $$\dfrac{dv}{dp}= -\dfrac{u_A}{2p_c} \left(\dfrac{p_m-p}{p_c}\right)^{-1/2} 
  +\dfrac{3u_A}{2p_c} \left(\dfrac{p_m-p}{p_c}\right)^{1/2}$$ 

  $$=\dfrac{u_A}{2p_c} \left(\dfrac{p_m-p}{p_c}\right)^{-1/2} \left[-1 + 
  3\left( \dfrac{p_m-p}{p_c} \right) \right] . \tag{11}$$ 

  The maximum of the curve, where this slope is zero, is thus defined by 

  $$\dfrac{p_m-p}{p_c}=\dfrac{1}{3} \tag{12}$$ 

  as indicated by the dotted line in Fig.\ 2. At this value of $\Delta p$, the 
  value of the volume flow rate is 

  $$v_{max}=\dfrac{2u_A}{3 \sqrt{3}} . \tag{13}$$ 

  If we want a time-stepping simulation model which includes the reed 
  resonance, we need to go back to equations (1) and (3) to see how to proceed. 
  At each time step of the simulation, we first compute a new value for the 
  reed displacement $y$. The most efficient way to do this, based on the 
  history of the forcing term $-\Delta p$, is to use a recursive IIR digital 
  filter, as described in section 9.5.2. If this calculation gives a value $y > 
  H_0$, then in fact the reed closes at that moment and we use $y=H_0$ instead. 

  Now we use equation (1) with this value of $y$, which gives a different 
  nonlinear function of pressure $p$: the factor $(H_0-y)$ is now a known 
  constant, and only the square root term gives non-trivial variation with $p$. 
  We thus need to find the intersection of this new function with the sloping 
  straight line, to give the new values of $p$ and $v$: an example is shown in 
  Fig.\ 3. The rest of the simulation model is unchanged: we still use 
  convolution with a reflection function to model the linear acoustics of the 
  tube, and give a value of $p_h$ which determines the position of the sloping 
  straight line. 

  \fig{figs/fig-8e036b05.png}{Figure 3. Sketch of the graphical construction to 
  be used in place of the one shown in Fig. 2 when running a simulation 
  including the reed resonance. The sloping straight line is exactly as before, 
  with a position determined by $p_h$, while the red curve shows the function 
  given by equation (1) when the factor $(H_0-y)$ is a known constant.} 

  There is one final thing to mention. The flow of air through the reed gap 
  given by equation (1) is not the only contribution to the total volume flow 
  rate into the instrument. The reed also induces a ``sweeping flow'' component 
  by its own motion. For a reed of the kind we are looking at here, that 
  contribution to the inward volume flow rate is proportional to the reed tip 
  velocity $\dot{y}$, scaled by an ``effective reed area'' $A_r$, so that 
  equation (1) becomes 

  $$v=-w(H_0-y) \sqrt{2 |\Delta p|/\rho_0} \mathrm{~sign} (\Delta p)+A_r 
  \dot{y} . \tag{14}$$ 

  We will neglect this extra term for now, in the interests of simplicity, but 
  we will need to include it in later sections when we discuss some subtleties 
  of brass and free reed instruments. 

  \sectionreferences{}[1] Jean-Pierre Dalmont and Cyrille Frappé, “Oscillation 
  and extinction thresholds of the clarinet: Comparison of analytical results 
  and experiments”, Journal of the Acoustical Society of America \textbf{122}, 
  1173—1179 (2007). 