  A simple model for a transverse flute, to allow us to investigate the effect 
  of the cork position, follows straightforwardly from earlier calculations we 
  have seen in sections 11.1.1 and 11.1.3. The geometry of interest is sketched 
  in Fig.\ 1. Three cylindrical tubes meet at the point O, which we will use as 
  our origin of coordinates: an open-ended tube of length $L$ and 
  cross-sectional area $S$, a closed-ended tube of length $b$ and 
  cross-sectional area $S$, and a short open-ended tube of length $h$ and 
  cross-sectional area $C$ (representing the embouchure hole, including its end 
  corrections). 

  From section 11.1.1 we already know general expressions for the acoustical 
  pressure and volume flow rate at a frequency $\omega$, in any cylindrical 
  pipe. The total pressure $p$ in the pipe can be written 

  $$p=Ae^{i(\omega t -- kx)}+Be^{i(\omega t + kx)} , \tag{1}$$ 

  and the corresponding total volume flow rate is then 

  $$v=\dfrac{AS}{Z_0}e^{i(\omega t -- kx)}-\dfrac{BS}{Z_0}e^{i(\omega t + kx)} 
  \tag{2}$$ 

  where $S$ is the cross-sectional area, $Z_0$ is the characteristic impedance 
  for sound waves in air, and the wavenumber $k=\omega/c$ where $c$ is the 
  speed of sound. Distance along the pipe is measured by $x$, and $t$ is time 
  as usual. 

  The coefficients $A$ and $B$ describe (complex) amplitudes of waves 
  travelling in the positive and negative $x$ directions, respectively. We can 
  use corresponding expressions for our three tubes. We can measure $x$ 
  outwards from the origin O along each of the three tubes, and we can use 
  coefficients $(A_1,B_1)$, $(A_2,B_2)$ and $(A_3,B_3)$ respectively, as 
  labelled in Fig.\ 1. For the embouchure tube we replace $S$ in equation (2) 
  by $C$. 

  We now need 6 equations for these 6 unknown wave amplitudes. At the open end 
  of the main tube we require $p=0$ at $x=L$, so 

  $$A_1 e^{-ikL} + B_1 e^{ikl} =0. \tag{3}$$ 

  At the closed end of the shorter tube we require $v=0$ at $x=b$, so 

  $$A_2 e^{-ikb} -- B_2 e^{ikb} =0. \tag{4}$$ 

  At the junction, we require several things. The pressures in all three tubes 
  must be the same at $x=0$, so that 

  $$A_1+B_1 = A_2 + B_2 = A_3 + B_3 . \tag{5}$$ 

  Also, the combined volume outflow rate from O must be zero, so that 

  $$\dfrac{C}{Z_0}(A_3 -- B_3) + \dfrac{S}{Z_0}(A_1 -- B_1 + A_2 -- B_2) = 0 . 
  \tag{6}$$ 

  Finally, our aim is to calculate the input admittance at the open end of the 
  short tube representing the embouchure hole. We can impose unit pressure at 
  this point by setting $p=1$ at $x=h$ so that 

  $$A_3 e^{-ikh} + B_3 e^{ikh} =1, \tag{7}$$ 

  and then the admittance $Y$ is given by the value of volume flow rate at this 
  point: 

  $$Y=\dfrac{C}{Z_0}\left( -A_3 e^{-ikh} + B_3 e^{ikh} \right) \tag{8}$$ 

  where a negative sign has been introduced because for input admittance the 
  relevant volume flow rate is into the tube, not out of it. 

  After some straightforward but tedious algebra, the linear simultaneous 
  equations (3)--(7) can be solved to give the admittance in the form 

  $$Y = \dfrac{S (\cot kL -- \tan kb ) -- C \tan kh}{i Z_0 \left[ \tan kh (\cot 
  kL -- \tan kb ) +1 \right]} . \tag{9}$$ 

  In order to plot the admittance shown in Figs.\ 7, 8 and 9 of section 11.8, 
  some parameter values were estimated to give a reasonable match to the 
  measured admittance. The length $L$ was 0.6 m, the cork position was chosen 
  so that $b$ was 18 mm, the main tube was assumed to be circular with internal 
  radius 10 mm, the embouchure hole was taken to be circular with radius 5 mm, 
  and the effective length $h$ was 12 mm. In addition, the values of $k$ used 
  to compute the admittance were complex, allowing for wall damping in the tube 
  using the model due to Fletcher as described in section 11.1.1. 

  In order to plot the pressure distributions shown in Fig.\ 10 of section 
  11.8, the analysis here was used in a slightly different way. At each chosen 
  frequency, the simultaneous equations (3)--(7) were solved to obtain values 
  for the coefficients $A_1, B_1, A_2$ and $B_2$. These were substituted in the 
  relevant versions of equation (1) to give the pressure distributions in the 
  long tube and the closed tube. These pressures are, of course, complex. For 
  simplicity, the plots show the imaginary part of this complex pressure, which 
  contains the dominant component. 