  We have already seen a derivation of the modal density for bending modes of a 
  plate, in section 3.2.4. But there is a neat argument due to Weinreich [1] 
  which shows how this result can be generalised to a range of systems of 
  interest to us. Figure 1 shows a repeat of Fig.\ 1 from section 3.2.4, 
  indicating the modes of a rectangular plate in 2D wavenumber space: they mark 
  out a regular grid. But something very similar can be said for many other 
  systems. The modes of a stretched string or a pinned-pinned bending beam 
  correspond to points that are regularly spaced in wavenumber, along a line 
  (i.e. in 1D wavenumber space). An acoustic volume, on the other hand, has 
  points that form a regular grid filling 3D wavenumber space, according to eq. 
  (6) of section 4.2. 

  \fig{figs/fig-3d9226ce.png}{\caption{Figure 1. Modes of a rectangular plate 
  of dimensions $0.6 \times 0.4$ m, plotted in wavenumber space (red stars). 
  The blue curve encloses the set of modes with natural frequencies below a 
  chosen frequency.}} 

  If we now want to ask how many modes have natural frequencies below some 
  chosen value $\omega$, we first convert this into a limit of wavenumber $k$ 
  using the equation of motion of the system in question. For any system 
  satisfying the wave equation, in 1D, 2D or 3D, the answer is that wavenumber 
  is simply proportional to frequency. But for bending problems, either for 
  beams or for plates, the relation is that frequency is proportional to the 
  square of wavenumber. So for all these systems, we can say that 

  \begin{equation*}k \propto \omega^\alpha \tag{1}\end{equation*} 

  \noindent{}where $\alpha=1$ for the wave equation and $\alpha = 1/2$ for 
  bending problems. 

  Now to find out the number of modes below $\omega$ we have to count the 
  number of points on the wavenumber grid lying within a distance $k$ of the 
  origin. The points are uniformly distributed over the line/plane/space, so 
  for high frequencies this mode count is approximately proportional to the 
  enclosed length/area/volume, depending on whether the dimension $d$ is 1, 2 
  or 3. The result for the mode count $N(\omega)$ is that 

  \begin{equation*}N(\omega) \propto k^d \propto \omega^{\alpha d} . 
  \tag{2}\end{equation*} 

  Now the modal density $n(\omega)$, which is the inverse of the mean spacing 
  between adjacent modes, is given by the derivative of $N(\omega)$, so that 

  \begin{equation*}n(\omega) \propto \omega^{\alpha d -1} . 
  \tag{3}\end{equation*} 

  The results can be collected in a table: 

  A stretched string and an acoustic pipe both have constant modal density on 
  average, and they share this behaviour with a bending plate. A bending beam 
  has modal density that decreases as frequency increases. A membrane (like a 
  drum or the head of a banjo) has modal density increasing proportional to 
  frequency, and an acoustic volume has modal density increasing proportional 
  to the square of frequency, as seen in Fig.\ 17 of section 4.2. 

  The strictly regular grid in wavenumber space corresponds to particular 
  special cases for each type of system. However, as was argued for the plate 
  in section 3.2.4, the fact that the points cover the relevant wavenumber 
  space in a statistically uniform manner remains true for the general case of 
  all these systems. 

  \sectionreferences{}[1] G. Weinreich, section 3.1.7 of ``Mechanics of musical 
  instruments'', ed. A. Hirschberg, J. Kergomard and G. Weinreich; CISM Courses 
  and Lectures no. 355, Springer-Verlag (1995) 