  Consider a plane wave at frequency $\omega$ propagating on an infinite 
  isotropic plate and driving the motion of air on one side of the plate. The 
  plate motion is governed by this equation, from section 3.2.3: 

  \begin{equation*}\rho h \dfrac{\partial^2 w}{\partial t^2}+EK 
  \left[\frac{\partial^4 w}{\partial x^4}+2\frac{\partial^4 w}{\partial x^2 
  \partial y^2} +\frac{\partial^4 w}{\partial y^4} 
  \right]=0\tag{1}\end{equation*} 

  \noindent{}for a plate of thickness $h$, with a constant $K$ given by 

  \begin{equation*}K=\frac{h^3}{12(1-\nu^2)} \tag{2}\end{equation*} 

  \noindent{}where $E$ is Young's modulus, $\rho$ is density and $\nu$ is 
  Poisson's ratio. If the wave takes the form $w=e^{i \omega t -- i k x}$ then 
  eq. (1) requires 

  \begin{equation*}EK k^4=\rho h \omega^2. \tag{3}\end{equation*} 

  The speed of the wave will be given by 

  \begin{equation*}\frac{\omega}{k} = \left( \dfrac{EK}{\rho h} \right)^{1/4} 
  \sqrt{\omega} ,\tag{4}\end{equation*} 

  \noindent{}rising proportional to the square root of frequency. 

  Now pressure in the air above the plate will satisfy a two-dimensional 
  version of the wave equation: 

  \begin{equation*}\frac{\partial^2 p}{\partial t^2}= c^2 \left[ 
  \frac{\partial^2 p}{\partial x^2} + \frac{\partial^2 p}{\partial z^2} \right] 
  \tag{5}\end{equation*} 

  \noindent{}where $z$ is the coordinate normal to the plate. The pressure will 
  vary harmonically in time at the same frequency as the plate wave, and it 
  must have a wavenumber in the $x$ direction equal to $k$, for continuity with 
  the plate motion. What remains is to find out what happens in the $z$ 
  direction. We can look for a solution of the form 

  \begin{equation*}p=p_n(z) e^{i \omega t -i k x} . \tag{6}\end{equation*} 

  Equation (5) then requires 

  \begin{equation*}\omega^2 = c^2\left( k^2 -\frac{d^2 p_n}{d z^2} 
  \right)\end{equation*} 

  \noindent{}so that 

  \begin{equation*}\frac{d^2 p_n}{d z^2} + (k_a^2-k^2) p_n =0 
  \tag{7}\end{equation*} 

  \noindent{}where $k_a=\omega/c$ is the wavenumber of sound waves at frequency 
  $\omega$. 

  The solutions of eq. (7) depend on the sign of the bracketed term. If $k_a > 
  k$, the term is positive and solutions are sinusoidal. In that case, waves 
  propagate away from the plate at an angle governed by the ratio of $k$ to 
  $\sqrt{(k_a^2-k^2)}$. If, on the other hand, $k_a < k$ then the bracketed 
  term is negative, and the solution of interest is an exponential decay of 
  sound pressure away from the plate. There is no propagating wave: the sound 
  field is evanescent. Examples of these two cases were illustrated with 
  animations in Figs.\ 11 and 12 of section 4.3. 

  The crossover between the two regimes occurs when $k_a = k$, in other words 
  when 

  \begin{equation*}\frac{\omega_c}{c} = \left( \dfrac{\rho h}{EK} \right)^{1/4} 
  \sqrt{\omega_c} \tag{8}\end{equation*} 

  \noindent{}from eq. (3). It follows that this critical frequency is given by 

  \begin{equation*}\omega_c = c^2 \left( \dfrac{\rho h}{EK} \right)^{1/2} . 
  \tag{9}\end{equation*} 