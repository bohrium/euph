

  It is time to return to the difficult questions raised in Section 6.1. 
  Ultimately, the quality of a musical instrument is not determined by 
  acoustical measurements or perceptual thresholds: judgements by people are 
  what matter. There are psychoacoustical techniques for putting such 
  judgements on a quantitative and verifiable footing, but the experimenter is 
  entering a minefield. The experiments are hard to design and carry out. The 
  requirement for sufficient data to obtain statistically significant 
  conclusions places one set of constraints, while the effort to identify 
  possible sources of bias and then work around them gives another, even more 
  challenging, set. 

  Meanwhile, the experimenters may have to brace themselves for a storm of 
  protest at the results. There are pieces of received wisdom, “well-known 
  facts”, surrounding many types of musical instrument. There is quite often a 
  “myth of a golden age” of classic, never-equalled instruments, whether it 
  concerns violins made by Antonio Stradivari or Gibson mandolins signed by 
  Lloyd Loar. Of course, these are precisely the things we may be most 
  interested to explore with scientific rigour. But there will be some who 
  believe in this received wisdom with quasi-religious fervour, and they often 
  react with vociferous outrage if the test results do not agree with their 
  prior views. 

  The experimenters then need to tread a very careful line. Musical judgements 
  are subtle, and it is always likely that there is at least a grain of truth 
  in the received wisdom. If a first attempt to test one of these ideas finds 
  no evidence, then good experimenters will try to set aside the aggressive 
  tone of the criticism that comes their way but listen very carefully to the 
  ideas that are suggested about what was wrong with their experiment. 
  Scientific progress often requires patience and persistence, and they may 
  design a second version of the experiment in the light of comments, and see 
  if that gives a different result. And so on… But if a sequence of such 
  efforts still fails to give clear evidence in favour of the received wisdom, 
  things begin to look black for the traditionalists. 

  We will look at an example where a sequence of studies illustrates this 
  process, but before that we look at a simpler example [1] — although not 
  without some controversy. Traditional acoustical guitars, whether classical 
  or steel-strung folk-style, have favoured the use of tropical hardwoods to 
  make the back and sides of the soundbox. A particular favourite is Brazilian 
  rosewood, Dalbergia nigra. Nowadays any use of tropical hardwoods is viewed 
  with suspicion, and in any case many species, including Brazilian rosewood, 
  are now on the CITES list which places very severe limits on any 
  international trade. Guitar makers are exploring many alternative materials, 
  and it is obviously of interest to them to know whether they will be losing 
  out in terms of sound if they stop using the traditional timber species. 

  To do a systematic experiment, you first need to involve a friendly guitar 
  maker with sufficient experience to make a set of instruments like the ones 
  illustrated in Figs.\ 1 and 2. The six guitars seen here each use a different 
  timber for the back and sides, while everything else was matched as closely 
  as possible across the whole set. The selected timbers have a variety of 
  appearances, as can be seen clearly in Fig.\ 2. They also cover a wide range 
  of price and sustainability credentials. 

  \fig{figs/fig-22c245ad.png}{\caption{Figure 1. The six guitars for the 
  back-wood experiment. Image copyright Michael English, reproduced by 
  permission.}} 

  \fig{figs/fig-b272cbce.png}{\caption{Figure 2. Back view of the six guitars, 
  showing the different timbers. Image copyright Michael English, reproduced by 
  permission.}} 

  A large number of experienced guitarists were recruited to perform playing 
  tests of two different types. The tests were “blinded”: a lot of care was 
  taken to make it hard for the player to identify the guitars other than 
  through playing and listening. Following a procedure pioneered in the violin 
  studies to be described a little later, the players wore welder’s goggles and 
  performed in a dimly-lit space. The idea was to allow them to see just enough 
  to be able to handle the instruments safely, but not to be able to recognise 
  the back wood despite the strong differences of colour and pattern revealed 
  in Fig.\ 2. 

  In the first test, the players were given each guitar in turn, given time to 
  explore and get used to it, then asked to give numerical ratings for various 
  qualities: “overall sound”, playability, and then a list of specific 
  qualities like brightness, warmth and richness. Some of the guitarists 
  repeated the whole test after having a rest, to allow the consistency of 
  their judgements to be checked. The second experiment was an “ABX” test. The 
  player would be handed guitar “A”, given a bit of time with it, then it would 
  be swapped for guitar “B”. Finally, they would be handed one or other of 
  these guitars, and asked to decide whether it was A or B. All six guitars 
  were used for the rating test, but because of constraints on time only three 
  were used for the ABX test (selected to represent the range of price and 
  sustainability). 

  The results of both tests were subjected to a battery of statistical 
  analysis: the details are given in [1]. The result? The players were not 
  convincingly able to distinguish the guitars at better than chance levels, in 
  either test. Now, we need to be careful in interpreting a statement like 
  this. The experiment does not prove that no player under any circumstances 
  would be able to discriminate between any of these instruments. But it does 
  say that any such discriminatory ability is subtle or rare, to the extent 
  that it was not revealed convincingly by this fairly careful test involving 
  53 different guitarists. 

  However, this is not quite the end of the story. We already saw, in section 
  6.5, that estimates have been made of the “ultimate” threshold for 
  discriminating guitars on the basis of changes to the body modes. These 
  estimates were made using synthesised sounds based on measured body 
  behaviour. The same approach was applied to the six guitars: acoustical 
  measurements were made of them all, in the same way described earlier (see 
  section 5.1), and then snatches of music were synthesised on them all. Those 
  synthesised sounds could be used for another version of the ABX test, but 
  this time we have a clear expectation about whether differences should be 
  perceptible or not. 

  The set of measured bridge admittances is shown in Fig.\ 3. Also included is 
  a guitar of generically similar kind, but not part of the set of six. The 
  plot shows that the set of six were all clearly different from the extra 
  instrument shown in the blue curve, but rather similar to each other: a 
  tribute to the skills of the guitar maker. For the purposes of the ABX 
  listening test, the measured admittances were not used directly. Instead, 
  attention was focussed purely on the three “signature modes” giving strong 
  peaks in the frequency range below 400 Hz. We have already seen in section 
  5.3 what the mode shapes look like corresponding to these peaks. The 
  frequency, amplitude and Q factor of these three modes was extracted from the 
  measurements. A reference guitar response was chosen, and then modified to 
  produce six versions matching the signature modes of the six guitars, while 
  all other details were the same. These were used to generate synthesised 
  sound files. 

  \fig{figs/fig-570ca900.png}{\caption{Figure 3. Bridge admittances of the six 
  guitars (red). For comparison, the admittance of an unrelated guitar of the 
  same general type is shown in blue.}} 

  The thresholds found in the earlier study [2] give clear predictions of 
  whether the differences in signature modes are sufficient to be audible. The 
  result is that the extra guitar should be clearly different from any of the 
  set of six, but that differences among the set were quite small and barely 
  above the threshold of perception. Among the six, there were two that stood 
  out with signature mode frequencies that were systematically about half a 
  semitone lower and higher than the average for the group: these happened to 
  be (respectively) the guitars built using Sapele and Indian rosewood for the 
  backs and sides. These represent the extremes, and the difference between 
  them should be big enough to be perceptible by a skilled listener. 

  The formal listening test confirmed these predictions. The extra guitar could 
  be distinguished from the others reliably. The Indian rosewood and Sapele 
  guitars could be distinguished from each other, but not as reliably. The 
  other members of the set of six could not be distinguished above chance 
  levels. You can hear some of the synthesised sounds in Sounds 1, 2 and 3. The 
  music is a short extract from the tune “Tears in heaven” (a favourite of one 
  of the students who did the work for the threshold project [2]). 

\audio{}

\audio{}

\audio{}

  There is a final twist to the story. Do we conclude that Indian rosewood and 
  Sapele backs and sides will produce different-sounding guitars? No, that 
  would probably be misleading. Looking back at the discussion of signature 
  modes of guitars in section 5.3, we can see that while the first two of these 
  modes might plausibly be affected by the behaviour of the back, this is not 
  the case for the highest of the three modes. That mode has motion largely 
  confined to the top plate, and it produces virtually no net volume change so 
  that it should not couple very well to back plate motion. But for the two 
  guitars in question, all three of these modes were a little higher or a 
  little lower than the average, by a similar factor. 

  That suggests that the origin of the difference on which the perceptual 
  result was based does not lie in the backs at all. More likely, it is due to 
  small differences in the top plates (including the bridge, and the braces 
  glued to its underside). The guitar maker has done a masterly job in using 
  the six different back woods to make very similar guitars, despite 
  significant differences in density and stiffness properties of the woods. We 
  should not be too surprised: instrument makers are used to the variability of 
  wood, and part of their skill set is to know how to compensate for those 
  variations to produce a consistent end product. We should also note that this 
  skilled guitar maker is still convinced that back wood does make a difference 
  to the sound. The results show that any such influence is subtle, and most 
  players are incapable of hearing it during normal playing, but we should 
  certainly not dismiss his opinion hastily. But nevertheless it is clear that 
  guitar makers can safely experiment with more sustainable materials, without 
  fear of acoustical disaster. 

  We now turn to a more controversial subject. The popular perception of a 
  “secret of Stradivari” is very widespread, and the pattern of market values 
  supports the idea that there is something special about certain old Italian 
  violins. There is a long history of public “tests” in which a Stradivari and 
  something else was played behind a curtain, and the audience asked to vote on 
  which is better. None of this would qualify as serious science, which 
  requires double-blind testing, sufficient test subjects and repeat tests, and 
  results verifiable by independent researchers. 

  Only in recent years have convincing experiments begun to be performed. The 
  first experiment took advantage of the availability of high-class instruments 
  and violinists at a violin competition, in Indianapolis in 2010 [3]. It 
  involved six violins: two by Stradivari, one by Guarneri “del Gesu”, and 
  three by contemporary makers. These were used for playing tests that were 
  somewhat similar to the guitar tests just described. The experiment needed to 
  allow players to handle the different instruments under test in a safe and 
  natural way, but without being able to see which was which and thus bring in 
  additional information and bias. This is where the dim lighting and welder’s 
  goggles come in: that procedure was first used in this experiment. The 
  goggles can be seen in action in Fig.\ 4, showing one of the participants in 
  the second experiment (to be described shortly). 

  \fig{figs/fig-4045aae8.png}{\caption{Figure 4. Violinist Ilya Kaler taking 
  part in the second experiment. Image copyright Stefan Avalos, reproduced by 
  permission.}} 

  The experimenters took the very reasonable view that the most acute 
  discrimination between instruments is likely to come from players, rather 
  than from external listeners, however expert. There is a simple reason for 
  this, which we will explore in some detail in Chapter ?: the player is inside 
  a feedback loop, able to adjust details of bowing to try to create the sound 
  they want, but the listener only hears the end result. A skilled player can 
  coax a good sound, at least on certain notes, from more or less any violin, 
  but the player will still be well aware that they have to try a lot harder on 
  some violins than others to get this effect. 

  The players were given several different tasks: to choose their favourite and 
  to rate the instruments against one another based on various different 
  criteria. The results were a surprise, not least to the experimenters 
  themselves. At that time most people expected to find some degree of 
  preference for the famous old instruments: the debate was about how big the 
  difference would prove to be. But the cautious conclusion of the authors 
  after this first experiment was that no statistically significant difference 
  between the two groups, old and modern, was found in terms of preference. In 
  fact the raw data suggested that certain of the modern instruments were 
  slightly preferred to any of the old ones. 

  These results caused a media storm. Some famous names weighed in with 
  negative comments. Among the invective, they raised a number of perfectly 
  reasonable objections to the details of the experiment. The tests were 
  carried out in a hotel room, whereas the natural home of the great old 
  instruments is the concert hall. There was no possibility to try the 
  instruments in an ensemble or with an accompaniment. There weren’t very many 
  instruments, and the set of participating violinists were rather a mixed bag 
  in terms of standard and experience. 

  To address some of these objections, the team organised a second experiment 
  involving 10 first-rate soloists in a rehearsal room and then in a concert 
  hall with an option on piano accompaniment [4]. This time, six old Italian 
  violins (including five by Stradivari) and six modern instruments were pitted 
  against each other. A careful process of selection was used for both groups 
  of instruments, choosing the preferred ones from large initial pools (the 
  authors have kept the exact identities of all instruments secret). 

  This second experiment was more carefully conducted than the first, and the 
  results were in complete agreement with the previous findings. They also gave 
  some additional information. When asked to choose a violin that might 
  plausibly replace their own for an upcoming tour, six of the soloists chose 
  new violins and four chose Stradivaris. A single new violin was easily the 
  most-preferred of the 12. On average, soloists rated their favourite new 
  violin more highly than their favourite old one for playability, 
  articulation, and projection, and at least equal in terms of timbre. Finally, 
  the 10 soloists failed to distinguish new from old at better than chance 
  levels. 

  The doubters were still not convinced. This time, they raised an objection 
  based on a long-held belief about “projection”. It is often claimed that the 
  classic Italian instruments can be quiet under the ear, but that their 
  audibility improves for a distant listener, out in the concert hall. To 
  explore this idea, a third experiment was organised: in fact, two versions of 
  the experiment were run, in concert halls in Paris and New York [5]. In each 
  case, three contemporary violins were pitted against three by Stradivari. In 
  the Paris experiment, the instruments were tested both with and without 
  orchestral accompaniment. 

  The key difference in these experiments was that the audience out in the 
  concert hall rated the various instruments. There were 55 and 82 
  participating listeners in the Paris and New York experiments, respectively. 
  Figure 5 shows a view of the scene during the New York experiment. Pairs of 
  instruments were played, and the listeners had to vote which was heard 
  better, and which was preferred. Another stage of the experiment involved 
  presenting new/old pairs of instruments, and asking the listeners to decide 
  which was which. 

  \fig{figs/fig-d2931578.png}{\caption{Figure 5. A scene in the New York phase 
  of the third experiment. Image copyright Hubert Raguet, reproduced by 
  permission.}} 

  The results were very clear. The listeners preferred the new violins over the 
  old, and also found that the new violins projected better. Results for 
  projection with and without the orchestral accompaniment were strongly 
  correlated. Furthermore, preferences expressed by the audience were in good 
  agreement with those from the players. Finally, the audience members could 
  not distinguish new from old instruments at better than chance levels. 

  So there seems to be no secret of Stradivari. The best of the classic Italian 
  instruments are still very good, of course, but the best of contemporary 
  instruments can hold their own when given a level playing field on which to 
  compete, and even sometimes out-perform the old instruments. Here is what the 
  authors wrote after the third experiment: “A belief in the near-miraculous 
  qualities of Old Italian violins has preoccupied the violin world for 
  centuries. It may be that recent generations of violin-makers have closed the 
  gap between old and new, or it may be that the gap was never so wide as 
  commonly believed. Either way, the debate about old versus new can perhaps be 
  laid aside now in favor of potentially more fruitful questions. What, for 
  example, are the physical parameters determining the playing qualities of any 
  violin, regardless of its age or country of origin?” [5]. 



  \sectionreferences{}[1] S. Carcagno, R. Bucknall, J. Woodhouse, C. Fritz, and 
  C. J. Plack . ``Effect of back wood choice on the perceived quality of 
  steel-string acoustic guitars.'' Journal of the Acoustical Society of America 
  \textbf{144}, 3533-3547, (2018).~ DOI 10.1121/1.5084735. The article may be 
  found here: \tt{}https://doi.org/10.1121/1.5084735\rm{} 

  [2] J. Woodhouse, E. K. Y. Manuel, L. A. Smith, A. J. C. Wheble and C. Fritz. 
  ``Perceptual thresholds for acoustical guitar models''. Acta Acustica united 
  with Acustica\textbf{ 98}, 475-486, (2012).~ DOI 10.3813/AAA.918531. 

  [3] C. Fritz, J. Curtin, J. Poitevineau, P. Morrel-Samuels and F.-C. Tao. 
  ``Players preferences among new and old violins''; Proceedings of the 
  National Academy of Sciences of the USA 1\textbf{09}, 760-763 (2012). 

  [4] C. Fritz, J. Curtin, J. Poitevineau, H. Borsarello, I. Wollman, F.-C. Tao 
  and T. Ghasarossian. ``Soloist evaluations of six Old Italian and six new 
  violins''; Proceedings of the National Academy of Sciences of the USA 
  \textbf{111}, 7224-7229 (2014). 

  [5] C. Fritz, J. Curtin, J. Poitevineau and F.-C. Tao. ``Listener evaluations 
  of new and Old Italian violins''; Proceedings of the National Academy of 
  Sciences of the USA \textbf{114}(21):2016194439 (2017). DOI 
  \tt{}10.1073/pnas.1619443114\rm{}. 