

  In this section, we will first look at an instrument which is a bit of cheat. 
  The ``musical saw'' is ordinarily played with a violin bow, but its behaviour 
  when tapped gives a very illuminating insight into the physics of a 
  particular kind of shell vibration, which we will then apply to a genuine 
  tuned percussion instrument, the Caribbean steel pan. 

  The first demonstration is well worth trying for yourself. Take an ordinary 
  carpenter's hand-saw. Sit down comfortably, then brace the handle of the saw 
  with your knees, grip the narrow end of the saw in one hand, and bend it so 
  that it takes up an S-shape: see Fig.\ 1 for the general idea. Now with 
  something like a pencil in your other hand, tap the saw right in the middle 
  of the S. If you have bent it by a suitable amount, you should hear a 
  surprisingly long-ringing sound, with a note that you can change by adjusting 
  the severity of the S-bend. But now keep hold of the saw and let it 
  straighten out: if you tap again you will hear only a dull thud or rattling 
  sound. In case you can't try this yourself, Sound 1 gives a simple recorded 
  example: I first tap the saw without bending it, then again with the S-bend 
  in place. 

  \fig{figs/fig-1c087951.png}{\caption{Figure 1. How to hold a musical saw.}} 

  \aud{auds/aud-6668735d-plot.png}{\caption{Sound 1. The sound of tapping the 
  saw without the S-bend, then with it. Notice the changing pitch in the later 
  sounds: the note can be modulated by changing the bend details.}} 

  What is going on? How do we create this clear ringing note seemingly out of 
  nowhere by curving the saw-blade like this? Figure 2 shows a laboratory test, 
  which allows the pattern of vibration to be visualised using a holographic 
  method. The ``saw'' is being held in the required S-shape by some clamps, and 
  then it is being made to vibrate at the resonant frequency we heard in the 
  pencil-tapping experiment, using a vibration actuator which is out of sight 
  behind the saw in the picture. Using a cunning trick involving a laser (which 
  is where the green colour comes from), the dark curves show a contour map of 
  the vibration amplitude. The next link describes how it is done. 

  \fig{figs/fig-f1aea612.png}{\caption{Figure 2. A holographic contour map of 
  the vibration of a musical saw. Image copyright Bernard Richardson, 
  reproduced by permission.}} 

  The two broad white bands running along the length of the saw are nodal 
  lines, where the vibration goes to zero. So we can work out what is 
  happening. If you concentrate on a strip across the width of the saw in the 
  vibrating region, the motion is just like the lowest mode of the free-free 
  bending beam see in Fig.\ 1 of section 3.2: down in the middle and up on both 
  edges, then reversing. Motion of this kind is happening over a chunk in the 
  middle of the saw, but then it mysteriously fades away as you move towards 
  either end. 

  To understand this fading, we need to remember the discussion of corrugated 
  sheet in section 3.2. The effect of the curvature was to make the sheet much 
  stiffer when you bend across the corrugations, compared to bending parallel 
  to the corrugations. Now recall that our saw is in an S-shape. It has an 
  inflection in the middle of the S, where the curvature is zero, but as you 
  move away from that point in either direction you encounter stronger and 
  stronger curvature. This region around the inflection is precisely where the 
  vibration is occurring in Fig.\ 2. 

  We can imagine setting up vibration near the inflection, with the pattern of 
  free-free bending across the width of the saw. Now imagine this pattern of 
  vibration trying to propagate along the saw, away from the inflection. It 
  encounters the growing curvature, and this increases the stiffness which 
  produces the restoring force for the bending vibration. Eventually, it can 
  reach a point where this curvature stiffening effect gets so strong that it 
  is no longer possible to vibrate across the width at this frequency. Our 
  travelling vibration pattern cannot get past this point: it experiences a 
  kind of total internal reflection, and is turned round and travels the other 
  way. It goes back to the inflection, then out on the other side. But it meets 
  increasing curvature again in that direction, until it reaches another point 
  of total internal reflection. So the vibration is confined to this region 
  around the inflection, bounded by the two points where the curvature reaches 
  a critical value and causes reflection. That is the pattern in Fig.\ 2. This 
  kind of system is known generically as a waveguide. The next link uses a 
  waveguide system that can be analysed more readily, to illustrate the 
  behaviour in quantitative form. 

  To see why the reflection from points of critical curvature accounts for what 
  we heard in the pencil-tapping experiment, we have to think about vibration 
  damping. A saw is made of high-carbon steel, which has very low vibration 
  damping. But we were holding the saw blade with our fingers, and at the other 
  end the steel blade was rivetted to a wooden handle. The handle and your 
  fingers both give efficient ways to dissipate any vibrational energy in the 
  saw. So when you tap it without the S-bend, you hear a dull thud or a rattle 
  from the handle. But now bend into the S shape. A vibration mode like Fig.\ 2 
  is created, confined to the middle region of the saw blade. Now the 
  vibrational energy can't reach your fingers or the handle, and the sound you 
  hear is characterised by the very low intrinsic damping of the steel. The 
  result is the long-ringing note we heard. 

  This idea of vibration being confined to a region of a curved shell through 
  the stiffening effects of curvature will allow us to understand the physics 
  behind our next instrument, the Caribbean steel pan. Steel pans are made from 
  mild steel, traditionally using an old oil drum. The maker shapes the top 
  with the hammer, and creates a pattern of curvature which not only gives a 
  musical note, but somehow allows many different notes to be played on the 
  same piece of steel. It is a truly remarkable piece of vibration engineering. 

  The first step in the process of turning an oildrum into a steel pan is to 
  ``sink'' the top. Mild steel is quite ductile, so by working hard enough with 
  a hammer it is possible to turn the flat top of the drum into a bowl shape. 
  By this stage, the ductility of the steel is likely to have been used up, 
  though the process of strain hardening. To restore some ductility for the 
  next step, the steel needs to be annealed, in other words heated to a 
  suitable temperature and allowed to cool slowly. The traditional way to do 
  this involves a bonfire containing old tyres: the carbon from the burning 
  tyres helps with controlling the properties of the steel. 

  \fig{figs/fig-ad28e75f.png}{\caption{Figure 3. Steel pan tuner Dudley Dickson 
  at work, in 1987. He is working on one pan with his hammer, and he has a 
  finished pan by his elbow to act as a pitch reference.}} 

  The next step is to mark out where the various notes are to be on the pan: 
  this depends on the type and register of the particular pan being made. Now 
  the clever part begins. Each note area is worked with the hammer, starting 
  out on the reverse side in order to undo the bowl curvature from the sinking 
  stage. The result is that each note is a flattish area, surrounded by curved 
  steel in the rest of the bowl. That is the secret: it is a kind of 
  two-dimensional version of the musical saw, and it can work in a similar way 
  to allow some modes of vibration to be confined to single patches on the pan 
  surface. 

  This confinement effect is dramatically illustrated in the set of holographic 
  images in Fig.\ 4. The approximately circular area at the bottom of the pan 
  is the note $A_3$, with nominal frequency 220 Hz. In the first picture we can 
  see that there is indeed a confined mode close to this frequency. But there 
  is more: another mode at roughly double that frequency, with a single node 
  line, another at the next doubling of frequency, with two node lines, and in 
  between a mode with a frequency close to 2.5 times the fundamental, and 
  musically consonant with it. The response near the double-octave frequency, 
  in the lower right pane of Fig.\ 4, shows activity in two other areas as 
  well. This is no mistake: these are the notes $A_4$ and $A_5$, tuned in their 
  own patches, sufficiently accurately that they all resonate with each other 
  at a frequency around 880 Hz. 

  \fig{figs/fig-16fbaffe.png}{\caption{217 Hz}} 

  \fig{figs/fig-4bfe9168.png}{\caption{441 Hz}} 

  \fig{figs/fig-16ae6e89.png}{\caption{540 Hz}} 

  \fig{figs/fig-b77202bc.png}{\caption{886 Hz}} 

  A similar pattern is seen in Fig.\ 5, with a sequence of confined modes 
  involving the $A_4$ area. But by the highest frequency illustrated here, at 
  1822 Hz, something different has happened. We now see vibration spread over 
  much of the surface of the pan, rather than being confined to one area or a 
  small number of related areas. This is exactly what we would expect from the 
  description, and the theory presented in section 3.5.1. The bowl curvature 
  presents a barrier to bending waves, but as you go higher in frequency and 
  shorter in wavelength, that barrier becomes less effective. The situation in 
  the pan is, in this respect, different from the musical saw or the idealised 
  beam on an elastic foundation. In those problems, curvature or foundation 
  stiffness increased inexorably with distance from the neutral point. So at 
  any frequency there would eventually be a point where internal reflection had 
  to occur. But in the pan, the bowl curvature is fixed. Once the wavelength is 
  short enough to overcome that barrier, the confinement effect is lost. 

  \fig{figs/fig-814e730c.png}{\caption{424 Hz}} 

  \fig{figs/fig-f4fa8f99.png}{\caption{886 Hz}} 

  \fig{figs/fig-50bd9330.png}{\caption{1035 Hz}} 

  \fig{figs/fig-ad6be964.png}{\caption{1822 Hz}} 

  So the pan-maker has not only managed to tune one confined mode to the 
  desired frequency in each note region, he has also tuned up to three other 
  modes to harmonically-related frequencies. It seems remarkable that the 
  process offers enough degrees of freedom to achieve this. The area and aspect 
  ratio of the note region give two variables to play with, but to tune four 
  modes in a single patch requires subtle adjustments to the detailed 
  topography of the patch. 

  Based on what we have heard with the marimba and the church bells, we would 
  expect this set of harmonically-related frequencies to allow a clear sense of 
  musical pitch from each note. We can confirm this with a simulated example, 
  in Sound 2. Based on the measured frequencies of a typical steel pan note, 
  and choosing appropriate Q-factors and impact time for the excitation, the 
  result is quite recognisably like the familiar sound of a steel band. 

  \aud{auds/aud-b6692e4e-plot.png}{\caption{Sound 2. Synthesised notes based on 
  measured frequency ratios for a steel pan note like the one illustrated here. 
  The Q-factor was set at 200 and a relatively soft impact was assumed, to 
  match the expected behaviour of mild steel hit with typical steel-band 
  drumsticks.}} 

