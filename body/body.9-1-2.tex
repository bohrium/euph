  Raman's argument applies strictly to bowed-string motion based on two 
  assumptions: that the string is an ideal, textbook string, and that the 
  friction force is governed by a ``Stribeck law'' in which the friction force 
  varies with relative sliding speed according to a curve like the example 
  shown schematically in Fig.\ 1. Neither of these assumptions is exactly true 
  for a real bowed string, as we will see later in this chapter, but they give 
  a good first shot at explaining a lot of bowed-string waveforms. 

  \fig{figs/fig-97b7ee76.png}{\caption{Figure 1. Schematic plot of friction 
  force as a function of sliding speed, where $v\_b$ is the bow speed. The 
  black line and stars relate to Raman's argument, explained in the text.}} 

  The periodic string motions in which Raman was interested were all observed 
  to involve the bowed string oscillating at the same frequency as the free 
  string. As a consequence, he argued, the friction force at the bow must 
  remain essentially constant: the modes of a string have very low damping, so 
  any significant variation of the friction force at the frequency of string 
  modes would produce a resonant response to high amplitude. For the extreme 
  case of the undamped textbook string, the predicted amplitude would be 
  infinite. Helmholtz motion is an example: it is a possible free motion of an 
  ideal string. So one way to think about Helmholtz motion and all the other 
  regimes we will find in this section is that they are free motions of the 
  string which happen to be able to tolerate the presence of the bow. 

  Figure 1 then tells us that the string velocity at the bowed point must 
  alternate, in some pattern, between two fixed values. During an episode of 
  sticking, the string velocity must match the bow speed and lie somewhere on 
  the vertical portion of the friction curve. But the velocity must integrate 
  to zero over a complete cycle, in order that there is no mean sideways motion 
  of the string. So episodes of sticking must be balanced by episodes of 
  slipping with a negative string velocity. If the friction force is to remain 
  constant during this process, the point on the curve corresponding to this 
  slipping speed must lie on a horizontal line with the relevant sticking 
  point, as indicated by the two black stars in the figure. 

  This argument shows that any possible motion of the string at the bowed point 
  must involve spells of constant velocity interrupted by jumps, all of the 
  same magnitude. Now we can use D’Alembert’s solution for the general motion 
  of an ideal string (section 5.4.2) to deduce what must be happening along the 
  rest of the string. The string displacement $w(x,t)$ at position $x$ and time 
  $t$ must be the sum of a right-travelling wave $f$ and a left-travelling wave 
  $g$, both with fixed shape: 

  \begin{equation*}w(x,t)=f(t-x/c)+g(t+x/c) \tag{1}\end{equation*} 

  \noindent{}where $c$ is the wave speed on the string. The string has fixed 
  ends at $x=0$ and $x=L$, so 

  \begin{equation*}f(t)+g(t)=0 \tag{2}\end{equation*} 

  \noindent{}and 

  \begin{equation*}f(t-L/c)+g(t+L/c)=0 , \tag{3}\end{equation*} 

  \noindent{}so 

  \begin{equation*}g=-f \tag{4}\end{equation*} 

  \noindent{}and 

  \begin{equation*}f(t)=f(t+2L/c) . \tag{5}\end{equation*} 

  The string velocity $v(x,t)$ can also be written in travelling-wave form: 

  \begin{equation*}v(x,t)=f^\prime(t-x/c) -- f^\prime(t+x/c) . 
  \tag{6}\end{equation*} 

  \noindent{}from eqs. (1) and (4), where primes denote the derivative. 

  Our measurements are of the force exerted on the bridge. If the string 
  tension is $T$, this is given by 

  \begin{equation*}T \left.\dfrac{\partial w}{\partial x} \right|_{x=0} = 
  \dfrac{T}{c} \left[ g^\prime(t) -- f^\prime(t) 
  \right]=-\dfrac{2T}{c}f^\prime(t) . \tag{7}\end{equation*} 

  So, apart from a constant multiplier, the bridge force wavefrom directly 
  reveals the form of the travelling wave $f^\prime(t)$. 

  For the particular case of Helmholtz motion we already know that the bridge 
  force is a sawtooth wave. So this sawtooth also gives the form of the 
  travelling velocity waves. The result is animated in Fig.\ 2. The two 
  travelling waves are shown at the top, over a range that is three times as 
  long as the physical string, indicated by the vertical lines. The black shape 
  at the bottom shows $v(x,t)$ for the string. 

\moobeginvid\begin{tabular}{ccc} \vidframe{ 0.30 }{ vids/vid-f05b86f9-00.png }&\vidframe{ 0.30 }{ vids/vid-f05b86f9-01.png }&\vidframe{ 0.30 }{ vids/vid-f05b86f9-02.png } \end{tabular}\caption{Figure 2. Helmholtz motion visualised in terms of travelling velocity waves. The travelling waves are shown over a wider range than the physical length of the string, indicated by the vertical lines. The lower animation shows the varying velocity distribution on the string.}\mooendvideo

  This may not be immediately recognisable as corresponding to Helmholtz 
  motion. To understand the plot, try laying a vertical ruler against the 
  screen, selecting a particular position on the string. This is your chosen 
  bowing point. Now look at how the velocity at that point varies through the 
  cycle of vibration. If your chosen point is near the left-hand end, where you 
  would normally bow a violin string, you will see that it spends most of the 
  cycle with a relatively small positive velocity, and the remainder with a 
  larger negative velocity. This is exactly what were were expecting: the 
  amplitude of motion will be scaled so that the positive velocity matches the 
  bow speed. The negative sliding speed has the correct value so that the 
  integrated velocity over the cycle is zero. This description will hold 
  whatever point you select as your bowed point: the only thing that will 
  change is the amplitude scaling in order to match the positive velocity to 
  the bow speed. 

  We can see the string vibration much more clearly if we integrate the 
  sawtooth waveforms to obtain the corresponding travelling wave contributions 
  to the string displacement. Each linear ramp in the sawtooth integrates to a 
  parabolic section of curve, and where the sawtooth had jumps, the integrated 
  function has a sharp corner, in other words a slope discontinuity. Using 
  these travelling waves, the corresponding animation for string displacement 
  is shown in Fig.\ 3, and now it is clear that it does indeed reproduce 
  Helmholtz motion. 

\moobeginvid\begin{tabular}{ccc} \vidframe{ 0.30 }{ vids/vid-12682a59-00.png }&\vidframe{ 0.30 }{ vids/vid-12682a59-01.png }&\vidframe{ 0.30 }{ vids/vid-12682a59-02.png } \end{tabular}\caption{Figure 3. Animation of Helmholtz motion corresponding to Fig. 2, but showing the travelling waves making up the string displacement, and the resulting string motion.}\mooendvideo

  Raman used this approach, via travelling waves of velocity, to catalogue all 
  the possible idealised bowed-string waveforms. The two waves from eq. (6) 
  must always add up to give a velocity waveform at the bowed point consisting 
  of constant velocity segments, interrupted by jumps. He showed that this can 
  only occur if the travelling waves of velocity take the form of a linear 
  ramp, interrupted by jumps. The ramp segments always have the same slope, and 
  the only distinction between different regimes of vibration comes in the 
  number and disposition of the jumps. We will show some examples, to 
  illustrate the two types of bridge force waveform shown in section 9.1. 

  Helmholtz motion is the only possible solution with a single velocity jump. 
  The next simplest possibility has two jumps. Figure 4 shows an animation 
  corresponding to Fig.\ 2, for a typical example of such motion. We know from 
  eq. (7) that the blue curve, the right-travelling velocity wave, mirrors the 
  waveform of bridge force, and we can recognise that waveform as corresponding 
  to Fig.\ 5 of section 9.1, describing a typical case of double-slipping 
  motion. If you do the ruler trick again, selecting a bowing point and 
  watching how the string velocity varies at that point, you will quickly see 
  that the motion does indeed involve two slipping episodes in every cycle. 
  Figure 5 shows the corresponding animation for string displacement. 

\moobeginvid\begin{tabular}{ccc} \vidframe{ 0.30 }{ vids/vid-d804e880-00.png }&\vidframe{ 0.30 }{ vids/vid-d804e880-01.png }&\vidframe{ 0.30 }{ vids/vid-d804e880-02.png } \end{tabular}\caption{Figure 4. Animation similar to Fig. 2, for a typical case of double-slipping motion.}\mooendvideo

\moobeginvid\begin{tabular}{ccc} \vidframe{ 0.30 }{ vids/vid-b9c70e9d-00.png }&\vidframe{ 0.30 }{ vids/vid-b9c70e9d-01.png }&\vidframe{ 0.30 }{ vids/vid-b9c70e9d-02.png } \end{tabular}\caption{Figure 5. Animation of double-slipping motion corresponding to Fig. 4, showing displacement rather than velocity.}\mooendvideo

  Such motion is classified as Raman's ``second type''. There is a special case 
  of this motion that has direct relevance to violinists. If the two velocity 
  jumps are arranged in a regular and symmetrical manner, the resulting motion 
  is illustrated in Fig.\ 6 and 7. This is still double-slipping motion, but 
  because the two slips in each cycle occur with equal spacing, the result is a 
  sound with half the period: in other words, a note that plays an octave 
  higher. Now look at the animation of the string motion in Fig.\ 7: it 
  consists of two ``Helmholtz motions'', going on simultaneously in the two 
  halves of the string. The string remains stationary at its midpoint. This is 
  the motion that arises when a violinist plays a ``harmonic'', by lightly 
  touching a finger at the mid-point of the string. 

\moobeginvid\begin{tabular}{ccc} \vidframe{ 0.30 }{ vids/vid-33830942-00.png }&\vidframe{ 0.30 }{ vids/vid-33830942-01.png }&\vidframe{ 0.30 }{ vids/vid-33830942-02.png } \end{tabular}\caption{Figure 6. Animation similar to Figs. 2 and 4, for the special case of double-slipping motion corresponding to the playing of an octave ``harmonic'' on a violin or cello string.}\mooendvideo

\moobeginvid\begin{tabular}{ccc} \vidframe{ 0.30 }{ vids/vid-25476915-00.png }&\vidframe{ 0.30 }{ vids/vid-25476915-01.png }&\vidframe{ 0.30 }{ vids/vid-25476915-02.png } \end{tabular}\caption{Figure 7. Animation of symmetrical double-slipping motion corresponding to Fig. 6, showing displacement rather than velocity. The played ``harmonic'' involves ``miniature Helmholtz motion'' in the two halves of the string simultaneously.}\mooendvideo

  For a final example, we look at a case of ``S-motion''. The particular case 
  shown in Figs.\ 8 and 9 would be classified by Raman as of 7th type, based on 
  the number of velocity jumps. The blue velocity waveform in Fig.\ 8 can be 
  seen to have the same general form as the measured bridge-force waveforms in 
  Fig.\ 6 of section 9.1. If you do the ``ruler trick'' with Fig.\ 8, you may 
  be able to see that some possible bowing positions have a single slip per 
  cycle in this vibration regime, while others show more than one slip. In 
  practice, S-motion usually appears with a single slip: the particular bowing 
  position then governs which Raman higher type is excited. We will see more 
  about this in section 9.3. 

\moobeginvid\begin{tabular}{ccc} \vidframe{ 0.30 }{ vids/vid-58476c9f-00.png }&\vidframe{ 0.30 }{ vids/vid-58476c9f-01.png }&\vidframe{ 0.30 }{ vids/vid-58476c9f-02.png } \end{tabular}\caption{Figure 8. Animation in the same format as Figs. 2, 4 and 6, showing a typical example of ``S-motion''. This particular example would be classified by Raman as ``7th type'' based on the number of velocity jumps.}\mooendvideo

\moobeginvid\begin{tabular}{ccc} \vidframe{ 0.30 }{ vids/vid-6d5b1d2a-00.png }&\vidframe{ 0.30 }{ vids/vid-6d5b1d2a-01.png }&\vidframe{ 0.30 }{ vids/vid-6d5b1d2a-02.png } \end{tabular}\caption{Figure 9. Animation showing string displacement for the same S-motion case as in Fig. 8.}\mooendvideo