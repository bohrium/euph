  Whenever a vibrating structure has two (or more) modes at exactly the same 
  frequency, these are called degenerate modes. For the ideal circular drum 
  most of the modes appear in degenerate pairs. The exception is the modes with 
  only nodal circles, but no nodal diameters. These modes, which include the 
  lowest-frequency mode of the drum, are single. 

  For the drum mode discussed in the main text, with a single nodal diameter, 
  the displacement around a circle of constant radius on the drum varies like 
  $\cos \theta$, where $\theta$ is the usual polar coordinate angle. The second 
  incarnation of the mode varies like $\sin \theta$. (Recall that both are 
  vibrating at the same frequency, so we don't need to write down the time 
  dependence because it is the same for everything.) Now suppose we have the 
  mixture $a\cos\theta+b\sin\theta$. Choose $r$ and $\phi$ so that 
  $a=r\cos\phi$ and $b=r\sin\phi$. Then the combination is $r \left(\cos\phi 
  \cos\theta+\sin\phi \sin\theta\right)=r\cos\left(\theta-\phi \right)$, the 
  same pattern rotated by angle $\phi$. 

  As an aside, if you measure the natural frequencies of a real drum (or any 
  other structure for which simple theory predicts degenerate modes) you will 
  almost invariably find pairs of frequencies that are very close, but not 
  exactly equal. The degeneracies are split, by the action of some physical 
  mechanism that does not precisely satisfy the assumption of perfect circular 
  symmetry. Some examples: the distribution of tension in a real drum head will 
  never be precisely uniform, the membrane itself will have properties that 
  vary slightly from point to point, and the shape of the rim of the drum will 
  not be a perfect circle. Nothing has absolutely perfect symmetry, every 
  aspect has some finite precision associated with it. 