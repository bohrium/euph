  If we have measured the bridge admittance $Y(\omega)$ of a particular cello, 
  we can use that in an enhanced version of the Raman-Schelleng calculation of 
  minimum bow force, set out in section 9.3.1. As before, we start by assuming 
  that the string is executing Helmholtz motion. If the bow speed is $v_b$, the 
  string has tension $T$, length $L$ and impedance $Z_0$,the bowing position is 
  $\beta$ and the fundamental frequency is $f_0$ Hz, then the sawtooth waveform 
  of bridge force can be written as a Fourier series in the form: 

  \begin{equation*}f_{br}(t)=\sum_{n=1}^\infty{a_n \sin 2 \pi n f_0 t} 
  \tag{1}\end{equation*} 

  \noindent{}with 

  \begin{equation*}a_n = (-1)^{n+1} \dfrac{Tv_b}{n \pi f_0 \beta L} . 
  \tag{2}\end{equation*} 

  The displacement response at the bridge can then be written in terms of $Y$ 
  and the real part of a complex form of the Fourier sine series as 

  \begin{equation*}\Re \sum_{n=1}^\infty{\dfrac{(-1)^n Z_0v_b}{n^2 \pi^2 f_0 
  \beta} Y(2 n \pi f_0) e^{2n \pi i f_0 t}} . \tag{3}\end{equation*} 

  We can now use the same approximation as in section 9.3.1 to deduce the 
  perturbation force at the bow, by assuming that $\beta$ is small so that we 
  can approximately treat the section of string between bow and bridge as being 
  straight: 

  \begin{equation*}f_{pert}(t)=\dfrac{T}{\beta L}\Re 
  \sum_{n=1}^\infty{\dfrac{(-1)^n Z_0v_b}{n^2 \pi^2 f_0 \beta} Y(2 n \pi f_0) 
  e^{2n \pi i f_0 t}} + K\end{equation*} 

  \begin{equation*}=\dfrac{2 v_b Z^2_0}{\pi^2 \beta^2}\Re 
  \sum_{n=1}^\infty{\dfrac{(-1)^n }{n^2} Y(2 n \pi f_0) e^{2n \pi i f_0 t}} + K 
  \tag{4}\end{equation*} 

  \noindent{}where $K$ is a constant. Its value is determined in the same way 
  as before, by requiring that $f_{pert}(\pm 1/2f_0)=0$, so that 

  \begin{equation*}K=-\dfrac{2 v_b Z^2_0}{\pi^2 \beta^2}\Re 
  \sum_{n=1}^\infty{\dfrac{Y(2 n \pi f_0)}{n^2} } . \tag{5}\end{equation*} 

  We can immediately deduce the new expression for the minimum bow force: the 
  criterion is that this perturbation force is just sufficient that, at some 
  stage in the cycle, it takes the friction force up to the limiting value. The 
  result is 

  \begin{equation*}f_{min}=\dfrac{2 v_b Z^2_0}{\pi^2 \beta^2 (\mu_s -- \mu_d)} 
  \times \left[ \Re \sum_{n=1}^\infty{\dfrac{Y(2 n \pi f_0)}{n^2} } 
  \right.\end{equation*} 

  \begin{equation*}\left. -- \max \left\lbrace 
  \Re\sum_{n=1}^\infty{\dfrac{(-1)^n }{n^2} Y(2 n \pi f_0) e^{2n \pi i f_0 t}} 
  \right\rbrace \right] \tag{6}\end{equation*} 

  \noindent{}where ``max'' means the maximum value of this function of time, 
  taken over a complete cycle of the periodic waveform. 

  This expression looks very complicated, but we can relate it back to 
  Schelleng's original formula by looking at an important special case. In our 
  study of the wolf note, we were mainly interested in a single body resonance, 
  and in the case where the cellist bows the note coinciding exactly with that 
  resonance. Suppose this resonance is described in terms of an oscillator with 
  mass $m$, stiffness $k$ and dashpot strength $c$. The admittance then takes 
  the simple form 

  \begin{equation*}Y(\omega)=\frac{i \omega}{k + i \omega c -- \omega^2 m} . 
  \tag{7}\end{equation*} 

  At the resonance frequency, $k=\omega^2 m$ so that $Y=1/c$. Furthermore, 
  looking at the two Fourier series summations in equation (6) we see that, 
  with the factor $1/n^2$, they both converge very rapidly. So we probably get 
  a reasonable approximation to this resonant case if we simply keep the 
  fundamental term $n=1$. Equation (6) then reduces to 

  \begin{equation*}f_{min} \approx \dfrac{2 v_b Z^2_0}{\pi^2 \beta^2 (\mu_s -- 
  \mu_d)} \max \left\lbrace Y(2 \pi f_0) (1+e^{2 \pi i f_0 t} 
  \right\rbrace\end{equation*} 

  \begin{equation*} = \dfrac{4 v_b Z^2_0}{c \pi^2 \beta^2 (\mu_s -- \mu_d)} . 
  \tag{8}\end{equation*} 

  This looks exactly like Schelleng's formula (equation (9) of section 9.3.1), 
  with $c$ in place of his assumed resistance $R$, and a numerical factor 
  $8/\pi^2 \approx 0.8$. 