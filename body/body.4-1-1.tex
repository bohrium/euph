  We first look at the simplest case, of a one-dimensional sound field in which 
  the variations of pressure, density and particle displacement all depend only 
  on the $x$-coordinate. We will assume that the air properties are the same 
  throughout space, and that the fluctuations in pressure and density due to 
  the sound wave are small compared to the steady atmospheric values. For the 
  moment, we simply look at the behaviour of sound waves propagating in empty 
  space: we will think about sound sources and the influence of walls and 
  obstacles a bit later. 

  We can write density as $\rho_0 + \rho'(x)$ and pressure as $p_0 + p'(x)$ 
  where $\rho_0$ and $p_0$ are the steady density and pressure of air, and the 
  quantities with primes are the small deviations from those steady values due 
  to the sound wave. Now consider a small element of the air, lying between 
  positions $x$ and $x+\delta x$, sketched in Fig.\ 1 with the wave inside a 
  pipe with cross-sectional area $A$. As a result of the sound wave, the 
  particles that were initially at the ends of this element are displaced by a 
  small distance, to $x+\xi(x)$ and $x + \delta x + \xi(x + \delta x)$ 
  respectively. 

  \fig{figs/fig-8b3d92ae.png}{Figure 1. A small element of a one-dimensional 
  sound wave, showing particle displacements.} 

  We can apply conservation of mass to this element of air: the product of 
  density and volume must remain constant, so 

  $$\rho_0 A \delta x = (\rho_0 + \rho') A [\delta x -- \xi(x) + \xi(x+ \delta 
  x)] $$ 

  $$\approx (\rho_0 + \rho') A \left[\delta x + \frac{\partial \xi}{\partial x} 
  \delta x \right] . \tag{1}$$ 

  Cancelling $A \delta x$ and ignoring products of small quantities because we 
  want a linearised relation, this reduces to 

  $$0 \approx \rho' + \rho_0 \frac{\partial \xi}{\partial x} .\tag{2}$$ 

  Now we can express $\rho'$ in terms of the pressure change $p'$. At typical 
  acoustic frequencies, sound waves fluctuate sufficiently quickly that the air 
  behaves adiabatically (or isentropically). This means that the thermodynamic 
  relation 

  $$p V^\gamma = \mathrm{constant} \tag{3}$$ 

  can be assumed for any small fluid volume $V$, where $\gamma$ is the ratio of 
  specific heats for air, a constant with the value 1.4. In terms of density, 
  this means 

  $$p \rho^{-\gamma} = \mathrm{constant} \tag{4}$$ 

  so that for our element, 

  $$p_0 \rho_0^{-\gamma} = (p_0 +p') (\rho_0 + \rho')^{-\gamma} = (p_0 +p') 
  \rho_0 ^{-\gamma} (1+ \rho'/\rho_0)^{-\gamma} \tag{5}$$ 

  so that 

  $$0 \approx p' -- p_0 \gamma \frac{\rho'}{\rho_0} \tag{6}$$ 

  using the binomial theorem and ignoring products of small terms. Substituting 
  in eq. (2) then gives 

  $$\frac{\partial \xi}{\partial x} = -- \frac{p'}{p_0 \gamma} . \tag{7}$$ 

  We can obtain a second relation between $p'$ and $\xi$ by applying Newton's 
  law to our small volume. The pressure acts on the two end faces, giving a net 
  force which must be balanced by mass times acceleration. So 

  $$A p(x) -A p(x+ \delta x) = \rho_0 A \delta x \frac{\partial^2 \xi}{\partial 
  t^2} \tag{8}$$ 

  so that 

  $$- \frac{\partial p'}{\partial x} \approx \rho_0 \frac{\partial^2 
  \xi}{\partial t^2} . \tag{9}$$ 

  Differentiating this with respect to $x$ and then using eq. (7), we obtain 

  $$\frac{\partial^2 p'}{\partial x^2} = -\rho_0 \frac{\partial^2}{\partial 
  t^2} \left[ \frac{\partial \xi}{\partial x} \right]= \rho_0 
  \frac{\partial^2}{\partial t^2} \left[ \frac{p'}{p_0 \gamma} \right] 
  \tag{10}$$ 

  So finally, if we define a constant $c$ satisfying 

  $$c^2 =\frac{\gamma p_0}{\rho_0} \tag{11}$$ 

  we obtain the governing equation 

  $$\frac{\partial^2 p'}{\partial t^2}= c^2 \frac{\partial^2 p'}{\partial x^2} 
  .\tag{12}$$ 

  We can recognise this as exactly the same equation we found for the ideal 
  stretched string in section 3.1.1, and so we already know that $c$ is the 
  wave speed, in other words in this case it is the speed of sound. Standard 
  values for the properties of air at sea level and 15$^\circ$C are 
  $\rho_0=$1.225 kg/m$^3$ and $p_0=$1.01 MPa, leading to $c=340$ m/s, the 
  expected value for the speed of sound. 

  It is not difficult to generalise this derivation to the general case of 
  sound waves in three dimensions. We now consider a small rectangular element 
  of air, with sides $\delta x$, $\delta y$ and $\delta z$. The pressure is 
  still a scalar quantity, but the particle displacement becomes a vector 
  $\underline{\xi}(x,y,z)$ with components $(\xi_1,\xi_2,\xi_3) $ in the three 
  axis directions. We follow the same sequence of steps. For the equivalent of 
  eq. (1), we need to allow for the volume change associated with each of the 
  three pairs of faces of the fluid element: the result is 

  $$\rho_0 \delta x \delta y \delta z \approx (\rho_0 + \rho') \left[1 + 
  \frac{\partial \xi_1}{\partial x} \right] \delta x \left[1 + \frac{\partial 
  \xi_2}{\partial y} \right] \delta y \left[1 + \frac{\partial \xi_3}{\partial 
  z} \right] \delta z \tag{13}$$ 

  and so 

  $$0 \approx \rho_0 \left[\frac{\partial \xi_1}{\partial x} + \frac{\partial 
  \xi_2}{\partial y} + \frac{\partial \xi_3}{\partial z} \right] + \rho' 
  \tag{14}$$ 

  or, in the notation of vector calculus involving the divergence operator, 

  $$-\rho_0 \nabla \cdot \underline{\xi} = \rho' = \frac{p'}{c^2} \tag{15}$$ 

  using eqs. (6) and (11). (If you are hazy about vector calculus, the next 
  link gives an introduction to the main concepts.) 

  To apply Newton's law to the fluid element, we simply have the equivalent of 
  eqs. (8) and (9) for each component separately, leading to 

  $$\rho_0 \ddot{\underline{\xi}} = -- \nabla p' \tag{16}$$ 

  in terms of the gradient of pressure. Taking the divergence of this equation 
  and then using eq. (15), we obtain 

  $$\nabla \cdot \nabla p' = -\rho_0 \nabla \cdot \ddot{\underline{\xi}} = 
  \frac{\partial^2}{\partial t^2} \left[ \frac{p'}{c^2} \right] . \tag{17}$$ 

  The result is the wave equation in three dimensions: 

  $$\frac{\partial^2 p'}{\partial t^2} = c^2 \nabla^2 p' .\tag{18}$$ 