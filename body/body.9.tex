  This chapter will explore the details of how a bowed string behaves. This is 
  a subject with a long history, and in section 9.1 we will see the earliest 
  stages, extending back to Helmholtz in the 19th century. Helmholtz first 
  revealed the counter-intuitive way that a bowed string normally vibrates. 

  In subsequent sections, we add layer upon layer of complication as we try to 
  understand enough about bowed strings to engage with the questions that a 
  string player is most interested in. In section 9.2 we meet the first 
  important step beyond Helmholtz's early work, which gave a first glimpse of 
  how a violinist can change the timbre of a note by adjusting their bowing. 
  The effort to understand this problem also led to the first computer 
  simulation models of bowed strings, in the 1970s. 

  In section 9.3 we make a first step toward investigating issues of 
  ``playability'': phenomena that underlie a player's perception that a 
  particular instrument, or string, or even a particular single note, might be 
  ``easier to play'' than another. John Schelleng introduced an important 
  graphical presentation of the parametric limits a player must stay within, if 
  they hope to produce Helmholtz motion in the string. 

  In section 9.4 a particular issue of playability is examined: the notorious 
  ``wolf note'', especially prevalent in cellos. The physics governing a 
  classic wolf note is explained, and possible ways to control a troublesome 
  wolf are discussed. 

  In section 9.5 we move on to the important but difficult topic of transient 
  behaviour of bowed strings. Players must put in many hours of practice to 
  master a range of styles of bowing, all of which can go horribly wrong in 
  inexperienced hands. We devote most of our attention to a particular family 
  of bowing transients, based around an important study by double-bassist Knut 
  Guettler which led to a graphical presentation of what a player needs to do 
  in order to produce a note with a ``perfect start''. Guettler's predictions 
  are compared with laboratory measurements, and then we start on the long and 
  difficult road of constructing a computer simulation model capable of giving 
  predictions that match those measurements. 

  In section 9.6 we focus on the knottiest challenge to such modelling: to 
  understand how exactly the friction behaviour of violinist's rosin behaves. 
  This is an unfinished story: we still don't have a fully satisfactory answer. 

  Finally, in section 9.7 we add the ingredient which has been missing so far: 
  the violin bow. The modern bow is a sophisticated piece of engineering, with 
  features that have a profound influence on the ability of a player to perform 
  some of the virtuosic tricks violinists are famous for. 

