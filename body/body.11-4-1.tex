  To see how to express the input impedance of a tube in terms of modal 
  parameters, we start with the ``Webster horn equation'' from section 4.2.3. 
  We need to augment the version of the equation derived there, by including 
  the term describing forcing input in the form of a volume flow source $q$ per 
  unit length. The result, given in section 7.2.1 of Chaigne and Kergomard [1], 
  is 

  $$\dfrac{S}{c^2} \dfrac{\partial^2 p}{\partial t^2} = 
  \dfrac{\partial}{\partial x}\left(S\dfrac{\partial p}{\partial x} \right) + 
  \rho_0 \dfrac{\partial q}{\partial t} \tag{1}$$ 

  where $p$ is the pressure, $S(x)$ is the cross-sectional area of the tube, 
  $c$ is the speed of sound and $\rho_0$ is the density of air. 

  We are interested in the pressure modes $p_n(x) e^{i \omega_n t}$ when the 
  tube has a closed end at $x=0$. Substituting in equation (1) without the 
  source term, these modes must satisfy 

  $$-\frac{S \omega_n^2}{c^2}p_n = \dfrac{\partial}{\partial 
  x}\left(S\dfrac{\partial p_n}{\partial x} \right) . \tag{2}$$ 

  As usual, the mode shapes are orthogonal with respect to an integral deduced 
  from the potential energy. From section 4.1.3, the potential energy can be 
  expressed in the form 

  $$V=\dfrac{1}{2} \dfrac{1}{\rho_0 c^2} \int_0^L{S p^2 dx} \tag{3}$$ 

  so we can deduce that 

  $$\dfrac{1}{2} \dfrac{1}{\rho_0 c^2} \int_0^L{S p_n(x) p_m(x) dx} = 0 
  \mathrm{~~~for~~~} n \ne m . \tag{4}$$ 

  We can choose to normalise the modes using the same expression, so that 

  $$\dfrac{1}{\rho_0 c^2} \int_0^L{S p_n^2(x) dx} = 1. \tag{5}$$ 

  Now go back to equation (1), and include a source term $q e^{i \omega t} 
  \delta(x)$ describing a concentrated sinusoidal source at $x=0$ (using the 
  Dirac delta function --- see section 2.2.8). The pressure $p(x) e^{i \omega 
  t}$ must satisfy 

  $$-\dfrac{S}{c^2} \omega^2 p = \dfrac{\partial}{\partial 
  x}\left(S\dfrac{\partial p}{\partial x} \right) + \rho_0 i \omega q 
  \delta(x). \tag{6}$$ 

  Now we seek a solution of this equation in the form of a modal expansion 

  $$p(x) = \sum_n{a_n p_n(x)} \tag{7}$$ 

  with complex coefficients $a_n$. If we multiply equation (6) by $p_m(x)$ and 
  integrate over the range $0 \rightarrow L$ we find 

  $$-\dfrac{\omega^2}{c^2} \sum_n{a_n \int_0^L{S p_n p_m dx}}=$$ 

  $$\sum_n{a_n \int_0^L{p_m \dfrac{\partial}{\partial x}\left(S\dfrac{\partial 
  p_n}{\partial x} \right) dx}} + i \omega \rho_0 q p_m(0) \tag{8}$$ 

  where the properties of the delta function have been used to simplify the 
  final term. Now we can use equations (4), (5) and (6) to deduce 

  $$-\dfrac{\omega^2}{c_2}a_m \rho_0 c^2 = -\dfrac{\omega_m^2}{c_2}a_m \rho_0 
  c^2 + i \omega \rho_0 q p_m(0) \tag{9}$$ 

  so that 

  $$a_m (\omega_m^2 -- \omega^2) = i \omega q p_m(0). \tag{10}$$ 

  Substituting in equation (7) gives 

  $$p(x,\omega)= i \omega q \sum_m{\dfrac{p_m(0) p_m(x)}{\omega_m^2 -- 
  \omega^2}} \tag{11}$$ 

  so that the input impedance we want is 

  $$Z(\omega) = \dfrac{p(0)}{q} = i \omega \sum_m{\dfrac{p_m^2(0)}{\omega_m^2 
  -- \omega^2}} . \tag{12}$$ 

  In the usual way (see section 2.2.7) we can tweak this undamped result to 
  allow for small modal damping with Q-factor $Q_m$ for mode $m$, to give 

  $$Z(\omega) \approx i \omega \sum_m{\dfrac{p_m^2(0)}{\omega_m^2 + i \omega 
  \omega_m/Q_m -- \omega^2}} . \tag{13}$$ 

  This has exactly the same form as the modal expression for the input 
  admittance of a mechanical system: see for example equation (12) of section 
  2.2.7 (which would need to be multiplied by a factor $i \omega$ to convert it 
  to admittance). 

  \sectionreferences{}[1] Antoine Chaigne and Jean Kergomard; “Acoustics of 
  musical instruments”, Springer/ASA press (2013) 