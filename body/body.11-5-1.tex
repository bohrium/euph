  The measured input impedance of a trombone without its mouthpiece, shown in 
  the blue curve of Fig.\ 6 in section 11.5, gives modal fits that reveal 
  something interesting about the usual model for energy dissipation of plane 
  sound waves in tubes. The measured Q-factors are plotted against frequency 
  using black stars in Fig.\ 1, on a log-log scale. The red circles show modal 
  fits to the synthesised impedance (shown in the red curve of Fig.\ 6 of 
  section 11.5). That synthesis model used modal Q-factors based on the usual 
  formula for boundary dissipation in cylindrical tubes that are not extremely 
  narrow, described in section 11.1.1 and frequently cited in the standard 
  literature on brass and woodwind instruments, for example Fletcher and 
  Rossing [1] and Campbell, Gilbert and Myers [2]. 

  \fig{figs/fig-14ef20c7.png}{\caption{Figure 1. Modal frequencies and 
  Q-factors fitted to a measured input impedance of a trombone without 
  mouthpiece (black stars), and to a synthesised version of that impedance 
  using the conventional model of energy dissipation (red circles). Power law 
  trends are indicated, with exponents 0.5 (dashed line) and 0.7 (solid 
  line).}} 

  The damping model is usually stated in terms of a spatial decay rate, so a 
  first step is to convert that into a temporal decay rate, which in turn can 
  be translated into a Q-factor. The argument used in this work runs as 
  follows. The damping model gives the spatial variation of a harmonic signal 
  at frequency $\omega$ as 

  $$e^{i\omega x/c-\alpha x} \tag{1}$$ 

  where 

  $$\alpha \approx 1.2 \times 10^{-5} \sqrt{\omega}/a, \tag{2}$$<br> 

  $c$ is the speed of sound and $a$ is the radius of the tube. On the other 
  hand, in a free decay of a mode with frequency $\omega_n$ and Q-factor $Q_n$, 
  the temporal variation is proportional to 

  $$e^{i \omega_n t -\omega_n t/Q_n} . \tag{3}$$ 

  If we now assume that the frequency-wavenumber relation $k=\omega/c$ 
  continues to hold for this complex frequency and complex wavenumber, we find 

  $$Q_n \approx \frac{\omega_n}{2 \alpha c} . \tag{4}$$ 

  The Q-factors indicated by the red circles, based on this damping model, are 
  always well within a factor of 2 of the corresponding measured results, and 
  ordinarily this would be regarded as excellent agreement for any predictive 
  model of damping. But in fact the plot suggests that this theory may be 
  missing something, because it does not predict the correct trend with 
  frequency. The measured results show a clear and orderly pattern, which 
  roughly follows a power law, but the power is more like 0.7 (indicated by the 
  solid blue line) than the value 0.5 arising from the square root in equation 
  (2) (shown by the dashed blue line). However, some doubt has been cast on the 
  calibration accuracy of this particular measured impedance, so pending 
  confirmation of that we should not worry too much about the discrepancy. 

  \sectionreferences{}[1] Neville H Fletcher and Thomas D Rossing; “The physics 
  of musical instruments”, Springer-Verlag (Second edition 1998) 

  [2] Murray Campbell, Joël Gilbert and Arnold Myers, “The science of brass 
  instruments”, ASA Press/Springer (2021) 