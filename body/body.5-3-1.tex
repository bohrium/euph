  To see the effect of driving a system through a resonant mass-spring system, 
  we need to analyse the idealised version shown in Fig.\ 1. A force $Fe^{i 
  \omega t}$ is applied to a mass $m$. This is connected via a spring of 
  stiffness $k$ to the original system, which has admittance $Y_v(\omega)$ (v 
  for `violin'). The aim is to calculate the new admittance $Y_b(\omega)$ (b 
  for `bridge') at the forcing position. We can denote the displacement of the 
  mass by $x_b e^{i \omega t}$ and the displacement of the underlying system by 
  $x_v e^{i \omega t}$. For the rest of the analysis, we will drop the $e^{i 
  \omega t}$ factors. 

  \fig{figs/fig-33adb3bf.png}{\caption{Figure 1. Schematic sketch of a system 
  driven through a mass-spring oscillator.}} 

  The force from the spring is $k(x_b-x_v)$: upwards at the top and downwards 
  at the bottom. So Newton' s law for the mass states that 

  $$-\omega^2 m x_b = F -k(x_b -x_v) \tag{1}$$ 

  while the force and velocity on the `violin body' must satisfy 

  $$i \omega x_v = k (x_b -- x_v) Y_v(\omega) . \tag{2}$$ 

  We don't really want $x_v$, so we eliminate it between these two equations. 
  From eq. (2), 

  $$x_v(i \omega +k Y_v)=kY_v x_b .\tag{3}$$ 

  Substituting in eq. (1) then gives 

  $$-\omega^2 m x_b = F -k x_b + \dfrac{k^2 Y_v x_b}{i \omega + kY_v} 
  .\tag{4}$$ 

  We are trying to calculate 

  $$Y_b(\omega) = \dfrac{i \omega x_b}{F} \tag{5}$$ 

  and so using eq. (4), 

  $$Y_b(\omega) = \dfrac{i \omega}{k -- \dfrac{k^2 Y_v}{i \omega +k Y_v} -- 
  \omega^2 m}$$ 

  so that after rearranging, 

  $$Y_b(\omega) = \dfrac{i \omega + kY_v}{k-\omega^2 m + i \omega k Y_v} 
  .\tag{6}$$ 