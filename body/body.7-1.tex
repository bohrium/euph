  After that interlude looking at hearing and psychoacoustics, we can return to 
  unfinished business with stringed instruments. We aren’t ready to look at 
  bowed-string instruments yet: we still want to learn as much as we can using 
  linear methods, and the bowed string is very definitely nonlinear. But we 
  have only scratched the surface of the extraordinary variety of musical 
  instruments from around the world that use plucked or struck strings. We have 
  looked at the guitar and the banjo, but what about the harp or the lute, the 
  piano or the harpsichord, let alone the Indian sitar, the Chinese pipa, or 
  the west African kora? 

  We will not, of course, deal with every instrument on an individual basis. 
  But there are some common themes that will be explored in this chapter. A 
  useful starting point is to ask “How do you recognise what instrument is 
  being played?” As we have already heard with the comparison of the guitar and 
  banjo, it is sometimes possible to distinguish which is which of two 
  instruments from the sounds of just a few notes. But is that always true? 
  Probably the sound of a sitar would be recognised straight away, at least in 
  comparison to something like a guitar or a piano. But what about comparing a 
  guitar to a harp, or a lute, or a mandolin? 

  If you ask a musician a question like “Why does a lute sound different from a 
  guitar or a harp?”, they will usually assume that the main reason is 
  something to do with the obvious difference in body shape between the 
  instruments. But we should take note of the dictum quoted at the head of this 
  page. Out of the many factors that contribute, body shape is in fact 
  surprisingly unimportant for this question of telling instruments apart. It 
  would be different if we were asking about the difference of sound between 
  similar instruments; two guitars, two harps, two violins. In that case, the 
  other factors that we will discuss in a moment are all fixed, so that 
  relatively subtle variations in body size, shape, thickness distribution and 
  material become the dominant influence. 

  When it comes to recognising what instrument is being played in a real 
  musical context, by far the most important factors are to do with what the 
  performer does. Each instrument offers particular possibilities to the 
  player, and also imposes particular constraints. On top of that, there are 
  questions of culture, training and repertoire. I suggest that you recognise 
  that a guitar is being played when you hear “guitarish” things, as opposed to 
  “harpish” things or, in the extreme, “violinish” things. 

  That all seems a bit vague, so let’s look at some specific examples. A 
  fretted instrument like the guitar has only a limited number of strings, and 
  the player will often play successive notes on the same string. That 
  obviously means that the previous note is cut off when the next one is 
  played. But on a harp you have a separate string for each note. Unless the 
  player chooses to stop the vibration, each string will continue to ring on, 
  and the sound will overlap with succeeding notes. This gives the listener an 
  immediate clue to what instrument is being played. 

  But performers like to play games to confound the listener. In some fretted 
  instruments, a performance technique called “campanella” may be used. It is 
  especially common in instruments with re-entrant tuning like the baroque 
  guitar or the ukulele. The tuned frequencies of the strings are not all in a 
  descending sequence as you strum from the “top string” to the “bottom 
  string”; instead, the pitch jumps up at some stage. The player can take 
  advantage of a string that is tuned high like this: something like a scale 
  can be played by interleaving notes on different strings, so that notes ring 
  on in a harp-like manner. Or, of course, a bell-like manner, explaining the 
  name “campanella”. 

  There are plenty of other examples of performance details that give an 
  immediate clue about what instrument is being played. The characteristic harp 
  glissando is an example. Somewhat more subtle, there are characteristic 
  features of keyboard playing, tending to produce mannerisms different from 
  those of a finger-plucked instrument or, indeed, a plectrum-strummed 
  instrument. 

  As usual, you should not believe any claims like these without evidence. 
  Computer-synthesised sounds give a clean way to make comparisons (accepting 
  that no such sounds are really a perfect representation of real instruments). 
  Listen to the four sounds below, and give an immediate vote for what 
  instrument is being played. Perhaps your first impression is that the first 
  two are played on harps of some kind, while the second pair are played on 
  guitars, or possibly on some kind of keyboard instrument? 

  In fact, the first pair use harp string data and ``harpish'' music, but the 
  first uses the measured bridge admittance of the small harp in Fig.\ 1, while 
  the second uses the admittance of a guitar. The second pair uses guitar 
  string data and ``guitarish'' or ``lutish'' music, but this time the first 
  uses the guitar admittance, while the second uses the harp admittance. So 
  Sound 2 has ``harp strings on a guitar body'', while Sound 4 has ``guitar 
  strings on a harp body''. To my ears, the first pair both sound like harps, 
  and if anything the second one sounds like a better harp. In the second pair 
  of sounds, I am aware of the lack of bass response in Sound 4, but it doesn't 
  immediately strike me as ``sounding like a harp''. 

  Now for a more difficult task. If clues from musical style are removed by 
  playing a single note on a variety of stringed instruments, can you reliably 
  decide what you are listening to? Try the experiment yourself. Figure 1 shows 
  a collection of 7 stringed instruments. Below it are 7 sound examples: the 
  note E$_4$ has been played on each of these instruments, using a typical 
  performance technique in each case (finger pluck, fingernail pluck, plectrum 
  pluck etc.) The recordings are not of studio quality, but they are all made 
  in the same room under very similar conditions. The notes have been trimmed 
  to 1 second length, scaled so that they all have the same peak signal level, 
  and then saved as the set of sound files. Can you identify which of the 7 
  instruments in the picture is playing each sound? I will reveal the answers 
  at the end of the section, to allow you to try the test ``blind''. 

  Even if you could not immediately name all the instruments, you probably 
  found that the 7 sounds were all quite distinctly different. So what are the 
  factors that allow you to distinguish them? I have already suggested that the 
  difference of body shapes visible in the picture will only be part of the 
  story. There are several things to do with strings that can be important. 
  Most obviously, the strings may be driven in different ways in different 
  instruments. A harpsichord sounds immediately different from a piano, partly 
  because of a very different choice of string gauges, but also because a 
  harpsichord string is plucked by a narrow plectrum, while a piano string is 
  struck by a felt-covered hammer. 

  We have already seen in the banjo study that the impedance of the strings 
  relative to the soundboard can have a strong influence on the decay rates of 
  played notes, and that decay rates can be very important for recognising an 
  instrument. The importance of decay rates is part of a larger issue, the 
  importance of transients. We will have more to say about this in Chapter 9, 
  when we reach a discussion of bowed strings, but for now a simple sound 
  demonstration will make the point. Listen to Sound 12. What instrument are 
  you hearing? Now listen to sound 13, and this time you will be in no doubt 
  about what you are hearing. The only difference between these two sounds is 
  that Sound 12 is played backwards. It has identical frequency content, the 
  only difference is the way everything varies in time. But the perceptual 
  effect is very striking: your ability to recognise instruments depends 
  critically on these details of the time dependence. 

  There are other important aspects of string choice. Metal strings sound 
  different from polymer strings, and at a more subtle level polymer strings 
  don’t all sound the same: gut, nylon and fluorocarbon strings are all 
  different, as will be explored in section 7.2. Some instruments have strings 
  in pairs, or even (in the case of the piano) triples. The strings of a pair 
  or triple may be tuned to nominal unison, or in some cases they are tuned an 
  octave apart. In section 7.3 the significance of multiple stringing will be 
  explored, drawing inspiration from a classic study by Gabriel Weinreich [2]. 

  Finally, instruments like the piano and the lute can exhibit subtle but 
  audible effects of nonlinearity, modifying the sound from what would be 
  predicted using linear theory. In section 7.4 we will have a preliminary look 
  at these effects, preparing the ground for Chapter 8 in which we take a deep 
  breath and start to look properly at nonlinear phenomena. That will prepare 
  the way for later chapters, in which we look at bowed strings and also at the 
  extensive family of wind instruments, all of which are intrinsically 
  nonlinear. 

  So what were the 7 sounds? 

  Sound 5 was a classical/flamenco guitar. The note was the open top string, 
  played with a fingernail using an apoyando stroke. The string is plain nylon. 

  Sound 6 was probably the easiest to recognise: it was the piano. This note is 
  in the range where the piano is strung with plain steel wire, three strings 
  per note. The recording reveals that my piano is tuned slightly flat! 

  Sound 7 was the harp, plucked with a fingertip. The string is made of nylon. 
  We will find out in section 7.4 that this particular sound is influenced by 
  nonlinear effects, so that it doesn't entirely sound the same as the linearly 
  synthesised ``harp'' in Sound 1. 

  Sound 8 was the banjo: the second fret on the top string was plucked with a 
  fingertip. The string is plain steel, far thinner than the strings of the 
  piano. 

  Sound 9 was a Neapolitan mandolin. This instrument is tuned in the same way 
  as a violin, and the note here was played at the second fret on the 3rd 
  course of strings, plucked with a plectrum. This pair of nominal unison 
  strings have steel cores, over-wound with wire. 

  Sound 10 was a lute, played with a fingertip at the 2nd fret on the second 
  course, a unison pair of plain nylon strings. 

  Sound 11 was a pizzicato note on a violin. The note was fingered in first 
  position on the 3rd (D) string, and plucked with a fingertip. The string has 
  a stranded polymer core, over-wrapped with flat metal tape to give a smooth 
  outer surface. 



  \sectionreferences{}[1] Evan Davis: a comment made during a talk on cello 
  acoustics to the Oberlin Violin Acoustics Workshop, 2021. 

  [2] Gabriel Weinreich ``Coupled piano strings''; Journal of the Acoustical 
  Society of America \textbf{62}, 1474--1484 (1977). 