

  Even without the added complication of room acoustics associated with a 
  performance space, the way that sound is radiated by a musical instrument (or 
  any other vibrating structure) is a complicated business. The details depend 
  very strongly on the Helmholtz number: the size of the sound source compared 
  to the wavelength. The sequence of behaviour is somewhat similar to the 
  shadowing example we saw earlier, in section 4.1. Sources with very small 
  Helmholtz number give simple behaviour, amenable to analysis. Intermediate 
  Helmholtz number, around unity, gives the most complicated behaviour; but 
  then very high Helmholtz number restores a measure of simplicity. For musical 
  instruments, as we will see shortly, very small Helmholtz number is somewhat 
  of a rarity. Despite this, the analysis of that case gives rise to some 
  concepts of such importance that we will devote a disproportionate amount of 
  space to it. 

  \textbf{A. Small sources: monopoles, dipoles, quadrupoles} 

  When sound is generated in a region that is very small compared to the 
  wavelength, there will be a near-field region within which there may be 
  vigorous motion of the air, but from the point of view of the far-field 
  radiated sound the details of this local motion don't matter very much. As 
  was explained in section 4.1.3, the non-radiating part of the near-field 
  disturbance is associated with motion of the air that is essentially 
  incompressible. You can picture it by imagining the same source distribution 
  vibrating very slowly, but in water rather than air. The water will flow back 
  and forth around the vibrating object, but it will ``get out of the way'' 
  rather than being compressed to form sound waves. This kind of near-field 
  flow will carry kinetic energy, and may have a strong influence on the 
  reaction forces experienced by the vibrating object, but it will not 
  influence the far-field sound. 

  That far-field sound can be described in terms of a combination of rather 
  simple patterns, two of which we have already met. If the vibration of the 
  source involves a net change of volume during the cycle, then the far-field 
  sound will have a component looking exactly like the pulsating sphere example 
  we saw in section 4.1. This is called a monopole source. It radiates sound 
  equally in all directions. 

  But if the sound source involves vibration without any net volume change, it 
  can't generate monopole sound. We saw an example in Fig.\ 4 of section 4.1, 
  with two adjacent spheres pulsating in opposite phases so that the net volume 
  change is zero. This is called a dipole source. The radiated sound is weaker 
  because of cancellation between the two sources, and it is no longer the same 
  in all directions. It has a figure-of-eight directional dependence, which we 
  can represent in a polar plot like the ones we saw for shadowing, in Fig.\ 7 
  of section 4.1. Figure 1 shows two examples, corresponding to different 
  orientations of the pair of sources. The two sources are indicated 
  schematically by red and blue symbols connoting the two different phases of 
  pulsation. 

  \fig{figs/fig-2f8e05ff.png}{Figure 1. Two small pulsating spheres, with equal 
  magnitudes but opposite phases, create a dipole sound source. Two possible 
  orientations are shown here. The blue curves show how the pressure amplitude 
  varies with direction: the radial distance to the curve gives the strength in 
  the corresponding direction.} 

  We can take the cancellation idea one stage further. If two dipoles with 
  opposite sign are placed side by side, with all spacings between spherical 
  sources small compared to the wavelength, the result is a quadrupole source. 
  Two examples are shown in Fig.\ 2: the detailed directional pattern depends 
  on whether the two cancelling dipoles are placed side by side, or end-to-end. 

  \fig{figs/fig-92a67b68.png}{Figure 2. If two dipoles in opposite phases are 
  placed close together so that they tend to cancel, the result is a quadrupole 
  source. There are two varieties, depending on the spatial arrangement: on the 
  left is a tesseral quadrupole, on the right is an axial quadrupole. Red and 
  blue symbols represent the sources and their phases, as in Fig. 1. The blue 
  curve shows the directional variation of radiated sound pressure.} 

  You may be wondering what these arrangements of pulsating spheres have to do 
  with musical instruments. The first step in answering that question is to 
  realise that there is something more familiar that generates a dipole field. 
  If we take a single rigid object and shake it from side to side, what kind of 
  radiated sound field will be generated? There will be no net change of 
  volume, so it is not a monopole source. But as it vibrates in one direction, 
  it is pushing at the air on the front face, while pulling on it with the back 
  face. This is a little like having a positive source at the front, and a 
  cancelling negative source at the back: exactly the recipe for a dipole 
  source. The next link gives details, showing that this intuitive argument is 
  in fact the right answer. A vibrating rigid object is a dipole source, 
  provided that it is small compared to the wavelength of sound associated with 
  the frequency of vibration. 

  We can go further. There is a familiar object which illustrates a quadrupole 
  sound source: the tuning fork. When a tuning fork vibrates at the frequency 
  marked on it, the two blades of the fork are moving in opposite directions, 
  as sketched in Fig.\ 3. Each blade is a dipole source, and the two are 
  arranged along a straight line so that they tend to cancel: just like the 
  right-hand diagram in Fig.\ 2. 

  \fig{figs/fig-80220266.png}{Figure 3. Vibration of a tuning fork, and the 
  Stokes experiment using a table knife to interfere with the near-field air 
  flow and make the sound louder} 

  You can easily use a tuning fork to appreciate an important feature of a 
  quadrupole sound field. Bang a fork to get it vibrating, then hold it at arms 
  length. You will hardly be able to hear it: the far-field sound radiation is 
  very weak because of the cancellation effect. But bring it very close to your 
  ear, and you will hear it loud and clear. Now move it slowly away from your 
  ear. The sound dies away much more quickly than would happen with the sound 
  of the small loudspeaker in your mobile phone, for example. When you hold the 
  fork near your ear you are in the near field, and for reasons explained in 
  the previous link the near-field components of dipoles and quadrupoles decay 
  faster than the $1/r$ variation of far-field sound. Furthermore, quadrupoles 
  decay even faster than dipoles. This is what makes the sound of the tuning 
  fork behave in that unfamiliar way. 

  Figure 3 also illustrates a demonstration of the influence of quadrupole 
  cancellation on the loudness of the radiated sound, attributed to Stokes. 
  Make a tuning fork vibrate vigorously, then bring something like a table 
  knife very close to one corner of the vibrating fork (but without touching 
  it). The right-hand sketch in Fig.\ 3 shows a top view of the tuning fork, 
  and indicates a good position of the knife. Once the blade gets near enough, 
  you should hear the sound get louder. The knife blade has interfered with the 
  near-field flow of air, and the cancellation can't work so thoroughly. 

  \textbf{B. Musical applications of monopoles and dipoles} 

  It is useful to summarise what we have learned so far. Any source of sound 
  that is small compared to the wavelength will work most effectively if it 
  produces fluctuating volume, so that it is a monopole source. Dipole and 
  quadrupole sources are progressively less efficient at generating far-field 
  sound. But even a monopole source has important limitations. As was shown in 
  the previous link, the sound pressure does not depend explicitly on the size 
  of the source, but on the product of frequency and total volume flux. So the 
  bigger the volume flux, the louder the sound. But for any practical device 
  there is always a limit on the volume flux that can be achieved, and then the 
  factor of frequency tells us that the sound pressure will inevitably drop off 
  at lower frequencies. A loud sound source at low frequency simply cannot be 
  small. 

  A familiar sound generator capable of high volume flux and hence loud sound 
  is the siren. Air is pumped under pressure through a small outlet, and the 
  flow is periodically interrupted by a rotating disc with a ring of holes in 
  it, or some similar system. The result is that the volume flux of the air 
  flow from the outlet is modulated, making a strong monopole source. A 
  somewhat similar effect is produced by our own vocal chords, modulating air 
  flow from our lungs: a singer can indeed be quite loud. Even more striking, 
  perhaps, is bird song. The British wren, for example, is a tiny bird with a 
  startlingly loud and piercing song. Both factors are significant: the bird's 
  vocal apparatus produces a modulated volume flux, but they also sing at 
  relatively high frequency. But even so, the size of a wren's mouth is small 
  compared to the wavelengths of sound they produce. 

  One other non-musical example is worth mentioning here. Have you ever noticed 
  that the only time you hear sound from flowing water is when it has bubbles 
  in it? A bubble in water can have many possible resonances, but one of them 
  exactly matches our ``pulsating sphere'' example. A small spherical bubble 
  can pulsate symmetrically, because the air inside it is not very stiff 
  compared to the surrounding water. Unsteady forces arising from other aspects 
  of the flowing water can excite this resonance of the bubble, and it makes a 
  monopole sound source. Without bubbles the water flow is essentially 
  incompressible, so however complicated the flow pattern is, little sound is 
  made. A violent river in spate can be full of whirlpools and other exciting 
  flow details, while being ominously silent. A trickle of water falling over a 
  pebble and making a few bubbles can immediately be heard. 

  In terms of musical instruments, the closest parallel with the siren or the 
  bird song might be an instrument like the clarinet. Sound is radiated from 
  the bell at the end of the instrument or, more commonly, from open toneholes 
  further up the tube. In both cases there is oscillating airflow of some kind 
  through rather small orifices, giving a monopole sound field. (If more than 
  one hole radiates significant sound, the spacing may violate the ``small 
  Helmholtz number'' requirement, but we won't enquire into this detail now.) 
  For brass instruments, essentially all the radiated sound comes out of the 
  bell. For low frequencies the Helmholtz number can be quite small and the 
  sound field omnidirectional, but the higher overtones in the sound will 
  violate the assumption. The sound becomes increasingly directional, beaming 
  predominantly forwards as is only too familiar to anyone who has sat in front 
  of the brass section in an orchestra. 

  The situation with stringed instruments is very different. A vibrating string 
  cannot radiate much in the way of sound waves into the surrounding air; it is 
  far too thin compared to the wavelengths of sound in air at audio 
  frequencies. Most musical strings are only around 1 mm in diameter, whereas 
  the wavelength of sound in the mid-audio range is of the order of hundreds of 
  millimetres, and even at the very highest frequency audible to humans, around 
  20 kHz, it is about 17 mm. So a string is a very weak dipole source of sound. 
  A simple illustration of this weakness is given by the (lack of) sound of an 
  unplugged electric guitar. 

  The purpose of the body of an acoustic stringed instrument is to extract some 
  of the energy from the vibrating string, and use it to excite a more 
  efficient sound source (the details will be explored in Chapter 5). So it is 
  no use having an instrument body that is too small, or it will still be a 
  weak radiator of sound, especially for the lowest notes of the instrument. On 
  the other hand, players do not want instruments to be too cumbersome to 
  handle and carry around. The result is that instrument makers have probably 
  settled on a compromise in which the bodies of instruments are just big 
  enough to allow sufficient radiation of sound at the lowest frequencies of 
  interest. 

  We can estimate some numbers to test this idea. Of course, there is no very 
  precise value for the ``radius'' of a guitar or violin body. But a simple 
  estimate can be made by noting that for a sphere, the conventional Helmholtz 
  number is the radius times the wavenumber, and this is the same thing as the 
  ratio of the circumference to the wavelength (because the factors of $2 \pi$ 
  cancel out). Putting a tape measure round a guitar at the widest part of the 
  body, roughly along the line of the bridge, gives a ``circumference'' of 
  approximately 0.9 m. Doing the same thing for a violin gives 0.5 m. For the 
  guitar, the lowest note in normal tuning is $E_2$ at 82.4 Hz, giving a 
  Helmholtz number around 0.22. The frequency at which this measure of 
  Helmholtz number would reach unity is about 380 Hz. For the violin, the 
  lowest tuned note is $G_3$ at 196 Hz, giving a Helmholtz number about 0.29. 
  It would reach unity around 680 Hz. 

  Both the guitar and the violin have ``smallish'' Helmholtz numbers at their 
  lowest frequencies. The lowest useful resonances of a guitar are given by the 
  pair of modes we looked at in section 4.2, based on coupling of the Helmholtz 
  mode of the cavity to the first mode of the top plate. These two modes 
  typically occur around 100 Hz and 200 Hz, and we now see that the estimated 
  Helmholtz number remains below unity over enough of a frequency range to 
  encompass these two modes, but only just. One would want both modes to 
  involve volume change so that they behaved like monopole sources, and that 
  indeed is what we saw in the animation of Fig.\ 8 in section 4.2. There is a 
  similar story for the violin, although the mode shapes are rather more 
  complicated: some images will be shown in section 5.3. 

  As a final example of using volume fluctuation to improve sound radiation, we 
  can look back at the xylophone which we discussed in section 3.3. Here is a 
  repeat of the picture shown there: 

  \fig{figs/fig-60068067.png}{Figure 4. A xylophone, showing the resonators 
  beneath each note.} 

  Each note of the instrument has a resonator beneath it, which is an 
  open-closed tube tuned to the fundamental frequency of the note. We can now 
  see what these resonators are for. A xylophone bar will vibrate with no 
  volume change, and it is small enough to have small Helmholtz number, so on 
  its own it would be a rather weak dipole source of sound. But the acoustic 
  resonance of the tube does involve net volume change: air will flow in and 
  out of the open end. By tuning the resonator to the same frequency as the 
  bar, the two modes will couple together in a somewhat similar way as happened 
  in the guitar body, resulting in combined motion that involves some volume 
  change and thus creates monopole radiation. 

  \textbf{C. Larger sound sources} 

  As frequency goes up, so does the Helmholtz number. From a mathematical 
  standpoint it is still possible to use monopoles, dipoles, quadrupoles and so 
  on to describe the far field radiated sound, but it rapidly ceases to be very 
  useful. Suppose we have a violin body, vibrating at some chosen frequency and 
  sending out sound waves. Choose a distance that is far enough away to be in 
  the far field, and imagine a sphere at that radius, enclosing the violin. 
  With a bit of effort, the sound pressure can be measured at a lot of points 
  covering that sphere, leading to a directivity pattern for the sound. Some 
  examples for a violin are shown in Figs.\ 5--8, based on measurements by 
  George Bissinger using a procedure described in [1]. 

\moobeginvid\begin{tabular}{ccc} \vidframe{ 0.30 }{ vids/vid-6f219ff6-00.png }&\vidframe{ 0.30 }{ vids/vid-6f219ff6-01.png }&\vidframe{ 0.30 }{ vids/vid-6f219ff6-02.png } \end{tabular}\caption{Figure 5. Directivity pattern of a violin at 280 Hz, the frequency of the Helmholtz-like resonance. The violin is rotating about the axis of the neck, with green showing radiation on the front side and red on the back side. The orientation of the image matches the way the measurements were made: the neck of the violin is pointing downwards. The radial scale is logarithmic, with a range of 30 dB from the centre to the maximum. Measured data is courtesy of George Bissinger.}\mooendvideo

\moobeginvid\begin{tabular}{ccc} \vidframe{ 0.30 }{ vids/vid-6c7d6e50-00.png }&\vidframe{ 0.30 }{ vids/vid-6c7d6e50-01.png }&\vidframe{ 0.30 }{ vids/vid-6c7d6e50-02.png } \end{tabular}\caption{Figure 6. Radiation pattern as in Fig. 5, at frequency 803 Hz. Measured data is courtesy of George Bissinger.}\mooendvideo

\moobeginvid\begin{tabular}{ccc} \vidframe{ 0.30 }{ vids/vid-61768f76-00.png }&\vidframe{ 0.30 }{ vids/vid-61768f76-01.png }&\vidframe{ 0.30 }{ vids/vid-61768f76-02.png } \end{tabular}\caption{Figure 7. Radiation pattern as in Fig. 5, at frequency 980 Hz. Measured data is courtesy of George Bissinger.}\mooendvideo

\moobeginvid\begin{tabular}{ccc} \vidframe{ 0.30 }{ vids/vid-085df3ec-00.png }&\vidframe{ 0.30 }{ vids/vid-085df3ec-01.png }&\vidframe{ 0.30 }{ vids/vid-085df3ec-02.png } \end{tabular}\caption{Figure 8. Radiation pattern as in Fig. 5, at frequency 1124 Hz. Measured data is courtesy of George Bissinger.}\mooendvideo

  There is a set of mathematical functions called spherical harmonics, which 
  can be used a bit like the sines and cosines of a Fourier series (remember 
  them from section 2.2.1?). It is always possible to express the measured 
  directivity pattern as a combination of these functions. These functions 
  match exactly the far-field radiation of monopoles, dipoles etc. The snag is 
  that once you are in the far field, all these components of the sound field 
  decay at the same rate, proportional to $1/r$. We can no longer make the 
  argument that the monopole component will dominate over the dipole component, 
  and the dipole will in turn dominate over the quadrupole component. That 
  argument relied on the fact that when the Helmholtz number was very small, 
  there was a significant region in which the near-field behaviour dominated, 
  and it was the respective near fields of monopole, dipole and quadrupole 
  terms that had very different decay rates, leading to the hierarchy of 
  dominance. 

  Instead, the directivity patterns have complicated shapes, progressively more 
  so as frequency goes up. We get an idea of this from the examples above. 
  Figure 5 shows the measured pattern at the lowest resonance frequency, the 
  Helmholtz-like resonance called ``A0'' in the cryptic jargon of violin 
  acoustics (see section 5.3). The pattern is virtually spherical: this mode 
  has net volume change, making it a monopole source, and it has a sufficiently 
  small Helmholtz number that this monopole term dominates the behaviour. 
  Figures 6--8 show progressively higher frequencies, giving examples of the 
  kind of thing that happens as frequency goes up. They all have non-spherical 
  shapes of one kind or another. Figure 6 is elongated in the vertical 
  direction (along the length of the violin body), Fig.\ 7 shows more radiation 
  towards the front (in green) than towards the back (in red). Figure 8 shows 
  something like a 4-lobed pattern: I have chosen a different viewing angle 
  from the other cases to show it most clearly. Finally, Fig.\ 9 shows the 
  directional pattern for the same violin, evolving as a function of frequency. 

\moobeginvid\begin{tabular}{ccc} \vidframe{ 0.30 }{ vids/vid-204d2b50-00.png }&\vidframe{ 0.30 }{ vids/vid-204d2b50-01.png }&\vidframe{ 0.30 }{ vids/vid-204d2b50-02.png } \end{tabular}\caption{Figure 9.  The directional pattern of the same violin as Figs. 5--8, but this time shown as a function of frequency, from 250 Hz up to 4 kHz. As in the earlier animations, green shows radiation on the front side and red on the back side. The orientation of the image matches the way the measurements were made: the neck of the violin is pointing downwards. The radial scale is logarithmic, with a range of 30 dB from the centre to the maximum.}\mooendvideo

  The patterns continue to become more complicated as frequency goes up. They 
  also vary rapidly with frequency, and by mid-kHz frequencies the pattern 
  varies fast enough to show significant changes from one semitone to the next, 
  and even within the range of a violinist's vibrato. That means that when a 
  note is played with strong vibrato, among the other complicated things that 
  happen the directional pattern of each varying harmonic of the sound may have 
  narrow lobes that are swinging around like beams from a lighthouse. This 
  phenomenon, combined with the acoustics of a concert hall, may contribute to 
  an effect that has been called directional tone colour [2]. 

  \textbf{D. Sources very large compared to the wavelength} 

  When frequency gets so high that the size of the source region is very large 
  compared to the wavelength, simplified models can often be used to give clues 
  about how the radiated sound will behave. The reason is that once the source 
  region is large enough, it may not make all that much difference to the local 
  behaviour if we assume it is infinitely large. Infinite systems may seem 
  physically implausible, but they often lead to easier mathematical models. 

  An important example of this is the idea of a baffle. We already mentioned 
  the possibility of using the result for sound radiation from a small monopole 
  source to describe more complicated sound fields. The idea is simple enough. 
  Imagine a vibrating violin body, like the one producing the measured 
  radiation patterns in Figs.\ 5--8. If we know the pattern of vibration on the 
  surface of the instrument, we could imagine dividing that up into a lot of 
  very tiny pistons, each one capturing the vibration in its own little area. 
  Each of these pistons will have very small Helmholtz number, and will radiate 
  monopole sound waves. We could imagine using a computer program to add up all 
  these contributions to give the complete sound field. 

  But there is a snag: the body of the violin will produce a shadowing effect, 
  so that the pistons on the front surface will not be able to radiate sound 
  very effectively towards the back, and vice versa. Solving this shadowing 
  problem is much more complicated than just adding up all the monopoles from 
  the little pistons. But at very high frequency, we might be able to get away 
  with a trick. There is a special case, for which the shadowing problem goes 
  away. If the vibrating object was an infinitely-extended plane, rather than a 
  complicated 3D shape like a violin, there is no issue of shadowing. The 
  little piston contributions can be added up, using a formula called the 
  Rayleigh integral which is explained in the next link. 

  If the top of the violin was surrounded by a rigid plane stretching off to 
  infinity, we could use the Rayleigh integral. There is no such plane, of 
  course, but at very high frequencies it might not matter very much to assume 
  that there was an '' infinite baffle'' like this, in order to get at least an 
  approximate idea of how much sound is radiated by the violin top. We won't 
  look at a complicated shape like a violin, but there is a very simple problem 
  which illustrates the idea, sketched in Fig.\ 10. We have a circular piston 
  embedded in an infinite baffle, and the piston is made to vibrate in and out 
  as a rigid body: this is a crude model of a loudspeaker in a rigid enclosure. 

  \fig{figs/fig-be2b3436.png}{Figure 10. A piston set in a plane baffle} 

  As explained in the previous link, the Rayleigh integral can be used for this 
  problem to give a closed-form mathematical answer to the directional pattern 
  of sound radiation in the far field. Some examples are plotted in Fig.\ 11, 
  associated with different values for the Helmholtz number $ka=2 \pi f a/c$, 
  where $a$ is the radius of the piston and $f$ is the frequency in Hz. When 
  $ka$ is small the radiation is essentially omnidirectional, as we should by 
  now expect. As $ka$ increases, the pattern gets more and more focussed into a 
  beam of sound heading straight ahead from the piston, with very little being 
  radiated in other directions. This shows immediately why your hi-fi 
  loudspeakers need a ``tweeter'', a small loudspeaker unit whose job is to 
  radiate the higher frequencies. A smaller value of $a$ means that the beaming 
  effect is put off until higher frequency, otherwise you would only be able to 
  hear the high-frequency sound by sitting directly in front of your 
  loudspeaker. 

  \fig{figs/fig-4debd45b.png}{Figure 11. Directional radiation patterns for the 
  baffled piston, for Helmholtz number $ka$ equal to 0.3 (blue); 1 (red); 3 
  (yellow); 10 (magenta) and 30 (green)} 

  Real loudspeakers do not, of course, have this infinite baffle. But this 
  simple model gives a good semi-quantitative impression of how a speaker will 
  behave: once the beaming effect gets established, shadowing by a real 
  finite-sized loudspeaker cabinet doesn't matter very much because most of the 
  sound is being projected forwards. The same model gives at least a 
  qualitative idea of the sound radiation from the bell of a brass instrument 
  like a trumpet. For this case there isn't even a cabinet looking a little bit 
  like the infinite plane baffle, but the trend shown in Fig.\ 11 still gives 
  the right impression, of sound increasingly concentrated into a beam heading 
  straight ahead as frequency goes up. 

  An interesting, and slightly counter-intuitive, example of directional 
  radiation is illustrated in Fig.\ 12. In the days before GPS navigation, an 
  important way to improve safety for ships in dangerous waters was to use 
  foghorns to give an audible warning. Lord Rayleigh, who we have met several 
  times before, was consulted about an observation that was puzzling the 
  authorities: ship's captains reported that in certain positions they were 
  unable to hear the foghorn, even when they were quite close. Rayleigh 
  realised the problem: the horns were too large, so that the sound radiation 
  was strongly directional. At certain angles, the radiated sound was very 
  weak. The cure was to use a horn with an elliptical section. The aim is to 
  spread the sound as uniformly as possible in the horizontal plane, while 
  concentrating it in the vertical so that sound energy was not ``wasted'' by 
  being sent up into the sky. This requires a horn that is larger than a 
  wavelength in the vertical direction, but smaller than a wavelength in the 
  horizontal direction, exactly as seen in the picture. 

  \fig{figs/fig-ea42589d.png}{Figure 12. An elliptical foghorn, based on a 
  design by Lord Rayleigh. This example is near Whitby in Yorkshire.} 

  To finish this chapter, we look at another important phenomenon which is most 
  easily understood by thinking about infinite systems. This concerns the 
  radiation of sound by the vibration of a bending plate. We are accustomed to 
  the idea that there is a single speed of sound, the same for all frequencies. 
  But for bending waves in plates or beams, things are more complicated. At low 
  frequency, bending waves travel slowly, but the speed increases as frequency 
  goes up. This has an important consequence. For sufficiently low frequencies, 
  waves in a plate travel slower than the speed of sound. As explained in the 
  next link, this means that they cannot generate far-field sound waves in the 
  adjacent air. But at higher frequency, the plate waves move faster than the 
  speed of sound, and now they can generate sound waves. Animations of the two 
  cases are shown in Figs.\ 13 and 14. The crossover frequency, where the plate 
  waves move exactly at the speed of sound, is called the critical frequency of 
  the plate. 

\moobeginvid\begin{tabular}{ccc} \vidframe{ 0.30 }{ vids/vid-1c1ff49c-00.png }&\vidframe{ 0.30 }{ vids/vid-1c1ff49c-01.png }&\vidframe{ 0.30 }{ vids/vid-1c1ff49c-02.png } \end{tabular}\caption{Figure 13. Sound field generated by a travelling wave in a plate at a frequency 10\% below critical}\mooendvideo

\moobeginvid\begin{tabular}{ccc} \vidframe{ 0.30 }{ vids/vid-42c71f71-00.png }&\vidframe{ 0.30 }{ vids/vid-42c71f71-01.png }&\vidframe{ 0.30 }{ vids/vid-42c71f71-02.png } \end{tabular}\caption{Figure 14 Sound field generated by a travelling wave in a plate at a frequency 10\% above critical}\mooendvideo

  This has important consequences for many things, including the transmission 
  of sound through buildings or into the passenger compartments of vehicles. 
  The design of soundproof partitions and double glazing systems must take the 
  critical frequency into account, because the behaviour will be different 
  above and below that frequency. 

  Our interest in the idea is mainly concerned with radiation of sound by the 
  bodies of stringed instruments. Things are made a little complicated by the 
  fact the soundboards of stringed instruments are usually made of wood like 
  spruce, which has very different stiffnesses in the directions along the 
  grain and across the grain. For such plates there is not a single critical 
  frequency: waves travelling at different angles to the grain will have 
  different critical frequencies, so the transition is spread out over a range 
  of frequencies. The broad behaviour is still the same, though. At high 
  frequency, any bending plate can radiate sound efficiently, but at low 
  frequency it is much less efficient. 

  Now, the argument from an infinite plate suggests that at low frequency a 
  vibrating plate could radiate no sound at all! This would be rather 
  embarrassing for the design of musical instruments. Fortunately, finite 
  plates escape from this fate: a plate of finite size can always radiate some 
  sound even at frequencies well below critical. The efficiency with which they 
  can do so varies from mode to mode, governed by another variant of Helmholtz 
  number. What matters here is the wavelength of plate vibration compared to 
  the size of the instrument. Low modes involving just a few half-wavelengths 
  can radiate relatively well, but modes involving more wavelengths behave in a 
  way that is closer to the infinite system, and do not radiate well until the 
  critical frequency is passed. 



  \sectionreferences{}[1] George Bissinger and John Keiffer: ``Radiation 
  damping, efficiency and directivity for violin normal modes below 4 kHz'', 
  Acoustics Research Letters Online \textbf{4} (2003), DOI 10.1121/1.1524623. 

  [2] Gabriel Weinreich: ``Directional tone color'', Journal of the Acoustical 
  Society of America, \textbf{101}, 2338 (1997). 