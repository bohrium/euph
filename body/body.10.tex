  This chapter is about measurements and experiments, in all their guises 
  relevant to the study of musical instruments. The first section gives a 
  discussion of some important background material: What are the different 
  types of measurement and experiment? What are the different motivations for 
  wanting to do measurements? What are the pitfalls you need to avoid in the 
  design of any experiment? 

  The next section looks at various experimental techniques for “seeing the 
  invisible”: microscopy of various kinds, CT scanning (similar to the use of a 
  medical scanner) to see inside things, and a method for using acoustical 
  measurement to reconstruct the internal bore profile of a wind instrument. 

  After that comes a section devoted to the particular question of describing 
  and measuring the properties of wood. Many musical instruments are, of 
  course, made of wood, and the selection of suitable wood is a big concern for 
  many instrument makers. There are two separate threads to this concern. 
  First, for many instruments the vibrational properties of wood are directly 
  important: density, various kinds of stiffness, and the level of internal 
  damping. These factors can all be quantified by measurement, and this may 
  help makers when selecting wood. The second thread is to do with 
  sustainability. Many of the traditional timbers used by instrument makers are 
  tropical hardwoods, and in many cases these are now threatened, and trade in 
  them is legally restricted. Makers are very interested in alternative 
  materials, whether that means other timber species that are more sustainable, 
  or new man-made materials. 

  Section 10.4 addresses something we have encountered many times already. For 
  many purposes, it is valuable to measure a frequency response function of 
  some kind. This section gives an account of some aspects of such 
  measurements. There are many options for actuators and sensors, each with 
  their own advantages and disadvantages. The choice depends in part on having 
  a clear sense of what exactly you want to get out of your measurement, and 
  how accurately you need to know it. Finally, there is a subsection devoted 
  specifically to measurements involving microphones and radiated sound. Any 
  such measurements raise important issues to do with the acoustics of the room 
  in which the measurement is done, as some examples illustrate. 

  Section 10.5 described one particular use to which frequency response 
  functions may be put: experimental modal analysis. We have already seen some 
  examples of measured mode shapes; this section gives a more detailed 
  description of how the measurement method works. 

  The final section addresses a different type of measurement technique: it 
  illustrates some ways to visualise fluid flows and acoustic sound fields. 
  Fluid flows will become important in the next chapter, when we look in some 
  detail at wind instruments. 

  In writing this rather wide-ranging chapter I have benefited enormously from 
  conversations with several friends and colleagues: especially Evan Davis and 
  Claire Barlow. Many of their good ideas are incorporated into what follows. 
  Other colleagues have also freely contributed material for inclusion here, in 
  areas that lie outside my own comfort zone: I thank Murray Campbell, David 
  Sharp, Mico Hirschberg, Earl Williams, Lily Wang, George Stoppani and Rudolf 
  Hopfner from Violinforensic of Vienna. 

