

  Now to see some examples of the psychoacoustical approach applied to 
  questions specifically about musical instruments. These questions divide into 
  two categories: the ones we would really like to ask, and the ones that are 
  straightforward to answer. The first category includes all questions 
  involving judgements of quality: “which violin do you prefer?”; “which sound 
  is more rich?” (or “lively” or “nasal” or “shrill”); “is this an old Italian 
  violin or a modern one?” We will put off all such questions until the next 
  section. 

  For now, we will concentrate on the second category. This involves questions 
  about the threshold of perception, or “just-noticeable difference”, when some 
  parameter of the sound is varied. It is clearest to explain through an 
  example. Back in section 5.4 we met the idea that a real musical string does 
  not have perfectly harmonic overtones, because of the effect of bending 
  stiffness (the details were given in section 5.4.3). Does this effect matter 
  for the sound of real strings? A natural way to start exploring that question 
  is to ask “what is the smallest bending stiffness that produces an effect 
  people can hear?” 

  The good thing about this question is that there is a reasonably 
  straightforward way to answer it. We generate a lot of sounds with varying 
  inharmonicity corresponding to different levels of bending stiffness, then 
  persuade a lot of volunteers to listen to pairs of these sounds, one with and 
  one without the bending stiffness. They are asked to say which is which, then 
  the level of bending stiffness is reduced, and they do it again. Eventually, 
  the stiffness will get so small that their responses will be random: they 
  can’t actually hear the difference, and they are simply guessing. This is the 
  threshold of perception. 

  To give an idea of what kind of sounds might be used, Sound 1 gives a 
  sequence of vaguely string-like synthesised sounds with different levels of 
  inharmonicity due to bending stiffness. The first note, and every alternate 
  one after that, has perfectly harmonic overtone frequencies. In between each 
  of these pairs is a sound with some inharmonicity, and the amount increases 
  each time. You probably can’t hear any difference with the first pair, but by 
  the end of the sequence you probably hear a very clear difference. Somewhere 
  in between would be your threshold of perception — for a note with this 
  particular frequency, amplitude, decay time and relative energy level between 
  the overtones. If any of those things was changed, the threshold might be 
  different. 

  To do this experiment seriously requires much more care, of course. The 
  threshold is not a sharply-defined thing. As it is approached, you get less 
  and less good at recognising if there is a difference or not. In order to 
  arrive at a particular value for the threshold, which can be compared with 
  estimates made by other people, an agreed definition must adopted which is 
  based on statistical distributions and the probability of making a correct 
  judgement in a particular case. Usually, an ingenious procedure is followed 
  in which the sounds are presented to the test subject in a 
  carefully-constructed sequence, which automatically converges to an estimate 
  of the threshold. Levitt [1] gives the statistical analysis to define 
  precisely what that estimate means in terms of the probability of success. 

  In a typical test, at each step the subject hears three sounds, and has to 
  pick the odd one out: a so-called “three-alternative, forced-choice” test. 
  The order of the three sounds is randomised. If the subject gets it right 
  three times in a row, the magnitude (of bending stiffness in our example) is 
  reduced. If they get it wrong just once, it is increased. The factor by which 
  the bending stiffness changes will also be varied, so that once the sounds 
  are in the vicinity of the threshold, finer gradations are used. After a 
  predetermined number of up-and-down reversals, the experiment is stopped and 
  a suitable average taken of the values of bending stiffness in the final 
  stages. 

  A careful study of this kind was carried out by Järveläinen et al [2], using 
  string-like synthesised sounds somewhat similar to the ones in Sound 1 here. 
  They covered the range of playable notes on a guitar, and compared the 
  deduced thresholds with the measured inharmonicity of notes on a classical 
  guitar. The result was that inharmonicity in the real guitar exceeded the 
  threshold for perception on all notes and all strings, most strongly for 
  notes on the 3rd and 6th strings of the guitar. 

  But by now you should not be surprised to learn that there are complications. 
  The same team carried out another study, using sounds that were based much 
  more accurately on the played guitar notes. This brought in factors like the 
  nature of the starting transient, and the frequency dependence of the decay 
  rates of the different overtones. Using these more complicated but more 
  realistic sounds, they found that the thresholds of perception were higher 
  [3]: sufficiently so that the inharmonicity of most played notes was below 
  threshold. 

  These later results do not invalidate the earlier ones, but they show that 
  considerable care is needed over the interpretation of such tests. We can 
  give arguments to suggest that both results are of interest: it all depends 
  on exactly what aspect of perception you are trying to probe. To argue for 
  the importance of the original results, we first remind ourselves that a 
  musician may develop very finely-tuned perception for sound details on their 
  own instrument. In the context of the inharmonicity question, we must 
  remember that the player can vary the way they play the notes on a particular 
  string, and as they strive for the sound they ``hear in their head'' they 
  will hear many variants of the sound, not just the particular sounds chosen 
  for the psychoacoustical test. 

  It can thus be helpful to know the threshold of perception for a particular 
  change, when we arrange the details of the test procedure to give the 
  listener the best possible chance of hearing it. This should result in what 
  we might think of as the ultimate threshold: beyond that point, it is not 
  humanly possible to perceive this particular change, however subtle and 
  finely-tuned the musician’s feature detectors may be. 

  On the other hand, the second inharmonicity experiment shows that under 
  different circumstances the realistic threshold of perception may be 
  significantly higher. The subject cannot achieve the ultimate threshold 
  because of what is called informational masking: other aspects of the sound 
  compete for the attention of your hearing system, and somehow confuse your 
  ability to hear the particular thing under study. 

  What other questions could we apply this kind of approach to? Well, by 
  recalling something from Chapter 2 we can explore an important issue for 
  instrument makers, to do with the effect of changes they may make to the 
  constructional details of a stringed instrument body. Provided the vibration 
  amplitude is small enough, as it usually is, we can treat the body using 
  linear theory. We then showed in section 2.2 that all we need to know about 
  the body is the behaviour of the vibration modes: natural frequencies, mode 
  shapes and damping factors. Constructional details influence the modes, and 
  the modes determine the sound. So it makes sense to explore the perception 
  threshold for changes in these modal parameters. In the light of the 
  discussion in section 5.3, it would also be interesting to explore perception 
  thresholds associated with any formant-like features. 

  Some initial studies on this question have been done, for the guitar and for 
  the violin. These two instruments require rather different approaches, so we 
  will discuss them separately. Back in section 5.4 we already met some of the 
  computer-synthesised sounds used in the guitar investigation [4]: they are 
  repeated below as Sounds 2, 3 and 4. Based on the measured response of a 
  particular guitar, these sounds show the effect of raising or lowering all 
  the body mode frequencies by one semitone (6\%). This change is clearly 
  audible, and it is not surprising to learn that the threshold for such a 
  shift of all mode frequencies turned out to be significantly lower: about a 
  1\% shift for the most acute listeners, and the most acute of them all could 
  detect a change as small as 0.3\% under the best conditions. Furthermore, 
  these most acute listeners could detect a frequency shift around 1\% for 
  moving just a few of the modes: specific tests were carried out shifting some 
  “signature modes” in the frequency range 150—250 Hz, and a cluster of modes 
  lying in the band 500—1000 Hz. 

  When a similar threshold experiment was carried out in which the modal 
  damping factors (or Q factors) rather than their frequencies were changed, 
  listeners were much less sensitive. The threshold for hearing a change in 
  modal damping turned out to be around 20\%. To give an idea of the effect on 
  sound of changing the damping of all body modes, Sounds 5--8 give examples. 
  Sound 6 is the reference case. Sounds 5 and 7 have all the Q factors changed 
  by a factor of 2, down and up respectively. Even with this large change, the 
  sound is not very different. Finally, Sound 8 illustrates the effect of 
  multiplying all modal Q factors by 4, and at last there is a clear change in 
  sound, to something rather ``boomy''. But we can conclude that modal damping 
  factors are far less important to sound than modal frequencies. 

  To explore the same question about changing the body modes of a violin 
  requires a different method. The problem is that realistic synthesis of 
  violin playing is much more challenging than was the case for guitar playing. 
  The essential reason is that plucking a string can be treated quite well 
  using linear theory, but bowing a string definitely can’t be. Violin playing 
  is strongly nonlinear, introducing all manner of complications, as we will 
  discuss in detail in Chapter ?. For the moment, we just need to note that 
  attempts to do psychoacoustical testing using synthesised violin sounds 
  results in a severe case of informational masking: listeners are so disturbed 
  by the fact that it doesn’t really “sound like a violin” that they do not 
  give their best judgements about the thing the experimenter is trying to 
  test. 

  However, we can rescue our experiment by noting that while the motion of a 
  bowed string is nonlinear, the resulting vibration and sound radiation by the 
  violin body is probably not. So we can address questions about the 
  sensitivity of sound to modifying the body modes by splitting the system into 
  these two components. Instead of trying the synthesise the string motion, we 
  can measure it directly using a laboratory version of an electric violin. 
  Small force-measuring sensors can be embedded into the bridge of a violin, 
  just underneath each string notch: Fig.\ 1 shows what it looks like. 

  A violinist can then play in the normal way, and the signal from the force 
  sensors can be recorded. An example of what this sounds like is given in 
  Sound 9. It is quite recognisable as violin playing, but the sound is rather 
  muffled and characterless. We can then take the measured response of a violin 
  body, or a simulated response after some desired modification, and combine it 
  with the measured force to give a prediction of the ``sound'' of the violin. 
  This combining process is called convolution: it was described, with an 
  example, back in section 2.2.8. The virtue of this approach is that it allows 
  many different ``virtual violins'' to be heard, while the input from the 
  violinist remains exactly the same. 

  I put ``sound'' in quotes in the previous paragraph, because what you get 
  from the convolution calculation depends on what kind of body response has 
  been measured. If it is the bridge admittance, like the examples discussed in 
  Section 5.3, then what would be calculated is the waveform of structural 
  vibration at the bridge. On the other hand, if the body response has been 
  measured with a microphone at some particular position, then what will be 
  computed will represent the sound that same microphone would pick up when the 
  violin is played. 

  It might seem more natural to use responses measured by microphone, so that 
  we really do model ``sound'' by the convolution process. However, as we have 
  seen in earlier problems, there is a snag: if we want to do experiments in 
  which virtual changes are made to the instrument response, we must use a type 
  of response for which we have a good theoretical model. This is the case for 
  an admittance, or other structural response, but it is not true for a 
  microphone response. So for psychoacoustical experiments, admittance is the 
  best response to use. 

  Examples of the output of convolution using measured bridge admittance of 
  three different violins are given in Sounds 10, 11 and 12. They are all based 
  on the recorded string signal from Sound 9. None of these violins was of 
  particularly high quality, but the sounds are distinctly different in the 
  three cases. All three sound significantly different from the original string 
  sound. 

  This “virtual violin” approach has been used to perform a range of threshold 
  experiments [5], somewhat similar to the guitar-based tests described 
  earlier. Both the amplitude and frequency of body modes were changed: for 
  individual modes at low frequency (the “signature modes” A0, B1- and B1+ 
  shown in Section 5.3), and also for blocks of modes lying in different 
  frequency bands. Some examples of the effect of shifting all mode frequencies 
  can be heard in Sounds 13--16. The emphasis throughout was to fine-tune the 
  details of the tests in order to obtain the best possible discrimination, 
  with a view to estimating “ultimate thresholds”. It was found that lower 
  thresholds were obtained from tests based on single notes, rather than more 
  extended snatches of music. Less surprisingly, it was found that listeners 
  with musical training did better than others. 

  Results for the best 5 listeners in each test were analysed. In broad 
  summary, the thresholds were in the region of 2--4 dB in amplitude and 1--4\% 
  in frequency: towards the high end of those ranges for individual modes, 
  towards the low end for bands. These results for frequencies are comparable 
  with the values found in the guitar tests, although consistently a little 
  higher. Perhaps the more complex sounds of violin playing, compared to 
  synthesised guitar ``playing'', led to some informational masking. 
  Encouragingly, it was shown that good predictions for the various thresholds 
  could be obtained based on differences between excitation patterns, as 
  discussed in Section 6.4. 

  The examples described in this section have demonstrated that some 
  interesting things can be learned by measuring thresholds of perception. The 
  tests are somewhat laborious to carry out, but there is a well-established 
  methodology: it is mostly a matter of taking endless care over details, and 
  putting in the hours of work. Without a doubt, there is considerable scope 
  for more studies of this kind: for example, the approach could be applied to 
  the various parametric changes illustrated in Section 5.5 for a model of the 
  banjo. But it is time to turn to the second category of questions we would 
  like to attack by psychoacoustical tests. 



  \sectionreferences{}[1] H. Levitt. ``Transformed up-down methods in 
  psychoacoustics''; Journal of the Acoustical Society of America \textbf{49}, 
  467--477 (1971). 

  [2] H. Järveläinen, V. Välimäki and M. Karjalainen. ``Audibility of the 
  timbral effects of inharmonicity in stringed instrument tones''; Acoustics 
  Research Letters Online \textbf{2}, 79--84 (2001).<span 
  style="text-decoration: underline;"></span> 

  [3] H. Järveläinen and M. Karjalainen. ``Perceptibility of inharmonicity in 
  the acoustic guitar''; Acta Acustica united with Acustica \textbf{92}, 
  842–847 (2006). 

  [4] J. Woodhouse, E. K. Y. Manuel, L. A. Smith and C. Fritz. ``Perceptual 
  thresholds for acoustical guitar models''. Acta Acustica united with Acustica 
  \textbf{98}, 475-486, (2012).~ DOI 10.3813/AAA.918531 

  [5] C. Fritz, I. Cross, B. C. J. Moore and J. Woodhouse. ``Perceptual 
  thresholds for detecting modifications applied to the acoustical properties 
  of a violin''; Journal of the Acoustical Society of America \textbf{122}, 
  3640–3650 (2007). 