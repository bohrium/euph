  \tt{}Euphonics\rm{} 

  The science of musical instruments                           ISSN 2977-5612 

  Figure 1 shows a violin bridge equipped with four force sensors, to allow the 
  transverse force from the vibrating strings to be captured as electrical 
  signals. Two small rectangles of piezoelectric crystal are used, oriented to 
  respond to compressive force through the thickness. Such sensors have high 
  electrical impedance, so the output of the sensor needs to be fed into the 
  same kind of pre-amplifier that would be used with a piezoelectric 
  accelerometer: a ``charge amplifier''. 

  \fig{figs/fig-ee256746.png}{\caption{Figure 1. A violin bridge with four 
  force sensors installed. The wires carrying the signals away can be seen 
  behind the bridge.}} 

  The construction of each sensor is shown in diagrammatic form in Fig.\ 2. The 
  string rests on a small triangular piece of insulating polymer, such as 
  perspex. This triangle has conducting copper wrapped around its two lower 
  faces. These make contact with one face of the two crystal pieces. The other 
  faces of the crystal pieces rest on two more copper electrodes, fixed to a 
  V-shaped supporting piece of perspex. Wires are connected to these, to take 
  the signal to the charge amplifier. In reality these wires need to be either 
  coaxial cable or a twisted pair, to minimise electrical hum pickup. All the 
  contacting surfaces will be held firmly together by the static force from the 
  string once the sensor is assembled on the instrument, but for initial 
  assembly it can be helpful to use electrically-conducting adhesive of some 
  kind, for example the carbon-loaded sticky pads used to secure specimens in a 
  scanning electron microscope. 

  \fig{figs/fig-385edc34.png}{\caption{Figure 2. Diagram of a bridge force 
  sensor. Blue rectangles indicate blocks of piezoelectric crystal, red lines 
  indicate copper conducting strip, and attached wires. The shaped pieces above 
  and below the crystals are made of insulating polymer such as perspex.}} 

  The sensor will respond either to the vertical component of force from the 
  string, or the horizontal component: it will depend on the relative 
  orientations of the two crystal pieces. The crystals are connected in series, 
  so that the combined output will be the sum of the two voltages generated. A 
  downward vertical force component will put both pieces in compression, and if 
  that is the desired output then the two crystal pieces would need to be 
  placed opposite ways up, so that both give signals with the same polarity in 
  the series combination. But it is more usual to want the transverse force 
  component. A transverse force will increase the compression in one crystal 
  while reducing the compression in the other. For this output, the two 
  crystals need to be arranged the same way up: the series electrical 
  connection will then result in a voltage corresponding preferentially to 
  transverse force. When making sensors like this, it is easy to lose track of 
  which way up the crystals are! A useful tip is to mark one face with graphite 
  pencil (which is electrically conducting and will not interfere with the 
  contacts). 