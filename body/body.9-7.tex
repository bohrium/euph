

  So far, we have talked a lot about bowing but we haven’t actually included a 
  real violin bow in the discussion. The measurements have used a rosin-coated 
  rod, and the computer simulations have assumed a rigid “bow” acting at a 
  single point on the string. We must now look to see what is different if a 
  conventional violin bow is used in place of a rigid rod. 

  First, we need to understand the anatomy of bows. A carefully-shaped stick, 
  usually made from a wood called pernambuco, has a cranked tip carved into it. 
  This tip carries one end of the ribbon of bow-hair. The other end is carried 
  by the “frog”, a block of wood (usually) which can slide along the stick, 
  secured by a threaded screw within the stick which can be used to tension the 
  hair. In violin bows of professional standard, the ribbon consists of 150—200 
  strands of horsehair, in a band approximately 10~mm wide. 

  The key figure in the development of the modern violin bow was \tt{}François 
  Tourte\rm{}, working in France in the late 1780s. Before Tourte, bows for 
  stringed instruments usually had a slight upward curve in the stick. Tourte 
  reversed this, introducing the down-curved stick we see today. You can see a 
  typical modern bow compared to an earlier bow (a treble viol bow) in Fig.\ 1. 
  Figure 2 shows a close-up of the two bow tips. 

  \fig{figs/fig-e7417636.png}{Figure 1. A modern violin bow (lower), and a 
  treble viol bow (upper)} 

  \fig{figs/fig-8e49ce95.png}{Figure 2. Closeup of the the tips of the two bows 
  in Fig. 1.} 

  An important consequence of the downward curvature of the bow stick was 
  highlighted by Askenfelt and Guettler [1]: the Tourte form allows 
  significantly higher tension in the bow hair. As the hair is shortened, the 
  tension tends to rotate the tip, and thus bend the stick upwards. If the 
  curvature of the stick is already upwards, this extra curvature tends to make 
  the stick less able to resist the axial force of the hair tension. But the 
  Tourte stick gets straighter, not more curved, as the hair tension is 
  increased within the usual working range. 

  Players are in no doubt that the bow is an important thing in its own right. 
  They will have definite preferences for particular bows, and they may be 
  prepared to spend a surprisingly large amount of money on a bow: a 
  significant fraction of the value of the instrument itself. The physics 
  behind this influence of the bow has proved quite elusive to understand. Some 
  things are known, but if anything the question of preferences between bows 
  remains even more mysterious than the corresponding question about 
  preferences between violins (or cellos, or whatever). There are several 
  possible ways that the bow can influence the sound and playing “feel” of a 
  violin, and I will group them into three categories for this discussion. 

  An expert violinist requires their bow to perform all kinds of sophisticated 
  tricks, and my first category of physical attributes of a bow contains things 
  which affect how the bow “feels in the hand”, and which thus contribute to 
  the ease (or otherwise) of performing these tricks. If you hold a violin bow 
  in the normal way, then wave it around in the air in the plane of the bow, 
  elementary mechanics tells us that just three parameters are enough to 
  determine how it will feel and behave. One of them is the total mass, and the 
  other two are determined by how that mass is distributed: the stick is 
  tapered, and it has additional mass near the ends in the form of the frog and 
  the tip. 

  We could use various possible parameters to characterise this mass 
  distribution, but there are two particular ones that are commonly used by 
  bow-makers. First, and most intuitive, is the balance point (or “centre of 
  mass”). The second parameter is something called the “centre of percussion”, 
  and as well as completing our characterisation of the mass distribution it 
  also relates to an important aspect of performance with a bow. Some kinds of 
  bowing make use of the way the bow can rebound from the string. Examples are 
  the bowing styles known as ricochet, spiccato and sautillé. You can see 
  demonstrations of these, and many other kinds of bowing, in \tt{}this YouTube 
  video\rm{}: go to the times 0:20 for ricochet, and 1:51 for 
  spiccato-sautillé. 

  The definition of the centre of percussion goes like this. Imagine hanging 
  the bow from the point marked “pivot” in Fig.\ 3: it is the position where 
  the player will usually place their thumb when holding the bow in the normal 
  way for classical music. Now think of tapping the bow-hair with a pencil. If 
  you tap at a position close to the pivot, there will be a reaction force at 
  the pivot, and it will be in the opposite direction to your tap. But if you 
  tapped right at the other end, near the tip, the reaction force would be in 
  the same direction as your tap, because it has to restrain the bow from 
  rotating about somewhere near the middle. Somewhere in between these two 
  tapping positions, there will be one particular place where there is no 
  reaction force at all. This is the “centre of percussion”. It is marked with 
  a green arrow in Fig.\ 3: it is about 2/3 of the way down the bow from the 
  pivot point. The next link gives some technical details about all this. 

  \fig{figs/fig-98160f26.png}{Figure 3. The violin bow from Fig. 1, labelled to 
  show the balance point and centre of percussion} 

  Another way to describe the effect would be to think of lying the bow on its 
  side on an ice rink so that it could slide around any way it wanted, and then 
  tapping the hair in a horizontal direction with your pencil. When you tap 
  exactly at the centre of percussion, the bow will start to move by rotating 
  about the “pivot” point, even though there is no pivot present this time. If 
  you tap anywhere else, the “pivot” point will move: in one direction if you 
  are closer to the frog, and in the opposite direction if you are further from 
  the frog. 

  When a player wants to perform ricochet or spiccato notes, they need the 
  natural bouncing behaviour of the bow to do some of the work for them. This 
  in turn depends on the position of the bowing point along the bow, partly 
  through the influence of the centre of percussion. It seems reasonable that 
  very rapid ricochet playing might be easier if the bowing point is fairly 
  near the centre of percussion, so that the bow naturally wants to rotate 
  around the player’s thumb during the bouncing action. Askenfelt and Guettler 
  did an interesting study of “the perfect spiccato” [2], and they suggest that 
  such strokes are ``always played well inside the centre of percussion (about 
  10~cm)''. For the bow in Fig.\ 3, that position would lie halfway between the 
  balance point and the centre of percussion. They also point out that such 
  bowing techniques only became possible with the higher hair tension available 
  from a Tourte bow. Viol players do not do such flashy bowings! 

  There is another aspect of the dynamics of a bouncing bow that is very 
  important for a player aiming to perform rapid spiccato or ricochet. When the 
  bow is in contact with a string, it has a resonant bouncing frequency 
  determined by the inertia of the bow and a stiffness coming mainly from the 
  bow-hair tension. This resonance frequency varies strongly with position 
  along the bow, and also with bow-hair tension. Askenfelt and Guettler [1] 
  gave a simplified theoretical expression for this: Fig.\ 4 shows a version of 
  their plot. The next link describes their model. They tested a professional 
  violinist with a wide variety of bows, asking them to play the same rapid 
  spiccato with each one. They found that the player did indeed adjust the hair 
  tension and bowing point for each bow so as to create essentially the same 
  bouncing frequency every time. 

  \fig{figs/fig-f3ebfffe.png}{Figure 4. Bow-bouncing resonance frequency as a 
  function of position along the bow hair, according to the simplified theory 
  of Askenfelt and Guettler [1].} 

  We see a hint of a different aspect of how a bow “feels in the hand” if we 
  make a link with material from the two previous sections. Figure 5 shows a 
  measured Guettler diagram using a real bow in the Galluzzo experimental rig, 
  compared with the one we have seen before measured on the same rig with the 
  rigid rod (repeated here as Fig.\ 6). The green and magenta lines will be 
  explained in a moment. At first glance, these two plots look fairly similar. 
  They both show a vaguely wedge-shaped region of transients that led to 
  successful Helmholtz motion, and they both include quite a few white or 
  yellow pixels indicating very short transient length. Both show some 
  “speckly” texture, suggesting a degree of “sensitive dependence”. 

  \fig{figs/fig-2c47e34c.png}{Figure 5. Guettler diagram, measured in the 
  Galluzzo rig using a real cello bow.} 

  \fig{figs/fig-af307b6d.png}{Figure 6. Measured Guettler diagram as seen in 
  section 9.5, using a rosin-coated rod.} 

  One clear difference between the two figures is in the position and shape of 
  the lower boundary of the region of coloured pixels. Figure 5 shows more 
  black pixels than Fig.\ 6 in the lower left corner, but the pattern is 
  opposite on the right-hand side: Fig.\ 6 shows a boundary curve that rises 
  more steeply than the one in Fig.\ 5. We can suggest a tentative explanation 
  for at least part of this difference, and it will tell us something 
  interesting about the “feel” of a bow. 

  Back in section 9.6, we looked at alternative models for friction. But one 
  aspect of that discussion was deferred until later: I said we should review 
  the Amontons-Coulomb “law” that friction force is proportional to the normal 
  force (i.e. to the bow force in our case). The time has now come to look at 
  this issue. The Amontons “law” is so familiar to anyone who has studied 
  mechanics or physics at any level that many people are surprised to learn 
  that there is any question about it. But the result is a purely empirical 
  finding, and the generally accepted explanation involves something 
  unexpectedly subtle. 

  Amontons published his “laws of friction” in 1699 (although in fact this was 
  a re-discovery: they were stated in the notebooks of Leonardo da Vinci some 
  200 years earlier, but Leonardo never published them). The first two of these 
  laws state that the force of friction is proportional to the applied load, 
  and independent of the apparent area of contact between the sliding surfaces. 
  The original experimental evidence gathered by Amontons and others involved a 
  classic experiment with an object like a brick resting on an inclined plane 
  (such as a wooden plank): the angle of the plank is slowly increased, until 
  the brick starts to slide. 

  If you were to examine the surfaces of the brick and the wooden plank under a 
  microscope, neither of them would be smooth. Instead, they would be covered 
  with small lumps and bumps known in the jargon as asperities. The result 
  would be something like what is sketched in Fig.\ 7: actual contact between 
  the two surfaces only occurs near the tips of these asperities. The real area 
  of contact will be much smaller than the apparent area of contact. 

  \fig{figs/fig-8520ace5.png}{Figure 7. Schematic sketch of one rough surface 
  sliding over another, with contact only at the tips of asperities.} 

  The accepted explanation for Amontons’ laws then goes like this. The friction 
  force is determined by the shear strength of the interface between the two 
  materials, acting over the real area of contact. So under the simplest 
  assumption, friction force is directly proportional to real area of contact. 
  When the applied load is increased so that the surfaces are pressed together 
  more firmly, there is a bit of deformation near the asperity tips, with the 
  result that each individual contact gets a bit bigger, and also additional 
  asperities may come into contact. In a classic study by Greenwood and 
  Williamson [3], it was shown that under reasonable assumptions about the 
  statistical description of the surface roughness, the real area of contact 
  ends up being proportional to the applied load. Putting these two things 
  together, the friction force is proportional to the load, as Leonardo and 
  Amontons found. 

  What does all this have to do with our two Guettler diagrams? Well, in the 
  experiment with the rosin-coated rod we would probably not expect this 
  argument based on rough surfaces to work. When a string is bowed by a rod, 
  the geometry is like two cylinders, crossing at right angles. The apparent 
  area of contact is very small, so the contact pressure (force per unit area) 
  will be very large. Rosin is not a very hard material, especially when it has 
  been warmed up by friction. With this large contact pressure, any asperities 
  that may have originally been present on the surfaces of the rosin-coated 
  string and rod will be squashed flat. The real area of contact is then the 
  same as the apparent area of contact. 

  Under those circumstances, the argument based on rough surfaces goes out of 
  the window. Instead, we would expect the material near the contact to behave 
  in a way first described in 1882 by Heinrich Hertz (the same scientist that 
  our unit of frequency is named after). You can find a description of this 
  “Hertzian contact”, and of the Greenwood-Williamson model of rough surfaces, 
  on \tt{}this Wikipedia page\rm{}. For a Hertzian contact, the friction force 
  is not proportional to the applied load: instead, it is proportional to the 
  2/3 power of that load. 

  So now look again at the Guettler diagram in Fig.\ 6. Guettler’s original 
  argument suggested that the wedge of short transients would be bounded by a 
  straight line — but he was assuming the Amontons-Coulomb law. What his 
  criterion really calls for is that the friction force is proportional to the 
  acceleration. If the rod-string contact is Hertzian, the boundary should show 
  bow force proportional to the acceleration raised to the power 3/2: a rising 
  curved line, not a straight line. The magenta line in Fig.\ 6 shows what that 
  looks like: it doesn’t do a bad job of tracking the lower boundary of the 
  coloured pixels. 

  Now what about the Guettler diagram measured with a real bow? We can make a 
  guess. The ribbon of bow-hairs must produce an effect which is rather like 
  the rough surfaces of Greenwood and Williamson. Each individual contact 
  between the string and a single hair might play the role of an asperity 
  contact. As the bow force is increased, more hairs might come into contact 
  with the string. The result would be to resurrect something like the 
  Amontons-Coulomb law. If that were exactly true, the boundary would be a 
  radial straight line in the Guettler diagram. The green line in Fig.\ 5 shows 
  an example. It doesn’t track the boundary very well at low values of bow 
  force, where perhaps things are more complicated, but with the eye of faith 
  it does a reasonable job of tracking the boundary at higher forces. Figure 5 
  also repeats the magenta line from Fig.\ 6: it is obvious that this line 
  rises far too steeply to be consistent with the pattern of these 
  measurements. 

  This effect might, just possibly, be the main reason for using the 
  complicated arrangement of a ribbon of horsehairs to make a bow. A bow could 
  instead be strung with something like a large-diameter gut string (rather in 
  the style of an archery bow). The string could be chosen to have the same 
  total mass as the horsehair bundle, so that if it was adjusted to the same 
  tension it would have the same bouncing frequency. Such a string will accept 
  a coating of rosin just as well as horsehair does. But, despite matching all 
  aspects of the mechanical behaviour we have talked about so far in this 
  section, it is virtually impossible to play an instrument with such bow! I 
  have tried the experiment, and there is an overwhelming impression that you 
  can’t press hard enough to get the string response you are expecting. That 
  impression would be a natural consequence of the contrast between Hertzian 
  contact and Amontons' law. 

  It is time to move on to the second category of physical consequences of 
  using a bow: these are effects directly arising from the finite width of the 
  ribbon of bow-hair. The most important effect is best introduced by an 
  example. Listen to Sound 1: this is a recording using the bridge-force sensor 
  of the open G string of a violin. The player gradually slides to bow towards 
  the bridge, getting very close at the end of the sample. In the sound, you 
  should be able to hear a growth in harshness of the sound, ending with a 
  definite “crunch”. This example is intentionally extreme, but players 
  sometimes make deliberate use of this harsh sound at moderate level. 

  Figure 8 shows two extracts from this measured waveform, from early and late 
  in the sequence. Both show the familiar Helmholtz sawtooth waveform, but they 
  both have something extra superimposed. The upper trace shown occasional 
  small spikes in the waveform, which differ in detailed placement from cycle 
  to cycle. The lower trace shows a similar thing in more extreme form: it is 
  not surprising that this irregular waveform might sound harsh or noisy. 

  \fig{figs/fig-960ba1c0.png}{Figure 8. Two samples of the waveform of Sound 1, 
  extracted early and late in the sequence. Both show the Helmholtz sawtooth, 
  with irregular ``spikes'' superimposed} 

  We get a clue about what might be happening from Fig.\ 9. The upper diagram 
  shows the moment in an ideal Helmholtz motion when the travelling corner has 
  just gone past the bow, setting off towards the player’s finger. The ribbon 
  of bow-hair is indicated by the yellow stripe: deliberately shown very wide, 
  to make the effect clear. The lower diagram shows what might happen later in 
  the cycle, shortly before the Helmholtz corner gets back to the bow. In the 
  ideal version of Helmholtz motion, the string shape would follow the dotted 
  line. But if the portion of string in contact with the bow has been sticking 
  throughout this time, over the entire width of the bow-hair, the string would 
  need to take up a zig-zag shape more like the solid line. 

  \fig{figs/fig-a5619725.png}{Figure 9. Sketch of what would need to happen if 
  the string were to stick to the entire width of the bow during Helmholtz 
  motion. The upper image shows the situation just after the Helmholtz corner 
  has passed the bow. The lower image shows what would happen by the time the 
  Helmholtz corner was arriving back near the bow.} 

  There are sharp corners at the edges of the bow-hair. These produce 
  concentrated forces, in opposite directions on the two edges. This feels 
  physically unrealistic: surely we might expect the string to have slipped in 
  some region near the edges of the bow, before things got this extreme? Those 
  localised slips, at the edge of the bow facing towards the bridge, are 
  responsible for the “spikes” we saw in the bridge-force waveforms of Fig.\ 8. 
  The effect becomes more extreme when the bow moves closer to the bridge. 

  We can confirm this idea by extending the computer simulation model to allow 
  for a bow of finite width, using an approach first used by Roland Pitteroff 
  [4,5]. Models of this kind are by no means as highly developed yet as the 
  single-point models we have used so far, but nevertheless we can get useful 
  qualitative results. Some details are given the next link: we can make 
  versions of the simulation model based round all three of the friction models 
  introduced in section 9.6. 

  Figure 10 shows one example of such a simulation, for a case approximately 
  matching the results shown in Fig.\ 8. This particular example was made using 
  the friction-curve model, initialised with ideal Helmholtz motion and then 
  allowed to run for a while for the motion to settle down. The bridge force 
  waveform shows an irregular pattern of spikes superimposed on a more-or-less 
  steady Helmholtz sawtooth, qualitatively similar to the results in Fig.\ 8. 

  \fig{figs/fig-965bd2b3.png}{Figure 10. Simulated example of a bridge force 
  waveform with a finite-width bow, showing Helmholtz motion with irregular 
  ``spikes'' caused by localised slips on the side of the bow facing the 
  bridge.} 

  Figure 11 shows the corresponding pattern of sticking and slipping across the 
  width of the bow. The time axis matches Fig.\ 10, and the top edge of the 
  plot corresponds to the edge of the bow closest to the bridge. Black pixels 
  indicate sticking and white ones indicate slipping. The main Helmholtz slips 
  show up as top-to-bottom white stripes, while the partial slips causing the 
  spikes in Fig.\ 10 show up as white streaks in the upper part of the plot. 
  There are a small number of red pixels, indicating slipping in the reverse 
  direction: these occur occasionally near the bottom edge of the plot, as 
  Fig.\ 9 might have led us to expect. 

  \fig{figs/fig-b05afff4.png}{Figure 11. The pattern of sticking (black) and 
  slipping (white) across the width of the bow for the waveform section plotted 
  in Fig. 10. Red pixels show slipping in the reverse direction. The edge of 
  the bow nearest to the bridge is at the top.} 

  This model allows us to make a preliminary exploration of something 
  important. We have already discussed the effects of several variables a 
  player can control during bowing: the bow force, speed and acceleration, and 
  the position of the bow on the string. But there is one more variable, which 
  we have ignored up to now: players do not necessarily press the bowhair flat 
  against string. Instead they often tilt the bow, so that the distribution of 
  normal force is not uniform across the width. In extreme cases, they may tilt 
  sufficiently that the hairs only contact the string over part of the width. 

  Figures 12, 13 and 14 give an impression of the consequence of a tilted bow. 
  Figure 12 shows a somewhat zoomed view of Figs.\ 10 and 11, with a flat bow. 
  It shows Helmholtz motion, with some irregular “spikes” in evidence. Figure 
  13 shows a case in which all parameters are the same, except that the bow has 
  been tilted in the sense that players usually use. The normal bow force 
  varies linearly across the width of the bow, falling to zero at the edge 
  nearest to the bridge and rising to double the mean level at the opposite 
  edge. Figure 14 shows the converse case, with “incorrect” tilting so that the 
  force is higher at the edge nearest to the bridge. 

  Comparing Figs.\ 12 and 13, the bridge force waveforms look quite similar, 
  but the stick-slip patterns are significantly different. With the flat bow, 
  the white streaks denoting partial slipping at some of the bow-hairs extend 
  across almost the whole width of the bow. For the tilted case, the edge of 
  the bow nearest to the bridge is slipping for most of the time (occasionally 
  in the reverse direction) but the white streaks do not extend so far across 
  the bow. Instead, there is a somewhat larger region (at the bottom of the 
  stick-slip plot) where the bow is sticking throughout the nominal stick 
  period of the Helmholtz motion. This may give the player an extra sense of 
  security: they can “keep a grip” on the string, with less danger of 
  occasional partial slips reaching right across the bow and producing stronger 
  “spikes” in the bridge force. 

  Figure 14 shows precisely this danger in action, when the bow is tilted in 
  the opposite direction. There are now a few partial slips in each cycle which 
  reach right across the bow, and the bridge force shows far more marked 
  spikes. It is easy to believe that such a difference would be clearly 
  audible. Unless a player actively wants the “noisy” sound of spikes, they 
  should avoid this kind of “incorrect” bow tilt. 

  A more extensive collection of simulation examples is given in the previous 
  link. All three friction models are illustrated, and some results are shown 
  from scanning the Schelleng diagram, to see how Helmholtz-like motion varies 
  with bow force and bow position, and with the three different friction 
  models. That link also gives some examples of transient simulations and 
  Guettler diagrams. It is worth showing a selection of Guettler diagrams here, 
  because that gives an opportunity to summarise the state of the art in the 
  simulation of bowed-string transients, wrapping up the discussion developed 
  in sections 9.5 and 9.6. 

  Figure 15 shows a set of 6 Guettler diagrams. Three of them (in the left-hand 
  column) use finite-width models, based on the three different friction models 
  we discussed in section 9.6. Each of these is compared to a corresponding 
  point-bow model in the right-hand column — these point-bow results are not 
  quite the same as the ones showed in section 9.5, because they are using the 
  parameters for a violin G string rather than a cello D string (for 
  computational reasons explained in the previous link). All cases use the bow 
  position $\beta = 0.0899$ as in earlier figures. The three rows in the plot 
  show different friction models: the friction-curve model in the top row, the 
  original thermal model in the middle row, and the modified thermal model at 
  the bottom. In the previous link, you can see a selection of detailed 
  waveforms associated with all 6 of these plots. 

  These plots give strong hints of interesting behaviour, even though the study 
  of transient response from finite-bow models is still in its infancy and 
  further development of the models can be expected. Looking first at the 
  right-hand column, we see a pattern similar to the results for the cello D 
  string discussed in section 9.6. The friction-curve model shows relatively 
  few successful transients. The thermal model is somewhat better, except that 
  the points are all concentrated on the left-hand side of the plot. Further to 
  the right, the phenomenon of “rounded bottoms” in the sawtooth waves occurs, 
  as illustrated in the previous section. The modified thermal model does the 
  best, with a good spread of non-black pixels. 

  The pattern with the finite-width simulations in the left-hand column is 
  rather different. For the friction-curve model, everything gets worse 
  compared to the point-bow model. There are hardly any non-black pixels. The 
  original thermal model, though, behaves impressively well. There is an almost 
  solid wedge of coloured pixels, including some bright colours indicating 
  short transients. The modified thermal model this time seems to make things 
  worse. Although it gives a “Guettler wedge” in roughly the same place as the 
  original thermal model, the colours are less bright and there are more black 
  pixels mixed in. 

  The resulting picture is tantalising. The results reinforce the message that 
  the friction-curve model gives very poor transient performance, and is not 
  good enough for any serious study of playability issues. The thermal models 
  perform much better, and it looks as if the finite-width model may have 
  eliminated the problem of “rounded bottoms” to give very promising behaviour. 
  The differences between the two versions of the thermal model are relatively 
  slight, so even though neither model is likely to be correct in full physical 
  detail, we are perhaps converging on a model that is fairly robust in the 
  face of changing model details. This gives encouragement for future work 
  using these models to study playability. 

  Finally, we turn to the third category of physical effects associated with a 
  bow: things connected with vibration of the bow-stick and bow-hair. Now we 
  enter a world of smoke and mirrors: many players and bow-makers have a 
  deeply-held belief that the bow has a direct influence on the tone of a 
  violin (not just on the ease of performing tricky bowings). But in fact there 
  is at present very little hard evidence for how such an influence on tone 
  might arise. 

  Of course the stick and hair of a bow can vibrate in various ways: the stick 
  can show bending and twisting behaviour, while the hair is rather similar to 
  a string, and can show transverse and longitudinal vibration. When these 
  things are all coupled together, quite a range of vibration resonances can 
  arise. But a bow, like a string, is very thin compared to the wavelength of 
  sound, so it can make very little radiated sound in its own right. Any 
  influence must, surely, come through interaction with the vibrating string. 

  One obvious route by which bow vibration might influence the bowed string 
  relies on the cranked shape of a bow-tip (recall Fig.\ 2). Bending modes of 
  the stick will naturally produce some rotation of this tip, and that will 
  couple to longitudinal motion in the bow hair — bow hair is by no means rigid 
  along its length. The time-varying friction force causing the string 
  vibration is, of course, in the longitudinal direction along the bowhair, so 
  (particularly when bowing near the tip) that friction force will be able to 
  interact with stick vibration. But is this important? The effect would be 
  confined to a frequency band close to each resonance. When players and 
  bow-makers talk about the influence of a bow on the sound, they do not seem 
  to be talking about note-specific effects (rather like the wolf note, 
  discussed in section 9.4). 

  The physics of bow vibration has been explored in some detail by Colin Gough 
  [6]. Many of the effects have been incorporated into a simulation model for a 
  bowed string by Hossein Mansour [7]. But both authors only make very cautious 
  statements about the possibility of a clear-cut influence of the bow on the 
  sound of a violin. The physical effects they find seem very subtle. 

  But I don’t want to end this chapter on that negative note. We need to remind 
  ourselves, yet again, that effects which are physically subtle can sometimes 
  turn out to be perceptually significant to a musician. I will show some 
  results about an effect that is not yet fully understood, but which seems to 
  be connected with an aspect of bow vibration. I will put forward a 
  speculative semi-explanation, arrived at in discussion with colleagues and 
  friends Murray and Patsy Campbell. 

  \fig{figs/fig-bd02fd43.png}{Figure 15. A renaissance bass viol made by 
  Richard Jones, being played by Patsy Campbell during the ``warble'' 
  investigation. Image copyright Murray Campbell, reproduced by permission.} 

  They are both multi-talented musicians, among other things being keen viol 
  players (see Fig.\ 15). They pointed out a phenomenon that is sometimes heard 
  in viols, and regarded as a nuisance by players. Listen to the three notes in 
  Sound 2: they were all played on the same string of the viol, with nominally 
  steady bow speed and force. In all three, perhaps most prominently in the 
  middle note, you can hear a kind of “wah-wah” effect which we will call 
  “warble”. 

  The effect is somewhat reminiscent of the cello wolf note, discussed in 
  section 9.4. But the spectrogram shown in Fig.\ 16 shows that the effect is 
  quite different from a wolf. A classic wolf note normally only affects a 
  single note (near a strong body resonance), and it involves strong modulation 
  in the fundamental frequency component as the Helmholtz motion alternates 
  with double-slipping motion. Neither of those things is true for the viol 
  warble. Figure 16 shows that for all three notes the fundamental, and the 
  second and third harmonics, show little variation in time. The 4th harmonic, 
  though, shows conspicuous and fairly regular modulation in all three notes, 
  most prominently in the middle note. Some of the higher harmonics also show 
  clear periodic modulation, not all with the same rate of repetition. 

  \fig{figs/fig-68ab64ed.png}{Figure 16. Spectrogram of Sound 2, showing the 
  ``warble'' as fairly regular modulation of some of the harmonics, especially 
  numbers 4 and above} 

  What is going on? Our tentative explanation involves transverse vibration of 
  the bow-hair ribbon. A first step is to see that such transverse vibration 
  can be excited by bowing, and can have audible consequences. We get rather 
  clear evidence of that from a different test result. The viol bow seen in 
  Fig.\ 15 was fitted with a small accelerometer, taped to the bow-hair near 
  the frog so as to respond to transverse bow-hair vibration in the plane of 
  the bow. The signal from this accelerometer was recorded simultaneously with 
  two others: the output of a bridge-force sensor of the kind we have seen 
  before, and the sound recorded by a microphone near the instrument body. 

  A recording was made of a bow change near the frog while bowing an open 
  string on the viol: in other words, an up-bow followed by a down-bow. Figure 
  17 shows a spectrogram of the response from the accelerometer: you can hear 
  the sound in Sound 3. The spectrogram shows vertical stripes marking the 
  harmonics of the played note, but it also shows a prominent pattern of 
  curving lines dotted out by bright spots on these vertical stripes. These 
  lines mark the fundamental and the first few harmonics of a tone that rises 
  in frequency as the bowed point approaches the frog, then falls again after 
  the bow change. You can hear this rising then falling tone very clearly in 
  Sound 3. 

  \fig{figs/fig-1bde7566.png}{Figure 17. A spectrogram of bowhair vibration 
  measured by a small accelerometer near the frog, during a bow change 
  upbow--downbow. The curving lines of bright spots show the frequency 
  variation of harmonics of transverse vibration of the hair between the string 
  contact and the frog.} 

  The rising-falling frequency you are hearing corresponds to the lowest 
  resonance of the segment of bow-hair between the string and the frog: that 
  segment gets shorter and then longer during the bow change, producing the 
  rising and falling frequency. The bright spots in the spectrogram show that 
  whenever one of the resonance frequencies of the short segment of bow-hair 
  matches a harmonic of the played note, that bow-hair resonance is strongly 
  excited. If we had placed another accelerometer near the tip, it would have 
  picked up the corresponding vibration of the other segment of bow-hair: this 
  would have produced a sound with the opposite trend, falling and then rising. 

  The rising-falling tone can be heard fairly clearly in the microphone 
  recording, and seen in a corresponding spectrogram, but there is absolutely 
  no trace of it in the bridge force recording. Putting these facts together, 
  we learn three interesting things, none of them obvious. First, transverse 
  bow-hair vibration is being strongly excited by bowing the string, despite 
  the fact that the time-varying friction force is longitudinal in the 
  bow-hair, approximately perpendicular to that transverse direction. Second, 
  this bow-hair vibration is somehow creating radiated sound that can be picked 
  up by a microphone and heard: but it is not doing this by influencing the 
  usual transverse force at the bridge. Third, the fact that the bridge force 
  seems unaffected suggests that the bow-hair vibration is having little 
  influence on the stick-slip process of string vibration, the thing we have 
  mainly been talking about throughout this chapter. 

  We can suggest a mechanism to explain the last two of these observations. 
  Transverse vibration of the bow-hair will exert a force on the string at the 
  bowed point, approximately perpendicular to the friction force. This can 
  excite string vibration in that perpendicular direction. But remember that 
  the bright spots in Fig.\ 17 always occur at harmonics of the played note, 
  which will also be (approximately, at least) resonance frequencies of the 
  string in the perpendicular direction, so this perpendicular string vibration 
  might be quite strong. 

  Now think what happens at the bridge. The new perpendicular string vibration 
  will not be detected by the bridge force sensor, because that is oriented to 
  capture transverse vibration. But the perpendicular string vibration will 
  exert a force on the bridge in that perpendicular direction, and so it can 
  excite some body vibration (and thus create radiated sound). Possibly, the 
  body response to such perpendicular forcing might be particularly strong in 
  an instrument without a soundpost, as is the case for some of these 
  early-pattern viols. 

  Returning to the viol warble, that too is not evident in a bridge force 
  recording. This suggests that it might arise through a somewhat similar 
  route, involving transverse vibration of the bow-hair and forcing at the 
  bridge in the perpendicular direction. Support for this idea comes from a 
  detail you can see in Fig.\ 15: notice that the bow stick has several white 
  markers attached to it. These were empirically placed, to mark positions 
  where the warble effect seemed most prominent. They form a suspiciously 
  regular pattern, which corresponds to the nodal points of the 8th resonance 
  of transverse hair vibration (count the segments between the markers). 

  Now, a nodal point of a mode of the entire length of bow-hair is also a point 
  where both portions (to the left and right of the string contact) will 
  resonate at the same frequency. So perhaps particularly strong perpendicular 
  string vibration is excited when the bow is near these positions? Does this 
  lie at the heart of the warble phenomenon? We don’t know for sure, yet. This 
  proposed mechanism involves string vibration in two directions and bridge 
  force in two directions, and gathering definitive data to see all those 
  things simultaneously needs another round of instrumentation and 
  data-gathering. 

  There is one more complication to mention: informal exploration of the warble 
  effect suggests that it happens most strongly if the bow is not perpendicular 
  to the string. Now, string teachers have always told pupils to avoid such 
  “skewed” bowing. At least in the context of viol playing, perhaps avoiding 
  warble is the main physical reason behind that advice? 



  \sectionreferences{}[1] Anders Askenfelt and Knut Guettler, “The bouncing 
  bow: an experimental study”, Catgut Acoustical Society Journal \textbf{3}, 6, 
  3—8 (1998) 

  [2] Knut Guettler and Anders Askenfelt, “On the kinematics of spiccato and 
  ricochet bowing”, Catgut Acoustical Society Journal \textbf{3,} 6, 9—15 
  (1998) 

  [3] J. A. Greenwood and J. B. P. Williamson, “Contact of nominally flat 
  surfaces”, Proceedings of the Royal Society of London, Series A: Mathematical 
  and Physical Sciences, \textbf{295}, 300-319 (1966) 

  [4] R. Pitteroff and J. Woodhouse, “Mechanics of the contact area between a 
  violin bow and a string. Part II: simulating the bowed string”; Acta Acustica 
  united with Acustica, \textbf{84}, 744—757 (1998). 

  [5] R. Pitteroff and J. Woodhouse, “Mechanics of the contact area between a 
  violin bow and a string. Part III: parameter dependence”; Acta Acustica 
  united with Acustica, \textbf{84}, 929—946 (1998). 

  [6] Colin E. Gough, ``Violin bow vibrations'', Journal of the Acoustical 
  Society of America \textbf{131}, 4152--4163 (2012). 

  [7] Hossein Mansour, Jim Woodhouse and Gary P. Scavone, “Enhanced wave-based 
  modelling of musical strings, Part 2 Bowed strings”; Acta Acustica united 
  with Acustica, \textbf{102}, 1094–1107 (2016). 