

  In the following sections we will discuss models for the excitation of 
  various kinds of wind instrument, and for that we will need to know a little 
  about fluid flow. This section gives a qualitative overview of some important 
  fluid phenomena and terminology, introduced through pictures. A side link 
  will give some technical background about how the subject can be approached 
  mathematically. 

  \samsection{A. Laminar and turbulent flow} 

  So far our only contact with the world of fluid dynamics has been through the 
  very special case of linear acoustics. However, we will need to go beyond 
  this to understand such things as how a clarinet or flute mouthpiece works. 
  If you think about it for a moment, you already know that everyday fluid 
  dynamics must involve nonlinear effects. Look at the small waterfall in Fig.\ 
  1. You see smooth water coming over the lip of the fall, but it turns into 
  complicated turbulent flow at the bottom. The smooth initial flow is known as 
  “laminar flow”. 

  Figure 2 shows another example. This is a schlieren image of the plume of hot 
  air rising from a candle flame. At the bottom the flow is laminar, but there 
  is a rather abrupt transition to turbulent flow in the middle of the image, 
  without anything obvious happening there to trigger the transition (nothing 
  analogous to the waterfall edge, for example). This is a spontaneous 
  transition to chaotic behaviour, in the sense discussed in section 8.4. 

  Chaotic behaviour like this can only occur in nonlinear systems, although if 
  you recall the example of the double pendulum from section 8.4 you will 
  appreciate that it may only take a rather simple-looking nonlinear system to 
  exhibit such behaviour. The origin of the main nonlinearity in the governing 
  equation for fluid flow (known as the “Navier-Stokes equation”) is nothing 
  more than a quadratic term, as shown in the next link, but this is enough to 
  allow the full complexities of turbulence. 

  The possibility of turbulence is surely enough to tell us that we should not 
  expect easy mathematical solutions to the governing equations for fluid flow. 
  It then makes sense to explore approximations of different kinds: these can 
  simplify the mathematics under certain circumstances and allow us to make 
  some progress with understanding the physics behind fluid behaviour. 

  \samsection{B. Incompressible flow} 

  We have already met one approximation: when we derived the linear wave 
  equation back in section 4.1.1, we assumed that all relevant quantities like 
  pressure and flow velocity were small in some appropriate sense, and that 
  allowed us to ignore a lot of complicating factors. However, in order to 
  treat sound waves we certainly had to allow our fluid (usually air) to be 
  compressible. But in the waterfall seen in Fig.\ 1, the water behaves as if 
  it were incompressible. This turns out to make a major simplification in the 
  governing equations, and for many problems it is a useful approximation to 
  make. The previous link gives some details. 

  You might think that the incompressible approximation would have nothing to 
  do with the acoustics of wind instruments, since we are surely always talking 
  about sound waves? But this would be misleading. Compressibility is indeed 
  always important for the acoustic resonances of instrument tubes, and for the 
  internal pressure waveform when an instrument is played. But when we are 
  thinking about how a mouthpiece works, we are mainly concerned with a 
  different aspect of fluid flow, associated with the air blown into the 
  instrument by the player. We can usually get a rather good approximation to 
  the behaviour by ignoring the compressibility of the air moving through the 
  mouthpiece. The underlying reason for this is that the air-flow is very slow 
  compared to the speed of sound (around 340~m/s), or in other words the Mach 
  number is very small. This is the mathematical condition for compressibility 
  effects to be unimportant. 

  We will make use of an important result which takes its simplest form in the 
  case of incompressible flow. This is called Bernoulli’s principle. The 
  mathematical details are given in the previous link, but in words Bernoulli’s 
  principle says that if you follow a laminar air-flow along its streamlines, a 
  region where the flow speeds up is automatically associated with the pressure 
  going down, and vice versa. So if a jet of air is squeezed through a narrow 
  gap, for example between the reed and the lay of a clarinet mouthpiece, the 
  pressure will be lower there. This low pressure tends to make the reed close 
  towards the lay. We will see in section 11.3 that this is an important 
  ingredient of how a reed mouthpiece works. 

  \samsection{C. Viscosity} 

  There is another physical property of fluids we need to think about, called 
  viscosity. Figure 3 shows a familiar sight, honey running slowly off a spoon. 
  If you imagine doing the same thing with a spoonful of water, the behaviour 
  would be quite different: almost all the water would fall off the spoon as 
  soon as you tip it up. This contrasting behaviour happens because the 
  viscosity of honey is far greater than the viscosity of water. 

  But you may have noticed that I said “almost all water would fall off the 
  spoon”. Actually, a few drops of water continue to fall, after most of it has 
  gone. All normal fluids, including water, have some viscosity. When you tip 
  the spoon up, a thin layer of the water is reluctant to run off. It clings to 
  the surface, and only runs down gradually — very much like the honey, except 
  that with water it is only a very thin layer that shows the behaviour. The 
  same thing happens when you wash up crockery: if you leave it to drain for a 
  while, that gives the last thin layer of water time to drip off. 

  So what exactly is viscosity? It is the property of a fluid that resists 
  shear deformation. Figure 4 shows a sketch of a layer of fluid between two 
  metal plates. (In essence, this is a sketch of an apparatus you could use to 
  measure viscosity.) The upper plate is forced to move to the right. This 
  plate does not slide over the fluid, it carries it along with it: fluid in 
  contact with a solid surface is “stuck” to it by molecular forces. So the 
  fluid at the top of the layer moves rightwards with the plate, but the fluid 
  at the bottom is anchored to the bottom plate. Provided the motion is slow 
  enough, the fluid in between will show a linear velocity profile as sketched: 
  this can be described as uniform shearing flow. Now the viscosity relates to 
  the force necessary to move the plate, at a given speed and with a layer of 
  given thickness and area. 

  So now come back to our spoonful of water. What happens when you tip the 
  spoon? The force of gravity acts in the same way on every particle of the 
  water, so it all “wants” to move downhill at the same speed. But the water in 
  contact with the spoon surface is stuck to it, and cannot flow away. The 
  result is sketched in Fig.\ 5, which shows a magnified view of a small region 
  near the spoon surface. Most of the water flows down at the same velocity, 
  but a thin layer near the surface experiences shearing motion, which is 
  resisted by the viscosity of the water. 

  The result is a velocity profile of the kind sketched, featuring a boundary 
  layer: a layer very close to the solid surface in which significant shear 
  flow occurs. The thickness of this layer is determined by a balance between 
  the force of gravity and the viscous force resisting the shear. The lower the 
  viscosity, the thinner the layer. It is the lower part of this boundary layer 
  that is left behind after the washing up, to drain away slowly in your 
  dish-drainer. 

  A boundary layer related to what has just been described will form on the 
  wall of a wind instrument when it is played. The acoustic field inside the 
  tube involves cyclical longitudinal motion of the air. Very close to the tube 
  wall, though, the “no-slip boundary condition” applies: the air is anchored 
  to the wall. A viscous boundary layer is the inevitable result. Since shear 
  motion resisted by viscosity always dissipates energy, this is one of the two 
  main mechanisms for damping of the acoustic resonances of the pipe, as 
  mentioned in section 11.1. The boundary layer will be at its thinnest if the 
  internal wall of the tube is very smooth, but if that surface is rough then 
  the boundary layer is likely to be thicker, and the damping increased. 

  \samsection{D. Reynolds Number (and its relatives)} 

  Two effects we have described, nonlinearity and viscosity, are each described 
  by one term in the governing Navier-Stokes equation (see the previous link). 
  Fluid dynamicists use a quantity called the Reynolds Number to quantify the 
  relative importance of these two terms (we described it briefly back in 
  section 10.1.1). This number captures the ratio of strengths of the nonlinear 
  term to the viscosity term. The formula for Reynolds Number turns out to 
  involve the typical flow speed, multiplied by the typical length-scale, and 
  divided by the viscosity. 

  In a flow with very low Reynolds number, nonlinear effects can be neglected. 
  The flow will be dominated by viscosity effects, and it will be laminar 
  because viscosity has a stabilising effect on the instabilities leading to 
  turbulence. Examples would be our honey running off the spoon (because the 
  viscosity is large), or the swimming of a micro-organism in water (because 
  the length-scale is very small). A flow with very high Reynolds Number will 
  have the opposite behaviour: viscosity can be neglected, nonlinear effects 
  will be strong, and the flow is likely to be turbulent. Examples would be the 
  air flow through a jet engine (because the flow speed is very high), or the 
  swimming of a human being in water (because this time the length-scale is 
  much bigger than for the micro-organism, and the viscosity of water is quite 
  small). 

  The effective Reynolds Number will vary with depth through a structure like a 
  boundary layer. There will always be a viscous-dominated layer very close to 
  the wall, but it is perfectly possible (and indeed quite common) for the 
  outer part of a boundary layer to become turbulent. See \tt{}this video\rm{} 
  for a striking demonstration of a turbulent boundary layer. 

  Fluid dynamicists really like to use non-dimensional numbers like the 
  Reynolds Number. Another example, which is easy to visualise, is the Mach 
  Number. This is the ratio of the flow speed to the speed of sound: so slow 
  flows have low Mach Number, and a supersonic flow has a Mach Number bigger 
  than 1. 

  \samsection{E. Flow separation and vortices} 

  Another phenomenon that can occur is that the shape of the solid object may 
  create the conditions for the boundary layer to separate from the wall at 
  some position, and then give rise to a turbulent wake. Figure 6 shows an 
  example, in a wind-tunnel image of flow past a wing cross-section. The flow 
  below the wing remains in contact with the surface, but above the wing it 
  separates and produces a complicated wake structure. 

  Under some circumstances this separation process varies cyclically with time, 
  giving rise to a strikingly beautiful ``Kármán vortex street'', like the 
  example in Fig.\ 7. The image is taken from \tt{}this Wikipedia page\rm{}, 
  where you can find more detail. In the process of shedding vortices on 
  alternate sides, an alternating force is exerted on the solid body, in the 
  direction perpendicular to the flow. If the object is something like a 
  stretched string or a power cable, this force can set it into vibration. This 
  is the underlying mechanism of the Aeolian harp, for example. 

  Another phenomenon involving vortices is more directly relevant to musical 
  wind instruments. If a jet of air is blown out of a slot, as in a recorder 
  mouthpiece for example, each side of this air jet will be a shear layer: a 
  sudden jump in air flow speed. Well, there is a famous mathematical result in 
  fluid dynamics proving that an ideal shear layer is unstable. If the 
  interface between the two flow speeds is initially straight, but is then 
  perturbed a little, the perturbation will grow. 

  With a narrow jet, the instability usually carries the whole jet up and down 
  --- we will say more about this in section 11.8 when we discuss air-jet 
  instruments. For now, we can see one image for a rather wide air jet, in 
  Fig.\ 8. In this schlieren image, we can see that the two shear layers on 
  either side of the jet have generated elegant vortex shapes, in an 
  alternating pattern. The image is taken from the doctoral thesis of Sylvie 
  Dequand [1]. 



  \sectionreferences{}[1] Sylvie Dequand, “Duct aeroacoustics: from 
  technological applications to the flute”, Doctoral dissertation, Eindhoven 
  University of Technology (2001), \tt{}https://pure.tue.nl/ws/portalfiles/portal/3429613/445111.pdf\rm{} 