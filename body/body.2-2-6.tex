  There is a simple way to get an idea of the force waveform of a hammer tap, 
  and its corresponding frequency response. This is relevant to a drum, but it 
  is also important for more formal experimental vibration testing because an 
  instrumented hammer is a very common way to excite a structure for testing. 
  The hammer has a certain mass, $m$ say. While it is in contact with the drum 
  (or other structure), there is a contact force tending to make it bounce off. 
  This restoring force can be modelled as a spring with stiffness $k$. That 
  stiffness is made up of two components: the intrinsic stiffness of the 
  hammer, and the stiffness provided by the drum membrane or whatever else is 
  being struck. A hard drumstick will be stiffer than a soft beater. But also, 
  a drum membrane will present a lower stiffness than a structure like a piano 
  soundboard or a ship's hull. 

  During contact, this ``tip spring'' will be compressed, by an amount $x$ say. 
  The equation of motion of the hammer mass is then exactly the same as the 
  simple harmonic oscillator examined in section 2.2.2: 

  \begin{equation*}m \dfrac{d^2 x}{dt^2}+kx=0. \tag{1}\end{equation*} 

  The solution is thus 

  \begin{equation*}x=A \cos(\Omega t + \phi) \tag{2}\end{equation*} 

  \noindent{}where $\Omega=\sqrt{k/m}$, and $A$ and $\phi$ are constants 
  determined by the initial conditions: at the moment of first making contact, 
  $x=0$ and the velocity $\dot{x}$ is equal to the speed with which the hammer 
  is moving. 

  The force applied to the structure is $f=kx$: but this description only 
  applies while that force is positive. As soon as the solution suggests that 
  the force should go negative, the hammer will bounce off the structure 
  because it can't apply a tensile force (unless the hammer tip is sticky). So 
  the waveform of applied force is as sketched in Fig.\ 1: a half-cycle of a 
  cosine wave, at the (radian) frequency $\Omega$. 

  \fig{figs/fig-47b667ec.png}{\caption{Figure 1. The bounce of an idealised 
  hammer}} 

  The frequency spectrum of this hammer force is found by calculating the 
  Fourier transform $F(\omega)$ of the force $f(t)$: see section 2.2.1. To do 
  this calculation, it is convenient to choose $t=0$ to lie at the centre of 
  the force pulse, which then extends symmetrically from $t=-\pi / (2 \Omega)$ 
  to $t=+\pi / (2 \Omega)$. Then 

  \begin{equation*}F(\omega) = \dfrac{kA}{2 \pi} \int_{-\pi/2 \Omega}^{\pi/2 
  \Omega} \cos \Omega t~e^{-i \omega t} dt \tag{3}\end{equation*} 

  \begin{equation*} = \dfrac{kA}{4 \pi} \int_{-\pi/2 \Omega}^{\pi/2 \Omega} 
  \left[ e^{i \Omega t} + e^{-i \Omega t} \right]~e^{-i \omega t} dt 
  \end{equation*} 

  \begin{equation*}= \dfrac{kA}{4 \pi} \left[ \dfrac{e^{i (\Omega -- \omega) 
  t}}{i (\Omega -- \omega)} -- \dfrac{e^{-i (\Omega + \omega) t}}{i (\Omega + 
  \omega)} \right]_{-\pi/2 \Omega}^{\pi/2 \Omega} \end{equation*} 

  \begin{equation*}= \dfrac{kA}{2 \pi} \left[ \dfrac{\sin \frac{\pi}{2} \left(1 
  -- \frac{\omega}{\Omega} \right)}{\Omega -- \omega} + \dfrac{\sin 
  \frac{\pi}{2} \left(1 + \frac{\omega}{\Omega} \right)}{\Omega + \omega} 
  \right].\end{equation*} 

  But $\sin (\pi/2 + \alpha) = \cos \alpha$ and $\sin(\pi/2 -- \alpha)= \cos 
  \alpha$, so 

  \begin{equation*}F(\omega) = \dfrac{kA}{2 \pi} \cos \dfrac{\pi \omega}{2 
  \Omega} \left[ \dfrac{1}{\Omega -- \omega} + \dfrac{1}{\Omega + \omega} 
  \right] \end{equation*} 

  \noindent{}and finally, 

  \begin{equation*}F(\omega) = \dfrac{kA \Omega}{\pi} \dfrac{\cos \frac{\pi 
  \omega}{2 \Omega}}{\Omega^2 -- \omega^2}. \tag{4}\end{equation*} 

  To see what this means, two plots are given below. Figure 2 is a 
  ``universal'' plot of the prediction of eq. (4), with the ratio $\omega / 
  \Omega$ on the horizontal axis. The vertical axis shows the absolute value of 
  $F(\omega)$, normalised by its DC value $F(0)$. It is plotted on a decibel 
  scale (defined by $20\log_{10}|F|$, so that 20 dB corresponds to a factor of 
  10 in amplitude). The plot reveals that the force imparted by the hammer tap 
  is concentrated in the frequency range below about $2.5 \Omega$. 

  \fig{figs/fig-703fc628.png}{\caption{Figure 2: Frequency spectrum of an ideal 
  hammer tap}} 

  To see what this might mean for using different drumsticks to hit a drum, 
  Fig.\ 3 shows two versions of the frequency spectrum of the force, this time 
  on a frequency scale in Hz. The red curve is the result for a soft, heavy 
  beater with the frequency $\Omega$ corresponding to 300 Hz, while the black 
  curve is for a lighter and harder stick with $\Omega$ corresponding to 1500 
  Hz. This hard stick can excite the drum to far higher frequency, resulting in 
  a ``sharper'' sound. The soft beater will give a mellow, ``boomy'' sound just 
  as you would expect. 

  \fig{figs/fig-72baa61b.png}{\caption{Figure 3: Two versions of the frequency 
  spectrum from Fig.2, now plotted against a frequency scale in Hz for two 
  different values of the frequency $\Omega$: expressed in Hz, the red curve 
  corresponds to 300 Hz and the black curve to 1500 Hz.}} 