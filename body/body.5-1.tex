  \fig{figs/fig-eff9c116.png}{Figure 1: a selection of acoustic stringed 
  instruments} 

  Stringed instruments come in many forms: a few are illustrated in Fig.\ 1. 
  This chapter will largely be concerned with plucked-string instruments 
  (because we are sticking as long as we can with linear theory: remember that 
  idea from section 2.1?). But we will give a broad description of how an 
  acoustic stringed instrument works, and for that purpose all stringed 
  instruments have some things in common, whether plucked, hammered or bowed. 
  The process of bowing a string is definitely nonlinear (we will come back to 
  it in Chapter ?), but the body vibration and sound radiation characteristics 
  of a violin or cello are essentially linear. 

  The musician devotes virtually all their effort and skill towards making the 
  strings of their instrument vibrate in particular ways: at the right pitch, 
  at the right time, and with subtle control over the initial transient and the 
  tonal balance. But, as was explained in section 4.3, a vibrating string 
  cannot radiate much in the way of sound waves into the surrounding air. Most 
  strings are only around 1 mm in diameter, whereas the wavelength of sound in 
  the mid-audio range is of the order of hundreds of millimetres, and even at 
  the very highest frequency audible to humans, around 20 kHz, it is about 17 
  mm. Air simply flows around a vibrating string, making it a very weak dipole 
  sound source. 

  In order to make sound that is loud enough to be useful, some of the energy 
  in the vibrating string must be used in one way or another to cause vibration 
  of something bigger, comparable in size to the wavelengths of interest. In an 
  electric instrument, this larger vibrating object is the loudspeaker cone, 
  set into vibration electrically by an amplifier. But in an acoustic 
  instrument, it is usually woodwork of some kind. 

  As the string vibrates, the end in contact with the instrument body at the 
  bridge will have a time-varying angle of contact. The string is, of course, 
  under tension, and this varying angle means that a component of the tension 
  is exerted on the bridge, perpendicular to the string in its plane of 
  vibration. This force drives the body into vibration, and the vibrating 
  structure will radiate sound waves into the surrounding air. Figure 2 gives a 
  sketch to illustrate what is going on, for the situation where the string 
  vibrates in a plane perpendicular to the soundboard, so that the time-varying 
  angle lies in that plane. 

  \fig{figs/fig-802c7700.png}{Figure 2. Sketch of a vibrating string on an 
  instrument. The changing direction of the string tension at the bridge 
  produces an oscillating force on the bridge, which excites vibration of the 
  body.} 

  There is a complication. For definiteness, it will be described in the 
  context of the guitar. A string can indeed vibrate in the plane perpendicular 
  to the soundboard as sketched in Fig.\ 2, but it could also vibrate in the 
  plane parallel to the soundboard, or anywhere in between. If the string 
  vibrates in the plane parallel to the soundboard, the force on the bridge is 
  also parallel to the soundboard and this is not an efficient way to make the 
  soundboard vibrate. Vibration perpendicular to the soundboard is much better: 
  the force is now oriented in the best way to make the soundboard vibrate, and 
  the sound is likely to be louder. This difference lies behind the guitarist's 
  techniques known as apoyando and tirando: these different ways to pluck a 
  string with the fingers make a difference to the plane of vibration of the 
  string. An apoyando stroke (sometimes called a rest stroke) produces more 
  string vibration in the plane perpendicular to the soundboard, and gives a 
  louder and fuller sound. 

  Now most of the bowed instruments have a problem. As will be seen in Chapter 
  9, a bowed string vibrates, mainly, in the plane of the bow-hair. In order to 
  be able to access the strings without the bow hitting the body, in most bowed 
  instruments (like the violin) this means bowing approximately parallel to the 
  soundboard. But we have just seen that this is not a recipe for making a loud 
  sound. This is the reason that a violin or cello needs the familiar high 
  bridge, quite different from the squat bridge of a guitar. The string exerts 
  a force at the top of the bridge, approximately parallel to the soundboard, 
  as sketched in Fig.\ 3. This force is translated by a rocking action of the 
  bridge into forces at the bridge feet that are perpendicular to the 
  soundboard. 

  But there is still a problem. If a violin was as symmetrical internally as it 
  appears to be from the outside, this rocking action would tend to excite 
  motion of the top plate with no net volume change, as indicated schematically 
  by the red line in the left-hand plot. This would lead to poor sound 
  radiation at low frequency. But there are two important structures inside a 
  violin which destroy the symmetry. One is the soundpost, a wooden rod wedged 
  between the top and back plates, near one foot of the bridge by the highest 
  string. The second internal structure is the bassbar, a wooden beam running 
  approximately 3/4 of the length of the top plate, passing beneath the other 
  foot of the bridge. Both are indicated in the right-hand plot. At low 
  frequency the presence of the soundpost inhibits motion of the top plate, 
  leading to deformation as indicated schematically by the new red line, 
  involving net volume change and hence monopole radiation. 

  Some bowed instruments escape these problems. In the Chinese erhu, for 
  example, the bow is threaded between the two strings and they are bowed 
  perpendicular to the ‘soundboard’ (which is actually a stretched snakeskin 
  membrane in this particular instrument). There is no need for a high bridge: 
  the erhu has a low bridge more like a guitar bridge, and it is quite loud. 
  Figure 4 shows an example. 

  \fig{figs/fig-3f08d472.png}{Figure 4. A blind musician playing an erhu in 
  Hubei, China in 2006. Another erhu player is in the background. Photograph: 
  Anna Frodesiak, Public domain, via Wikimedia Commons} 

  There is an important consequence of the sequence of events just described, 
  and illustrated in Fig.\ 2. To a good first approximation, all interaction 
  between the string and the body occurs at a single contact point: the string 
  notch on the bridge. So to characterise the vibration behaviour of the body 
  in so far as it affects the string motion, what is needed is a measurement of 
  the frequency response function at this point. 

  Such a measurement was illustrated in Fig.\ 4 of section 2.2, and Fig.\ 5 of 
  that section showed some results for the toy drum. A (very) small hammer can 
  be used to tap the bridge by the string notch, and the response can be 
  measured with a laser vibrometer or an attached sensor with very low mass. 
  The waveform of force is measured by a built-in sensor in the hammer. The two 
  signals, force and response, are recorded into a computer, and the FFT is 
  used to convert them into frequency spectra. The output spectrum (FFT of the 
  response) is divided by the input spectrum (FFT of the force waveform), and 
  the result is the frequency response function we are seeking. The most common 
  version is based on an output signal proportional to the velocity at the 
  measurement point, and is called the bridge admittance or bridge point 
  mobility. 

  \fig{figs/fig-fbc9f2b5.png}{Figure 5. A banjo set up for measurement of the 
  bridge admittance. The hammer taps the bridge near the notch for the top 
  string and the red spot from the laser vibrometer is nearby. Vibration of the 
  strings, both in the playing lengths and in the afterlengths between the 
  bridge and the tailpiece, has been suppressed by weaving a piece of paper 
  through them.} 

  Figure 5 shows a typical setup, on a banjo. Figure 6 shows a slightly 
  modified procedure being used on a violin. As mentioned above, violin strings 
  vibrate predominantly in the plane of the bow-hair. To measure the most 
  relevant bridge admittance the force has to be applied in the bowing 
  direction, rather than perpendicular to the soundbox, and the response needs 
  to be similarly measured. It is easier in practice to measure the response 
  from the corner of the bridge on the other side, on the right in this 
  picture. This means that the result is not strictly a point admittance, but 
  provided the top portion of the bridge moves without deforming, it is a good 
  approximation to it [1]. 

  \fig{figs/fig-9a945901.png}{Figure 6. Bridge admittance being measured on a 
  violin. The force needs to be applied in the direction of bowing, because the 
  string vibration occurs predominantly in that direction.} 

  As with all measurements, there are some tricks of the trade for getting best 
  results. It is prudent to collect several sets of data, and use an averaging 
  procedure to combine them into a best estimate of the bridge admittance. Some 
  details of the process are given in the next link. A side effect of the 
  procedure is something called the coherence function, which gives a very 
  helpful measure of data quality: an example is shown in the link. We will 
  come back to a more general discussion of measuring frequency response 
  functions in section 10.4. 

  The hammer is held in a pendulum fixture to guarantee that it hits the same 
  spot every time. We want to characterise the body behaviour separately from 
  the strings, in order to couple them together in a controlled way later (see 
  section 5.4). So for the admittance measurement the strings are damped to 
  suppress their vibration: in this case using a piece of paper woven through 
  the strings. But we don't want to remove the strings entirely: on instruments 
  like the banjo or the violin, the tension of the strings is needed to hold 
  the bridge in place. A more subtle effect of the strings is that the body 
  vibration can be affected significantly by the stiffness of the strings along 
  their length: especially for metal strings, this stiffness can be quite high. 

  There is a final step: in order to obtain a bridge admittance that is 
  quantitatively correct, some kind of calibration procedure is needed. The 
  measurement must be repeated on some system for which the answer is already 
  known, so that a suitable scaling factor can be determined. The most common 
  approach is to weigh a mass, then hang it as a pendulum. Applying the 
  hammer/laser measurement method to this, we should obtain a result that 
  agrees with Newton's law: force=mass$\times$acceleration. We know the mass, 
  so we can deduce the true value of acceleration/force, and compare it with 
  the measured value to obtain a calibration factor. 

  It is time to see some examples. A typical measured and calibrated admittance 
  is shown in Fig.\ 7, for a violin. We will say something about what it shows 
  in a moment, but first it is helpful to convert it to a different form. In 
  the course of this chapter we will use three contrasting instruments to 
  illustrate the range of possibilities: a classical guitar, a violin and a 
  banjo. How can we make a fair comparison between instruments of different 
  types, employing different strings and tuned to different pitches? To give a 
  good answer to that question we need one more piece of background 
  information. For reasons explained in the next link, the strength of coupling 
  between a string and the instrument body is determined by the product of the 
  admittance we have measured and a property of the string called its wave 
  impedance or characteristic impedance. This impedance is defined as 

  $$Z_0=\sqrt{T m}$$ 

  where $T$ is the string tension and $m$ is its mass per unit length. 

  \fig{figs/fig-24f3f14d.png}{Figure 7. The magnitude of the bridge admittance 
  for a violin made by David J Rubio.} 

  To show the most useful comparison between different instruments, we can 
  scale the bridge admittance by the impedance of a typical string: we will use 
  the top string. We can also adapt the frequency scale to take account of the 
  different tunings: we will show a scale in semitones, starting from the 
  fundamental frequency of the lowest string. The result, for the three chosen 
  instruments, is shown in Fig.\ 8. The frequency scale covers 5 octaves, a 
  frequency ratio of $2^5=32$. The scaled admittance needs no units: it is a 
  dimensionless quantity. It directly describes the rate at which energy from 
  the vibrating string can `leak' through the bridge into the instrument body. 
  An informal description would be that the higher the level in the plot, the 
  louder the played note is likely to be. Some relevant parameters for these 
  three instruments are listed in Table 1. 

  \fig{figs/fig-834ce046.png}{Figure 8. Bridge admittance magnitude multiplied 
  by the wave impedance of the top string, for three different stringed 
  instruments. Red curve: the violin as in Fig. 5; purple curve: a classical 
  guitar made by Martin Woodhouse; blue curve: the banjo seen in Fig. 1. The 
  frequency scale is expressed in semitones, starting from the lowest tuned 
  note of each instrument. Successive octaves are indicated by ticks.} 

  The three curves in Fig.\ 8 all show a pattern of peaks and dips. At low 
  frequencies, there is a very simple interpretation of the peaks: they 
  correspond to individual vibration modes of the respective instruments. 
  However, things get more complicated at higher frequencies. Recall from 
  section 2.2.7 that each mode contributes a peak with a characteristic 
  bandwidth determined by its level of damping. It is usually characterised by 
  the half-power bandwidth, the frequency range between the two points where 
  the amplitude falls to $1/\sqrt{2}$ of the peak value. Energy is proportional 
  to the square of amplitude, so these correspond to the points where the 
  energy involved in the vibration has halved. On a decibel plot like the ones 
  shown here, these points occur where the amplitude has fallen by 
  approximately 3 dB (because $20 \log_{10} (1/\sqrt{2}) = -3.01$). In terms of 
  the modal Q-factor which we have used to characterise damping, this 
  half-power bandwidth is equal to the peak frequency divided by the Q-factor. 

  Now we can learn something interesting by combining two facts that have been 
  mentioned earlier. For any plate-based structure, like a guitar or a violin, 
  the spacing between adjacent modal frequencies is, on average, constant (see 
  section 3.2.4). On the other hand, it is an empirical observation that the 
  modal damping of such structures usually shows, approximately, a constant 
  Q-factor for all modes. So, roughly, the peaks are uniformly spaced along the 
  frequency axis, while the half-power bandwidth of each peak increases 
  approximately proportional to its centre frequency. The result is that the 
  first few resonant peaks may be well separated, because the spacing is large 
  compared to the bandwidth, but as we look higher in frequency, the spacing 
  stays the same while the bandwidth increases, so the peaks begin to overlap. 
  Eventually, there may be several modes with resonance frequencies within the 
  half-power bandwidth of each separate mode: this number is called the modal 
  overlap factor. 

  Once modal overlap becomes significant like this, the admittance function at 
  any particular frequency is influenced by several modal contributions. But 
  these will all be in different phases, so that interference will occur. We 
  can no longer rely on the fact that a peak corresponds to a resonance: peaks 
  will occur where several overlapping modes have responses that combine 
  constructively because of their phase relations. 

  Returning to Fig.\ 8, all three of the curves show this pattern. (The 
  argument given above doesn't apply directly to the banjo, which is not 
  plate-based, but it was shown in section 4.2.4 that the modal density for a 
  membrane increases with frequency, so the modal overlap increases even more 
  rapidly than in a plate-based system.) There is a great deal that can be 
  learned from the details in Fig.\ 8: so much so that we will need to take it 
  in easy stages. A detailed look at the similarities and differences of the 
  three instruments will be deferred until section 5.3. First, we will look at 
  a deceptively simple question, often asked of stringed instrument makers: ``I 
  like this instrument, but can you make it louder?'' 



  ~[1] A. Zhang and J. Woodhouse, ``Reliability of the input admittance of 
  bowed-string instruments measured by the hammer method'' ~ Journal of the 
  Acoustical Society of America \textbf{136}, 3371-3381, (2014). 