

  We now turn to something more workshop-based and close to the heart of many 
  instrument makers: the selection, cutting and characterisation of wood. We 
  will discuss which properties of wood might be most important and how they 
  might be measured. We will also try to relate these properties to the 
  cellular organisation of wood, which can only be made visible using the kind 
  of high-tech measurements discussed in the previous section. 

  We will mostly be concerned with the specific agenda of wood for the bodies 
  of stringed instruments, with a focus on properties that influence vibration 
  and sound. But we should also recognise that there are many other issues 
  surrounding wood for musical instruments, especially in connection with 
  sustainability. Many instruments have traditionally made use of tropical 
  hardwoods: rosewoods for guitar backs and sides, and for making marimba bars; 
  African blackwood for wind instruments; ebony for fingerboards of many 
  stringed instruments; pernambuco for making bows. 

  Most of these timbers are now threatened, and many are listed under the 
  \tt{}CITES convention\rm{} (which is short for the Convention on 
  International Trade in Endangered Species of Wild Fauna and Flora). As a 
  result, there is considerable interest in finding or developing alternative 
  materials. In some cases, synthetic materials can be considered: wind 
  instruments made of plastic and violin bows made of carbon fibre composite 
  can be good enough that they are finding favour with some musicians. In other 
  cases, more sustainable timbers are being explored, such as European walnut 
  for guitar bodies. In the case of fingerboards one possibility involves a 
  return to the earlier practice of using a composite construction, with a core 
  of lighter wood and just a veneer of ebony or some other hard-wearing wood 
  for the working surface. 

  Each of these applications involves its own set of desiderata and 
  constraints, and the scientific mindset can be applied to nearly all of them. 
  I will give a very brief review of some of the issues, before returning to 
  the primary agenda of properties affecting vibration and sound. There are two 
  issues which, in one way or another, tend to influence all applications: how 
  a wood looks, and how it behaves under a maker’s hands when they are trying 
  to shape it. Visual appearance is outside our remit here, but we can say some 
  things about behaviour on the woodworking bench. 

  If woodworking convenience was the only consideration, then instrument makers 
  would probably choose the same types of timber that wood-carvers prefer: 
  close-grained, not prone to splitting, not too hard so that controlled 
  cutting is easy, and capable of holding crisp details. That recipe points 
  towards timbers like boxwood, and fruit woods like apple, pear or plum. But 
  in fact these are not very common in musical instruments. In the stringed 
  instrument world they are mainly used for fixtures and fittings like pegs and 
  violin tailpieces. Those components are indeed often decorative, and boxwood 
  in particular is a common choice for up-market fittings. It is also a 
  traditional wood to make chessmen, and perhaps it acquired its name because 
  of use for making decorative snuffboxes. 

  There is an important exception to those rather dismissive comments, in the 
  world of woodwind instruments. In earlier times, boxwood and some fruit woods 
  were commonly used to make woodwind instruments like recorders and oboes, and 
  they remain a popular choice for recorder making. They can make handsome 
  instruments, with excellent musical qualities. However, they have a 
  disadvantage. These woods are rather sensitive to changes in moisture 
  content, which can arise from changes in the weather and also from the fact 
  that a player blows warm, moist air into them. 

  Absorption of moisture by the wood can lead to distortion or cracking, and it 
  can also impair the musical operation. The surface texture of the inside of 
  the bore may change, as a result of “grain raising”. This can increase the 
  damping of the internal resonances of the air in the instrument, which in 
  extreme cases can lead to the instrument “choking”. The standard control 
  measure for this problem is to oil the wood, traditionally by using a feather 
  to apply oil to the inside of the bore. 

  However, some of the most familiar woodwind instruments, like the clarinet 
  and the oboe, are black in colour, and are obviously made of a different 
  wood. This is usually African blackwood (Dalbergia melanoxylon), also known 
  as grenadilla. It is one of the very hardest and densest of all timbers, with 
  little or no discernible grain. This makes it relatively immune to 
  moisture-related problems, but at a price: the wood is very difficult to 
  work. It is so hard that it can blunt tools very quickly. The tubes of 
  clarinets and oboes have to be made using metal-working tools like tungsten 
  carbide lathe cutters. 

  Fingerboards of stringed instruments are another application where hardness 
  of the wood is at a premium. The problem this time is not to do with 
  moisture, but the wear and tear resulting from repeated vigorous contact with 
  the strings and the player’s fingers during normal playing. Hard woods like 
  ebony are often used for fingerboards. These are not quite as hard as African 
  blackwood, but they still present challenges in terms to keeping your tools 
  sharp. 

  The favoured wood to make violin bows is called pernambuco (previously 
  Caesalpinia echinata, renamed to Paubrasilia echinata in 2016). This choice 
  goes back to Tourte, the originator of the modern bow (see section 9.7). The 
  wood was originally imported to Europe in the 16th century, to make an 
  important dye. But that is not what attracted Tourte: a bow requires a 
  challenging combination of properties, and presumably he found that 
  pernambuco combined them to perfection. The long, slender bow-stick must be 
  strong enough, and be capable of being bent to the required cambered shape 
  using heat. The cranked tip (see Fig.\ 2 of section 9.7) is carved from the 
  same solid piece of wood, and it is crucial that this tip doesn't break off 
  easily, despite the fact that the grain of the wood runs across it at a very 
  narrow point. Pernambuco has been over-harvested, and trade in the wood may 
  be banned altogether before long. This is causing a serious problem for 
  bow-makers, although of course it was not their relatively modest usage that 
  led to the over-harvesting. 

  Percussion instruments like the marimba bring in a different aspect of wood 
  properties and choice. Unlike violin bows or guitar fingerboards, the bars of 
  a marimba are used directly to make sound. This means that the vibration 
  properties of the wood are important, including the damping. Damping 
  determines how well the sound rings on after a bar is struck, and this aspect 
  of the sound quality is a major factor in the choice of particular woods. 
  Again, the traditional choice has settled on some species of tropical 
  hardwood. 

  All the examples just reviewed have chosen woods that are denser and harder 
  than the wood-carver’s choice of boxwood or fruit woods. But when we turn to 
  the soundboards of stringed instruments, we find the opposite choice being 
  made: wood species that are somewhat difficult to work because they have very 
  low density, rather than very high density. We already know the main reason 
  for this choice. Back in section 5.2, with detail in section 5.2.1, we looked 
  at a simple criterion for choosing soundboard material based on loudness. If 
  you want to make a loud instrument, you need a material with low density but 
  high stiffness. The material selection charts we showed there pointed towards 
  softwoods. The most widespread choice for stringed instruments is Norway 
  spruce, but other species such as Sitka spruce are also used, and in the 
  guitar world it is common to use Western red cedar for soundboards. 

  Softwoods like Norway spruce give the instrument maker some problems. These 
  timbers are very prone to cracking along the grain, compared to hardwoods 
  which have a more “interlocking” cellular structure (compare Figs.\ 1 and 9 
  of section 10.2.1, for example). Also, the marked difference in hardness 
  between the “spring wood” and the “summer wood” in the annual growth rings 
  makes it more difficult to carve smooth curves. Your knife tends to move in 
  jumps through the softer wood, stopping on the harder layers. Violin makers 
  have to put up with this when carving details such as f-holes, but it would 
  make this wood a difficult choice for carving chessmen. 

  The vibration of a soundboard, like any other vibration, is governed by mass, 
  stiffness and damping. The material property governing the mass is the 
  density of the wood, which is very easy to measure by weighing a block of 
  wood of known volume. There is also a neat trick that violin makers sometimes 
  use (if the wood dealer allows them to) when selecting spruce for top plates. 
  You can deduce the density quickly by seeing how high the billet of wood 
  floats when immersed in a bucket of water, as sketched in Fig.\ 1. The volume 
  of water displaced must weigh the same as the block of wood, by Archimedes’ 
  principle. Armed with a tape measure you now find the proportion of the 
  length of the billet which is underwater. This tells you the density of the 
  wood billet relative to the density of water --- and we know that the density 
  of water at normal temperatures is 1000~kg/m$^3$. 

  \fig{figs/fig-6259a487.png}{Figure 1. Sketch of a spruce billet for a violin 
  top plate, being floated in water to find its density.} 

  “Stiffness” is a far more complicated thing, for a material like wood that 
  has different properties in different directions. Figure 2 reminds us of the 
  three principal directions in a tree, conventionally labelled L (for 
  longitudinal), R (for radial) and T (for tangential). For a straight-growing 
  tree without knots or other blemishes, the wood structure is symmetrical in 
  the LR plane, and also in the RT plane. If we disregard the curvature of the 
  annual rings, it is also approximately symmetrical in the LT plane. So a 
  small block of wood extracted from the tree has approximate symmetry in three 
  mutually perpendicular planes. In the language of materials science, such a 
  material is called “orthotropic”. 

  \fig{figs/fig-63aff26c.png}{Figure 2. The three principal directions L, R and 
  T in a tree. The left-hand black outline indicates the way that a radial 
  wedge would be cut for a violin soundboard. The right-hand one shows how a 
  constant-thickness board for a guitar soundboard might be sawn: such a board 
  might be close to radial, but more often it might deviate from that ideal 
  condition like the example shown here.} 

  There is now a piece of standard textbook theory which tells us how many 
  parameters are needed to give a complete description of the stiffness of such 
  a material (see for example \tt{}this Wikipedia page\rm{}). The answer is 9: more 
  complicated than we might have wished. We can visualise these parameters if 
  we think about cutting thin rods from our piece of wood, aligned in the three 
  directions L, R and T. We can imagine taking one of those rods, gripping the 
  ends in a testing machine, and applying various forces to it. 

  If we apply tension and measure how much the rod stretches, the relevant 
  stiffness is characterised by the Young’s modulus. (We met Young’s modulus 
  back in section 3.2.1 when we looked at bending beams.) We can measure 
  something else in this same test. As the rod is stretched lengthwise, it will 
  usually get a little thinner at the same time. In our orthotropic material, 
  the amount of thinning might be different in the two crosswise directions. 
  The ratio of the lengthwise stretch to a crosswise contraction is called 
  Poisson’s ratio. Finally, if instead of applying tension we twist the rod in 
  our test machine, we can measure the amount of torsion created by a given 
  torque. That is governed by a different stiffness called a shear modulus. We 
  can repeat these tests with each of our rods. In total, we will find three 
  Young's moduli, three shear moduli, and six Poisson's ratios. But there is a 
  mathematical relation between some of these, so that we only need three of 
  the Poisson's ratios to complete our set of parameters. 

  Luckily, we don’t need to think of measuring all 9 parameters for every piece 
  of wood we use. All soundboards of musical instruments are deliberately cut 
  quite thin: either approximately flat as in a piano or guitar, or arched as 
  in a violin or cello. Furthermore, the instrument maker will try quite hard 
  to have the grain direction (the L axis) lying in the plane of the plate. 
  Under those circumstances, we only need a reduced number of stiffness 
  parameters. A flat plate cut in a suitable way for a guitar or piano 
  soundboard has the 2D version of orthotropic symmetry: the structure is 
  symmetrical in two mutually perpendicular axes lying in the plate. A maker 
  would probably describe these axes as lying “along the grain” and “across the 
  grain”. 

  The corresponding piece of textbook theory now tells us that we need 4 
  stiffness values — and it turns out that one of those is not very important 
  (see McIntyre and Woodhouse [1]), so for a first approximation we only need 
  three. These can be visualised as corresponding to things that a guitar maker 
  would feel in their hands, by bending and flexing the soundboard wood. We can 
  bend the plate along the grain or across the grain. We can also twist it: we 
  will see a picture shortly of what that looks like. 

  Vibration resonance frequencies can give a very good way to measure 
  stiffnesses. The approach relies on have a test specimen of a suitable shape 
  that there is a simple theoretical expression for the frequencies in terms of 
  the geometry and material properties. The geometry is easy to measure 
  directly, and the density can then be determined by weighing. The only 
  unknown quantity in the formula for the frequencies will then be a stiffness. 
  You can turn the formula around: frequency is easy to measure quite 
  accurately, using an FFT app in your computer or your phone, and so you can 
  deduce the stiffness. 

  The simplest application of this approach is to beam-shaped samples, like the 
  ones shown in Fig.\ 3. These particular examples are made from the offcuts of 
  spruce after cutting violin top plates out of their billets. The lowest 
  resonance of a beam like this is something we have already looked at, back in 
  section 3.2. The mode shape is repeated here in Fig.\ 4. The resonance 
  frequencies of a bending beam like this were discussed in section 3.2.1: the 
  detailed formula is given in the next link. It can easily be used to 
  determine the Young’s modulus from a measured frequency. If you want to do 
  this measurement yourself, the link gives some suggestions for how to do the 
  test in order to get the most accurate and reliable answers. 

  \fig{figs/fig-065f8724.png}{Figure 3. Rectangular beams cut from the offcut 
  spruce of six different violin top plates. They are aligned in the 
  cross-grain direction.} 

\moobeginvid\begin{tabular}{ccc} \vidframe{ 0.30 }{ vids/vid-48a56e61-00.png }&\vidframe{ 0.30 }{ vids/vid-48a56e61-01.png }&\vidframe{ 0.30 }{ vids/vid-48a56e61-02.png } \end{tabular}\caption{Figure 4. The lowest mode of a bending beam with free ends.}\mooendvideo

  By cutting beam samples like this from your offcut wood, in the directions 
  parallel and perpendicular to the grain, you can deduce two of the three 
  stiffnesses that were mentioned above. However, it is not so easy to get the 
  third constant this way: it relies on twisting motion rather than bending 
  motion. If you are a guitar maker, though, you might have an easy way to 
  measure all three stiffnesses. The wood for guitar soundboards is often 
  supplied in the form of rectangular panels with essentially constant 
  thickness. This makes them ideal for a measurement of all three important 
  stiffnesses. 

  The reason is that a rectangular panel like this usually has three vibration 
  modes looking more or less like the animations in Figs.\ 5, 6 and 7. (To find 
  out why it is only “usually”, see the next link.) These three modes involve 
  precisely the three types of motion described above: bending along the grain, 
  bending across the grain, and twisting. The frequencies of these three modes 
  can thus be used to deduce stiffnesses, as described in the link, which also 
  explains the measurement procedure in detail. 

\moobeginvid\begin{tabular}{ccc} \vidframe{ 0.30 }{ vids/vid-2b5298c1-00.png }&\vidframe{ 0.30 }{ vids/vid-2b5298c1-01.png }&\vidframe{ 0.30 }{ vids/vid-2b5298c1-02.png } \end{tabular}\caption{Figure 5. A vibration mode of a rectangular plate with free edges. The shape is dominate by bending in one particular direction.}\mooendvideo

\moobeginvid\begin{tabular}{ccc} \vidframe{ 0.30 }{ vids/vid-1e38aa11-00.png }&\vidframe{ 0.30 }{ vids/vid-1e38aa11-01.png }&\vidframe{ 0.30 }{ vids/vid-1e38aa11-02.png } \end{tabular}\caption{Figure 6. Another vibration of a rectangular plate, involving bending in the perpendicular direction to the mode in Fig. 5.}\mooendvideo

\moobeginvid\begin{tabular}{ccc} \vidframe{ 0.30 }{ vids/vid-4b909508-00.png }&\vidframe{ 0.30 }{ vids/vid-4b909508-01.png }&\vidframe{ 0.30 }{ vids/vid-4b909508-02.png } \end{tabular}\caption{Figure 7. A vibration mode of a rectangular plate with free edges, this time dominated by twist. This mode  often has the lowest frequency of the three shown here.}\mooendvideo

  The first thing you learn from doing beam or plate tests like these does not 
  come as a surprise: the stiffness along the grain is always much higher than 
  the stiffness across the grain. This is a direct result of the cellular 
  structure of spruce. Remember the microscope images from section 10.2, and 
  the analogy with a bundle of drinking straws? It is intuitively clear that a 
  bundle of straws will feel very stiff if you press it lengthwise, but much 
  more squashy if you press across the width. The reason is that to deform the 
  bundle lengthwise, you need to compress every straw along its length. But to 
  deform the bundle sideways, each straw simply needs to distort into an oval 
  shape. The walls of the straws bend, rather than compressing or stretching, 
  and because they are very thin they do not provide much stiffness against 
  this bending. For more detail about this bending/stretching idea, see the 
  book on foams and honeycombs by Gibson and Ashby [2]. 

  The full story of the link between wood stiffnesses and cellular structure 
  is, needless to say, more complicated than this simple picture. Several 
  factors play a role. First, the annual ring structure introduces relative 
  stiff planes (the dense summer wood) interleaved with the spring wood which 
  is much less stiff. This layered structure has different effects on the 
  different stiffnesses. 

  One counterintuitive effect is that it should make the stiffness in the T 
  direction higher than the one in the R direction. The reason is that the 
  layers with alternating stiffness are connected ``in series'' in the R 
  direction but ``in parallel'' for the T direction. But the actual stiffnesses 
  are the other way round: the R stiffness is slightly higher than the T 
  stiffness. The explanation is that the effect of the layered structure is 
  counteracted by another effect, which turns out to be stronger. The cells in 
  the T direction are randomly staggered, but the cells in the R direction are 
  quite strongly aligned. This is clear in the micrographs shown in section 
  10.2, one of which is repeated below as Fig.\ 10. The alignment is partly to 
  do with how the cells divide in the growing tree, and partly to do with the 
  constraining effect of the medullary rays which interpenetrate the tracheids. 
  This alignment has the effect of increasing the R stiffness relative to the T 
  stiffness. For details of all this, see Kahle and Woodhouse [3]. 

  There is, of course, a link between the 9 stiffnesses that characterise the 
  solid wood, and the 4 needed for a thin soundboard. Investigating that link 
  reveals something interesting. There are a few published measurements of the 
  full set of 9 stiffnesses for spruce of suitable quality for soundboards. 
  Some of these results are surprisingly old [4]: they go back to the time when 
  aircraft were made of wood, and naturally enough aircraft manufacturers 
  wanted low density and high stiffness, just as instrument makers do. Using 
  some standard results from the mathematical theory of elasticity (see [1] for 
  details), we can calculate the 4 plate stiffnesses from the measured values 
  of the 9 stiffnesses of the solid. 

  We can use that calculation to investigate a question of immediate 
  significance to instrument makers. Suppliers of wood for musical instruments 
  will usually take great care to cut the tree parallel to the grain. However, 
  in the interests of getting the most out of each tree they do not necessarily 
  cut every soundboard blank exactly in the LR plane, so called “quarter-cut” 
  wood. Instead, the annual rings may go through the thickness of the plate at 
  some angle which we can call the “ring angle”. 

  How much difference does this angle make to the behaviour of the soundboard? 
  Figure 8 gives an example of the answer to that question. This shows the 
  three “plate stiffnesses” as a function of ring angle, going from $0^\circ$ 
  (perfectly quarter-cut) all the way round to $90^\circ$, when the annual 
  rings are in the plane of plate. The bending stiffness along the grain, shown 
  in red, doesn’t change very much. Neither does the twisting stiffness, shown 
  in black. But the blue curve for the bending stiffness across the grain shows 
  very dramatic variation (noting that the plot uses a logarithmic scale 
  because the range of values on the vertical axis is very large). As the ring 
  angle moves away from zero, the stiffness falls steeply — only $10^\circ$ is 
  enough to reduce the stiffness by a factor of 2. It continues to fall until 
  the ring angle reaches about $45^\circ$, reaching a value nearly 10 times 
  lower than the quarter-cut case. After that the stiffness climbs again, 
  ending up only about a factor 2 lower than where it started. 

  \fig{figs/fig-fbe59ba8.png}{Figure 8. The three important stiffnesses of a 
  flat plate, cut from solid spruce with a range of ring angles. The red line 
  shows the long-grain stiffness, the blue line shows the cross-grain 
  stiffness, and the black line shows the twisting stiffness.} 

  Figure 9 shows a spruce plate (sold for the purpose of making a guitar 
  soundboard) in which the ring angle is close to $45^\circ$. The stiffnesses 
  of this particular plate have been measured by the procedure described in the 
  previous link (see [1] for the detailed results), and the values were in line 
  with the predictions in Fig.\ 8. The ratio of long-grain stiffness to 
  cross-grain stiffness was nearly 70:1, whereas for quarter-cut spruce that 
  ratio would be more like 16:1. When you flex this plate in your hands, it is 
  scarily bendy in the cross-grain direction. 

  \fig{figs/fig-bfa05589.png}{Figure 9. A piece of Norway spruce supplied for a 
  guitar soundboard. Notice that the annual rings go through the thickness at 
  an angle close to $45^\circ$. This results in very low cross-grain 
  stiffness.} 

  We can understand why the cross-grain stiffness gets so low with this 
  intermediate ring angle by again thinking about the cell structure of spruce. 
  Figure 10 shows a repeat of Fig.\ 2 from section 10.2. Imagine a rod cut from 
  this wood, aligned with the horizontal axis in this image. The annual rings 
  would then cut across the rod with a ring angle in the vicinity of 
  $45^\circ$. 

  Now think what would happen if you tried to stretch this rod. The dense 
  portions of the annual rings would act like fairly rigid plates, but the very 
  open spring wood could easily deform by shearing motion. At the cell level, 
  the deformation would only involve bending the thin cell walls in the spring 
  growth into a slight S-shape. The effective stiffness for that shearing is 
  extremely low (see [3] for a careful analysis). The rod would stretch rather 
  in the way a deck of playing cards can be spread, by sliding each card over 
  its neighbour: each card corresponds to the dense layer in one annual ring. 
  Another analogy of this effect of an intermediate ring angle is with woven 
  fabric. Most fabric is quite stiff if you try to stretch it parallel to the 
  threads of the weave, either warp or weft. But at $45^\circ$ the fabric 
  stretches easily, by shearing motion: this is the basis of ``bias cutting''. 

  \fig{figs/fig-dac41904.png}{Figure 10. SEM image of the cell structure of 
  spruce, showing a section in the RT plane, the ``end grain''. This is a copy 
  of Fig. 2 from section 10.2.} 

  A consequence of the dramatic variation shown in Fig.\ 8 is that, so far as 
  cross-grain stiffness is concerned, the ring angle probably makes at least as 
  big a difference as the variation between one piece of wood and another. 
  Apart from anything else, the ring angle is not usually constant across the 
  entire width of a soundboard. Look again at Fig.\ 3, and see the variation in 
  ring angle within each of these 6 samples — and these particular beams were 
  all cut from the same tree! 

  If you want to maximise cross-grain stiffness, you need to keep the ring 
  angle close to zero. For a violin maker, this brings in another variable. The 
  arched top plate of a violin is (usually) carved out of solid wood. An 
  ingenious maker might try to choose wood where the variation of ring angle 
  “fans out” in a way that follows the arch of the plate. Figure 11 shows that 
  Stradivari might have done this, at least in the particular violin shown 
  here. The image is a close-up of Fig.\ 5 from section 10.2: a CT scan of the 
  violin, which shows the detailed configuration of each annual ring. In the 
  section between the f-holes, the rings do indeed fan out so that they follow 
  the arch to an extent. 

  \fig{figs/fig-5054f264.png}{Figure 11. Still from a video animating the 
  result of a high resolution CT scan of a Stradivari violin. This is a 
  close-up of Fig. 5 from section 10.2. Image copyright violinforensic, Rudolf 
  Hopfner, Vienna, reproduced by permission} 

  Having dealt with the mass and stiffness properties of wood, the remaining 
  material property relevant to vibration is the damping. Damping behaviour is 
  tied very closely to the stiffnesses that we have just discussed. The details 
  are a bit messy (see the next link), but we don’t really need to know more 
  than the fact that each stiffness has a damping factor associated with it. 
  They can be measured as part of the same beam or plate testing procedures 
  that we have already described. As well as measuring the frequency of each 
  beam or plate mode, you also measure its decay rate, or equivalently its Q 
  factor. 

  If you are wanting to select wood for a marimba bar, then this is indeed the 
  measure of damping that you will need to take into account. But if your 
  interest is in the damping of modes of a violin body, things are more 
  complicated. Damping is a measure of how rapidly energy is dissipated when 
  something vibrates. What we have talked about so far is material damping: 
  energy dissipation arising directly from the fact that the material, wood in 
  our case, is stretching, bending or twisting. But there are other physical 
  mechanisms for energy dissipation, and the total damping of a mode is the 
  combined effect of all of them, added together. 

  Any kind of fixture for holding your instrument while you test it is likely 
  to introduce extra damping. This was already mentioned in the previous side 
  links on beam and plate measurement: for damping measurements based on beam 
  or plate testing, it requires great care to support the beam or a plate in a 
  way that doesn’t add a lot of extra damping. We will come back to the issue 
  in section ?, when we talk about measuring frequency response functions. 
  Furthermore, if you attach any kind of sensor to the test specimen to measure 
  the vibration, that, too, will add damping. Energy may be dissipated by the 
  sensor itself or in a layer of adhesive by which it is attached, and also 
  energy may be lost along the the sensor’s signal cable. 

  If holding fixtures are so problematic, why have I suggested doing tests with 
  free, unsupported beams and plates? Why not clamp a beam at one end, and 
  measure the frequency and damping as it vibrates like one tine of a tuning 
  fork? The answer is quite surprising: clamping a beam introduces a subtle but 
  significant extra mechanism of energy dissipation. The sketch in Fig.\ 12 
  shows why. A beam (yellow) is held in a clamp (grey). Now think what happens 
  if the beam bends downwards as indicated by the black lines. This bending 
  involves stretching the top surface of the beam, and compressing the bottom 
  surface. Near the edge of the clamp, the beam wants to move, slightly, as 
  indicated by the red arrows. When this situation is analysed carefully, it 
  turns out that for a sharp-cornered clamp like the one sketched here, the 
  limit of friction will always be exceeded near the corners. The surfaces of 
  the beam must slip a little against the clamp jaws, and this will result in 
  energy dissipation by friction. 

  \fig{figs/fig-0999b7de.png}{Figure 12. Sketch of a vibrating beam held in a 
  clamp. When the beam bends, a small amount of sideways slipping occurs near 
  the corners of the clamp, so that energy is dissipated via friction.} 



  \sectionreferences{}[1] M. E McIntyre and J. Woodhouse, “On measuring the 
  elastic and damping constants of orthotropic sheet materials”, Acta 
  Metallurgica \textbf{36}, 1397—1416 (1988). 

  [2] L. J. Gibson and M. F. Ashby, “Cellular solids”, Pergamon Press (1988). 

  [3] E. Kahle and J. Woodhouse, “The influence of cell geometry on the 
  elasticity of softwood”, Journal of Materials Science \textbf{29}, 1250—1259 
  (1994). 

  [4] H. Carrington,”The elastic constants of spruce”, Philosophical Magazine 
  \textbf{45}, 1055—1057 (1923). 