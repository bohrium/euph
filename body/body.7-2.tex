

  Musical strings come in a great variety. Some are simple single-strand 
  monofilaments: made of steel for the piano or steel-string guitar, or of 
  various polymers for classical guitars, harps and so on. Others have a more 
  complicated construction, with a core of some kind over-wound with one or 
  more layers of wire or tape, as in the lower strings of a piano or guitar, or 
  any of the strings of a cello. Figure 1 shows some examples. The piano has 
  plain steel strings for the higher notes, three strings per note. Lower down, 
  the strings switch to having a steel core, over-wrapped with copper wire. The 
  lowest notes have just one string per note, slightly higher notes have two 
  strings per note. The classical guitar, on the other hand, has one string per 
  note throughout. The top three strings are plain nylon (which you can see is 
  transparent), while the three bass strings have a core of stranded nylon (not 
  visible in the picture), over-wrapped with copper wire which has been plated 
  to appear silver. Other instruments might have strings using natural gut, 
  either plain (as in many harps), or over-wrapped (as in some traditional 
  violin strings). Why is there so much variety, and how does a player or 
  instrument maker set about choosing the “right” strings for an instrument? 

  
  \fig{figs/fig-a30d90bc.png}{\caption{{``cameraPreset'':7,''cameraType'':''Wide'',''macroEnabled'':false,''qualityMode'':2,''deviceTilt'':0.18485130041882947,''customExposureMode'':0,''extendedExposure'':false,''whiteBalanceProgram'':0,''cameraPosition'':1,''shootingMode'':0,''focusMode'':0}}} 

  
  \fig{figs/fig-c864a8ae.png}{\caption{{``cameraPreset'':7,''cameraType'':''Wide'',''macroEnabled'':false,''qualityMode'':2,''deviceTilt'':-1.4527434907003576,''customExposureMode'':0,''extendedExposure'':false,''whiteBalanceProgram'':0,''cameraPosition'':1,''shootingMode'':0,''focusMode'':0}}} 

  Back in Chapter 5, we already met some of the key ideas. Now we will approach 
  the question more systematically, and support the discussion with sound 
  examples so that the perceptual effect of some of the key factors can be 
  experienced. The first stage is to put ourselves into the player’s position, 
  and try to understand what they want to achieve or to avoid when they choose 
  a string. As often with these musical questions, the answer is surprisingly 
  complicated. There are some aspects that are common to all stringed 
  instruments, plus some extras that are specific to instruments of particular 
  types. 

  The starting point is to be clear about what we already know, before we begin 
  the selection process. We have the instrument in front of us, so we know the 
  length of the string (by which we mean the vibrating length, between the 
  bridge and the nut). We also know what note the string is to be tuned to. 

  One thing we have already seen is that the impedance of a string is 
  important, because it determines how strongly the string vibration couples to 
  the instrument body. This in turn governs the loudness of the note when that 
  string is played. To remind you of two key facts, the impedance is governed 
  by the product of the tension and the mass per unit length of the string. 
  But, for a string of a given length, the fundamental frequency is governed by 
  the ratio of those two quantities. So that ratio is fixed by the fact that we 
  know the tuning. This seems to leave us with just one choice. We can choose a 
  heavier or a lighter string, and we will then adjust the tension in 
  proportion to the mass per unit length in order to tune correctly. A heavier 
  string needs a higher tension, and so has a higher impedance; a lighter 
  string vice versa. 

  But impedance is not the only thing that changes when you choose a heavier, 
  tighter string. We can start by thinking about monofilament strings. Once you 
  have chosen a material, with a particular density, the only choice you have 
  is the string diameter. This determines the mass per unit length, in 
  proportion to the cross-sectional area of the string. But, as we saw back in 
  section 5.4, other things change when you replace a thin string with a 
  thicker one. First, the bending stiffness increases. That makes the natural 
  frequencies deviate more markedly from ideal harmonic relations. 

  It is easy to use computer-synthesised sounds to illustrate the effect. We 
  can use the synthesis method described in section 5.4, and in the computer we 
  can try combinations that would be difficult to achieve on a real physical 
  instrument. Sound 1 gives three guitar-like notes, and the only thing that 
  has changed between them is the diameter of the string. The chosen note 
  represents the third string of a guitar, played at the 12th fret: in other 
  words, the note G (392 Hz) on a string of length 325 mm. To avoid 
  complications associated with damping, which we will discuss in a moment, all 
  modes of the string are assumed to have the same Q-factor with value 3000. 
  This is a number typical of a steel guitar string. The string is coupled to a 
  guitar body, but the admittance has been artificially reduced by a factor of 
  1000, to create an essentially rigid bridge (again to avoid confusion from 
  extra effects). 

\audio{}

  These ``strings'' are made of steel, with the appropriate density and Young's 
  modulus. The first note assumes a string of 0.3 mm diameter: a typical value 
  for the top string of a guitar, much thinner than you would expect for the 
  third string. The second note assumes a string of 0.6 mm diameter. It sounds 
  perceptibly different from the first note, with a slightly ``jangly'' 
  quality. This diameter is about what you might expect if you were to fit a 
  solid steel third string to a guitar, and follow the trend from the first and 
  second strings. But sets of steel guitar strings do not usually use a solid 
  third string, probably to avoid the jangly sound just illustrated. Instead, a 
  thinner core is used, with over-wrapped wire to increase the mass without 
  increasing the bending stiffness so much. 

  The third note in Sound 1 assumes a steel string of diameter 1 mm, comparable 
  to a piano string. Now the sound is drastically different. You probably hear 
  the ``pitch'' going up compared to the earlier notes, although if you listen 
  out for it you can hear that the fundamental frequency hums on at exactly the 
  same pitch as the other two notes. This strange, percussion-like sound is an 
  extreme example of the effect of bending stiffness: no-one would want to 
  choose a guitar string making a sound like this, unless the player is looking 
  for special timbres along the lines of the ``prepared piano''. 

  This example brings out another factor that influences string choice. The 
  three ``strings'' used here require very different tensions to tune them to 
  the assumed note: the tensions would be 36 N, 143 N and 398 N respectively. 
  The middle value is typical for actual steel guitar strings. The first 
  tension would probably strike a player as far too low: such a string would be 
  very floppy, and in danger of giving fret buzzes all the time. The third one, 
  on the other hand, has such a high tension that it is probably more than the 
  structure of a guitar body is designed to take. It might also feel 
  uncomfortably tight to the player, but that issue is probably less 
  significant than the danger of the bridge coming unstuck from the soundboard! 
  This highlights an important thing for the process of string selection: for a 
  given string on a given instrument, there will be a minimum and a maximum 
  acceptable tension. We will return to that issue in a bit, when we build up a 
  design chart for string selection. 

  But first we need to be aware of a second thing that happens when the string 
  diameter is changed: the damping of the string vibration changes. This 
  doesn't matter very much for metal strings, but for nylon or gut strings it 
  can make a very big difference to the sound, to the extent that it is a 
  critical factor to take into account in the process of selecting strings. 

  Back in section 5.4, we discussed two sources of damping in a vibrating 
  string. One is associated with the air surrounding the string: energy is lost 
  to viscosity of the air. It was shown in section 5.4.5 that this effect is 
  strongest at low frequencies. The second source of damping arises from the 
  fact that string materials like nylon or gut have very high intrinsic 
  damping. Think of the sound you would hear if you tapped a nylon wine beaker: 
  it would be a dull thud, quite unlike the ringing sound of a glass one. This 
  high material damping can influence our vibrating string through the effect 
  of bending stiffness: it depends on exactly the same parameter that 
  determines the inharmonicity we just heard in Sound 1 (see section 5.4.4 for 
  the details). Energy loss by this mechanism increases rapidly at high 
  frequency. 

  The combined effect of the two things is that the damping as a function of 
  frequency follows a U-shaped curve: from highish values at low frequency due 
  to the air viscosity, it falls with increasing frequency until the bending 
  effect comes in, making the curve turn round and grow. Somewhere in between 
  it will hit a minimum value where the string can ring the best. Strings of 
  different diameters will follow different versions of this curve. Fat 
  strings, with high bending stiffness, will show the rising trend sooner and 
  more strongly. 

  This expectation is borne out by measurements. Figure 2 shows some results 
  for three different nylon strings, all attached to the same guitar body. The 
  discrete symbols show measured values of frequency and loss factor, in three 
  colours for the different strings. The black points are for a very thin 
  string, the red ones for a regular guitar top string: both are acceptably 
  ``musical-sounding''. The green points, though, show the results for a very 
  thick string, which gave a very unsatisfactory dull thump when plucked. The 
  results show why: except for the very lowest frequency, where the three 
  strings show comparable values of loss factor, the green points indicate very 
  high damping, increasing rapidly with frequency. 

  \fig{figs/fig-a7e94a11.png}{\caption{Figure 2. Measured loss factors as a 
  function of frequency for three different nylon strings, all fitted to the 
  same guitar. Curves in corresponding colours show the prediction for these 
  three strings, using the theory from section 5.4.}} 

  Figure 2 also shows, as solid lines, the predictions of the theory from 
  section 5.4. The U-shaped curves are evident, and the curves follow the 
  measured points satisfyingly well. The dash-dot lines show the contribution 
  to the loss factor from the bending effect on its own, rising rapidly with 
  frequency. This rising trend creates a high-frequency roll-off in the sound 
  of each plucked string. This can be used to formulate a criterion for a 
  ``musically acceptable'' string, which distinguishes the red and black 
  strings from the unacceptable green one. We will come to the details shortly. 

  I said earlier that this damping effect is not important for metal strings. 
  To see why, Fig.\ 3 shows results corresponding to Fig.\ 2, but for a steel 
  guitar string. The scales are the same as in Fig.\ 2. This time the loss 
  factor is still falling when we reach 10 kHz. The internal damping of the 
  kind of steel used for musical strings is very low, so we simply don't see a 
  dramatic rising trend of damping associated with string bending stiffness. 
  Comparing Figs.\ 2 and 3, it is no surprise that a steel string can sound 
  quite different from a nylon string, even a very thin one. The steel string 
  can deliver long-ringing sound up to very high frequencies, most evident if 
  it is played with a hard plectrum in order to excite frequencies that high. 

  \fig{figs/fig-a13e29bb.png}{\caption{Figure 3. Measured loss factor as a 
  function of frequency, in the same format as Fig. 2, for a typical steel 
  guitar string.}} 

  \textbf{Building a string selection chart} 

  We are now ready to pull together the things we have seen about string 
  selection in graphical form to make a design chart. We will develop this in 
  stages, for the particular case of nylon strings. For reasons explained in 
  the next link, it will turn out that we can express all the things we need in 
  terms of a single parameter, which is the product of the string length (in 
  metres) and the tuned frequency (in Hz). The value of this parameter, 
  remember, is something we already know for any given string. Also, the value 
  does not change if you play different notes on the same string: the 
  fundamental frequency of a string is proportional to the length so the length 
  cancels out. We will put this product along the horizontal axis of a plot, 
  and on the vertical we can indicate the string diameter. The result of the 
  first stage of construction is shown in Fig.\ 4. Two things have been added 
  to the axes just described. First, the set of curves in shades of copper 
  colours are lines of constant tension: the colour code for values of tension 
  is shown in the sidebar. 

  \fig{figs/fig-d0fcac35.png}{\caption{Figure 4. The string selection chart for 
  nylon, first stage. The curves show lines of constant tension. The vertical 
  lines indicate the ultimate strength of a nylon string. Beyond the solid 
  black line the string will eventually break; beyond the dashed line it will 
  break almost immediately.}} 

  Two vertical lines have also been added to the plot. These are connected with 
  the ultimate strength of a nylon string, before it breaks. Somewhat 
  surprisingly, if you take a nylon string of any diameter and stretch it until 
  it breaks, the highest frequency you will reach is the same, regardless of 
  the string diameter. You need higher tension with a thicker string, of 
  course, but breaking is determined by the stress in the string (force per 
  unit area), and it turns out that stress depends only on the position on the 
  horizontal axis. The algebra lying behind this claim is given in the next 
  link. There are two versions of the vertical line. To the left of the solid 
  line, you can expect a nylon string to last more or less for ever without 
  breaking. But once you are to the right of the line the string will continue 
  to stretch, more or less rapidly, and eventually it is likely to break. The 
  further right you move, the shorter the lifetime. The dashed line marks the 
  most extreme string I could find in actual use on a professionally-played 
  instrument. 

  The next stage of constructing our selection chart is shown in Fig.\ 5. A new 
  set of curves has been added, in magenta dashes. These are lines of constant 
  impedance. They have a somewhat similar falling trend to the constant-tension 
  curves, but quite different in detail. Again, the algebra behind these lines 
  is explained in the previous link. 

  \fig{figs/fig-9c4b88df.png}{\caption{Figure 5. The string selection chart for 
  nylon, second stage. The magenta curves show lines of constant impedance.}} 

  The final step is to add a limit on string choice based on the damping effect 
  described above and to be illustrated shortly in Sound 2. The details are 
  described in the next link. This time, the limiting condition for a 
  musically-acceptable string does not give a single curve in the chart, 
  because it also depends on the length of the string. Figure 6 shows some 
  examples of the result. The upward-trending black curve is the limit for the 
  open string of a guitar, with length 650 mm. The sound of that open string 
  should be acceptable if it lies below this black curve. The dashed curve is 
  the corresponding limit for playing at the 12th fret of the guitar, so that 
  the vibrating length has halved to 325 mm. We see immediately that a given 
  string will be in increasing danger of crossing the limit line as you play it 
  in higher and higher positions: the string's point on the chart stays fixed, 
  but the limit curve moves progressively downwards. 

  \fig{figs/fig-226c224f.png}{\caption{Figure 6. The complete string selection 
  chart for nylon. The damping threshold curves plotted here correspond to 
  guitar strings: the solid black curve is for an open string of length 650 mm, 
  the dashed curve for the same string played at the 12th fret, with length 325 
  mm. Blue stars show the top three strings for a typical set of classical 
  guitar strings. Red stars indicate the three strings used in the synthesised 
  plucks in Sound 2.}} 

  Figure 6 also includes some discrete points. The blue stars mark the position 
  in the chart of the top three strings of a typical classical guitar set. It 
  can be seen that this set of strings has been chosen as some kind of 
  compromise between constant tension and constant impedance. The third string, 
  the leftmost of the three points, comes the closest to the limit lines 
  associated with damping: indeed, the chart suggests that if that third string 
  is played up to the 12th fret the sound quality will become muffled to an 
  extent that may be considered marginal. This is indeed an effect familiar to 
  guitarists. 

  But, of course, you want to hear the effect for yourself. Sound 2 gives three 
  synthesised notes, corresponding to the three red stars in the plot. The note 
  played corresponds to the 12th fret, so the important comparison is with the 
  dashed limit line. The first sound goes with the lowest star. It is well 
  below the line, and sounds fine. The second corresponds quite closely to the 
  actual third string played at this fret, and it is close to the line. The 
  sound is clearly less ``twangy'' than the first note. The third note is well 
  above the line, and illustrates what I mean by ``not musically acceptable''. 

\audio{}

  To see what the chart in Fig.\ 6 is telling us, a schematic version is shown 
  in Fig.\ 7. The string needs to be chosen to be not too tight and not too 
  slack. Obviously, the string should not break, or at least not too quickly. 
  Finally, it should lie below the limit line associated with damping, for the 
  highest note to be played on the string (with the shortest vibrating length). 
  The player's choice of string diameter allows them to move along a vertical 
  line in the chart. Choosing a thicker string allows a louder sound (because 
  the impedance is increased), but if the string lies in the left half of the 
  plot, you run the risk of getting too close to the damping limit. So a 
  thicker string tends to be louder but duller, and conversely a thinner string 
  is quieter but can be brighter-sounding. 

  \fig{figs/fig-18eb9983.png}{\caption{Figure 7. Schematic version of the 
  selection chart, showing the main constraints on string choice.}} 

  It should be emphasised that the damping limit plotted in Fig.\ 6 is not a 
  precise one. The lines plotted there give an indication of when the sound 
  quality may start to deteriorate, in the manner illustrated by Sound 2. But 
  the criterion, explained in the previous link, is based on a very crude 
  assumption: an ``acceptable'' string should have at least 10 overtones with 
  damping lower than a certain threshold value. For the strings of a guitar, 
  this empirical criterion works quite well. But in a moment we will look the 
  harp, where the range of string lengths and tuned frequencies is far larger. 
  For that case it doesn't entirely make sense to have a criterion based on a 
  fixed number of overtones: 10 overtones of a low string on a concert harp 
  might only take you to 1~kHz, whereas 10 overtones on the top string of the 
  same harp would take you beyond 30~kHz, well beyond the range of human 
  hearing. Under those circumstances, you would guess that for the low strings 
  you would like to be well below the threshold as plotted here, but by the 
  time you reach the highest strings you could afford to be significantly above 
  the threshold line. 

  All of this has been about solid, monofilament strings. As the chart makes 
  clear, for an instrument with fixed-length strings like a guitar there is an 
  intrinsic problem associated with choosing bass strings. The length is fixed, 
  so as the tuned note gets lower you automatically move to the left in the 
  chart. We have already seen that the third string of a classical guitar is 
  close to the limit for a solid nylon string, if you want to be able to play 
  it in high positions. For the lower strings, if you want a bright and twangy 
  sound, you have to do something different. One choice is to use steel strings 
  rather than nylon strings, but that takes you into the different sound world 
  of the steel-string guitar, and we have also heard (in Sound 1) that thick 
  steel strings don't really work, for a different reason. Although the damping 
  problem goes away, the inharmonicity due to bending stiffness produces an 
  undesirable sound. 

  You are left with two choices. Most modern instruments, like the classical 
  guitar, opt to use over-wound strings for the lower notes. This construction 
  allows the string to be heavier without too much penalty in terms of bending 
  stiffness. The result is the typical appearance of a set of classical guitar 
  strings, as we saw in Fig.\ 1. 

  Things are more challenging if you play a period instrument like a lute, and 
  you want to aim for an ``authentic'' sound and feel. At least in the earlier 
  part of the lute's heyday, the technology of making wire-wrapped strings may 
  not have existed, or at least such strings would have been expensive and not 
  widely available. Players used plain gut strings, and the result in terms of 
  damping was very similar to the nylon strings we have been looking at. To 
  help with this intractable problem, lute-players used a trick. The strings of 
  a lute are in pairs called courses. For the lower courses, instead of using a 
  pair of fat strings tuned in unison they could use a thick-thin pair, with 
  the thinner one tuned an octave higher. When you pluck the two strings 
  together, you get the necessary low fundamental frequency from the thick 
  string, but the thinner string can supply higher overtones to add brightness 
  to the sound, because it lies further to the right in the chart and so is 
  further away from the damping limit line. 

  \textbf{Case study: the harp} 

  The harp makes an interesting case study in the use of the selection charts. 
  Unlike the guitar, a harp has strings of different lengths for each note. 
  This raises a new issue, to be taken into account when choosing strings: the 
  ``feel'' of the strings to the player. A simple way to characterise feel is 
  to ask ``how hard do you have to pull the string, in order to create a 
  desired displacement before you let go and start the vibration?'' A simple 
  static force-balance calculation reveals that this measure of feel is 
  governed by the ratio of tension to length. So if a harpist wants the feel to 
  stay roughly the same across the range of the instrument, they need to choose 
  higher tensions for the longer strings. 

  This is indeed what they do. Figure 8 shows the same selection chart for 
  nylon strings, annotated with a set of recommended string diameters for a 
  concert harp, from a string manufacturer. The pattern is quite different from 
  what we saw for the strings of a classical guitar. The tensions get very high 
  indeed for the lower, longer strings. The harp body has to be designed to 
  withstand this amount of internal force; and even so, a common mechanical 
  failure in harps involves the soundboard pulling away from the rest of the 
  body. 

  \fig{figs/fig-dfb7ea11.png}{\caption{Figure 8. String selection chart for 
  nylon strings on a harp. The length of harp strings increases for the 
  lower-frequency notes, but not enough to compensate fully for the frequency 
  difference, so that longer, lower-frequency strings still lie towards the 
  left-hand side of the plot as they did for the guitar strings in Fig. 6.}} 

  Because all the strings have different lengths, it is hard to show the 
  associated damping limit curves. In this plot, changing colours have been 
  used to indicate the pattern. The discrete points for the strings are 
  coloured red and blue, in alternating batches. A limit curve is plotted for 
  the length associated with the last string of each colour batch, in the 
  corresponding colour. It can be seen that these lines always fall rather 
  close to the change of colour in the batches of points. What this means is 
  that all these strings fall close to the suggested damping limit. 

  But many harpists do not choose nylon strings at all: they prefer the sound 
  of natural gut strings. Figure 9 shows the corresponding selection chart for 
  gut strings: it is similar in some ways to the nylon chart, but details 
  matter here and when you look in detail, every string material requires a 
  different version of the chart. The chart in Fig.\ 9 is again annotated with 
  discrete points based on the string manufacturer's recommended choice of 
  gauges for a concert harp. The same scheme of changing colours is used to 
  indicate the pattern of the damping limit curves. 

  \fig{figs/fig-86b07a9e.png}{\caption{Figure 9. String selection chart for gut 
  strings on a harp}} 

  This time, the colour changes between the batches of discrete points all 
  occur before the corresponding limit line, even though the actual values of 
  tension are higher than in the nylon example. Comparing with Fig.\ 8, each 
  colour change for the strings falls consistently further below the plotted 
  limit line for gut strings compared to nylon strings: gut strings allow the 
  harp to have a brighter sound than is possible with nylon strings. Probably 
  this is at least part of the reason why players continue to prefer gut 
  strings, despite their higher cost and higher sensitivity to changes in 
  humidity and temperature. 

  The most obvious difference between the charts for nylon and gut is in the 
  shapes of the damping limit curves. For nylon the curves bend down, whereas 
  for gut they are essentially straight. The reason for this difference lies in 
  the way the two materials respond to changes in tension. Man-made polymers 
  like nylon consist of long-chain molecules, tangled together a bit like a 
  bowl of spaghetti. As a string stretches under tension, these chains get 
  straightened out; more and more so as the tension is increased. This has a 
  consequence for an important mechanical property of the string, the Young's 
  modulus, which determines the stiffness along the length of the string. As 
  the molecules straighten, the string gets stiffer. This is ultimately what 
  causes the damping limit lines to bend over (for details, see [1]). But gut 
  does not behave like this: its Young's modulus stays essentially the same 
  whatever the tension. That results in the straight lines seen in Fig.\ 9. 

  Now look again at the points representing the harp strings in the two 
  materials. Because of the desire to use high tensions in order to improve the 
  feel, the string choices push into a region of the charts where they are 
  significantly different. The straight lines in the harp chart gave a bit more 
  ``headroom'' in the string choice, compared to the nylon chart. It looks as 
  if harpists exploit this to the full. 

  Just to complete the story, we should note that harpists actually use a more 
  complicated combination of strings. They do indeed prefer gut strings in the 
  middle range of the instrument. But a comparison of Figs.\ 8 and 9 reveals 
  something we haven't mentioned: the black lines indicating the breaking 
  strength are further to the left for gut than for nylon. So for the very 
  highest strings, nylon is generally used in order to improve the life 
  expectancy. For the very lowest strings, a different issue arises. The charts 
  indicate the use of very fat strings, which influences both the damping 
  effect we have discussed, and the inharmonicity we met earlier. For both 
  reasons, harpists typically choose over-wrapped strings for the lowest notes, 
  usually with a steel core. 

  \textbf{What about bowed instruments?} 

  The strings with the most complicated construction are not found on 
  plucked-string instruments, but on instruments like the violin or cello. The 
  vibrating string length of a cello is similar to that of a guitar, but you 
  can pay as much for a single cello string as for several full sets of 
  high-quality guitar strings. What is going on? The same range of physical 
  behaviour involving impedance, inharmonicity and damping is still important. 
  We will see in a bit that there are some extra details that also matter in a 
  bowed string, but in themselves these are probably not the main issue that 
  has driven string design towards such complicated and expensive 
  constructions. 

  A more important aspect of the explanation lies, probably, in the different 
  ways that a violinist and a guitarist interact with their strings during 
  performance. Think for a moment about the sounds made by a beginner on these 
  two instruments. A complete novice can pluck a single note on a guitar, and 
  obtain a satisfyingly “musical” sound. But, notoriously, a beginner on the 
  violin is likely to produce some very unpleasant noises before they learn how 
  to create an acceptable sound. 

  With a plucked string, the performer’s input is mainly limited to setting the 
  initial shape of the string. Once they let go, the string “does its own 
  thing”, producing a mixture of almost-harmonic overtones, each with its 
  characteristic decay profile. In a bowed string, by contrast, the player is 
  interacting with the string throughout the motion, all the time the bow is in 
  contact. The player is trying to make the string do particular tricks (with a 
  range of bowing techniques with names like sautillé and martelé). The string 
  may have other ideas, and respond to a wrong bow gesture with an undesirable 
  squeak or groan rather than the crisp note the player had in mind. Once a 
  player has learned to control these different ways to bow a string, they can 
  be rewarded by an instrument that offers enormous flexibility of musical 
  expression. But first they need to put in many hours, weeks and years of 
  practice: as the saying goes, there is no free lunch. 

  This is a first glimpse into the wild and complicated world of seriously 
  nonlinear instruments: we will engage with it properly in the next few 
  chapters. For the moment, all we need take away is that a violinist or 
  cellist cares more than a guitarist about some subtleties of string 
  behaviour. They are constantly fighting against a string that doesn’t really 
  want to do what they have in mind. Any small improvement in the string’s 
  mechanical behaviour might make their life easier, and allow them to play 
  better and more reliably. A similar thing happens with sports equipment: for 
  a tennis player or a cellist, any small enhancement that improves their 
  performance is valued, and worth paying serious money for. 

  I mentioned that there are some properties of a string that we have not 
  considered up to now, which can matter for a violinist or cellist. To satisfy 
  your curiosity, I will give a brief account now. But things will become 
  clearer when we look at bowed strings in detail, in Chapter 9. A bowed string 
  is driven into vibration by the friction force between the bow-hair and the 
  string, mediated by the rosin with which a player coats the bow-hair. This 
  introduces two new things to consider in string design and selection. First, 
  the surface of the string has to be suitable to interact with the rosin. It 
  helps if the surface is smooth, so wrapped strings need to be made with 
  ribbons of tape rather than with round wire. The physics and chemistry of the 
  string's surface may also play a role: bowed instruments do not generally use 
  nylon strings, for example, because rosin does not stick effectively to them. 

  The second consequence of driving a string by friction from a bow is that a 
  new type of string vibration can come into play. The friction force is 
  applied tangentially across the surface of the cylindrical string. So as well 
  as exciting transverse string vibration, it can excite torsional vibration in 
  which ``waves of twisting'' travel along the string. Torsional string 
  vibration does not (usually) contribute directly to much of the sound from an 
  instrument, but it does influence the details of the string motion. It can 
  provide an additional source of damping, and it can also contribute more 
  subtle effects: more on this in section 9.5.3. 



  \sectionreferences{}[1] J. Woodhouse and N. Lynch-Aird ``Choosing strings for 
  plucked musical instruments''. \tt{}Acta Acustica united with Acustica 
  \textbf{105}, 516-529, (2019)\rm{}.~ 