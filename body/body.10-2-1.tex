  This section shows a collection of scanning electron microscope images 
  relating to wood and woodworking. All the microscopy was carried out by 
  Claire Barlow. 

  First, a few pictures to show a bit more detail of the cell structure of 
  Norway spruce, following on from the brief description in section 10.2. 
  Figure 1 shows a cut in the LR plane in the top half and one in the RT plane 
  at the bottom, separated by a corner. The landscape is dominated by 
  tracheids, and there is an annual ring boundary in the centre of the image. 

  \fig{figs/fig-af12f4b8.png}{Figure 1. Specimen of Norway spruce, cut to 
  reveal the LR plane (top half) and the RT plane (bottom half).} 

  Looking carefully at the top half of this image, we can see some more 
  details. Dotted around, for example towards the top right-hand corner, you 
  may be able to make out a lot of small doughnut shapes. These are called 
  bordered pits. At the centre of each doughnut there is a hole through the 
  cell wall, connecting one tracheid with the one next door. These allow fluids 
  to travel through the wood, essential in the growing tree. But they are quite 
  complicated: each bordered pit has a kind of flap valve at its centre, 
  allowing the tree to close the holes. When the tree is cut down and the wood 
  dries out, these flap valves usually stick themselves permanently to one face 
  or the other, effectively sealing the hole. That is why dry wood cannot 
  easily be soaked with water. Water can only get into tracheids that have been 
  exposed on the surface, but it can’t get into the others because all the 
  bordered pits are closed. 

  Figure 2 shows a close-up view of a group of bordered pits. They are all 
  sealed shut, but two of them are sealed against the upper surface while two 
  are sealed against the lower surface, so they look further away. Figure 3 
  shows an example of a bordered pit which has been split in half by the 
  process of preparing the specimen. The valve flap, with its ``spider's web'' 
  support network, is sealed against the lower surface. Figure 4 shows a 
  similar example, except that this is a different species of softwood and in 
  this case the valve has not sealed shut but remained open in the dried wood. 

  \fig{figs/fig-1b160d61.png}{Figure 2. A group of bordered pits in Norway 
  spruce} 

  \fig{figs/fig-45252b64.png}{Figure 3. A bordered pit that has lost its top 
  half, so that the membrane forming a flap valve is visible} 

  \fig{figs/fig-0f02f769.png}{Figure 4. A bordered pit in a different species 
  of softwood, in which the valve has remained open rather than sealing shut} 

  Figure 5 shows a closer image of an “end grain” surface cut in the RT plane. 
  You are looking into the “tunnels” of a lot of tracheids. The two horizontal 
  lines running across the image are rays. If you look careful at the lower one 
  of these, then look into the tracheids just above it, you can make out ridges 
  running across the floor of those tracheids, where the other ray cells of 
  this stack are passing underneath. You may also be able to see some small 
  holes: these are piceoid pits, which connect tracheids to rays and allow 
  fluids to travel radially in the growing tree. 

  \fig{figs/fig-863a0b66.png}{Figure 5. Typical view of a cut in the RT plane 
  in Norway spruce.} 

  Figure 6 shows an even closer view of a similar RT cut. This time you can see 
  the joins between the tracheids. Each tracheid has a wall made up of helical 
  windings of cellulose (and other stuff). Adjacent tracheids are glued 
  together by a thin layer called the middle lamella. When wood is pulped to 
  make paper, the intention of the process is to keep individual tracheids 
  (“fibres”) intact, while ungluing them from their neighbours. The fibres can 
  then be persuaded to glue themselves together in a kind of mat: that is your 
  sheet of paper. In Fig.\ 6 you can also see a couple of triangular holes, 
  just above and below the ray running across the lower part of the picture. 
  These are probably the very last bits of the pointed ends of tracheids that 
  have been cut off during the preparation of the specimen. 

  \fig{figs/fig-40a1f9a5.png}{Figure 6. Close-up of a similar cut to Fig. 5.} 

  Figure 7 shows a different view, cut approximately in the LR plane. Figure 8 
  shows a corresponding view of a cut in the LT plane. In combination, these 
  two pictures give a good idea of the distribution of rays among the 
  tracheids. In Fig.\ 7 you can see parts of several rays, exposed on the 
  surface and then diving down under the next layer of tracheids. Figure 8 
  shows the columns of ray cells in end view, as lines of small holes among the 
  tracheid tubes. 

  \fig{figs/fig-47eb4d43.png}{Figure 7. Norway spruce, cut approximately in the 
  LR plane.} 

  \fig{figs/fig-28b0e33a.png}{Figure 8. Norway spruce, cut in the LT plane} 

  Figures 9 and 10 give a glimpse of the denser and more complicated cellular 
  structure of a hardwood. These are two views of the kind of maple (Acer 
  platanoides) normally used for the backs and sides of violins and their 
  relatives. A conspicuous feature in Fig.\ 9 is the set of large pores, 
  distributed rather uniformly throughout the volume for this timber. Some 
  other hardwood species, such as oak, are called “ring porous” because they 
  have pores concentrated in bands parallel to the annual growth rings. 

  \fig{figs/fig-31abaa71.png}{Figure 9. Typical structure of violin-quality 
  maple Acer platanoides.} 

  \fig{figs/fig-b81dd986.png}{Figure 10. Closer view of maple, as in Fig. 6.} 

  The remaining pictures are intended to give an impression of what happens to 
  the cellular structure as a result of common woodworking operations. Figure 
  11 shows something that can go wrong. A piece of Norway spruce has been 
  clamped in a vice, too tightly for the structure to withstand. A band of 
  collapsed cells runs across the picture. The intact tracheids in this image 
  also show a good selection of bordered pits and piceoid pits. 

  \fig{figs/fig-14411e2b.png}{Figure 11. A sample of Norway spruce that has 
  been clamped too hard in a vice, causing a line of cells to collapse.} 

  Figure 12 shows a related effect, brought on by a different operation. The LR 
  surface of the wood, in the top half of the picture, has been smoothed with a 
  heavy cabinet scraper. The end-grain part of the picture, in the lower half, 
  shows that the pressure of the scraper has collapsed the tracheids with thin 
  walls, in the first few layers below the surface. However, the denser cells 
  near the annual ring boundary have been cut rather than collapsing. 

  Figure 13 shows what happens when wood like this is moistened with water. 
  This is a process described by woodworkers as “raising the grain”, and the 
  picture shows you why. The water has made the collapsed cells pop back more 
  or less into their original shape. The result is a step at the annual ring 
  boundary: the cells that had been squashed now stand proud of the surface. 
  This would be repeated at every annual ring, giving the wood surface a ridged 
  texture. 

  \fig{figs/fig-1d303b74.png}{Figure 12. A sample of Norway spruce, showing LR 
  and RT planes as in Fig. 1. The top surface was prepared using a heavy 
  cabinet scraper, and this has collapsed the thin-walled tracheids in the top 
  two layers or so.} 

  \fig{figs/fig-0f65af8b.png}{Figure 13. A similar Norway spruce sample to the 
  one in Fig. 12, but this has been moistened to ``raise the grain''. The 
  collapsed cells have recovered their shape, making a step on the top surface 
  at the boundary of the annual ring.} 

  Figure 14 shows the different kind of surface texture which is generated by 
  smoothing the wood with abrasive paper with a very fine grit. There are some 
  fuzzy strands where the cell walls have been shredded by the abrasive 
  particles, but mostly the surface is rather smooth. Figure 15 shows the 
  result of using another kind of “abrasive”. Traditionally-minded violin 
  makers sometimes use the dried skin of a dogfish, a kind of small shark. 
  Dogfish skin indeed feels very abrasive to the touch. But the picture reveals 
  that the effect is rather different from the abrasive paper. The cell-wall 
  material is soft enough that it has been “smeared” by the action of polishing 
  with the dogfish skin: the action is more like burnishing that abrasion: 
  somewhat similar to the effect of the heavy scraper in Fig.\ 12. 

  \fig{figs/fig-912779cc.png}{Figure 14. The LR surface of a Norway spruce 
  sample which has been prepared with fine-grit abrasive paper.} 

  \fig{figs/fig-d2a182a0.png}{Figure 15. The LR surface of a Norway spruce 
  sample which has been prepared by rubbing with dogfish skin, a natural 
  ``abrasive''.} 

  We can see what is going on in Figs.\ 16 and 17. Figure 16 shows the surface 
  of pristine dogfish skin. It is covered with structures called denticles: 
  they are thought to give the living shark lower drag as it swims through the 
  water. Figure 17 shows what has happened after the dogfish skin has been used 
  extensively as an “abrasive”. The denticles have worn down to stubs. The 
  material of the denticles, unlike the grit in abrasive paper or the steel of 
  a cabinet scraper, is no harder than the cell-wall material of wood. So when 
  dogfish skin is used for smoothing a surface, both the wood and the dogfish 
  skin are worn away. 

  \fig{figs/fig-a18421c1.png}{Figure 16. The surface of a new sample of dogfish 
  skin, showing the ``denticles''.} 

  \fig{figs/fig-773e0e9f.png}{Figure 17. The surface of dogfish skin which has 
  been used to smooth wood, wearing the denticles down.} 