  The first step in modelling a soprano saxophone, or any other conical reed 
  instrument, is to calculate the input impedance for a truncated cone. This 
  will play the same role as the impedance of a cylindrical tube, which we 
  calculated in section 11.1.1 and used when building an idealised model of the 
  clarinet. Figure 1 shows the geometry: measured from the point of the 
  complete cone, the two ends of the truncated cone are at positions $x_1$ and 
  $x_2$, so the length of the tube is $L=x_2-x_1$. Although the diagram is 
  drawn with a rapidly flaring cone, in reality the instruments we are 
  interested in have much more gentle flares: the angle at the apex of the cone 
  is about $4^\circ$ for a soprano saxophone, about $1.5^\circ$ for an oboe and 
  as little as $0.8^\circ$ for a bassoon. These angles are all small, so in our 
  analysis we will make no distinction between horizontal distance $x$ and 
  radial distance $r$. 

  \fig{figs/fig-9685d652.png}{Figure 1. Sketch of the geometry of the truncated 
  cone.} 

  We already know from section 4.1.2 that the pressure inside the tube, for 
  sinusoidal excitation at frequency $\omega$, can be written as a sum of 
  right-travelling and left-travelling spherical waves: 

  $$p=A_R \dfrac{e^{i \omega t -ikr}}{r} + A_L \dfrac{e^{i \omega t +ikr}}{r} 
  \tag{1}$$ 

  where $A_R$ and $A_L$ are (complex) coefficients and the wavenumber 
  $k=\omega/c$ as usual, where $c$ is the speed of sound. The particle velocity 
  $v$ then must satisfy equation (2) from section 4.1.3, so that 

  $$\rho_0\dfrac{\partial v}{\partial t}=-\dfrac{\partial p}{\partial r} 
  \tag{2}$$ 

  where $\rho_0$ is the density of air. Substituting, 

  $$\rho_0 i \omega v = A_R \dfrac{e^{i \omega t -ikr}}{r^2} + A_L \dfrac{e^{i 
  \omega t +ikr}}{r^2}$$ 

  $$+ ik A_R\dfrac{e^{i \omega t -ikr}}{r}- ik A_L\dfrac{e^{i \omega t 
  +ikr}}{r} \tag{3}$$ 

  so that 

  $$v=\dfrac{1}{i \omega \rho_0}\left[ \dfrac{p}{r}+\dfrac{ik}{r}\left(A_R e^{i 
  \omega t -ikr} -- A_L e^{i \omega t +ikr} \right) \right] . \tag{4}$$ 

  The tube is open at the right-hand end, so $p=0$ at $r=x_2$, giving 

  $$A_R e^{-ikx_2} + A_L e^{ikx_2} = 0 \tag{5}$$ 

  so that 

  $$A_L=-A_R e^{-2ikx_2} . \tag{6}$$ 

  Now at $r=x_1$ we have 

  $$p=\dfrac{e^{i \omega t}}{x_1} A_R \left[ e^{-ikx_1} -- e^{ik(x_1-2x_2)} 
  \right] \tag{7}$$ 

  and 

  $$v= \dfrac{e^{i \omega t}}{ik\rho_0 c x_1^2} A_R \left[ e^{-ikx_1} -- 
  e^{ik(x_1-2x_2)} \right] $$ 

  $$+ \dfrac{e^{i \omega t}}{i\rho_0 c x_1} A_R \left[ e^{-ikx_1}+ 
  e^{ik(x_1-2x_2)} \right] . \tag{8}$$ 

  So the input impedance $Z$, and its inverse the admittance $Y$, satisfies 

  $$\dfrac{1}{Z}=Y=\dfrac{S_1v}{p}=\dfrac{S_1}{\rho_0 c} \dfrac{e^{-ikx_1} 
  \left(1+\dfrac{1}{ikx_1}\right) + e^{ik(x_1-2x_2)} \left(1-\dfrac{1}{ikx_1} 
  \right)}{e^{-ikx_1} -- e^{ik(x_1-2x_2)}}$$ 

  $$=\dfrac{1}{Z_t}\left[ \dfrac{1+e^{-2ikL}}{1-e^{-2ikL}}+ 
  \dfrac{1}{ikx_1}\right]=-\dfrac{i}{Z_t}\left(\cot kL + \dfrac{1}{kx_1} 
  \right) \tag{9}$$ 

  where $S_1$ is the cross-sectional area at position $x_1$ and $Z_t=\rho_0 
  c/S_1$ is the characteristic impedance of a cylindrical tube with that area. 

  This gives the input impedance expression we want, but we should also note a 
  further approximation. For low frequencies such that $kx_1 \ll 1$, then $kx_1 
  \approx \tan kx_1$ so that 

  $$Y \approx -\dfrac{i}{Z_t}\left(\cot kL + \cot kx_1 \right) . \tag{10}$$ 

  This is the ``cylindrical saxophone'' approximation: it is precisely the 
  expression for the input admittance at distance $x_1$ from the end of a 
  cylindrical pipe of length $x_1+L$ which is open at both ends. Using a 
  standard trigonometric formula for $\cot A + \cot B$ we deduce that the zeros 
  of $Y$, which are the peaks of $Z$, occur where $\sin k(x_1+L)=0$, so that 
  they are exactly harmonically spaced. The same expression (10) (apart from 
  needing a different constant in place of $Z_t$) gives the impedance at 
  position $x_1$ on a string of length $x_1+L$, which is the basis of the 
  analogy with a bowed string. 

  If we want to couple our truncated cone to a mouthpiece cavity with volume 
  $V$, we simply need to replace $Y$ from equation (9) by 

  $$Y_m=Y+\dfrac{i \omega V}{\rho_0 c^2} \tag{11}$$ 

  because the pressure has to be the same at the tube mouth and in the cavity, 
  while the total volume flow rate is the sum of the two contributions, so the 
  two admittances must be added. The expression for the admittance of the 
  enclosed air volume follows from equations (2) and (3) of section 4.2.1. The 
  required value of $V$ to match the volume of the missing portion of the cone 
  is 

  $$V=x_1 S_1/3 . \tag{12}$$ 

  However, equation (11) is only valid at low frequency, because it relies on 
  the assumption $kx_1 \ll 1$. At very high frequencies the influence of the 
  mouthpiece should tend to zero, so for the purposes of computation equation 
  (11) is replaced by an ad hoc version including a low-pass filter: 

  $$Y_m=Y+\left( \dfrac{\omega_c^2}{\omega^2+\omega_c^2} \right) \dfrac{i 
  \omega V}{\rho_0 c^2} \tag{13}$$ 

  with a cutoff frequency $\omega_c$ chosen suitably: for the saxophone model 
  discussed here this cutoff was placed at 1~kHz. 

  For the purposes of simulation, we need to deduce a reflection function from 
  the input impedance. The procedure was described in section 11.3.2. It 
  involves using an inverse FFT, and for that purpose the damping needs to be 
  sufficient to avoid numerical difficulties. As was done in the clarinet case, 
  the baseline level of damping was determined by textbook expressions for 
  visco-thermal damping, but this was enhanced in an ad hoc way at higher 
  frequencies. The resulting input impedances are shown in Fig.\ 2, for models 
  with and without the mouthpiece compensation. The difference between the two 
  looks very small in this plot, but plots shown in section 11.3 demonstrate 
  that it is big enough to have a significant effect on the results of 
  simulations. 

  \fig{figs/fig-12d3ef54.png}{Figure 2. The input impedance used for the 
  simulations: the red curve is for the truncated cone only, the blue one 
  includes the effect of the mouthpiece.} 

  The assumed level of damping was set in such a way as to resemble measured 
  results. Figure 3 shows a measured input impedance of a soprano saxophone 
  fingered for its lowest note, with all tone-holes closed, again taken from 
  Joe Wolfe’s \tt{}web site\rm{}. The general level and the extent of 
  peak-to-valley excursions at low frequencies give quite a good match to Fig.\ 
  2. But the real instrument has far higher damping above about 2.5~kHz, 
  largely as a result of the radiation characteristics of the bell, not 
  included in the simplified model here. 

  For comparison, Fig.\ 4 shows the “cylindrical saxophone” impedance from 
  equation (10), as a green curve superimposed on the two impedance curves from 
  Fig.\ 2. 

  \fig{figs/fig-841bc722.png}{Figure 4. The input impedances from Fig. 2, with 
  the addition of the result for the ``cylindrical saxophone'' model (green 
  curve).} 

  From the impedance functions shown in Fig.\ 2, it is straightforward to 
  calculate the corresonding impulse responses and the reflection functions, as 
  described in section 11.3.2. The results are shown in Figs.\ 5 and 6, for the 
  models with and without the mouthpiece, using the same colour code as in 
  Fig.\ 2. The reflection functions are more spread out in time than the 
  corresponding result for a cylindrical tube (shown in Fig.\ 4 of section 
  11.3.2). 

  \fig{figs/fig-453e1716.png}{Figure 5. Impulse responses deduced from the 
  input impedances of Fig. 2: the red curve without the mouthpiece, the blue 
  curve including the mouthpiece.} 

  \fig{figs/fig-3989fd7f.png}{Figure 6. Reflection functions deduced from the 
  input impedances of Fig. 2: the red curve without the mouthpiece, the blue 
  curve including the mouthpiece.} 

  Finally, while discussing the ingredients of the saxophone simulations this 
  is a convenient place to summarise the parameter values chosen for the reed 
  model. The width $w$ was set at 12~mm, the reed stiffness $K_r$ was chosen to 
  be significantly softer than the clarinet reed at 5~kPa/mm, the reed 
  resonance frequency was set at 3~kHz, and the reed Q-factor at 2.5. 