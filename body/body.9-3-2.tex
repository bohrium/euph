  To understand the experimental results shown in section 9.3 and 9.6, it is 
  useful to know a bit about the design and capabilities of the bowing machine 
  used in the studies. The bow motion is provided by a linear motor, which 
  propels a trolley carrying a system that simulates the player’s wrist. A 
  clamp that can hold either a conventional bow or a rosin-coated perspex rod 
  is flexibly mounted, and a vibration shaker provides an actuation force for 
  controlling the bow force. The system can be seen in Fig.\ 1, in this case 
  with the perspex rod in place. Strain gauge sensors monitor the force applied 
  to the bow clamp. Through a combination of open-loop control, closed-loop 
  feedback compensation and careful hardware design, the bowing machine can 
  change bow acceleration with a response time of around 10 ms while 
  maintaining constant bow force with an accuracy of $\pm3 \%$. Full details of 
  the mechanical design, control strategy and calibration procedure for this 
  rig can be found in [1]. 

  \fig{figs/fig-e7acb058.png}{Figure 1. The bowing machine, showing the perspex 
  rod ``bow'' extending into the distance, held by the aluminium clamping block 
  on a supporting structure which is flexibly mounted using the steel rule 
  visible in the foreground. The cylindrical object with the two wires attached 
  is a vibration shaker, providing the bow force.} 

  In all the experiments to be described here, this machine was used to bow the 
  D string of a cello. The cello is held in a supporting frame, which 
  approximately mimics how a player would hold the instrument. The cello sits 
  on its endpin as usual, and adjustments to the projecting length of that 
  endpin are used to set the bowing point $\beta$. The body of the cello is 
  supported by padded clamps providing the player’s “knees” and “left hand”. 
  The string to be bowed is aligned to be vertical, perpendicular to the line 
  of the bow, which is pressed against it in a horizontal plane. The 
  arrangement can be seen in Figs.\ 2 and 3. The bridge of the cello is 
  equipped with a piezoelectric bridge-force sensor, as described in section 
  9.1.1. 

  \fig{figs/fig-426ac110.png}{Figure 2. The bowing machine with the cello in 
  place.} 

  \fig{figs/fig-867cce77.png}{Figure 3. General view of the cello in its 
  supporting frame, with the bowing machine behind.} 

  The machine is designed for bow strokes in which the bow is in contact with 
  the string throughout: it cannot do “bouncing bow” strokes. But within that 
  family of bow gestures, the machine can equal or exceed the capabilities of a 
  human. For the experiments to be described here, two types of bowing gesture 
  were used. 

  For the Schelleng diagram tests, the bow speed was set to the constant value 
  0.05 m/s, and a carefully-tailored initial bowing gesture was used to 
  establish Helmholtz motion prior to each measurement. The force and speed 
  were then adjusted smoothly to the desired values, and these were sustained 
  for two seconds. Bridge force was measured for 0.1 s at the end of that time, 
  and used to classify the string’s motion. Changes in the value of $\beta$ 
  were made by hand, so each column of the Schelleng diagram was measured as a 
  separate run. 

  The Guettler diagram tests, to be described in section 9.6, were more 
  straightforward. The required gesture in each case involved a constant bow 
  force, and a bow speed starting from rest and then growing with a constant 
  acceleration. The bridge force for the first 0.25 s of the transient response 
  was recorded in each case. The values of force and acceleration were varied 
  to scan the Guettler diagram, and the entire $20 \times 20$ grid of points 
  was run in a single experiment, controlled by the computer. 

  \sectionreferences{}[1] Paul M. Galluzzo and Jim Woodhouse; 
  ``High-performance bowing machine tests of bowed-string transients'', Acta 
  Acustica united with Acustica \textbf{100}, 139--153 (2014) 