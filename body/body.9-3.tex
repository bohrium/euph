

  A beginner on the violin soon finds out that drawing a bow across a string 
  doesn’t always make a satisfactory musical note. They must learn to control 
  the parameters of bowing to stay within certain limits: those limits are the 
  subject of this section. For the moment we will ignore transient effects, and 
  concentrate on steady bowing with the aim of producing a steady note. Once 
  the instrument, the strings, the bow and the particular note have all been 
  chosen, the player has four things to control in a steady bow stroke: the bow 
  speed, the bow force, the position of the bow on the string, and the angle of 
  tilt of the ribbon of bowhair relative to the string. We will ignore tilt for 
  now: we come back to it in section 9.6. 

  That leaves bow speed, force and position to consider. If the speed and the 
  position are kept constant while the bow force is varied, a player quickly 
  finds out that the force must lie within a certain range in order to achieve 
  a satisfactory Helmholtz motion of the string. Press too hard, and instead of 
  a note the violin will make a raucous “crunch” noise. Press too lightly, and 
  a more subtle problem arises. A musical note is still produced, but the tone 
  quality changes in a way that is likely to be criticised by your violin 
  teacher: it is often described as “surface sound”, or “not getting into the 
  string properly”. The vibration regime responsible for this “surface sound” 
  is one we have already met: it is the double-slipping motion shown in Fig.\ 5 
  of section 9.1. 

  We deduce that there must be a minimum bow force and a maximum bow force for 
  Helmholtz motion. We will soon see that both these force limits change, 
  depending on the values of the other two control parameters: bow speed and 
  bow position. The first steps in understanding these limits were taken by 
  Raman, and then around 1970 his work was refined and extended by John 
  Schelleng [1] (1892--1979: he was only four years younger than Raman). 
  Schelleng presented his main results in the form of a diagram, which we will 
  see shortly. Schelleng, like Cremer, had a distinguished career in his “day 
  job”: in his case, research work in the Bell Telephone Laboratory. But, like 
  Cremer, he was a keen amateur string player, and in his retirement he devoted 
  a lot of his time to researching violin acoustics as a founding member of a 
  group of enthusiasts based in the USA who called themselves the “\tt{}Catgut 
  Acoustical Societ\rm{}\tt{}y\rm{}”. 

  Schelleng’s condition for maximum bow force is the simpler of the two limits: 
  it follows from the graphical construction relating to frictional hysteresis, 
  explained in section 9.2. In order for Helmholtz motion to be possible under 
  the constraints of the hysteresis rule, the Helmholtz slip speed must be 
  outside the range of the hysteresis. The details of the resulting formula are 
  given in the next link. 

  The condition for minimum bow force is a little more complicated. For an 
  ideal textbook string, there would be no minimum bow force, because Helmholtz 
  motion is a possible free motion of the string, not requiring any support 
  from the bow. However, as soon as we allow the system to have some energy 
  dissipation, steady motion is only possible if the bow somehow compensates 
  for the energy loss in each cycle. 

  Schelleng's condition is calculated by assuming that this energy loss all 
  takes place at the bridge, by coupling to the body of the instrument. 
  However, he did not allow for the detailed dynamical behaviour of the body. 
  Instead, he used an approximation rather similar to Weinreich's model for 
  double decays in a piano (see section 7.3): he modelled the body as a 
  mechanical resistance (or ``dashpot'') with an impedance assumed to be much 
  greater than the impedance of the string. We will return to this 
  approximation in the next section, and see how the result is modified by 
  using a more realistic model for the body vibration. But for the moment we 
  will follow Schelleng's approximate treatment. 

  The chain of logic runs like this. If we assume an ideal Helmholtz motion, 
  with a given bow speed and position, we already know what the waveform of 
  bridge force will be: it is the sawtooth we have seen several times now. That 
  force will produce some motion in the dashpot representing the body, with a 
  velocity waveform which will mirror the sawtooth shape. This body motion will 
  in turn create some additional force back at the bowed point. For Helmholtz 
  motion to be possible, this additional force during a sticking interval must 
  not exceed the limit of friction, otherwise it would trigger a second slip. 
  This argument gives the condition we are after: the details of the 
  calculation are given in the previous link. 

  According to Schelleng’s formulas, both the maximum and the minimum bow force 
  are proportional to the bow speed. So a faster bow speed means you need to 
  press harder, but because both limits change in the same way, the relative 
  range of possible bow force stays the same. The dependence of the force 
  limits on bow position is more interesting. We will describe it in terms of 
  the parameter $\beta$ (``beta'') introduced earlier, the position of the 
  bowed point as a fraction of the string length. According to Schelleng's 
  formulas, the maximum force is proportional to $1/\beta$, while the minimum 
  force is proportional to $1/\beta^2$. 

  These different dependencies on $\beta$ are the ingredients of Schelleng’s 
  diagram. If we plot the two force limits in a graph of bow force versus bow 
  position, and if we choose to use logarithmic scales for both these 
  variables, we get something like Fig.\ 1. Both bow force limits appear as 
  straight lines in the plot, and the line for the minimum bow force is 
  steeper. Helmholtz motion is only possible within the wedge-shaped region 
  between the two lines. As the bow is moved closer to the bridge, the player 
  needs to press harder, and also to control the force more carefully because 
  the available range decreases. The two lines meet at some point. At least 
  according to this approximate analysis, it is not possible to produce 
  Helmholtz motion if the bow is closer to the bridge than this meeting point. 

  \fig{figs/fig-bb3b52cf.png}{Figure 1. Schelleng's diagram} 

  Schelleng’s diagram is useful to beginning string players. The interaction 
  between bow force and bow position is not intuitively obvious, and the 
  graphical representation can help to fix the idea clearly in the mind. For 
  example, the diagram gives an immediate explanation of a common beginner’s 
  error, illustrated in Fig.\ 2. The player may keep the bow speed and the bow 
  force under control, but if they do not concentrate on the bow position they 
  may find themselves wandering backwards and forwards along a line like the 
  blue one in this plot. This can result in going above the maximum force or 
  below the minimum force, even though the force itself has not changed. 

  \fig{figs/fig-a72af2af.png}{Figure 2. Schelleng's diagram as in Fig. 1, 
  illustrating what happens if the bow position varies while the bow force 
  stays constant: the player may stray along the blue line and find themself 
  outside the Helmholtz region.} 

  Schelleng’s diagram gives us a first tool to ask questions about 
  “playability”. What makes one instrument “easier to play” than another? One 
  possible answer to that question is that an instrument with a larger 
  Helmholtz wedge in the Schelleng diagram might strike a violinist as being 
  easier to play. So it is of interest to ask whether the two bow force limits 
  would be expected to vary from instrument to instrument. Schelleng’s 
  approximate formula for the maximum bow force does not depend on anything to 
  do with the instrument body. The only things that appear in the formula (see 
  the previous link for details) are the bow speed and position, the 
  characteristic impedance of the string, and some information about the 
  friction coefficients. So choosing a different string or a different brand of 
  rosin for the bow might make a difference, but changing the body of the 
  violin should not. 

  The story is different for Schelleng’s formula for the minimum bow force. The 
  result depends upon the impedance of the dashpot representing the body, and 
  hence its potential for energy dissipation. Choosing a different value for 
  that impedance would result in shifting the line up or down, as indicated in 
  Fig.\ 3. The bigger the body impedance, the further down the line is shifted. 
  In the extreme case when the impedance becomes infinitely large, the line 
  would be pushed off the diagram entirely. Helmholtz motion would then be 
  possible for any bow force below the maximum. This is the case we already 
  mentioned, of an ideal string with no energy loss, which has no minimum bow 
  force. 

  \fig{figs/fig-68ec83c4.png}{Figure 3. Three alternative positions for the 
  minimum bow force line in Schelleng's diagram.} 

  There is a snag with trying to apply this idea to compare two different 
  violins: a violin body does not really behave like a dashpot, so it is hard 
  to know what value for the dashpot impedance we should be using in the 
  formula. We will return to this question in section 9.4, when we will show 
  how to extend Schelleng’s analysis to take account of the details of body 
  response, for example from a measured bridge admittance. 

  In this section and the two previous ones, we have described several features 
  of the physics of a bowed string, and made some predictions. We have by no 
  means exhausted the list of physics-related questions, but this is a good 
  point to look at some experimental results to assess how we are doing so far. 
  We will look at detailed results from a measured version of the Schelleng 
  diagram: they will shed some light on all the things we have discussed. 

  It is a rather laborious business to scan the Schelleng diagram 
  experimentally. It can really only be done with some kind of bowing machine, 
  which allows the speed and force of a bow stroke to be controlled in a 
  precise and repeatable way. Two such experiments have been carried out, both 
  as part of PhD projects: one by Paul Galluzzo, working in Cambridge [2], and 
  the other by Erwin Schoonderwaldt, working in Stockholm [3]. 

  Both give somewhat similar results, which is reassuring, but neither is 
  absolutely ideal for giving an overview of normal bowing of a violin string. 
  Schoonderwaldt’s experiment used a real bow on a violin D string, but the 
  string was attached to a rather rigid laboratory monochord rig. As we have 
  just seen, this might have major consequences for the position of the minimum 
  bow force line. Galluzzo’s experiment used a cello D string mounted on a real 
  cello, so the bow force limits should be realistic. However, he did not use a 
  real bow. His experiment was designed primarily to give experimental data for 
  probing theoretical models of bowed-string transients, and it is easier to 
  make detailed comparisons if the dynamics of the bow and the finite width of 
  the ribbon of hair are removed from the picture. So Galluzzo’s Schelleng 
  diagram test was carried out using a rosin-coated perspex rod rather than a 
  real bow. 

  I will show some results from the Galluzzo experiment. The bowing machine is 
  described in the next link, together with the procedure used in the Schelleng 
  diagram experiment. The most important aspect of the procedure was that the 
  ``bow'' was manipulated in a way that guaranteed Helmholtz motion at the 
  start of each note, with the aim of finding out whether it was possible to 
  maintain that motion subsequently. Results were collected for a $20 \times 
  20$ grid of points in the Schelleng plane: 20 values of $\beta$, 
  logarithmically spaced between 0.02 and 0.18, and 20 values of bow force, 
  logarithmically spaced between 0.1~N and 3~N. For each of these points, the 
  bridge force was recorded and then analysed to determine what the string had 
  decided to do. 

  \fig{figs/fig-3b8a22fe.png}{Figure 4. A measured version of Schelleng's 
  diagram. The colour code shows red for Helmholtz motion, orange for 
  double-slipping, yellow for decaying motion, black for non-periodic raucous 
  motion and white for S-motion. Blue arrows mark columns shown in detail in 
  Fig. 4. Green arrows mark columns investigated in Figs. 5--8.} 

  The results are shown in Fig.\ 4, using different colours to indicate 
  different vibration regimes. Helmholtz motion is shown in red, occupying a 
  large region in the middle of the plot. Below the Helmholtz region, orange 
  pixels indicate double slipping (or triple slipping, etc.). Yellow pixels 
  mark cases where the string did not really seem to want to vibrate at all. 
  The string was set into vibration by the initial bowing gesture imposed by 
  the bowing machine, but for these points it seems to be decaying away. Above 
  the Helmholtz region, black pixels mark cases where the motion was not 
  periodic, so some kind of “raucous” motion had occurred. Finally, white 
  pixels mark cases of S-motion. 

  To make sense of this classification, Fig.\ 5 shows extracts from the 
  measured bridge-force waveforms for a representative selection of four 
  columns of the Schelleng diagram (marked with blue arrows in Fig.\ 4). Each 
  panel of this plot shows all 20 values of bow force, stacked up in the same 
  way as in the Schelleng diagram. The left-hand panel shows the Helmholtz 
  sawtooth in the top few cases, then some double or triple slipping waveforms. 
  By the bottom of the stack, the waveforms look almost sinusoidal, as they 
  slowly decay away. Near the middle of the stack are a few cases where it is 
  really not at all clear how they should be classified: but they are certainly 
  not Helmholtz motion. 

  \fig{figs/fig-2090123e.png}{Figure 5. Extracts of waveforms from four of the 
  columns of Fig. 4, indicated there by the blue arrows.} 

  The second panel of Fig.\ 5 shows a similar sequence, but it is clear that 
  the Helmholtz sawtooth reaches further down the stack, just as indicated in 
  Fig.\ 4. The third panel shows double-slipping at the bottom, then a block of 
  Helmholtz motion cases, but at the top it shows something different. There is 
  an abrupt transition to S-motion waveforms. The right-hand panel is different 
  again. There are still some recognisable Helmholtz sawtooth cases in the 
  middle, but moving upwards we see a mixture with S-motion cases and 
  non-periodic “raucous” cases. At the bottom, nothing really looks like 
  double-slipping, but the waveforms shapes get progressively peculiar and 
  indistinct. For the purposes of Fig.\ 4 these have all been marked in yellow, 
  but it is not clear whether they are really “decaying”. 

  Having seen these examples of the waveforms lying behind Fig.\ 4, we can give 
  a general description of what that figure shows. Helmholtz motion is confined 
  within a wedge-shaped region, with double-slipping motion found below the 
  wedge, and raucous motion above it. So far, so good: all this is 
  qualitatively as predicted by Schelleng. In some regions, especially when the 
  bowing point is a long way from the bridge (i.e. larger values of $\beta$), 
  the red wedge is interrupted by columns of S-motion. It appears that for 
  certain values of $\beta$, the string prefers S-motion to Helmholtz motion, 
  especially with higher values of bow force. This is consistent with the early 
  measurements that inspired Raman’s study, as we saw in section 9.1. It is 
  also consistent with Lawergren's experiments and interpretation. 

  This behaviour doesn't contradict the Schelleng analysis, but we need to be 
  careful about what that analysis does and does not show. Schelleng's bow 
  force limits were both found by asking the question ``when does it become 
  impossible for Helmholtz motion to happen?'' But just because Helmholtz 
  motion is possible within the Schelleng wedge doesn't mean that it will 
  necessarily happen. When more than one type of string vibration is possible 
  (such as Helmholtz motion and S-motion), the question of which one you 
  actually get from a given bow gesture can be very subtle. We will come back 
  to this question in later sections. 

  In the lower part of Fig.\ 4 where the bow force is low, the classification 
  of regimes becomes rather indefinite. Is the string capable of producing a 
  version of Helmholtz motion in the bottom right-hand corner? Possibly a human 
  player could use more subtle bow control than the bowing machine; but the 
  patterns we have seen suggest that this would be a dangerously unreliable 
  region for a player to try to use. Indeed, the whole of the right-hand third 
  of the plot suggests that players would have a hard time in one way or 
  another, if they were trying to produce Helmholtz motion. This is the regime 
  known in violin jargon as sul tasto (over the fingerboard): this is a 
  specialised style of playing used for deliberate colouristic effects. We can 
  now see that those effects are probably associated with a kaleidoscopic 
  selection of different S-motion regimes. 

  There is more we can do with the Galluzzo data: we can use it to test some 
  ideas from section 9.2 about the effect of bow force and bow position on the 
  spectrum of a note, and also look for the predicted pitch flattening effect. 
  For this investigation, we concentrate on the cases flagged in Fig.\ 4 as 
  Helmholtz motion. 

  The sawtooth waveform makes it relatively easy to detect individual cycles of 
  vibration by looking for the flyback of the sawtooth then calculating the 
  time at which each one crosses zero. Having identified single periods, we can 
  use the FFT to calculate the frequency spectrum (in other words, the 
  magnitude of the Fourier series coefficients for that cycle). 

  Figure 6 shows some examples, for all the Helmholtz cases in the column of 
  Fig.\ 4 indicated by the left-hand green arrow. The colours of the plotted 
  lines change from red through blue to green, in sequence as the bow force 
  increases in that column. The spectrum level in dB is plotted against the 
  harmonic number, but it is easy to convert this into a frequency scale given 
  than the played note is the open D in every case, with a nominal fundamental 
  frequency just below 147~Hz. So the 10th harmonic is around 1.5~kHz, and the 
  upper limit of the plot at the 46th harmonic is a little below 7~kHz. 

  \fig{figs/fig-77280e7e.png}{Figure 6. Spectrum amplitude as a function of 
  harmonic number, for cases of Helmholtz motion in column 6 of Fig. 4, marked 
  with a green arrow. The colours of the plotted lines shade progressively from 
  red to blue to green, as the cases move upwards along this column. The 
  vertical dashed line marks the expected frequency of Schelleng's ripples, at 
  harmonic numbers close to $1/\beta$.} 

  Because a logarithmic scale has been used for the horizontal axis, the 
  pattern of harmonic amplitudes for an ideal sawtooth wave would appear as a 
  straight line with slope $-1$. All the plotted lines follow this prediction 
  up to at least the 5th harmonic, but then they start to spread out, in their 
  colour order. The curves in redder colours, corresponding to the lower values 
  of bow force, fall away first, while the blue and green curves stick with the 
  straight line for longer. This is Cremer’s corner-rounding effect in action. 
  We are seeing a systematic change in the spectrum with bow force, exactly as 
  predicted by the discussion in section 9.2. 

  But then something changes at higher frequencies. The vertical dashed line 
  shows the harmonic number corresponding to $1/\beta$, which is the expected 
  frequency of Schelleng’s ripples (recall Fig.\ 13 of section 9.2). Around 
  about that point in the spectrum plot, all the curves start to show peaks and 
  dips, associated with the influence of these ripples. 

  Figures 7 and 8 show corresponding plots for the other two columns of Fig.\ 4 
  marked by green arrows. They show similar features to Fig.\ 6, but the dashed 
  lines are progressively further to the left because $\beta$ is larger (i.e. 
  the bow is further from the bridge). The result is that the disruption 
  associated with the Schelleng ripples happens at a lower harmonic number than 
  in Fig.\ 6, in a region of the plot where the systematic spreading of curves 
  due to the corner-rounding effect has not really got started. There is still 
  a clear tendency for the curves at higher frequency to be spread out in their 
  colour sequence, but this is superimposed on the peaks and dips associated 
  with the Schelleng ripples. 

  \fig{figs/fig-59275fef.png}{Figure 7. Spectrum plot in the same format as 
  Fig. 6, for column 11 of Fig. 4.} 

  \fig{figs/fig-99e8ecc3.png}{Figure 8. Spectrum plot in the same format as 
  Fig. 6, for column 16 of Fig. 4.} 

  The combined message of these three plots is that the frequency spectrum of 
  bridge force, and hence indirectly of the sound of the played note, is 
  influenced in a systematic way by the two phenomena we have discussed. Bow 
  force has an effect through the corner-rounding mechanism, and bow position 
  has an effect through the pattern of ripples — recall also that the magnitude 
  of ripples is influenced by the bow force. These two interacting effects are 
  responsible for at least part of the “sound palette” available to a violinist 
  [2]. 

  Finally, we can use the same data to investigate the pitch flattening effect. 
  Figure 9 shows some results for the set of Helmholtz motions examined in 
  Fig.\ 7, indicated by the middle green arrow in Fig.\ 4. This plot shows the 
  frequency shift, in cents, as a function of bow force. The plot looks a 
  little “noisy”: the individual points have some uncertainty associated with 
  them because of the rather crude way that the period of the sawtooth wave has 
  been measured, but the falling trend is clear and convincing. By the highest 
  bow force, just below the Schelleng maximum force, the pitch has flattened by 
  some 10~cents relative to the value with low bow force. It looks as if the 
  string was tuned slightly flat, by about 2~cents, so those points at low bow 
  force do not fall around zero. Don’t forget that these results are for an 
  open string: a finger-stopped note would have had more corner-rounding, and 
  thus a larger potential for flattening. 

  \fig{figs/fig-a72a62a2.png}{Figure 9. Pitch flattening, expressed in cents, 
  for the Helmholtz motion cases in column 11 of Fig. 4 (the same set that gave 
  the spectra shown in Fig. 7).} 

  The conclusion from all these comparisons with measurements is that the 
  simple models we have looked at so far seem to give qualitatively correct 
  predictions for several aspects of real bowed-string behaviour. This is 
  encouraging, but we shouldn’t go overboard with excitement about this level 
  of agreement. We will see in subsequent sections that some 
  musically-important details of bowed-string sound are influenced by effects 
  that we haven’t yet included in our discussion. Furthermore, when we come to 
  look at bowed-string transients, starting in section 9.5, we will find that 
  it is remarkably challenging to go from qualitative to quantitative agreement 
  between measurements and computer models. 



  \sectionreferences{}[1] John C. Schelleng; ``The bowed string and the 
  player'', Journal of the Acoustical Society of America \textbf{53, }26–41 
  (1973). 

  [2] Paul M. Galluzzo; On the playability of stringed instruments, \tt{}PhD 
  Dissertation\rm{}, University of Cambridge (2003). 

  [3] Erwin Schoonderwaldt; ``The violinist’s sound palette: spectral centroid, 
  pitch flattening and anomalous low frequencies'', Acta Acustica united with 
  Acustica, 95, 901--914 (2009) 