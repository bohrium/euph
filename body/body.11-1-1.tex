  We will calculate the input impedance of a cylindrical pipe with rigid walls, 
  of length $L$ and cross-sectional area $S$. The pipe runs from $x=0$, where 
  the impedance is to be determined, to $x=L$, where it has an open end. 
  Actually, we will assume that $L$ includes the end correction for the open 
  end, so the physical length of the pipe is really slightly shorter than that. 

  At $x=0$ we drive the pipe with a volume flow rate $Ve^{i \omega t}$. This 
  will generate an outgoing sound wave with acoustic pressure $Ae^{i(\omega t 
  -- kx)}$ above ambient, where $A$ is a complex constant and $k=\omega/c$. 
  There will also be a returning reflected wave with pressure $Be^{i(\omega t + 
  kx)}$, where $B$ is another complex constant. Each of these travelling waves 
  has an associated particle velocity $u(x,t)$, satisfying equation (7) from 
  section 4.1.3: 

  $$-\rho_0 \dfrac{\partial u}{\partial t} = \dfrac{\partial p'}{\partial x} 
  \tag{1}$$ 

  where $\rho_0$ is the mean density of air. For a wave $e^{i(\omega t -- kx)}$ 
  this requires 

  $$u=\dfrac{1}{Z_0}e^{i(\omega t -- kx)} \tag{2}$$ 

  where $Z_0=\rho_0 c$ is the characteristic impedance of sound waves in air. 
  The corresponding result for a left-travelling wave $e^{i(\omega t + kx)}$ is 

  $$u=-\dfrac{1}{Z_0}e^{i(\omega t + kx)} . \tag{3}$$ 

  Now the total pressure $p'$ in the pipe is 

  $$p'=Ae^{i(\omega t -- kx)}+Be^{i(\omega t + kx)} , \tag{4}$$ 

  and the total volume flow rate is 

  $$v=\dfrac{AS}{Z_0}e^{i(\omega t -- kx)}-\dfrac{BS}{Z_0}e^{i(\omega t + kx)} 
  . \tag{5}$$ 

  At $x=L$ we require $p'=0$ for the open end, so 

  $$Ae^{-ikL} +Be^{ikL} =0 \tag{6}$$ 

  or 

  $$B=-Ae^{-2ikL} . \tag{7}$$ 

  At $x=0$, the input impedance we want, $Z(\omega)$, is now given by the ratio 
  of pressure to volume flow rate: 

  $$Z(\omega)=\dfrac{A+B}{(S/Z_0) 
  (A-B)}=\dfrac{Z_0}{S}\dfrac{1-e^{-2ikL}}{1+e^{-2ikL}}=\dfrac{iZ_0}{S}\tan kL 
  . \tag{8}$$ 

  So far, we have assumed that there is no dissipation in the problem. As a 
  final step, we can introduce a realistic level of damping by the simple 
  expedient of allowing $k$ in equation (8) to become complex and to depend on 
  frequency. There are three sources of energy dissipation for our sound waves 
  in the pipe. One is the result of energy loss at the open end by sound 
  radiation. The other two come from effects internal to the pipe, as the waves 
  propagate. First, the particle velocity $u$, parallel to the axis of the 
  pipe, results in some dissipation from viscous friction on the walls of the 
  pipe. 

  The second effect comes from something we haven't previously mentioned. When 
  we derived the wave equation back in section 4.1.1, we used the approximation 
  that at acoustic frequencies of interest the changes in density and pressure 
  are adiabatic (or isentropic). This means that the density changes happen 
  sufficiently rapidly that there is no change in total energy of a small 
  element of air. The universal gas law governing the thermodynamics then 
  requires that the temperature of the air also changes. When air (or any other 
  gas) is compressed, it gets hotter; and when it is expanded, it gets cooler. 
  This principle is used in refrigerators, for example. 

  So as a sound wave propagates along the pipe, there are small temperature 
  fluctuations associated with it. There is thus a tendency for heat diffusion 
  to occur from the warmer to the cooler regions, leading to energy 
  dissipation. Furthermore, the pipe wall is a reservoir of heat, and it 
  remains more or less at a constant temperature. So the air in contact with 
  the walls will have a significant temperature gradient, and again heat 
  diffusion will result. 

  The combination of these three energy dissipation mechanisms is quite 
  complicated: see sections 8.2--8.5 of Fletcher and Rossing [1] for a detailed 
  discussion. For our purposes we can use a simple approximation that these 
  authors recommend as appropriate to most musical wind instruments. Assuming 
  that the pipe is neither extremely narrow (i.e. it is not a capillary), nor 
  extremely wide, the loss from sound radiation can be neglected, and the other 
  two effects can be allowed for in a combined formula in which 

  $$k=\omega/c+i \alpha(\omega) \tag{9}$$ 

  with 

  $$ \alpha \approx 1.2 \times 10^{-5} \sqrt{\omega}/a \tag{10}$$ 

  where $a$ is the pipe radius. 

  We can deduce something else from the first part of equation (8). If our pipe 
  were semi-infinite, so that no reflected wave ever came back, we would have 
  $B=0$ and the input impedance would be simply $Z_0/S$. So in order to plot 
  the impedance, it makes a lot of sense to normalise it by this value for an 
  infinite tube. That is what was done in Fig.\ 13 of section 11.1, reproduced 
  here as Fig.\ 1. We can now see that the mean trend of this decibel plot is 
  very obviously the 0 dB line. This neatly illustrates Skudrzyk's theorem (see 
  section 5.3.2): the logarithmic mean of a driving-point impedance or 
  admittance is given by the behaviour of the corresponding infinite system. 

  \fig{figs/fig-e105f4bf.png}{\caption{Figure 1. Normalised input impedance 
  $|Z|S/Z\_0$ of a tube of length 1~m and internal diameter 20~mm, calculated 
  using the theory from this section.}} 

  \sectionreferences{}[1] Neville H Fletcher and Thomas D Rossing; “The physics 
  of musical instruments”, Springer-Verlag (Second edition 1998) 