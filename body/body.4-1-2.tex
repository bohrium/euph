  The wave equation, derived in section 4.1.1, can be written in terms of 
  spherical polar coordinates $(r,\theta,\phi)$: the pressure field $p$ must 
  satisfy 

  $$\frac{1}{r^2} \frac{\partial}{\partial r} \left( r^2 \frac{\partial 
  p}{\partial r} \right) + \frac{1}{r^2 \sin \theta} \frac{\partial}{\partial 
  \theta} \left(\sin \theta \frac{\partial p}{\partial \theta} \right) $$ 

  $$+ \frac{1}{r^2 \sin^2 \theta} \frac{\partial^2 p}{\partial \phi^2} = 
  \frac{1}{c^2} \frac{\partial^2 p}{\partial t^2}. \tag{1}$$ 

  If we look for a spherically-symmetric solution of the form $p(r,t)$, eq. (1) 
  requires 

  $$\frac{1}{r^2} \frac{\partial}{\partial r} \left( r^2 \frac{\partial 
  p}{\partial r} \right) = \frac{1}{c^2} \frac{\partial^2 p}{\partial t^2}. 
  \tag{2}$$ 

  But this equation can be rearranged into the form 

  $$ \frac{\partial^2 (rp)}{\partial r^2} = \frac{1}{c^2} \frac{\partial^2 
  (rp)}{\partial t^2}, \tag{3}$$ 

  so that the combination $rp$ satisfies the one-dimensional wave equation. Two 
  possible solutions would be 

  $$rp=Ae^{i \omega t \pm i k x} \tag{4}$$ 

  where $A$ is a constant amplitude and, as usual, the physical solutions would 
  be obtained by taking the real part of this complex expression. When the 
  $\pm$ sign is $+$, this solution describes a spherical wave travelling 
  inwards from infinity towards the origin: this solution is not of physical 
  interest. The solution we want is thus 

  $$p=\frac{A}{r} e^{i \omega t -- i k r} \tag{5}$$ 

  which describes a wave radiating outwards, with amplitude decaying 
  proportional to $1/r$. The quantity $k$ is called the wavenumber: it is 
  related to the wavelength $\lambda$ by $k=2 \pi /\lambda$, and to the 
  frequency $\omega$ via the speed of sound: $k=\omega /c$. 

  The next step is to find the particle motion associated with this pressure 
  field. The motion will be purely radial, with displacement $\xi(r) e^{i 
  \omega t}$, and it must obey eq. (16) of section 4.1.1, and so 

  $$- \omega^2 \rho_0 \xi e^{i \omega t} = -\frac{\partial p}{\partial r} = 
  A\left( \frac{1}{r^2} + \frac{ik}{r}\right) e^{i\omega t -- ikr} \tag{6}$$ 

  or 

  $$\xi e^{i \omega t} = -- \frac{A}{\rho_0 \omega^2 r} \left( \frac{1}{r} 
  +\frac{i \omega}{c} \right) e^{i\omega t -- ikr} . \tag{7}$$ 

  If this sound field is being created by small-amplitude pulsation of a sphere 
  of radius $a$, then the amplitude $A$ is determined by requiring that the 
  particle displacement matches that of the sphere at $r=a$. If the modulated 
  sphere radius takes the form $a+a' e^{i \omega t}$, then 

  $$A=-\dfrac{a' a \rho_0 \omega^2 e^{i \omega a/c}}{\left( \frac{1}{a} 
  +\frac{i \omega}{c} \right)} . \tag{8}$$ 

  The pressure on the surface of the sphere is then 

  $$p(a,t)=-\dfrac{a' \rho_0 \omega^2}{\left( \frac{1}{a} +\frac{i \omega}{c} 
  \right)} e^{i \omega t}. \tag{9}$$ 

  The term in the denominator here means that the relative phase of the 
  pressure and displacement is frequency-dependent. At high frequency the term 
  $i \omega /c$ dominates, while at low frequency or when $a$ is small, the 
  term $1/a$ dominates. The turnover point between these two limits is governed 
  by the ratio of the magnitudes of these two terms: $\omega a/c = ka$. This 
  parameter is called a Helmholtz number: it characterises the size of the 
  sound source in wavelength terms. 

  Similar behaviour arises from the bracketed term in eq. (7), and this time 
  the turnover point between limiting regimes is governed by the dimensionless 
  ratio $\omega r/c = kr$, which describes the distance of the observer in 
  wavelength terms. For large values of this parameter the observer is in the 
  far field of the sound source, and the behaviour is dominated by the term $i 
  \omega /c$. For small values, the observer is in the near field of the 
  source, where the behaviour is dominated by the term $1/r$. However, this 
  near-field region only exists if the Helmholtz number is also small, 
  otherwise it is not possible for $r$ to get small enough. So the Helmholtz 
  number governs whether the sound field involves a near field, or whether the 
  far field essentially starts immediately from the surface of the sphere. If 
  there is a near field, the parameter $kr$ determines how far out this regime 
  extends. 