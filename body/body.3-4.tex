

  There are no mainstream tuned percussion instruments based on flat plates, 
  although when we come to stringed instruments we will find many examples of 
  more-or-less flat soundboards. But there are certainly percussion instruments 
  based on curved shells. In this section we will look at a kind of 
  ``percussion instrument'' which develops the theme of vibration engineering 
  to align overtones into near-harmonic patterns: church bells. Bells must be 
  among the oldest tuned percussion instruments, both in Europe and (with an 
  entirely different tradition) in China. European bells are still cast from 
  bronze in a very traditional manner: indeed, a bell foundry is a place that 
  feels like a glimpse of an earlier century. 

  \fig{figs/fig-e144a5d6.png}{} 

  \fig{figs/fig-4b54661d.png}{} 

  \fig{figs/fig-9b8964c1.png}{} 

  \fig{figs/fig-34fb4460.png}{} 

  Molten bronze is poured into moulds to make each bell. After casting, the 
  bells are tuned using a vertical lathe to adjust the thickness profile of the 
  walls. The principle is similar to adjusting the undercutting of a marimba 
  bar, but the technology is very different with such large objects. Once the 
  bell is tuned, it will be installed in its church tower, with the associated 
  equipment for ringing it. 

  \fig{figs/fig-9a9b536e.png}{} 

  \fig{figs/fig-39be45ac.png}{} 

  British tower bells are rung by rope from a chamber further down the tower 
  from the bells. They are traditionally rung ``full circle'': the bells rest 
  facing more or less vertically upwards, as seen in Fig.\ 4, then on each 
  stroke the bell is made to swing a complete circle to come to rest in a 
  similar position but with the clapper now resting on the other side, ready 
  for the return stroke. Actually, during normal ``change ringing'' the bell 
  does not quite come to the top of the travel each time: the ringer has to 
  develop the skill of stopping it in the right place to swing back at the 
  right moment for its next place in the complicated mathematical sequence that 
  constitutes a change ringing ``method''. Sound 1 gives a brief snatch of 
  change ringing, for illustration. 

  \fig{figs/fig-b8071aa0.png}{} 

  \fig{figs/fig-e96eac7c.png}{} 

\audio{}

  There is some interesting science involved in this type of ringing, which 
  lies behind how the bells should be hung and adjusted to give good sound and 
  good handling from the ringers' perspective. We will come back to this later 
  (see section ?), but for now we concentrate on the natural frequencies of the 
  bells and their tuning. There are no simple mathematical models that can cope 
  with the complicated geometry of a traditional bell, so we begin with 
  empirical measured frequencies. We will draw heavily on an extensive study of 
  church bell tuning by Hibbert, described on \tt{}his web site\rm{}. 

  It turns out that the tuning of European church bells has a rather 
  complicated history. Bell-makers settled on the classic form of bell, no 
  doubt after much empirical experimentation, many centuries ago. As we shall 
  see in a moment, this form more or less guarantees a sufficient level of 
  near-harmonic relations between certain of the strongly-excited overtones 
  that a listener can identify a pitch when the bell is struck. But some 
  bell-makers use a more sophisticated pattern of tuning, resulting in a 
  stronger set of harmonic relations and a correspondingly stronger pitch 
  sensation. This approach, called ``true-harmonic tuning'' in the English 
  tradition, seems to have been discovered, lost and rediscovered over a long 
  period of time, perhaps more than once. No doubt the underlying reason was 
  that the details of this tuning method were trade secrets, carefully guarded 
  from competitors. 

  Combined with the fact that church bells can have a very long lifetime, this 
  means that there are bells to be found around Europe spanning the centuries, 
  and embodying a variety of styles and accuracy of tuning. Hibbert has 
  measured the natural frequencies of a very large number of them. The key 
  behaviour of a representative pair of bells is shown in Fig.\ 5. This is a 
  synthesised example, based on measured frequencies of two bells: one 
  true-harmonic tuned, and the other not. The frequencies have been scaled so 
  that both give the same ``strike note'', to be explained and illustrated 
  shortly. 

  \fig{figs/fig-43877cfd.png}{\caption{Figure 5. Frequency spectra of two 
  synthesised bell-like sounds: the red curve uses frequency ratios and levels 
  measured by Hibbert, the black curve uses the same levels but measured 
  frequency ratios for a bell with ``true-harmonic'' tuning, given by Fletcher 
  and Rossing [1]. In both cases the actual frequencies have been scaled to 
  give the same ``strike note'', $G\_4$ (392 Hz). All modes have the same 
  Q-factor, 1000. The numbers above certain peaks denote the number of 
  wavelengths of bell vibration around the circumference.}} 

  The red curve in Fig.\ 5 shows the frequency pattern and amplitude levels of 
  a bell measured by Hibbert (from a church in Ipswich). The black curve shows 
  the pattern of a true-harmonic bell. This is a synthesised example, using the 
  strongest overtone frequencies from the measurements but disregarding various 
  complicating factors. It is immediately clear that the two bells have very 
  similar behaviour at higher frequencies, but that the first few modes fall at 
  quite different frequencies. 

  The peaks that match in the two bells are annotated with numbers in the plot: 
  these give the number of full wavelengths of deformation of the corresponding 
  mode shape around the circumference of the bell. For a bell with a perfectly 
  circular cross-section, an argument based on this symmetry behaviour shows 
  that the circumferential variation must take the form $\cos n \theta$ or 
  $\sin n \theta$ for each mode, just as we saw for the wineglass example in 
  section 3.2. (Remember $\theta$ is the angle round the bell, pronounced 
  ``theta''.) The annotations give the values of $n$. 

  The geometry of the bell is more complicated than the wineglass, but 
  Rayleigh's argument that inextensional motion tends to govern the 
  low-frequency modes still applies, at least approximately. The labelled modes 
  involve motion that is approximately inextensional, and largely confined near 
  the free rim of the bell. This has two important consequences: it is the 
  position where the clapper strikes, so they are strongly excited during 
  normal ringing; and their relative frequencies fall in a rather regular 
  pattern that only depends on the geometry of the bell close to the rim. 

  This regular pattern is responsible for the ``strike note'' of the bell, 
  through an interesting psychoacoustical phenomenon often called the ``missing 
  fundamental''. For both bells, the peaks labelled 4, 5 and 6 have frequencies 
  in the approximate ratio 2:3:4. The higher labelled peaks continue this 
  regular pattern: the two bells differ slightly, but not a great deal. 

  But at lower frequencies the two bells are very different. The true-harmonic 
  tuned bell has been adjusted, using a lathe like the one seen in Fig.\ 4, to 
  bring the first 4 mode frequencies into deliberate harmonic relations to the 
  sequence just described: relative to the 2:3:4 frequencies, these 4 modes 
  have frequencies 0.5, 1, 1.2 and 1.5. The first two are tuned to the 
  ``missing'' fundamental, and an octave below that. The other two, with ratios 
  6/5 and 3/2, are less obvious at first sight, but they make up the notes of a 
  minor chord and so are musically ``harmonious'' with the strike note and its 
  harmonic series. The corresponding ratios for the non-tuned bell are 0.56, 
  0.89, 1.17 and 1.56: sufficiently different that any sense of harmoniousness 
  is absent. 

  The sound of one note of the non-tuned ``bell'' with the red curve is given 
  in Sound 2. The same thing for the tuned bell is given in Sound 3. I hope you 
  agree that Sound 3 gives a more clear and definite sense of pitch, while 
  Sound 2 is rather ambiguous. Indeed, hearing single notes like this does not 
  give a very convincing impression that they have the same strike pitch, as I 
  have claimed. But listen to the same pair of ``bells'' in Sounds 4 and 5, 
  where scaled versions with different fundamental frequencies have been used 
  to play a short passage at roughly the speed typical of normal 
  change-ringing, as in Sound 1. Now it is perhaps more persuasive that the 
  perceived pitch of the two tunings is essentially the same. The trouble with 
  the long-ringing single notes in Sounds 2 and 3 is that you tend to 
  concentrate on the slow-decaying lower frequency components, which are the 
  ones most different between the two ``bells''. 

\audio{}

\audio{}

\audio{}

\audio{}

  I have kept putting ``bells'' in quotes here, because these synthesised 
  sounds aren't in fact terribly convincing as church bell sounds. Perhaps they 
  sound a little like tubular bells, used in orchestral performances. But there 
  is something about the sound heard in Sound 1 which is surely not reproduced 
  well here. That is an issue that will return again and again in this story: 
  synthesised sounds based on what appears to be the main ingredients of the 
  underlying physics often do not sound very convincing. This should not be 
  interpreted as a failure of science, but as a challenge to do better. 
  Something important for perception has been missed, and tracking that down 
  requires persistence. But a convincing response should not simply involve 
  arbitrary fudging of the sounds to be more ``realistic'': we want to know 
  what the missing physics really is. We will not forget this challenge, but we 
  will not pursue it any further for church bells for the moment. It is too 
  early in this story to get tempted into messy details of any single example. 



  \sectionreferences{}[1] Neville H Fletcher and Thomas D Rossing; ``The 
  physics of musical instruments'', Springer-Verlag (Second edition 1998) 