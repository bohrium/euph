  Duffing's equation describes a single-degree-of-freedom oscillator with a 
  nonlinear spring. For weak nonlinearity it is reasonable to expect that the 
  spring force-displacement relation could be expanded in a Taylor series, and 
  that this might then be approximated by discarding higher powers of 
  displacement. For a spring force that behaves in a symmetrical manner under 
  tension or compression, we expect only odd powers of displacement in the 
  Taylor expansion, so the simplest approximation is to include just a cubic 
  term alongside the linear variation. 

  We can write the resulting equation in this form: 

  $$\ddot{x}+p \eta \dot{x} +p^2 x +\mu x^3 = F(t) \tag{1}$$ 

  where $p$ is the angular frequency of the linear oscillator resonance, $\eta$ 
  is the linear damping factor, $\mu$ is the nonlinear coefficient, and $F(t)$ 
  is the externally applied excitation force. If $\mu > 0$ it is a hardening 
  spring, if $\mu < 0$ it is a softening spring. We will consider the case of 
  harmonic excitation with amplitude $a$ and frequency $\omega$: 

  $$F(t)=a \cos \omega t . \tag{2}$$ 

  We will first look at the undamped case, with $\eta =0$. The cubic spring law 
  means that the nonlinearity generates only odd harmonics. So a simple 2-term 
  harmonic expression to try would be 

  $$x \approx \alpha \cos \omega t + \beta \cos 3\omega t . \tag{3}$$ 

  Substituting in eq. (1) gives 

  $$-\omega^2 \left(\alpha \cos \omega t + 9\beta \cos 3\omega t\right) + p^2 
  \left(\alpha \cos \omega t + \beta \cos 3\omega t\right)$$ 

  $$+\mu \left(\alpha \cos \omega t + \beta \cos 3\omega t\right)^3 = a \cos 
  \omega t. \tag{4}$$ 

  We now expand out the cubed term, and make use of three standard 
  trigonometrical formulae: 

  $$\cos \theta \cos \phi = \frac{1}{2} \left[\cos (\theta -- \phi) + \cos 
  (\theta + \phi) \right], \tag{5}$$ 

  $$\cos^2 \theta = \dfrac{1}{2}\left(1 + \cos 2 \theta \right) \tag{6}$$ 

  and 

  $$\cos^3 \theta =\frac{3}{4}\cos \theta + \frac{1}{4} \cos 3 \theta . 
  \tag{7}$$ 

  The result is a linear combination of terms in $\cos n\omega t$ for various 
  odd-number values of $n$. We equate the coefficients of all the terms in 
  $\cos \omega t$ to give one relation between the coefficients $\alpha$ and 
  $\beta$, and we do the same with the terms in $\cos 3 \omega t$ to obtain 
  another. We simply ignore the terms with $n$ values greater than 3; but if we 
  wanted information about higher values we could extend the approach to 
  include more terms in the initial guess (3). 

  The two equations that result are: 

  $$-\alpha \omega^2 + \alpha p^2 + \mu \left[\frac{3}{4}\alpha^3 + 
  \frac{3}{4}\alpha^2 \beta +\frac{3}{2}\alpha \beta^2 \right] = a \tag{8}$$ 

  and 

  $$-9\beta \omega^2+ \beta p^2 + \mu \left[\frac{1}{4}\alpha^3 + 
  \frac{3}{2}\alpha^2 \beta +\frac{3}{4} \beta^3 \right] = 0 . \tag{9}$$ 

  We can make a further approximation: we expect the third harmonic generated 
  by the nonlinearity to be small in magnitude, so that $ |\beta | \ll |\alpha 
  |$. The terms in square brackets, which are also multiplied by the small 
  parameter $\mu$, can thus be simplified to give 

  $$\alpha (p^2 -- \omega^2) +\dfrac{3}{4} \mu \alpha^3 \approx a \tag{10}$$ 

  and 

  $$\beta (p^2-9 \omega^2) + \dfrac{1}{4} \mu \alpha^3 \approx 0 . \tag{11}$$ 

  Equation (10) is a cubic equation for $\alpha$ for given excitation level 
  $a$. Once $\alpha$ is found, eq. (11) gives the associated value of $\beta$. 

  In principle there is a formula for the roots of a cubic equation, but for 
  the purposes of this discussion it is simpler to compute a numerical example. 
  Figure 1 shows a plot for a hardening spring using the parameter values 
  $a=1$, $p=1$, $\eta =0$ and $\mu = 10^{-7}$. Damping is zero here so that 
  both curves, linear and nonlinear, head off to infinity. The plot shows the 
  fundamental amplitude $\alpha$ only: the third harmonic governed by $\beta$ 
  affects the frequency spectrum of the motion, but makes little difference to 
  the amplitude. Figure 3 of section 8.2 shows a matching case except with 
  $\eta = 0.002$, while Fig.\ 5 of that section shows the corresponding case 
  for a softening spring, with $\mu = -10^{-7}$. 

  \fig{figs/fig-edeb97a7.png}{Figure 1. Amplitude of response to sinusoidal 
  driving of a linear oscillator (black) and a Duffing oscillator with a 
  hardening spring (red). Both cases are undamped.} 

  The final step is to add the damping term in. For that, we need to replace 
  eq. (3) by something involving both $\sin \omega t$ and $\cos \omega t$, 
  because damping introduces phase shifts. Having seen how the previous 
  calculation worked out, we can save effort by looking only at the fundamental 
  component of the Fourier series: so this time we try 

  $$x=\alpha \cos \omega t + \gamma \sin \omega t . \tag{12}$$ 

  We substitute into eq. (1), and make similar use of trigonometrical 
  identities. Ignoring all higher harmonic components, we can obtain two 
  equations for $\alpha$ and $\gamma$ by equating coefficients of all terms in 
  $\cos \omega t$, and all terms in $\sin \omega t$. The result is 

  $$(p^2 -- \omega^2) \alpha +p \eta \gamma \omega +\frac{3}{4} \mu \alpha 
  (\alpha^2 + \gamma^2) = a \tag{13}$$ 

  and 

  $$(p^2 -- \omega^2) \gamma -p \eta \alpha \omega +\frac{3}{4} \mu \gamma 
  (\alpha^2 + \gamma^2) = 0 . \tag{14}$$ 

  With a little manipulation, these two equations can be combined to yield a 
  single equation for the amplitude $z$, defined by $z^2=\alpha^2 + \gamma^2$: 

  $$\left[ \left(\omega^2 -- p^2 -- \frac{3}{4} \mu z^2 \right)^2 + (p \eta 
  \omega)^2 \right] z^2 = a^2. \tag{15}$$ 

  This cubic equation for $z^2$ was solved numerically to generate the damped 
  response shown in Figs.\ 3, 4 and 5 of section 8.2. 

  Finally, to add the blue circles indicating the ``backbone curve'' in Figs.\ 
  3 and 5 of section 8.2 we need to find the condition that defines the 
  ``nose'' of the leaning curves. We take our cue from the linear harmonic 
  oscillator: the peak response occurs at the point where the motion is exactly 
  $90^\circ$ out of phase with the excitation. In other words, we want the 
  point where $\alpha = 0$. Substituting this into eqs. (13) and (14) and doing 
  minor manipulations, we obtain the equation 

  $$\gamma^4 +\dfrac{4p^2}{3 \mu} \gamma^2 -\dfrac{4a^2}{3 \mu p^2 \eta^2}=0 
  \tag{16}$$ 

  which is a quadratic equation for $\gamma^2$. Once $\gamma$ is known, the 
  corresponding frequency satisfies 

  $$p \eta \gamma \omega = a. \tag{17}$$ 

  These two equations, evaluated with different values of $a$ representing a 
  slow decay of the oscillation, give the blue circles shown in Figs.\ 3 and 5 
  of section 8.2. 