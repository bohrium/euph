  In order to have a model for banjo synthesis with full control over the model 
  parameters, a very simple model has been developed based on results for a 
  rectangular membrane. It might seem odd to use the rectangular shape, when we 
  have relatively easy expressions for the mode shapes and natural frequencies 
  of a more realistic circular membrane (see section 3.6.1). However, it turns 
  out that a model giving realistic sound needs to pay careful attention to 
  damping, including the damping of the membrane due to sound radiation. There 
  are no simple expressions for this radiation damping of a circular membrane, 
  but Leppington et al. [1] have provided simple formulae for the rectangular 
  case. These approximate expressions were calculated by an asymptotic method, 
  but they give sufficiently accurate results for this application. 

  The datum membrane is chosen to have the same area, tension and mass per unit 
  area as the head of the Deering banjo. Specifically, it has tension 5.33 
  kN/m, mass per unit area 308 g/m$^2$, and dimensions $249 \times 226$ mm in 
  the directions normal and parallel to the strings, respectively. An allowance 
  is made for the added mass resulting from the air in contact with the 
  membrane, as we discussed briefly in section 3.6 in the context of the 
  kettledrum. It is done using a result for an infinite membrane (see reference 
  [2] for details). 

  Radiation damping of the membrane is modelled using the formulae of 
  Leppington et al. [1], as explained above. These formulae are calculated 
  using the Rayleigh integral, which we met in section 4.3.2. They assume an 
  infinite baffle around the membrane, which is of course not physically 
  accurate for a banjo. However, comparisons of admittance computed from the 
  model with the measured admittance of the real banjo gives reasonably good 
  agreement. A constant ``background structural damping'' with loss factor $4 
  \times 10^{-3}$ is added to suppress unrealistically high modal Q-factors. A 
  final detail concerning damping of this datum case will be discussed in 
  section 5.5.2. 

  The ``bridge'' is modelled as a concentrated mass of 1.5 g, with a footprint 
  $10 \times 10$ mm centred at position (105,80) in mm relative to a corner of 
  the rectangle. An additional stiffness 20 kN/m is applied to the bridge to 
  represent the combined effect of the axial stiffness of the strings, a 
  similar stiffness contributed by in-plane stretching of the head membrane, 
  and also an effect associated with the string tension. All these stiffness 
  components are sensitive to the break angle over the bridge. Finally, a 
  ``dashpot'' (mechanical resistance) with impedance 1 Ns/m is added, to 
  increase the damping of the strongest resonances to a level comparable with 
  the measurements. Reference [2] gives a detailed discussion of the effect of 
  these successive additions, and of the choice of parameter values. 

  \sectionreferences{}[1] F. G. Leppington, E. G. Broadbent and K. H. Heron, 
  “The acoustic radiation efficiency of rectangular panels”. Proceedings of the 
  Royal Society of London Series A-Mathematical Physical and Engineering 
  Sciences, \textbf{382}, 245–271 (1982). 

  [2] Jim Woodhouse, David Politzer and Hossein Mansour. ``Acoustics of the 
  banjo: theoretical and numerical modelling'', Acta Acustica\textbf{5}, 16 
  (2021). The article is available here: 
  \tt{}\rm{}\tt{}https://doi.org/10.1051/aacus/2021008\rm{} 