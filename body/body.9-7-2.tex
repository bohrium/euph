  A simple model of the bouncing frequency of a bow was presented by Askenfelt 
  and Guettler [1]. We used this model to plot Fig.\ 4 of section 9.7, to show 
  how the bouncing frequency varies depending on the bow-string contact 
  position. The model can easily be derived: the arrangement is shown 
  schematically in Fig.\ 1. The bow stick with its tip and frog is treated as a 
  rigid frame, holding the bowhair which has tension $T$. This frame is pivoted 
  at one end, indicated by the blue circle in Fig.\ 1, and it is assumed to 
  have moment of inertia $I$ about this point. 

  \fig{figs/fig-b2fed42a.png}{\caption{Figure 1. Schematic sketch of the 
  bouncing bow model}} 

  At a given moment, the frame has rotated by a small angle $\theta$ about the 
  pivot. At a distance $x$ along the bowhair from the frog, the hair is in 
  contact with a string. The string is assumed to have a much higher tension 
  than the bowhair, so that for a first approximation it can be treated as 
  rigid. This rigid string has pressed into the bowhair by a distance $y 
  \approx x \theta$. This generates a restoring force 

  $$F \approx T \frac{y}{x} +T \frac{y}{L-x} \approx T \theta + T \frac{x}{L-x} 
  \theta = T \theta \frac{L}{L-x} . \tag{1}$$ 

  Now taking moments about the pivot we obtain the governing equation: 

  $$Fx \approx T \theta \frac{Lx}{L-x} =-I \ddot{\theta} . \tag{2}$$ 

  This is the simple harmonic equation, so we deduce immediately that the 
  frequency $\omega$ of bouncing motion satisfies 

  $$\omega^2=\frac{T}{I}~\frac{Lx}{L-x} . \tag{3}$$ 

  Of course, the string is not really rigid, so that in practice the frequency 
  will be a little lower than this estimate. 

  This very simple model of bow bouncing has been extended by Gough [2], using 
  a combination of finite-element computation and measurements on real bows. 
  Needless to say, the full pattern of behaviour is revealed to be more 
  complicated. However, for the purposes of qualitative description it is 
  reassuring to find that the plot shown in Fig.\ 4 of section 9.7 is changed 
  only very slightly for bow-string contact positions up to the point indicated 
  as the ``typical spiccato position''. 

  But there is a more important effect for actual ricochet or spiccato playing. 
  This model has assumed that the bow remains in contact with the string 
  throughout the motion, but the whole point of these bowing techniques is that 
  the bow bounces off the string for a short time in each cycle, then makes 
  contact again to start the next note. This ``flying time'' will add to the 
  period of the bouncing as we have just calculated it, by an amount depending 
  on details of the player's gesture. 

  \sectionreferences{}[1] Anders Askenfelt and Knut Guettler, “The bouncing 
  bow: an experimental study”, Catgut Acoustical Society Journal \textbf{3}, 6, 
  3—8 (1998) 

  [2] Colin E. Gough, ``Violin bow vibrations'', Journal of the Acoustical 
  Society of America \textbf{131}, 4152--4163 (2012). 