  In the main text, sound examples are given for variations of parameters of 
  immediate interest to players and makers of banjos. However, the synthesis 
  models also require some assumptions about internal ``housekeeping'' 
  variables. For completeness, these are discussed now, and some sound examples 
  are provided to illustrate the effect of these variables. There are two main 
  topics here: representing ``sound'', and fine-tuning the treatment of 
  damping. 

  The synthesis model does not attempt to calculate radiated sound, it simply 
  calculates the motion of the body at the bridge following a plucked note. 
  This calculation includes the physics relevant to all the parametric 
  variations considered here, but in order to allow perceptual judgements, 
  sound files must be created that give a plausible approximation to the sound 
  that would reach a listener or a microphone. 

  The actual radiation of sound by the banjo, or indeed any other vibrating 
  structure, is rather complicated. It will vary significantly with frequency, 
  and also with position of the observing point. There is no simple, universal 
  model for sound radiation, analogous to the general formula for structural 
  response as a linear combination of modal contributions (as in section 
  2.2.5). It is possible in principle to compute radiated sound pressure: some 
  results from a detailed Finite Element and Boundary Element (FE/BE) model 
  were presented in refs. [1,2]. An example of synthesised sound based on this 
  model will be presented shortly, but the method is too computationally 
  intensive to be used for wide-ranging parametric explorations. 

  Measurements in which bridge response and radiated sound were simultaneously 
  measured suggest that the trend with frequency of the sound broadly follows 
  that of the bridge velocity. However, when synthesised sounds were made in 
  which bridge velocity was the output variable, several different listeners 
  concurred in thinking that the sound was not bright enough. A very simple 
  filtering procedure was thus used to boost the high frequencies: the 
  predicted body velocity signal is scaled by $(i \omega)^\beta$ where the 
  power $\beta$ can be chosen for best effect. The value for the studies here 
  is $\beta=0.4$. The effect of changing the value of $\beta$ can be heard in 
  the sound examples below. Note that in this particular set of files, the 
  loudness level is not preserved between examples. Changing the value of 
  $\beta$ makes a big difference to peak levels, and so each sound file has 
  been auto-scaled. 

  Next, we give some examples of sounds based on the FE/BE model. Sound X.7 
  uses the computed bridge admittance, plotted as the black line in Fig.\ 1, 
  filtered with the standard power-law filter. Sound X.8 uses the same 
  admittance filtered using the detailed FE/BE prediction. The red line in 
  Fig.\ 1 shows the sound pressure at a typical point, and the dashed blue line 
  shows the ratio, which gives the transfer function from bridge velocity to 
  field pressure. This is the transfer function used to filter the sound of 
  Sound X.8. Notice the broadly horizontal trend of this dashed curve, 
  illustrating the comment above that sound very roughly seems to follow 
  velocity. Sound X.9 is similar to Sound X.8, but it uses a magnitude-only 
  version of the computed frequency response for the filtering process. 

  There is no doubt that the FE/BE model captures many aspects of the banjo 
  response and sound radiation far more accurately than the super-simple square 
  banjo model. However, the results illustrate a theme we have encountered 
  before and will encounter many more times: it is never obvious which features 
  of a model have the most perceptual significance. In the present state of 
  this FE/BE model, the resulting sounds do not achieve a high degree of 
  realism. The reason is perhaps associated with transient effects arising from 
  narrow spectral features in the computed transfer functions. This links to 
  the topic we will turn to next: the perceptual quality of synthesised sounds 
  can be very sensitive to details of the damping model, and these are often 
  not amenable to theoretical prediction. 

  Subsection 5.5(I) of the main text gives a general discussion of some 
  problems associated with the treatment of damping within the synthesis 
  models, and particularly in the square banjo model. In order to produce a 
  datum case of that model which compared well in a visual sense with the 
  bridge admittance of the real banjo, and which also produced a sound quality 
  which was not marred by distracting ``zinginess'' artefacts, damping was 
  incorporated in two different ways. Both involve a choice of parameter 
  values, and the effect of those choices is illustrated in the remaining sound 
  examples here. 

  The first concerns the modelling of the effect of the bridge. Once a good 
  value of the mass and the added stiffness had been chosen in order to get the 
  low-frequency formant in about the right place, the individual peaks within 
  that formant were still rather too pronounced for realism. So additional 
  damping was incorporated in the form of a mechanical resistance (or 
  ``dashpot'') at the bridge. A range of values of the dashpot coefficient are 
  illustrated in the next set of sound examples. Sound X.10 shows the effect 
  with no dashpot; Sound X.12 illustrates the choice of dashpot used in the 
  datum model; Sound X.14 shows the effect of a much stronger dashpot. 

  Even with this dashpot included, synthesis with the square banjo model 
  produced distracting ``zinginess'' which has been mentioned several times. 
  The purpose of using the square banjo model for parametric investigations is 
  to give a useful impression of the perceived effect on sound of physical 
  changes, for example to membrane tension or bridge mass. For that purpose, it 
  is essential that the datum model gives an acceptably banjo-like sound, so 
  this problem of zinginess arising from low loss at high frequency needs to be 
  addressed. The problem arises largely because of the concentrated mass used 
  to represent the bridge in the model. At high frequency, the admittance is 
  dominated by the effect of this mass, and so the real part becomes very 
  small. This is not the case in the measured admittance, largely because of 
  the effect of the bridge hill discussed in section 5.3. This hill feature was 
  well captured by the FE/BE model, using a detailed model of the bridge, but 
  the simple concentrated mass used in the square banjo model cannot reproduce 
  it. 

  This issue has been addressed with an unashamed fudge. This takes its 
  inspiration from the way measured admittance is processed. The laser 
  vibrometer and data-logging system result in a small time delay between the 
  two data channels, the hammer signal and the velocity response. This needs to 
  be compensated in the computer in order to avoid a similar problem of the 
  real part of the admittance becoming small or even negative at higher 
  frequencies, leading to ``zinginess''. 

  No such physical delay is present in the theoretical model, but it was found 
  that if it is processed in the same way, shifting the phase to represent a 
  very short delay of 20 $\mu$s, the mean value of the real part of the 
  admittance is changed to have a magnitude comparable with the measured value. 
  Being purely a phase shift, this change has no effect on the magnitude of the 
  admittance. The result of this, admittedly non-physical, phase compensation 
  is illustrated by the next set of sound examples. Sound X.16 in this set is 
  based on the original prediction of the model, and illustrates the original 
  problem. Sound X.15 makes things worse by using a 20 $\mu$s compensation with 
  the wrong sign: the zinging sound is very much in evidence. Sound X.17 is the 
  case used as the datum: the zinging is largely suppressed. Sounds X.18 and 
  X.19 illustrate the effect of increasing the delay further. 

  \sectionreferences{}[1] Jim Woodhouse, David Politzer and Hossein Mansour. 
  “Acoustics of the banjo: measurements and sound synthesis”, Acta Acustica 5, 
  15 (2021). The article is available here: 
  \tt{}https://doi.org/10.1051/aacus/2021009\rm{} 

  [2] Jim Woodhouse, David Politzer and Hossein Mansour. “Acoustics of the 
  banjo: theoretical and numerical modelling”, Acta Acustica 5, 16 (2021). The 
  article is available here: \tt{}https://doi.org/10.1051/aacus/2021008\rm{} 