  The displacement $w(x,t)$ of an ideal string with tension $T$ and mass per 
  unit length $m$ satisfies the differential equation 

  \begin{equation*}\frac{\partial^2w}{\partial x^2}= \frac{1}{c^2} 
  \frac{\partial^2w}{\partial t^2}\tag{1}\end{equation*} 

  \noindent{}where $c=\sqrt{T/m}$ is the wave speed, from eq. (5) of section 
  3.1.1. We can make a change of variables: define 

  \begin{equation*}p=x+ct,\mathrm{~~~~}q=x-ct. \tag{2}\end{equation*} 

  Now use the chain rule repeatedly: 

  \begin{equation*}\frac{\partial w}{\partial x}=\frac{\partial w}{\partial 
  p}\frac{\partial p}{\partial x}+\frac{\partial w}{\partial q}\frac{\partial 
  q}{\partial x}\end{equation*} 

  \begin{equation*}=\frac{\partial w}{\partial p}+\frac{\partial w}{\partial 
  q}\end{equation*} 

  \noindent{}so 

  \begin{equation*}\frac{\partial^2 w}{\partial x^2}=\frac{\partial^2 
  w}{\partial p^2}+2\frac{\partial^2 w}{\partial p \partial q}+\frac{\partial^2 
  w}{\partial q^2} . \tag{3}\end{equation*} 

  Similarly 

  \begin{equation*}\frac{\partial w}{\partial t}=c \frac{\partial w}{\partial 
  p} -c \frac{\partial w}{\partial q}\end{equation*} 

  \noindent{}so that 

  \begin{equation*}\frac{\partial^2 w}{\partial t^2}= c^2 \left[ 
  \frac{\partial^2 w}{\partial p^2} -- \frac{\partial^2 w}{\partial p \partial 
  q} \right]-c^2 \left[ \frac{\partial^2 w}{\partial p \partial q} 
  -\frac{\partial^2 w}{\partial q^2} \right] . \tag{4}\end{equation*} 

  Substituting eqs. (3) and (4) into eq. (1) thus gives 

  \begin{equation*}\frac{\partial^2 w}{\partial p^2}+2\frac{\partial^2 
  w}{\partial p \partial q}+\frac{\partial^2 w}{\partial q^2} = 
  \frac{\partial^2 w}{\partial p^2}-2\frac{\partial^2 w}{\partial p \partial 
  q}+\frac{\partial^2 w}{\partial q^2}\end{equation*} 

  \noindent{}which reduces to 

  \begin{equation*}4\frac{\partial^2 w}{\partial p \partial 
  q}=0.\tag{5}\end{equation*} 

  This equation can be integrated immediately. But we have to remember that 
  these are partial derivatives, not ordinary derivatives. So instead of 
  picking up arbitrary constants of integration, we pick up arbitrary 
  functions. For example, from the equation 

  \begin{equation*}\frac{\partial w}{\partial p}=0\end{equation*} 

  \noindent{}we would deduce that $w$ could be any arbitrary function of $q$. 
  So the general solution of eq. (5) is 

  \begin{equation*}w(p,q) = f(p) +g(q)\end{equation*} 

  \noindent{}where $f$ and $g$ are arbitrary functions. The final result is 
  that the free motion of the string must be of the form 

  \begin{equation*}w(x,t) = f(x+ct) + g(x-ct),\tag{6}\end{equation*} 

  \noindent{}in other words the sum of a wave of any shape travelling leftwards 
  at speed $c$ and another wave of any shape travelling rightwards at speed 
  $c$. 