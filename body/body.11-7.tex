

  This is a good point in our story to make a small digression to investigate 
  the question of how a wind instrument actually makes sound. All the 
  simulation models we have developed led to predictions of the time-varying 
  sound pressure inside the instrument, in the mouthpiece or in the acoustical 
  cavity in the case of our free-reed models. But that is not the sound you 
  hear when the instrument is played: for that, we need to think about how the 
  instrument radiates sound into the surrounding air. 

  First, we need to recall and extend some things we learned back in Chapter 4. 
  As will become clear shortly, we are mainly interested in sound sources that 
  are small compared to the wavelength of sound, and that allows us to use an 
  important simplification. There is a rather general theory for sound 
  radiation from this kind of “compact sound source”, which gives a hierarchy 
  of three different patterns of sound radiation. The next link gives some 
  details. 

  Whenever there is a net change in volume, for example because of air flow in 
  and out of an open tone-hole, it behaves like a monopole sound source. An 
  idealised example of a monopole source, which we looked at in Chapter 4, is a 
  pulsating sphere. A monopole is the most efficient type of sound source, and 
  if it is present it will normally dominate over the other two kinds of source 
  that we will come to in a moment. A monopole source radiates sound equally in 
  all directions. The strength of the source is proportional to the rate of 
  change of the volume flow rate, which means that sound radiation tends to be 
  stronger at higher frequencies. 

  If there is no change of volume, then the most likely thing to occur is a 
  dipole sound source. The example we saw back in section 4.3 was the result of 
  a small rigid sphere oscillating backwards and forwards. Any other small 
  vibrating rigid body will have a similar effect. Even if there is no physical 
  vibration driving the sound, a dipole sound source can arise if an 
  oscillating force is applied to the air, for example the drag force 
  associated with oscillating flow round a sharp edge. A dipole sound source 
  has a directional pattern, in a figure-of-8 shape. Figure 1 shows two 
  examples of the pattern: it is a copy of Fig.\ 1 from section 4.3. The 
  orientation of the figure-of-8 is determined by the direction of vibration of 
  the rigid body, or by the direction of the oscillating force applied to the 
  air. 

  \fig{figs/fig-49aff88b.png}{\caption{Figure 1. The directional pattern of 
  sound radiation from a dipole source, shown in two possible orientations. 
  This plot is reproduced from Fig. 1 of section 4.3.}} 

  Finally, in the absence of volume change or rigid boundaries that might apply 
  a force, it is still possible for sound to be generated by the air-flow 
  itself. The most familiar example is the sound generated by jet engines. The 
  exhaust flow from the jet is turbulent, in other words a complicated and 
  quasi-random flow pattern. Bernoulli’s principle tells us that the variations 
  of flow speed within this turbulence must be associated with variations in 
  pressure. However, most of the pressure changes are confined within the area 
  of turbulence. Only a very small fraction of the energy of the flow is 
  converted into sound that radiates away from the jet so that it can be heard 
  by a distant listener. Small it may be, but of course the noise from jet 
  engines can be a major source of nuisance and noise pollution. 

  Because of the importance of aircraft noise, considerable research effort has 
  been devoted to understanding and ameliorating the problem. We do not need to 
  go into the details here, but some of the understanding resulting from that 
  research can be applied to certain aspects of the sound of musical 
  instruments. The cornerstone of the theory of aerodynamic sound generation is 
  a result known as “Lighthill’s acoustic analogy”. The previous link explains 
  a little of the background. The important upshot for our purposes is that a 
  sound source like this has the character of a quadrupole source. We gave a 
  simple example of a quadrupole source back in section 4.3: the sound radiated 
  by a vibrating tuning fork. As was explained there, a quadrupole source will 
  have a directional pattern, typically involving four lobes of strong sound 
  radiation which can fall in a variety of detailed configurations. 

  To understand the importance of this monopole/dipole/quadrupole hierarchy, we 
  need to recall another thing we learned in section 4.3. For a monopole source 
  radiating sound at a single frequency, the sound pressure amplitude falls 
  inversely with distance. But for dipole and quadrupole sources, the pattern 
  is more complicated. They have a “near-field” region within which the sound 
  pressure decays more rapidly, and only when you get into the “far-field” 
  region does this decay rate switch over to the same inverse-with-distance 
  pattern as the monopole. The distinction between near and far field regions 
  relies on the ratio of distance to the wavelength of sound: when that ratio 
  is small, you are in the near field, when it becomes large you are in the far 
  field. 

  Specifically, in the near field of a dipole source the sound pressure decays 
  like the square of distance, while for a quadrupole source it decays like the 
  cube of distance. Now suppose you had sources of the three types which had 
  comparable pressure amplitudes at the edge of the source region. Within the 
  near field, the sound from the dipole source will decay more rapidly that 
  that of the monopole source, and the sound from the quadrupole source will 
  decay even faster. The result is that by the time you reach the far field, 
  where all three sources have similar behaviour with distance, the monopole 
  source is the loudest, followed by the dipole source, followed by the 
  quadrupole source. 

  Now to apply these ideas to the various wind instruments. The first 
  instrument we looked into was the clarinet, playing its lowest note with all 
  the tone-holes closed. For that, the sound radiation behaviour is fairly 
  simple. All the complicated fluid dynamics around the reed and mouthpiece is 
  hidden inside the player’s mouth, and so does not contribute significantly to 
  external sound radiation. The acoustic pressure fluctuations inside the pipe 
  can only radiate sound at the open end of the tube, the clarinet bell. The 
  diameter of this bell is rather small compared to the wavelength of sound 
  except at the very highest audible frequencies, so we can expect the dominant 
  sound radiation to have a monopole character. The strength is governed by the 
  volume flow rate out of the bell, more specifically by the rate of change of 
  that volume flow rate. The bell is approximately a nodal point of pressure, 
  but it is an antinode for particle velocity, so this flow rate can be large. 
  The net result is that the sound from this note on the clarinet can be 
  strong, and omni-directional. 

  Now what happens if a more typical note is played on the clarinet, with some 
  tone-holes open? Each individual tone-hole is small compared to the 
  wavelength of any audible sound, so the air-flow through each hole will 
  create a monopole acoustical source. But the spacing between the tone-holes 
  is not so small, so when we think about the combined sound radiation from all 
  open holes as well as from the bell, we have to take into account wave 
  interference effects. Furthermore, we saw in section 11.1 that the first open 
  tone-hole does not act in the same way as sawing the tube short at that 
  point. Other open or closed tone-holes further down the tube have an 
  influence on the pitch (for example allowing fork fingerings to used). This 
  demonstrates that they are involved in the internal pressure distribution, so 
  all open tone-holes may radiate some sound. 

  The situation is shown schematically in Fig.\ 2. The relative phases of these 
  various sound sources will determine exactly how they add up to give the 
  combined sound radiation. This will result in some directionality in the 
  sound field. Two extreme cases are easy to explain. If all the monopole 
  sources were exactly in phase, they would add most strongly in directions 
  perpendicular to the tube, broadside on to the row of tone-hole sources. Not 
  just on the side where the holes are, though: the diameter of the tube is 
  small compared to the wavelength of sound at most audible frequencies, so 
  each individual monopole source will send sound more or less equally in 
  directions all around the tube. 

  \fig{figs/fig-a8c72566.png}{\caption{Figure 2. Sketch of sound radiation from 
  a clarinet note with several open tone-holes}} 

  The opposite extreme would occur if the sound inside the tube were a pure 
  travelling wave from mouthpiece to bell, with no reflected wave travelling 
  back. In that case, each successive tone-hole would have a short time delay 
  relative to the previous one, determined by the hole spacing and the speed of 
  sound. These time delays would be reflected in the phases of the monopole 
  sources, so that they would add together most strongly in the direction of 
  travel of the wave. The strongest sound radiation would then be in the 
  forward-facing direction. 

  The real situation will fall between these extremes: beyond the first open 
  tone-hole the internal sound field will be dominated by a travelling wave 
  towards the bell, but there will be a component of reflected wave. As first 
  pointed out by Benade [1], the result is a directional sound pattern with its 
  maximum amplitude on a forward-facing cone, making a small angle with the 
  axis of the tube. Some direct measurements of clarinet directivity [2] 
  confirmed this pattern, with a cone angle of the order of $20^\circ$. 

  Turning to brass instruments, the position is in some ways simpler. Apart 
  from unusual instruments like the cornetto, brass instruments do not have 
  tone-holes, and essentially all the sound is radiated from the bell of the 
  instrument, as sketched in Fig.\ 3. But the bell of a brass instrument like a 
  trombone or trumpet is much bigger than a clarinet bell. This means that 
  although it is small compared to the wavelength of sound at very low 
  frequencies, this ceases to be the case somewhere in the mid-range of 
  frequencies. 

  \fig{figs/fig-fbe688ea.png}{\caption{Figure 3. Sketch of sound radiation from 
  the bell of a brass instrument.}} 

  The result is that the sound radiation is omnidirectional at low frequency, 
  but becomes increasingly directional at higher frequency. The behaviour will 
  be similar to an example we saw back in section 4.3, of the sound radiation 
  by a circular piston set in a plane baffle: the plot from that section is 
  reproduced here as Fig.\ 4. The curves in different colours show the 
  behaviour at different frequencies, indexed by the “Helmholtz number”, which 
  for this problem we can visualise as the bell circumference divided by the 
  wavelength of sound. The practical consequence is something quite familiar. 
  If you stand behind a trumpeter the sound is fairly mellow, but if you walk 
  round to the front so that you are in the direct line where the bell is 
  pointing, the sound becomes a lot more brilliant. Instruments that are used 
  in musical contexts that emphasise mellowness and bass sound, like the French 
  horn or the tuba, have bells that do not point directly towards the audience. 

  \fig{figs/fig-81adec1c.png}{\caption{Figure 4. The directivity pattern of a 
  circular piston set in a plane baffle, reproduced from Fig. 11 of section 
  4.3. The curves are for Helmholtz number (bell circumference divided by the 
  wavelength of sound) equal to 0.3 (blue); 1 (red); 3 (yellow); 10 (magenta) 
  and 30 (green)}} 

  Sound radiation from the free reeds presents a very different situation from 
  the woodwind reeds and the brass instruments. There is no doubt that a free 
  reed is small compared to the wavelength of sound at audible frequencies, so 
  it will be a compact sound source of some kind. We can see three different 
  contributions that might be relevant. First, there is a mean flow of air past 
  the reed, and this is modulated by the nonlinear valve effect as the reed 
  moves towards or away from the base plate. This modulated volume flow rate 
  will constitute a monopole source of sound, rather like the sound from a 
  siren, in which a flow of air is interrupted by a rotating disc with holes in 
  it. 

  A free reed can also make sound arising directly from its own motion, 
  interacting with the air. This produces two effects, as we noted earlier for 
  the idealised example of an oscillating sphere. But the result is not quite 
  the same as for the sphere example. The reed is not vibrating in empty space, 
  because the base plate acts as a kind of baffle. So the surface motion of the 
  reed does not generate a dipole field: the air-flow created on one side 
  cannot cancel with the flow on the other side because the plate is in the 
  way. As a result, the volume flow rate driven directly by the reed motion 
  should be added to the volume flow rate from the “siren” effect to contribute 
  to the strength of the monopole source. 

  But the reed (and the oscillating sphere) also exerts a force on the air, and 
  this will create a dipole sound field. For the reasons explained above, we 
  might expect this dipole contribution to be relatively unimportant in the far 
  field. This is especially true at the fundamental frequency of the note, 
  because the near field will extend a relatively long way at this lowest 
  frequency, giving plenty of scope for the near-field decay of the dipole 
  field. But the higher harmonics of the note will have progressively smaller 
  near-field regions (because the size of the region is governed by the 
  wavelength of sound), so that the dipole contribution might be more 
  significant. 

  Finally, we come to the air-jet instruments like the flute. We will discuss 
  those in detail in the next section, but we can anticipate a little in order 
  to say a bit about their sound radiation behaviour. In some respects a flute 
  is rather similar to a clarinet. Volume flow from the end of the tube and 
  from any open tone-holes will produce local monopole sources in much the same 
  way as the earlier discussion. 

  But there is an extra ingredient, concerning the air-jet mouthpiece. Figure 5 
  is a repeat of a schematic diagram from section 11.1, showing the typical 
  geometry. An air jet emerges from a channel, crosses a hole and then impinges 
  on a sharp edge. The hole is called the “mouth” in a recorder, and the 
  “embouchure hole” in a transverse flute. The channel is provided by the 
  instrument maker in a recorder or a flue organ pipe, while it is formed by 
  the player’s lips in the case of a transverse flute. 

  \fig{figs/fig-64791292.png}{\caption{Figure 5. Schematic sketch of the 
  mouthpiece region of an air-jet instrument, from Fig. 7 of section 11.1}} 

  There are two types of sound source associated with the mouth opening. The 
  first is similar to a tone-hole: there is a net volume flow of air in and out 
  of the hole, and this provides a monopole source. The second is more 
  complicated. The air jet may be laminar or turbulent, depending on the 
  instrument and the particular playing technique in use. Especially when the 
  jet is turbulent, it provides an aerodynamic sound source. This component is 
  responsible for a broad-band “noisy” component of the sound: it may be 
  described as “breathy” or “wind noise”. In some musical contexts, this 
  “breathy” noise is an important part of the characteristic sound of the 
  instrument — a particular example is the end-blown Japanese flute called the 
  \tt{}shakuhachi\rm{}. 

  The turbulent flow constitutes a quadrupole source, and in empty space this 
  would be a very weak sound source for the reasons discussed above. But the 
  jet interacts with the sharp edge, and the resulting force generates a dipole 
  sound source, significantly more effective at radiating sound. Even so, this 
  component of the sound can be made more prominent in a recording using a 
  close microphone, which can be within the near field. 

  You may by now be wondering whether there is another way that wind 
  instruments can make sound, which I have been ignoring. Don’t the walls of 
  the instrument vibrate, and radiate sound rather like the body of a stringed 
  instrument? There is certainly a widespread belief among players and makers 
  of wind instruments that the material of which an instrument is made is 
  somehow important for the sound. I will discuss the issue of wall vibration, 
  and then digress slightly to discuss other ways in which the material of a 
  wind instrument might affect its sound. 

  The fact is that wall vibration in wind instruments is generally not very 
  important. It is not that the solid material of an instrument cannot vibrate 
  — the important issue is that those vibrations usually cannot be excited by 
  the internal pressure field created by playing the instrument. The bore 
  diameter of all wind instruments is generally small compared to the 
  wavelength of sound. There are perhaps exceptions for the larger sections of 
  instruments like the euphonium, but at the mouthpiece end, all instruments 
  have small bores. This means that the internal pressure distribution is 
  almost entirely in the form of more-or-less plane waves, with uniform 
  pressure across any cross-section of the bore. 

  Uniform pressure like this cannot easily excite vibration in the circular 
  wall of a woodwind or brass instrument, for the reason illustrated 
  schematically in Fig.\ 6. The tube would have to “inflate” in a symmetrical 
  manner, as indicated by the dashed line in the upper sketch. That would 
  involve stretching the wall material, which is resisted by a very high 
  stiffness. If the tube cross-section had been square, like the lower sketch, 
  things would be different. Now, the tube can be “inflated” by bending the 
  side-walls as sketched. If the walls were thin, the stiffness associated with 
  that bending action would be much lower. If you can remember back to section 
  10.3, we applied a somewhat similar argument to explain why softwoods like 
  spruce are much stiffer along the grain direction than in any cross-grain 
  direction: long-grain deformation involves stretching the cell walls, but 
  cross-grain deformation can take place by bending them, with much lower 
  stiffness. 

  \fig{figs/fig-69175a77.png}{\caption{Figure 6. Sketches of the deformation of 
  a circular (above) and square (below) tube in response to internal 
  pressure.}} 

  Of course something like a clarinet tube will have vibration resonances, but 
  the most obvious ones at relatively low frequencies would involve the tube 
  behaving as a bending beam. Bending vibration like this cannot be driven by 
  internal pressure variations in a narrow tube, for a reason that follows 
  directly from a symmetry argument. If we ignore the holes, keywork and so on, 
  the circular tube is symmetric in any plane through the central axis. Uniform 
  pressure variations are symmetrical with respect to a mirror reflection in 
  that plane, but bending motion in the plane is anti-symmetric (i.e. it 
  reverses if you reflect it in a mirror formed by the plane). A symmetric 
  force cannot excite an anti-symmetric motion. In the real world, with 
  tone-holes and so on, the mirror symmetry is not perfect. That means that 
  there might be weak but non-zero coupling between internal pressure and tube 
  bending, but the effect is very small. Even then, bending vibration of the 
  tube would not radiate much sound because the tube diameter is small compared 
  to the wavelength of sound. For practical purposes, the sound of a wind 
  instrument is not affected by wall vibration. 

  Admittedly, there are some exceptions under unusual circumstances. Some 
  wooden organ pipes have a square cross-section rather like the lower sketch 
  in Fig.\ 6. Way back in 1909, Dayton Miller [3] did an ingenious experiment 
  to demonstrate that wall vibrations of such pipes could sometimes couple to 
  the internal pressure, and have a detrimental effect on tone. In one 
  experiment he made a double-walled pipe, with a space in between that could 
  be filled with water. He demonstrated that for certain heights of the water 
  filling, the sound was affected because a bending resonance of the walls 
  matched a harmonic of the played note. Organ builders take some pains to 
  avoid such coincidences of frequency, mainly by making sure that the wooden 
  walls of pipes are quite thick. 

  A rather different kind of “exception” is associated with the convoluted 
  tubing of brass instruments. Figure 7 shows a U-bend in a tube, as a simple 
  example of the kind of thing that can be found in a trumpet or French horn. 
  Suppose the pressure inside this U-bend is raised. That pressure exerts an 
  equal force per unit area on all the walls of the tube. There is no net force 
  in the vertical direction or in the direction perpendicular to the diagram, 
  because the arrangement of pipe walls is symmetrical in those two planes and 
  so the pressure force cancels out. 

  \fig{figs/fig-3bf2fe19.png}{\caption{Figure 7. A U-bend in a tube, seen 
  sideways on and end-on.}} 

  But this does not happen in the horizontal direction. The right-hand sketch 
  in Fig.\ 7 shows an end-on view of the U-bend. The total projected area of 
  pipe wall facing away from the viewer is somewhat bigger than the total 
  projected area facing towards the viewer, because of the two holes shaded 
  grey. This means that the raised pressure produces a net force towards the 
  right in the left-hand sketch. So the time-varying pressure when a note is 
  played produces an oscillating force, which may excite vibration in the rest 
  of the tube-work. This may not radiate a lot of sound (although axial 
  vibration of the instrument bell might behave a bit like a piston), but it 
  can certainly be felt by the player’s hands. 

  However, wall vibration is by no means the only way that the sound and 
  playing properties of a wind instrument might be influenced by the material 
  they are made of. We can start with the broad distinctions: instruments may 
  be made of various kinds of metal, wood or plastic. A key conclusion from 
  what we have said previously is that the acoustical behaviour is determined 
  almost entirely by the precise shape of the internal bore of the instrument 
  (including details like the internal profile of closed tone-holes). There is 
  no reason in principle why an identical bore shape could not be built from 
  any of these materials. 

  But in practice things are not so simple. Different tools and working methods 
  are used with the different types of material, and these tend to leave their 
  trace in subtle differences of detail. And there is the most important factor 
  of all, the skill of the instrument maker, and how long they are prepared to 
  spend tweaking details to get things just right. Think about the extreme 
  cases. On the one hand a plastic recorder is likely to be mass-produced, with 
  relatively low standards of quality control. These can still be quite good 
  instruments, depending on how much care was originally taken in making the 
  mould template and also on how good a job the manufacturing process makes of 
  reproducing tiny details. 

  But contrast that with a hand-made wooden recorder, let alone a high-end 
  instrument like the famous golden flute of James Galway. Such instruments 
  are, of course, far more expensive than the plastic recorder. What you are 
  paying for is not, mainly, the cost of the material as such, but the skill of 
  the instrument maker. The intrinsic cost and beauty of the material tends to 
  move up in step with the level of effort devoted to making the instrument, 
  but it is not the gold or the boxwood as such that determines the quality of 
  the final product: it is the combination of the material characteristics with 
  the maker’s skill and dedication. 

  So what are these all-important details that a skilled maker must get right? 
  It is hard to give an exhaustive and definitive answer to that question, but 
  we can give some examples. First, and simplest, are the mechanical things: 
  keys, valves and trombone slides must work smoothly without leaking, key-pads 
  must seal their holes crisply. But these are not really material-related 
  issues. More immediately relevant is the internal surface finish of the bore. 
  One important source of energy dissipation in an instrument tube comes from 
  viscous effects in the boundary layer. Surface roughness of any kind will 
  tend to increase dissipation and reduce the Q-factors of tube resonances. 
  Roughness may come from manufacturing: tool marks from the process of boring 
  a hole in a piece of wood, whiskers of wood or metal left over from drilling 
  tone-holes, or surface imperfections from casting a plastic instrument. 

  In the case of a wooden wind instrument, the internal surface finish may also 
  vary with exposure to the warm, moist air that the player blows down the 
  tube. Some types of wood are relatively immune to being influenced by 
  humidity — but these are generally tropical hardwoods, which are beginning to 
  raise other important issues of sustainability and the restrictions imposed 
  by the CITES rules. Historical wind instruments were (and are) often made of 
  European hardwoods such as boxwood or pearwood, and those timbers certainly 
  are susceptible to the effects of humidity, which can “raise the grain” on 
  the internal surface of the bore and thus increase energy dissipation. A 
  standard way to guard against this is to oil the wood, traditionally by 
  applying oil to the inside of the bore with a long feather. 

  More subtly, the inside edges of tone-holes may be square, or chamfered, or 
  rounded — see the sketches in Fig.\ 8. The same is true for other 
  discontinuities in the internal shape, such as the exit of the air-jet 
  channel in a recorder. Adjusting these edge details will have a small 
  influence on the tuning of resonances, but it will also influence the 
  shedding of vortices by the air-flow around the hole, which in turn 
  influences energy dissipation and the Q-factors. Tweaking these things is a 
  major part of the process of “voicing” an instrument. Naturally, the voicing 
  process will use different tools in a wooden or a metal instrument, which may 
  in itself have consequences for the end result. It would be unusual to carry 
  out such voicing at all on a plastic instrument — but it is not impossible. I 
  have a clear memory of seeing Arthur Benade demonstrating voicing adjustments 
  on a plastic recorder, back in the 1970s. 

  \fig{figs/fig-c72a6f41.png}{\caption{Figure 8. Sketches of tone-hole 
  cross-sections with square, chamfered and rounded edges on the inside of the 
  bore.}} 



  \sectionreferences{}[1] Arthur H. Benade, “On the mathematical theory of 
  woodwind finger holes”, Journal of the Acoustical Society of America 
  \textbf{32}, 1591—1608 (1960) 

  [2] Fumiaki Ehara and Shigeru Yoshikawa, “Radiation directionality 
  measurement of clarinets made of different wall material”, Forum Acusticum 
  2005, 525—528 (2005), available at 
  \tt{}http://www.conforg.fr/acoustics2008/cdrom/data/fa2005-budapest/paper/449-0.pdf\rm{} 

  [3] Dayton C. Miller, “The influence of the material of wind instruments on 
  the tone quality”, Science \textbf{29}, 151—171 (1909). 