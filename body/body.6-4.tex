

  In the loudness test just described, the listener only ever heard one sine 
  wave at a time. But most sounds contain many frequency components 
  simultaneously: the separate harmonics of a complex tone, or the sounds of 
  two instrument playing together, or the sound of someone talking in the 
  presence of background noise. Naturally, this makes the business of 
  understanding perception more complicated --- but it is not quite as 
  complicated as one might have feared. 

  A very important idea comes into play, based around the description we gave 
  in section 6.2 of the behaviour of the basilar membrane within our inner ear. 
  With a sine wave at some particular frequency, the response of the basilar 
  membrane is strongest in a particular region --- but of course this region 
  has a finite spread. So a range of hair cells will respond to some extent to 
  this sine wave, and conversely a given hair cell will respond to some extent 
  to a range of frequencies, spread around its frequency of maximum 
  sensitivity. This leads to the idea of an auditory filter. 

  The mechanical response at a certain point on the basilar membrane will have 
  a frequency response function, similar to the ones we have looked at for 
  musical instruments. We would expect it to have the shape of a rather simple 
  band-pass filter, showing a peak of response at the ``resonant'' frequency, 
  and lower response for frequencies progressively further away on either side. 
  This might be expected to translate into some kind of frequency response 
  related to the firing rate of the neuron attached to a hair cell at that 
  point on the basilar membrane. 

  There are various ways to map out the characteristics of these auditory 
  filters. It is possible to insert a small electrode into a single neuron in 
  the auditory nerve, and record directly the firing rate: if the threshold 
  sound level to give detectable firing is measured for different frequencies 
  of sine wave, the result is called a neural tuning curve. Such curves do 
  indeed have the form we have anticipated, with a peak sensitivity at a 
  frequency corresponding to the placement on the basilar membrane of the hair 
  cell attached to the particular neuron. 

  Alternatively, psychoacoustical methods of testing can be used to find a 
  filter shape directly from listening tests. Most methods are based around the 
  phenomenon of masking. If a loud sound and a quiet sound happen 
  simultaneously, to what extent is the quiet sound covered up by the loud 
  sound? This can be investigated by a rather similar experiment to the one 
  just described for assessing loudness. A test subject, with their headphones, 
  is exposed to a fixed sine wave at some chosen frequency, and a second sine 
  wave at a different frequency. This time, the two sounds are simultaneous 
  rather than being presented alternately. The goal is to measure the threshold 
  for detecting the quiet sound. 

  What is discovered is that if the frequencies of the two sounds are 
  sufficiently different, the threshold for detecting the quiet sound is 
  exactly the same as it would have been in the absence of the loud sound: in 
  other words, at the level given by the threshold curve in Fig.\ 1 of the 
  previous section 6.3. But if the two frequencies are close together, then the 
  quiet sound is significantly masked by the loud sound. 

  However, in practice using a sine wave for a masking experiment like this 
  does not give the most useful results. The main problem is that when two sine 
  waves at slightly different frequencies occur simultaneously, the phenomenon 
  of beats arises. The effect is most striking for sine waves of the same 
  amplitude, as illustrated in Fig.\ 1. The strong modulation of the amplitude 
  is at a frequency which is the difference of the two separate sine waves. In 
  this example the sine waves are at 200 Hz and 203 Hz, so the beat frequency 
  is 3 Hz. You can hear the waveform in Sound 1. 

  If the two amplitudes are not equal, a weaker version of the same phenomenon 
  occurs. An example is shown in Fig.\ 2, where the second sine wave has 1/10 
  the amplitude of the first one. You can hear the waveform in Sound 2. The 
  modulation is still clearly visible in the plot. It is subtle but audible in 
  the sound. This is the effect which can interfere with an experiment to probe 
  masking. Basically, we might be using a different feature detector to notice 
  the beats, not the one the experimenter is trying to exercise. 

  It is better to use random noise of some kind for the masking signal, to 
  avoid this issue of the subject detecting the test sound via beats. In this 
  context ``noise'' has a technical meaning that is more specific than the 
  colloquial usage. One thing we might do is replace the sinusoidal masking 
  tone with narrow-band noise, which would mean adding together sine waves at a 
  range of closely-spaced frequencies, all with the same amplitude but with 
  random phases. There is also an ingenious technique due to Patterson [1] 
  which uses the opposite pattern, called notched noise: a combination of sine 
  waves at all frequencies except for a range around the test tone. 

  By carrying out experiments by these various methods, auditory filter shapes 
  can be mapped out. Some examples are shown in Fig.\ 3. For very low-level 
  sound these filters behave approximately linearly, but for louder sounds the 
  filter characteristics change progressively with sound level because of 
  nonlinear effects. The filter shapes become more asymmetric, and the degree 
  of ``tuning'' changes. Tuning is sharpest for low-level sounds: it is 
  strongly influenced by the action of the active outer hair cells. As with so 
  much of the subject, it is all rather complicated: for some details, see 
  Moore [2]. Figure 3 is plotted with a logarithmic frequency scale, which 
  reveals that the filters centred on higher frequencies tend to have very 
  similar shapes. This means that their bandwidth is approximately proportional 
  to their centre frequency. But at lower frequency this relative bandwidth 
  gets wider. 

  The bandwidth of each auditory filter defines an important quantity known as 
  the critical bandwidth. Roughly, any two frequency components falling within 
  a critical bandwidth of each other are too close to be resolved on the 
  basilar membrane. We will see shortly that this has a number of perceptual 
  consequences. The precise numerical definition of the critical band depends 
  on exactly what definition of bandwidth is used. Historically, different 
  authors basing their conclusions on different test procedures have used 
  different definitions. But we need not go into these details. 

  We will use one particular definition, the equivalent rectangular bandwidth 
  or ERB. This is defined as the width of an idealised rectangular filter shape 
  which would pass the same total power as the actual auditory filter. If, 
  instead of the decibel plot of Fig.\ 3, the filter shapes were plotted on a 
  linear vertical scale of squared amplitude, the ERB would simply be the width 
  of a rectangle with the same area underneath it as the filter curve, and the 
  same maximum height. What this means in terms of the decibel plot in Fig.\ 3 
  is that the ERB is quite close to the 3 dB bandwidth of each peak, the same 
  measure we used earlier for the bandwidth of resonances of mechanical 
  systems. A plot of the ERB as a function of centre frequency is shown in 
  Fig.\ 4. It is believed that a frequency scale based on ERBs corresponds to a 
  scale of distance along the basilar membrane: 1 ERB corresponds to about 0.9 
  mm. 

  Armed with the set of auditory filters, we can construct something important. 
  For any given sound input, be it a sine wave or a Beethoven symphony, we can 
  use the set of filters to map out the pattern of motion of the basilar 
  membrane, known as the excitation pattern. If the input sound is steady this 
  will be a fixed pattern, but for normal sounds entering your ears the pattern 
  will vary in time. 

  We will start by illustrating a steady example. The left-hand plot in Fig.\ 5 
  shows a computed excitation pattern for a steady sawtooth wave at 440 Hz, at 
  sufficiently low amplitude that we can use the linear approximation to the 
  auditory filters. The first few of the successive harmonics, at multiples of 
  440 Hz, are clearly separated, but at higher frequency the pattern blurs into 
  a continuum. The right-hand plot shows why. The ERB at each successive 
  harmonic frequency is plotted, normalised by the spacing between harmonics. 
  The value reaches 1 around the 8th harmonic: above that, successive harmonics 
  are within 1 ERB of each other, so that they are not clearly resolved in the 
  excitation pattern. 

  Next we look at a simple time-varying example. Back in section 2.4, we saw a 
  spectrogram of a violin note, being played with vibrato. The spectrogram is 
  repeated as Fig.\ 6, and the associated sound appears as Sound 3. 

  We can analyse the same sound as a time-varying excitation pattern. To 
  achieve this, we can take advantage of an approximate form of the auditory 
  filters, called \tt{}gammatone filters\rm{} [3]. These have a neat 
  mathematical expression, convenient for computing, and they have been shown 
  to give a good match to the measured auditory filters. The result is shown in 
  Fig.\ 7. It looks rather like a spectrogram, and it has been plotted in a 
  similar way to Fig.\ 6. The horizontal axis shows an ERB-based frequency 
  scale which is not very different from a logarithmic scale. Time runs 
  vertically upwards. The excitation magnitude is indicated by colour, on a 
  decibel scale shown in the sidebar. 

  The harmonics of the violin note can be seen, but they are much more blurred 
  than the version in Fig.\ 6. The ``wobble'' caused by vibrato is still 
  clearly visible in the first few harmonics. However, by about the 7th 
  harmonic it is getting hard to see the separate stripes in the plot, whereas 
  they remained clearly visible in the spectrogram. The reason is exactly the 
  same as the example in Fig.\ 5: this frequency is around about where the 
  spacing between successive harmonics is comparable with the critical 
  bandwidth so that they are not resolved in the excitation pattern. The 
  spectrogram representation shows the underlying physics more clearly, but the 
  ``auditory spectrogram'' of Fig.\ 7 relates much more closely to the way a 
  human listener will perceive the sound. 

  For the next example, it would be a good idea to listen (rather carefully) to 
  Sound 4 before you know what it contains. 

  What you were listening to was a pair of sine waves, with the same amplitude 
  and gradually changing frequencies. They are always centred on 800 Hz. They 
  start 160 Hz apart, and the two frequencies move symmetrically inwards at a 
  steady rate until they just come together at the end of the file. How would 
  you describe what you hear? To my ears, the two separate tones, gradually 
  converging, are clear at the beginning. But somewhere around the middle of 
  the file the perception changes. It is no longer clear that there are two 
  tones, but there is gradually increasing ``roughness'' in the sound, which 
  resolves into slowing beats right at the end. 

  Figure 8 shows the excitation pattern plot generated from this sound, using 
  the gammatone filters again. In the lower part of the diagram, two nearby but 
  separate stripes can be seen, converging as time goes on and looking rather 
  like a pair of trousers. Somewhere near the middle of the plot, these two 
  stripes merge: this is not surprising, because the ERB at 800 Hz is 111 Hz. 
  At the top of the plot, a pattern of modulation in time can be seen. It is 
  quite fast when first visible, and gradually slows down. These are beats, 
  exactly as shown earlier in Fig.\ 1, except that the beat frequency is 
  changing with time. The beats are at the difference of the two frequencies, 
  so the beat rate slows down towards zero at the end. This example illustrates 
  two things. First, an apparently simple combination of two sine waves can 
  produce perceptions of different kinds: from changing tones to something 
  involving ``roughness'', and eventually clear pulsation in time. Second, we 
  get an indication that the excitation pattern plot reveals at least some 
  aspects of these different perceptions; although we can see nothing very 
  obvious associated with the sense of ``roughness''. 

  What we have seen is that the excitation pattern captures some aspects of 
  sound perception, although by no means all. There is a striking success story 
  related to loudness. In section 6.3 we saw a bit of the complexity of 
  loudness relating only to single sine waves. One might have expected many 
  more layers of complexity with more complicated sounds, but in fact there is 
  a very successful loudness model by Moore and Glasberg [4], which builds on 
  the excitation pattern. For this purpose, the linear gammatone approximation 
  is not good enough, though: the nonlinear variation of auditory filter 
  characteristics with level must be included. The Moore/Glasberg model first 
  calculates the excitation pattern, steady or time-varying as appropriate, 
  then aggregates that pattern into a single measure of loudness. 

  For the rather elusive concept of ``timbre'', the excitation pattern is 
  useful, but certainly does not tell the whole story. We have already seen an 
  example, in Sound 4 and Fig.\ 8. The impression of ``roughness'' did not 
  correspond to anything obvious in the excitation pattern plot. But on the 
  other hand we will see examples in the next section where analysis of 
  excitation patterns gives strong clues about whether a small change to a 
  sound will be audible. A summary of current understanding seems to be that if 
  two excitation patterns are sufficiently different, the two sounds are almost 
  certainly audibly different, but the converse is not true: two sounds can be 
  distinct but have very similar excitation patterns. 

  A clear example of this comes from something that you may have been wondering 
  about. A typical ERB is a few semitones, when expressed in musical jargon 
  (remember that a semitone is a frequency ratio of about 6\%). Music as we 
  understand it would hardly be possible if we could not distinguish two sounds 
  a semitone apart! In fact, our ability to discriminate pitch between two 
  steady periodic sounds with different periods is far more acute than the ERB. 

  This acuity of pitch perception is usually expressed in terms of cents: a 
  cent is a hundredth of an equal-tempered semitone, so 1 cent corresponds to a 
  frequency ratio of about 0.06\%. Well, we cannot discern a difference as 
  small as 1 cent, but under the best conditions people can discern pitches 
  about 5 cents different. (Strictly, I am speaking rather loosely here: what I 
  mean is ``people can discern a pitch difference between two frequencies 
  differing by about 5 cents''.) This value is (approximately) the threshold 
  for pitch discrimination when two sounds are heard one after another. If the 
  two sounds are heard simultaneously, we may be even more acute to 
  ``out-of-tuneness'', because the phenomenon of beats comes into play again. 
  So when you are tuning your violin, or playing or singing in an ensemble, you 
  need to get pitches remarkably accurate, as all musicians (and parents of 
  budding musicians) know only too well. 

  How do we manage to distinguish two pitches as close as 5 cents (i.e. a 
  frequency ratio of about 0.3\%)? There has been much debate and controversy 
  about that question over the years. The answer almost certainly involves the 
  interplay of two factors. One is the one we have already been talking about: 
  there is relatively coarse segregation in frequency already present on the 
  basilar membrane, and there are subtle features of the exact pattern of 
  neural excitation by different hair cells that may encode a more precise 
  estimate than the ERB would suggest. But there is a second factor. At least 
  at lowish frequencies (up to the mid-kHz range), the firing of individual 
  nerve fibres can be synchronised with the phase of the input sound, 
  presumably by being phase-synchronised in some way with local motion of the 
  basilar membrane. This means that information about the frequency is reaching 
  your brain in this form, as well as the information coded into which hair 
  cells are being activated. Somehow, somewhere, these two sources of 
  information are probably being combined in your brain to allow you to 
  discriminate pitches with the precision that is observed. 



  \sectionreferences{}[1] Roy D. Patterson; ``Auditory filter shapes derived 
  from noise stimuli'', Journal of the Acoustical Society of America 
  \textbf{59}, 640--654 (1976). 

  [2] Brian C. J. Moore; ``An Introduction to the Psychology of Hearing'', 
  Academic Press (6th edition 2013). 

  [3] The gammatone software used here is by Malcolm Slaney: ``Auditory Toolbox 
  Version 2'', Technical Report \#1998-010, Interval Research Corporation 
  (1998), \tt{}http://cobweb.ecn.purdue.edu/~malcolm/interval/1998-010/\rm{} 

  [4] The Moore/Glasberg loudness model has developed over time, and there are 
  many references. A key one is Brian R. Glasberg and Brian C. J. Moore; “A 
  model of loudness applicable to time-varying sounds”, Journal of the Audio 
  Engineering Society \textbf{50}, 331--342 (2002). 