

  Having covered some background material relevant to a wide range of musical 
  wind instruments, it is time to turn to a more detailed look at the main 
  families of instruments. We will begin with the reed instruments, starting 
  with the clarinet because this is arguably the most-studied member of the 
  family. We will find some surprisingly close analogies with the behaviour of 
  a bowed string, which we studied in Chapter 9. Later in this section we will 
  look at the soprano saxophone, as an example of an instrument with a conical 
  bore rather than a cylindrical bore like the clarinet. This too will have an 
  analogy with bowed string behaviour. If you have read Chapter 9 before 
  reaching this point, these analogies will make immediate sense to you. But if 
  you have dipped in at this point without knowing much about bowed strings, 
  you might find it helpful to break off here and take a look at the next link 
  before continuing this section. It gives a short, non-technical summary of 
  some key points of bowed-string behaviour. 

  We have already looked at some aspects of the linear acoustics of wind 
  instrument tubes, in section 11.1, so the main emphasis in this section will 
  be on the excitation mechanism of reed instruments, and the consequences of 
  that for sound and playing behaviour. For an immediate impression of how 
  important the excitation mechanism is, look at \tt{}this rather amusing 
  video\rm{}. In it, Joe Wolfe demonstrates what happens if you swap the 
  mouthpieces of a flute and a clarinet. The clear conclusion is that any 
  roughly cylindrical instrument played with a clarinet mouthpiece sounds very 
  much like a clarinet, while one played with a flute mouthpiece sounds like a 
  flute. 

  For both the clarinet and the saxophone examples, we will use computer 
  simulations based on an idealised model to produce diagrams somewhat 
  analogous to the Schelleng diagram for a bowed string (see the previous link 
  for an explanation). These give an indication of how a player might perceive 
  the variation of behaviour, depending on how they control the way they blow 
  into the instrument. 

  \textbf{A. Simplest model of the clarinet} 

  We have already given a schematic description of how a clarinet mouthpiece 
  works, back in section 8.5. That description was qualitatively correct, but 
  now we can look at a version based on explicit modelling. Figure 1 shows a 
  sketch. The reed is a tapered, flexible piece of natural cane or synthetic 
  material, which is clamped against the rigid tube wall. The player seals 
  their lips around the mouthpiece and reed, and they blow air in. As indicated 
  in the sketch, there is a concentrated air jet in the narrow gap, then inside 
  the mouthpiece it breaks up into turbulent eddies and dissipates the kinetic 
  energy. (Look back at Figs.\ 1 and 2 in section 11.2 for a reminder about 
  laminar and turbulent flows: the concentrated air jet is laminar in the 
  narrow gap past the reed tip.) For the purposes of this simple description, 
  we will assume that the pressure inside the player's mouth is simply 
  constant. This is not always true, though, in real performance: expert 
  players may manipulate the resonances of their vocal tract to make subtle 
  adjustments. 

  \fig{figs/fig-cf1e4646.png}{\caption{Figure 1. Sketch of the air flow into a 
  clarinet mouthpiece}} 

  We get a reasonable approximation to the physical effects if we assume that 
  the simplest form of Bernoulli’s law applies to the jet in the region where 
  it passes over the reed surface. The high speed of the jet produces a reduced 
  pressure compared to the mouth pressure acting on the lower face of the reed. 
  This pressure difference tends to draw the reed inwards, and eventually to 
  close it against the “lay”, the rigid lip and side plates of the mouthpiece. 
  As explained in the next link, this gives an explicit formula for the volume 
  flow rate of air through the gap, as a function of the pressure difference 
  between the player’s mouth and the inside of the mouthpiece. 

  This function is plotted in Fig.\ 2, using parameter values appropriate for a 
  clarinet reed. The exact shape depends on a small number of parameters: the 
  width and height of the initial gap between reed and lay, and the stiffness 
  of the reed. Of course, the reed has resonances — the previous link show how 
  to include the lowest resonance in our modelling. We will come back to this a 
  little later, at which time the resonance frequency and damping factor will 
  need to be added to the list of parameters. But for the moment we will assume 
  that the playing frequency is well below the reed’s first resonance, in which 
  case the reed will behave like a simple spring. 

  \fig{figs/fig-a705d421.png}{\caption{Figure 2. The simplest model of a 
  clarinet mouthpiece: the flexible reed provides a nonlinear valve, with a 
  relation between pressure difference and volume flow rate like the red curve 
  here. The sloping dashed lines indicate two possible positions of the line 
  whose intersection with the red curve determines the new values of pressure 
  and flow rate. Note that many authors plot this function with the horizontal 
  axis running in the reverse direction: do not be confused if you are 
  comparing this plot with textbooks like [1]!}} 

  Figure 2 also reminds us of another key ingredient of the model developed in 
  section 8.5. In a simulation of the transient response, at each time step we 
  first calculate the incoming pressure wave, returning down the clarinet tube 
  after reflecting from the tone-holes and bell. This incoming pressure 
  determines the position of a sloping straight line: two examples of possible 
  positions are shown in Fig.\ 2. The pressure and volume flow rate at the next 
  time step are determined by finding the intersection between the straight 
  line and the nonlinear curve. The details governing the position and slope of 
  the straight line were set out in section 8.5.3. 

  The plot in Fig.\ 2 shows the volume flow rate as a function of the pressure 
  difference between the inside and the outside of the mouthpiece. However, for 
  a model of the sound generation inside the instrument we want to express it 
  is a function of the actual pressure $p(t)$ just inside the mouthpiece. To 
  achieve this, we simply need to add the player’s mouth pressure. (Remember 
  that for this introductory account we are regarding this mouth pressure as 
  being constant: we are neglecting the effect of resonances inside the 
  player's mouth.) As the player increases their mouth pressure, the curve is 
  shifted progressively to the right, as shown in Fig.\ 3. 

  \fig{figs/fig-606f2404.png}{\caption{Figure 3. The nonlinear valve 
  characteristic from Fig. 2, plotted now as a function of pressure $p(t)$ 
  inside the mouthpiece for four different values of the player's mouth 
  pressure: lowest for the red curve, highest for the black curve. The values 
  of mouth pressure are 0.8~kPa, 2.8~kPa, 4.8~kPa and 6.8~kPa.The blue curve 
  shows the case close to the threshold for the clarinet to start to sound.}} 

  We have already talked about the result of this shift, back in section 8.5. 
  When the mouth pressure is low, as in the red curve in Fig.\ 3, the 
  mouthpiece behaves like a mechanical resistance, tending to dissipate the 
  energy associated with any pressure variation inside the tube. But once the 
  pressure is high enough to shift as far as the blue curve, the tangent to the 
  curve where it crosses the vertical axis starts to slope in the opposite 
  sense, and the mouthpiece then behaves like a negative resistance, tending to 
  amplify disturbances. This is the threshold condition for the clarinet to 
  start to make sound. Blowing harder still can produce the green curve, and 
  then the black curve. In both these cases the tangent is sloping steeply and 
  the amplification effect is very strong. 

  Blowing harder and shifting a little further, the vertical axis would pass 
  through the horizontal section of the curve where the reed is fully closed. 
  The tangent is horizontal, and there is no amplification. Under those 
  conditions, the instrument is likely to remain silent: it is ``choked'' by 
  blowing too hard. However, we will see shortly that things are a little more 
  complicated than that. With some kinds of initial articulation transient by 
  the player, it is perfectly possible to produce sound from the instrument 
  with a mouth pressure high enough to be in this region. This is a first hint 
  of ``playability'' issues in our model clarinet, something we return to in 
  more detail later. 

  Figure 4 illustrates possible effects on the nonlinear valve characteristic 
  of varying two other key parameters in the model. The datum for comparison is 
  the green curve, which is the same as the green curve in Fig.\ 3. The blue 
  curve shows the effect of the player choosing a stiffer reed — reeds are sold 
  in various grades of “hardness”. The stiffer reed, set to the same initial 
  gap, requires a bigger pressure difference before it closes, so the curve is 
  higher. The red curve shows the effect of setting the stiffer reed with its 
  tip closer to the lay so that the closure pressure remains the same as in the 
  green curve. The curve is lower, because the volume flow is reduced through 
  the narrower gap. 

  \fig{figs/fig-4198acc0.png}{\caption{Figure 4. Variations in the nonlinear 
  valve characteristic from Fig. 2 when the reed properties are changed. The 
  green curve is the same as the green curve in Fig. 3. The red curve shows the 
  effect of reducing the initial reed opening from 0.6~mm to 0.4~mm while 
  keeping the same closure pressure (so that a slightly stiffer reed is 
  implied). The blue curve shows the effect of increasing the reed stiffness 
  without changing the gap, so that the closing pressure is increased.}} 

  Referring back to Fig.\ 1, we can see that the player might have some 
  influence on these two parameters during performance. The player can choose 
  the precise contact position of lip/teeth with the reed, and also how hard to 
  “bite”. Biting harder will reduce the gap, and the lip/teeth contact will 
  influence the effective stiffness of the reed. It will also change the reed 
  resonance frequency, and its damping factor. 

  \textbf{B. Why is a clarinet like a violin?} 

  Now we can start to draw interesting parallels with a bowed violin string. 
  The simplest model of a bowed string, discussed in section 9.2, involves a 
  very similar set of ingredients to the clarinet model just outlined. There is 
  a nonlinear function, the “friction curve”, and in a transient simulation the 
  values of force and string velocity at the next time step are determined by 
  the intersection of this curve with a straight line. The position of the 
  straight line is determined by incoming reflected waves on the string. 

  Furthermore, a violinist can influence the friction curve through two of 
  their main control variables: the effects are illustrated in Fig.\ 5. Look 
  first at the red curve, which shows the typical shape of a friction curve. 
  The player can change this to the blue curve by increasing their bow speed, 
  or alternatively they can change it to the green curve by reducing the bow 
  force. Comparing this figure with Figs.\ 3 and 4, we see obvious parallels. 
  Bow speed behaves like mouth pressure, shifting the nonlinear curve sideways. 
  Bow force behaves a bit like the two effects explored in Fig.\ 4: the 
  nonlinear function stays a similar shape, but the peak value can be increased 
  or decreased. 

  \fig{figs/fig-62ea4e1d.png}{\caption{Figure 5. Variations on the friction 
  curve, for the simplest bowed-string model. The red curve has normal force 
  10~N and bow speed 0.05~m/s; the blue curve has the same normal force but 
  increased bow speed 0.1~m/s; the green curve has the same bow speed as the 
  red curve, but reduced normal force 4~N.}} 

  So two of the violinist’s control variables have clear analogues in the 
  clarinet. However, this is not true of the third major variable, the bow 
  position on the string. Schelleng’s diagram (see the first side link above, 
  and section 9.3) showed us that the bow position has a very strong influence 
  on how a violin string behaves. But the mouthpiece of a clarinet is not 
  plugged into the tube at an intermediate point, analogous to the position of 
  a violin bow. Instead, it drives the tube at one end. 

  There is an analogue in the bowed string, but it is a very extreme and 
  unusual case: a clarinet behaves rather like a string bowed at its exact 
  centre. The comparison is illustrated in the sketches in Fig.\ 6. The bowed 
  string will respond with symmetrical motion in the two halves. If the wave 
  speeds are the same on the string and in the clarinet tube, it is easy to see 
  that the fundamental pitches agree. The clarinet will have a quarter-wave in 
  the length, while the string will have a half-wave in a length which is twice 
  as long. 

  \fig{figs/fig-e66c4f69.png}{\caption{Figure 6. The analogy between a clarinet 
  and a string bowed at its mid-point}} 

  This comparison tells us something very interesting. When we first started 
  looking at the behaviour of bowed strings, we described the work of Helmholtz 
  and Raman. Well, the descriptions we used can be carried over directly to the 
  clarinet problem. First, we have Helmholtz’s observation of how a bowed 
  string moves. For the special case of bowing at the mid-point, the Helmholtz 
  motion will have a pair of “Helmholtz corners” travelling in opposite 
  directions on the string, to give the required symmetrical motion in the two 
  halves. At the bowed point there will, as usual, be an alternation between 
  sticking and slipping, once per cycle. The resulting waveform of string 
  velocity at the bowed point will be a symmetrical square wave, which is the 
  relevant special case of the usual rectangular pulse waveform. 

  Raman’s argument, explained in section 9.1.2, gives an explanation of this 
  waveform for the idealised case with no dissipation. We will phrase the 
  description in terms of the clarinet example, but it is exactly the same as 
  described previously for the bowed string. If the tube has no damping, then 
  the only way it is possible to have steady, periodic oscillation of the 
  internal pressure is if the forcing, from the waveform of volume flow 
  injected at the mouthpiece, is exactly constant. If it were not constant, it 
  would contain Fourier components at the resonant frequencies of the tube, and 
  in the absence of damping these would evoke a response which would grow 
  without limit, contradicting the assumption of steady motion. 

  The result is that the volume flow can only switch between two values, 
  positioned on a horizontal level on the nonlinear valve curve. For the 
  clarinet, the pressure waveform (analogous to the string velocity waveform in 
  the bowed string) will be a symmetrical square wave, so the pair of points 
  must have equal and opposite pressure. Figure 7 illustrates what must happen: 
  this shows the same four curves as Fig.\ 3, and three of them are annotated 
  with circles in corresponding colours showing where the pair of points must 
  lie. But for the curve with the lowest mouth pressure, the red curve, it not 
  possible to find such a pair. We conclude that the clarinet version of ideal 
  Helmholtz motion is possible with mouth pressures corresponding to the blue, 
  green and black curves, but not for the red curve. 

  \fig{figs/fig-dcfea7bb.png}{\caption{Figure 7. The four curves from Fig. 3, 
  annotated to illustrate Raman's argument. With no energy dissipation in the 
  tube, the pressure waveform must be a square wave, alternating between two 
  values on the same horizontal level. This is possible for the blue, green and 
  black curves, as indicated by the circles, but it is not possible for the red 
  curve so ``Helmholtz motion'' is not possible with this mouth pressure. Plot 
  adapted from Fig. 9.10 of Chaigne and Kergomard [1].}} 

  As a reminder of the corresponding argument for the bowed string, Fig.\ 8 
  shows a copy of Fig.\ 1 from section 9.1.2. This illustration is not for a 
  mid-point bow but for a more normal bow position close to one end of the 
  string. The Helmholtz motion then involves a rectangular velocity waveform 
  alternating between two speeds, but it is not a symmetrical square wave. 
  Instead, the proportions of time spent sticking and slipping are in the same 
  ratio as the position of the bowing point as a fraction of the string length. 
  But it would be easy to find a different horizontal level on this plot, with 
  a symmetrical pair of intersections on either side of the vertical axis. That 
  line would define the velocities and friction force level for mid-point 
  bowing of a string with no energy dissipation, directly analogous to the 
  clarinet case shown in Fig.\ 7. 

  \fig{figs/fig-97b7ee76.png}{\caption{Figure 8. A copy of Fig. 1 from section 
  9.1.2, showing a schematic plot of friction force as a function of sliding 
  speed, where $v\_b$ is the bow speed. The black line and stars relate to 
  Raman's argument, explained in the text.}} 

  The analogy with the bowed string continues: we can go beyond the 
  super-simplified Raman model by using an equivalent approach to the one 
  described in section 9.2, in which the Helmholtz corner was allowed to become 
  rounded, rather than being ideally sharp. In fact, we can do that more 
  systematically for the clarinet than we were able to do for the violin 
  string. There is a mathematical trick, described in the next link, that 
  allows a measured or simulated input impedance to be processed to reveal the 
  “reflection function” needed for an accurate simulation. 

  For the purposes of this section, we will use a sort of “one-note clarinet”: 
  a cylindrical tube of the right length and bore diameter for a clarinet, but 
  without finger holes (either open or closed). We can calculate the input 
  impedance of such a tube, with plausible damping, in the way described in 
  section 11.1.1. The details are given in the link, with some plots. The 
  result of simulations using this reflection function should give a reasonably 
  good representation of the lowest note of a clarinet --- certainly good 
  enough to bring out some interesting aspects of the physics. 

  For a first example, we can look at (and listen to) three simulated notes 
  based on the blue, green and black curves in Fig.\ 3, using three different 
  mouth pressures while keeping everything else fixed. Each note was 
  synthesised using a very gentle initial transient, in which the pressure and 
  flow velocity history inside the tube were initialised to values close to the 
  intersection of the curve with the vertical axis. This means that the note 
  builds up very slowly when the mouth pressure is only just above the 
  threshold, as in the blue curve of Fig.\ 3. This can be seen very clearly in 
  Fig.\ 9, which shows the synthesised pressure inside the mouthpiece for the 
  three notes: the first note has a very long initial transient, much longer 
  than the other two notes. An artificial exponential decay to silence has been 
  imposed on each note, in order to separate them. You can hear the sequence of 
  three notes in Sound 1. (A side note: Fig.\ 3 is the first of many waveform 
  plots in this section and the next. To help avoid confusion if you flip back 
  and forth between these, every caption ends with a brief statement of which 
  model you are looking at.) 

  \fig{figs/fig-b1c1c56a.png}{\caption{Figure 9. Three synthesised notes on the 
  ``one-note clarinet'', using three different values of mouth pressure. The 
  three notes correspond to the blue, green and black curves of Fig. 3, in that 
  order. The exponential decay at the end of each note has been artificially 
  added in order to separate the notes clearly. [Idealised clarinet, 
  quasi-static reed model.]}} 

  \aud{auds/aud-9bdbfa20-plot.png}{\caption{Sound 1. The synthesised notes 
  shown in Fig. 9.}} 

  Waveform details cannot be seen at the resolution of Fig.\ 9, but the 
  envelope shape of each note is very clear. Figure 10 shows a zoomed view of 
  the periodic parts of the three waveforms, using the same colour code as the 
  corresponding curves in Fig.\ 3. It is immediately clear that all three 
  waveforms have a recognisable similarity to the square wave predicted by 
  Raman’s argument. It is also clear that the amplitude increases as the mouth 
  pressure increases, as suggested by the circle markers in Fig.\ 7. But the 
  waveforms do not only differ in amplitude: the blue curve, corresponding to 
  mouth pressure just above threshold, is much more rounded than the other two. 
  This more rounded appearance is associated with a smaller amplitude of the 
  higher harmonics, compared to the amplitude of the fundamental. The result is 
  a reduction of brightness of the sound, as you should be able to hear clearly 
  in the first note of Sound 1, compared to the other two. 

  \fig{figs/fig-275f3dfe.png}{\caption{Figure 10. Zoomed view of the steady 
  portion of the waveforms of the three notes from Fig. 9, using colours 
  corresponding to those in Figs. 3 and 7. [Idealised clarinet, quasi-static 
  reed model.]}} 

  We can easily perform a corresponding synthesis using the Raman model in 
  place of the realistic reflection function for the “one-note clarinet”. (The 
  previous link gives details of how this was done.) The results, in the same 
  format as we have just seen, are shown in Figs.\ 11 and 12, and Sound 2. 
  Comparing Fig.\ 11 with Fig.\ 9, the amplitudes and envelope shapes look 
  quite similar in the two cases. But the results certainly do not sound the 
  same, and we can immediately see why by comparing Fig.\ 12 with Fig.\ 10. The 
  only periodic waveform that can be produced by the Raman model is a perfectly 
  sharp square wave. So the three notes only differ in amplitude and length of 
  initial transient: the final periodic waveforms are all equally rich in 
  higher harmonics, and they sound identical apart from a difference in 
  loudness. 

  \fig{figs/fig-3e42e64c.png}{\caption{Figure 11. Synthesised notes using the 
  Raman model, with everything else identical to the results in Fig. 9. [Raman 
  model clarinet, quasi-static reed model.]}} 

  \aud{auds/aud-1b9de81c-plot.png}{\caption{Sound 2. The sound of the three 
  notes in Fig. 11.}} 

  \fig{figs/fig-48039a2a.png}{\caption{Figure 12. Zoomed view of the steady 
  portions of the three waveforms in Fig. 11, using the same colour coding as 
  in Fig. 10. [Raman model clarinet, quasi-static reed model.]}} 

  Next, we look at similar synthesised notes for the three variations of the 
  nonlinear valve characteristic plotted in Fig.\ 4. Each group of three notes 
  illustrates the green curve, the red curve, and the blue curve, in that 
  order. The results are plotted in Figs.\ 13—16, in the same format as the 
  ones we have just seen, using the realistic synthesis model and then the 
  Raman model. Again, you can hear the corresponding sounds of the pressure 
  waveforms inside the mouthpiece in Sounds 3 and 4. This time, the amplitudes 
  of the three notes in each group are very similar because the mouth pressure 
  is the same in all cases. For the realistic model, you can hear subtle 
  difference of tone quality, while for the Raman model the sounds are 
  virtually indistinguishable. 

  \fig{figs/fig-403f529b.png}{\caption{Figure 13. Three synthesised notes on 
  the ``one-note clarinet'', using three different variations of the nonlinear 
  valve characteristic of the mouthpiece. The three notes correspond to the 
  blue, red and blue curves of Fig. 4, in that order. The exponential decay at 
  the end of each note has been artificially added in order to separate the 
  notes clearly. [Idealised clarinet, quasi-static reed model.]}} 

  \aud{auds/aud-8c58caf3-plot.png}{\caption{Sound 3. The synthesised notes 
  shown in Fig. 13.}} 

  \fig{figs/fig-fe2a4a4b.png}{\caption{Figure 14. Zoomed view of the steady 
  portions of the three waveforms in Fig. 13, using the same colour code as in 
  Fig. 4. [Idealised clarinet, quasi-static reed model.]}} 

  \fig{figs/fig-12b1c201.png}{\caption{Figure 15. Synthesised notes using the 
  Raman model, with all other details the same as in Fig. 13. [Raman model 
  clarinet, quasi-static reed model.]}} 

  \aud{auds/aud-ed094ac1-plot.png}{\caption{Sound 4. The synthesised notes 
  shown in Fig. 15.}} 

  \fig{figs/fig-f9918133.png}{\caption{Figure 16. Zoomed view of the steady 
  portions of the three waveforms in Fig. 15, using the same colour code as in 
  Figs. 4 and 14. [Raman model clarinet, quasi-static reed model.]}} 

  \textbf{C. The pressure--gap diagram} 

  We have seen in the preceding plots, and heard in Sounds 1 and 3, that the 
  mouth pressure and the details of the valve characteristic both have audible 
  effects on the sound of our synthesised clarinet. We learn some interesting 
  things by taking a more systematic look at this. The mouth pressure is 
  obviously a key control parameter for the player, and we noted a bit earlier 
  that the “bite strength” might be another. The most obvious effect of a 
  player biting harder on the reed is to decrease the gap between the reed and 
  the lay. We can use these two physical parameters — mouth pressure and reed 
  gap — to make a plot that plays a somewhat similar role to the Schelleng 
  diagram for bowed strings. The first published version of a plot in this 
  form, as far as I have been able to find, is in a paper by Almeida, George, 
  Smith and Wolfe [2]. They used this format to present the results of 
  measurements on a real clarinet, using a test rig that allowed controllable 
  variation of pressure and bite force. 

  We can approach this in much the same spirit as our bowed-string studies back 
  in Chapter 9, by doing what the nonlinear dynamics folk sometimes call 
  “carpet bombing”. We can take a grid of points in the pressure—gap plane, 
  perform a simulation for each grid point, then analyse the results and plot 
  some interesting quantity in this plane, to see how it varies. Figure 17 
  shows an example. The horizontal axis shows mouth pressure, while the 
  vertical axis shows the gap. This vertical axis is being regarded as a 
  surrogate for the bite force, so the values of the gap have been plotted 
  decreasing upwards, so that the bite force increases upwards as in the plots 
  by Almeida et al. [2]. At the very top of the plot, the bite has got so hard 
  that the reed is closed completely. 

  \fig{figs/fig-8bf11a8e.png}{\caption{Figure 17. Pressure--gap diagram, 
  colour-shaded to show the frequency content of the final waveform for cases 
  where a note was produced, rather than silence. This example is based on the 
  ``one-note clarinet'' model, with the kind of gentle initial transient used 
  for Figs. 9, 11, 13 and 15. [Idealised clarinet, quasi-static reed model.]}} 

  Each simulation has first been analysed to test whether it produced “note” or 
  “silence”: black pixels mark points where there was no significant amplitude 
  of the pressure waveform at the end of the simulated time span. The coloured 
  pixels show where a note was obtained, and they have been coloured to 
  indicate something about the frequency content of the last few periods of the 
  pressure waveform. The scale, indicated in the colour bar, shows how high up 
  the series of harmonics you need to go before the level drops to 10~dB below 
  that of the ideal square wave. So bright colours indicate bright sound, and 
  dull reds indicate “mellow” or “muted” sound. (Ignore the various coloured 
  lines for the moment --- they will be explained shortly.) 

  The qualitative resemblance between Fig.\ 17 and the Schelleng diagram is 
  quite striking. In a plane with axes showing two key control variables for 
  the player, acceptable notes are only produced within a wedge-shaped region. 
  Within that wedge the high-frequency content, and hence the brightness of the 
  sound, varies significantly — albeit in a different pattern to the variation 
  within the Schelleng diagram. The two diagrams encapsulate important aspects 
  of violin and clarinet playing, which a beginner struggles to master and an 
  expert learns to use with great subtlety. 

  Figure 18 shows a different plot based on the same set of simulations. This 
  time, the colour shading indicates the frequency of the final waveform. The 
  variations are not large in this case, but there is a definite tendency for 
  the note to play progressively flat as conditions move down the middle of the 
  wedge towards the bottom right-hand corner. The measurements of real clarinet 
  behaviour by Almeida et al. [2] show a similar (but far stronger) tendency to 
  flattening, in the same region of the diagram. 

  \fig{figs/fig-774f5caa.png}{\caption{Figure 18. The same set of simulations 
  as in Fig. 17, now colour-shaded to indicate the frequency of the final 
  waveform. [Idealised clarinet, quasi-static reed model.]}} 

  The physical origin of this flattening phenomenon in our idealised example is 
  far from clear. The two effects we already know about, which might have 
  produced such an effect, are absent in this example. The resonances of our 
  cylindrical tube are accurately harmonic, apparently ruling out 
  “Benade-style” effects arising from collaborative regimes between inharmonic 
  frequencies. However, such effects might well have been present in the 
  measurements by Almeida et al. [2], potentially explaining why they saw a 
  much bigger effect. The second possible effect takes us back to the bowed 
  string. In section 9.2 we saw that when the straight line crosses the 
  friction curve in more than one place, the result is a hysteresis loop which 
  causes the pitch of the note to flatten (see especially Fig.\ 9 in that 
  section). But Fig.\ 2 in the present section shows that the corresponding 
  straight line is too steep to cross the nonlinear valve characteristic in 
  more than one place, so the effect cannot arise. 

  Naturally, we want to know what determines the shape and position of the 
  wedge-shaped region in these plots. We get a useful clue from Fig.\ 19, which 
  shows another pair of plots like Figs.\ 17 and 18. The only difference is 
  that this time each simulation was started with a more vigorous initial 
  transient, giving a “kick” to get oscillation started. We see that this has 
  expanded the region of possible notes. There is still a wedge-shaped region, 
  but the top edge of the wedge has moved upwards. Roughly speaking, the wedge 
  in Figs.\ 17 and 18 was confined between the lowest (magenta) line and the 
  solid cyan line, but in Fig.\ 19 it extended upwards towards the top line 
  (which is actually a solid red line and a dashed white line, very close 
  together). 

  \fig{figs/fig-5eaf3f3d.png}{} 

  \fig{figs/fig-a00d1b17.png}{} 

  Those coloured lines were not drawn on the plots simply in order to follow 
  the data. Instead, they represent thresholds between different types of 
  behaviour of the simple clarinet model, which have been calculated by 
  Dalmont, Gilbert, Kergomard and Ollivier on the basis of bifurcation studies 
  [3]. We can describe them all in words. The magenta line is the one we 
  already knew about, from the discussion in section 8.5. This is the threshold 
  of oscillation, which occurs when the mouth pressure is big enough for the 
  tangent slope on the vertical axis to change sign. The plotted line is simply 
  the condition for the maximum of the valve characteristic to lie in that 
  position: when there is energy dissipation, pixels that spontaneously 
  generate a note must lie at least a little above this line. 

  The dashed green line represents the condition Dalmont et al. [3] called the 
  “beating reed threshold”. Below this line, the reed remains open throughout 
  the vibration cycle, but above the line the reed is closed for part of the 
  time. The line is based on their analysis, calculated using the Raman model, 
  but it should give a good guide to the behaviour in the more realistic model, 
  as we will see shortly. 

  The solid cyan line represents the condition Dalmont et al. [3] called the 
  “inverse oscillation threshold”: it shows when the mouth pressure, for a 
  given gap, is just enough to close the reed completely against the lay. It is 
  easy to see why this line gives an upper limit in the case of a very gentle 
  initial transient, as in Figs.\ 17 and 18: beyond this line, the mouth 
  pressure is high enough that the reed is closed initially, and the gentle 
  transient means that the internal pressure is initialised with the same 
  pressure — so the reed stays closed for ever after, and no note is produced. 

  But with a more vigorous transient, as in Fig.\ 19, the reed has a chance to 
  open after the initial closure. For a pixel lying not too far above the cyan 
  line, it is possible for a note to get going, and then to be sustained. 
  However, for a pixel above the highest (red) line there is simply no possible 
  solution (at least within the simplification of the Raman model, which 
  Dalmont et al. [3] used to derive the condition). This line represents what 
  they call the “extinction threshold”. The dashed white line, lying very close 
  to the red line, is what they called the ``saturation threshold''. This is 
  the condition for the pressure amplitude to be a maximum, because (within the 
  Raman model) the pressure and volume flow rate when the reed is open fall 
  exactly at the peak of the nonlinear valve characteristic of Fig.\ 2. 

  In approximate summary, it may be possible to achieve a note anywhere between 
  the magenta and red/white lines, but above the cyan line it is also possible 
  to get silence — so the player has to use the right kind of articulatory 
  gesture to achieve a note. The space between the cyan and red/white lines is 
  dangerous territory where note production may feel unreliable, especially to 
  a beginner. Indeed, the advice for a beginner would be similar to the 
  corresponding advice to a violinist based on the Schelleng diagram: learn to 
  stay near the middle of the wedge-shaped region in order to get reliable 
  notes every time, and only explore towards the boundaries of the region when 
  you have built up a bit of experience. 

  It is instructive to see comparable pressure—gap diagrams based on the Raman 
  model. Figure 20 shows a pair of diagrams directly comparable to the ones in 
  Fig.\ 19. The region within which a note is achieved is rather similar in the 
  two cases, confined between the magenta and red/white lines as just 
  explained. But the figures are much less pretty with the Raman model! If a 
  note is achieved, it always has a square-wave pressure waveform, rich in 
  harmonics, and it always has exactly the same frequency. So the colour 
  shading is completely uniform in both plots. 

  \fig{figs/fig-7c08d9b1.png}{} 

  \fig{figs/fig-0934fea9.png}{} 

  Figure 21 shows Raman model simulations comparable to Figs.\ 17 and 18, with 
  a very gentle initial transient. Again, the region in which notes are found 
  is quite similar with both models, but now it stops at the cyan line for the 
  reason explained above. Towards the bottom of the plots, both models show 
  quite a few black pixels above the magenta line, more than in Figs.\ 19 and 
  20. The reason for this is simple: these pixels lie only just above the 
  threshold of excitation, and the initial transient is so gentle that the 
  growth is very slow. The pressure amplitude does not get sufficiently high by 
  the end of the simulations to qualify as a “note”. 

  \fig{figs/fig-5b745a6a.png}{} 

  \fig{figs/fig-86d7ab42.png}{} 

  Finally, Figure 22 shows what happens with the lossless version of the Raman 
  model. “Notes” now extend beyond the red/white line, all the way off the top 
  of the plot. This behaviour is exactly what the theory predicts (see 
  reference [2]). The red/white line here is entirely misleading: it has been 
  plotted in the same position as the other plots, to guide the eye when 
  comparing, but the two thresholds represented by these lines move “to 
  infinity” when there is no energy dissipation. There is no upper threshold, 
  exactly as the simulation results suggest. Notice, by the way, that the lower 
  limit of “notes” is slightly different than in Fig.\ 20. Non-black pixels now 
  extend right down to the magenta line, rather than stopping just above it. 
  That is exactly what we expect from the original discussion of the threshold 
  of oscillation: with no losses, the threshold lies at the peak of the 
  nonlinear characteristic, rather than just beyond the peak. 

  \fig{figs/fig-b2562c00.png}{} 

  \fig{figs/fig-8c6bb2f6.png}{} 

  \textbf{D. Transients of the clarinet model} 

  Up to now, as we have developed the clarinet model we have been finding 
  surprisingly close analogies with the behaviour of a bowed string. But when 
  we turn to examine transient behaviour, we will find a strong contrast. 
  Figure 23 shows the transient waveforms of pressure and volume flow rate, for 
  one of the cases used to make up Figs.\ 17 and 18. Specifically, this case 
  has mouth pressure 3~kPa and reed gap 0.4~mm. Comparing with Fig.\ 17 reveals 
  that this puts it in the middle of the wedge region, above the beating reed 
  threshold (dashed green line). This is a case with a gentle initial 
  transient, and it can be seen that the pressure variations start small, and 
  grow quite slowly. But after a few period-lengths it settles to a periodic 
  waveform looking recognisably like a square wave. 

  \fig{figs/fig-57a0a56b.png}{\caption{Figure 23. A clarinet transient from the 
  set used for Figs. 17 and 18, with mouth pressure 3~kPa and reed gap 0.4~mm. 
  [Idealised clarinet, quasi-static reed model.]}} 

  Figure 24 shows the same information, plotted (in red) on top of the 
  nonlinear valve characteristic. This shows what part of that characteristic 
  is used by this particular note. It can be seen that the red line extends 
  only a rather short distance along the flat portion, where the reed is 
  closed. This fits in with what we learned by looking at Fig.\ 17: we are 
  above the beating reed threshold, but not very far above it. 

  \fig{figs/fig-e7782555.png}{\caption{Figure 24. The transient from Fig. 23, 
  showing the region of the nonlinear valve characteristic that it occupies. 
  [Idealised clarinet, quasi-static reed model.]}} 

  Now compare Fig.\ 23 with Figs.\ 25, which shows the corresponding transient 
  computed with the Raman model. The pressure waveform is very square, as we 
  expect from the Raman model, but it shows a very close resemblance to the 
  pressure waveform in Fig.\ 23. The waveforms of volume flow rate look more 
  different at first sight, but actually they are about as similar as they 
  could be, as I will now explain. 

  \fig{figs/fig-0d3db5c4.png}{\caption{Figure 25. The same transient as in Fig. 
  23, simulated using the Raman model. [Raman model clarinet, quasi-static reed 
  model.]}} 

  The Raman model jumps abruptly between different vales of pressure and volume 
  flow rate, whereas in the more realistic model they both vary smoothly and 
  continuously. The result is made particularly clear in Fig.\ 26, which is the 
  equivalent of Fig.\ 24: the points on the nonlinear characteristic are 
  sparse, rather than forming a solid red line. By the end of this portion of 
  transient, the Raman model is simply alternating between two points, which 
  are the leftmost and rightmost of the red stars in Fig.\ 26. This means that 
  the waveform of volume flow in Fig.\ 25 is a square wave, whereas the 
  corresponding waveform in Fig.\ 23 has to go up and over the hump of the 
  curve each time the pressure switches from high to low. This produces a pair 
  of “rabbit ears” in each cycle of the blue waveform in Fig.\ 23. If you 
  ignore those, the waveform looks much more recognisably like the one in Fig.\ 
  25. 

  \fig{figs/fig-5025d8ac.png}{\caption{Figure 26. The Raman model transient 
  from Fig. 25, superimposed on the nonlinear valve characteristic. The final 
  periodic state corresponds to alternation between the rightmost and leftmost 
  pair of points. [Raman model clarinet, quasi-static reed model.]}} 

  These examples demonstrate two things that appear to be typical: the 
  transients of this model clarinet are always rather “simple”, and the Raman 
  model usually matches the more realistic model quite well. Both these things 
  are far from being true for bowed-string transients! Bowed-string transients 
  are usually complicated, and the details are highly sensitive to everything 
  you can think of — the player’s gesture, the details of the model of the 
  linear system (string and instrument body), and also the details of the 
  nonlinear model used to describe the friction force. In particular, switching 
  to Raman's model completely changes everything. 

  As a reminder of some of the discussion of this issue from sections 9.5—9.7, 
  Fig.\ 27 shows a repeat of Fig.\ 6 from section 9.5. This shows three 
  measured transients, differing only in the normal force between bow and 
  string. The middle plot (in red) is the only one that echoes the simplicity 
  of the clarinet transients. The upper plot (black) shows a “scratchy” 
  transient featuring irregular string motion before eventually settling into a 
  periodic waveform; the lower one (blue) shows a transient that leads to an 
  entirely different periodic regime of string vibration (“double-slipping 
  motion”). Neither of these phenomena has any obvious parallel within the 
  clarinet model, and it seems that there is no need to look for an analogue of 
  the Guettler diagram (see section 9.5). 

  \fig{figs/fig-9cdc4aff.png}{\caption{Figure 27. Three measured transients of 
  a bowed string, reproduced from Fig. 5 of section 9.5.}} 

  What makes the clarinet so different from a bowed string in terms of 
  transient behaviour? There are three factors that probably contribute. First 
  is the nature of the nonlinearity. The reed valve characteristic is more 
  ``benign'' than the rather vigorously nonlinear friction curve. However, we 
  saw in section 9.6 that the friction-curve model is probably not realistic. A 
  different model is needed to give a satisfactory account of the frictional 
  behaviour of violin rosin, and the thermal models that were discussed in that 
  section behave in a more benign way than the original friction-curve model. 
  So it is not clear how important this factor might be for real bowed-string 
  transients. 

  The second factor is a disparity of damping between a clarinet tube and a 
  violin string. The Q-factor of the fundamental mode of our model clarinet 
  tube is 26, whereas the Q factor of the fundamental of a typical 
  finger-stopped violin string is of the order of 300. Damping has a 
  stabilising influence on self-excited vibrations, and this factor of 10 
  difference may well help to make a bowed violin string more “twitchy” than 
  the clarinet. 

  But the third factor is probably the most important. The clue to this lies in 
  Raman’s argument. Raman originally formulated this argument in response to 
  published measurements showing a range of possible vibration waveforms of a 
  bowed string. In the simplest version of his argument, he imagined the bowed 
  point dividing the string length in a ratio $1:n$, where $n$ is a whole 
  number. Any possible periodic motion of the string, within his idealised 
  model, will then have $n$ segments of constant velocity per period. The 
  argument leading to Fig.\ 8 above means that the velocity within each of 
  these segments can only take one of two possible values. Raman used this as 
  the basis for a classification scheme for all possible motions of a 
  (lossless) bowed string. He showed that there are indeed many possibilities. 

  But our clarinet model corresponds to bowing a string at its mid-point. In 
  the lossless case, each cycle of a periodic motion at the fundamental 
  frequency can then only have two segments of velocity. So there is only one 
  possible way to have a two-velocity solution: it must be the kind of square 
  wave we have been looking at. The lossless clarinet, or the mid-point bowed 
  string, simply does not allow other periodic solutions (unless they have 
  different periods, for example following a “period-doubling bifurcation” to 
  give a note an octave lower). So one major aspect of the discussion of 
  playability of bowed-string transients does not arise for the clarinet. We do 
  not need to ask whether a given gesture leads to Helmholtz motion or to some 
  other periodic regime like double-slipping motion, because there are no such 
  alternative regimes! 

  \textbf{E. Influence of the reed resonance} 

  We haven’t finished with clarinet models yet. So far, everything has been 
  based on the simplest mouthpiece model, in which the reed moved as if it was 
  a simple spring. But of course the reed has its own resonance frequencies, 
  and we already showed (back in section 11.3.1) how to incorporate the first 
  reed resonance into a mouthpiece model. It is easy enough to incorporate this 
  extra ingredient into a computer model, so we can look at some simulated 
  results. For all the results to be shown here, the reed resonance frequency 
  is chosen to be 2.8~kHz, based on a suggestion by Chaigne and Kergomard [1] 
  (see section 9.2.2.2). The only other new parameter we need is the damping of 
  the reed resonance. It is far from clear what value this damping should have, 
  so we will look at results for several different values of the reed Q-factor. 
  In practice, the reed damping is no doubt influenced by contact with the 
  player's lip, so it may vary with changes of embouchure. 

  Figure 28 gives a first example of a transient simulated with this new model. 
  It is presented in the same format as Figs.\ 23 and 25, with the mouthpiece 
  pressure in the top graph and volume flow rate in the lower graph. We will 
  explain shortly which precise transient this is: for the moment, we just want 
  to note some qualitative things about it. The pressure waveform looks quite 
  similar to the one in Fig.\ 23, tending quite quickly to a slightly 
  rounded-off square wave. But in the first few cycles you can see the effect 
  of the reed resonance “ringing on” briefly after each jump in pressure. 

  \fig{figs/fig-3b4f20fb.png}{\caption{Figure 28. A typical transient from the 
  clarinet model, allowing for the reed resonance. [Idealised clarinet, dynamic 
  reed model.]}} 

  The lower graph in Fig.\ 28 looks a little different from its counterpart in 
  Fig.\ 23, because the pairs of “rabbit ears” in the final periodic motion are 
  no longer symmetrical: the second “ear” is much bigger than the first one. A 
  different view of this phenomenon is given by Fig.\ 29, which shows the same 
  data plotted in the pressure/flow-rate plane. The nonlinear valve 
  characteristic that we used in the earlier models is also shown, in blue. 
  Rather obviously, the new model shows a loop above and below this blue curve: 
  it rarely follows the curve. 

  \fig{figs/fig-b88adf8b.png}{\caption{Figure 29. The final third of the 
  transient from Fig. 28, plotted in red stars on top of the quasi-static 
  nonlinear reed characteristic that we have seen before. [Idealised clarinet, 
  dynamic reed model.]}} 

  The explanation lies in the damping of the reed resonance. When the pressure 
  jumps abruptly in the Helmholtz-like square wave, the reed no longer responds 
  immediately, like the quasi-static spring model we used earlier. Instead, the 
  damping imposes a slight lag. When the reed is open before the pressure jump, 
  it stays open a little longer than before, so that the air flow through the 
  gap is bigger. But if the reed is closed before the pressure jump, it stays 
  closed a little longer and the air flow is reduced. The result is a 
  hysteresis loop, followed in an anti-clockwise direction in this plot. This 
  loop is reminiscent of the hysteresis loop we met back in section 9.2, for 
  the frictional behaviour of a bowed string. In the bowed string, this effect 
  was responsible for the ``flattening effect'', so it should not come as a 
  surprise that we will see shortly that it causes the pitch of our clarinet to 
  fall: the note plays a little flat. 

  To complete the set of plots of this example transient, Fig.\ 30 shows the 
  corresponding displacement of the reed. Positive values here correspond to 
  the reed moving towards the lay, and when the value reaches 0.6~mm the reed 
  closes — because that was the choice of initial gap used in this particular 
  example. Throughout the plot you can see some evidence of transient ringing 
  of the reed resonance, whenever the displacement changes abruptly in response 
  to a pressure jump in the mouthpiece. 

  \fig{figs/fig-4767df83.png}{\caption{Figure 30. The reed displacement $y(t)$ 
  from the transient of Fig. 28. [Idealised clarinet, dynamic reed model.]}} 

  The natural next step is to use the new model to generate pressure-gap 
  diagrams, to reveal the behaviour over a region of parameter space. Figures 
  31, 32 and 33 show three examples that are directly comparable with Fig.\ 19. 
  They use three different Q-factors for the reed: 1.25, 2.5 and 5 
  respectively. These may all seem very low values of Q (or very high damping), 
  but remember that the player's lip is in contact with the reed, and flesh has 
  very high damping. All details of the plots are the same as Fig.\ 19, except 
  that the range of behaviour encoded by the colours is different. The 
  left-hand plot of each pair, representing the frequency content via the 
  harmonic number that first falls 10~dB below the ideal square wave, has a 
  scale that runs up to 25, compared to 50 in Fig.\ 19. All the waveforms from 
  the new model tend to be less rich in higher harmonics: we will see an 
  explicit spectrum plot in a moment. 

  \fig{figs/fig-cc93e883.png}{} 

  \fig{figs/fig-8234ad81.png}{} 

  \fig{figs/fig-97cb6e3a.png}{} 

  \fig{figs/fig-16803032.png}{} 

  \fig{figs/fig-fa7d3ca4.png}{} 

  \fig{figs/fig-600c1dd3.png}{} 

  The right-hand image of each pair is colour-shaded to show the frequency of 
  the final periodic waveform. In this case, the scale represented by the 
  colours is longer than in Fig.\ 19. The right-hand plot of Fig.\ 19 did show 
  a slight tendency to flattening towards the bottom right-hand corner, but the 
  total range of frequencies covered by that plot was 128.8—129.1~Hz, a total 
  range of about 1/4~Hz. The corresponding range for the new plots (all three 
  give a similar range) is 127.9—129.0~Hz, a range of 1.1~Hz. 

  To get an idea of what lies behind these plots, and how the new results 
  relate to the earlier model, Fig.\ 34 shows the first few period-lengths of a 
  particular transient for all the models. Specifically, it is the one in the 
  bottom right-hand corner of the pressure-gap diagrams, marked by a blue 
  circle in the right-hand plot of Fig.\ 33. The black curve is the result from 
  the earlier model, where the reed was treated as a spring. The other three 
  are drawn from the sets used to generate Figs.\ 31—33: the green curve has 
  the reed’s $Q=1.25$, the blue curve has $Q=2.5$ and the red curve has $Q=5$. 
  The changing damping of the reed resonance is immediately apparent in the 
  extent of the “ringing” after each pressure jump. It is reassuring to see 
  that the blue and red curves track quite close to the black curve --- this is 
  a useful check on the accuracy of coding of the new computer model. The blue 
  curve here shows exactly the same transient used to generate Figs.\ 28, 29 
  and 30 earlier. 

  To get an idea of what lies behind these plots, and how the new results 
  relate to the earlier model, Fig.\ 34 shows the first few period-lengths of a 
  particular transient for all the models. Specifically, it is the one in the 
  bottom right-hand corner of the pressure-gap diagrams, marked by a blue 
  circle in the right plot of Fig.\ 33. The black curve is the result from the 
  earlier model, where the reed was treated as a spring. The other three are 
  drawn from the sets used to generate Figs.\ 31—33: the green curve has the 
  reed’s $Q=1.25$, the blue curve has $Q=2.5$ and the red curve has $Q=5$. The 
  changing damping of the reed resonance is immediately apparent in the extent 
  of the “ringing” after each pressure jump. It is reassuring that the blue and 
  red curves lie quite close to the black curve. 

  \fig{figs/fig-27ce5471.png}{\caption{Figure 34. The first few period-lengths 
  of the transient lying at the bottom right-hand corner of Figs. 19, 31, 32 
  and 33, corresponding to a reed gap of 0.6~mm and a mouth pressure 4~kPa. The 
  black curve is for the old model without the reed resonance. The other three 
  are from the new model, for the three cases shown in Figs. 31--33: green has 
  $Q=1.25$, blue has $Q=2.5$ and red has $Q=5$. [Idealised clarinet, dynamic 
  reed model.]}} 

  Even on the short time-scale of this plot, you can see the flattening effect 
  developing. All four pressure waveforms are synchronised at the start, but by 
  the right-hand side the black curve (from the old model, with less 
  flattening) has already got slightly ahead of the three coloured curves. The 
  effect is easiest to see by comparing the near-vertical lines where the 
  pressure jumps up or down. 

  All four of these transients led to a periodic waveform of pressure when the 
  simulation was run for longer. Taking a chunk of this eventual waveform and 
  applying the FFT, we can easily generate corresponding frequency spectra. The 
  results are shown in Fig.\ 35, with the same colour code as in Fig.\ 34. The 
  curves have been separated vertically for clarity, with the old model at the 
  top, then the three cases of the new model in descending order of Q-factor. 

  \fig{figs/fig-0b5202c7.png}{\caption{Figure 35. Frequency spectra of the 
  eventual periodic pressure waveforms following the four transients shown in 
  Fig. 34. The colour code is the same as in Fig. 34. The curves have been 
  separated by 60~dB for clarity. The blue curve (for $Q=2.5$) is marked with 
  red stars to indicate the first few odd-numbered harmonics, and green stars 
  for the corresponding even-number harmonics. [Idealised clarinet, dynamic 
  reed model.]}} 

  It is clear that the reed damping is having a significant effect on the 
  frequency content of the note. With the old model (black curve) sharp peaks 
  at the harmonics extend over the whole range of this plot. But with the reed 
  resonance included in the model, the peak amplitudes die away with rising 
  frequency — slowly with $Q=5$ (red curve), faster with $Q=2.5$ (blue curve), 
  and very fast with $Q=1.25$ (green curve). This effect is quite audible, 
  especially with the lowest Q-factor. Figure 36 shows a sequence of three 
  notes from these simulations including the reed resonance, combined in the 
  same way as in Figs.\ 9, 11, 13 and 15 (with an artificial exponential decay 
  imposed to separate the notes). The sequence has $Q=1.25$ first, then 
  $Q=2.5$, then $Q=5$. You can hear the resulting sound in Sound 5. 

  \fig{figs/fig-afde75ad.png}{\caption{Figure 36. A sequence of three notes, 
  combining the three simulations from Figs. 34 and 35 which include the reed 
  resonance. The sequence is $Q=1.25$, then $Q=2.5$, then $Q=5$. [Idealised 
  clarinet, dynamic reed model.]}} 

  \aud{auds/aud-eff63229-plot.png}{\caption{Sound 5. Audio demonstration of the 
  three notes from Fig. 36.}} 

  Returning to Fig.\ 35, there is another feature we should mention. Our model 
  clarinet tube is closed at one end and open at the other, so that the 
  resonance frequencies match odd-numbered harmonics 1, 3, 5, 7… only. The tube 
  has no resonances at the even-numbered harmonics. However, the nonlinear 
  action of the reed valve can generate all harmonics, not just the 
  odd-numbered ones. So the internal spectrum of the pressure, like the 
  examples shown in Fig.\ 35, may contain even as well as odd harmonics. This 
  is indeed what the plot shows. As a guide to the eye for identifying these 
  harmonics, the blue curve has been annotated with red stars indicating the 
  frequencies of the first few odd-numbered harmonics, and green stars marking 
  the corresponding even harmonics. 

  All four curves are dominated by odd harmonics at low frequency, but at 
  higher frequencies the even harmonics become increasingly visible. But this 
  plot shows the spectrum of the internal pressure, so the odd harmonics are 
  supported and boosted by the tube resonances. The sound of the instrument, 
  based on the external sound pressure radiated from the open end of the tube, 
  may be significantly different. As Benade pointed out many years ago, the 
  even-numbered harmonics correspond to antiresonances of the tube, not to 
  resonances. At those frequencies the pressure is a maximum near the open end 
  (or an open tone-hole), not a minimum as it is for the resonant frequencies. 

  The result is that any component of the internal pressure corresponding to an 
  even harmonic may be radiated much more efficiently than the corresponding 
  components from the odd harmonics. This effect goes some way to compensating 
  for the even-odd pattern of harmonic amplitudes: the sound of a clarinet may 
  contain significant levels of some even harmonics. This is indeed found to be 
  the case when a clarinet sound spectrum is measured: Fig.\ 37 shows just such 
  a measurement, of the lowest note of a clarinet with all holes closed, played 
  forte. It is taken from Joe Wolfe's \tt{}web site\rm{}. At the lowest 
  frequencies, the odd-numbered harmonic peaks are higher than their 
  neighbouring even-numbered peaks, but this even-odd pattern rapidly 
  disappears as you move up in frequency. 

  \fig{figs/fig-a254e30e.png}{\caption{Figure 37. Measured sound spectrum of 
  the lowest note of a clarinet, with all tone-holes closed. Image taken from 
  https://newt.phys.unsw.edu.au/music/clarinet/E3.html, reproduced by 
  permission of Joe Wolfe.}} 

  Looking all the way back to Fig.\ 33, there is another feature that we should 
  comment on. Along the lower edge of the wedge-shaped region of coloured 
  pixels is a small wedge of white pixels. One of these is highlighted by a 
  green circle. These white pixels represent what a clarinettist would call 
  “squeaks”: instead of eliciting the expected tone based on the tube 
  resonances, you sometimes get a much higher, and discordant, frequency. This 
  has more to do with the resonance of the reed than with those of the tube. 
  Indeed, you can make a squeak with a bare mouthpiece, detached from the tube. 

  A region in the pressure-gap diagram corresponding to squeaks rather than 
  notes was found in the measurements by Almeida et al. [2]. It is rather 
  encouraging that the simulation model produces a similar prediction, in 
  roughly the same region of the diagram. If we take the transient marked by 
  the green circle and assemble three notes in the same manner as Fig.\ 36 and 
  Sound 5, we can hear the effect in action. The same sequence of reed 
  Q-factors is used, and the assembled waveform is plotted in Fig.\ 38. You can 
  hear the result in Sound 6. You will notice that the first note, with 
  $Q=1.25$, shows a decaying transient rather than a sustained note --- you 
  just hear a kind of ``thump''. The second note, with $Q=2.5$, gives a fairly 
  normal clarinet-like tone, while the third one, with $Q=5$, gives the squeak. 

  \fig{figs/fig-ce4ff3f1.png}{\caption{Figure 38. A sequence of three notes in 
  the same format as Fig. 36, for the three transients at the position marked 
  by a green circle in Fig. 33. The sequence is $Q=1.25$, then $Q=2.5$, then 
  $Q=5$. The first gives no note, just a transient decay. The second give a 
  normal note, the third gives a squeak. [Idealised clarinet, dynamic reed 
  model.]}} 

  \aud{auds/aud-34f3dca1-plot.png}{\caption{Sound 6. Audio demonstration of the 
  three notes from Fig. 38.}} 

  \textbf{F. Conical reed instruments} 

  So far, all our effort has been devoted to the model clarinet, based on a 
  cylindrical tube. But many reed instruments are based on tapered, conical 
  tubes — the saxophone, the oboe and the bassoon, for example. We already know 
  about one key difference between a conical tube and a cylindrical tube: we 
  saw in section 4.2 that the resonance frequencies of a conical tube, tapering 
  down to point, form a complete harmonic series. By contrast, a closed 
  cylindrical tube like our clarinet model has resonance frequencies that are 
  odd-numbered harmonics only. 

  We will study one idealised example of a conical reed instrument, to compare 
  with the clarinet model we have already discussed. This will be based on a 
  soprano saxophone, which has a conical tube somewhat longer than a clarinet. 
  The lowest note, with all tone-holes closed, is a lot higher than the 
  corresponding note of the clarinet, though, because the lowest mode has a 
  half-wavelength in the length of the tube, whereas a clarinet has a 
  quarter-wavelength. 

  Our model will use a similar level of simplification to the clarinet model: 
  the aim in both cases is to bring out the essential physics of the two types 
  of instrument with minimal complication, and draw some conclusions about the 
  behaviour as a player might perceive it. Figure 39 shows a soprano saxophone, 
  and immediately below it is the first stage of idealisation. With all holes 
  closed, and ignoring the bell, we can approximate the internal bore shape by 
  a straight-sided cone. But of course the real instrument does not taper all 
  the way to a point. Instead, it is truncated and terminated by a mouthpiece: 
  a small cavity carrying a flexible reed rather similar to the clarinet reed. 

  \fig{figs/fig-54327c0c.png}{\caption{Figure 39. A soprano saxophone, with 
  sketches of various stages of idealisation. Saxophone image copyright Joe 
  Wolfe, reproduced by permission.}} 

  The third image in Fig.\ 39 shows the truncated cone, without the mouthpiece. 
  It also shows the “completion” of the cone in a dotted red line, extending 
  some way beyond the physical length of the instrument. The input impedance of 
  a truncated cone like this can be calculated, as explained in the next link. 
  The answer, slightly unexpectedly, involves the length of the “missing” 
  portion of the cone as well as the physical length of the tube. 

  The resulting formula for the impedance tells us several useful things. Its 
  inverse, the input admittance, has resonance peaks at exactly the same 
  frequencies as a cylinder of the same length. However, these are the 
  resonance frequencies that would occur with both ends of the tube open. For 
  our saxophone model the reed and mouthpiece will be like a closed end, just 
  as in the clarinet model. The resonances are then given by the peaks of 
  impedance, not of admittance. As we will see in some detail shortly, those 
  peaks are not harmonically spaced — potentially bad news for our saxophone. 

  The final image in Fig.\ 39 shows a rather unexpected approximation to the 
  acoustic behaviour, known as the “cylindrical saxophone” [1,4] and explained 
  in the previous link. When the frequency is sufficiently low that the 
  wavelength of sound is much bigger than the length of the missing piece of 
  cone, the formula for the impedance becomes the same as the impedance of a 
  cylindrical tube with the same length as the completed cone, open at both 
  ends and excited at the position where the actual cone was truncated. So we 
  can imagine a saxophone mouthpiece plugged in to the side-wall of the 
  cylindrical tube, as sketched in the figure. This cylindrical approximation 
  is not accurate for the saxophone at higher frequencies, and we won’t rely on 
  it when we come to run simulations. But it is illuminating because it tells 
  us something qualitative about how a saxophone might behave. (Indeed, you can 
  buy a real musical instrument based, rather loosely, on the idea of the 
  cylindrical saxophone: the \tt{}Yamaha Venova\rm{}.) 

  The important thing for our immediate purpose is that the “cylindrical 
  saxophone” frequency response applies equally well to a string, bowed at the 
  same intermediate position. This allows us to extend the analogy between wind 
  instruments and a bowed string, to incorporate the main ingredient that was 
  missing when we talked about the clarinet. In the clarinet case, the 
  equivalence was to the very special case of a string bowed at its mid-point. 
  But the new analogy suggests that a saxophone (or an oboe or a bassoon) might 
  behave a bit like a string bowed at a more normal position, relatively close 
  to one end. 

  Immediately, we can make some guesses. Based on what we already know about 
  bowed strings, the analogy suggests that the saxophone model might behave 
  very differently from the clarinet model. First, the equivalent of the 
  Helmholtz motion will no longer be a square wave of pressure. Instead, we 
  expect something resembling the velocity waveform of a bowed string: a pulse 
  wave, with the pulse occupying a proportion of each period determined by the 
  position (as a fraction of the total tube length) of the “mouthpiece” in the 
  bottom sketch of Fig.\ 39. 

  But something more fundamental will also change. For the clarinet, or the 
  mid-point bowed string, we argued (following Raman) that the square wave was 
  the only possible periodic waveform at the natural period of the tube or 
  string. But now we have a “bowing point” that divides the length unequally — 
  for the numerical cases we will see shortly, this division is roughly in the 
  ratio 1:8. This opens the floodgates to a wealth of alternative waveforms, 
  just as Raman found for a bowed string. So we might expect to see far more 
  complicated behaviour of the saxophone model, compared to the clarinet model. 
  We will shortly see some simulation results, and we will be on the lookout 
  for behaviour that reminds us of the response of a bowed string, such as the 
  appearance of periodic waveforms other than the Helmholtz motion. 

  There are a couple more details to mention, to complete a model of the actual 
  (non-cylindrical) saxophone that we can use for simulation studies. One of 
  these concerns the mouthpiece cavity, indicated by the sketch immediately 
  below the photograph in Fig.\ 39. We have already mentioned that the 
  impedance peaks for the truncated cone are not distributed according to an 
  accurate harmonic series. But, at least for the first few modes, we can 
  improve things a little by including the mouthpiece cavity in the model, and 
  making a careful choice of its volume. 

  The approximation again works best when the wavelength of sound is very long 
  compared to the missing portion of the cone. If the cone had been complete, 
  then of course the air trapped in that final section would have been rigidly 
  enclosed. We already know how such a confined volume of air behaves — we 
  thought about it when looking at the Helmholtz resonator back in section 4.2. 
  The trapped air behaves like a simple spring, and its stiffness depends only 
  on the total volume, not on the shape. We also know that the complete cone 
  would have harmonically-spaced resonances, so if we choose a mouthpiece 
  cavity with the same volume as the “lost” section of cone we should find that 
  at least the first few resonances still have approximately harmonic spacing. 
  Sure enough, measurements of the volumes of preferred mouthpieces for a range 
  of conical reed instruments (including some allowance for the extra 
  ``effective volume'' arising from the reed's flexibility) conform quite well 
  to this predicted pattern, matching the volume lost by truncating the cone. 

  The final ingredient for a simulation model is the reed. But this one is easy 
  — a soprano saxophone reed and mouthpiece are not very different from those 
  of a clarinet, so we can use the same model with minor adjustments to 
  parameter values. The details of the values used here are given in the 
  previous link, but I have used some guesswork in choosing them: published 
  data on saxophone reeds is in far shorter supply than on clarinet reeds. But 
  this should not matter very much, since the aim here is to bring out 
  qualitative behaviour, not to match quantitatively the behaviour of any 
  particular saxophone — that is a task for future research. 

  So let us look at some simulation results. Figure 40 shows a successful 
  transient, in the same format as Figs.\ 23 and 28: pressure inside the 
  mouthpiece in the upper plot, volume flow rate through the reed in the lower 
  one. The pressure waveform rapidly settles to a periodic state, and it is 
  indeed a pulse wave as we anticipated from the “cylindrical saxophone” 
  argument. The lower plot reveals that the reed closes once per cycle, for a 
  rather short time that more or less matches the length of the pressure pulse. 

  \fig{figs/fig-3741b03b.png}{\caption{Figure 40. A transient of the saxophone 
  model, corresponding to the 6th pixel in the row of Fig. 43 marked with a 
  green line. Upper plot: pressure inside the mouthpiece; lower plot: volume 
  flow rate through the reed. [Idealised saxophone, dynamic reed model.]}} 

  To see the other “cylindrical saxophone” prediction in action, we only have 
  to change the mouth pressure a little. Figure 41 shows a transient with a 
  slightly increased mouth pressure, and Fig.\ 42 shows the result after a 
  further small increase. You can hear these three transients in Sound 7. The 
  transient in Fig.\ 41 shows what in violin terms we would call “double 
  slipping” or “surface sound”. The waveform settles to a periodic state 
  involving two pulses per period, and two corresponding short episodes when 
  the reed closes. But the sound is at the same pitch as the Fig.\ 40 note, not 
  an octave higher. The reason is that the two pulses are not equally spaced in 
  the cycle, so the periodicity of the note is unchanged. Pressure waveforms 
  looking very much like this have been reported from real saxophones [5]. 

  \fig{figs/fig-a1d83878.png}{\caption{Figure 41. A saxophone model transient 
  in the same format as Fig. 40, corresponding to the 8th pixel in the row of 
  Fig. 43 marked with a green line. [Idealised saxophone, dynamic reed 
  model.]}} 

  \fig{figs/fig-aebd86eb.png}{\caption{Figure 42. A saxophone model transient 
  in the same format as Figs. 40 and 41, corresponding to the 9th pixel in the 
  row of Fig. 43 marked with a green line. [Idealised saxophone, dynamic reed 
  model.]}} 

  \aud{auds/aud-b3a38f31-plot.png}{\caption{Sound 7. The three simulated 
  saxophone notes from Figs. 40, 41 and 42, assembled in a similar format to 
  the earlier sound demonstrations in this section.}} 

  But in Fig.\ 42 we see a different pattern. This time, the early part of the 
  transient waveform shows groups of four pressure pulses (a “quadruple slip” 
  motion in bowed-string terms). But by the end of the plot, two of these 
  pulses have faded away and disappeared, and the remaining two are equally 
  spaced. The result is that the final note sounds an octave higher, at least 
  by the end. In the early part of this note, the third note in Sound 7, you 
  can still hear a trace of the original pitch before it moves up to the 
  octave. 

  The natural next step is to compute a pressure-gap diagram for the model 
  saxophone: the result is shown in Fig.\ 43. I will explain the colour shading 
  in a moment, but first we can note the wedge-shaped region of coloured 
  pixels: the saxophone has a region of “note” rather than “silence” that looks 
  broadly similar to what we already saw for the clarinet. The magenta and cyan 
  lines have the same meanings as in previous pictures: they mark the threshold 
  of vibration, and the threshold beyond which it could be possible for the 
  reed to remain permanently closed. But the other thresholds shown in earlier 
  pressure-gap diagrams are omitted because they have no direct counterpart 
  here: they were derived from calculations with the Raman model, and a full 
  set of equivalent conditions for the saxophone model has yet to be derived 
  (but see section 9.4.8.3 of Chaigne and Kergomard [1] for some efforts in 
  that direction). The horizontal green line in Fig.\ 43 indicates the row from 
  which the transients of Figs.\ 40—42 were drawn. Counting from the left-hand 
  side, these transients correspond to pixels numbers 6, 8 and 9. Pixel 6, 
  corresponding to Fig.\ 40, is the first coloured pixel beyond the magenta 
  threshold line. 

  \fig{figs/fig-da16d508.png}{\caption{Figure 43. Pressure-gap diagram for the 
  saxophone model, with the same kind of vigorous initial transient used for 
  Fig. 19 in the case of the clarinet. The green line marks the row containing 
  the transients shown in Figs. 40, 41 and 42. The white lines mark cases 
  illustrated in Fig. 44. The colour indicates the frequency of the final 
  waveform as a multiple of the nominal frequency, based on analysis of 
  autocorrelation. [Idealised saxophone, dynamic reed model.]}} 

  To relate this to the waveforms we have seen, I need to explain the colour 
  shading. First, we should look at a bigger set of waveforms. Figure 44 shows 
  the full set of waveforms for the coloured pixels lying along the three white 
  lines in Fig.\ 43. Three nominal period-lengths are plotted, from the end of 
  each individual simulation. For each of the three sets, the top line shows a 
  pulse waveform similar to the one in Fig.\ 40 (it just looks a little 
  different because the vertical scale has been squashed). As you go down each 
  stack of waveforms, the pulse waveform develops into shapes with more pulses. 
  But before you reach the bottom of each stack you see a return to pulse 
  waveforms — but upside down compared to the ones at the top of the stacks. 

  \fig{figs/fig-756d8db3.png}{\caption{Figure 44. Some sample waveforms from 
  the simulations lying behind Fig. 43. The three columns correspond to all the 
  coloured pixels in the three rows marked with white lines. Left-to-right rows 
  in Fig. 43 become top-to-bottom stacks here, so that mouth pressure increases 
  as you move down each column. The last three period-lengths of each pressure 
  waveform are plotted. [Idealised saxophone, dynamic reed model.]}} 

  A more detailed example of one of these inverted pulse waveforms is shown in 
  Fig.\ 45: this is from the right-most coloured pixel along the green line in 
  Fig.\ 43. You can see what has happened: the reed opens and closes once per 
  cycle as before, but now it is closed nearly all the time, opening only for a 
  short pulse. This is the opposite behaviour to the case in Fig.\ 40. 

  \fig{figs/fig-0a8559a9.png}{\caption{Figure 45. A saxophone model transient 
  in the same format as Figs. 40--42, corresponding to the 27th pixel in the 
  row of Fig. 43 marked with a green line. [Idealised saxophone, dynamic reed 
  model.]}} 

  With these waveforms in mind, the colour shading of Fig.\ 43 can be 
  explained. The last few period-lengths of each simulation were used to 
  compute something called the autocorrelation function. The signal is 
  multiplied, sample by sample, with another version of the same signal with a 
  time lag, and the products are all added together. After normalising to allow 
  for the absolute magnitude of the signal, the result is a number between $-1$ 
  and $1$, and this number varies as the lag is changed. If the signal is 
  periodic, then the autocorrelation reaches the value $1$ when the lag matches 
  the period. So the period of the waveform can be deduced by looking for the 
  smallest lag which yields a value of the autocorrelation above some chosen 
  threshold. If the nominal period is divided by this lag, the result would be 
  1 for any signal at the nominal frequency, 2 for a signal an octave higher 
  (i.e. at the second harmonic), 3 for a signal at the 3rd harmonic and so on. 

  This number is what has been used to colour the pixels in Fig.\ 43. If you 
  compare with the scale in the colour bar, you can see that the chosen colours 
  are almost invariably close to the values 1, 2, 3, 4 or 5. Now scan along the 
  lowest of the three white lines, and compare pixel by pixel with the 
  waveforms in the 3rd column of Fig.\ 44. The pulse waves at the start and at 
  the end all give the dark red colour connoting the value 1: these cases all 
  correspond to ``normal'' sound at the nominal frequency. The 4th coloured 
  pixel along the line gives a bright red colour, connoting the value 2. Sure 
  enough, the 4th waveform from the top shows a symmetrical “double slip”, 
  sounding an octave higher. Continue this comparison, and the colour code 
  should make sense. 

  So why have I chosen this way to colour the diagram? The answer to that lies 
  in the behaviour we hope for in our saxophone. We might expect to find a 
  region of the plane where it plays the “right” note. But then adjacent to 
  that, we would not be surprised to find a region where the pitch jumps by an 
  octave: this would correspond to “over-blowing to the second register”. A bit 
  more, and we might over-blow to the third register, indicated by an orange 
  colour connoting the value 3. We see exactly this pattern if we scan 
  downwards from the top edge of the wedge region: a broad wedge with the value 
  1 (corresponding to “inverted” pulse waveforms like Fig.\ 45), then a thin 
  line of red followed by a line of orange. After that it gets rather messy, 
  because of the complicated waveforms revealed by Fig.\ 44. Near the bottom of 
  the wedge we see another solid region of the value 1, from the pulse 
  waveforms arranged as in Fig.\ 40. 

  So this first attempt at a simulation model for the soprano saxophone’s 
  lowest note shows encouraging behaviour. It is indeed the case that low notes 
  on a saxophone rather readily give ``bugling'' behaviour, where the note may 
  jump progressively from register to register. The clarinet, by contrast does 
  not show this behaviour: exactly what our plots such as Figs.\ 31--33 
  predict. When those plots show a note, it is always close to the expected 
  pitch for the first register. To shift register on a clarinet, you need to 
  open a register hole. 

  But possibly you think I am glossing over the regions of the saxophone 
  diagram where the pattern becomes complicated? Possibly so --- we simply do 
  not have the data to know whether a real saxophone behaves in a rather messy 
  way like these simulations, but we might suspect that the real instrument is 
  a bit better-behaved: it is what we should expect, given that we have used a 
  very crude model, lacking the subtlety of details that have been designed 
  into real saxophones. 

  We can get an inkling of the sensitivity of the pattern to small details of 
  the tube impedance by comparing with another case. We can’t make our model 
  better without a bit of effort (we'll come to that in the next section), but 
  we can very easily make it a bit worse, and see what effect that has. We can 
  omit the allowance for the mouthpiece volume, and simply use the acoustical 
  behaviour of a truncated cone (as described in the previous link). This is 
  quite a small change to the model, but as you can see from the results in 
  Figs.\ 46 and 47 it has a significant effect. The wedge region is smaller, 
  and a lot of the details have changed. However, Fig.\ 47 shows that the 
  repertoire of waveforms is essentially the same: pulse waves both ways up, 
  “double slipping”, and so on. 

  \fig{figs/fig-260824f8.png}{\caption{Figure 46. Pressure-gap diagram in the 
  same format as Fig. 43, generated from the saxophone model without including 
  the effect of the mouthpiece volume. [Idealised saxophone, dynamic reed 
  model.]}} 

  \fig{figs/fig-df48cd98.png}{\caption{Figure 47. Some sample waveforms from 
  the simulations lying behind Fig. 46, in the same format as Fig. 44. The 
  three columns correspond to all the coloured pixels in the three rows marked 
  with white lines. Left-to-right rows in Fig. 46 become top-to-bottom stacks 
  here, so that mouth pressure increases as you move down each column. The last 
  three period-lengths of each pressure waveform are plotted. [Idealised 
  saxophone, dynamic reed model.]}} 

  If we dig a little deeper into this comparison of results with and without 
  allowing for the mouthpiece volume, we can see an interesting example of 
  Benade’s idea of the playing frequency being determined by a collaboration 
  and compromise between the resonance frequencies of the tube, when these are 
  not exactly harmonically spaced. We will compare the spectra of notes 
  produced by the two versions of the model, for a particular case: the chosen 
  example has the same mouth pressure and reed gap as in Fig.\ 40, 
  corresponding to a pixel just above the threshold of excitation. The two 
  curves in Fig.\ 48 show the spectra deduced by taking an FFT of a long 
  stretch of simulated periodic sound, for the two cases: the black curve 
  includes the mouthpiece, the green one is without it. Both curves show sharp 
  peaks at the harmonics of the note, and it is immediately apparent that the 
  two models have ``chosen'' to play at different frequencies. To see why, we 
  need to compare them with the corresponding input impedances. 

  \fig{figs/fig-b3ba6406.png}{\caption{Figure 48. Spectra of the final periodic 
  waveforms from simulations corresponding to the same case as Fig. 40. The 
  black curve shows the model including the effect of the mouthpiece, the green 
  curve is without this effect. The sharp peaks mark the harmonics of the two 
  notes. [Idealised saxophone, dynamic reed model.]}} 

  We look first at the note simulated without allowing for the mouthpiece 
  cavity. Figure 49 shows the green spectrum from Fig.\ 48, compared with the 
  input impedance used in the simulation (blue curve) and also the 
  ``cylindrical saxophone'' impedance (dashed red curve). This dashed red curve 
  has resonances that are exactly harmonically spaced, so the blue/red 
  comparison gives an immediate visualisation of the “imperfection” of the 
  conical tube. Figure 49 shows that the played note has ``chosen'' a frequency 
  to match the highest peaks in the impedance, with its 3rd and 4th harmonics. 
  This has the result that the fundamental and the 2nd harmonic are sharp 
  compared to the tube resonances, while the higher harmonics are flat. 

  \fig{figs/fig-66576033.png}{\caption{Figure 49. The green curve is the same 
  as in Fig. 48, showing the frequency spectrum of a played note from the 
  saxophone model without including the mouthpiece effect. The blue curve is 
  the input impedance used in the simulation, the dashed red curve is the 
  corresponding input impedance of the ``cylindrical saxophone'' approximation. 
  [Idealised saxophone, dynamic reed model.]}} 

  Figure 50 shows the corresponding comparison for the case allowing for the 
  mouthpiece. The black curve is the same one as in Fig.\ 48, the blue curve 
  shows the input impedance that has been used in the simulation, and the 
  dashed red curve again shows the input impedance of the “cylindrical 
  saxophone” model. Compare this carefully with the blue curve. The first two 
  peaks are virtually identical. The third peak shows a small separation, and 
  then the higher frequencies show increasingly drastic divergence. Making the 
  same comparison in Fig.\ 49, the blue and red curves are separated a little 
  more clearly. 

  \fig{figs/fig-fec1c19a.png}{\caption{Figure 50. The black curve is the same 
  as in Fig. 48, showing the frequency spectrum of a played note from the 
  saxophone model including the mouthpiece effect. The blue curve is the input 
  impedance used in the simulation, the dashed red curve is the corresponding 
  input impedance of the ``cylindrical saxophone'' approximation. [Idealised 
  saxophone, dynamic reed model.]}} 

  This difference in impedance peaks with and without the mouthpiece seems like 
  a small effect, but now compare with the black curve in Fig.\ 50. Our 
  “saxophone” has chosen to play at a frequency that matches the first three 
  peaks of the impedance curve: the apparently small effect of the mouthpiece 
  has made these peaks sufficiently close to harmonic that their combined 
  influence is able to ``beat'' the fact that the amplitude of the first peak 
  is relatively low. (That low amplitude of the first peak is the reason that 
  it is notoriously difficult to start low notes on a saxophone, oboe or 
  bassoon very quietly.) The result is a simulation with a fundamental 
  frequency very close to the first resonance of the tube. The higher harmonics 
  of the played note diverge progressively from the peaks of the blue curve. 
  They continue to coincide with the peaks of the dashed red curve, because 
  those are accurately harmonic: as in Fig.\ 49, the blue/red comparison gives 
  a direct visualisation of the “imperfection” of the conical tube, even with 
  the mouthpiece compensation. 



  \sectionreferences{}[1] Antoine Chaigne and Jean Kergomard; “Acoustics of 
  musical instruments”, Springer/ASA press (2013) 

  [2] Andre Almeida, David George, John Smith and Joe Wolfe, “The clarinet: How 
  blowing pressure, lip force, lip position and reed ‘hardness’ affect pitch, 
  sound level, and spectrum”, Journal of the Acoustical Society of America 
  \textbf{134}, 2247—2255 (2013) 

  [3] Jean-Pierre Dalmont, Joël Gilbert, Jean Kergomard and Sébastien Ollivier, 
  “An analytical prediction of the oscillation and extinction thresholds of a 
  clarinet”, Journal of the Acoustical Society of America \textbf{118}, 
  3294—3305 (2005). 

  [4] Chaigne and Kergomard report that the ``cylindrical saxophone'' 
  approximation was first proposed by A. Gokhshtein, ``Self-vibration of finite 
  amplitude in a tube with a reed'', Soviet Physics --- Doklady \textbf{24}, 
  739--741 (1979). 

  [5] Tom Colinot, Philippe Guillemain, Christophe Vergez, Jean-Baptiste Doc 
  and Patrick Sanchez, “Multiple two-step oscillation regimes produced by the 
  alto saxophone”, Journal of the Acoustical Society of America \textbf{147}, 
  2406–2413 (2020) 