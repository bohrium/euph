  The idea of complex numbers starts with a mathematical fiction that seems 
  quite bizarre, but it leads to something remarkably powerful and useful, so 
  it is worth getting grips with. You will already know that you can take the 
  square root of any positive number, but whenever you square an ordinary 
  number you get a positive result, so there is no way to calculate the square 
  root of a negative number like $-1$. Well, suppose you could do it --- let us 
  see what happens if we define a symbol $i$ to be the square root of $-1$, so 
  that 

  \begin{equation*}i^2=-1 . \tag{1}\end{equation*} 

  (Some people call this $j$ rather than $i$, but I will always use $i$.) 

  Don't worry about what this thing $i$ might mean: just bear with me for a bit 
  as we play some mathematical games to see if it can do anything useful. The 
  next step is to introduce combinations of ordinary numbers and $i$, called 
  complex numbers: for example $5+3i$ or $2-2i$. The jargon is that $i$ is an 
  ``imaginary number'', as is $3i$ or $-2i$. These are the ``imaginary parts'' 
  of the complex numbers $5+3i$ and $2-2i$, while the $5$ and the $2$ are the 
  corresponding ``real parts''. 

  As a first step to see if anything makes sense, let's try multiplying these 
  two complex numbers together. We can treat $i$ just as if it was a 
  mathematical variable like $x$, and expand out the brackets as usual: 

  \begin{equation*}(5+3i)(2-2i) = 10 +6i -10i -6 i^2=10 -4i +6 = 16-4i 
  \tag{2}\end{equation*} 

  \noindent{}where we have used equation (1) to get rid of $i^2$. So the result 
  of multiplying two complex numbers is another complex number --- very 
  reassuring. 

  Next, we should recall that any square root could actually have a $\pm$ sign 
  in front of it: for example $\sqrt{4}=\pm2$. So $-i$ must also be a square 
  root of $-1$. This suggests that the two complex numbers $x+iy$ and $x-iy$ 
  might be closely related: we have just reversed the sign of $i$, to make the 
  other choice for the square root of $-1$. These two complex numbers are 
  called complex conjugates of each other. Now see what happens when we 
  multiply the two together: 

  \begin{equation*}(x+iy)(x-iy)=x^2 +ixy -ixy -i^2 y^2 = x^2 + y^2. 
  \tag{3}\end{equation*} 

  This result has a simple geometrical interpretation, in a useful graphical 
  representation of complex numbers called the ``Argand diagram''. If we take 
  the complex number $x+iy$, we can treat $x$ and $y$ as the Cartesian 
  coordinates of a point, and plot it as in Fig.\ 1. The horizontal axis shows 
  $x$, the real part of the complex number, so it is called the ``real axis''. 
  The vertical axis shows the imaginary part $y$, so it is called the 
  ``imaginary axis''. The complex conjugate of our complex number is also shown 
  in the figure: it is the mirror-reflection of $z$ in the real axis. 

  \fig{figs/fig-d808aa0a.png}{\caption{Figure 1. A complex number $z=x+iy$ in 
  the Argand diagram (red dot). The complex conjugate $z^*$ is shown as a blue 
  dot. The magnitude $r$ and the phase (or argument) $\theta$ are also shown.}} 

  But we don't always use Cartesian coordinates --- it is sometimes useful to 
  use polar coordinates $r$ and $\theta$ such that 

  \begin{equation*}x=r \cos \theta, \mathrm{~~~} y=r \sin \theta. 
  \tag{4}\end{equation*} 

  This polar representation of our complex number is indeed useful. The radius 
  $r$ is called the ``magnitude'' of the complex number, and the angle $\theta$ 
  is called its ``argument'' or its ``phase''. Now notice that we can 
  reinterpret equation (3) as saying that a complex number multiplied by its 
  complex conjugate gives the square of the magnitude. In symbols, if we denote 
  the complex number by $z=x+iy$ and the complex conjugate by $z^*=x-iy$, then 

  \begin{equation*}z z^* =r^2 =|z|^2 \tag{5}\end{equation*} 

  \noindent{}where $|z|$ is another common notation for the magnitude $r$ of 
  the complex number. 

  We can use this result to complete our toolkit for elementary manipulations 
  of complex numbers. We have already seen how to multiply two complex numbers 
  --- but what about dividing? So suppose we have complex numbers $a$ and $b$, 
  and we want $a/b$. All we need to do it multiply on the top and bottom of the 
  fraction by the complex conjugate of $b$: 

  \begin{equation*}\dfrac{a}{b}=\dfrac{a b^*}{b b^*}=\dfrac{a b^*}{|b|^2}. 
  \tag{6}\end{equation*} 

  The denominator is now an ordinary ``real number'' $|b|^2$, so we can see 
  that the ratio $a/b$ is indeed just another complex number, which we could 
  work out. 

  There is one more really important result involving complex numbers that we 
  will need, and it relates to the combination $\cos \theta + i \sin \theta$ 
  that appears in the polar expression 

  \begin{equation*}z=x+iy=r(\cos \theta +i \sin \theta) . 
  \tag{7}\end{equation*} 

  To get to this result we need a short digression on ``Taylor series'' or 
  ``power series''. It is often useful to express a function like cosine in 
  terms of a sum of powers: is it possible to write 

  \begin{equation*}\cos \theta =\sum_{n=0}^\infty{a_n \theta^n} 
  \tag{8}\end{equation*} 

  \noindent{}where the coefficients $a_0$, $a_1$, $a_2,...$ are constants? 
  Well, if we put $\theta =0$ in this equation, the left-hand side is 
  $\cos(0)=1$, while the right-hand side is simply $a_0$ because all the other 
  terms disappear. So we must have $a_0=1$. Now what happens if we 
  differentiate equation (8)? We get 

  \begin{equation*}-\sin \theta = \sum_{n=1}^\infty{n a_n \theta^{n-1}} 
  \tag{9}\end{equation*} 

  \noindent{}and now if we set $\theta=0$ we find 

  \begin{equation*}a_1 = -\sin(0)=0. \tag{10}\end{equation*} 

  Differentiating again and setting $\theta=0$ we find 

  \begin{equation*}a_2 = -\dfrac{\cos(0)}{2} = -\dfrac{1}{2}, 
  \tag{11}\end{equation*} 

  \noindent{}and we can continue this process to find the power series 

  \begin{equation*}\cos \theta = 1-\dfrac{\theta^2}{2}+\dfrac{\theta^4}{2 
  \times 3 \times 4}-\dfrac{\theta^6}{2 \times 3 \times 4 \times 5 \times 6} 
  +...\tag{12}\end{equation*} 

  If we go through the same process starting with the function $\sin \theta$, 
  we find that the corresponding power series is 

  \begin{equation*}\sin \theta = \theta -\dfrac{\theta^3}{2 \times 3} + 
  \dfrac{\theta^5}{2 \times 3 \times 4 \times 5}+... \tag{13}\end{equation*} 

  Finally, we can repeat the process with the exponential function $e^x$ and 
  find 

  \begin{equation*}e^x = 1+x+\dfrac{x^2}{2}+\dfrac{x^3}{2 \times 
  3}+\dfrac{x^4}{2 \times 3 \times 4}+... \tag{14}\end{equation*} 

  Now comes the magic. If we substitute $x=i\theta$ into equation (14), we find 

  \begin{equation*}e^{i \theta} = 1+i 
  \theta-\dfrac{\theta^2}{2}-i\dfrac{\theta^3}{2 \times 3}+\dfrac{\theta^4}{2 
  \times 3 \times 4}+... \tag{15}\end{equation*} 

  Alternate terms are real and imaginary, and if we collect those real and 
  imaginary terms together into two separate series, we recognise both of them 
  from equations (12) and (13): we have the very striking result 

  \begin{equation*}e^{i \theta} = \cos \theta + i \sin \theta . 
  \tag{16}\end{equation*} 

  Notice that this allows us to rewrite equation (7) in the compact form 

  \begin{equation*}z=x+iy=r e^{i \theta}. \tag{17}\end{equation*} 

  We will make extensive use of the result (16) in the development of the 
  theory of vibration ( sections 2.2.2 and 2.2.7 are the first important 
  examples), but to close this section I will just give one example of the 
  power of the result to derive some useful results in trigonometry. Suppose we 
  have two angles $A$ and $B$, and we multiply together the two results 
  corresponding to equation (16): 

  \begin{equation*}e^{iA} e^{iB} = (\cos A +i \sin A)(\cos B +i \sin B). 
  \tag{18}\end{equation*} 

  Multiplying out the brackets and using familiar properties of the exponential 
  function, this becomes 

  \begin{equation*}e^{i(A+B)} = \cos A \cos B -\sin A \sin B +i(\sin A \cos 
  B+\cos A \sin B) \tag{19}\end{equation*} 

  But 

  \begin{equation*}e^{i(A+B)} = \cos (A+B) + i \sin (A+B) 
  \tag{20}\end{equation*} 

  \noindent{}so equating the real parts and the imaginary parts of equation 
  (19) we can deduce immediately that 

  \begin{equation*}\cos (A+B)= \cos A \cos B -\sin A \sin B 
  \tag{21}\end{equation*} 

  \noindent{}and 

  \begin{equation*}\sin (A+B) = \sin A \cos B+\cos A \sin B . 
  \tag{22}\end{equation*} 

  These results can, of course, be proved by trigonometry without using complex 
  numbers, but it takes far, far longer than this simple calculation! 