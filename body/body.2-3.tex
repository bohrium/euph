

  Since one feature of human hearing has been mentioned, it is convenient to 
  have a short digression on musical pitch and its relation to frequency. Just 
  as we hear loudness on a logarithmic scale, so we respond to pitch on a 
  logarithmic scale too: when a musician talks about the relation of two 
  different notes, they use language which describes the ratio of the two 
  frequencies. At least, this is more or less true. For the moment I will 
  ignore quibbles of detail and tell the simple but approximate story. Our 
  brains are wired in a way which means that we have a special liking for 
  frequencies in simple whole-number ratios, and there are musical terms to 
  describe these. The simplest is the ratio 2:1: two notes with frequencies in 
  this ratio are said to be an octave apart. The ratio 3:2 describes a fifth, 
  4:3 a fourth, 5:4 a major third, 6:5 a minor third. Notes with a frequency 
  ratio close to one of these, but not quite accurate, are likely to sound out 
  of tune. 

  Now there is an arithmetic snag. If you choose a starting note and then go up 
  in seven octaves and back down again in 12 fifths, you arrive back almost at 
  the starting point, but not quite. If you are a singer or a violinist this is 
  no problem. You can vary the pitch continuously, and it doesn't matter if 
  these two notes are slightly different. But if you play an instrument with 
  fixed and defined pitches (via a keyboard, for example) then you need to play 
  the same note for both ends of this sequence. That can only be achieved by 
  some kind of fudging of the frequencies, known as a tempering system. To tune 
  a keyboard instrument, it is simply not possible to make all note 
  combinations (or intervals) be perfectly in tune. Some compromise is needed, 
  and different choices may be made depending on the style and period of music 
  to be played. We won't go into any detail here, we will only describe the 
  simplest and most common system, called equal temperament. 

  The argument goes like this. We hear in ratios, so let's try to find a way of 
  tuning so that the ratio from each note to the one above is the same. We will 
  keep octaves at exactly the 2:1 ratio, so the question is: how do we 
  subdivide the octave into a number of equal steps, so that we include 
  reasonably good approximations to the simple ratios described above: 3:2, 4:3 
  and so on? A bit of empirical exploring with a calculator reveals that the 
  smallest number of subdivisions which works well enough for most purposes is 
  12. So that is why a piano has 12 keys to the octave (ignoring any 
  distinction between black and white keys, which are purely for ergonomic 
  convenience of the player). In a similar way, a guitar has 12 frets to the 
  octave. The frequency ratio needed to achieve this can easily be worked out. 
  We need a number which when multiplied by itself 12 times ends up with the 
  value 2 (for the octave). So we need the ``twelfth root of 2'', written 
  $\sqrt[12]{2}$, which is equal to approximately 1.059, or about 6\%. This 
  ratio is the equal-tempered semitone. Any two adjacent notes on a piano have 
  frequencies different by about 6\%. 

  Applied to the earlier example: to go up seven octaves and come back down in 
  12 equal steps to reach exactly the starting point, each step needs to have a 
  ratio corresponding to seven equal-tempered semitones, which is 1.498 rather 
  than the ``perfect fifth'' ratio 1.5. Close, but slightly tweaked. For this 
  interval, equal temperament does pretty well. Its most conspicuous failing 
  comes with the major third: four equal-tempered semitones gives a ratio 
  1.2599, whereas the perfectly tuned major third should have the ratio 5/4 = 
  1.25. That sounds fairly close, but not close enough for an expert. To 
  describe small frequency discrepancies like this, it is usual to further 
  subdivide the semitone into 100 small intervals called cents. The most acute 
  musical ears can discriminate about 2 cents, 1/50 of a semitone, or a 
  frequency change of about 0.1\%. But the discrepancy between 1.2599 and 1.25 
  is about 14 cents, very clearly audible. This has consequences for the design 
  of the piano and other fixed-pitch stringed instruments, which we will come 
  to in Chapter ?. 

  Since we will be showing a lot of graphs of things plotted along a frequency 
  axis, it will be useful to have a translation table between musical notes and 
  frequency in Hz. So far we have only defined ratios of frequencies, so to get 
  an agreed pitch standard we need to adopt a convention. The usual modern 
  convention is to tune one of the notes A to the frequency 440 Hz. This is the 
  note the oboe will usually play before a concert starts, to allow the 
  orchestra to tune up. Starting from this value, it is easy to work out the 
  frequency of any other equal-tempered note. For reference, a table is given 
  here. Musical notes are labelled with their usual names A, B, C etc. for the 
  ``white notes'', with intervening semitones, the ``black notes'', called 
  things like ``A sharp'' or ``B flat'', written A$\sharp$ or B$\flat$. Within 
  equal temperament those two names mean exactly the same note, with a 
  frequency one semitone above A and one semitone below B. The table also shows 
  the conventional scheme for numbering the octaves, so we can refer to the A 
  in a particular octave as, for example, A$_4$. 

