  In this section we give a brief but general account of sound generation, 
  concentrating particularly on compact sources (i.e. the source region is 
  small compared to the wavelength of sound, or equivalently the Helmholtz 
  number is small). This description draws heavily on Chapter 7 of the textbook 
  by Dowling and Ffowcs Williams [1]. 

  The first step is to recall a result from section 4.3.1: the sound pressure 
  radiated by a small sphere vibrating sinusoidally at frequency $\omega$ is 

  \begin{equation*}p=\dfrac{i \omega \rho_0 q}{4 \pi r} e^{i \omega(t-r/c)} 
  \tag{1}\end{equation*} 

  \noindent{}where $r$ is the distance to the listener from the centre of the 
  sphere, $q$ is the total volume flux displaced by the pulsating sphere, and 
  $\rho_0$ is the density of air. Noting that $i \omega q$ is simply the 
  frequency-domain expression for the rate of change $\dot{q}$, we can see that 
  all the frequency components of a general waveform $\dot{q}(t)$ have 
  identical behaviour: an outgoing spherical wave at speed $c$, with amplitude 
  decreasing inversely with distance. We can deduce the time-domain equivalent 
  formula for the observed pressure at distance $r$: 

  \begin{equation*}p(t)=\dfrac{\rho_0 }{4 \pi r} \dot{q}(t-r/c) . 
  \tag{2}\end{equation*} 

  We will shortly use this result in a similar manner to the Rayleigh integral 
  (see section 4.3.2), to determine the total sound pressure generated by a 
  spatially distributed sound source region by performing a suitable 
  integration. 

  The next step is to revisit the derivation of the linearised wave equation 
  from section 4.1.1, and add in any possible source terms. The derivation 
  involved two fundamental equations: mass conservation and momentum balance 
  (or Newton's law). Each of those could have an associated source term. 

  The equation of mass conservation then reads 

  \begin{equation*}\dfrac{\partial \rho}{\partial t} + \rho_0 \nabla \cdot 
  \underline{v} = m \tag{3}\end{equation*} 

  \noindent{}where $\underline{v}$ is the air velocity vector. This is equation 
  (15) of section 4.1.1, with the addition of the term $m$ on the right-hand 
  side. This allows for the possibility of mass being added to the system: it 
  describes the rate at which mass is added, per unit volume, at a given 
  position in the source region. The kind of application we are interested in 
  involves air flowing out of a tone-hole, in which case if the volume flow 
  rate is $q$ per unit volume, then $m=\rho_0 q$. 

  In a similar vein, the equation of momentum balance now reads 

  \begin{equation*}\nabla p + \rho_0 \dfrac{\partial \underline{v}}{\partial t} 
  = \underline{f} , \tag{4}\end{equation*} 

  \noindent{}which is the same as equation (16) from section 4.1.1 with the 
  addition of the term $\underline{f}$ describing external force per unit 
  volume that may be applied to the air at a given position. 

  Now we proceed much as before. We can substitute $\rho = p/c^2$ in equation 
  (3), then eliminate $\underline{v}$ between equations (3) and (4) to leave an 
  equation for $p$. To achieve this we take $\partial/\partial t$ of equation 
  (3) and we take the divergence of equation (4), then subtract one from the 
  other. The result is 

  \begin{equation*}\dfrac{1}{c^2} \dfrac{\partial^2 p}{\partial t^2} 
  \mathrm{~}-\mathrm{~} \nabla^2 p = \rho_0 \dot{q} \mathrm{~}-\mathrm{~} 
  \nabla \cdot \underline{f} .\tag{5}\end{equation*} 

  We see the familiar wave equation, with two source terms on the right-hand 
  side. We can conclude that within the approximation of linearised acoustics 
  these are the only two possible ways in which sound can be generated. To see 
  what they mean, we can consider them in turn. The first term, $\rho_0 
  \dot{q}$, can be related directly to the result in equation (2) derived from 
  the pulsating sphere. The sphere problem had a concentrated excitation 
  involving $\rho_0 \dot{q}$, while equation (5) has a distributed version of 
  exactly the same thing, expressed per unit volume. So, just as we did with 
  the Rayleigh integral in section 4.3.2, we can integrate equation (2) over 
  the volume $V$ containing the source region, where $\rho_0 \dot{q}$ is 
  non-zero. The result is 

  \begin{equation*}p(\underline{x},t)=\rho_0 \int \int 
  \int_V{\dfrac{\dot{q}(\underline{y},t-|\underline{x} -- \underline{y}|/c)}{4 
  \pi |\underline{x} -- \underline{y}|} d^3 \underline{y}} 
  \tag{6}\end{equation*} 

  \noindent{}where the position vectors $\underline{x}$ and $\underline{y}$ 
  point to the receiver position and an element within the source region, as 
  indicated in Fig.\ 1, and $d^3 \underline{y}$ denotes the element of volume 
  with respect to the position $\underline{y}$. The equation looks at first 
  sight more complicated than equation (2), but this is simply because we need 
  the more cumbersome expression $|\underline{x} -- \underline{y}|$ in place of 
  the radial distance $r$: each element of the source is at a different 
  distance from the receiver. 

  \fig{figs/fig-9d9e3ced.png}{\caption{Figure 1. Sketch of a source region, an 
  element within it (green cube), and the position vectors of this element and 
  of the receiver where the pressure is to be observed.}} 

  If the source region is small compared to the wavelength of sound and 
  receiver is very distant from the source, the variation in $|\underline{x} -- 
  \underline{y}|$ for different positions $\underline{y}$ is very small. Under 
  those circumstances, equation (6) reduces to a simple approximate form: 

  \begin{equation*}p(\underline{x},t) \approx \dfrac{\rho_0}{4 \pi R} 
  \dot{Q}(t-R/c) \tag{7}\end{equation*} 

  \noindent{}where 

  \begin{equation*}\dot{Q}=\int \int \int_V{\dot{q} \mathrm{~} d^3 
  \underline{y}} \tag{8}\end{equation*} 

  \noindent{}and 

  \begin{equation*}R=|\underline{x} -- \underline{Y}| \tag{9}\end{equation*} 

  \noindent{}where 

  \begin{equation*}\underline{Y}=\dfrac{1}{V} \int \int \int_V{\underline{y} 
  \mathrm{~} d^3\underline{y}} \tag{10}\end{equation*} 

  \noindent{}is the mean position vector indicating the middle of the source 
  region. Equation (7) is exactly the same as equation (2): it describes 
  omnidirectional monopole sound radiation with a strength $\dot{Q}$ given by 
  simply adding all the distributed contributions (equation (8)). 

  Returning to equation (5), we now look at the effect of the second source 
  term. We can use the same integration approach as in equation (6), to obtain 

  \begin{equation*}p(\underline{x},t)=- \int \int \int_V{\dfrac{\nabla \cdot 
  \underline{f}(\underline{y},t-|\underline{x} -- \underline{y}|/c)}{4 \pi 
  |\underline{x} -- \underline{y}|} d^3 \underline{y}} . 
  \tag{11}\end{equation*} 

  If we then looked for a similar approximation to equations (7) and (8), we 
  would hit a snag. The equivalent of equation (8) would involve the integral 

  \begin{equation*}\int \int \int_V{\nabla \cdot \underline{f} \mathrm{~}d^3 
  \underline{y}}, \tag{12}\end{equation*} 

  \noindent{}and we can easily show that this integral is zero. We apply the 
  divergence theorem (see section 4.1.4 equation (11)) to a surface $S$ which 
  entirely encloses the source volume $V$, far enough out that 
  $\underline{f}=0$ everywhere on the surface. Then 

  \begin{equation*}\int \int \int_V{\nabla \cdot \underline{f} \mathrm{~}d^3 
  \underline{y}} = \int \int_S{\underline{f} \cdot d\underline{A}} = 0 . 
  \tag{13}\end{equation*} 

  So the net monopole strength of this kind of sound source is zero: in fact, 
  it describes a dipole source. To see this clearly, we use a trick --- closely 
  related to the trick we used in section 4.3.1 when we first met dipoles. 
  Suppose the equation we are trying to solve had been 

  \begin{equation*}\dfrac{1}{c^2} \dfrac{\partial^2 p_1}{\partial t^2} -- 
  \nabla^2 p_1 = \underline{f} \tag{14}\end{equation*} 

  \noindent{}rather than equation (5). Then we would know the solution by our 
  integration procedure: it would be 

  
  \begin{equation*}p_1(\underline{x},t)=\int_V{\dfrac{\underline{f}(\underline{y},t-|\underline{x} 
  -- \underline{y}|/c)}{4 \pi |\underline{x} -- \underline{y}|} d^3 
  \underline{y}} . \tag{15}\end{equation*} 

  Now we can take the divergence of both sides of equation (14) to deduce the 
  solution we really want: 

  \begin{equation*}\dfrac{1}{c^2} \dfrac{\partial^2 p}{\partial t^2} -- 
  \nabla^2 p = -- \nabla \cdot \underline{f} \tag{16}\end{equation*} 

  \noindent{}with 

  \begin{equation*}p(\underline{x},t) = -\nabla \cdot p_1(\underline{x},t) = 
  -\nabla \cdot \int_V{\dfrac{\underline{f}(\underline{y},t-|\underline{x} -- 
  \underline{y}|/c)}{4 \pi |\underline{x} -- \underline{y}|} d^3 \underline{y}} 
  \tag{17} .\end{equation*} 

  From here, we can obtain an approximate form for a compact source region, 
  similar to equations (7) and (8): 

  \begin{equation*}p(\underline{x},t) \approx -\nabla \cdot \left[ 
  \dfrac{\underline{F}(t-R/c)}{4 \pi R} \right] \tag{18}\end{equation*} 

  \noindent{}where 

  \begin{equation*}\underline{F} = \int \int \int_V{\underline{f} \mathrm{~} 
  d^3 \underline{y}} \tag{19}\end{equation*} 

  \noindent{}and $R$ is given by equation (9). 

  To put this result in a more intuitive form, suppose we choose to align our 
  $z$-axis with the direction of the net force $\underline{F}$. Write the 
  magnitude of the force $F(t)$. Then equation (18) becomes 

  \begin{equation*}p(\underline{x},t) \approx -\dfrac{\partial}{\partial z} 
  \left[ \dfrac{F(t-R/c)}{4 \pi R} \right] . \tag{20}\end{equation*} 

  As we did in section 4.3.1 we can relate a small change in $z$ to a small 
  change in $R$ via the polar angle $\theta$ with the $z$-axis: $dR=dz \cos 
  \theta$. So finally 

  \begin{equation*}p(\underline{x},t) \approx -- \cos \theta 
  \dfrac{\partial}{\partial R} \left[ \dfrac{F(t-R/c)}{4 \pi R} 
  \right]\end{equation*} 

  \begin{equation*} = \dfrac{\cos \theta}{4 \pi}\left[ \dfrac{1}{Rc} 
  \dfrac{dF}{dt} + \dfrac{F}{R^2} \right] . \tag{21}\end{equation*} 

  This is exactly the form we found for a dipole source in section 4.3.1. In 
  the far field, only the first term in the square brackets matters. 

  We have seen that within the scope of linear acoustics there are only two 
  types of sound source: anything producing a net variation in volume gives a 
  monopole source, and anything applying a net force to the fluid gives a 
  dipole source. But there is a third kind of source, if we take account of the 
  nonlinearity of the full governing equations for fluid motion. The underlying 
  theory was developed by Sir James Lighthill [2] back in the 1950s, when noise 
  from jet aircraft was beginning to be a major concern. 

  Lighthill's key insight was that you could take the full, nonlinear equations 
  governing fluid flow, and manipulate them into a form 

  \begin{equation*}<\mathrm{Wave~equation}>\mathrm{~}= 
  \mathrm{~}<\mathrm{Other~stuff}>\end{equation*} 

  \noindent{}and then interpret the $<\mathrm{Other~stuff}>$ as a source term, 
  creating sound waves. This formulation is usually called ``Lighthill's 
  acoustical analogy''. We will give a brief account here, but we need not go 
  deeply into the rather complicated mathematics associated with this subject. 
  We have laid the groundwork for the analysis in section 11.2.1. The steps are 
  very similar to the derivation of the wave equation given above: we will take 
  the equations of mass conservation and momentum balance and combine them to 
  give a single equation, in this case one governing the behaviour of fluid 
  density rather than pressure. 

  We will use the notation of section 11.2.1, denoting the three Cartesian 
  coordinates by $(r_1,r_2,r_3)$ rather than $(x,y,z)$ so that we can express 
  the equations in a compact form. Equation (5) from that section expressed 
  conservation of mass in the form 

  \begin{equation*}\dfrac{\partial \rho}{\partial t} = -- \nabla \cdot (\rho 
  \underline{u}) = -\sum_{j=1}^3\dfrac{\partial}{\partial r_j} \left( \rho u_j 
  \right) . \tag{22}\end{equation*} 

  Equation (20) from that section expressed momentum balance in the form 

  \begin{equation*}\rho \dfrac{Du_i}{Dt}=\sum_{j=1}^3{\dfrac{\partial 
  \sigma_{ij}}{\partial r_j}}\mathrm{~~~for~~}i=1,2,3 \tag{23}\end{equation*} 

  \noindent{}where $\sigma_{ij}$ is the stress tensor, and for our present 
  purpose we have not included the term expressing a body force --- we have 
  already dealt with external forces in the earlier discussion. Expanding out 
  the convective derivative $D/Dt$, this equation gives 

  \begin{equation*}\rho \dfrac{\partial u_i}{\partial t}=- \rho 
  \sum_{j=1}^3{u_j \dfrac{\partial u_i}{\partial r_j}} + 
  \sum_{j=1}^3{\dfrac{\partial \sigma_{ij}}{\partial 
  r_j}}\mathrm{~~~for~~}i=1,2,3 . \tag{24}\end{equation*} 

  Now we can combine equations (22) and (24) to show 

  \begin{equation*}\dfrac{\partial}{\partial t}(\rho u_i) =\dot{\rho} u_i + 
  \rho \dot{u_i}\end{equation*} 

  \begin{equation*}= -\sum_{j=1}^3\dfrac{\partial}{\partial r_j} \left( \rho 
  u_j \right) u_i \mathrm{~}-\mathrm{~} \rho \sum_{j=1}^3{u_j \dfrac{\partial 
  u_i}{\partial r_j}} + \sum_{j=1}^3{\dfrac{\partial \sigma_{ij}}{\partial 
  r_j}}\end{equation*} 

  \begin{equation*}= \sum_{j=1}^3 \dfrac{\partial}{\partial r_j} 
  \left[\sigma_{ij} \mathrm{~}-\mathrm{~} \rho u_i u_j \right] 
  \mathrm{~~~for~~}i=1,2,3 \tag{25}\end{equation*} 

  Now we can take $\partial/\partial t$ of equation (22) and subtract the 
  divergence of equation (25) to obtain 

  \begin{equation*}\dfrac{\partial^2 \rho}{\partial t^2} = -\sum_{i=1}^3 
  \sum_{j=1}^3 \dfrac{\partial^2}{\partial r_i \partial r_j} \left[\sigma_{ij} 
  \mathrm{~}-\mathrm{~} \rho u_i u_j \right] . \tag{26}\end{equation*} 

  Finally, if we subtract from both sides of this equation a term 

  \begin{equation*}c^2 \nabla^2 \rho = c^2 \sum_{i=1}^3 \dfrac{\partial^2 
  \rho}{\partial r_i^2} \tag{27}\end{equation*} 

  \noindent{}we can express the result in the form we are looking for: 

  \begin{equation*}\dfrac{\partial^2 \rho}{\partial t^2} \mathrm{~}-\mathrm{~} 
  c^2 \nabla^2 \rho = \sum_{i=1}^3 \sum_{j=1}^3 \dfrac{\partial^2 
  T_{ij}}{\partial r_i \partial r_j} \tag{28}\end{equation*} 

  \noindent{}where Lighthill's stress tensor $T_{ij}$ is defined by 

  \begin{equation*}T_{ij} = \rho u_i u_j \mathrm{~}-\mathrm{~} \sigma_{ij} 
  \mathrm{~}-\mathrm{~} c^2 \rho \delta_{ij} \tag{29}\end{equation*} 

  \noindent{}where $\delta_{ij}$ denotes the unit matrix as usual. 

  We need a slightly more general expression for $\sigma_{ij}$ than the one 
  given in equation (21) of section 11.2.1, because we do not want to assume 
  incompressible flow. We need not go into details, but it is always possible 
  to write 

  \begin{equation*}\sigma_{ij} = -p \delta_{ij} + \tau_{ij} 
  \tag{30}\end{equation*} 

  \noindent{}where $\tau_{ij}$ is a symmetric matrix (or, more correctly, 
  tensor) called the deviatoric viscous stress tensor. It vanishes for inviscid 
  flow. If we assume the usual relation $p=c^2 \rho$, the final form of the 
  Lighthill tensor is 

  \begin{equation*}T_{ij} = \rho u_i u_j \mathrm{~}-\mathrm{~} \tau_{ij} . 
  \tag{31}\end{equation*} 

  Looking back at equation (18), we saw that a dipole field was given by the 
  divergence of a monopole field. Equation (28) now shows that the sound field 
  generated by the nonlinearity of flow involves a double divergence, which 
  immediately tells us that it will be quadrupole in character. 

  \sectionreferences{}[1] A. P. Dowling and J. E. Ffowcs Williams; “Sound and 
  sources of sound”, Ellis Horwood (1983). 

  [2] M. J. Lighthill; “On sound generated aerodynamically. I and II”. 
  Proceedings of the Royal Society of London, Series A \textbf{211}, 564-587 
  (1952) and \textbf{222}, 1-32 (1954). 