  The justification for treating small vibration by linear theory lies in 
  Lagrangian mechanics. The equations of vibration of any system can be derived 
  from expressions for the potential and kinetic energy in the system. For 
  small motion, it is justifiable to make a series expansion of those energy 
  expressions and keep only the first term that does not vanish. The result is 
  that the kinetic energy is approximately a quadratic expression in the 
  velocities, while the potential energy is a quadratic expression in the 
  displacements. Applying Lagrange's equations to these approximate energy 
  expressions leads automatically to linear differential equations describing 
  the motion. 

  Some details of this calculation will be given in section 2.2.5, but the 
  essential idea can be illustrated using the simple harmonic oscillator 
  discussed in section 2.2.2, by approaching that problem in terms of energy. 
  The system is conservative (i.e. has no dissipative forces acting), so that 
  the total energy is constant. The kinetic energy is 

  \begin{equation*}T=\frac{1}{2} m \dot{x}^2 \tag{1}\end{equation*} 

  \noindent{}where $\dot{x}=dx/dt$. The potential energy stored in the spring 
  is 

  \begin{equation*}V=\frac{1}{2} k x^2. \tag{2}\end{equation*} 

  This may be seen by noting that the work done in a small displacement $\delta 
  y$ from displacement $y$ is $ky\delta y$, and integration from zero to the 
  desired displacement $x$ gives the expression (2). Now energy conservation 
  requires 

  \begin{equation*}\dfrac{d}{dt} [T+V]=0 \tag{3} \end{equation*} 

  \noindent{}which gives 

  \begin{equation*}m \ddot{x} +kx=0 \tag{4}\end{equation*} 

  \noindent{}exactly as in eq. (1) of section 2.2.2, derived via Newton's laws. 

  The energy approach gives a natural way to generalise this simple example to 
  cover small vibrations of a wide range of systems. Consider any system with 
  just one degree of freedom, in other words one whose displaced position can 
  be completely described by a single parameter $q$ (which might, for example, 
  describe the displacement of a point or the angle of rotation of a rigid 
  body). It will generally be the case that the kinetic energy of the system 
  takes a similar quadratic form to expression (1): 

  \begin{equation*}T=\frac{1}{2} M \dot{q}^2 \tag{5}\end{equation*} 

  \noindent{}where the constant $M$ will be a mass if $q$ describes a linear 
  displacement, or a moment of inertia if $q$ describes a rotation. 

  The potential energy is less immediately obvious, but if the system is in a 
  position of stable equilibrium when $q~=~0$ then certainly the function 
  $V(q)$ must have a minimum at $q~=~0$. Provided this function is reasonably 
  well-behaved, we can expect it to be well approximated by the dominant term 
  in its Taylor expansion provided $q$ remains sufficiently small: 

  \begin{equation*}V(q) \approx V_0+\frac{1}{2} V_2 q^2 + 
  \mathrm{terms~of~order~} q^3. \tag{6} \end{equation*} 

  There is no linear term in this expansion by virtue of the equilibrium at 
  $q~=~0$, and in order for that equilibrium to be stable we must have 

  \begin{equation*}V_2 \ge 0. \tag{7} \end{equation*} 

  The constant $V_0$ is of no dynamical consequence, and can be set to zero for 
  simplicity. Provided $V_2$ is non-zero we are left with an approximate 
  expression for the potential energy which takes the same form as that of the 
  simple mass-spring oscillator in eq. (2). 

  In some cases a similar approximation may be needed in the kinetic energy: 
  the quantity $M$ may depend on the displacement $q$, but for the case of 
  small displacements away from equilibrium the dominant term is obtained 
  simply by setting $q~=~0$ within this expression. The next term in a Taylor 
  expansion would be of order $q\dot{q}^2$, small by comparison. 

  Having thus obtained approximate expressions for both kinetic and potential 
  energy which match those of the simple spring-mass system, it follows that 
  for any such system, provided the displacement away from equilibrium is 
  sufficiently small, the free motion will obey a differential equation 
  analogous to eq. (4), and the solution will take the sinusoidal form of eq. 
  (5) of section 2.2.2, with a natural frequency 

  \begin{equation*}\Omega=\sqrt{V_2/M}. \tag{8} \end{equation*} 