  It is worth showing a little detail on linear vibration theory, because this 
  leads to a formula which will repeatedly be useful as we look at the 
  behaviour of musical instruments. For simplicity, we will show the results 
  for a discrete vibrating system: one that has only a finite number of 
  variables (or degrees of freedom). This is not an important restriction: 
  continuous systems like stretched strings or wooden plates can always be 
  approximately described by discrete systems, provided we allow enough 
  variables. For example, we might do the equivalent of sampling the string 
  motion at regularly spaced points. If those points are close enough together 
  then we won't be missing anything important in the gaps between the points. 
  Another way to approximate a continuous system by a discrete system is to 
  build a finite element model of it, perhaps using a commercial software 
  package. The details of that method do not matter here, the important thing 
  is that the bit of theory we are about to look at, for general discrete 
  systems, would apply automatically to any of these approximate treatments of 
  continuous systems. 

  The rest of this discussion will be quite technical, but the end result is 
  very important. If the detail is too much, then skip to eq. (11) and its 
  discussion. All you really need to know is that $u_j^{(n)}$ represents the 
  amplitude of the $n$th vibration mode, at a position labelled by $j$, and 
  that $\omega_n$ is the natural frequency of that mode. But here is the fuller 
  discussion. Suppose we have a system with $N$ degrees of freedom, described 
  by parameters (called ``generalised coordinates'') $q_1,q_2,\dots,q_N$. The 
  first step is to calculate approximate expressions for the potential energy 
  $V$ and the kinetic energy $T$ in terms of the $q$s and their time 
  derivatives, $\dot{q}$s, assuming that the motion is small as outlined in 
  section 2.2.3. The result for the potential energy can be written 

  $$V=\dfrac{1}{2}\sum_j \sum_k K_{jk} q_j q_k, \tag{1} $$ 

  a quadratic expression involving a matrix $K$ whose terms are $K_{jk}$, 
  called the stiffness matrix. (The notation $\sum_j(\dots)$ means ``add up the 
  terms $(\dots)$ for all the values of $j$.'') We can always choose the values 
  so as to make $K$ symmetric. The total kinetic energy $T$ is the sum of terms 
  like ``$\frac{1}{2}mv^2$'' for each bit of mass making up the system. It 
  follows that it will be described by a quadratic expression 

  $$T=\dfrac{1}{2}\sum_j \sum_k M_{jk} \dot{q}_j \dot{q}_k . \tag{2}$$ 

  Again, we can always choose to make the matrix $M$, whose terms are $M_{jk}$, 
  symmetric. It is called the mass matrix. 

  Now we can use Lagrange's equations: for free motion of the system, 

  $$\dfrac{d}{dt}\left( \dfrac{\partial T}{\partial 
  \dot{q}_j}\right)+\dfrac{\partial V}{\partial q_j}=0 \tag{3}$$ 

  so that 

  $$\sum_k M_{jk} \ddot{q}_k + \sum_k K_{jk} q_k =0. \tag{4}$$ 

  These are linear equations, representing a set of coupled mass-spring 
  oscillators. Finding the vibration modes allows us to uncouple them. A 
  vibration mode is simply defined to be a solution with sinusoidal time 
  dependence: 

  $$q_j(t)=u_j e^{i\omega t}. \tag{5}$$ 

  So we require 

  $$-\omega^2 \sum_k M_{jk} u_k + \sum_k K_{jk} u_k =0, \tag{6}$$ 

  or in matrix-vector form, 

  $$K \mathbf{u}=\omega^2 M \mathbf{u}, \tag{7}$$ 

  which is a standard mathematical equation called a ``generalised eigenvalue 
  problem''. For the purposes of hand calculation, the usual solution procedure 
  is: 

  (i) solve 

  $$\det[K -\omega^2 M ]=0 \tag{8}$$ 

  for the natural frequencies $\omega=\omega_n$; 

  (ii) for each allowed $\omega_n$, solve the simultaneous equations 

  $$K \mathbf{u}^{(n)}=\omega_n^2 M \mathbf{u}^{(n)}, \tag{9}$$ 

  for the mode shape $\mathbf{u}^{(n)}$. 

  Because $M$ and $K$ are both symmetric, there are standard results from 
  linear algebra which show that 

  (i) the values of $\omega_n^2$ are all real, as are the eigenvectors; 

  (ii) the eigenvectors are orthogonal with respect to both the mass and 
  stiffness matrices. This means that 

  $$ \mathbf{u}^{(m)T} M \mathbf{u}^{(n)} = 0 \mathrm{~~~~if~} n \ne m 
  \tag{10}$$ 

  and 

  $$ \mathbf{u}^{(m)T} K \mathbf{u}^{(n)} = 0 \mathrm{~~~~if~} n \ne m 
  \tag{11}$$ 

  where $.^T$ denotes the transpose. 

  Mode shapes have no built-in amplitude, they can be scaled by any factor. It 
  is convenient to fix the scale factor using ``mass normalisation'': scale 
  each mode so that 

  $$\mathbf{u}^{(n)T} M \mathbf{u}^{(n)}=1 . \tag{12}$$ 

  The archetypal vibration measurement, on a musical instrument or any other 
  structure, is to apply a sinusoidal force to one point on a structure and 
  observe the response at another point. Suppose we apply a sinusoidal force 
  $F$ at frequency $\omega$ to the $j$th generalised coordinate. Now the 
  frequency response function we want is the response at ``point'' $k$ to this 
  forcing at ``point'' $j$. It can be expressed rather simply in terms of the 
  mode shapes and frequencies: 

  $$G(j,k,\omega)=\dfrac{q_k}{F}=\sum_n 
  \dfrac{u_j^{(n)}u_k^{(n)}}{\omega_n^2-\omega^2}. \tag{13}$$ 

  The word ``point'' was deliberately written in quotes in the sentence above, 
  because eq. (13) applies to a much wider range of problems than is 
  immediately apparent. We will return to that question after some qualitative 
  comments about what eq. (13) tells us. 

  The substance of eq. (13) can be described in words. First, the frequency 
  response function for any chosen driving point $j$ and measuring point $k$ 
  can be found once we know the mode shapes and natural frequencies: no other 
  information is needed about the system. Modal information tells you 
  everything. Second, the amplitude of a given modal response is given by the 
  product of the (mass-normalised) mode shape at the driving point with the 
  same mode at the observation point. (Those could be the same point if $j = 
  k$, in which case it would be called a ``driving-point response''.) For the 
  drum problem, if the measuring point is kept fixed while the drummer hits at 
  different positions, the ``amount'' of each mode in the mixture will change 
  in a way which is easily visualised: it simply follows the modal amplitude at 
  the tapping point. Third, each of the separate terms inside the summation in 
  eq. (13) looks exactly like the response of a simple mass-spring oscillator 
  as shown in eq. (12) of section 2.2.2. Comparing the two expressions, we see 
  that the $n$th mode has a natural frequency $\omega_n$ and an effective mass 
  which is the inverse of the mode-shape product $u_j^{(n)}u_k^{(n)}$. 

  This equivalence extends to the free motion of the system, as well. The most 
  general free motion involves a mixture of the vibration modes, with each mode 
  behaving just like a mass-spring oscillator, bouncing away at its own 
  frequency quite independent of what any other modes might be doing. The total 
  response is simply the sum of these separate modal responses. 

  The components of the vector $\mathbf{u}$ are not necessarily displacements 
  at points on the structure: they might be rotation angles if we are looking 
  at a problem involving torsional motion, or they might be less obvious things 
  like the coefficients in a Finite-Element model of a structure. The transfer 
  function (13) applies equally well to all these cases, provided the correct 
  definition is used for the ``force'' $F$. For internal consistency, this has 
  to be the correct ``generalised force'' to appear on the right-hand side of 
  Lagrange's equation. The definition of the generalised force $Q_j$ associated 
  with a generalised coordinate $q_j$ is as follows. Imagine a small movement 
  of the system in which $q_j$ increases by a small amount $\delta q_j$ while 
  all the other generalised coordinates remain unchanged. Suppose the work done 
  on the system by external forces during this displacement is $\delta W$: then 
  define $Q_j$ so that 

  $$\delta W=Q_j \delta q_j. \tag{14}$$ 

  If $q_j$ is actually the displacement of a particular point on the system, 
  then $Q_j$ is indeed the force applied at that point, as implied by the 
  description during the derivation of the transfer function (13) above. If 
  $q_j$ is the rotation angle of a system component, then $Q_j$ is the external 
  moment applied to that component. 