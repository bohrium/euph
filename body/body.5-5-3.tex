  The main study of the banjo was all conducted with a single instrument and a 
  single bridge. One conclusion was that the dynamic behaviour of the bridge, 
  together with its mass and height, is responsible for creating formants. As a 
  postscript to that study, a number of alternative banjo bridges were fitted 
  to the standard instrument, and the bridge admittances at the 1st and 3rd 
  string positions measured. These measurements could then be used to 
  synthesise sound examples. The combination of measurements and synthesised 
  sounds can give a first impression of the range of variation possible, and of 
  how different these can sound. Some of the bridges are commercially 
  available, others were made or modified specifically for this project. All 
  the bridges have the same height, and all except the last one have the same 
  mass (2.2~g). The bridge makers involved are Greg Deering, Donald New, 
  Jeffrey Weitzl and Bart Veerman. The selection and adjustment was made by my 
  co-conspirator on the banjo project, David Politzer. 

  Figure 1 shows the original Deering bridge, together with the most extreme 
  variant: a solid block of poplar, shaped to give the correct mass and height. 
  The white patches visible in this figure, and in others to follow, are pieces 
  of reflective tape added to allow the laser vibrometer measurements. Figure 2 
  shows the measured admittances with this ``bridge'', contrasted with the 
  corresponding results with the Deering bridge. The same format will be used 
  in all subsequent admittance plots. Results for the featured bridge, the 
  poplar block in the case, are plotted in solid lines while the Deering 
  results are plotted in dotted lines, the same in all subsequent plots. The 
  black and blue curves show the admittances at the first string notch, the red 
  and magenta curves show the admittances at the 3rd string notch, at the 
  centre of the bridge. 

  \fig{figs/fig-72f3bbb8.png}{Figure 1. The original bridge, and a ``bridge'' 
  made from a solid block of poplar with the same mass and height} 

  \fig{figs/fig-4ad5cc78.png}{Figure 2. Bridge admittances for the poplar block 
  bridge (solid curves), compared to the original bridge (dotted curves). Black 
  and blue: string 1 position; red and magenta: string 3 position.} 

  For the poplar block bridge, lacking a hard ebony cap like the other bridges, 
  it was not possible to obtain a hammer impact of a very high bandwidth: the 
  poplar is simply too soft. This is the reason that the measured admittances 
  become very noisy at high frequency. However, the comparison of the plots 
  reveals exacatly what would have been predicted based on the earlier study. 
  Because the mass and height were the same as the original bridge, the 
  low-frequency formant is very similar with both bridges. Up to about 2~kHz, 
  the pairs of curves agree quite closely. Above that, the Deering bridge shows 
  formants (discussed in detail in section 5.5). But the poplar block shows 
  very little evidence of such features, except for a small hump around4~kHz. 

  The final step is to produce synthesised sound examples. The same strategy 
  has been used in all cases. The two measured admittances have been used to 
  compute estimates of admittances at the 5 string notches. Strings 1 and 3 
  were directly measured. String 5 is assumed to have the same admittance as 
  string 1 because all the bridges are symmetrical. For strings 2 and 4, a 
  simple average of the results for strings 1 and 3 was used. The resulting 
  sound clips for the original bridge and the poplar block are given in Sounds 
  1 and 2. To my ears, the two sounds are similar but by no means identical. I 
  should emphasise that there is no claim that these synthesised sound will 
  capture all nuances of the actual bridges. For example, the softer wood of 
  the poplar block may well have a local influence on the string where it 
  enters the bridge notch: such effects are not included in the sounds here. 

  The next set of bridges, Shown in Fig.\ 3, are variations on a theme. The 
  starting point was three nominally identical bridges with a violin-like 
  shape. The first bridge is the original. For the second, the middle leg has 
  been trimmed back so that it does not touch the banjo head when fitted. The 
  third bridge has a similar shortening of the leg, and then has a cut in the 
  lower part to reduce the bending stiffness. 

  \fig{figs/fig-e4b7e4f2.png}{Figure 3. Three variations on a ``violin-shaped'' 
  bridge: the second has the middle leg trimmed short, and the third has a cut 
  to reduce the bending stiffness} 

  The corresponding admittances are shown in Figs.\ 4, 5 and 6. Again, in all 
  three cases the low-frequency formant shows a close correspondence with the 
  original Deering bridge. But the higher formants are significantly different, 
  as would be expected with these different designs. The clearest difference is 
  in the formant appearing strongly in the bridge-centre measurements. With the 
  Deering bridge (magenta dots) this formant appeared around 5~kHz. In the 
  first two of the present trio, it is seen around 8.5~kHz, then when the cut 
  is added for the third bridge in this set it falls to around 6~kHz. This 
  formant was shown to be associated with a first bending resonance of the 
  bridge, modified by contact with the banjo head. These frequency variations 
  all follow a pattern that makes sense in terms of that interpretation. 

  \fig{figs/fig-26e5a6db.png}{Figure 4. Admittances of the first bridge from 
  Fig. 3, compared with the original Deering bridge in the same format as Fig. 
  2.} 

  \fig{figs/fig-1bd6a186.png}{Figure 5. Admittances of the second bridge from 
  Fig. 3, compared with the original Deering bridge in the same format as Fig. 
  2.} 

  \fig{figs/fig-613b19c6.png}{Figure 6. Admittances of the third bridge from 
  Fig. 3, compared with the original Deering bridge in the same format as Fig. 
  2.} 

  The variations seen when comparing the black and blue curves are less easy to 
  guess. With the Deering bridge, a formant-like feature was seen around 
  3.5~kHz. The finite-element modelling described in Section 5.5 suggested that 
  this was associated with detailed motion of the feet of the bridge. Without 
  carrying out similar detailed modelling on the other bridge shapes to be 
  shown here, it is very hard to guess what we expect to happen to this feature 
  --- and indeed a feature looking somewhat similar was seen in Fig.\ 2 with 
  the poplar block, which of course does not have feet at all. What is clear in 
  Figs.\ 4, 5 and 6 is that something changes when the middle leg is shortened: 
  a clear feature is seen in Figs.\ 5 and 6 around 3~kHz, whereas Fig.\ 4 
  showed a broader and less prominent feature, somewhat higher in frequency. 

  Synthesised sounds based on these three bridges are given in Sounds 3, 4 and 
  5. It is hard to put into words one's impressions of such sounds, but I think 
  I hear a progressive change from ``more muffled'' to ``more clear'' through 
  the sequence of sounds. But exactly which features of the response are giving 
  me that impression is very hard to say. It would require some careful 
  investigation and systematic listening tests to resolve that question: 
  remember the discussion of the traps and pitfalls of psychoacoustical testing 
  from chapter 4. 

  The next two bridges, shown in Fig.\ 7, are commercially-available bridge 
  with designs that are radically different from the 3-foot design of the 
  Deering bridge. Admittances for the two are shown in Figs.\ 8 and 9, and 
  corresponding synthesised sounds in Sounds 6 and 7. The first bridge shows 
  admittances that are similar in general form to the Deering original, but 
  different in details. The second bridge is more different, as is not 
  surprising from this design which has obvious higher bending stiffness. 

  \fig{figs/fig-3eee34f1.png}{Figure 7. Two bridges of unusual design: a 
  Weitzel 6-10, and a Donald New ``spillway dam''.} 

  \fig{figs/fig-ce165220.png}{Figure 8. Admittances of the first bridge from 
  Fig. 7, compared with the original Deering bridge in the same format as Fig. 
  2.} 

  \fig{figs/fig-9c45f599.png}{Figure 9. Admittances of the second bridge from 
  Fig. 7, compared with the original Deering bridge in the same format as Fig. 
  2.} 

  The final set of bridges is shown in Fig.\ 10. Admittances are shown in 
  Figs.\ 11, 12 and 13, and synthesised sounds are given in Sounds 8, 9 and 10. 
  The first pair of bridges have a design similar to the Deering bridge, but 
  made of a different wood. The difference between the two is that in the first 
  one, the feet are shaped with a curved contour while the second one is flat. 
  The third bridge is the only one in the set tested here which has a different 
  mass: roughly half the mass of all the other bridges. Both measurements and 
  sound examples suggest very little difference between the first pair. But the 
  third bridge behaves quite differently, as we would expect. The lower mass 
  makes the low-frequency formant extend to higher frequency, as is clear in 
  Fig.\ 13. The corresponding sound (Sound 10) is strikingly different from all 
  the others. 

  \fig{figs/fig-9c565c9f.png}{Figure 10. Three bridges by Bart Veerman. The 
  first pair are similar except that the first has a curved base contour while 
  the second is flat. The third bridge is the only one of this set with a 
  different mass: about 1.2~g compared to 2.2~g for all the others.} 

  \fig{figs/fig-7bdc7af6.png}{Figure 11. Admittances of the first bridge from 
  Fig. 10, compared with the original Deering bridge in the same format as Fig. 
  2.} 

  \fig{figs/fig-4ddf407e.png}{Figure 12. Admittances of the second bridge from 
  Fig. 10, compared with the original Deering bridge in the same format as Fig. 
  2.} 

  \fig{figs/fig-f71124ee.png}{Figure 13. Admittances of the third bridge from 
  Fig. 10, compared with the original Deering bridge in the same format as Fig. 
  2.} 