  In order to complete our simulation model of a reasonably realistic 
  ``one-note clarinet'' we need a suitable reflection function. In order to 
  obtain this, we need to clarify the relationships between four different 
  acoustical quantities that can be used to characterise the linear acoustics 
  of a tube: the impedance, the impulse response, the reflection coefficient 
  and the reflection function. 

  Suppose we inject a sinusoidal pressure wave into the tube at the mouthpiece 
  end. This wave travels down the tube, and generates a reflected wave 
  travelling back in the reverse direction. As measured at the mouthpiece, we 
  can denote these $P_R e^{i \omega t}$ and $P_L e^{i \omega t}$ respectively 
  (we are envisaging the outgoing wave as travelling to the right, and the 
  reflected wave travelling to the left). The two wave amplitudes are linked by 

  $$P_L=R(\omega) P_R \tag{1}$$ 

  where $R(\omega)$ is the reflection coefficient. Note that $P_R$, $P_L$ and 
  $R$ may all be complex quantities, because they contain phase information as 
  well as amplitude information. 

  We can use a result from section 4.1.1 to obtain the corresponding complex 
  amplitudes of the two travelling waves of volume flow rate: these will be 

  $$V_R=P_R/Z_t,\mathrm{~~~~}V_L=-P_L/Z_t \tag{2}$$ 

  in terms of the characteristic impedance of the tube, $Z_t$. Now we can write 
  down an expression for the input impedance $Z$: 

  $$Z(\omega) = \dfrac{P_R+P_L}{V_R+V_L}=\dfrac{Z_t(P_R+P_L)}{P_R-P_L}=Z_t 
  \dfrac{1+R}{1-R}. \tag{3}$$ 

  Inverting this relation gives 

  $$R=\dfrac{Z-Z_t}{Z+Z_t}. \tag{4}$$ 

  Both $Z$ and $R$ are defined in the frequency domain. The inverse Fourier 
  transforms of these two quantities are both of some interest to us. We know 
  from section 2.2.8 that the inverse transform of any input-output frequency 
  response of a linear system is the corresponding impulse response: the 
  response of the output variable when the input is a unit delta function 
  $\delta(t)$. So the inverse transform of the impedance is the impulse 
  response $g(t)$, the pressure response at the mouthpiece to a pulse of volume 
  flow. The inverse transform of the reflection coefficient is the reflection 
  function $r(t)$ that we need: the reflected pressure wave when a pressure 
  spike is injected into the tube. 

  It is time to see examples. The starting point is the impedance of a 
  clarinet-like tube, which we can calculate as described in section 11.1.1. We 
  choose a tube of length 0.66~m and internal diameter 15~mm, matching the 
  dimensions of a clarinet. Strictly the length should be augmented by suitable 
  end corrections, without which our ``instrument'' will play a bit sharp, but 
  this is not important for the purpose of representing the essential physics 
  of a clarinet. 

  The resulting impedance is shown in the red curve of Fig.\ 1. We want to use 
  an inverse FFT to calculate the impulse response, and in order to avoid 
  numerical artefacts we need to reduce the amplitude by the highest frequency. 
  This has been done in the blue curve. The damping of the tube has been 
  artificially raised, starting at 6~kHz and increasing thereafter. The result 
  is that the curve has settled down to a flat line at the highest frequency 
  plotted. This flat line corresponds to the characteristic impedance $Z_t$, 
  appropriate to a semi-infinite tube of the same diameter. (With the decibel 
  scale of the plot, this illustrates Skudrzyk's theorem: see section 5.3.2.) 

  \fig{figs/fig-f1c60553.png}{\caption{Figure 1. The scaled input impedance 
  $Z/Z\_t$ of a clarinet-like cylindrical tube, normalised by the 
  characteristic impedance. The red curve shows the result according to the 
  method described in section 11.1.1, the blue curve shows a modified version 
  with increased damping at higher frequencies.}} 

  Although this increase of damping is entirely artificial, something of the 
  kind happens in a real clarinet because of the influence of the bell. The 
  efficiency of sound radiation at the bell rises with frequency, so that the 
  impedance does indeed tend to flatten out. Figure 2 shows an illustration. 
  This is a measured input impedance of a clarinet fingered for its lowest 
  note, with all the tone-holes closed, taken from Joe Wolfe's \tt{}web 
  site\rm{}. (We should mention a small complication: the measurement is on a 
  $B\flat$ clarinet, which is a transposing instrument. The note in question is 
  written as $E_3$, but it sounds at $D_3$, 147~Hz.) This measured impedance 
  shows peaks reducing in amplitude by the highest frequency plotted here, 4 
  kHz. This real effect is far more drastic than the artificial one imposed in 
  Fig.\ 1. 

  \fig{figs/fig-b27b4438.png}{\caption{Figure 2. The measured input impedance 
  of a clarinet with all tone-holes closed, taken from 
  https://newt.phys.unsw.edu.au/music/clarinet/E3.html and reproduced by 
  permission of Joe Wolfe. Note that the frequency range here only runs up to 
  4~kHz, much shorter than in Fig. 1.}} 

  Taking the inverse FFT of the impedance plotted in blue in Fig.\ 1 gives the 
  impulse response shown in Fig.\ 3. The initial delta function can be seen, 
  appearing as a single digital sample of height 1 in this discrete-time 
  representation. The initial pulse travels down the tube and returns, inverted 
  by the reflection at the end. It then reflects at the mouthpiece end and 
  travels off down the tube again, not inverting at this reflection because the 
  mouthpiece is a closed end. After another delay, it returns after another 
  inversion at the far end so that it now appears as a positive pulse. The 
  pattern then repeats, with alternating negative and positive pulses getting 
  gradually lower and more rounded. 

  \fig{figs/fig-37a64b13.png}{\caption{Figure 3. Impulse response of an 
  idealised clarinet tube, obtained by inverse FFT of the blue impedance from 
  Fig. 1.}} 

  If we process the impedance with equation (4) to obtain the reflection 
  coefficient, then take the inverse FFT of that, we obtain the result shown in 
  Fig.\ 4. This is the reflection function we need for the simulation model. It 
  is clear from the plot that the processing has had the desired effect: the 
  multiple reflections have all been removed, just leaving the single negative 
  pulse representing a single reflection. This is the function used in the 
  simulations described in section 11.3. One thing to note: the pulse seen in 
  Fig.\ 3 is symmetrical, but in reality this would not quite be true because 
  of a factor neglected in this analysis. Interactions with the thermal and 
  viscous boundary layers in the tube affect the wave speed as well as the 
  damping, causing wave dispersion and leading to an asymmetric pulse. For 
  details, see for example Fletcher and Rossing [1], section 8.2. 

  \fig{figs/fig-78a02e24.png}{\caption{Figure 4. Reflection function 
  corresponding to the impulse response in Fig. 3, obtained by processing the 
  impedance with equation (4) then taking an inverse FFT.}} 

  For this reflection function, we find numerically that 

  $$\int_0^\infty{r(t) dt}=-0.966$$ 

  so that it does not satisfy the unit-area condition given in equation (5) in 
  section 8.5.3. In section 11.3 we also showed simulations done with a 
  corresponding Raman model, and it is now easy to describe how that model was 
  created. Raman's model has a delta-function reflection function 

  $$r_{Raman}(t)=-\lambda \delta(t-\tau)$$ 

  with amplitude $\lambda \le 1$ and delay $\tau$. So to create a Raman model 
  similar to the reflection function in Fig.\ 4, we choose $\tau$ to match the 
  timing of the (negative) peak response, and $\lambda=0.966$ to match the 
  area. In section 11.3 we also showed some results with the lossless version 
  of the Raman model: that model uses the same delay, but with $\lambda=1$. 

  \sectionreferences{}[1] Neville H Fletcher and Thomas D Rossing; “The physics 
  of musical instruments”, Springer-Verlag (Second edition 1998) 