

  After that preamble we are ready to look in detail at some specific types of 
  measurement. We will start with a group of methods concerned with exploring 
  structural details that, for one reason or another, cannot be seen with the 
  naked eye. There have been huge developments in recent decades in ways to 
  visualise hidden details, and we will show examples of some of the most 
  striking in this section. The measurements rely on seriously expensive 
  laboratory-grade equipment, but the results are so useful to instrument 
  makers that images like the ones we will see are becoming quite familiar in 
  that world. 

  First, we will look at microscope imagery, applied to the cellular structure 
  of wood and how this is modified by wood-working procedures. Traditional 
  light microscopes have, of course, been with us for a very long time. Wood 
  cells can be viewed that way, but there are two drawbacks. First, the use of 
  light puts a limit on the highest magnification you can use: the wave nature 
  of light means that it is not possible to image details once their length 
  scale becomes comparable with the wavelength. Secondly, at least with 
  traditional microscopes, there is a severe limitation on the depth of field 
  of the images you obtain. 

  Both these drawbacks can be circumvented by using a scanning electron 
  microscope (SEM). Shorter length-scales can be seen, although there is still 
  a limitation for viewing a relatively fragile material like wood, because to 
  see shorter and shorter scales you need to fire your electrons with higher 
  and higher energy. Beyond a certain point this causes damage to the specimen 
  you are trying to see. Secondly, for technical reasons to do with the way 
  electron microscope lenses work, SEM images can have a very large depth of 
  field. This gives images with a very striking and satisfying 
  three-dimensional character. 

  I will show samples of SEM images here: the next link gives a more extensive 
  picture gallery. I am grateful to Claire Barlow for doing the microscopy for 
  all these images. Figure 1 shows an image we saw earlier, a general view of 
  the cell structure of Norway spruce (Picea abies), a wood commonly chosen for 
  soundboards of stringed instruments of all kinds. Three red arrows identify 
  the three principal directions in the tree: the L direction runs vertically 
  in the tree, the R direction runs radially, and the T direction runs 
  horizontally around the tree trunk in a direction tangential to the annual 
  growth rings and the bark of the tree. 

  \fig{figs/fig-6f1b7c5a.png}{\caption{Figure 1. SEM image of the cell 
  structure of spruce, showing the three principal directions L, R and T.}} 

  Softwoods like Norway spruce have a relatively simple structure, dominated by 
  two types of cells. The majority are tracheids, long tubes running vertically 
  in the tree, tapering down to a point at both ends. Running perpendicular to 
  the tracheids, oriented in the radial direction in the tree, are rays. In 
  spruce the ray cells are typically stacked in vertical columns, just one cell 
  thick. You can see one of these stacks in Figure 1, just left of centre in 
  the lower part of the image. It has been sliced through by the knife cut used 
  to prepare the wood sample for the microscope. We will see a clearer example 
  in a moment, in Fig.\ 3. 

  In Fig.\ 1, the R arrow is shown pointing in the outward direction from the 
  centre of the tree. This is the direction in which the tree grows. New 
  tracheids are formed on the outer edge of the tree trunk, just below the 
  bark. In the spring the tree grows rapidly, making new tracheids that are 
  large and thin-walled. As the growing season progresses, the growth rate 
  slows down and the new tracheids have thicker walls and less internal space, 
  until growth stops altogether. It bursts back into action the following 
  spring. This modulation of the tracheid geometry is responsible for the 
  annual growth rings of the tree: it can be seen clearly in Fig.\ 1, and in 
  more detail in Fig.\ 2 which shows about 2 complete annual cycles. 

  \fig{figs/fig-dac41904.png}{\caption{Figure 2. SEM image of the cell 
  structure of spruce, showing a section in the RT plane, the ``end grain''.}} 

  Figure 3 shows a different view of all these features. This time the wood has 
  been cut approximately in the LR plane: this orientation of cut would 
  normally form the visible outer surface of the soundboard in a violin or 
  guitar. Tracheids make the obvious “tunnels” heading towards the top 
  right-hand corner of the image. Immediately on the ``roof'' of these tunnels 
  we see a ray: a stack of about 5 cells have been exposed by the cut, heading 
  towards the top left corner. There are other anatomical details visible in 
  these pictures, but I won't discuss them here: see the previous link. 

  \fig{figs/fig-571e5856.png}{\caption{Figure 3. SEM image of the cell 
  structure of spruce showing a detail in the LR plane. The scale bar here is 
  in micrometres (µm): 1~µm is a thousandth of a millimetre.}} 

  In section 10.1 I said that every measurement has pitfalls for the unwary, 
  with the possibility of misleading results. I will conclude this discussion 
  of SEM images with a few words about problems that can arise with those. A 
  normal electron microscope relies on the specimen being able to conduct 
  electricity, but of course wood is an insulator. In order to obtain images 
  like these, the specimen has to be coated with a thin layer of conducting 
  material, usually gold. Not enough coating, and the images will show 
  artefacts of ``charging'': the electrons that have been fired at the specimen 
  cannot escape, and they give an electrical charge which can distort the 
  image. To much coating, and fine details of the structure you are trying to 
  see might be covered up. 

  Another snag is that the SEM works best when the chamber containing the 
  specimen is pumped out to give a high vacuum. This removes air molecules that 
  might interfere with the electrons, but vacuum might also affect your 
  specimen. Wood contains water, so the structure you see might be altered by 
  vacuum dessication. The vacuum can also have the effect of sucking other 
  material out from the interior of the sample: the blob visible centre-right 
  in Fig.\ 2 might be resin that has been sucked out of the wood in this way. 

  It seems to have been in the 1980s that someone first had the idea of putting 
  a violin in a medical scanner, to reveal a different kind of “hidden detail”: 
  see for example the article by John Waddle and Steven Sirr [1]. Medical 
  scanners use x-rays and rely on a processing method called “computed 
  tomography”, so they are often known as “\tt{}CT scanners\rm{}”. A CT scan 
  can reveal many things: detailed shape and thickness information, and also 
  things like woodworm damage and old repair work. 

  \fig{figs/fig-66a21135.png}{\caption{Figure 4. Three old Italian violins 
  entering a CT scanner. Image copyright Sam Zygmuntowicz, reproduced by 
  permission.}} 

  CT scans formed part of the ambitious “Strad3D project”, in which three 
  famous old violins were subjected to every kind of test and measurement that 
  was available in 2006. All the results can be seen on \tt{}this web 
  site\rm{}. Figure 4 shows the three violins stacked up and entering the CT 
  scanner. Access to scanner time was still at a premium in those days, hence 
  the combined scan of three instruments at once. You can see a video of the 
  resulting scan \tt{}here\rm{}: go to the option “Three violins axial 
  compare”. 

  In the years since then, the technology has progressed in leaps and bounds. 
  Medical CT scanners are optimised for the particular business of imaging the 
  human body, but laboratory-based scanners have been developed that are 
  targeted at tasks like non-destructive testing of engineering components. 
  These can have significantly higher resolution than medical scanners, and the 
  processing software has also been developed differently to allow non-medical 
  aspects of behaviour to be examined. The result is that some stunning images 
  of musical instruments are now available. Figure 5 shows a screen shot from 
  \tt{}this YouTube video\rm{}, showing a “deconstructed” Stradivari violin. 

  \fig{figs/fig-679ca7c5.png}{\caption{Figure 5. Still from a video animating 
  the result of a high resolution CT scan of a Stradivari violin. Image 
  copyright violinforensic, Rudolf Hopfner, Vienna, reproduced by permission.}} 

  You can see many interesting details by studying Fig.\ 5. You can see 
  individual annual rings in the wood of the top, the back and the bass bar. 
  This shows how Stradivari chose and cut his wood. You can see the soundpost 
  (leaning at a slightly unexpected angle, presumably deliberate). On the 
  underneath of the top plate, around the end of the soundpost, you can see a 
  reinforcing patch put in at some stage by a restorer. In the lower left, you 
  see the label in the instrument -- but the one thing the CT scan does not see 
  is the writing on it! 

  Figure 6 shows something else that can be extracted from the high resolution 
  CT scan data. It shows a thickness map of the back plate of the violin, 
  something violin makers are always very interested in. Makers are also very 
  interested in the arching shape of the top and back plates. It used to be 
  common for violin makers to have in their workshops plaster casts of famous 
  violins like the one shown here. But these days, they may have 3D-printed 
  replicas of plates made from CT scan data. 

  \fig{figs/fig-dbfbffdb.png}{\caption{Figure 6. Thickness map of the back 
  plate of the Stradivari violin seen in Fig. 5, deduced from the CT scan data. 
  Numbers in black are individual thickness values, deduced from the scan. 
  Image copyright violinforensic, Rudolf Hopfner, Vienna, reproduced by 
  permission.}} 

  What are the snags that can cause misleading results in a CT scan? The main 
  thing is that x-rays aren’t very good with metal. If you watched the video 
  from the Strad3D project, you may have noticed vigorous “starburst” flashes 
  at certain points: these occurred where the viewing section passed through 
  metal objects like the fine-tuners attached to the tailpiece in order to tune 
  the top string. 

  A more subtle issue to be aware of in the back of your mind is that the 
  images you see are not a direct result of observation: they are reconstructed 
  from the raw data by sophisticated computer software. The programmer of that 
  software has made some choices, which influence what you see. As a simple 
  example, the colour in Fig.\ 5 is created by software. It has nothing 
  directly to do with the actual colour of the wood, it has been chosen for 
  good visual effect (very much like those famous images from the Space 
  Telescope). More detail of the procedure can be found in an article by Rudolf 
  Hopfner [2]. 

  There is another piece of equipment originally developed for non-destructive 
  testing, which has been applied to musical instruments. The method is called 
  pulse reflectometry, and it is useful for long, thin structures like 
  pipelines or cables. With a pipeline, for example, you send a pulse of 
  vibration or sound into the wall of the pipe, or through the fluid contained 
  within the pipe, and you record the reflections that come back. By analysing 
  the pattern, you can detect things like blockages, corrosion or breaks in the 
  pipeline. That can tell you what has gone wrong, and also where you need to 
  dig the road up to repair the damage. 

  This method has been applied to the rather different problem of 
  reconstructing the bore profile of musical wind instruments. The bore profile 
  is the most important thing to know about a wind instrument, either for the 
  purposes of making a replica of a historic instrument or for understanding 
  the acoustical behaviour of an instrument. A measurement of the bore profile 
  can also shed light on possible internal faults such as leaky valves in brass 
  instrument. 

  We can get an idea of how the method works by a hand-waving argument. Suppose 
  first that we have a cylindrical pipe with an abrupt change in 
  cross-sectional area, like the sketch in Fig.\ 7. Now suppose that a pulse of 
  acoustic pressure is sent into the pipe at the left-hand end. We will assume 
  that it is a plane wave, with constant pressure across any particular 
  cross-section of the pipe. When this pulse reaches the jump in cross-section, 
  a reflected pulse will be generated which will travel back down the pipe. 
  There will also be a transmitted pulse, which carries on into the second 
  section of the pipe. 

  \fig{figs/fig-9f7f8bb2.png}{\caption{Figure 7. Sketch of a circular pipe with 
  a sudden jump in cross-sectional area.}} 

  We can learn three things by observing the reflected pulse, with a microphone 
  near the left-hand end where the original pulse was sent in. The time delay 
  between the original pulse and the reflected pulse tells us how far down the 
  pipe the jump in section occurred. The amplitude of the reflected pulse can 
  be used to work out the ratio of the two cross-sectional areas at the jump, 
  using a standard piece of textbook theory. Finally, the sign of the reflected 
  pulse tells us whether the cross section jumped to a larger or a smaller 
  value than the original pipe: if there is a reduction of area, the reflected 
  pressure pulse has the same sign as the original pulse, but if there is an 
  increase in area the reflected pulse will be inverted. Putting these three 
  things together, we have full information about the jump in cross-section. 

  Now for the hand-waving part. In a musical instrument, the bore might have 
  jumps at some positions, but it will also have continuous changes along its 
  length. Well, we can imagine approximating any bore profile by a kind of 
  staircase profile, with small jumps separated by short sections of parallel 
  pipe. Each of those small jumps will generate a reflection which can be 
  analysed as I have just described. So with a bit of effort in the computer 
  software, we should be able to reconstruct the entire bore profile. Needless 
  to say, there are details of this processing that need to be thought about 
  very carefully: see for example the discussion by Sharp and Campbell [3]. 

  But the process can indeed be made to work reliably, and we can show an 
  example. Figure 8 shows the experimental setup. A trumpet is being tested. 
  The input acoustic wave is generated by a loudspeaker, controlled by a 
  computer. This is connected to the trumpet via a long length of tube with a 
  constant cross-section. The tube is coiled up in the picture: the sound waves 
  inside the tube do not care about gentle curves, a fact which is equally 
  relevant to the trumpet itself. The bore profile that will be reconstructed 
  is “unwound”. 

  \fig{figs/fig-30cfe18b.png}{\caption{Figure 8. A pulse reflectometry setup. A 
  trumpet is being tested. The input sound wave is generated by a loudspeaker, 
  visible in the top right corner. This is connected to trumpet by a long 
  length of copper tubing, and somewhere in the middle of that tube a 
  microphone is fitted: its electrical cable is visible. Image copyright David 
  Sharp, reproduced by permission.}} 

  Somewhere in the middle of the long tube, a microphone is placed to observe 
  the incoming sound pulse and also the reflected sound returning from the 
  trumpet. The reason for this long tube is to minimise the influence of 
  possible complicating factors. For example, the reflected wave from the 
  trumpet will travel back down the tube past the microphone, and then when it 
  reaches the loudspeaker it will be reflected back into the tube. The tube 
  between the microphone and the loudspeaker needs to be long enough that this 
  next reflection is not confused with the reflected wave from the trumpet, 
  which we are trying to measure as accurately as possible. 

  Figure 9 shows two reconstructed bore profiles of a trumpet. The red curve 
  describes the shape when no valves are depressed, so that the total length of 
  the tube is as short as possible. The blue curve shows the result of 
  depressing all the valves, to give the longest combined length of tube. The 
  profile is expressed in terms of an effective radius, on the assumption that 
  the cross-section is always circular. This will not be entirely true in 
  practice, but it makes no difference to the validity of this measurement 
  provided any local perturbations to the cross-sectional shape are small 
  compared to the wavelengths of sound that we are interested in. Wiggles can 
  be seen, especially in the blue curve, marking positions where the details of 
  the valve mechanisms make small changes to the effective cross-section. 
  Figure 10 shows a 3D rendering of the bore profile from the blue curve in 
  Fig.\ 9. In both Figs.\ 9 and 10, the flaring bell of the trumpet is not 
  shown out to its actual total radius: the assumptions behind the 
  reconstruction process start to break down when the flare becomes too rapid, 
  so the results are less accurate. 

  \fig{figs/fig-b1a378be.png}{\caption{Figure 9. Reconstructed bore profiles of 
  an Amati-Kraslice trumpet with no valves depressed, and with all three valves 
  depressed. Data courtesy of David Sharp.}} 

  \fig{figs/fig-2400f2e0.png}{\caption{Figure 10. A 3D rendering of the bore 
  profile of the trumpet shown as a blue curve in Fig. 9. The proportions look 
  a little odd because of different scales in the axial and cross-sectional 
  directions.}} 



  \sectionreferences{}[1] John R. Waddle and Steven A. Sirr, “X-ray 
  computerized transaxial tomographic analysis of stringed instruments”, Catgut 
  Acoustical Society Journal (series II), \textbf{3}, 2, 3—8 (1996). 

  [2] Rudolf Hopfner, “Hi-res revelations”, The Strad \textbf{129}, 1533, 54—59 
  (January 2018). 

  [3] David B. Sharp and D. Murray Campbell, “Leak detection in pipes using 
  acoustic pulse reflectometry”, Acta Acustica united with Acustica 
  \textbf{83}, 560–566 (1997). 