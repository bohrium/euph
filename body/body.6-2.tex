

  Before we start to grapple with psychoacoustics, it is useful to know a 
  little about how human hearing works. Figure 1 shows a diagram of the main 
  components. Sound waves from the outside world enter through the external 
  part of the ear, into the auditory canal. This is an acoustic duct, which 
  functions rather like an ear trumpet to concentrate the energy of the sound 
  wave before it reaches the tympanic membrane, or eardrum. The sound level at 
  the eardrum is boosted by up to 15 dB compared to the incident sound wave, 
  peaking around 3 kHz. The eardrum is a membrane that is set into vibration by 
  the incoming sound wave. The next part of the system is the middle ear, 
  containing three tiny bones (the malleus, incus and stapes) which act as a 
  kind of lever system to increase the force available from the vibration. 

  Now we enter the most important region, the inner ear. The complicated shape 
  drawn in purple indicates a cavity within the bone of your skull. It has two 
  main components. The semi-circular canals are associated with our sense of 
  orientation and balance: it has components which behave rather like the 
  accelerometers and gyros that you may have inside your mobile phone, to sense 
  accelerations and rotational movement. 

  The component we are most interested in is the second one, the cochlea. It is 
  wound up into a shape like a snail shell: indeed, the word ``cochlea'' 
  derives from the Greek for a snail. The cochlea is a bony cavity, and it is 
  filled with fluid. It has two ``windows'': flexible membranes that hold the 
  fluid in, but able to bulge in and out . The final bone of the chain in the 
  middle ear, the stapes, presses against one of these windows, the oval 
  window, and it pushes it in and out as a result of the incoming sound wave. 
  The fluid inside the cochlea is virtually incompressible, so to accommodate 
  this movement there is a second window, the round window, which can bulge in 
  the opposite sense to the oval window. 

  To see the point of this, we need to know a bit about what is inside the 
  cochlea. Figure 2 shows an ``unrolled'' view of it. Glossing over a lot of 
  complicated anatomical details (\tt{}see Wikipedia\rm{} if you want to know 
  more), the crucial component for our purpose is called the basilar membrane. 
  This membrane bridges across the cochlea from side to side. The oval window 
  introduces acoustic disturbances on one side of it, while the round window 
  allows them to ``escape'' on the other side. The resulting fluid flow in the 
  cavity exerts a force across the basilar membrane, making it vibrate. 

  Now for the important part: the mechanical properties of the basilar membrane 
  vary, by an enormous factor, along the length of the cochlea. At the end near 
  the two windows, it is narrow and stiff, while at the far end it is wider and 
  floppier. The result is that the basilar membrane functions as a kind of 
  mechanical frequency analyser. If the incoming sound wave is sinusoidal, the 
  vibration response of the basilar membrane is concentrated in a particular 
  region: near the base for a high frequency, near the far end for a low 
  frequency. Some positions of maximum response for different frequencies are 
  indicated in Fig.\ 2 (but note that more recent research suggests that the 
  lowest indicated frequency may be a bit misleading: the peak frequency at the 
  end of the cochlea may be 50 Hz). 

  The basilar membrane carries a large number of hair cells, which are the 
  neural sensors for vibration. So, at least to an extent, the information 
  reaching the brain along the auditory nerve is already sorted out into its 
  frequency content, simply because nerve fibres originating from hair cells 
  attached at different positions on the basilar membrane will tend to respond 
  most strongly to sounds in different frequency ranges, appropriate to the 
  position of each one. We will see in section 6.4 that some aspects of this 
  mechanical filtering action of the basilar membrane carry over rather 
  directly into the way we hear. 

  When you listen to sound that is sufficiently quiet, this mechanical response 
  of eardrum, middle ear and basilar membrane involves small-amplitude 
  vibration, and can be described reasonably well by the kind of linear theory 
  we have been using up to now. But the sound does not need to become very loud 
  before the response starts to exhibit progressive nonlinearity. In any case, 
  from here on the process of hearing is very definitely nonlinear. The nervous 
  system communicates in the form of electrical pulses, so the brain is 
  essentially a digital device. The information from individual hair cells is 
  coded, somehow, in the rate and detailed timing of the pulses generated and 
  sent off along the auditory nerve by the neurons connected to them. 

  However, this is not the end of mechanical effects that influence hearing. 
  Not all the hair cells are sensors. Some of them, indeed the majority of 
  them, behave like loudspeakers: they cause additional motion of the basilar 
  membrane in response to nerve signals coming back from the brain or from a 
  more local reflex action. These active hair cells are known as outer hair 
  cells, whereas the sensing hair cells are inner hair cells. Sometimes, the 
  action of the outer hair cells results in sound coming out of the ears: 
  so-called \tt{}otoacoustic emissions\rm{}. Some kinds of tinnitus are the 
  result of real sounds generated in the ear in this way. The details of the 
  excitation of the basilar membrane by these active hair cells are still a 
  matter of current research, but it appears that this process is crucial to 
  the phenomenal range of loudness that our ears are capable to responding to. 

  For more detail of all aspects of cochlear mechanics, see the comprehensive 
  review by Ni et al. [1]. 

  There is one more thing we need to mention about ears and hearing. So far, we 
  have talked about how one single ear works, but of course we have two ears. 
  It is by combining the information from them that we are able to tell the 
  direction a sound is coming from — at least up to a point. Figure 3 shows in 
  schematic form the head of a listener. The lower part of the figure shows a 
  top view of the head, with a sound wave passing by in the direction of the 
  red arrow. 

  We can see that the sound wave interacts with the listener’s ears in slightly 
  different ways, and these give two possible clues to the direction the sound 
  has come from. First, there is a timing difference: the sound wave reaches 
  one ear before the other. If the sound is a continuous signal like a sine 
  wave, this will produce a phase difference between the signals at the two 
  ears. On the other hand if the sound contains transient events like clicks, 
  the listener’s brain may be able to detect the time difference directly by 
  comparing the signals from the two ears. 

  The second clue comes from loudness. In the case shown in Fig.\ 3 the sound 
  reaches the listener’s right ear directly, but in order to reach the left ear 
  it has to get round the head. There will be some kind of sound shadow behind 
  the head, so that this left-ear signal will be less loud. The loudness 
  difference will depend on the Helmholtz number associated with the size of 
  the head and the wavelength of the sound, as was discussed in section 4.1 
  (see Fig.\ 8 there). 

  Our brains make use of both these signals to estimate the direction a sound 
  is coming from. The details are, as ever, complicated: see the book by Brian 
  Moore [2] for more information. One consequence of the use of loudness and 
  timing differences between the ears is a curious but familiar experience when 
  using headphones. If the sound is the same in both ears (a monaural 
  recording), then obviously both the timing and the loudness are exactly the 
  same in both ears. Your brain then makes the logical deduction that the sound 
  source must be inside your head, and that is the perception you will usually 
  have. 

  There is one more important twist in the way we localise a source of sound. 
  We are very often listening to sounds within a room, whether that is a 
  domestic room or a concert hall. This means that you hear each sound many 
  times over: the direct sound from the source is followed by echoes from the 
  walls, the ceiling, or the furniture. Our brains have evolved to cope with 
  this potential confusion of direction — presumably not because of our need to 
  hear music clearly, but because of an evolutionary need in the distant past 
  to know where a predator is approaching from (echoes can come from trees or 
  rocks, as well as from the walls of a room). 

  At least for echoes that follow the direct sound within a short time (about 
  50--100 ms), our brains recognise that the later arrivals are just further 
  copies of the same sound, and they are fused together in your perception. You 
  hear the sound as coming from the direction of the first arrival, which is 
  the direct sound path. The later echoes add to the loudness, and perhaps the 
  clarity, of the perception, but you are not explicitly aware of them as 
  separate echoes. This effect was first described by Lothar Cremer, who we 
  will meet in chapter 9 because he also worked on how violin strings vibrate. 
  Cremer called it the “law of the first wavefront”; but these days it is more 
  commonly called the “precedence effect”. To see more about it, go to 
  \tt{}this Wikipedia page\rm{}. 

  One immediate application of the precedence effect is in the design of sound 
  reinforcement systems for lecture halls, churches and so on. There will 
  usually be a number of loudspeakers spread around the space. If the sound to 
  each loudspeaker is delayed by an amount corresponding to the travel time of 
  sound from the microphone, they will behave like echoes. Listeners should 
  then hear the sound as if it came from the microphone position, enhanced in 
  loudness and clarity by the reinforcement. 

  Echoes that arrive after about 50--100~ms are too late for the precedence 
  effect. They are not fused into the single perception, and instead contribute 
  to that sense of reverberation which is a particularly familiar quality of 
  sound in large spaces like cathedrals. If a single strong reflection arrives 
  after this time, it may be heard as a discrete echo — and then you are able 
  to determine the direction it has come from. 

  The precedence effect relies to a considerable degree on our ability to 
  discriminate an early echo as coming from a different direction to the direct 
  sound. This ability in turn relies on having information from both ears. A 
  consequence is that if you remove this extra information by listening to a 
  monaural recording or by listening through a keyhole, the echoes seem more 
  more prominent. The impression is that the sound is more reverberant, and 
  details are less clear. This is one reason why recording studios are usually 
  very “dry” spaces, with a lot of sound absorption and little reverberation. A 
  recording made in a less dead space is likely to sound unexpectedly 
  reverberant when it is replayed through a loudspeaker, because the 
  information about the direction of early echoes has been lost. 



  \sectionreferences{}[1] Guangjian Ni, Stephen J. Elliott, Mohammad Ayat, and 
  Paul D. Teal: ``Modelling cochlear mechanics'', BioMed Research 
  International, Volume 2014, Article ID 150637, 42 pages 
  \tt{}http://dx.doi.org/10.1155/2014/150637\rm{} 

  [2] Brian C. J. Moore; ``An Introduction to the Psychology of Hearing'', 
  Academic Press (6th edition 2013). 