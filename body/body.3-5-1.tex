  The kind of holographic image illustrated here is made by taking a long 
  exposure of the object while it is being driven in steady sinusoidal 
  oscillation at the resonant frequency of a vibration mode. A laser beam is 
  split in two: one beam illuminates the vibrating object, while the other is 
  sent by a different route to act as a reference beam. The hologram is formed 
  on a piece of photographic emulsion coated on glass, where the two beams are 
  brought back together. The image is formed as an interference pattern between 
  the light reflected from the object and light from the reference beam: when 
  the two beams are in phase, the combination is bright, but when they are 
  exactly in opposite phase, cancellation occurs and the combination is dark. 

  The recording period covers many cycles of the object, so the holographic 
  image involves a collection of all the object positions between the two 
  extremes of its motion --- a sort of “blur”. However, the object spends most 
  of its time at the two extremities of its motion, and it is these two 
  positions which contribute the greatest to the holographic image and it is 
  essentially these which create the interference patterns (the “fringes”) 
  observed in the final image. The fringe contrast (and hence visibility) falls 
  off for higher- order fringes because of the “blur” (the fringe intensities 
  are actually described by the square of the zero-order Bessel function $J_0$, 
  which we will meet in section 3.6.1 in connection with vibration modes of a 
  drum: see Fig.\ 2 in that section). The fringes map out contours of equal 
  vibration amplitude with adjacent bright (or dark) fringes representing a 
  further (approximately) quarter-wavelength amplitude change. Nodal lines 
  stand out as very intense fringes: they are brighter than others because 
  there is no “blurring” in these positions. 