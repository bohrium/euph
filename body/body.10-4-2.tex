  \textbf{A: Philosophy} 

  Making decent measurements is a skill of a similar kind to making a decent 
  instrument. Doing it by recipe is like making a violin from instructions in a 
  book — it will work, but it probably won’t be great. Some people have more 
  natural talent than others, but all beginners at instrument making or 
  acoustical measurement benefit from occasional hands-on guidance from someone 
  with more experience. In the case of measurements the issue is not so much 
  “which buttons to press when”, but how to recognise quickly that something 
  looks wrong, then concentrate on fixing that rather than wasting time 
  collecting a lot of misleading data. 

  As was emphasised in section 10.4B, you should always ask yourself the key 
  questions “What do I need to know?” and “How well do I need to know it?” 
  before starting any test project. Asking these questions can greatly simplify 
  the selection of a test method. For example, a test which only requires 
  tracking of resonant frequency changes or relative level changes is much 
  simpler than one that involves absolute admittance levels, which requires a 
  fully calibrated force and velocity measurement system. 

  \textbf{B: Digital choices} 

  You will no doubt use some kind of FFT package, which can sample your signals 
  and turn them into frequency spectra. In any of these packages, there are a 
  couple of choices you will need to make before you even collect any data: the 
  sampling rate, and the length of time to sample for. Both have important 
  consequences. 

  The sampling rate governs the frequency range of your tests: your top 
  frequency will be half the sampling rate, known as the Nyquist frequency. If 
  the signal you are measuring contains significant energy at frequencies above 
  the Nyquist frequency, you may get seriously misleading results because of a 
  phenomenon called aliasing. Figure 1 shows an example. The black curve shows 
  a sine wave with a frequency of 4.5~kHz. The red stars show what happens if 
  you sample this sine wave with a sampling rate of 5~kHz. The Nyquist 
  frequency is then 2.5~kHz, lower than the frequency of our signal. You can 
  see that something remarkable has happened: the red stars do indeed dot out a 
  sine wave, but it is at the wrong frequency, a much lower frequency than it 
  should be. 

  Specifically, the frequency marked by the red stars is 500~Hz, which is the 
  difference between the sampling rate and the true frequency of the sine wave. 
  This aliasing effect is not just a mathematical artefact: you can hear it. 
  Sound 1 is a sine wave which sweeps up in frequency from zero to 7~kHz. Sound 
  2 is the result of sampling this same sine wave sweep at a sampling rate of 
  5~kHz. The note rises exactly as before until it reaches the Nyquist 
  frequency, 2.5~kHz. It then turns round and goes back down again! Eventually 
  this aliased sound reaches zero frequency (when the true frequency reaches 
  the sampling rate). It then turns round again and starts rising. The case 
  plotted in Fig.\ 1 occurs shortly before this second reversal, when the sweep 
  reaches 5~kHz and you hear a tone at 500~Hz. 

  You absolutely do not want aliasing to be going on in your measurement! 
  Frequency peaks will appear in entirely the wrong places. So you must choose 
  a sampling rate high enough that all the frequencies making up your actual 
  signal lie below the Nyquist frequency. But how do you know, before you have 
  started making the measurement? There is a simple trick. Make a guess, and do 
  a measurement at that sampling rate. Save it. Now increase the sampling rate 
  — for example, you might double it. Repeat the same measurement, and compare 
  the spectrum with the one you saved. Do they look different? Does the new 
  spectrum show any significant output in the frequency range which is above 
  the Nyquist frequency of the first measurement? If either answer is “yes”, 
  the original sampling rate was too low. Now double the rate again, and 
  repeat. By exploring in this way, you can quite rapidly find an acceptable 
  sampling rate. 

  Next, you have to decide how long to collect data for, or equivalently how 
  many samples to collect. This sampling time governs the frequency resolution 
  of your results, in a very straightforward way: the step between frequency 
  points will be the inverse of your sampling time. So if you collect data for 
  1~s your resolution will be 1~Hz, with 2~s it will be 0.5~Hz, and so on. 
  Figure 2 shows a typical hammer-tap response of an instrument body (a banjo 
  in this particular example). You can see that the response has decayed to low 
  levels by 0.5~s, and if this is the complete data sample, the resolution will 
  be 2~Hz. 

  Now you have to consider a trade-off. Higher resolution seems desirable 
  because it leads to smoother plots, but it requires you to store a larger 
  amount of data, and more insidiously there are implications for the amount of 
  noise included in your signal. Figure 2 shows why. If you collected data for 
  2~s, in order to increase your frequency resolution to 0.5~Hz, you would be 
  including an extra 1.5~s in which the real signal was essentially zero. But 
  the actual measured signal will never be zero, because there will always be 
  some level of noise — both physical noise being picked up by your sensor, and 
  electrical noise from your instrumentation. So the price of the increased 
  resolution will be extra noise. 

  A standard fix for this is called “zero padding”: you set a threshold level 
  below which you think the signal will be dominated by noise, and whenever the 
  measured signal falls below that threshold, you replace the measured values 
  by zero. This trick is particularly important for the force signal from an 
  impulse hammer. The real force is only non-zero for the short time when the 
  hammer is in contact with the instrument. Outside that time the force must be 
  zero, so it makes sense to detect that contact time via a threshold, and then 
  zero out all the rest of the signal before you do your FFT to deduce the 
  spectrum of the hammer force. 

  Returning to Fig.\ 2, it is easy to see that there is also a minimum time for 
  which you should collect the data. For this particular example, suppose you 
  only collected 0.1~s of the signal. You would miss quite a significant part 
  of the decay. This lost data will distort the frequency spectrum, as well as 
  giving you a very low resolution of 10~Hz. You need to collect enough data to 
  catch “most” of the decay, although the exact meaning of “most” will depend 
  on the actual noise level in your measurement. But as a rule of thumb, the 
  amount plotted in Fig.\ 2 feels about right for the minimum sampling time. 

  \textbf{C: What could possibly go wrong?} 

  C.1: Support fixtures 

  There are other things you need to think about when designing your 
  experiment. You probably need to support the instrument you are testing in 
  some way. All support fixtures are likely to change the behaviour to a 
  greater or lesser extent, and you should obviously seek to minimise that 
  influence. The most realistic, of course, is to do a measurement while a 
  player holds the instrument in the normal manner. But that is impractical for 
  some measurements, and in any case it will introduce a source of variability 
  if you want to compare results for many instruments, collected over a long 
  period of time. No two players will hold the instrument in exactly the same 
  way, and even if you use the same player there no guarantee that they will 
  hold an instrument in precisely the same way several months apart. 

  So usually people design some kind of supporting fixture. You can try to make 
  something that mimics a player’s hold: perhaps a foam-lined support around 
  the neck where a hand might hold it, perhaps some kind of soft grip on the 
  chinrest of a violin… The main thing is to be alert to the issue — ideally 
  you should try to play the instrument, at least a little, while it is in the 
  rig and with any attached instrumentation in place. Does it sound normal, or 
  is it muted or muffled? Is something buzzing or rattling? An example of a 
  suitable holding arrangement is shown in Fig.\ 3. This treble viol is resting 
  on soft foam on its end block, and is supported by a foam-lined ``hand'' on 
  the neck. 

  Another approach that is often used is to suspend the instrument from some 
  kind of frame using rubber bands. This approach can give good isolation of 
  the instrument from the holding fixture, but it is not without its own snags. 
  Rubber bands are basically stretched strings, and they have their own 
  resonances. Sometimes, those resonance frequencies show up in the measured 
  frequency response, and that can be puzzling if you don't think of the rubber 
  bands as a possible cause. 

  Another snag is that the instrument, behaving like a rigid mass, will have 
  several modes in which it bounces or twists on the rubber bands. These are 
  normally at very low frequency so they may not matter as resonances, but they 
  can interfere with the measurement process because they make take a long time 
  to settle after each hammer tap. Your next tap may miss its mark, because the 
  instrument is still bouncing around. 

  There is one further issue about support structures if you want to use the 
  wire-break method of testing. You don't want the instrument to fall over when 
  you pull the wire to break it! So for that kind of testing you need a rather 
  firm support --- rubber band suspension probably won't work very well. If an 
  instrument is being held by a player in the usual way, that generally gives a 
  sufficiently firm support. But artificial support frames need to be a bit 
  more robust than the one seen in Fig.\ 3. For example, the lower bouts of 
  this viol could be held between foam-lined side supports, mirroring the way a 
  player of this instrument would grip the body with their legs. 

  C.2: String damping 

  If you are testing a stringed instrument, you need to think carefully about 
  whether you want the strings to be damped or not. It all depends on what you 
  want to do with your measured frequency response. There is, of course, an 
  argument in favour of leaving strings undamped, since that is what happens to 
  the non-played strings of an instrument during normal performance. So if your 
  aim if to characterise the sound of the instrument, in some sense, perhaps 
  you want to include the contribution from the ringing strings. 

  However, for other purposes you definitely do not want undamped strings: they 
  are at best a nuisance and a distraction, and at worst they invalidate what 
  you are trying to do. Think back to the synthesised guitar and banjo sounds 
  presented in earlier chapters. Many of those used measured body response, but 
  the strings were added in the computer program. This allowed variations in 
  string choice and playing details to be simulated, while keeping the body 
  response constant. In other cases, variations were made in the body response 
  while keeping the strings and playing details the same. None of that would 
  have been possible if the body measurement already included the effects of 
  coupling to undamped strings. 

  More commonly, you do not want undamped strings in a frequency response 
  measurement because they interfere with the measurement. The strings ring on 
  for a lot longer than the body modes, so you need to allow a significantly 
  longer sampling time in your measurement protocol. This long ringing gives 
  another effect: the strings produce extra narrow peaks and dips in the 
  response, determined by the resonances of the strings. Figure 4 shows an 
  example of the bridge admittance of a violin, with and without string 
  damping. If you want to identify peaks as body modes, and perhaps to do modal 
  analysis (which we will come to in section 10.5), these extra peaks are a 
  distraction: notice how the string modes have had a particularly strong 
  effect on the low-frequency ``signature modes''. Furthermore, if you want to 
  compare your measurements with ones done by other people, the comparison is 
  cleaner and simpler without string effects: the other instruments may have 
  been fitted with different strings, they may not have been tuned the same, 
  and so on. 

  However, it is important that the strings are in place and under normal 
  tension during a measurement. The violin gives a particularly clear example 
  here. The bridge is held in place by the strings, and the static stresses 
  produced by the string tension also influence other aspects of the behaviour. 
  The tightness of the soundpost would be different without string tension, for 
  example. Also, the geometry of the violin means that the longitudinal 
  stiffness of the strings is a contributing factor to the vibration modes of 
  the body. We met some consequences of this kind of influence of the strings 
  when we discussed the banjo in section 5.5 — the banjo shares with the violin 
  the arrangement of a “floating” bridge and a tailpiece to anchor the strings. 

  If you damp the strings, you need to do it carefully. If you were to do it by 
  putting foam between the strings and the fingerboard, for example, you would 
  be adding damping to the body modes as well as damping the strings. A method 
  that often works very effectively is to weave a piece of paper or light card 
  (like a business card) over and under the strings. This can be done without 
  making contact with the fingerboard or body, and it can produce very 
  effective damping via frictional rubbing of the strings against the card. An 
  example can be seen in Fig.\ 3. A useful check is to strum the strings with 
  the card in place — if you can still hear a pitch associated with a string 
  frequency, the damping is not enough. 

  C.3: Checks 

  There are some things you can do to check that a measured frequency response 
  looks reliable. The first is something we mentioned back in section 5.1.1. If 
  you are doing a hammer test and your software package allows it, you should 
  repeat each test several times, and calculate an average response. This will 
  reduce the effects of noise, and it also gives you the coherence function, 
  which is an indication of how well your measurements are behaving. Coherence 
  close to 1 (or 0~dB if you are looking at plots with a decibel scale) 
  indicates that your measurement passes at least some of the tests for 
  reliability. 

  The coherence plot often gives a very direct indication of the bandwidth over 
  which your frequency response can be trusted. At high frequencies your hammer 
  taps will not be putting very much energy into the instrument, and this will 
  lead to an increase in noise and a reduction in coherence. Figure 5 shows an 
  example. In this case, the coherence suggests that the measured bridge 
  admittance is reliable up to about 6~kHz. 

  There are two other checks that can be made if your measurement is of a 
  structural response, such as admittance. If you are doing a driving-point 
  measurement (tapping essentially at the same position as the sensor) then the 
  frequency response function has an important property which can be checked if 
  your software package allows you to view the phase response, or else the 
  complex version of the function. Figure 6 illustrates this property for an 
  example of the driving-point admittance at the bridge of a violin. It shows 
  two alternative views of the same response: on the left the real and 
  imaginary parts of the complex admittance are plotted, while on the right the 
  phase angle is plotted. 

  Notice that the real part of the admittance (blue curve on the left) is 
  always positive, while the phase angle always stays within the range between 
  $-90^\circ$ and $+90^\circ$. These both follow from the requirement that if 
  you apply a force to the violin at the measurement point, energy must always 
  flow into the violin, never out of it. Any measurement of a driving-point 
  admittance should show this behaviour, and it is a useful thing to check. One 
  thing you may find is that the real part of your admittance is always 
  negative, rather than always positive. This is simply an indication that when 
  you come to calibrate your measuring system (see subsection E) you will need 
  a negative calibration factor to correct the error. 

  If your measurement is of accelerance rather than admittance (i.e. you are 
  measuring acceleration rather than velocity in response to a hammer tap), 
  there is an equivalent property but the details are slightly different. It is 
  now the imaginary part, not the real part, which must always be positive. The 
  equivalent condition on the phase angle is that it must lie in the range 
  between $0^\circ$ and $180^\circ$. 

  On the other hand, if your tapping position is not the same as your sensor 
  position then you can do a different check. There is a very general 
  reciprocal theorem, which tells us that if you swap the positions of tapping 
  and measurement, the frequency response should be exactly the same. So do the 
  measurement you planned, then move your sensor to where you were tapping, and 
  tap with your hammer at the original sensor position. Compare the results of 
  the two measurements: they should be identical (within the accuracy limits of 
  the measurement you are doing). 

  \textbf{D: Reproducibility} 

  All measurements have noise and accuracy limits. To avoid over-interpreting 
  your measurements, and as a way of getting to know your particular 
  measurement rig and its quirks, it is really important to go through 
  something like the checklist given below. The important thing is the idea and 
  the mindset, rather than the exact numbers of how many of each kind of test 
  you do. But don’t shortcut this learning and checking stage, the time it 
  takes will be amply repaid in mistakes avoided later. 

  D.1: Repeat tests 

  Set your rig up for the measurement you intend. But don’t just do the test 
  once, repeat the measurement several times. Display the results on top of 
  each other — on the screen if your software allows it, but if necessary print 
  the plots out, stack them and hold them up to a window. Look carefully at the 
  detailed comparison. How different are they? Close enough for what you wanted 
  to know, or would you like the reproducibility to be better? Is the 
  difference more or less uniform? Perhaps it is different in different 
  frequency ranges, or the peaks are more reproducible than the dips, or there 
  is some other pattern. Patterns are always trying to tell you something….so 
  think about it. Can you guess what kind of thing might cause this pattern? Is 
  it something you could tighten up in your procedure? Try it! 

  You don’t have to go through this rigmarole every time you make a test, but 
  you should do it occasionally — more often when you are learning, less often 
  as you gain experience. 

  In this repeat-testing process, think about the limits on the resolution of 
  the measurement. If the frequency step is 1.5 Hz then looking for changes 
  smaller than about 6 Hz will be problematic. Set the FFT analyzer so that it 
  can clearly resolve the differences that are being measured, to the best of 
  the analyzer’s capabilities. This is no different than selecting a scale for 
  weighing things: if the scale only reads out in tenths of a gram, weighing 
  items that are on the order of a half of gram or less will have poor 
  resolution. 

  D.2: Deliberate variations 

  Now do a deliberate experiment about variability. Keep the same test 
  instrument (or wood sample, or whatever you are trying to measure), and 
  repeat the test with “irrelevant” things deliberately varied. The first thing 
  to do is to simply take the instrument (or whatever) out of the rig, put 
  everything away, and then take it out again as if it was another day’s 
  testing, and try to put it all back together exactly as you had it before. 
  The next thing to do is to move the rig around in the room, or to a different 
  room. Move the furniture around a bit. Choose days with different temperature 
  and humidity. Anything else you can think of. For all these tests, ask the 
  same kind of questions as before. How repeatable are the results? Are there 
  patterns to the variation? 

  Now you will have gained some experience about how far you can trust your 
  measurement, and in the course of doing that you will also have got better at 
  controlling what you are doing. Just keep thinking of the analogy with 
  instrument making, and remember your first instrument: you were proud of it 
  at the time, but many established makers would like to get that first 
  instrument back and destroy it (or at least take their label out). Your first 
  measurements may be similar. There is no shame in that, everyone is the same. 

  The caution can be extended: when you make your second instrument the “it 
  really shouldn’t take this long” voice is heard in the back of your mind. As 
  it in many things in life, and for measurements in particular, ‘haste makes 
  waste’. 

  A good idea for a final step in checking out your test rig and data analysis 
  processes before you get down to serious measurement is to try to reproduce 
  the results of a known test. This is best done as a non-destructive test such 
  as adding lumps of mass to the instrument. Look around for a simple 
  experiment someone else has done and see if you can get comparable results. 
  The violin in your rig is different from the one tested by the previous 
  author, so what you want to see is a similar trend, and broadly similar 
  numbers. 

  D.3: Real tests 

  Now get back to what you were trying to do. Probably you wanted to measure 
  differences between instruments, or changes after some procedure. Are the 
  differences you measure bigger than the ones you saw in these reproducibility 
  tests? If they are, you (probably) have meaningful data. If not, you may 
  still have meaningful data, but it isn’t obvious. Look for the patterns again 
  — does the deliberate variation produce a different pattern from the ones you 
  saw before, or does it look suspiciously similar to the effect of the weather 
  changing or moving the furniture? 

  Hypothesis testing is the best way to explore the data. A hypothesis is based 
  on your current understanding and expectations. A hypothesis is a guess, 
  hopefully an educated guess. It maybe the null hypothesis “If I do this 
  nothing will change” or based on conjecture “If I block one f-hole A0 will 
  change by a fifth”. The hypothesis may be right or wrong. If the hypothesis 
  is correct you most likely understand how the system is working (great!). If 
  the hypothesis turns out to be incorrect, what you learn from the experiment 
  will help you pose a new hypothesis. After you have checked the hypothesis, 
  look around — did something else change in a pattern? For example, in most 
  cases changing something that moves a low frequency mode will also change the 
  system at high frequencies. 

  \textbf{E. Calibration} 

  You have now learned to make measurements that you understand and trust, at 
  least to an extent. Now you may want to compare with someone else’s 
  measurements of what is supposed to be the same thing. Maybe something 
  published, or maybe you are exchanging results with a friend with a similar 
  test rig. The first thing you need is some kind of calibration procedure, to 
  fix the absolute level of your plots so that it makes sense to compare. The 
  essence of any calibration procedure is to apply your measurement procedure 
  exactly as you were doing it, with no changes whatever, to measure something 
  that you already know the answer to. You compare what you get with what you 
  expect. If all is well, the pattern should look right, but you will need to 
  apply a scaling factor to your measurements to convert to the right answer. 
  This is the calibration factor for your rig: write it down and apply exactly 
  the same factor to all measurements made with the same rig and the same 
  settings. 

  If you change a setting by twiddling a knob or switching a range switch, you 
  should repeat the calibration experiment — don’t trust the manufacturer to 
  have marked the dial accurately enough, check it out yourself. Any 
  instrumentation rig that has ‘knobs’ which vary the gain continuously (as 
  opposed to switching between preset ranges) will be a constant source of 
  problems with maintaining calibration. One approach is to arrange to work 
  with the knob right at the top of the range, which is something you can 
  reliably repeat every time. Unfortunately top of the range settings can have 
  other consequences related to signal limiting or signal distortion. The best 
  advice to work around the ‘knob’ problem is find a good safe setting, then 
  tape over the knob securely, do your calibration measurement in that 
  condition, and never touch the knob again! 

  E.1: Absolute and relative calibration 

  There are two levels of calibration you might do. The “gold standard” is an 
  absolute calibration, which turns your results into internationally agreed 
  units: for example admittance (or mobility) in N~s/m (or s/kg, which is the 
  same thing although it looks different). If you can easily do such a 
  calibration, as you can for admittance, then it is obviously the best thing. 
  For many measurements, especially for sound pressure measured by a 
  microphone, it is not so easy to do an absolute calibration using reasonably 
  cheap equipment. Then you have to think what you are really trying to do. Do 
  you want to express your radiated sound power in absolute terms (in 
  milliwatts, for instance)? Then you need to bite the bullet and find a way to 
  get an absolute calibration. But perhaps you just want to be able to compare 
  measured sound levels with someone else’s results from a similar rig, being 
  sure that differences are correctly captured. Then all you really need is the 
  relative calibration factor which would convert your friend’s results onto 
  the same scale as yours. 

  E.2: How do you do it? 

  There is a choice of two approaches: calibrate each part of the system 
  separately, or do an overall calibration. The second approach is usually much 
  easier. With enough effort, you can calibrate each separate piece of 
  equipment — for example accelerometer, hammer, conditioning amplifiers, 
  computer card, and software FFT algorithm — then combine all these factors to 
  give the overall factor for your frequency response function measurement. 
  Some of these factors may be supplied by manufacturers, but the others may 
  take some effort to determine. The chain is no stronger than its weakest 
  link, so this approach is quite error-prone if you miss a step out or get one 
  thing wrong. 

  Usually, it is much simpler to do a single calibration for your entire 
  frequency response function in one go. For an absolute calibration, you need 
  to find a system for which the particular property you are measuring is 
  already known — either because of some fundamental law of nature, or because 
  someone else has already calibrated it using some standard test. 

  For admittance, you can use Newton’s law of motion “force equals mass times 
  acceleration”. If you measure a system which is essentially just a mass 
  moving in a straight line, then the acceleration per unit force is simply the 
  inverse of the mass, which you can determine with a weighing scale (but 
  remember that the standard unit of mass is the kilogram, not the gram, so 
  express the mass in kilograms!). In practice your mass might be suspended on 
  strings as a pendulum, or supported on soft foam. 

  You have to be careful that your hammer tap really does just make it move in 
  a more-or-less straight line, without any significant rotation or 
  cross-direction motion. You must also remember to choose a mass which is 
  convenient to measure using the same settings of gain etc. that you use for 
  your actual measurements. But then with care you can obtain an absolute 
  calibration and compare your results with, for example, the calibrated bridge 
  admittances of violins and guitars seen in earlier chapters here, or with 
  published ones by researchers like Erik Jansson or George Bissinger. (To 
  compare with Jansson measurements you need to read the small print — his 
  plots are not given in calibrated units, but he gives the required 
  calibration correction in the text.) 

  For a relative calibration of a hammer-to-microphone measurement, for 
  example, you need to choose some kind of stable and easily reproducible 
  system which makes a noise when tapped. By agreeing to tap in a given 
  position and place your microphone at certain place, you should get the same 
  answer every time (but check it out by the approach of section 2). Once that 
  is verified, then the same structure or a careful copy of it can be tested by 
  your friend. You should obtain matching frequency response plots, except for 
  a scale factor (i.e. a shift on a dB scale). That scale factor is what you 
  need to convert your measurements ready to be compared. 

  Pay careful attention in two-channel measurements where one spectrum is being 
  divided by another — it is easy to get the input and output cables swapped. 
  Mobility (velocity over force, admittance) is quite different from impedance 
  (force over velocity). Whereas mobility peaks at resonances, impedance peaks 
  at anti-resonances. If the signature modes have suddenly jumped when you do a 
  measurement on your standard test article, check the channel assignments: 
  another reason to add that standard test article to your test rig kit bag. 

  A final note: with all these calibration factors, of any kind, it is fatally 
  easy to multiply by the factor when you meant to divide, or vice versa. You 
  need to be awake when you think about how to do this! It always sounds 
  obvious when someone explains, but this is a very common mistake. 