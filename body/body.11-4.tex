

  In the previous section, we investigated the lowest notes of one cylindrical 
  reed instrument (the clarinet) and one conical one (the soprano saxophone). 
  In the interests of learning something about the playing behaviour of such 
  instruments, we used a reasonably realistic model of the reed mouthpiece in 
  order to perform computer simulations. However, in both cases our model of 
  the instrument tube was highly idealised: a perfect cylinder and a perfect 
  truncated cone respectively. In this section, we move closer to real 
  instruments: we will look at a new set of simulations using the same reed 
  model, but making direct use of measured input impedance of a clarinet and a 
  saxophone, fingered for the lowest note with all tone-holes closed. 

  \samsection{A. Input impedance, modes and modal simulation} 

  The first step is to extract information from the input impedance in a form 
  we can use in a simulation model. There is more than one way in which this 
  could be done. In section 11.3 we did simulations based on the “digital 
  waveguide” method: the same approach we had used earlier, when we simulated a 
  bowed string in Chapter 9. For this approach, the input impedance from an 
  idealised tube model was converted into a “reflection function”, as described 
  in detail in section 11.3.2. 

  We could do the same thing with a measured input impedance, but there are two 
  reasons why I will in fact choose to do something different this time. First, 
  the process of converting an impedance to a reflection function always 
  involves approximations and compromises, so that the result may be less 
  accurate than we would really like. This didn’t matter very much when we were 
  dealing with idealised models, because we weren’t trying to achieve an 
  accurate match to the behaviour of a particular tube. Instead, we were 
  looking for qualitative insights into how systems generically similar to a 
  clarinet or a saxophone might work. But the main point of using a measured 
  impedance function is that it should represent, accurately, the particular 
  instrument that was measured. 

  The second reason is more pragmatic. The methods generally used to measure 
  input impedance can be relied upon to give a reasonably accurate value of the 
  impedance magnitude, but with some kinds of measuring apparatus they are less 
  accurate when it comes to the phase of the complex frequency response 
  function. (This is not in fact a problem for the impedances we are about to 
  use, but it will become more of an issue in section 11.5, when we look at 
  brass instruments.) If sufficient information to run a simulation can be 
  extracted without relying on complete accuracy of the measured phase, this 
  would circumvent some potential problems. 

  There is indeed a way to achieve this. An acoustic input impedance, in common 
  with many other frequency response functions such as the bridge admittances 
  of stringed instruments discussed extensively in Chapter 5, can be expressed 
  in terms of modal properties: resonance frequencies, modal damping factors 
  and mode shapes. The underlying theory for the case of mechanical vibration 
  was outlined in section 2.2.5. The corresponding theory for the case of an 
  acoustic input impedance is given in the next link. 

  Now we can make use of something we have already seen. In section 10.5, we 
  met the idea of experimental modal analysis. A measured frequency response 
  function can be analysed to extract the modal parameters that make it up, 
  provided we know the mathematical formula for the relevant modal expansion. A 
  very simple approach has been adopted here, individually fitting each visible 
  peak in the measured impedance using a variant of circle fitting (see section 
  10.5). The details of the procedure are given in the next link. 

  The results are shown in Figs.\ 1 and 2, for input impedance measurements of 
  a clarinet and a soprano saxophone provided by Joe Wolfe. In both cases the 
  measurement is shown in red, and the reconstructed impedance based on the 
  modal expansion is shown in blue. The agreement is excellent at low 
  frequency. There is gradual divergence at higher frequencies, probably caused 
  mainly by the effect of missing modal contributions — associated either with 
  resonance frequencies outside the measurement range or with peaks too small 
  to circle-fit. There is no doubt that a closer fit could be achieved by a 
  more elaborate procedure, but given that we are mainly interested in the 
  lowest playable note with this fingering on each instrument, these fits are 
  probably entirely adequate. 

  Rather than using these fitted impedances to calculate a reflection function, 
  the information is used directly to perform simulations based on the impulse 
  response, the pressure response to a sharp pulse of volume flow through the 
  reed. Once the modal parameters are known, there is a simple formula for this 
  impulse response. Each mode gives a sinusoidal contribution at its resonance 
  frequency, decaying at a rate determined by its damping factor. Adding these 
  modal contributions together gives the total impulse response: the results 
  corresponding to Figs.\ 1 and 2 are shown in Figs.\ 3 and 4. 

  Impulse responses like these can be used directly in a simulation algorithm, 
  as described in the previous link. At each time step, the pressure inside the 
  mouthpiece can be computed by convolution of the impulse response with the 
  history of the volume flow rate through the reed. But, as we have noted 
  earlier (see section 9.5.2) we don’t need to compute this convolution 
  directly, which might be rather slow — instead, we can use a recursive 
  digital filter (“IIR filter”) for each mode in the expansion, and add the 
  results together. The result is a modal-based simulation which actually runs 
  faster than the ones we used in section 11.3, based on the reflection 
  function. 

  This modal approach works well in the context of wind instrument synthesis, 
  but the same approach would not have been so useful for the bowed string 
  studies we saw earlier. There are two main reasons. First, for these wind 
  instruments we only need to take a relatively small number of modes into 
  account: about 15 for the two fits shown in Figs.\ 1 and 2. For a typical 
  violin string, with a violin body coupled to it, we would need to include a 
  lot more modes, and the process of fitting the parameters of all these modes 
  would be more challenging. 

  The second reason concerns the modal damping: the damping of these tube modes 
  is far higher than the corresponding damping of a string. The effect is 
  immediately visible in Figs.\ 3 and 4: the impulse responses die away in a 
  small fraction of a second, whereas a string (following a pluck, for example) 
  rings on for far longer. The longer “memory” of a string makes the response 
  to bowing much more “twitchy”: the predictions of a simulation model are 
  sensitive to errors in the modal parameter values, and to other small details 
  in the model. 

  The modal parameters deduced from the fitting process are of some interest in 
  their own right. Figure 5, 6 and 7 show the frequencies, Q-factors and 
  amplitudes for the two instruments. Figure 5 is perhaps the most interesting. 
  For the idealised models we used in the previous section, a (complete) 
  conical tube had harmonically-spaced resonance frequencies filling a full 
  harmonic series: frequency ratios 1,2,3,4… The cylindrical tube of an 
  idealised clarinet also had harmonically-spaced resonances, but only 
  occupying the odd-numbered terms of a harmonic series with frequency ratios 
  1,3,5,7… The actual frequencies deduced from the input impedances used in the 
  idealised models from section 11.3 are included in Fig.\ 5: ``clarinet'' in a 
  dashed black curve, ``saxophone'' in a dotted black curve. 

  Figure 5 shows that the two real instruments show behaviour somewhat 
  intermediate between the idealised patterns shown by the two green lines. 
  Both sets of measurements lie between these two lines. The saxophone, which 
  is reasonably close to conical, (blue line) stays close to the lower green 
  line until about the 5th mode. But the clarinet, which differs more strongly 
  from the ideal cylindrical shape, (in red) is already departing significantly 
  from the upper green line by the third mode. The idealised ``saxophone'' 
  (black dotted curve) looks very similar to the actual measurements, deviating 
  slightly above the lower green line. However, the idealised ``clarinet'' has 
  perfectly harmonic frequencies, and looks quite different from the actual 
  measurement. 

  Are these departures from harmonic behaviour significant for the behaviour of 
  the real instruments? One preliminary way to assess that question is to take 
  into account the fact that the resonance peaks in the impedance have a finite 
  bandwidth, associated with their damping. The standard measure of this effect 
  is the half-power bandwidth (see section 2.2.7). These bandwidths are 
  indicated in Fig.\ 5 by the “error bar” symbols around each resonance 
  frequency. 

  If the green lines do not pass through the vertical range indicated by these 
  bandwidth markers, we can conclude that the inharmonicity is likely to be 
  significant. For the clarinet, this happens very quickly. If the clarinet 
  plays a note based on the fundamental resonance frequency of the tube, the 
  nonlinear action of the reed will generate exact harmonics of that frequency. 
  Because of the inharmonicity, these will not fall sufficiently close to the 
  higher resonances of the tube to gain strong resonant reinforcement. This 
  would be expected to impact on the frequency content of the played note: it 
  will be dominated by the fundamental, much more than was the case for the 
  idealised model with harmonic resonances, so the sound will be less bright. 

  Inharmonicity may also have an impact on the playing frequency via 
  “Benade-style” cooperative interactions between the resonances. Figure 5 
  shows that the resonances of the clarinet fall progressively flat, compared 
  to the ideal harmonic behaviour. Suppose the player changes what they are 
  doing in such a way that the amplitudes of harmonics generated by the reed 
  increase (perhaps as a result of blowing harder). We can predict that the 
  playing frequency is likely to respond by shifting downwards as the influence 
  of the higher tube modes becomes greater: the note will play progressively 
  flat. But Figure 5 shows that the saxophone exhibits the opposite tendency, 
  so we might expect our soprano saxophone to play sharp rather than flat under 
  similar circumstances. We will soon see simulation results, which will allow 
  these predictions to be tested. 

  Figure 6 shows the Q-factors deduced from the mode-fitting process. This 
  time, there is no big difference between the two instruments. Both curves are 
  rather featureless, and one might guess that the behaviour would not change 
  very much if we simply took a constant value for the modal Q-factor around 
  30, for all modes of both instruments. 

  The dashed black curve in this figure shows the Q-factors used in the 
  idealised clarinet model from section 11.3, while the dotted black curve 
  shows the corresponding idealised saxophone model. The tube damping in the 
  clarinet model was based on formulae for wall losses (ignoring energy loss 
  from sound radiation) originally derived by Rayleigh, and summarised in 
  section 8.2 of Fletcher and Rossing [1]. We can see from the plot that these 
  formulae give results that, apart from the first two resonances, look nothing 
  like the measurements on the real clarinet (red curve)! The results for the 
  idealised saxophone, on the other hand, are quite close to the values for the 
  saxophone measurements. This is no coincidence: the damping behaviour in the 
  idealised saxophone model was tweaked in an ad hoc manner to give an 
  impedance curve resembling the measurement. 

  Figure 7 shows a similar comparison of modal amplitude factors, for the two 
  measurements and also the two idealised models from section 11.3. The 
  idealised model of a cylindrical tube predicts that the amplitude should be 
  exactly the same for every mode. This is, of course, not exactly true for the 
  measured clarinet, but the measured results are not very far away from this 
  super-simple theoretical pattern. For the conical saxophone, the amplitude is 
  predicted to vary with mode number in a pattern that rises and then falls. 
  The measured amplitudes match the idealised prediction quite well for the 
  first three modes, but then they fall much more sharply than predicted. The 
  measurements rise again for modes 13, 14 and 15, but it is not clear how 
  reliable these fitted values are. Looking back at Fig.\ 2, these high modes, 
  above 3~kHz, were only visible as small wiggles in the impedance curve. They 
  are likely to be affected by non-negligible modal overlap in this frequency 
  range, and the simple circle-fitting strategy would then be too crude. 

  \samsection{B. Simulation results for the clarinet} 

  Finally, we are ready to see some results of simulation using the new model 
  based on measured impedance. We start with the clarinet. First, we need a 
  “sanity check” to make sure that the new simulation model gives results that 
  are recognisably related to the ones we saw in section 11.3 with the 
  idealised model. Figure 8 shows a typical “cold start” transient which 
  resulted in a note being played. The upper plot shows the mouthpiece 
  pressure, and after a few period-lengths this settles into a periodic 
  waveform looking quite like the square wave we have come to expect. The lower 
  plot shows the corresponding waveform of volume flow rate into the 
  mouthpiece. We see that the reed closes once per cycle, and we also see that 
  for a brief interval during each cycle air is flowing in the reverse 
  direction: back into the player’s mouth through the open reed. 

  But, as we saw before, it is hard to learn much from an individual 
  simulation. So we move directly on to pressure-gap diagrams, to give an 
  impression of the “playability” of this clarinet, fingered for its lowest 
  note. Figure 9 shows a first example. This shows the behaviour following cold 
  start transients, over the same ranges of reed gap and mouth pressure that we 
  saw in section 11.3. Recall that the vertical axis in this plot shows the 
  reed gap before the vibration starts but after the player has adjusted their 
  bite pressure. Higher bite pressure makes a smaller gap, so bite pressure 
  increases upwards in this plot, and all the similar ones we will see shortly. 
  At the very top of the plot, the bite has got so hard that the reed is 
  entirely closed. 

  The particular transient shown in Fig.\ 8 corresponds to the bottom 
  right-hand pixel of this diagram, with mouth pressure 4~kPa and reed gap 
  0.6~mm. The colour shading here is done on the same basis as Figs.\ 43 and 46 
  from section 11.3: it shows the playing frequency of each pixel deduced from 
  the autocorrelation of the final waveform, expressed as a multiple of the 
  nominal frequency. In this case, the colour is uniformly red, connoting the 
  value 1: every single case that produced a note turns out to play at a 
  frequency close to the nominal. (We will have more to say more about the 
  playing frequency shortly.) 

  The lines included on this figure have exactly the same meanings as in 
  corresponding plots in the previous section: they show various 
  theoretically-predicted thresholds. The magenta line is the lower limit for 
  possible excitation of a note. Any note must lie above this line: how close 
  it is possible to get depends on the amount of energy dissipation. As we saw 
  in Fig.\ 6, the new model has higher dissipation at every resonance than the 
  old idealised model, so the coloured pixels in Fig.\ 9 cannot get as close to 
  the magenta line as before. The green dashed line shows the “beating reed 
  threshold”, above which (according to the approximate Raman model) the reed 
  closes completely at some stage in the cycle. The cyan line shows the 
  “inverse oscillation threshold” above which it is possible for the reed to be 
  held permanently shut by the mouth pressure. The highest line, in red, is the 
  “extinction threshold” beyond which (according to the Raman model) no 
  oscillation at the fundamental frequency is possible. Since we are dealing 
  with cold starts, we expect the playable region to be more or less confined 
  between the magenta and cyan lines --- which indeed it is. 

  You may wonder how these thresholds based on the Raman model are calculated 
  for our new model. The answer is that, since we are expecting a “Helmholtz 
  motion” dominated by the fundamental, we can simply use the measured Q-factor 
  of the lowest mode and choose a Raman model that matches this value. As later 
  results will confirm, this gives an acceptable approximation to the various 
  thresholds. 

  Figure 9 gives an answer to one question, but this is not the only question 
  of interest to a clarinettist. We would also like to know the extent of the 
  largest region within which it is possible to sustain a note, once it has 
  been started. This question would be the direct analogue of the calculation 
  behind the Schelleng diagram for a bowed string — outside Schelleng’s 
  wedge-shaped region it is simply not possible to sustain Helmholtz motion in 
  a bowed string. But the calculation, and the diagram, does not tell you 
  anything about what kind of transient you need in order to start Helmholtz 
  motion in the first place — for that, we needed something like the analysis 
  behind the Guettler diagram. 

  To address this question for the clarinet model, we can change how we start 
  each simulation. We want to encourage oscillation approximately at the 
  fundamental frequency. The new mode-based simulation gives us a very simple 
  way to do this, better and more versatile than the approach we used in 
  section 11.3. Recall that each mode is represented separately by a recursive 
  digital filter. For a cold start, we initialise all these filters with the 
  value zero. But if we want to encourage oscillation near the fundamental 
  frequency, we can initialise that particular filter with a non-zero value. 
  This would represent a situation where there was pre-existing excitation of 
  the fundamental mode, while all the other modes were silent. 

  The result of doing a set of simulations with this kind of initialisation is 
  the pressure-gap diagram shown in Fig.\ 10. Compare this with Fig.\ 9. The 
  lower edge of the red wedge is in the same position as before, but the top 
  edge has moved up to lie close to the (barely visible) red line. Provided you 
  can get the note started, the clarinet is capable of playing at (or at least 
  near) the nominal frequency anywhere within this larger wedge-shaped region. 

  There are many other choices for how to colour-shade diagrams like this, to 
  bring out different aspects of the behaviour. Figure 11 shows the same set of 
  simulations as in Fig.\ 10, coloured to show the behaviour of the playing 
  frequency. The colour scale now shows the deviation, expressed in cents, of 
  the playing frequency relative to the nominal frequency of this note on the 
  clarinet. That note is written $E_3$, but because the $B\flat$ clarinet is a 
  transposing instrument it plays at $D_3$, 146.8~Hz. 

  The plot suggests that the note tends to play sharp, but there several things 
  to be said about that. First, the fitted frequency of the lowest tube mode 
  was 150.5~Hz, rather close to the highest playing frequency in this plot. But 
  that high frequency may be a little misleading because the measurement of 
  impedance was made with dry air at room temperature. Things may change a 
  little with the warm, moist air laden with $\mathrm{CO}_2$ that a player 
  blows into the instrument. There may also be an overall shift of frequency 
  because the measurement of impedance does not include the small end 
  correction arising from the flexibility of the reed. So we should be cautious 
  about comparing absolute frequencies like this. A more reliable thing that 
  Fig.\ 11 shows is that blowing harder and moving diagonally down the spine of 
  the wedge-shaped region results in the note getting some 30~cents flatter. 
  This is the effect we anticipated, in the discussion of Fig.\ 5. This pattern 
  of significant flattening is very much in line with the measurements by 
  Almeida et al. [2]. 

  Another choice of colour shading is illustrated by Fig.\ 12. This is an 
  attempt to show something about the frequency content of the pressure 
  waveform. The plot is directly comparable with Fig.\ 17 from section 11.3, 
  and later plots in that section. It shows the highest-numbered harmonic of 
  the waveform that has an amplitude no more than 10~dB below that of an ideal 
  square wave. Turn back to Fig.\ 17 of section 11.3, and look at the colours. 
  Comparing with the new plot demonstrates something else we predicted earlier. 
  Whereas the old plot has a lot of bright yellow and white pixels, the new one 
  never goes beyond red. The idealised model used before had perfectly harmonic 
  tube resonances, but the measured clarinet has significant inharmonicity. 
  This results in a big reduction in high-frequency content in the pressure 
  waveform. 

  To give a direct impression of the waveforms that lie behind these various 
  plots, Fig.\ 13 shows the full set of pressure waveforms along the three 
  lines marked in green in Fig.\ 10. To help interpret this plot, notice that 
  the lowest waveform in the right-hand column is the same as the one shown in 
  Fig.\ 8. It is a recognisable version of the square wave, and all three 
  columns of Fig.\ 13 show waveforms like this in the middle of their range. 
  But near the edges of the wedge-shaped region, for example the waveforms at 
  the top and bottom of the left-hand column, the shape looks more like a sine 
  wave. 

  I will show one final group of plots relating to this clarinet model. If we 
  make a version of Fig.\ 9 over an extended range of blowing pressure and reed 
  gap, we get the result shown in Fig.\ 14. Notice that a few white pixels have 
  appeared, indicating a frequency approximately three times the nominal. In 
  this set of notes generated by cold-start transients, a few cases have 
  spontaneously chosen a playing frequency based around the second resonance of 
  the tube, rather than the fundamental resonance. In musician's terms, our 
  simulated clarinet is over-blowing to the second register (without the use of 
  the register key). 

  Figure 15 shows a version of Fig.\ 10 over this same extended range. When the 
  simulation is “primed” with some oscillation at the fundamental frequency, it 
  is perfectly capable of sustaining a note in the first register over the 
  whole wedge-shaped region. Now compare this with Fig.\ 16, in which the 
  simulations were primed with the second tube resonance rather than the first. 
  Now we see a large number of white pixels: there is a big region of the 
  diagram over which it is possible to sustain a second-register note, once 
  started. This white region lies entirely within the red region of Fig.\ 15: 
  in that range the clarinet is capable to sustaining two different regimes of 
  oscillation (and maybe others too). 

  These three plots tell us something important about playing the clarinet. 
  None of the initial transients used in these simulations will be an accurate 
  representation of what a human player really does. They do not “sing” into 
  the tube to prime the first or second mode, nor are they capable of switching 
  on their blowing pressure in the instantaneous way that the computer 
  simulations do it. Instead, skilled players learn to manipulate subtle 
  details of their transient to shape the sound, and to land on the regime they 
  are seeking with a musically-acceptable transient sound. The study of such 
  articulatory gestures in wind instruments is in its infancy. Up to a point 
  these issues can be explored by experiment and by the kind of simulation 
  studies started here, but there remains a research challenge. No-one has yet 
  carried out an analysis of wind instrument transients comparable to Knut 
  Guettler’s bowed-string study that gave rise to the Guettler diagram. 

  \samsection{C. Simulation results for the saxophone} 

  We can now look at some similar plots for the soprano saxophone, based on the 
  input impedance shown in Fig.\ 2, and the associated modal parameters 
  described earlier. As an introduction, Fig.\ 17 shows the pressure waveform 
  of a successful cold-start transient, which develops into a periodic 
  pulse-like waveform. This is the expected form of the “Helmholtz motion” for 
  this instrument. 

  Figure 18 shows a pressure-gap diagram for cold-start transients. Unlike the 
  corresponding results for the clarinet in Fig.\ 9, the saxophone shows 
  regions where a note in the first, second or third registers is produced — at 
  least according to this correlation analysis. We will see shortly that the 
  reality is a bit more complicated. The behaviour shown by Fig.\ 18 is similar 
  to what we saw earlier with the idealised saxophone model, but the real 
  instrument looks slightly better-behaved: the regimes appear in well-defined 
  and contiguous regions, giving the player a fighting chance of achieving the 
  one they are aiming for. But the comment we made about clarinet playing 
  applies here too: real transients will not be as abrupt as the computer’s 
  version of a cold start. A skilled player will learn to fine-tune their 
  initial articulation so that the register they are aiming for is achieved 
  reliably. But the plot makes some contact with the experience of players: Joe 
  Wolfe writes ``the very lowest notes on the saxophone are quite difficult to 
  start softly --- they tend to start at the octave. Loud, ugly starts on the 
  bottom notes are a problem on all saxes. Similar problems beset the oboe and 
  bassoon. In contrast, the clarinet will start readily on ppp on the lowest 
  notes.'' 

  Figure 19 shows three sets of waveforms, in the same format as Fig.\ 13. They 
  correspond to the three rows of Fig.\ 18 marked by horizontal lines. It is 
  worth looking rather closely at these waveforms, because the pattern of 
  behaviour is rather complicated. The right-hand column gives the most detail. 
  The top four waveforms correspond to red pixels in Fig.\ 18, and they show 
  the pulse waveform characteristic of Helmholtz motion. Indeed, the second of 
  these is the case we have already seen in Fig.\ 17. 

  Next, we come to four yellow pixels in Fig.\ 18, and Fig.\ 19 reveals that 
  these all show two pulses per period, “double-slipping motion” in 
  bowed-string terms. Of the next 6 pixels in Fig.\ 18, 5 are white. Fig.\ 19 
  reveals that these all show rather similar waveforms, with three pulses per 
  cycle — very much what we would expect for the third register. The odd one 
  out is an isolated red pixel, for which Fig.\ 19 shows a kind of double-pulse 
  waveform, but the two pulses are not equally spaced so that the periodicity 
  of this waveform does indeed correspond to the fundamental frequency, not the 
  the second register. 

  Next, we come to three yellow pixels in Fig.\ 18, and Fig.\ 19 shows that the 
  corresponding waveforms look very similar to the earlier “double slipping” 
  waveforms. Finally, there are three red pixels. But these do not show what we 
  might have been expecting from the discussion of the idealised saxophone in 
  section 11.3. They do not show the “inverted Helmholtz” motion, with a single 
  upward pulse in each cycle. Instead, the first two of them show a kind of 
  asymmetric double pulse waveform rather like the isolated red pixel we passed 
  earlier. The final waveform is different: it shows, somewhat unexpectedly, a 
  kind of square wave, rather like what we saw from the clarinet in Fig.\ 13. 

  It is useful to look at the frequency spectra of a selection of these 
  waveforms. Figure 20 shows these, for the 5 cases marked by circles in Fig.\ 
  18. They have been spread out in the plot for clarity. The top one, in blue, 
  shows strong peaks at all harmonics of the fundamental frequency. The second, 
  in orange, is a bit more surprising. We have described this as a note in the 
  second register, implying a frequency twice that of the fundamental tube 
  resonance. But the frequency spectrum reveals that although the even-numbered 
  peaks are much stronger than the odd-numbered ones, nevertheless the 
  odd-numbered peaks are not entirely missing. So this note is dominated by 
  frequency components in a harmonic series an octave higher than the first 
  register, but it still contains traces of that fundamental frequency. This 
  pattern is repeated in the third spectrum in Fig.\ 20, in yellow. This note 
  has been described as being in the third register. Well, the third peak is 
  certainly the strongest, but all the harmonics of the fundamental are present 
  at some level. The remaining spectra match what has already been described: 
  the fourth is rather like the second, and the last one (in green) has strong 
  peaks at all harmonics, like the first one. 

  So, you should be wondering, what do these all sound like? Sound 1 gives you 
  a chance to hear all the simulated notes along the bottom row of Fig.\ 18. 
  Each simulated note is 0.4~s long, and these have simply been strung together 
  to make this sound file. You should hear quite clearly the “bugling” effect 
  of going up through the first three registers, and then down again. But if 
  you pay attention to individual notes, you can hear changes in tone quality 
  which are associated with the waveform and spectrum details we have just 
  discussed. 

  Figure 21 shows a version of the same pressure-gap diagram as Fig.\ 18, 
  colour-shaded to show the frequency deviation in cents from the nominal 
  frequency of the relevant register (i.e. twice the nominal frequency for the 
  yellow pixels in Fig.\ 18, and three times the fundamental for the white 
  pixels). We can see that the third-register notes tend to play a little flat, 
  compared to the ideal harmonic series. Looking back at Fig.\ 20, the spectra 
  confirm this: if you look closely at the peak positions, the yellow spectrum 
  has all its peaks at slightly lower frequencies than the other four. 

  We will say a bit more about frequency deviation in a moment, in the context 
  of our next set of plots. Just as we did with the clarinet, we now want to 
  ask where in this pressure-gap plane it is possible to sustain a note in the 
  first register, once it has been started. We can investigate this in exactly 
  the same way as we did before, by computing a pressure-gap diagram with the 
  filter representing the fundamental tube mode primed with a non-zero value. 
  The result is shown in Fig.\ 22, and just as in the clarinet case it shows a 
  large area, uniformly coloured red connoting the first register. One thing 
  can be noticed immediately: the red pixels in this plot extend a little below 
  the magenta threshold line, something we have not seen before. But in Fig.\ 
  18, with cold-start transients, the pixels were all above the line. Figure 22 
  is suggesting that once you get this lowest note started on the saxophone, 
  you may be able to sustain it with the pressure reduced below the line. 

  Figure 23 shows a version of the same pressure-gap diagram, shaded to show 
  frequency deviation. Figure 24 shows another version, shaded to show 
  something about the frequency content of each waveform — but this time that 
  has been done in a different way, by plotting the spectral centroid of each 
  final periodic waveform. (If you printed the spectrum on cardboard and cut it 
  out, the spectral centroid would be the centre of gravity of the cardboard 
  shape.) 

  The two plots reveal some interesting features. First, Fig.\ 24 shows a 
  pattern of variation that looks superficially similar to the corresponding 
  plot for the clarinet, seen in Fig.\ 12. But think carefully — it actually 
  shows the exact opposite behaviour! The “core” of the wedge-shaped region in 
  Fig.\ 24 shows darker colours, so the spectral centroid is lower there. For 
  the clarinet, the sound was predicted to be brighter in the core of the 
  wedge, but for the saxophone it is predicted to be less bright. 

  To see what is going on to cause the wedge of orange colour in the middle of 
  Fig.\ 24, we need to look at waveforms in detail. Figure 25 shows a plot in 
  the same format as before. If you scan down the middle column of this plot, 
  you see that at first the waveforms show the pulse-like Helmholtz motion that 
  we have seen before. But the waveform shapes gradually morphs into something 
  like a square wave, and then as you continue down the column it gradually 
  turns into the inverted pulse waveform that we saw in section 11.3. This 
  morphing of the waveform while maintaining the same playing pitch is probably 
  telling us something illuminating about about the tonal flexibility of the 
  saxophone. The orange colours in Fig.\ 24 correspond, more or less, to the 
  square wave cases. A symmetrical square wave has weakened even-number 
  harmonics, and that is probably responsible for reducing the spectral 
  centroid. 

  We haven’t finished with Figs.\ 23 and 24 yet: there is another interesting 
  thing to be noticed. Earlier in this section when we were discussing the 
  resonance frequencies of the clarinet and saxophone, we suggested that the 
  two instruments might show opposite trends of pitch deviation, when the 
  blowing conditions are changed. We have seen one of our guesses confirmed: 
  the clarinet tends to go flat when blown in a more vigorous way, causing more 
  harmonic generation by the nonlinear reed behaviour. So what about the 
  saxophone? The instrument behaves in a more complicated way than the 
  clarinet, as we have just seen, but we can see something relevant to our 
  guess if we look at the behaviour near the lower edge of the wedge-shaped 
  region. Figure 23 shows reddish colours near this lower edge, getting 
  progressively yellower as you move to the right. This means that the note 
  plays sharper as the blowing pressure is increased, in this zone fairly close 
  to the threshold. This is indeed where an increase in nonlinear effects could 
  be anticipated, as the beating reed threshold is approached and passed. 

  As blowing pressure is increased further, the waveform morphs into the square 
  wave as we just saw. At that stage, the frequency deviation shifts back 
  towards more orange colours. Blowing harder still produces the complicated 
  series of waveform changes we just described. These changes interact with the 
  pitch deviation in a complicated manner, leading to a pattern of stripes in 
  Fig.\ 23. 

  \samsection{D. Multiphonics} 

  So far, we have given an impression that when a reed instrument plays a note, 
  the pressure signal will settle down to a periodic waveform of some kind. But 
  this is by no means always the case. Sometimes, the result is something 
  called a “quasi-periodic waveform”. These may be produced by mistake when a 
  player is trying to produce a regular note, but sometimes they are used 
  deliberately for musical effect — they are then usually called 
  “multiphonics”. We will look at one example of a multiphonic fingering for a 
  clarinet, taken yet again from the data set on Joe Wolfe’s web site. He 
  labels this particular example “D4B5”. 

  The input impedance for this fingering is shown in Fig.\ 26. By using an 
  unconventional fingering, the player has created a rather irregular 
  distribution of resonance peaks. It is hard to guess what will happen now 
  when they blow into the mouthpiece. Figure 27 shows a pressure-gap diagram 
  from a set of simulations, coloured to show the spectral centroid. The 
  portion of the plane with an orange colour corresponds to more-or-less steady 
  notes, but in the strip along the lower edge of the wedge-shaped region, 
  something else happens. 

  We will look at three examples, drawn from the bottom row of this diagram and 
  marked by three circles. The first half-second of the simulated pressure 
  waveforms of these three cases are plotted in Fig.\ 28. The top plot (in 
  blue) corresponds to the orange pixel in the right-hand corner of Fig.\ 27. 
  After quite a short transient, this settles into a periodic waveform. But the 
  other two cases lead to non-periodic signals. The bottom plot, in yellow, has 
  a very long transient before it settles into the quasi-periodic state. 

  Figure 29 shows the frequency spectra of the same three waveforms, with the 
  same layout and colour coding. The blue curve shows a series of harmonic 
  peaks. But the other two show a complicated mixture of frequency components. 
  In Sound 2, you can listen to the three waveforms in succession. The first is 
  a fairly normal-sounding note, but the other two are not. You can perhaps 
  hear more than one pitch in these sounds: this is, of course, the origin of 
  the term “multiphonic”. 

