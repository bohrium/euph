  Figure 1 shows a sketch of a double pendulum: two identical rods of length 
  $2L$ and mass $m$ are freely hinged together at one end, and the linked pair 
  is freely pivoted to a fixed support at one end. We can describe the position 
  of the two pendulums at any instant with the two angles $\theta$ and $\phi$ 
  as shown. 

  \fig{figs/fig-b31febf6.png}{\caption{Figure 1. Sketch of the double 
  pendulum}} 

  We can derive the governing equations for free motion by using Lagrange's 
  equations. First, we need expressions for the potential and kinetic energies. 
  In terms of Cartesian x-y axes with an origin at the fixed pivot, the centre 
  of mass of the top pendulum is at the point 

  \begin{equation*}[L \sin \theta, -L \cos \theta] \tag{1}\end{equation*} 

  \noindent{}while that of the lower pendulum is at the point 

  \begin{equation*}[2L \sin \theta + L \sin \phi, -2L \cos \theta -- L \cos 
  \phi]. \tag{2}\end{equation*} 

  We can immediately write down an expression for the potential energy, which 
  is entirely due to gravity: 

  \begin{equation*}V=mgL(1 -- \cos \theta) +mgL[2(1- \cos \theta) + (1-\cos 
  \phi)] \tag{3}\end{equation*} 

  \noindent{}where the reference configuration with $V=0$ has been taken to be 
  the equilibrium position $\theta=\phi=0$. 

  For the kinetic energy, we first need the velocities of the two centres of 
  mass. For the top pendulum this has components 

  \begin{equation*}L \dot{\theta} [\cos \theta, \sin \theta] 
  \tag{4}\end{equation*} 

  \noindent{}while for the lower pendulum the components are 

  \begin{equation*}2L \dot{\theta} [\cos \theta, \sin \theta] + L \dot{\phi} 
  [\cos \phi, \sin \phi]. \tag{5}\end{equation*} 

  We can now write down an expression for the kinetic energy: for each pendulum 
  separately, we can use the general rigid-body expression $(1/2)m v_G^2 + 
  (1/2) I_G \Omega^2$ where $m$ is the mass, $I_G$ is the moment of inertia 
  about the centre of mass, $v_G$ is the magnitude of the velocity of the 
  centre of mass, and $\Omega$ is the angular velocity. 

  The result is 

  \begin{equation*}T=\dfrac{1}{2}mL^2 \dot{\theta}^2 (\cos^2 \theta + \sin^2 
  \theta) +\dfrac{1}{2}\left(\dfrac{1}{3}mL^2\right) 
  \dot{\theta}^2+\dfrac{1}{2}\left(\dfrac{1}{3}mL^2\right) 
  \dot{\phi}^2\end{equation*} 

  \begin{equation*}+ \dfrac{1}{2}mL^2 \left[ \left(2 \dot{\theta}\cos \theta + 
  \dot{\phi}\cos \phi\right)^2 + \left(2 \dot{\theta}\sin \theta + 
  \dot{\phi}\sin \phi\right)^2 \right] \tag{6}\end{equation*} 

  \noindent{}which, after multiplying out and simplifying, comes down to 

  \begin{equation*}T=\dfrac{1}{2}mL^2 \left[ \dfrac{16}{3} \dot{\theta}^2 + 
  \dfrac{4}{3} \dot{\phi}^2 + 4 \dot{\theta} \dot{\phi} \cos (\phi-\theta) 
  \right] . \tag{7}\end{equation*} 

  Now we use the Lagrange equations. Applied to the angle $\theta$, the 
  equation is 

  \begin{equation*}\dfrac{d}{dt}\left[ \dfrac{\partial T}{\partial 
  \dot{\theta}} \right] -- \dfrac{\partial T}{\partial \theta} + 
  \dfrac{\partial V}{\partial \theta} = 0 \tag{8}\end{equation*} 

  \noindent{}leading to 

  \begin{equation*}\dfrac{d}{dt}\left[\dfrac{16}{3}mL^2 \dot{\theta} + 2mL^2 
  \dot{\phi} \cos (\phi -- \theta) \right] \end{equation*} 

  \begin{equation*}-2mL^2 \dot{\theta} \dot{\phi} \sin (\phi -- \theta) + 3mgL 
  \sin \theta = 0 \tag{9}\end{equation*} 

  \noindent{}which expands and simplifies to the final result 

  \begin{equation*}\dfrac{16}{3}\ddot{\theta}+ 2\ddot{\phi}\cos (\phi -- 
  \theta) -- 2 \dot{\phi}^2 \sin (\phi -- \theta) + \dfrac{3g}{L} \sin \theta = 
  0. \tag{10}\end{equation*} 

  The corresponding Lagrange equation for the angle $\phi$ gives 

  \begin{equation*}\dfrac{d}{dt}\left[\dfrac{4}{3}mL^2 \dot{\phi} + 2mL^2 
  \dot{\theta} \cos (\phi -- \theta) \right] \end{equation*} 

  \begin{equation*}+2mL^2 \dot{\theta} \dot{\phi} \sin (\phi -- \theta) + mgL 
  \sin \phi = 0 \tag{11}\end{equation*} 

  \noindent{}which expands and simplifies to give 

  \begin{equation*}\dfrac{4}{3}\ddot{\phi}+ 2\ddot{\theta}\cos (\phi -- \theta) 
  + 2 \dot{\theta}^2 \sin (\phi -- \theta) + \dfrac{g}{L} \sin \phi = 0. 
  \tag{12}\end{equation*} 

  Equations (10) and (12) determine the free motion of the pendulum system. 
  Because we have two second-order differential equations, we need four initial 
  values: the most natural things to specify are the values at time $t=0$ of 
  $\theta$, $\dot{\theta}$, $\phi$ and $\dot{\phi}$. 