  One aspect of vibration behaviour which was made explicit by the transfer 
  function formula (11) of section 2.2.5 is that natural frequencies correspond 
  to resonances, in the sense that a small applied force at or near such a 
  frequency causes a large response. Unfortunately, the formula predicts 
  infinite response when the forcing frequency matches a natural frequency 
  exactly. This obviously unphysical prediction is the result of the fact that 
  we have ignored energy dissipation. If some energy dissipation, or damping, 
  is included in the analysis, the prediction of infinite response is reduced 
  to one of a large but finite response level. 

  This problem of infinite response at resonance was already seen in section 
  2.2.2, for the simple mass-spring oscillator without damping. That model can 
  easily be extended as sketched in Fig.\ 1, to include a mechanism for energy 
  dissipation: the usual way to do this is to add a third component to the 
  system, called a dashpot. The symbol in the diagram is meant to indicate a 
  loose-fitting piston in a cylinder, like the shock absorbers of a car or the 
  devices sometimes seen on the backs of doors so that they close without 
  slamming. When the piston is moved in the cylinder, fluid flow around it 
  causes energy loss via viscosity. In the idealised version suitable for our 
  linear system, the resisting force associated with this fluid flow is assumed 
  to be proportional to the relative velocity of piston and cylinder, with a 
  constant of proportionality called $c$ here. 

  \fig{figs/fig-1cb82063.png}{\caption{Figure 1: Idealised mass-spring-dashpot 
  oscillator.}} 

  The equation of motion of this modified system is 

  \begin{equation*}m \ddot{x}+c\dot{x} +kx=f(t) \tag{1} \end{equation*} 

  \noindent{}when driven by an external force $f(t)$. But we first examine free 
  motion, with $f=0$. To find the solution, we can try substituting $x(t)=e^{i 
  \alpha t}$ in the hope of solving for possible values of $\alpha$. Equation 
  (1) then requires 

  \begin{equation*}[-\alpha^2 m +i \alpha c + k] e^{i \alpha t} =0. \tag{2} 
  \end{equation*} 

  The exponential factor can be cancelled, leaving a quadratic equation for 
  $\alpha$ which can be solved with the usual formula: 

  \begin{equation*}\alpha = \dfrac{ic \pm \sqrt{-c^2 + 4 m k}}{2m}. 
  \tag{3}\end{equation*} 

  Provided the damping is light, in the sense that $c^2 \ll 4mk$, this reduces 
  to 

  \begin{equation*}\alpha \approx \pm \sqrt{\frac{k}{m}} + \frac{ic}{2m}. 
  \tag{4} \end{equation*} 

  Now note that $\sqrt{\frac{k}{m}}=\Omega$, the natural frequency of the 
  undamped oscillator found in section 2.2.2. It is convenient to define a 
  dimensionless quantity $\zeta$ so that 

  \begin{equation*}\frac{c}{2m} = \zeta \Omega \tag{5} \end{equation*} 

  \noindent{}which requires 

  \begin{equation*}\zeta=\frac{c}{2\sqrt{km}}. \tag{6} \end{equation*} 

  Then 

  \begin{equation*}e^{i \alpha t} \approx e^{\pm i \Omega t} e^{-\zeta \Omega 
  t}. \tag{7} \end{equation*} 

  This gives us our general solution for free motion of the damped oscillator: 
  in a similar form to eq. (5) of section 2.2.2 it is 

  \begin{equation*}x(t) \approx R \cos(\Omega t + \phi) e^{-\zeta \Omega t} 
  \tag{8} \end{equation*} 

  \noindent{}so that the motion consists of oscillation at frequency $\Omega$ 
  with initial amplitude $R$ and phase offset $\phi$, but the effect of the 
  damping is to make the oscillations die away exponentially with time. 

  The rate of decay is governed by the value of $\zeta$, called the damping 
  ratio. But this is not the only quantity used to characterise damping: two 
  related quantities are worth mentioning straight away. The loss factor, often 
  denoted $\eta$, is for this simple problem related by $\eta=2\zeta$. The 
  quality factor or Q-factor is another commonly-used measure of damping, 
  defined by $Q=1/\eta=1/2\zeta$. Low damping is associated with high Q-factor: 
  the name comes from the field of electrical engineering, where for example a 
  ``high-quality'' quartz crystal oscillator is one with very low damping so 
  that it gives an accurate frequency reference for your computer's internal 
  clock. 

  To understand the link between damping and accuracy of a frequency reference, 
  we need to look at the response of the damped oscillator to sinusoidal 
  excitation. The calculation is very similar to the one in section 2.2.2. If 
  $f(t)=F e^{i \omega t}$, the response must also be sinusoidal at the same 
  frequency: $x(t) = Xe^{i \omega t}$. Substituting in eq. (1) and cancelling 
  the exponential factor gives 

  \begin{equation*}[-\omega^2 m + i \omega c +k]X=F \tag{9}\end{equation*} 

  \noindent{}so that the frequency response function is 

  \begin{equation*}G(\omega) = \frac{X}{F} = \frac{1}{k+i \omega c -- \omega^2 
  m} \tag{10} \end{equation*} 

  \noindent{}which can be re-written in the alternative forms 

  \begin{equation*}G(\omega) = \frac{1/m}{\Omega^2+2i \omega \zeta \Omega -- 
  \omega^2 }= \frac{1/k}{1+2i \zeta [\omega/\Omega] -- [\omega/\Omega]^2 }. 
  \tag{11} \end{equation*} 

  The response function $G(\omega)$ is now complex, and we recall that the 
  physical answer is given by the real part of this complex expression. 

  \fig{figs/fig-b909b6a8.png}{} 

  \fig{figs/fig-371d6419.png}{} 

  Both amplitude and phase shift now vary with frequency: some examples are 
  plotted in Fig.\ 2, for different values of $\zeta$. The amplitude is plotted 
  on a dB scale, in normalised form such that the value at zero frequency (DC) 
  is unity, or 0 dB. The frequency scale shows the ratio $\omega/\Omega$. The 
  phase lag $\phi$ is plotted in degrees. The blue curves show the undamped 
  result: the amplitude goes off to infinity on this dB scale, and the phase 
  switches abruptly at the resonant frequency from zero (in phase) to 
  $-180^\circ$ (opposite phase). The other curves all show non-zero values of 
  $\zeta$. The amplitude at resonance becomes finite, and the peak value 
  reduces as $\zeta$ rises. The phase curves have the abrupt jump smoothed out, 
  over an increasing wide frequency range as $\zeta$ increases. 

  A simple calculation reveals a useful formula for the bandwidth associated 
  with damping. At the frequencies $\Omega(1 \pm \zeta)$, the amplitude has 
  fallen from the peak value by a factor $1/\sqrt{2}$, while the phase is 
  $90^\circ \pm 45^\circ$. These two frequencies are called the half-power 
  points: the energy in the oscillator is proportional to the square of the 
  amplitude, so it has reduced to 50\% of its peak value at these two 
  frequencies. The difference of these two frequencies is the half-power 
  bandwidth. The half-power bandwidth tends to zero as damping tends to zero, 
  or as Q-factor tends to infinity. This is the reason that a high-Q oscillator 
  can define a very precise resonant frequency. 

  To give an intuitive feel for the meaning of the damping ratio, some familiar 
  systems illustrating different orders of magnitude are as follows: 

  The last one gives an impression of being quite highly damped, but $\zeta_n 
  \approx 0.1$ is still a sufficiently small number for the purposes of the 
  ``small damping'' assumption that underlies the simple approach to damping 
  used here. ``Small damping'' is the rule, not the exception. 

  Turning to the more general problem, a simple strategy to incorporate some 
  damping into the response of systems with many degrees of freedom is to add 
  some damping to each modal oscillator separately, drawing inspiration from 
  the damped harmonic oscillator just discussed. Thus, we modify eq. (11) of 
  section 2.2.5 to 

  \begin{equation*}G(j,k,\omega) \approx \dfrac{q_k}{F}=\sum_n 
  \dfrac{u_j^{(n)}u_k^{(n)}}{\omega_n^2+2i\omega \omega_n \zeta_n-\omega^2}. 
  \tag{12}\end{equation*} 

  \noindent{}where the $\zeta_n$ are dimensionless numbers called modal damping 
  ratios. 