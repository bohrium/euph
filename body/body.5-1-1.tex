  To get the most accurate estimate of a bridge admittance, or any other 
  frequency response function of a linear system, it is advisable to make 
  multiple measurements and use some form of averaging. We are dealing with a 
  system like the sketch below, reproduced from section 2.1. (Of course, the 
  system we are measuring is not necessarily a drum!) 

  \fig{figs/fig-bd8fc8a5.png}{Figure 1: Schematic block diagram of an 
  input-output ``black box''.} 

  Suppose we have an input signal $x(t)$ and an output signal $y(t)$. We can 
  call their Fourier transforms $X(\omega)$ and $Y(\omega)$. If we could take 
  measurements for infinite time, with no contamination by noise, and if our 
  system really was perfectly linear, then the frequency response we want is 
  simply the ratio $Y(\omega)/X(\omega)$. But none of those things will be true 
  in a real measurement. We catch a finite chunk of the input and output 
  signals, and use the FFT to convert them to the frequency domain. The ratio 
  of these will indeed give us an estimate of the frequency response function, 
  but for mathematical reasons that we will not go into here, simply repeating 
  that process several times and taking the average of these individual 
  estimates of the complex response is not the best thing to do. 

  A better approach makes use of concepts from the field of random vibration, 
  concerned with questions like describing the vibration response of an 
  offshore oil platform exposed to ocean waves, or the noise inside a car as a 
  result of tyre noise from the rough surface of the road. We can define 
  something called the power spectral density of the input 

  $$S_{xx}(\omega) = \frac{1}{T}\left<|X(\omega)|^2 \right> = 
  \frac{1}{T}\left<X(\omega) X^*(\omega)\right> \tag{1}$$ 

  where $T$ is the length of the chunk of signal that has been processed, $X^*$ 
  denotes the complex conjugate of $X$, and $\left< ... \right>$ means `average 
  value over the multiple measurements'. Similarly, we define 

  $$S_{yy}(\omega) = \frac{1}{T}\left<Y(\omega) Y^*(\omega)\right> .\tag{2}$$ 

  Finally, a related quantity is the cross spectral density of the output and 
  input: 

  $$S_{xy}(\omega) = \frac{1}{T}\left<X(\omega) Y^*(\omega)\right> .\tag{3}$$ 

  Given these three averaged quantities, there are two different ways that we 
  might estimate the thing we really want, $Y/X$: either as $S_{yy}/S_{xy}$, or 
  as $S^*_{xy}/S_{xx}$. If everything is working perfectly, these two estimates 
  should be equal. A useful measure of whether this is happening is the 
  coherence function $C_{xy}$ defined as the ratio of the two: 

  $$C_{xy}(\omega)=\dfrac{S_{xy}(\omega)S^*_{xy}(\omega)}{S_{xx}(\omega) 
  S_{yy}(\omega)} .\tag{4}$$ 

  Since the two power spectral densities are by definition real and positive, 
  and the expression in the numerator is $|S_{xy}(\omega)|^2$, it is obvious 
  that this is a positive, real quantity, not a complex one. It always lies 
  between 0 and 1. 

  An example is shown in Fig.\ 2: this is a rather good-quality measurement, 
  showing coherence close to 1 up to about 7 kHz. In that range, where the 
  coherence is seen to drop a little it can be seen to correspond to 
  frequencies where the admittance was low, so that the signal-to-noise ratio 
  of the measurement was relatively poor. However, this plot is deliberately 
  shown over an extended frequency range, up to 20 kHz. The impulse hammer 
  hitting the wood of a violin bridge does not supply reliable input at these 
  higher frequencies, and the coherence falls off dramatically. The coherence 
  function thus gives an immediate impression of the effective bandwidth of the 
  measurement. 

  \fig{figs/fig-8f85ecd4.png}{Figure 2. Measured bridge admittance of a violin 
  by Andrea Postacchini (upper plot), and the associated coherence function 
  (lower plot)} 