  This chapter applies material from Chapter 2 to the question of percussion 
  instruments, and to the particular question of how a tuned percussion 
  instrument differs from a non-tuned one. It all hinges on the pattern of 
  resonant frequencies associated with the vibration modes. 

  Section 3.2 gives some background on the vibration behaviour of different 
  types of structure, then a variety of tuned percussion instruments are 
  investigated: xylophones, church bells, Caribbean steel pans and tuned drums 
  of various kinds. In each case, sound examples are given so that you can hear 
  directly the effect of tuning, or not tuning, the relevant resonance 
  frequencies. 

