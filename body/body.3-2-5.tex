  Back in section 2.2.5, we looked at the general theory of linear vibration of 
  discrete systems. In terms of a set of ``generalised coordinates'' 
  $q_1,q_2,\dots,q_N$ we argued that the potential energy has the form of a 
  quadratic expression 

  $$V=\dfrac{1}{2}\sum_j \sum_k K_{jk} q_j q_k \tag{1}$$ 

  involving a matrix $K$ whose terms are $K_{jk}$, called the stiffness matrix. 
  Similarly, the kinetic energy was expressed as a quadratic expression in the 
  generalised velocities $\dot{q}_1,\dot{q}_2,\dots,\dot{q}_N$ 

  $$T=\dfrac{1}{2}\sum_j \sum_k M_{jk} \dot{q}_j \dot{q}_k\tag{2}$$ 

  involving a matrix $M$ whose terms are $M_{jk}$, called the mass matrix. 
  These expressions can be written compactly in terms of the vector 
  $\mathbf{q}$ whose components are $q_i$: 

  $$V=\frac{1}{2} \mathbf{q}^T K \mathbf{q},~~~T=\frac{1}{2} \mathbf{\dot{q}}^T 
  M \mathbf{\dot{q}}\tag{3}$$ 

  where $\mathbf{q}^T$ denotes the transpose of the vector $\mathbf{q}$. 

  We will now give an outline proof of an important result about the ``Rayleigh 
  quotient'' 

  $$R=\frac{\mathbf{q}^T K \mathbf{q}}{\mathbf{q}^T M \mathbf{q}} .\tag{4}$$ 

  Notice that the denominator is not quite the kinetic energy: the expression 
  involves $\mathbf{q}$, not $\mathbf{\dot{q}}$. 

  We want to express $\mathbf{q}$ in terms of the mode shapes 
  $\mathbf{u}^{(n)}$, which were the solutions of eq. (7) of section 2.2.5. So 
  write 

  $$\mathbf{q}=\sum_n{\alpha_n \mathbf{u}^{(n)}} .\tag{5}$$ 

  Substitute this expression into eq. (4), and take advantage of the fact 
  (described in section 2.2.5) that any two modes $\mathbf{u}^{(n)}$ and 
  $\mathbf{u}^{(m)}$ are orthogonal with respect to both matrices $K$ and $M$. 
  The result simplifies down to 

  $$R=\frac{\alpha_1^2 \omega_1^2 + \alpha_2^2 \omega_2^2 + \dots + \alpha_N^2 
  \omega_N^2 }{\alpha_1^2 + \alpha_2^2 + \dots + \alpha_N^2 } .\tag{6}$$ 

  Now suppose that the vector $\mathbf{q}$ is an approximation to one of the 
  modes, say $\mathbf{u}^{(k)}$. That means that we could take $\alpha_k =1$, 
  and all the other $\alpha_n$ will be small: 

  $$\alpha_n \ll 1,~~~n \ne k .\tag{7}$$ 

  Then 

  $$R=\frac{\alpha_1^2 \omega_1^2 + \alpha_2^2 \omega_2^2 + \dots+ \omega_k^2 + 
  \dots + \alpha_N^2 \omega_N^2 }{\alpha_1^2 + \alpha_2^2 + \dots + 1 \dots + 
  \alpha_N^2 } .\tag{8}$$ 

  It is obvious that $R \approx \omega_k^2$, but the result we want is stronger 
  than that. The vector $\mathbf{q}$ differs from the exact mode shape by terms 
  of order $\alpha_1,~\alpha_2, \dots$ but $R$ only differs from $\omega_k^2$ 
  by much smaller terms, of the order of $\alpha_1^2,~\alpha_2^2, \dots$. 
  Roughly, we could say that if $\mathbf{q}$ has about 10\% errors, the 
  Rayleigh quotient $R$ will approximate $\omega_k^2$ with errors only around 
  1\%. 

  We can say more: suppose that in eq. (8) we replace all the terms 
  $\omega_2^2, \omega_3^2, \dots$ by $\omega_1^2$, where we assume we have 
  arranged the frequencies in order so that 

  $$\omega_1^2 \le \omega_2^2 \le \omega_3^2 \le \dots \le \omega_N^2 
  .\tag{9}$$ 

  Then it is obvious that we must have reduced the value of the Rayleigh 
  quotient, whatever the values of $\alpha_1, \alpha_2, \dots$, because both 
  numerator and denominator are sums of terms that are all positive because 
  they are squares. So 

  $$R \ge \frac{\alpha_1^2 \omega_1^2 + \alpha_2^2 \omega_1^2 + \dots+ 
  \omega_1^2 + \dots + \alpha_N^2 \omega_1^2 }{\alpha_1^2 + \alpha_2^2 + \dots 
  + 1 \dots + \alpha_N^2} = \omega_1^2 .\tag{10}$$ 

  In a similar way, if we replace all the $\omega_j^2$ by the largest value 
  $\omega_N^2$ we can deduce that $R \ge \omega_N^2$. 

  Gathering all these results up, we can deduce that if the quantity 
  $R=\frac{\mathbf{q}^T K \mathbf{q}}{\mathbf{q}^T M \mathbf{q}}$ is evaluated 
  using any vector $\mathbf{q}$, the result will be 

  (i) $\ge$ the smallest squared natural frequency; 

  (ii) $\le$ the largest squared natural frequency; 

  (iii) a surprisingly good approximation to $\omega_k^2$ if $\mathbf{q}$ is an 
  approximation to mode $\mathbf{u}^{(k)}$. (Formally, the Rayleigh quotient 
  $R$ is stationary near each mode.) 

  The last result can be used to estimate natural frequencies from fairly crude 
  guesses of mode shapes. As we will see in section 3.3.1, it can also be used 
  to reveal the effect on the natural frequencies of making small changes to 
  the system. We have demonstrated these results for discrete systems with a 
  finite number of degrees of freedom. But, as was explained in section 2.2.5, 
  results for discrete systems can always be extended to apply to continuous 
  systems like beams and plates. 