

  Some stringed instruments give the impression that they should be 
  approximated well using linear methods, but in fact crucial details for the 
  sound involve nonlinear effects. In this section, we dip into two examples. 
  Both of them were included in the set of plucked-string example sounds in 
  section 7.1: the harp and the lute. In both cases the nonlinear effects are 
  fairly weak, and we will be able to see roughly what is going on by simple 
  approaches. But a full discussion of nonlinear effects needs more care, and 
  after this introductory “toe in the water”, we will give a more systematic 
  description in the next chapter. 

  If you were able to pick out the harp from the set of instruments in Section 
  7.1, you will know that there must be something characteristic about the 
  sound, even of a single note. Probably, though, you find it hard to put the 
  distinctive quality into words. We can start by looking for physical features 
  that distinguish a harp from other plucked-string instruments. First, the 
  physical configuration is unusual. In the vast majority of plucked or struck 
  stringed instruments, the strings lie approximately parallel to the 
  soundboard. However, the strings of a harp meet the soundboard at an angle: 
  Fig.\ 1 shows the small harp used in this study, where the string-soundboard 
  angle is 35$^\circ$. 

  The other conspicuous physical feature that distinguishes the harp from 
  instruments like the guitar is that it has a separate string for each note. 
  This means that there are usually a large number of sympathetic strings 
  contributing to the sound of a given note, and the layout of strings also 
  influences many aspects of performance style. Two of these will be 
  particularly important here. Harpists normally pluck their strings closer to 
  the centre than is usual in other instruments. They can also impose a much 
  higher initial displacement than a guitarist is able to use, because there is 
  no equivalent of the danger of a “fret buzz”. The amplitude is only limited 
  by the spacing between strings, about 15~mm for the harp shown in Fig.\ 1. As 
  we will see shortly, string motion with large amplitude is likely to bring 
  nonlinear effects into play. 

  A harp string, like all other instrument strings, can vibrate in two 
  polarisations. One polarisation, in the plane of the strings, generates a 
  component of force that can excite soundboard vibration efficiently. String 
  vibration in the perpendicular plane, however, will drive the soundboard less 
  strongly. The player will normally excite string vibration with some mixture 
  of both polarisations, which means that harp notes might exhibit the double 
  decay phenomenon discussed in section 7.3. However, the effect does not in 
  fact seem likely to be a very strong one: Fig.\ 2 shows a plot similar to 
  Figs.\ 9--11 of section 7.3, using the string properties and measured bridge 
  admittance of the small harp. It can be seen that the red curve rarely rises 
  above the green curve. This means that the decay rates for vibration in the 
  two polarisations will not be significantly different. We must look elsewhere 
  for an explanation of ``harpish'' sound. 

  One candidate might involve longitudinal string vibration, something that is 
  often ignored in discussions of plucked instruments. The geometric 
  arrangement of strings and soundboard in a harp means that there is a direct 
  coupling between transverse and longitudinal motion in the string. The harp 
  soundboard will prefer to vibrate normal to its surface, moving the string's 
  termination in a direction that must involve both transverse and longitudinal 
  motion. 

  This coupling can be modelled using linear methods, similar to the synthesis 
  models described earlier (see section 5.4). The conclusion is that this 
  effect, too, seems to be rather minor: unimportant enough, in fact, that we 
  needn't go into details here: see [1] for the full story. Our quest for the 
  distinctive source of ``harpish'' sound seems to be thwarted at every turn. 
  However, we must not be too hasty. What has been mentioned and then dismissed 
  here is longitudinal string motion driven via linear vibration. But this is 
  where nonlinear effects start to appear: longitudinal string vibration, able 
  to drive the soundboard efficiently because of the string-soundboard geometry 
  of the harp, is also generated by another mechanism. 

  There are two aspects of nonlinear excitation of longitudinal string 
  vibration, and both have been shown to give audible effects in some musical 
  instruments. They both have their origin in the fact that when a string 
  vibrates transversely, the length of any small element of the string must 
  increase a little as a simple consequence of Pythagoras’ theorem: this is 
  called geometric nonlinearity. The first effect is that the small increase of 
  the total string length causes an increase in mean tension. Higher tension 
  raises the pitch of the transverse string vibration, of course, so during the 
  early part of a vigorously-plucked note a “pitch glide” can occur, starting a 
  little higher than the expected note. The pitch moves rapidly downwards as 
  the string motion decays, allowing the tension to fall back towards its 
  nominal level. 

  We can illustrate with a spectrogram. Figure 3 shows a portion of the 
  spectrogram of a strongly-plucked harp note: you can listen to it in Sound 1. 
  The spectrogram contains some stripes that are approximately vertical, 
  showing the frequency components of the note decaying away with time. But if 
  you look carefully, you can see that the stripes all bend a bit towards the 
  right in the bottom half of the plot. This is the pitch glide effect: in the 
  early part of the note the tension is a little higher so that all the 
  component frequencies are raised a bit. 

  The second effect of nonlinearity is more subtle. Dynamic axial forces are 
  generated in the string, leading to unexpected frequency components in the 
  sound, sometimes described as “phantom partials”. These forces can also 
  directly excite the longitudinal string resonances. Figure 3 shows examples 
  of both effects. Notice that the pattern of stripes is not regular. Starting 
  with the stripe around 2.7~kHz, instead of a single line there are two lines 
  close together. The next group near 3~kHz shows a pair slightly more widely 
  separated, and this progression of wider and wider spacing continues as you 
  move across towards the right-hand side of the plot. We will find out in a 
  bit how this pattern arises. 

  Figure 3 also shows a bright patch at early times (bottom of the plot) around 
  3.4 kHz. This is the frequency of the first longitudinal resonance of this 
  particular harp string. Conklin [2] has demonstrated the importance of 
  longitudinal string resonances for the tone of the piano, to the extent that 
  a piano designer needs to take the effect into account to achieve the best 
  tonal balance: some convincing sound demonstrations from Conklin's work can 
  be found in \tt{}<mark style="background-color:rgba(0, 0, 0, 0)" 
  class="has-inline-color has-black-color">this web link</mark>\rm{}. All these 
  effects of nonlinearity in string vibration are usually discounted for 
  non-metallic strings like nylon or gut harp strings, but Figure 3 
  demonstrates that they can in fact appear strongly in plucked harp notes. 
  They may well play a significant role in the distinctive character of harp 
  sound. 

  To investigate the pattern of phantom partials, and related nonlinear 
  effects, we will look at some more careful measured responses. To pluck a 
  string in the most well-controlled way, the best approach is to take a length 
  of very thin copper wire, loop it round the string, and pull gently until it 
  snaps. This method allows the position and direction of the pluck to be very 
  well controlled, and also gives a reliably repeatable amplitude between 
  plucks because the wire breaks at more or less the same force every time. 
  This kind of wire-break pluck will not sound the same as a normal harpist’s 
  finger pluck, of course, but it can reveal the underlying physics in the 
  clearest form. 

  To bring out the first piece of key behaviour we will look at a 
  carefully-chosen special case. The string was plucked at its mid-point, in 
  the direction in the plane of the strings and also in the perpendicular 
  direction. The responses were measured by a small accelerometer fixed to the 
  soundboard, very close to the attachment point of this string. The resulting 
  frequency spectra are shown in Figs.\ 4 and 5, for two different frequency 
  ranges. The pluck in the plane of the strings is shown in blue, the pluck in 
  the perpendicular plane in red. Because the two curves are sometimes on top 
  of each other, the positions of the main “string mode” peaks are indicated by 
  markers in corresponding colours. 

  Look first at Fig.\ 4, and concentrate initially on the blue curve. At first 
  glance, this shows what we expect to see: a sequence of tall and sharp peaks 
  corresponding to “string modes”, with some wiggles at lower level resulting 
  from the resonances of the harp body. But in fact the plot reveals a 
  surprise. Remember that the plucks were at the mid-point of the string. As we 
  found in section 5.4 (with details in section 5.4.1), linear theory would 
  predict that a mid-point pluck can only excite odd-numbered modes of the 
  string, because all the even-numbered ones have a nodal point at the centre. 
  But Fig.\ 4 shows peaks 1, 2 and 3 with roughly similar heights. After that, 
  mode 4 is indeed very weak, then mode 6 is moderately strong. 

  How does the high peak around mode 2 arise? To understand what has happened, 
  we need to know a bit more about the mechanics of nonlinear excitation of 
  longitudinal vibration. An approximate theoretical model is derived in the 
  next link. This model shows that the excitation of longitudinal motion is 
  governed by a function involving the square of the transverse motion: this is 
  the essential source of nonlinearity. A direct consequence of this square 
  law, also explained in the next link, is that the longitudinal motion may 
  contain extra frequencies compared to the transverse motion: it can involve 
  components at double the frequency of transverse components, and also the sum 
  and difference of pairs of transverse frequencies. 

  This gives an immediate clue about the unexpected peak number 2 in Fig.\ 4. 
  There are several ways that this frequency could be generated through the 
  square law: or, strictly, there are several ways that frequencies close to 
  this can be generated. If the transverse string motion consisted of perfect 
  harmonic frequencies, we could generate the frequency ``2'' by doubling the 
  fundamental frequency ``1'', but we could also obtain it as the difference 
  between pairs like 3 and 1, 5 and 3, 7 and 5, and so on. But things are a 
  little more complicated, because of the inharmonicity associated with bending 
  stiffness. These various combination frequencies all fall close to the 
  frequency ``2'', but they will all be slightly different from each other, 
  leading to a cluster of frequencies. This effect is hardly visible in Fig.\ 
  3, but a similar effect will become important shortly when we look at phantom 
  partials at higher frequency. The previous link explains the effect in more 
  detail. 

  Figure 4 contains direct evidence that this strong peak at approximately 
  double the fundamental frequency is caused by nonlinearly-generated axial 
  force. The key is to compare the peak heights of the blue and red curves. For 
  the odd-numbered peaks, the red curve is typically 10~dB lower than the blue 
  curve. But for the even numbers, the red and blue curves have very similar 
  heights. This is the expected pattern for nonlinear excitation. Figure~2 has 
  already shown that coupling to body vibration makes little difference to the 
  string damping, so the transverse string motion in the two plucks should be 
  very similar. This means that the nonlinear excitation of longitudinal motion 
  will also be very similar. The measurement is of soundboard acceleration. The 
  transverse string motion in the plane of the strings will excite the 
  soundboard more efficiently than the pluck in the perpendicular plane, giving 
  the observed pattern of heights of the odd-numbered peaks. However, 
  longitudinal force from both plucks will excite the soundboard identically, 
  so that approximately matching heights for the even-numbered peaks are 
  expected. That is exactly what the plot shows, both for the strong peak 2 and 
  for the smaller peaks 4 and 6. 

  The wire-break plucks used here were relatively weak compared to regular harp 
  playing. Even so, the nonlinearly-generated octave peak in Fig.\ 4 is of 
  similar height to peaks 1 and 3 which will be, at least in large part, 
  linearly excited. The conclusion is surely that nonlinear generation like 
  this is not a small perturbation to the sound of a harp. The special nature 
  of the frequency content from a mid-point pluck has been obliterated by the 
  nonlinear effect. Harp strings are indeed often plucked quite near the 
  mid-point, and they are also commonly plucked with far larger amplitude than 
  was the case in Figs.\ 4 and 5. Because of the square law nonlinearity, the 
  relative importance of nonlinear excitation in the sound is likely to be even 
  greater with more vigorous plucks. This mixture of linear and nonlinear 
  effects is a strong candidate for imparting a recognisable “harpish” sound 
  quality, and it may also give the harpist a way to vary the tone quality 
  depending on the strength of the pluck, even when the plucking point remains 
  the same. 

  Figure 5 shows the next frequency range for the same two spectra as in Fig.\ 
  4. This brings out some extra features: we have already seen a hint of them 
  in the spectrogram of Fig.\ 3. Instead of a single peak 8, we see a pair 
  labelled 8a and 8: these correspond to the close pair of lines near 2.7~kHz 
  noted in Fig.\ 3. After that we have pairs 9a and 9, 10a and 10, and so on. 
  The peaks labelled with an ``a'' are the phantom partials. They are generated 
  from the transverse modal frequencies of the string by frequency doubling or 
  by sum and difference combinations, and as we already noted these extra 
  frequencies don't quite match the linear modal frequencies of the string 
  because of the effect of inharmonicity from bending stiffness. 

  Recall from section 5.4 that the effect of bending stiffness is to make the 
  overtone frequencies of a stiff string progressively sharp, relative to an 
  ideal harmonic pattern. As a result, a frequency-doubled component based on, 
  for example, mode 5 will have a lower frequency than the actual mode 10, 
  because mode 10 has been sharpened by the stiffness effect. These are the two 
  frequencies we have labelled 10a and 10 in Fig.\ 5. Similarly, the sum of 
  frequencies 4 and 5 is a little lower than the actual frequency of mode 9: 
  these are the two frequencies labelled 9a and 9. But for our string plucked 
  at the mid-point, mode 4 was only driven very weakly, and that is the reason 
  that the peak 9a is very weak in Fig.\ 5: it relies on the strengths of both 
  peak 4 and peak 5. It is shown in reference [1] that the observed pattern of 
  peaks, including the phantom partials, can be very well explained across the 
  whole frequency range by the effects of bending stiffness. 

  Notice that all the ``a'' peaks in Fig.\ 5 show identical heights in the blue 
  and red curves, whereas the ``non-a'' peaks are higher in the blue curve. 
  This is the same pattern discussed earlier, for the peaks 2, 4 and 6 compared 
  to peaks 1, 3 and 5. Peaks with equal heights in the two spectra are a 
  hallmark of nonlinearly-generated longitudinal string motion. This suggests 
  that to be strictly consistent we should have labelled the peaks as 2a, 4a 
  and 6a in Fig.\ 4: for this special case of the mid-point plucks, these were 
  all ``phantom partials''. But we didn't recognise them as such because they 
  didn't obviously fall at unexpected frequencies. Another interesting detail: 
  the rounded peak labelled ``L'' in Fig.\ 5 is the first longitudinal 
  resonance of the string. It is not very prominent because it has high 
  damping, but notice that, again, it has virtually identical levels in the 
  blue and red curves. 

  The next step is to synthesise some sounds, so we can listen and compare them 
  with the measurements. There are several possibilities here. First, we can 
  use a version of the linear model developed in Chapter 5, extended to allow 
  for the string-soundboard angle of the harp, and the consequent coupling 
  between transverse and longitudinal motion. But then we want to give at least 
  an impression of the effect of the nonlinear coupling, generating phantom 
  partials. But nonlinear synthesis is a complicated business, still very much 
  an active research topic. So for the moment we will be content with the 
  simplest approximate model, as described in the previous link. 

  Even this is somewhat complicated. The procedure is first to find the 
  transverse string motion, then to use this to calculate the nonlinear driving 
  force for longitudinal motion. This approach involves several assumptions and 
  simplifications. First, the method is intrinsically approximate: the 
  governing equation derived in the previous link is based on discarding 
  everything except the leading-order effect. Second, it only works in one 
  direction: the transverse motion influences the longitudinal motion, but not 
  vice versa. Among other things, this means that the method is not capable of 
  giving the pitch glide effect. 

  Third, to compute the nonlinear driving force requires knowledge of the 
  transverse vibration at all positions on the string, not just near the 
  bridge. But the linear synthesis method we have been using does not provide 
  this: it only gives the motion at the bridge resulting from the combined 
  effects of string and soundboard. So we will use a crude approximation. We 
  saw in Fig.\ 2 that the transverse string motion in the harp is not very 
  strongly influenced by coupling to the soundboard. That being the case, we 
  can try using the simple closed-form response of a plucked string developed 
  in section 5.4.1, for the case of a rigidly terminated string. We can easily 
  extend that treatment to allow for the string’s damping and for the effect of 
  bending stiffness. 

  This approximate version of the transverse motion can now be used to 
  calculate the nonlinear driving force, then solve the equation from the 
  previous link for the longitudinal motion. Forces will then be exerted on the 
  harp soundboard by both transverse and longitudinal motion, and these can be 
  used in conjunction with the measured bridge admittance to give an estimate 
  of the soundboard acceleration. This, finally, can be compared directly to 
  the measurements from the wire-break plucks. 

  But there is still one more twist to the story. As shown in detail in the 
  previous link, it turns out that there are two contributions to the 
  longitudinal force exerted on the bridge by the vibrating string. The balance 
  between these two depends on the details of exactly how the string makes 
  contact with the soundboard. There are two extreme cases. One has both 
  effects fully active: this corresponds to assuming what is known as a pinned 
  boundary condition at the end of the string. The other case has only one of 
  the two effects, and corresponds to what is called a clamped boundary 
  condition. We will show results for both cases: in a real harp, the answer is 
  likely to be somewhere in between the two. 

  Figure 6 shows a selection of frequency spectra for these various synthesised 
  and measured responses, all for a pluck near the mid-point of the string. The 
  top curve, in green, shows the result of the simplified linear synthesis. 
  This curve shows why I chose to pluck near the mid-point, rather than exactly 
  at that point. In the computer it is easy to be accurate, but in the 
  measurements there is something of the order of 1~mm uncertainty about the 
  exact plucking point. For a special case like the mid-point this makes a 
  difference, especially when we plot results on a decibel scale showing a very 
  large range of amplitudes. If we had plucked at the exact mid-point, in this 
  simplified model of transverse motion, the even-numbered string modes would 
  have been completely absent. Instead, the assumed plucking point is 1~mm away 
  from the centre, to reflect the accuracy of the measurements and give a more 
  representative comparison. You can still see a strong even-odd pattern of 
  peak heights, but the even-numbered peaks are all visible to some extent in 
  the green curve. 

  The second and third curves in Fig.\ 6, in magenta and black, show the result 
  of the approximate nonlinear model with the two different boundary 
  conditions: pinned in magenta, clamped in black. All these synthesised curves 
  are to be compared with the measurements in the lowest two curves. These are 
  the same results as shown in Figs.\ 4 and 5, and the blue/red colour scheme 
  is the same as in those previous plots. 

  We have already talked about some the main features of the measured results, 
  especially the pattern of phantom partials. The three synthesised cases give 
  very different results for this pattern. The linear synthesis, of course, 
  does not show them at all. The magenta curve, for the nonlinear model with a 
  pinned boundary condition, shows a very high peak near the position of the 
  second string mode, as we expect from the earlier discussion. However, at 
  higher frequency it shows far too many phantom peaks, compared to either of 
  the measured plots. The black curve, for the clamped boundary condition, 
  comes far closer to matching the pattern of measurements. It is by no means a 
  perfect match --- look for example at the height around peak 4 --- but in the 
  main it shows encouraging agreement. 

  So, finally, what do these all sound like? Sounds 2--7 allow you to hear all 
  the cases we have been talking about. Sounds 2 and 3 are the two measured 
  wire-pluck waveforms. Don't forget that these are all giving the ``sound'' as 
  measured by an accelerometer on the harp soundboard, so we don't expect them 
  to sound the same as a microphone recording. The sharpness of the wire-break 
  plucks also gives them a lot of high-frequency content, in comparison with 
  the finger pluck in Sound 1. 

  Sound 4 is the result of a linear synthesis using the method of Chapter 5, 
  while Sound 5 gives the result of the simplified linear synthesis described 
  above. The simplified method does not take proper account of coupling to the 
  soundboard, so it lacks the initial ``thump'' coming from transient response 
  of the modes of the harp body. But it does contain part of the effect of the 
  body, because the string force has been converted into a version of the 
  soundboard acceleration using the bridge admittance: you can see in the green 
  curve of Fig.\ 6 some evidence of body resonances, in the low-level wiggles 
  at low frequency in between the narrow string peaks. 

  Sounds 6 and 7 give the two versions of the approximate nonlinear synthesis, 
  with pinned and clamped boundary conditions respectively. They sound, to my 
  ears at least, different from each other, and also clearly different from the 
  linear syntheses on the one hand, and from the measurements on the other. 
  What do we conclude? Perhaps that all the effects discussed in this section 
  have some influence on sound, but that we don't yet have a fully convincing 
  synthesis model. There is plenty of scope for further work here. 

  That is as much as we will say about the harp. But we now turn to another 
  plucked-string instrument with a distinctive sound, the lute. We will 
  discover that a different type of nonlinearity is significant here. Figure 7 
  shows the strings of a typical lute. This particular one has 8 courses: 
  during the long history of the lute the number of courses gradually increased 
  from 5 or 6 in the earliest years, up to 10 or more by the time of Bach. The 
  top string is single, the rest are in 2-string courses. The first 4 pairs are 
  tuned in unison, the lowest 3 in octaves. All the strings are plain 
  monofilaments; in this case they are all synthetic polymers, but originally 
  gut would have been used. 

  An intriguing puzzle is revealed by comparing the sounds of the lute and a 
  classical guitar. The top three strings of a classical guitar are nylon 
  monofilaments, apparently similar to the strings of the lute. A guitarist 
  normally uses fingernails to pluck the strings, while a lutenist uses the 
  flesh of the fingertips. When played in the usual way, both instruments make 
  a satisfyingly bright sound, but if a guitar is played using lute technique, 
  with finger flesh rather than nails, the result is very dull and 
  unsatisfying. The most important physical difference between the two 
  instruments is that the strings of the lute are thinner than the 
  corresponding strings of the guitar. 

  This poses a puzzle because of something we can deduce by putting together a 
  few things covered in earlier chapters. If we combine information from 
  sections 5.4.1 and 7.2.1, we can deduce that for an ideal textbook string, of 
  the kind undergraduates are usually taught about, the diameter of the string 
  should not make any difference whatsoever to the sound of a plucked note! 
  Specifically, we think about a string of a given length and material, tuned 
  to a given frequency, and plucked with a perfect step function of force like 
  a wire-break. We then calculate the response at the bridge of the instrument, 
  and it turns out that the string diameter simply cancels out. 

  We need some data. First, let’s listen to some sounds. An experiment was done 
  by fitting a guitar with four plain nylon strings of different diameters, all 
  tuned to the same frequency. Three of the strings were the top three from a 
  standard set of classical guitar strings. But all three were tuned to the 
  usual frequency of the top string (E$_4$, 329.7~Hz). A fourth string was 
  added, with a typical diameter for the top string of a lute, and this was 
  also tuned to the same note. This experiment was not done using an expensive 
  guitar! Tuning the second and, particularly, the third strings up to the top 
  E put more stress on the guitar and its tuners than it was really designed to 
  withstand. Fortunately, no disaster ensued. 

  We will concentrate on three of these strings: the light-gauge lute string 
  (“string 1” in what follows), the regular guitar top string (“string 2”) and 
  the tuned-up guitar second string (“string 3”). The diameters of these three 
  strings were respectively 0.50~mm, 0.69~mm and 0.78~mm. Sounds 8, 9 and 10 
  give single notes on these three strings, plucked with a thumb using lute 
  technique. As with the harp results earlier in this section, these “sounds” 
  are actually the result of recording the output of a small accelerometer 
  fixed to the guitar bridge. I think you will agree that there is a big 
  difference of sound between these examples: brightest for string 1, but quite 
  dull for string 3 even with the high-frequency boost provided by the 
  accelerometer recording. 

  Sounds 11, 12 and 13 are the result of plucking the same three strings with a 
  breaking wire. All three of them sound brighter than the corresponding thumb 
  plucks. Comparing Sounds 9 and 12 illustrates the lute/guitar contrast 
  commented on at the start of this discussion. These two sounds are both based 
  on the normal top E string of a guitar, and the difference of brightness in 
  the sound between the two is very marked. Even with the artificial “treble 
  boost” provided by the accelerometer recording, the thumb pluck gives a 
  rather unsatisfactory dull sound. The fingernail of a classical guitarist 
  will give an effect somewhere between a thumb pluck and a wire-break pluck, 
  giving the guitar a sufficiently bright sound. 

  The contrasting brightness between the thumb plucks and wire-break plucks is 
  no surprise: this is exactly the difference predicted by linear theory. We 
  saw in section 5.4.1 that the contribution of each mode to the pluck response 
  of a string is found by expressing the initial shape as a mixture of the mode 
  shapes, and for a string this amounts to expanding that initial shape as a 
  Fourier series. The sharp corner given to the string by the wire break 
  requires a lot of short-wavelength terms in that Fourier series. This in turn 
  creates a lot of high frequency content in the sound. The more rounded shape 
  from a thumb pluck only involves relatively long wavelengths, and so it 
  excites the higher modes of the strings rather little, and produces less high 
  frequency content in the sound. A fingernail pluck will fall somewhere 
  between the two in terms of sharpness. 

  So can we track down what is responsible for the difference of sound between 
  strings with different diameters? Several effects we have already encountered 
  might play a role. First, we know that strings of different diameters will 
  have different inharmonicity, and also different damping behaviour. Both 
  effects are associated with a difference of bending stiffness: a thicker 
  string will have higher stiffness, leading to greater inharmonicity and also 
  to a reduced cut-off frequency associated with damping. We have already seen 
  some data about this damping effect in Fig.\ 2 in section 7.2, reproduced 
  here as Fig.\ 8 to remind you. The strings shown in black and red here are in 
  fact string 1 and string 2 of the present study. (The string shown in green 
  is not relevant here, though. That string had diameter 1.68~mm, far thicker 
  than any of the four strings used in the guitar experiment.) 

  Both these effects associated with bending stiffness can be captured 
  accurately by linear synthesis: Fig.\ 8 reminds us that we have a good 
  theoretical model of the damping of these strings (shown in the lines which 
  follow the experimental points quite closely). Sounds 14, 15 and 16 give 
  synthesised results for the three strings, corresponding to wire-break plucks 
  at the same position as the measured responses in Sounds 11, 12 and 13. To my 
  ears, these all sound remarkably similar. This suggests that differences in 
  bending stiffness, inharmonicity and damping are not a very big contribution 
  to the contrast between the three strings that we heard in Sounds. 8, 9 and 
  10. 

  Perhaps the nonlinear effects we investigated in the harp might be important 
  in the guitar and lute as well? It is certainly true that “phantom partials” 
  can be seen in detailed spectra of the wire-break plucks of the strings of a 
  guitar or lute: see [3] for some examples. Sounds 17, 18 and 19 give 
  synthesised plucks on the three strings using the approximate nonlinear 
  approach used for the harp in Sounds 6 and 7. The three sounds are clearly 
  different from the linear versions in Sounds 14, 15 and 16. They are also 
  slightly different from each other, but nowhere near enough to match the 
  differences heard in the three wire-pluck measurements, let alone in the 
  three thumb plucks. 

  So what remains? The full detective story of tracking down the evidence is 
  explained in [3], but you may have already guessed the most likely answer 
  after listening to Sound 8 and Sound 11. The very thin string, under rather 
  low tension, is almost certainly hitting some other part of the structure 
  when it vibrates. There are various possibilities, but a plausible culprit is 
  the fret closest to the nut. The early part of the transverse string motion 
  following a pluck will be similar to the pattern for an ideal textbook 
  string: see the animation in Fig.\ 3 of section 5.4. After half a cycle, the 
  string shape becomes an approximate mirror image of the initial displacement. 
  The plucking point was 30~mm from the bridge, so the maximum displacement 
  will occur 30~mm from the nut, fairly close to the first fret at about 37~mm. 
  If the string hits a fret, a short force pulse is likely to occur. This would 
  excite additional vibration, with a very important extra feature. If the 
  original pluck was band-limited, like the thumb plucks, a fret impact could 
  excite vibration over a far wider bandwidth, just as was heard in the sound 
  files (and also seen in measurements: see [3]). 

  The original observation, that lighter-gauge strings sound brighter, would 
  then be explained by the simple fact that a player will naturally tend to 
  excite a lower-tension string to larger amplitude, and thus make impacts or 
  other nonlinear interactions more likely. In the case of wire plucks, the 
  pluck force is explicitly kept constant, so that different amplitudes of 
  string motion are excited. Musicians will probably not be thinking about 
  plucking force as such, but they will be aware of the loudness of sound they 
  wish to create. To achieve comparable loudness on strings of different 
  tensions will require a larger vibration amplitude in a lighter string. 

  A musician or instrument maker would probably describe a fret impact as a 
  “buzz”, but that term is usually applied to cases of multiple impacts with a 
  fret so that a sustained effect is clearly audible. Guitar makers normally go 
  to some lengths to avoid such buzzes. But we are suggesting here that a 
  similar effect at a low level is probably essential to the tone quality of 
  the lute, and other early plucked-string instruments in which the flesh of 
  the fingers is used, rather than fingernails or any kind of plectrum. The 
  brightness and crispness of tone associated with the lute and related 
  instruments relies on their light-gauge stringing, probably via some 
  manifestation of the effect discussed here. 

  This idea has some interesting consequences. First, certain design details of 
  the lute are perhaps clarified. The frets on a lute (or a viol, or other 
  stringed instruments of this early period) are not made of metal like the 
  frets of a guitar. Instead, they are made of lengths of gut tied around the 
  neck of the instrument. The result is softer than a modern metal fret, and 
  this may reduce the abruptness of a string-fret contact so that the contact 
  force is more band-limited and the sound is less obtrusive than a buzz on a 
  guitar. In early lutes, the frets were conventionally tied with a doubled 
  length of gut. This slightly curious feature is, perhaps among other things, 
  a “buzz promoter”. The vibrating length of the string is probably terminated 
  at the rear strand of the fret, but the vibrating string might buzz against 
  the front strand to give some increased brightness to the sound. 

  Another consequence is that the desired effect of a slight buzzing contact 
  might be extremely sensitive to the set-up of the instrument. The maker is 
  probably able to shape the sound by adjusting various small details, but it 
  is also likely that the sound is quite sensitive to changes in temperature 
  and humidity which may cause movement in the wood structure of the instrument 
  body and change the set-up details. To a musician, the instrument may seem 
  rather “twitchy” to changes in the weather, sounding harsh on some days but 
  more mellow on others. 

  There is another effect on the sound of a lute connected with humidity 
  sensitivity, to do with the way the instrument is played. The recommended 
  technique is to press down with the flesh of a finger pad on a string (or a 
  pair of strings in a course), then pull sideways. The “pluck” is thus 
  produced by frictional slipping against the skin. Now, it is well known that 
  the coefficient of friction of human flesh is sensitive to the state of 
  hydration: dry skin has lower friction. That means that when a player has dry 
  skin, they will have to press down harder to increase the normal force in 
  order to achieve the required frictional force for a pluck of a given 
  strength. This will change the orientation of the string vibration, and the 
  extra normal displacement may make buzzing more likely. The result is that 
  the instrument may sound harsh. 

  This fits well with the experience of players, who know that soaking their 
  hands to hydrate the skin can have a big effect on tone quality. Once this 
  mechanism is understood, it can be seen that players may have other choices 
  to achieve the same effect: there are various substances available to enhance 
  friction. Powdered rosin is one, but this makes the fingers feel undesirably 
  sticky. A better option may the “chalk” (magnesium carbonate powder) that 
  rock climbers use to improve friction in their finger-tips. The effect may be 
  longer-lasting than that of hydration, if playing in a dry environment. 



  \sectionreferences{}[1] J. Woodhouse, ``The acoustics of a plucked harp 
  string'', Journal of Sound and Vibration \textbf{523}, 116669, (2022).~~DOI 
  10.1016/j.jsv.2021.116669 

  [2] H. A. Conklin, ``Generation of partials due to nonlinear mixing in a 
  stringed instrument'' Journal of the Acoustical Society of America 
  \textbf{105}, 536–545, (1999). 

  [3] J. Woodhouse, ``Influence of damping and nonlinearity in plucked strings: 
  Why do light-gauge strings sound brighter?'', Acta Acustica united with 
  Acustica \textbf{103}, 1064–1079 (2017). 