  Here is an informal listening challenge to try. Necks were fabricated from 
  four different timbers, but all other details were nominally identical. The 
  necks were fitted in turn to the same banjo pot, and several recordings were 
  made of the same tune. The question now is: can a listener recognise which 
  neck is which? 

  \fig{figs/fig-5e7b23d5.png}{\caption{Figure 1. The four necks; from the top: 
  maple, mahogany, walnut and white oak. Image copyright David Politzer, 
  reproduced by permission.}} 

  \fig{figs/fig-70884846.png}{\caption{Figure 2. The test banjo, with a 
  mahogany neck fitted. Image copyright David Politzer, reproduced by 
  permission.}} 

  A set of these sound files has been trimmed to the same short snatch of music 
  in each case. First, you hear these played using necks which are identified 
  by their timber choice. After that, there are 5 files labelled A--E. These 
  contain one example of each of the four timbers, plus a second recording made 
  using one of them. Can you marry these up with the first four sounds? You 
  might want to use your best available audio reproduction option for this 
  task, either headphones or loudspeakers. After you have tried it, you can 
  find the key \tt{}here\rm{}. More detail of the experiment can be found 
  \tt{}here\rm{}. 

\audio{}

\audio{}

\audio{}

\audio{}

\audio{}

\audio{}

\audio{}

\audio{}

\audio{}

  If you can hear differences and do at least some identifications correctly, 
  that tells you something --- but we need to be a bit careful about exactly 
  what. You may be hearing differences that are specifically caused by the 
  different wood species in the necks, but you might also be hearing 
  differences associated with small differences in the nut grooves, action 
  height or other aspects of setup. Later, in section 7.4, we will meet the 
  example of the lute, where it seems that an important ingredient of the sound 
  quality is caused by a very subtle level of ``fret buzz'', contact between 
  the played strings and the higher frets. This example gives a warning against 
  leaping to conclusions about the cause of sound differences. 