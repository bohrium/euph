  To calculate the reflection characteristics of a single tone-hole, we can 
  consider an infinite cylindrical tube with cross-sectional area $S$ carrying 
  a short side-branch tube with effective length (including end correction) $h$ 
  and cross-sectional area $C$. The geometry is sketched in Fig.\ 1. The 
  spatial coordinate along the tube is $x$, and the junction is centred at 
  $x=0$. 

  \fig{figs/fig-5cca257c.png}{\caption{Figure 1. Sketch of the geometry: a 
  short open tube is joined at $x=0$ to an infinite cylindrical tube. Incident, 
  reflected and transmitted sinusoidal pressure waves are as indicated.}} 

  We send in a sinusoidal pressure wave with frequency $\omega$ from the 
  negative $x$ direction, with unit amplitude. When this interacts with the 
  tone-hole it generates reflected and transmitted pressure waves, with complex 
  amplitudes $R$ and $T$ as indicated in the figure. So the total acoustical 
  pressure field is 

  \begin{equation*}p'=e^{i(\omega t -kx)} + R e^{i(\omega t +kx)} 
  \mathrm{~~~for~~~}x \le 0 \tag{1}\end{equation*} 

  \noindent{}and 

  \begin{equation*}p'=T e^{i(\omega t -kx)} \mathrm{~~~for~~~}x \ge 0 
  \tag{2}\end{equation*} 

  \noindent{}where $k=\omega /c$ and $c$ is the speed of sound. Using equations 
  (2) and (3) from section 11.1.1, the associated volume flow rates (positive 
  to the right) are 

  \begin{equation*}u=\dfrac{1}{Z_0}\left(e^{i(\omega t -kx)} -- R e^{i(\omega t 
  +kx)} \right) \mathrm{~~~for~~~}x \le 0 \tag{3}\end{equation*} 

  \noindent{}and 

  \begin{equation*}u=\dfrac{T}{Z_0} e^{i(\omega t -kx)} \mathrm{~~~for~~~}x \ge 
  0 \tag{4}\end{equation*} 

  \noindent{}where $Z_0$ is the characteristic impedance of sound waves in air. 

  At $x=0$ the pressure is 

  \begin{equation*}p'=(1+R)e^{i \omega t} = Te^{i \omega t} . 
  \tag{5}\end{equation*} 

  This pressure acts on the side branch, which we know from section 11.1.1 has 
  an input impedance 

  \begin{equation*}Z_{\mathrm{hole}}=\dfrac{i Z_0}{C} \tan kh \approx \dfrac{i 
  Z_0}{C} kh \tag{6}\end{equation*} 

  \noindent{}where the final approximate expression is valid because $kh \ll 1$ 
  for our tone-hole. The result is a volume flow rate into the hole 

  \begin{equation*}u_{\mathrm{hole}}=\dfrac{C}{iZ_0 kh}T e^{i \omega t} . 
  \tag{7}\end{equation*} 

  Now if we disregard the compressibility of the small volume at the junction, 
  shown as a dotted box in Fig.\ 1, we can equate the volume flows into and out 
  of this box to obtain 

  \begin{equation*}\dfrac{S}{Z_0}(1-R)=\dfrac{S}{Z_0}T +\dfrac{C}{iZ_0 kh}T 
  \tag{8}\end{equation*} 

  \begin{equation*}=\left(\dfrac{S}{Z_0}+\dfrac{C}{iZ_0 kh} \right) (1+R) 
  \tag{9}\end{equation*} 

  \noindent{}using equation (5). Rearranging, we find 

  \begin{equation*}R=-\dfrac{\lambda}{\lambda+2ikh} \tag{10}\end{equation*} 

  \noindent{}where $\lambda=C/S$ is the area ratio of the hole to the tube. It 
  follows from equation (5) that 

  \begin{equation*}T=1+R = \dfrac{2ikh}{\lambda+2ikh} . \tag{11}\end{equation*} 

  It is reassuring to check that these results satisfy the conservation of 
  energy. The average energy intensity carried by a plane wave with complex 
  amplitude $A$ is $|A|^2/2Z_0$ by equation (11) of section 4.1.3, so the total 
  energy flux away from the tone-hole is 

  \begin{equation*}\dfrac{S(|R|^2 + |T|^2)}{2Z_0} = \dfrac{S(\lambda^2+ 
  4k^2k^2)}{2Z_0|\lambda+2ikh|^2} = \dfrac{S}{2Z_0} \tag{12}\end{equation*} 

  \noindent{}which is indeed equal to the total energy flux of the incoming 
  wave, which had unit amplitude. 

  It is interesting to note that the reflection and transmission of waves by 
  the tone-hole is governed by a single non-dimensional parameter 

  \begin{equation*}K=\dfrac{2hk}{\lambda} \tag{13}\end{equation*} 

  \noindent{}since 

  \begin{equation*}R=-\dfrac{1}{1+iK}, \mathrm{~~~}T=\dfrac{iK}{1+iK} . 
  \tag{14}\end{equation*} 

  When $K \ll 1$, $R \approx -1+iK$ and $T \approx iK$. This is the limit of a 
  large tone-hole, in which virtually all the energy is reflected. Conversely, 
  if $K \gg 1$, $R \rightarrow 0$ and $T \approx 1$. The most ``interesting'' 
  behaviour thus occurs when $K \approx 1$: this is the ``Goldilocks'' region 
  where reflection is not too weak, not too strong. We might expect tone-holes 
  designed with fork fingerings in mind to fall in this region, for example. 