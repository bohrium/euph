  To obtain an index of merit for the selection of a material for a soundboard, 
  we can combine two things that are covered elsewhere on this site. The aim is 
  to maximise the typical level of the bridge admittance of our soundboard, 
  keeping the plan dimensions of the soundboard fixed but adjusting the 
  thickness in order to keep the natural frequencies in roughly their usual 
  position. Specifically, we can keep the modal density fixed. We already know 
  from section 3.2.4 that the modal density of a rectangular plate of sides $a 
  \times b$ and thickness $h$ made from isotropic material with Young's modulus 
  $E$, Poisson's ratio $\nu$ and density $\rho$ is 

  $$n= \dfrac{ab}{2 \pi h} \sqrt{\dfrac{3 \rho (1-\nu^2)}{E}} . \tag{1}$$ 

  We are keeping $a$ and $b$ fixed, and we will ignore the small influence of 
  $\nu$. The conclusion is that to keep the modal density constant for 
  different materials we want to scale 

  $$h \propto \sqrt{\dfrac{\rho}{E}} . \tag{2}$$ 

  We now need a formula for the typical level of the admittance of the plate, 
  when driven at a point representing the bridge. We can make use of a result 
  that will be described properly a little later on, in section 5.3.2. It is 
  based on a general result called ``Skudrzyk's mean value method'', which says 
  that the mean value of the decibel plot of the plate admittance is exactly 
  the same as the admittance of an infinite plate with the same thickness and 
  material properties. This infinite-plate admittance has a simple closed-form 
  solution, given in eq. (2) of section 5.3.2: 

  $$Y_\infty=\frac{1}{4h^2} \sqrt{\frac{3(1-\nu^2)}{E \rho}} . \tag{3}$$ 

  This gives us what we want, a simple measure of the typical level of the 
  admittance which captures its dependence on the material properties. 
  Substituting for $h$ from eq. (1), the index of merit that we wish to 
  maximise is thus proportional to 

  $$\dfrac{E}{\rho} \times \sqrt{\dfrac{1}{E \rho}} = \sqrt{\dfrac{E}{\rho^3}} 
  . \tag{4}$$ 

  Since the square root is a monotonic function, we will obtain the same answer 
  by maximising the combination $E/\rho^3$. 

  We can write the quantity from eq. (4) in a different way. The speed of sound 
  in the material is given by 

  $$c=\sqrt{\dfrac{E}{\rho}}\tag{5}$$ 

  so that we can write 

  $$ \sqrt{\dfrac{E}{\rho^3}} = \dfrac{c}{\rho}. \tag{6}$$ 

  The final step is to take account of the fact that wood is by no means 
  isotropic. An orthotropic material has two different Young's moduli $E_1$ and 
  $E_2$ in two perpendicular directions: in the case of wood these would be 
  ``along the grain'' and ``across the grain''. The full details of orthotropic 
  plate vibration are more complicated: we will return to it later (see section 
  ?). But for now, it is good enough to use a standard approximation. We get 
  roughly the right answer if we replace $E$ everywhere in this section by the 
  combination $\sqrt{E_1E_2}$. This geometric mean of the two moduli has a very 
  simple representation on the logarithmic axes of the Ashby plots: it falls at 
  the mid-point of the two separate moduli, in much the same way that an 
  arithmetic mean would fall at the midpoint on a linear scale. 