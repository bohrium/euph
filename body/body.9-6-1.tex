  The most ingenious experiment for investigating the friction force driving a 
  bowed string was performed by Bob Schumacher, at Carnegie-Mellon University 
  in Pittsburgh. A regular violin E string was mounted on a rather rigid 
  fixture, allowing it to be tensioned with a tuning peg in the usual way. This 
  string was ``bowed'' using a glass rod which had been dip-coated in rosin 
  from solution. The rod was fixed to a movable carriage, driven by a 
  computer-controlled system so that a desired velocity-time profile could be 
  reliably reproduced on every individual bow stroke. The precise height of the 
  string's terminations could be adjusted with micrometer screws, to set the 
  normal force between rod and string. 

  At both ends of the string, force-measuring sensors were built in to the 
  holding fixture. These allow the forces exerted by the vibrating string on 
  the terminations to be monitored, without interfering in any way with the 
  string's vibration. The two force signals were collected by a digital 
  data-logger at a high sampling rate. As will now be explained, these signals 
  can be combined to give an estimate of the velocity at the friction force at 
  the bowed point on the string. 

  We can now make use of some equations from section 9.2.1. We define incoming 
  and outgoing velocity waves on the two sides of the bowed point by 
  $v_1^{(i)}(t), v_1^{(o)}(t), v_2^{(i)}(t)$ and $v_2^{(o)}(t)$: these are 
  illustrated schematically in Fig.\ 1, in a format we have used before. The 
  string velocity $v(t)$ at the bowed point is given by two equivalent 
  expressions: 

  $$v=v_1^{(i)}+v_1^{(o)}=v_2^{(i)}+v_2^{(o)} . \tag{1}$$ 

  We can also relate the friction force to these velocity waves. The outgoing 
  wave on one side is given by the incoming wave from the other side, plus the 
  new contribution arising from the friction force $f(t)$, so that 

  $$v_1^{(o)} = v_2^{(i)} + \dfrac{f}{2Z_0} \mathrm{~~~ and~~~} v_2^{(o)} = 
  v_1^{(i)} + \dfrac{f}{2Z_0} . \tag{2}$$ 

  Rearranging, we obtain two equivalent expressions 

  $$f=2Z_0(v_1^{(o)}-v_2^{(i)})=2Z_0(v_2^{(o)}-v_1^{(i)}). \tag{3}$$ 

  The conclusion is that if we can estimate the current values of the four 
  velocity waves, we can construct two different estimates of $v(t)$ and 
  $f(t)$. 

  \fig{figs/fig-139b71f1.png}{\caption{Figure 1. Schematic diagram of the 
  incoming and outgoing velocity waves at the bowed point on a string}} 

  For the simplest case of an ideal string with no bending stiffness, no 
  damping, and with rigid terminations, this is very easy. First we recall that 
  the force exerted by the string at a rigid termination is given by $2Z_0$ 
  times the incident velocity wave at the termination. Referring to Fig.\ 1, we 
  thus see that $v_1^{(i)}$ is simply given by the measured bridge force at the 
  point in the diagram labelled A, scaled by $1/2Z_0$ and delayed by the travel 
  time to the bowed point. The other three velocity waves are given similarly 
  in terms of the measured bridge force at the points labelled B, C and D. In 
  the case of points B and D, the signal is advanced rather than delayed: we 
  deduce the outgoing waves at the bow by looking at future values of bridge 
  force. 

  Of course, the real string does not exactly satisfy the three assumptions we 
  have just made. However, it turns out that only one we need to worry about 
  for this particular experiment is the effect of bending stiffness. The rig is 
  designed with rather rigid terminations, and energy loss in the string is 
  very small during the very short travel times involved in this argument. But 
  we do need to think about the consequences of bending stiffness, which means 
  that different frequency components in the velocity waves travel at somewhat 
  different speeds: higher frequencies, or shorter wavelengths, travel a little 
  faster. High frequencies are important for this reconstruction process, 
  because the transitions between sticking and slipping happen very fast, and 
  that automatically means that they are influenced by very high frequencies. 

  We can see the effect of this bending stiffness in the results shown in Fig.\ 
  2. This shows the two measured force waveforms, not when the string is bowed 
  but when it is set into vibration by a wire-break pluck at the position where 
  the bow will be placed. Look at the early part of the blue curve, showing the 
  force at the end of the string further from the bow. There is an initial flat 
  portion , then a big downward jump marking the moment when the main 
  ``corner'' from the pluck arrives at the sensor. But before that jump there 
  is a wiggly ``precursor''. This is the effect of bending stiffness: the high 
  frequency components of the jump have arrived a little earlier because they 
  travel faster. If you look carefully, you can see that the frequency is 
  changing through this precursor: highest at the front, then falling a little 
  until the main jump occurs. 

  \fig{figs/fig-f86c7707.png}{\caption{Figure 2. The two measured bridge force 
  waveforms following a wire-break pluck at the point where the ``bow'' will be 
  placed.}} 

  The good news is that there is an approximate mathematical expression for 
  this effect of frequency-dependent wave speed. We can use pluck responses 
  like the ones in Fig.\ 2 to determine the parameters of that expression, and 
  then use it as a basis for digital filters to remove the effect from our four 
  velocity wave estimates. The gory details can be found in reference [1]. 

  We can now show an example of the reconstruction procedure in action. The 
  following plots all refer to the same section of data used to generate Figs.\ 
  2 and 3 of section 9.6. First, Fig.\ 3 here shows the original measured 
  bridge forces at the two ends of the string during the bow stroke. They have 
  been separated vertically to make the plot easier to see. Both waveforms show 
  the characteristic Helmholtz sawtooth shape. 

  \fig{figs/fig-d105d72b.png}{\caption{Figure 3. The two measured bridge force 
  waveforms, for the same segment of data shown in Figs. 2 and 3 of section 
  9.6. The plots are offset for clarity.}} 

  From these, following the procedure outlined above, we can compute two 
  estimates of the string velocity $v(t)$ and the friction force $f(t)$. These 
  are shown in Fig.\ 4 and 5. There is not perfect agreement between the two, 
  but they are close enough to be reassured that the method is working 
  reasonably well. The average of each pair of waveforms then gives our final 
  best estimate, and those averages are what was used for the plots in section 
  9.6. 

  \fig{figs/fig-938973ba.png}{\caption{Figure 4. The two versions of 
  reconstructed string velocity, from the data of Fig. 3.}} 

  \fig{figs/fig-08112f36.png}{\caption{Figure 5. The two versions of 
  reconstructed friction force, corresponding to the velocities in Fig. 4.}} 

  \sectionreferences{}[1] J. Woodhouse, R. T. Schumacher and S. Garoff, 
  “Reconstruction of bowing point friction force in a bowed string”;  Journal 
  of the Acoustical Society of America  \textbf{108} 357–368 (2000). 