

  Before looking at specific tuned percussion instruments, it is useful to meet 
  some simple idealised systems that will form building blocks for real 
  instruments, and indeed for the vibration behaviour of many other engineering 
  structures. We have already looked at the stretched string --- but stringed 
  instruments are not usually regarded as percussion instruments. Instead, 
  these often consist of curiously-shaped pieces of metal or wood. Before 
  plunging into the complication associated with the curious shapes, there are 
  some idealised ``textbook'' cases that can be used to give insights into the 
  physics of vibration, and also to generate sound examples so we can start to 
  hear the consequences. These idealised cases come in a hierarchy of 
  increasing complication: beams, flat plates, and curved plates or shells. 
  There is one other idealised system relevant to musical instruments: the 
  stretched membrane, as in a drum or a banjo. We will defer that for the 
  moment, and look at it in Section 3.6 when we examine tuned drums. 

  The simplest of these systems is a slender bar, more usually called a beam in 
  the context of vibration. The behaviour of such beams is based on bending. 
  The restoring force for the vibration comes from the fact that if a beam is 
  made to bend a little, it will try to straighten out again. If you picture 
  the beam as built out of a bundle of fibres running along the length, the 
  result of bending the beam is that fibres on one side are stretched while the 
  ones on the other side are squashed. This produces forces of tension and 
  compression on the two sides, and hence a bending moment tending to 
  straighten the beam. Of course, if you bend a beam too vigorously it may 
  break or stay bent: but this is not what you want to happen when using it to 
  play music. The amplitude of vibration is small enough that the beam springs 
  back each time it is bent, then overshoots because of its inertia, and 
  repeats this process so that vibration occurs. 

  The simplest model for bending vibration of beams is called Euler-Bernoulli 
  theory. It is a linear model, appropriate for long, thin beams. As explained 
  in the next link, this approach can be used to find the vibration modes and 
  corresponding natural frequencies of something like a xylophone bar, for 
  which both ends are free: there is no external force or moment applied at the 
  ends. The first few mode shapes are shown in Fig.\ 1. These modes are 
  appropriate to a bar with a uniform cross-section, like the metal bars of a 
  toy xylophone. We will think about bars with varying cross-sections in the 
  next section. 

\moobeginvid\begin{tabular}{ccc} \vidframe{ 0.30 }{ vids/vid-3abd40ef-00.png }&\vidframe{ 0.30 }{ vids/vid-3abd40ef-01.png }&\vidframe{ 0.30 }{ vids/vid-3abd40ef-02.png } \end{tabular}\caption{Figure 1.  The first three modes of a free-free bending beam. The movie shows one cycle of the fundamental mode, alongside the second and third modes at the correct relative frequencies. Because the second and third modes do not have a period matching the fundamental mode, their animations will ``jump'' when the display loops. This simply a display artefact.}\mooendvideo

  Figure 2 shows how the natural frequencies of the free-free beam modes vary 
  with mode number. They are shown as ratios to the frequency of the lowest 
  mode. In order to check whether overtones fall close to harmonics of the 
  fundamental, it is useful to see these in numerical form: the first few are 
  1.00, 2.76, 5.40, 8.93, 13.34. We immediately see that they do not do so: 
  none of the ratios are very close to whole numbers. The other thing we can 
  see from these frequency ratios, and from Fig.\ 2, is that the modes are much 
  more sparse than we saw for the simple model of a vibrating string. There, 
  the $n$th overtone occurred at $n$ times the fundamental frequency, indicated 
  by the black dashed line in the plot. For the beam, as explained in the link 
  above, the approximate formula for the $n$th overtone frequency contains a 
  factor $(n+1/2)^2$ rather than $n$ so the spacing between adjacent 
  frequencies gets wider and wider. 

  \fig{figs/fig-88692804.png}{\caption{Figure 2. Frequencies of the first 10 
  modes of a free-free beam, expressed as ratios to the fundamental frequency 
  (red). Blue stars show the effect of rounding each ratio to the nearest whole 
  number: see main text for explanation. The black dashed line shows the 
  corresponding result for an ideal stretched string as examined in section 
  3.1.1.}} 

  From a given vibration fingerprint --- a recipe of modal frequencies, 
  amplitudes and decay rates --- it takes only a very simple computer program 
  to create a sound: this process is called additive synthesis and is the basis 
  of one kind of electronic musical instrument. We have already seen how to use 
  a computer for the reverse process: to record a sound, then use the FFT to 
  analyse it into its component sine waves to find out which frequencies are 
  present and how they relate to each other. Additive synthesis can be applied 
  to the results of such an analysis, to recreate a version of the sound and 
  find out if it really does sound similar to the original. More relevant to 
  the present purpose, the method can be used to make a sound based on 
  theoretical predictions of frequencies and mode shapes. The link below gives 
  some detail of how these sound examples were computed. 

  Sound 1, below, allows you to listen to a short computer-synthesised scale 
  based on the ideal beam frequencies given above. To generate these sounds, a 
  choice has to be made of modal damping factors. We will explore the effect of 
  changing the damping shortly, but for now a low level of damping has been 
  selected, giving a rather ``metallic'' sound. Specifically, we use 
  $\zeta=0.001$, corresponding to a Q-factor of 500 --- the same value is used 
  for all modes. (Greek letter $\zeta$, pronounced ``zeta''.) We will generally 
  use Q-factors to describe the damping levels in future examples: it is 
  perhaps simpler to appreciate the difference between Qs of 300 and 500, than 
  between $\zeta$ values of 0.0017 and 0.0010. 

  \aud{auds/aud-e08d5eda-plot.png}{\caption{Sound 1. A synthesised scale, based 
  on the frequencies of an ideal bending beam. The fundamental frequency is 196 
  Hz, the modes all have a Q-factor 500, and a hard striking hammer is 
  assumed.}} 

  We can use this beam example to explore a bit of ``virtual vibration 
  engineering''. We can modify the frequencies in the computer to make the 
  relations between overtones more harmonic, make a new sound file, and listen 
  to the effect. In this way, we get a first test of the idea that ``more 
  harmonic means more musical''. Sound 2 demonstrates the effect of rounding 
  each frequency ratio to the nearest whole number, so that all the overtones 
  fit exactly to a harmonic series. Everything else is identical to Sound 1. 
  The blue stars in Fig.\ 2 show the effect of this rounding: on that plot it 
  looks like a small change, but you probably find that the sound is very 
  significantly different. 

  \aud{auds/aud-eb929bdb-plot.png}{\caption{Sound 2. A synthesised scale with 
  all settings identical to Sound 1, except that each modal frequency has been 
  rounded to the nearest whole-number multiple of the fundamental frequency.}} 

  Is Sound 2 ``more musical''? But perhaps less complex and ``interesting'' 
  than Sound 1? The impression made by Sound 1, to my ears at least, is that I 
  can tell immediately that a scale is being played. However, comparing with 
  Sound 2 reveals that I wasn't as certain as I thought I was about the pitch 
  of each individual note in the scale. The notes of Sound 2 have exactly the 
  same fundamental frequencies as those of Sound 1, but the impression is that 
  all the notes sound lower in pitch. For Sound 2 the pitch of each note is 
  absolutely clear: I could hum them with confidence. Listening more closely to 
  Sound 1, I can hear different `pitches' within each note, including the 
  fundamental which, I reluctantly admit, does indeed sound the same as the 
  pitch of Sound 2. 

  What is going on? We can make a guess. Probably, the progressively wider 
  spacing of the overtones of Sound 1, as shown in Fig.\ 1, gives your brain a 
  challenge. It recognises that the change from one note to the next is playing 
  a familiar musical scale, so it expects ``musical'' notes. It therefore tries 
  to fit the non-harmonic frequencies into some kind of best approximation to a 
  harmonic series. It is perhaps using some kind of average spacing between 
  adjacent frequencies as the basis for hearing a ``pitch'', and because in 
  this case all the spacings are wider than the fundamental, you detect a 
  ``pitch'' which is higher. The details are, no doubt, complicated: almost 
  certainly the ``best approximation'' will depend on the relative loudness of 
  the various overtones. But it also depends on which aspect of the sound you 
  choose to concentrate on. It would not be surprising to discover that it is 
  different for different listeners: age, natural hearing acuity and training 
  probably all make a difference. 

  Now listen to a similar comparison pair of sound files for a different 
  theoretical structure: first with the actual set of frequencies, and then 
  with each frequency rounded to the nearest whole-number multiple of the 
  fundamental. These are computed from our second idealised system, a flat 
  plate, to be described shortly. The difference in sound is far bigger than 
  for the previous example. Sound 3 is very ``unmusical''. In fact, it is 
  somewhat reminiscent of the sound of the child's saucepan lids --- although 
  you can still hear that a scale is being played, even though the individual 
  notes do not give any clear sensation of a definite pitch. Sound 4 is 
  startlingly different. It certainly has notes with clear pitches, but it 
  doesn't really sound like a percussion instrument of any kind, it sounds more 
  like a plucked string. 

  \aud{auds/aud-1fc4e7b7-plot.png}{\caption{Sound 3. Synthesised musical scale 
  as for Sound 1, but using the set of natural frequencies of a rectangular 
  plate.}} 

  \aud{auds/aud-68f9dca9-plot.png}{\caption{Sound 4. The result of adjusting 
  all the natural frequencies from Sound 3 to the nearest whole-number multiple 
  of the fundamental frequency.}} 

  The vibration of a thin, flat plate is essentially a two-dimensional version 
  of the bending beam. Bending or twisting of the plate generates appropriate 
  restoring moments. The theory is a little more complicated than for the beam: 
  it is outlined in the next link. There are very few special cases of plate 
  vibration that can be solved easily by hand. By far the simplest of these is 
  the one that has been used to generate Sounds 3 and 4. A rectangular plate is 
  assumed to have hinged or simply-supported boundaries all the way round. The 
  mode shapes are then rather simple: two dimensional versions of the shapes we 
  saw in section 3.1 for the vibrating string. The mode shape is sinusoidal in 
  both directions: it can have 1,2,3,... half-wavelengths of deformation 
  parallel to one edge of the plate, and, independently, 1,2,3... 
  half-wavelengths parallel to the other edge. As you might expect, the lowest 
  mode has a single half-wavelength in both directions, and higher frequencies 
  involve progressively shorter wavelengths. 

  \fig{figs/fig-ab5beecb.png}{\caption{Figure 3. Frequencies of the first 50 
  modes of a simply-supported rectangular plate, expressed as ratios to the 
  fundamental frequency (red). Blue stars show the effect of rounding each 
  ratio to the nearest whole number, in the same way as shown in Fig. 2. The 
  black dashed line shows the corresponding result for an ideal stretched 
  string.}} 

  Based on this special case, a plot similar to Fig.\ 2 can be generated: it is 
  shown in Fig.\ 3. The plate has a lot more modes than the beam had, and the 
  plot shows a trend that follows a rather straight line. This straight line 
  means that the modal density, the average number of modes within a given 
  frequency band, is approximately constant. This is a general result for plate 
  vibration, not dependent on this special case of a simply-supported 
  rectangular plate. An outline of the proof of that claim is given in the next 
  link. It is the generality of this pattern of natural frequencies which 
  allows us to use the simple rectangular case to illustrate the sound of a 
  tapped plate. Within reason, any other shape of plate, with any other 
  boundary conditions, would show behaviour that was statistically similar to 
  this special case, and would tend to sound somewhat similar. 

  As in Fig.\ 2, the blue stars show the frequency shifts necessary to make 
  each overtone frequency an exact harmonic of the fundamental frequency. The 
  shifts look small on this plot, but we have already heard that the effect on 
  the sound is profound. In this case, the denser set of natural frequencies 
  means that most of the possible harmonics are represented in the altered 
  sound. Perhaps this is why it sounds rather like a plucked string, which also 
  has a sound containing most of the harmonics (or at least, almost harmonics, 
  as we will see in section 5.4.3.) 

  Before leaving beams and plates, we can use these two idealised systems to 
  illustrate two important effects: how the sound changes if the assumed modal 
  damping is changed, or if the hardness of the tapping hammer is changed. 
  Sound 5 is the same as Sound 1 (for the beam), except that the modal 
  Q-factors are now all set to 100 rather than 500. Sound 6, on the other hand, 
  keeps the original Q-factors but assumes a softer hammer, with a contact time 
  10 times longer than before. Sounds 7 and 8 give the same pair of 
  comparisons, starting from the plate case from Sound 3 and then changing 
  either the Q-factors or the hammer hardness. 

  \aud{auds/aud-366c3afd-plot.png}{\caption{Sound 5. The ideal beam as in Sound 
  1, but with all Q-factors reduced to 100 rather than 500.}} 

  \aud{auds/aud-d4c6d57e-plot.png}{\caption{Sound 6. The ideal beam as in Sound 
  1, but with the impact duration of the hammer tap increased by a factor 10 to 
  1 ms.}} 

  \aud{auds/aud-cf873b67-plot.png}{\caption{Sound 7. The rectangular plate as 
  in Sound 3, but with all Q-factors reduced to 100 rather than 500.}} 

  \aud{auds/aud-f0d7305d-plot.png}{\caption{Sound 8. The rectangular plate as 
  in Sound 3, but with the impact duration of the hammer tap increased by a 
  factor 10 to 1 ms.}} 

  For both cases with the softer hammer the sound becomes more mellow, as you 
  would expect. The softer impact concentrates the energy in a smaller number 
  of modes at low frequency. The change of Q-factor is more difficult to 
  characterise. You may find that it gives an impression of an object made of a 
  different material. A Q-factor of 500, as in the original sounds, is in the 
  range expected for metallic objects. A Q-factor of 100 is towards the high 
  end of the range for wooden objects. In the case of the beam (Sound 5), we 
  are used to hearing the sound of wooden xylophone or marimba bars, and 
  perhaps that is what you are reminded of. 

  For the plate, though, the sound is less clearly ``wooden''. That is, at 
  least in part, because the sound we are accustomed to hearing from a real 
  wooden plate is influenced by added damping associated with holding a 
  structure up while you tap it. In the xylophone or marimba, the bars are 
  supported rather carefully near the nodal points of the first two modes, so 
  that rather little energy dissipation occurs at the supports. For the plate, 
  the sound we usually hear involves many more modes and it is not possible to 
  support a plate in a way that falls near a node of all these modes. The 
  result is extra dissipation from the supports, so that the actual Q-factors 
  of a typical wooden plate will be significantly lower than the value 100 
  assumed here. 

  The final type of structure to be mentioned here involves curved plates, 
  usually called shells. Such shells are very common: in church bells or the 
  body of a violin, for example, but also in body panels for cars or 
  aeroplanes. This time, the detailed theory governing the vibration is too 
  complicated to go into here, but there are a few qualitative things about 
  shell vibration that are well worth outlining. 

  It is easy to get an idea of why shell theory is intrinsically more 
  complicated than beam or plate theory. We can start by thinking about the 
  beam. We have talked about bending vibration, but there is another way that a 
  beam can vibrate. If an axial force is applied, the beam can stretch and 
  shrink, and the corresponding dynamic behaviour would be axial or 
  longitudinal vibration. The same is true of a flat plate: as well as 
  out-of-plane bending and twisting motion, it could also undergo in-plane 
  stretching vibration. 

  But for a straight beam or a flat plate, these two types of vibration are 
  completely independent. The reason is a rather deep one, to do with symmetry. 
  The beam and plate both have an invisible plane of symmetry running through 
  the middle: it is often called the ``middle surface'' for the plate case. Now 
  think about this plane as a mirror, and imagine how the two types of motion 
  would appear in that mirror. Stretching motion would be completely the same 
  as its reflection: the motion is symmetrical in the plane. But bending motion 
  would reverse in the mirror, and go in the opposite direction: bending motion 
  is antisymmetric in the plane. So in both cases we have a symmetric 
  structure, capable of both symmetric and antisymmetric motion. It is a 
  fundamental principle of physics that these must be independent. 

  Now think of a curved beam, or a curved thin shell. The curvature means that 
  the structure no longer has the plane of symmetry. A physicist would say that 
  ``the symmetry is broken by the curvature''. The argument that stretching and 
  bending motion are independent goes out of the window. The general theory of 
  vibration of such shells involves coupled bending and stretching motion, and 
  that inevitably makes everything much more complicated. 

  A simple example may help to make this seem more clear (with acknowledgement 
  to Lord Rayleigh, who used this neat example when making a similar point). 
  Think about corrugated roofing sheet. It is made from a sheet of metal or 
  plastic with uniform thickness, but because of the pattern of corrugations it 
  bends much more easily across the corrugations than along them. You can 
  easily visualise what happens when you bend the sheet in the stiff direction: 
  it behaves like a bending plate, but one that is much thicker than the 
  original sheet. The effective thickness is governed by the corrugation depth. 
  What is happening locally during this bending is that material on the peaks 
  of the corrugations is being stretched, while material in the dips is being 
  compressed (or vice versa when you bend the other way). So the total 
  deformation of the curved sheet involves a mixture of bending and stretching. 

  But there is an interesting special case: the corrugated sheet may be quite 
  floppy when you bend it parallel to the corrugations. Why is this different? 
  The reason is that corrugated sheet has a very special pattern of curvature: 
  it is called a developable surface, meaning that you can deform flat sheet 
  into the corrugated pattern without any stretching. For example, you could 
  easily make a corrugated pattern like that from a sheet of paper, without 
  tearing it. If you now bend the corrugated sheet parallel to the 
  corrugations, the extra deformation is still developable. Such motion is 
  called inextensional. There is no additional stiffening associated with local 
  stretching of the material, every element of material experiences only 
  bending motion. 

  The reason this makes such a big difference to the floppiness of the sheet is 
  that stretching deformation of a thin shell is intrinsically much stiffer 
  then bending deformation. Think of kitchen foil: you can bend it with ease in 
  your fingers, but it is very hard to stretch it (provided your piece is flat, 
  and not wrinkled). The thinner the sheet, the bigger is this disparity 
  between stretching and bending stiffness. The main reason is that the 
  stretching stiffness is proportional to the thickness $h$, but the bending 
  stiffness, as we have seen previously, is proportional to $h^3$. As you make 
  $h$ smaller, the bending stiffness decreases at a far higher rate than the 
  stretching stiffness. 

  For certain vibration problems, we can make use of this idea to understand 
  the low-frequency vibration behaviour. If a shell is geometrically capable of 
  inextensional deformation, then you would expect the lower vibration modes to 
  take advantage of that kind of deformation. The reason is to be found in 
  something called Rayleigh's principle, explained in the next link. Among 
  other things, Rayleigh's principle says that the lowest-frequency vibration 
  mode shape is the one which minimises the potential energy of deformation, 
  for a given level of kinetic energy in the motion. Well, the total potential 
  energy in a shell is simply the sum of a bending term, proportional to $h^3$, 
  and a stretching term, proportional to $h$. When $h$ is small, $h^3$ will be 
  much smaller than $h$. If inextensional motion is possible, it is therefore 
  bound to be associated with particularly low potential energy. 

  A familiar example is a wine glass, of the roughly hemispherical kind. A 
  hemispherical shell with a free top edge is indeed capable of inextensional 
  motions, involving behaviour with angle $\theta$ round the circumference like 
  $\cos 2 \theta$, $\cos 3 \theta$, etc. (The Greek letter $\theta$, pronounced 
  ``theta'', is often used to denote an angle like this.) The note you hear if 
  you tap a wine glass with a pencil corresponds to the case with $\cos 2 
  \theta$: the circular rim becomes slightly oval as shown (with exaggerated 
  scale) in Fig.\ 4. Of course, it is also possible for the glass to vibrate 
  with angular variation $\sin 2 \theta$. As we saw in Section 2.2.4 when 
  looking at the behaviour of a drum, these two can be combined to produce the 
  same sinusoidal pattern rotated to any orientation, provided the glass is 
  perfectly circular. (You take advantage of this when you make a wine glass 
  sing by running a wet finger round the rim. Your moving finger drags a nodal 
  point around with it!) 

\moobeginvid\begin{tabular}{ccc} \vidframe{ 0.30 }{ vids/vid-8c5356ca-00.png }&\vidframe{ 0.30 }{ vids/vid-8c5356ca-01.png }&\vidframe{ 0.30 }{ vids/vid-8c5356ca-02.png } \end{tabular}\caption{Figure 4.  The pair of modes, with identical natural frequencies, giving rise to the note you hear when you tap a wineglass.}\mooendvideo

  As an aside, something similar happens with a coffee mug, but here the $\cos 
  2 \theta$ and $\sin 2 \theta$ modes occur at slightly different frequencies, 
  because the circular symmetry is broken by the handle of the mug. Find a mug 
  that rings nicely when tapped, then go round it tapping with a pencil every 
  $45^\circ$ starting from the handle. You should hear two different notes, 
  alternating as you go round. If you tap at other positions, you will excite 
  both modes simultaneously and it may be harder to hear clearly what is 
  happening. You should be able to see what is happening from the animations in 
  Fig.\ 4. 

