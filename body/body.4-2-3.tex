  We are interested in acoustic waves in a horn with a varying cross-sectional 
  area $S(x)$. We can give a simple derivation of the governing equation, based 
  on assuming that the acoustic disturbance still takes the form of an 
  approximately plane wave. This approximation is good when the section varies 
  slowly, but for accurate answers it needs more care as the bell is approached 
  and the horn flares more abruptly. More detail on this, and many other 
  aspects of the acoustics of horns, can be found in the books by Fletcher and 
  Rossing [1], and Chaigne and Kergomard [2]. 

  We follow the same steps as in section 4.1.1. Consider a small element of the 
  air inside a pipe with cross-sectional area $S(x)$, lying between positions 
  $x$ and $x+\delta x$, as sketched in Fig.\ 1. As a result of the sound wave, 
  the particles that were initially at the ends of this element are displaced 
  by a small distance, to $x+\xi(x)$ and $x + \delta x + \xi(x + \delta x)$ 
  respectively. 

  \fig{figs/fig-fc032ef7.png}{\caption{Figure 1. A small element of the 
  acoustic wavefield in a tube with varying cross-section, showing particle 
  displacements.}} 

  We can apply conservation of mass to this element of air: the product of 
  density and volume must remain constant, so 

  \begin{equation*}\rho_0 S \delta x = (\rho_0 + \rho') \left[ S(x) \delta x -- 
  S(x) \xi(x) +S(x+ \delta x) \xi(x+ \delta x) \right] \end{equation*} 

  \begin{equation*}\approx (\rho_0 + \rho') \left[ S \delta x + 
  \frac{\partial}{\partial x} \left(S \xi \right) \delta x \right] . 
  \tag{1}\end{equation*} 

  Cancelling $\delta x$ and ignoring products of small quantities, this reduces 
  to 

  \begin{equation*}0 \approx S \rho' + \rho_0 \frac{\partial}{\partial x} 
  \left(S \xi \right) . \tag{2}\end{equation*} 

  As before, we can assume that $p' = c^2 \rho'$ in terms of the pressure 
  change $p'$, and so 

  \begin{equation*}p' = -\dfrac{c^2 \rho_0}{S}\frac{\partial}{\partial x} 
  \left(S \xi \right) . \tag{3}\end{equation*} 

  We can obtain a second relation between $p'$ and $\xi$ by applying Newton's 
  law to our small volume. The pressure acts on the two end faces, giving a net 
  force which must be balanced by mass times acceleration. However, because of 
  the varying cross-section there is also a contribution to the force from the 
  pressure acting on the sloping walls of the element. This can be conveniently 
  expressed as the mean pressure multiplied by the projected area of the 
  element side-walls in the axial direction. The resulting equation is 

  \begin{equation*}S(x) [p_0+p'(x)] -S(x+ \delta x) [p_0+p'(x+ \delta x)] + 
  \end{equation*} 

  \begin{equation*}(p_0 +p') [S(x+ \delta x) -S(x)]= \rho_0 S \delta x 
  \frac{\partial^2 \xi}{\partial t^2} \tag{4}\end{equation*} 

  \noindent{}so that 

  \begin{equation*}- S\frac{\partial p'}{\partial x} \approx \rho_0 S 
  \frac{\partial^2 \xi}{\partial t^2} .\tag{5}\end{equation*} 

  Differentiating this with respect to $x$ and then using eq. (3), we obtain 

  \begin{equation*}\dfrac{\partial}{\partial x} \left( S\frac{\partial 
  p'}{\partial x} \right) = -\rho_0 \frac{\partial^2}{\partial t^2} \left[ 
  \frac{\partial }{\partial x} (S \xi) \right]= \dfrac{S}{c^2} \frac{\partial^2 
  p'}{\partial t^2} \tag{6}\end{equation*} 

  \noindent{}which is the Webster horn equation. 

  We can learn something useful by making a change of variables in this 
  equation: set $p'=\psi(x) S^{-1/2}$, and also also write the area in terms of 
  a radius via $S(x)=\pi a^2(x)$. Finally, we are interested in modes and 
  therefore we can assume harmonic time dependence $e^{i \omega t}$. Equation 
  (6) can then be rearranged into a form first suggested by Benade and Jansson 
  [3]: 

  \begin{equation*}\dfrac{d^2 \psi}{dx^2} + \left( \frac{\omega^2}{c^2} -- 
  \frac{1}{a} \dfrac{d^2a}{dx^2} \right) \psi = 0. \tag{7}\end{equation*} 

  This is the familiar simple harmonic equation. If the term in brackets is 
  positive, it has sinusoidal solutions which suggest local behaviour 
  corresponding to travelling waves. However, if that bracketed term is 
  negative, the solutions give exponential growth or decay, suggesting that the 
  local behaviour might be evanescent in character. For a given frequency 
  $\omega$, if this bracketed term reaches zero at some position along the 
  tube, that gives a limit beyond which waves should not propagate. 

  In order to produce the numerical example shown in the main text of section 
  4.2, a broadly plausible bore profile $a(x)$ was chosen (based on suggestions 
  from the literature [1]), and then a very simple finite-difference approach 
  was used. The pressure was discretised with a chosen step length, and the 
  derivatives on the left-hand side of eq. (6) were expressed via central 
  differences of these discrete values. Blocked and free boundary conditions 
  were respectively imposed at the two ends of the tube. The resulting discrete 
  equations were used to populate a stiffness matrix, whose eigenvalues and 
  eigenvectors were then computed. Convergence tests were used to find an 
  appropriate step length, giving a good compromise between accuracy and 
  conditioning of the matrix. 

  The specifics of the chosen bore profile are as follows. A straight tube of 
  length 0.5 m and radius 1 cm was joined to a second 0.5 m length of tube in 
  which the radius varied according to a power law with distance from the bell 
  (a so-called Bessel horn [1]). The chosen power was -0.85, so that the 
  behaviour was a little less extreme than a hyperbola. The function was 
  truncated 5 mm before the singularity to give a finite bell of 
  plausible-looking shape. With these parameters, the fundamental frequency was 
  around 122 Hz. 

  The function $\frac{1}{a} \frac{d^2a}{dx^2}$ appearing in eq. (7) was, of 
  course, zero throughout the straight section of pipe. Then, after a blip at 
  the junction with the flaring pipe, it was monotonically increasing all the 
  way to the bell. This function was used to compute the critical position 
  along the bore based on the natural frequency of each mode, corresponding to 
  the transition of behaviour of eq. (7). These are the positions plotted as 
  vertical lines in Fig.\ 15 of section 4.2. The systematic variation of these 
  lines with mode number is a direct consequence of the monotonic function. 

  \sectionreferences{}[1] Neville H Fletcher and Thomas D Rossing; ``The 
  physics of musical instruments'', Springer-Verlag (Second edition 1998) 

  [2] Antoine Chaigne and Jean Kergomard; ``Acoustics of musical instruments'', 
  Springer/ASA press (2013) 

  [3] Arthur H. Benade and Erik V. Jansson; On plane and spherical waves in 
  horns with nonuniform flare, Acustica \textbf{31}, 80--98 (1974). 