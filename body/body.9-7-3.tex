  There are several possible ways to extend the bowed-string simulation model 
  to allow for a finite-width bow. The model used here is based on one 
  developed by Roland Pitteroff [1]. We assume some number of separate, 
  discrete “bow-hairs”, spread over a chosen total width as sketched in Fig.\ 
  1. These “hairs” are not assumed to be rigid: instead, measured behaviour of 
  individual strands of horsehair is used to deduce an approximate model 
  consisting of a parallel spring-dashpot combination. This model is expressed 
  in terms of values per unit width across the bow by assuming the total number 
  of “active” hairs in potential contact with the string to be 50, representing 
  a single layer of hairs out of the total of 200 or so for a typical violin 
  bow. The properties of each “hair” in the discrete model are then deduced by 
  subdividing the total bow width into the chosen number. The details of all 
  this can be found in reference [2]. 

  \fig{figs/fig-b48d0faa.png}{Figure 1. Schematic sketch of a set of discrete 
  ``bow-hairs'', regularly spaced across the chosen section of the string.} 

  The model then uses a finite-difference approach to the section of string 
  lying under the bow, combined with the digital waveguide method to represent 
  the two sections of string outside the width of the bow. At each time step, 
  outgoing waves from the two edges of the bow are converted into incoming 
  waves by convolution with reflection functions as before. Both transverse and 
  torsional waves are included. 

  The short section of string under the bow is treated in the simplest possible 
  way: transverse and torsional waves are each assumed to obey the simple wave 
  equation, with the appropriate wave speeds. Local effects of damping and 
  bending stiffness within this short length are ignored: but for the string as 
  a whole they are allowed for via the reflection functions. Each wave equation 
  can be represented approximately by a central-difference form for the spatial 
  second derivative, involving the displacements of the “hairs” on either side 
  of the one being considered, and by a backward-difference form for the time 
  derivative. The spring-dashpot model for the “hairs” is also expressed in a 
  finite-difference form. Putting all this together, the new value of 
  displacement of string and bow-hair at each discrete “hair” can be 
  calculated, from a knowledge of the displacement of it and its neighbours at 
  previous time steps together with the friction force acting on that 
  particular “hair”. Details of all this can be found in reference [1]. 

  The calculation of the friction force at each “hair” involves nothing new: 
  each “hair” is treated exactly like the single-point “bow” we have studied in 
  earlier sections. There is only one minor twist. The particular way the 
  Pitteroff method was implemented assumes that the friction curve is a 
  hyperbolic function. This allows an analytic solution for the force and 
  velocity at the next time step, rather than relying on a numerical 
  approximation. In order to produce results that are reasonably comparable 
  with the earlier ones, it is therefore useful to make a hyperbolic fit to the 
  measured friction coefficient that we have seen before. The result is shown 
  in Fig.\ 2. The explicit functional form is 

  $$\mu(v) =\mu_d + \dfrac{(\mu_s -- \mu_d) v_0}{v + v_0}$$ 

  with $\mu_s = 1.2$, $\mu_d =0.35$ and $v_0=0.02$. So the maximum sticking 
  coefficient of friction and the asymptotic coefficient of sliding friction at 
  high speed are exactly the same as before, and the only difference lies in 
  the exact shape of the curve at intermediate speeds. (Note that, for 
  consistency with earlier plots, the curve is shown here as a function of 
  $-v$.) 

  \fig{figs/fig-9e480a55.png}{Figure 2. Measured steady-sliding friction 
  coefficient for violin rosin (stars), with the double-exponential curve fit 
  used in previous sections (red) together with an approximate hyperbolic fit 
  to the same data (blue curve).} 

  We can illustrate some typical results of this model. These simulations, and 
  the ones shown in section 9.7, use parameter values suitable for a violin G 
  string, rather than the cello D string that was used for most of the earlier 
  examples. The reason is a technical one, arising from the simulation method. 
  An open cello string has a length around 680~mm, and the bow width is about 
  13~mm. An open violin string has a length around 330~mm, and the bow width is 
  about 10~mm: a much bigger fraction of the string length than for the cello. 

  Now, the resolution for the spatial variation within the bow width obviously 
  depends on the number “bow-hairs” used. We don’t want this number to be too 
  small: the results here have 11 “hairs”. But having made that choice, there 
  is a lower limit on the chosen sampling frequency (or equivalently, an upper 
  limit on the time step length) in order to achieve numerical stability with 
  the finite-difference method. It turns out that with cello-like parameter 
  values, the required sampling rate becomes embarrassingly high, and the 
  process of simulation becomes slow, unwieldy and inaccurate. Everything is 
  much easier with violin-like parameter values. 

  Some individual simulation results were shown in section 9.7, but it is 
  useful to show some more systematic results here, to illustrate the effect of 
  options within the model. For a first set of results, we show some columns 
  from a simulated Schelleng diagram. Four particular values of bow position 
  $\beta$ (referred to the centre point of the finite bow) have been selected, 
  and for each $\beta$ value simulations have been run with 10 
  logarithmically-spaced values of bow force. For each simulation the model was 
  initialled with ideal Helmholtz motion, then it was run for 50 nominal 
  period-lengths to give the motion an opportunity to settle down into a 
  periodic state (or not, of course). Finally, we plot an extract of the bridge 
  force waveform for the last few period-lengths. 

  Figure 3 shows the results, for simulations using the friction-curve model 
  with the hyperbolic shape plotted in Fig.\ 2. The format of the plot is 
  similar to Fig.\ 5 of section 9.3, where we looked at a measured Schelleng 
  diagram (for a cello D string on the Galluzzo rig). Each column shows the 
  characteristic sawtooth waveform in the lower part, giving way to some kind 
  of irregular motion towards the top. This transition, a version of the 
  Schelleng maximum bow force, marks out a downward-slanting line across the 
  set of plots, much as we expect from the earlier discussion. As the maximum 
  bow force is approached, irregular “spikes” associated with differential 
  slipping across the finite width of the bow become more evident: we will see 
  more detail of this shortly. 

  \fig{figs/fig-529beb3e.png}{Figure 3. Examples of bridge force waveforms for 
  an open G string of a violin, simulated using a flat bow of width 10 mm, for 
  some representative columns of Schelleng's diagram. The four plots correspond 
  to columns 2, 8, 14 and 20 of the Schelleng diagrams shown in section 9.3: 
  the $\beta$ values at the centre of the bow, from left to right, are 0.0225, 
  0.0449, 0.0899 and 0.1800 respectively. Bow force values correspond to 
  alternate rows of the earlier Schelleng diagrams. Each simulation was 
  initialised with ideal Helmholtz motion. Simulations are based on the 
  friction-curve model, with the hyperbolic friction curve plotted in Fig. 2.} 

  More surprisingly, we do not see any clear example of a transition to double 
  slipping motion at low values of bow force. Instead, the sawtooth waveform 
  gets more and more rounded, and in the lower left of this figure they seem to 
  be decaying slowly away. This absence of a clear example of Schelleng’s 
  minimum bow force is probably not a consequence of the finite-width bow 
  model, though: similar behaviour has been seen in some simplified 
  single-point bowing models [3]. 

  Most likely, it is a result of other simplifications that have been made in 
  the development of this particular model. There are two possible candidates: 
  there is no representation of coupling to an instrument body, and the 
  intrinsic damping of transverse string vibration has been approximated by a 
  simplified model representing a constant Q-factor for all modes. Both these 
  omissions were for purposes of numerical efficiency: in particular the 
  digital filter method of representing string damping more accurately, 
  developed by Hossein Mansour [4], proved hard to adapt to the higher sampling 
  rate used here. 

  More interesting for the present purpose, we can produce corresponding 
  results using the alternative friction models developed in section 9.6. By 
  running the heat-flow calculation for each of the ``bow-hairs'' separately, 
  we can estimate the contact temperature across the width of the bow. We don't 
  really know the details of contact size between individual bow-hairs and the 
  string. For the purposes of these results, I have simply assumed the same 
  total area of contact as in the earlier single-point models, divided equally 
  between the discrete ``bow-hairs''. We can use the computed temperature 
  either with the original thermal model, or with the modified thermal model. 
  Results corresponding to Fig.\ 3 for these two cases are shown in Figs.\ 4 
  and 5, respectively. 

  \fig{figs/fig-b25f0a77.png}{Figure 4. Bridge-force waveforms from Schelleng's 
  diagram in the same format as Fig. 3, but simulated using the thermal 
  friction model from Section 9.6.} 

  \fig{figs/fig-64000b6e.png}{Figure 5. Bridge-force waveforms from Schelleng's 
  diagram in the same format as Figs. 3 and 4, simulated using the modified 
  thermal friction model from Section 9.6.} 

  The immediate impression from these plots is that all three friction models 
  give quite similar-looking results. However, if we look in a little more 
  detail we can see significant differences between the three models. We will 
  choose a particular case to illustrate: the third waveform from the top in 
  the second column of Figs.\ 3, 4 and 5. This corresponds to $\beta=0.0449$ 
  and bow force $1.23$~N. The following sequence of plots shows this bridge 
  force waveform from the three different friction models, accompanied by maps 
  to show what is going on under the bow. All cases have the stick-slip map, 
  and the two thermal models also have a temperature map. 

  \fig{figs/fig-f6894ea8.png}{Figure 6. Bridge force waveform for the chosen 
  case, simulated with the friction curve model.} 

  \fig{figs/fig-b369ff72.png}{Figure 7. The map of slipping and sticking within 
  the bow, corresponding to Fig. 6.} 

  \fig{figs/fig-2cebc965.png}{Figure 8. Bridge force waveform for the chosen 
  case, simulated with the original thermal friction model.} 

  \fig{figs/fig-a1a7e293.png}{Figure 9. The map of slipping and sticking within 
  the bow, corresponding to Fig. 8.} 

  \fig{figs/fig-19dc9edd.png}{Figure 10. The map of temperature distribution 
  within the bow, corresponding to Fig. 8.} 

  \fig{figs/fig-8485edc1.png}{Figure 11. Bridge force waveform for the chosen 
  case, simulated with the modified thermal friction model.} 

  \fig{figs/fig-596e1d0b.png}{Figure 12. The map of slipping and sticking 
  within the bow, corresponding to Fig. 11.} 

  \fig{figs/fig-e76d1998.png}{Figure 13. The map of temperature distribution 
  within the bow, corresponding to Fig. 11.} 

  Finally, we can show some transient results using the three finite-width 
  models. Each is compared to the corresponding point-bow model --- these 
  point-bow results are not quite the same as the ones showed in section 9.5, 
  because they are using the parameters for a violin G string rather than a 
  cello D string for reasons explained above. Figure 14 shows this set of 6 
  Guettler diagrams. All cases use the bow position $\beta = 0.0899$ as in 
  earlier figures. These are the same diagrams that were shown in section 9.7, 
  but now each Guettler diagram is annotated with a $3 \times 3$ grid of green 
  circles to mark the pixels corresponding to acceleration values 5, 11 and 17 
  (counting from the left-hand side), and force values 5, 10 and 15 (counting 
  from the bottom). The waveforms for these 9 cases are shown in Fig.\ 15, for 
  all 6 models. The 9 panels of that figure are laid out in the same 
  arrangement as the $3 \times 3$ grid of green circles. In each panel, results 
  for the 6 different models are identified by plot colour. 

  These waveforms show what lies behind the automatic classification in the 
  Guettler diagrams. The black curves, for the finite-width friction-curve 
  model, show that this model finds it very hard to settle into any kind of 
  periodic motion. Individual sticks and slips at the separate “bow hairs” are 
  all subject to the jumps arising from the frictional hysteresis mechanism, 
  and probably this injects too much “sensitive dependence” into the nonlinear 
  system, leading, often, to chaotic response. The corresponding point-bow 
  model (blue curves) only has one place for such jumps to occur, and the 
  behaviour is a little less wild. 

  All the thermal models show some evidence of periodic motion occurring by the 
  end of the time segment plotted here — not in every case, and not always 
  Helmholtz motion, but often enough to suggest more benign behaviour from the 
  perspective of a violinist. Comparing each waveform with the corresponding 
  pixel in the Guettler diagram gives reassuring evidence that the automatic 
  classification routine is behaving quite reliably. The only apparent anomaly 
  is associated with the point-bow thermal model (magenta curves). In the 
  middle row of Fig.\ 15, two of the waveforms look as if they lead to 
  Helmholtz motion, and yet the Guettler plot shows black pixels. This is a 
  manifestation of the “rounded bottoms” phenomenon highlighted in section 9.6: 
  the effect is not visible when plotted at this resolution, but the waveforms 
  are sufficiently different from a sawtooth form that they fail the “Helmholtz 
  test” used by the classification routine. 

  \sectionreferences{}[1] R. Pitteroff and J. Woodhouse, “Mechanics of the 
  contact area between a violin bow and a string. Part II: simulating the bowed 
  string”; Acta Acustica united with Acustica, \textbf{84}, 744—757 (1998). 

  [2] R. Pitteroff and J. Woodhouse, “Mechanics of the contact area between a 
  violin bow and a string. Part I: reflection and transmission behaviour”; Acta 
  Acustica united with Acustica, \textbf{84}, 543—562 (1998). 

  [3] J. Woodhouse; “On the playability of violins, Part II minimum bow force 
  and transients”; Acustica \textbf{78}, 137–153 (1993). 

  [4] Hossein Mansour, Jim Woodhouse and Gary P. Scavone, “Enhanced wave-based 
  modelling of musical strings, Part 1 Plucked strings”; Acta Acustica united 
  with Acustica, \textbf{102}, 1082–1093 (2016). 

  [5] R. Pitteroff and J. Woodhouse, “Mechanics of the contact area between a 
  violin bow and a string. Part III: parameter dependence”; Acta Acustica 
  united with Acustica, \textbf{84}, 929—946 (1998). 