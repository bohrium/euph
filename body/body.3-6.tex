

  The first instrument we looked at was a toy drum, back in section 2.2. Now it 
  is time to look more carefully at drums. The majority, like the ones in the 
  drum kit seen in Fig.\ 1, are not tuned to any definite pitch. They are 
  designed to make different sounds, of course, and drummers exploit these 
  differences for musical effect. But there are also tuned drums, and we will 
  look at two varieties using different principles to achieve tuning. 

  \fig{figs/fig-eab55c67.png}{\caption{Figure 1. A typical drum kit.}} 

  The family of mode shapes of an ideal drum was shown in section 2.2, and is 
  reproduced here in Fig.\ 2. The theory behind these mode shapes and natural 
  frequencies is presented in the next link. Each mode shape involves variation 
  with polar angle $\theta$ around the drum which takes the form $\cos n 
  \theta$ or $\sin n \theta$, where $n$ must take one of the values 0,1,2,3... 
  Each row of Fig.\ 2 corresponds to a different value of $n$, starting with 
  $n=0$ on the top row. For this value, $\sin n \theta =0$ but $\cos n \theta = 
  1$, so these modes are axisymmetric. 

  For all the higher values of $n$, $\cos n \theta$ and $\sin n \theta$ both 
  make sense, so each of these modes appears as a degenerate pair with the same 
  natural frequency: recall section 2.2.4 for an explanation of what that 
  means. But any real drum will have some departure from perfect circular 
  symmetry: the mass distribution and the tension will never be perfectly 
  uniform. The result is that if a measurement is made, these pairs of modes 
  will normally be slightly separated in frequency and show up as double peaks 
  in a frequency response plot. 

  The variation of each mode shape in the radial direction is given by a Bessel 
  function, a different one for each value of $n$. The relevant Bessel 
  functions are usually written $J_n$; plots of the first few can be found in 
  the link above. 

  \fig{figs/fig-26fec7a5.png}{\caption{Figure 2: The vibration modes of an 
  ideal circular drum, with no allowance for loading by air in contact with the 
  drum skin. All modes in a particular column have the same number of nodal 
  circles, while all modes in a given row have the same number of nodal 
  diameters. The first few are shown here: the pattern continues indefinitely 
  to the right and downwards. Above each mode shape is listed its natural 
  frequency, relative to the fundamental mode.}} 

  The frequency ratios in Fig.\ 2 do not reveal any obvious harmonic relations 
  between the natural frequencies. That remark would still be true if we took 
  account of an important factor that hasn't been mentioned yet. Drum membranes 
  were traditionally made of natural skin of one kind or another, but most 
  modern drums use a synthetic material familiar from trade names such as 
  Mylar. A Mylar drum head will only weigh a few grams, sufficiently light that 
  we cannot ignore the influence of the air in contact with the membrane on 
  both sides. A natural skin head would be somewhat heavier, but still very 
  light compared to the other percussion instruments we have looked at, or the 
  wooden soundboard of a typical stringed instrument. 

  Air motion is important for two reasons. The purpose of the drum is to make 
  audible noise, so of course the air carries sound waves away from the 
  vibrating drum. But there is also kinetic energy associated with local 
  movement of air close to the membrane. The inertia associated with this 
  kinetic energy will increase the effective mass per unit area of the 
  membrane, and thus reduce all the natural frequencies compared to the simple 
  theory presented in the link above. The effect depends on the wavelength and 
  other details of the mode shape, so it is different for each mode and 
  therefore changes the frequency ratios. 

  This added-mass effect from loading by the surrounding air does not, of 
  itself, create harmonic relations between the natural frequencies. However, 
  the first of our tuned drums, the orchestral kettledrum, makes ingenious use 
  of air motion in order to do exactly that. The shape of the ``kettle'' 
  beneath the drum membrane (see Fig.\ 3) has been developed over the centuries 
  to modify and control the added mass associated with air motion, by 
  interaction with internal acoustic resonances of the cavity. The result is 
  that a few of the natural frequencies fall close to harmonic relationships. A 
  skillful percussionist will fine-tune these relationships by careful 
  adjustment of the tension distribution in the membrane. 

  \fig{figs/fig-494d6e5a.png}{\caption{Figure 3. A typical kettledrum. This 
  type has a foot pedal that allows the membrane tension, and thus the pitch of 
  the drum, to be changed by the performer.}} 

  We can illustrate the results with some measurements and synthesised sound 
  examples. Christian et al. [1] measured natural frequencies of a kettledrum 
  membrane both with and without the presence of the kettle. The case without 
  kettle includes an effect of air loading, so the frequencies are already 
  different from the ideal case analysed in the link above. Adding the kettle 
  causes further change of these frequencies. 

  The most important frequency ratios for the final case, with the kettle, are 
  summarised in the red numbers in Fig.\ 4. These are expressed as ratios 
  relative to the mode with a single nodal diameter, but for reasons that will 
  rapidly become apparent this mode is given the nominal frequency 2. It can be 
  seen in the figure that three other modes fall into harmonic relations to 
  this one, with relative frequencies close to 3 and (for two different modes) 
  4. So we have harmonic relations, but this is another case of a ``missing 
  fundamental'': there is no mode with frequency 1 in these terms. Furthermore, 
  the axisymmetric modes in the top row are not tuned, so they will tend to 
  detract from the sense of definite pitch. For that reason, a percussionist 
  does not hit a kettledrum near the centre, which would maximise the sound 
  from these modes while giving little sound from the tuned modes (which all 
  have at least one nodal line through the centre). 

  \fig{figs/fig-f6db5bef.png}{\caption{Figure 4. Frequency ratios among the 
  vibration modes of two tuned drums. The modes are arranged in the same order 
  as in Fig. 1, but are shown here via nodal line patterns for greater clarity. 
  Red figures: kettledrum, taken from Christian et al. [1] and reproduced in 
  Table 18.2 of Fletcher and Rossing [2]; blue figures: mridanga (related to 
  the tabla), from Rossing and Sykes [3].}} 

  To illustrate the consequences of these various different sets of natural 
  frequencies, three sound examples appear below. Sound 1 uses the frequency 
  ratios of the ideal membrane without any influence of air, as in Fig.\ 2; 
  Sound 2 uses the measured frequencies from Christian et al. [1] without 
  kettle, and Sound 3 uses their measured frequencies with the kettle. Apart 
  from the different frequency ratios, all three sounds use identical 
  parameters. The same set of modes is included in each case, and the reference 
  mode with one nodal diameter is assigned the frequency 172 Hz to match the 
  measurements; all modes are given a Q-factor of 100; and all are given the 
  same amplitude except that the lowest mode has amplitude reduced to 30\% of 
  this value as a crude representation of the fact that a performer would try 
  to de-emphasise that mode. As with all the sound examples in this chapter, 
  the intention is not to produce the most realistic synthesis of a drum sound. 
  Rather, the intention is to allow a clean comparison of the effect of 
  changing the frequency ratios with the minimum of complicating factors. 

\audio{}

\audio{}

\audio{}

  To my ears, at least, the sounds are all reasonably drum-like. The 
  differences are fairly subtle, but I could hum the pitch of Sound 3 with 
  confidence, whereas the other two are more ambiguous. 

  So the kettledrum has illustrated one way to apply some acoustic engineering 
  to the design of a drum in order to produce a sound of definite pitch. But it 
  is not the only possibility: the champion makers of tuned drums come from 
  India. Several types of Indian drum use the same trick for tuning. The most 
  well known of these is the tabla, shown in Fig.\ 5. Both drums in this 
  picture have a black patch on the membrane. This patch is the result of the 
  drum-maker painstakingly applying layer after layer of a paste made from 
  over-cooked rice blended with a heavy black powder containing manganese and 
  iron oxide. By careful distribution of this progressively added mass, they 
  manage to tune at least 8 of the natural frequencies into harmonic relations: 
  twice as many as in the kettledrum. 

  \fig{figs/fig-d2fd2943.png}{\caption{Figure 5. A tabla. Image: Lestat, CC 
  BY-SA 3.0, via Wikimedia Commons}} 

  The smaller of the pair of drums in Fig.\ 5 exhibits the most impressive 
  level of harmonic tuning. The extraordinary number of tuned modes in such a 
  drum was first discovered by C. V. Raman [4], who also became India's first 
  Nobel prizewinner for his work on light scattering. We will make use of some 
  more recent results: Rossing and Sykes [3] followed the process of tuning a 
  Southern Indian drum called the mridanga, which exhibits the same pattern of 
  tuning that Raman found in the tabla. They recorded the frequencies of major 
  peaks in the sound of this drum throughout the process, which involved over 
  100 separate layers of paste being applied! 

  The key conclusion of their study is shown in the blue numbers in Fig.\ 4. In 
  addition to the four modes that were tuned in the kettledrum, the three 
  axisymmetric modes in the top row were tuned. At least one other mode outside 
  the range shown in Fig.\ 4 was also tuned: the mode with four nodal 
  diameters, which would be the next one in the first column, was tuned to 
  frequency 5 in the terms of this plot. Within the frequency range of these 
  tuned modes, there are no untuned modes. Furthermore, the combination of 
  frequencies no longer has a ``missing fundamental''. 

  Using these frequencies in a synthesis exactly like those for the previous 
  sound examples (except that the amplitude of the first mode is now the same 
  as all the others) gives the result in Sound 4. With so many tuned modes, and 
  with no untuned modes included, the sound is not immediately recognisable as 
  a drum of any kind. Perhaps it does not sound entirely like a tabla either 
  --- but that is partly because this pattern of repeated slow notes left to 
  ring on is not at all in the typical musical style of tabla playing. More 
  normal would be a blizzard of notes in complicated rhythmical patterns, 
  played with the fingertips and making use of the variety of sound possible by 
  striking in different positions on the drum head. 

\audio{}



  \sectionreferences{}[1] R. S. Christian, R. E. Davis, A. Tubis, C. A. 
  Anderson, R. I. Mills and T. D Rossing; ``Effects of air loading on timpani 
  membrane vibrations'', Journal of the Acoustical Society of America 
  \textbf{76}, 1336--1345 (1984) 

  [2] Neville H Fletcher and Thomas D Rossing; ``The physics of musical 
  instruments'', Springer-Verlag (Second edition 1998) 

  [3] T. D. Rossing and W. A. Sykes; ``Acoustics of Indian drums'', Percussive 
  Notes \textbf{19} (3) 18--31 (1982) 

  [4] C. V. Raman; ``The Indian musical drum'', Proceedings of the Indian 
  Academy of Sciences \textbf{A1}, 179--188 (1934) 