  There are many accounts of scientific discovery, and a lot of them go roughly 
  like this: back in the ``dark ages'' of the subject all was mystery, 
  misinformation and magic, then science got a grip on things and the magic was 
  dispelled. This progression has happened with gravity and the motion of the 
  planets; it has happened to a large extent with medicine; it is happening at 
  the moment with weather forecasting. Now, many people, scientists included, 
  don't want the magic to go away from music. Perhaps music demystified will 
  lose its emotional impact, or perhaps robots will replace craftsmen and make 
  world-beating violins. Is a book about the science of music and musical 
  instruments going to shine too bright a light into unwelcome corners? 

  Curiously, the answer is no. Musical acoustics has, to a large extent, 
  resisted the usual trend. It has been studied since at least the time of 
  Pythagoras, and yet it is still remarkably hard to give a straight answer to 
  a violin maker who asks what they should adjust in their woodwork to achieve 
  a particular desired sound quality, such as matching the sound and playing 
  qualities of a famous (and expensive) instrument by Stradivari or Guarneri 
  del Gesu. Certainly, some things are much better understood than they were. 
  There has also been no shortage of innovation: there are many thousands of 
  patents relating to musical instruments, including some entirely new types of 
  music-making devices. And yet, alongside that, the violin is virtually 
  unchanged since the late seventeenth century, and when you listen to the 
  language used by musicians of all persuasions when they are trying to choose 
  a new instrument you still hear plenty of mystery and magic (and probably 
  some misinformation). For a scientist, this is tantalising and occasionally 
  frustrating. 

  What is going on? This ``book'' will mainly be about physical descriptions of 
  vibration and sound from musical instruments, but the agenda and the special 
  flavour of the subject are set by something else: the remarkable subtlety and 
  elusive qualities of human ability, both in the perception of sound and in 
  motor actions to control breath, bows or fingers. Some exploration in these 
  areas is needed before we get down to the business of acoustics. 

  So what is music? It is widespread, perhaps universal in all human cultures. 
  It is one of those things that is hard to define, but you probably know it 
  when you see it, or more likely, hear it. Music is important to a lot of 
  people, and some people are very good at it, as performers or experienced 
  listeners. These people care a great deal about small details: of the music, 
  its performance, and the instruments on which it is played (including the 
  human voice). ``Small details'' will be a recurring theme in this book. A 
  musical instrument is a contrivance which allows a performer to use gestures 
  they are physically capable of performing to make sounds that they like. Of 
  course, ``like'' is a loaded word, the details being highly dependent on 
  cultural background and experience. Nevertheless, all skilled musicians are 
  probing the outer limits of human abilities. There are a lot of violinists 
  and pianists out there, and they are all striving to do better than the 
  others and get the good reviews and the recording contracts. A virtuoso is 
  someone who can do things other people can't do. Maybe they go only just 
  beyond the limits of other players, but that boundary is where all the 
  attention is focussed. 

  If the choice of instrument can make a difference, you can be sure that money 
  will be spent to achieve this competitive edge. It is very much like sport: 
  superstars are just that bit better at hitting a ball, or whatever, but that 
  bit is crucial. Manufacturers of sports equipment make huge efforts to tweak 
  design details to give an extra edge, for which serious money will be paid. 
  And they may resort to psychological ploys via careful advertising to 
  convince people to pay out for the latest product. In sport as in music, 
  ``psychological'' effects are not necessarily ``fictitious''. A player, 
  whether of tennis or the violin, may well perform better if they believe they 
  have superior equipment. Conversely, they are only too likely to make errors 
  if they believe that their competitors have an edge. 

  The science of acoustics is well developed and wide-ranging. Some aspects 
  involve medical or military applications, but a lot of it (at least, a lot of 
  the well-funded aspects) is devoted to reducing noise, whether from vehicles, 
  wind turbines or neighbours. Musical acoustics is different. It is very easy 
  to solve a violinist's ``noise problem'': put candle wax on their bow-hair 
  instead of rosin. (This really works, but don't try it incautiously --- it is 
  virtually impossible to get rid of.) But of course that is not the point. The 
  things that matter to violinists and their listeners all revolve around 
  making and controlling sound, down to the level of those fine details. This 
  makes the associated science particularly challenging. It does not satisfy a 
  musician or an instrument maker to be told roughly how things work, they will 
  always press for more and more detailed information. 

  Everything hinges on the way human perception is believed to work, not only 
  for sounds but for vision, touch and so on. We are each equipped with a huge 
  collection of ``feature detectors'', bits of neural circuitry that are tuned 
  in to spotting a particular thing when it occurs among the input data. These 
  different detectors are all running at once, each looking out for their 
  particular thing, and when they hear/see/feel it, they send off a message to 
  ``you''. Some of these feature detectors are more or less the same for almost 
  everyone, others may be rather specific to one person's particular background 
  and training. 

  Some examples may make this clear. In visual processing, we have 
  ``low-level'' detectors which allow us to recognise different colours, simple 
  shapes, and simple kinds of movement. But we also have high-level detectors 
  that are finely tuned to recognise, for example, a human face. That is why we 
  so easily see ``faces'' in flames, rock formations or clouds. Look at the 
  photograph below, near a waterfall in Iceland. Do you see a face in profile 
  in the shape of the foreground grass? The top pair of birds are in the eye 
  socket. 

  \fig{figs/fig-470ab9c9.png}{\caption{Figure 1. Do you see a face?}} 

  Your ``face detector'' is very specific: it only works really well if the 
  face is the right way up, and of a reasonably familiar type. It is hard to 
  recognise a photograph of a friend if it is upside down, and most ``western'' 
  folk are familiar with the feeling that ``all Chinese faces look the same'' 
  (and of course the Chinese say the same about European faces). Look at the 
  two faces in Fig.\ 2. Can you immediately be sure if these are two different 
  people, or if the picture is a composite of two images of the same man? But 
  if you turn your head round to view the picture the right way up, it will 
  probably be immediately clear that they are two different people --- they are 
  identical twins, Captains Scott and Mark Kelly, both NASA astronauts. The 
  simple act of turning a picture upside down would pose no problem to a 
  computer, but human perception behaves quite differently. 

  \fig{figs/fig-1064b04e.png}{\caption{Figure 2. 
  Image:http://campusdata.uark.edu/resources/images/articles/2017-01-31\_04-23-17-PMRS61352\_KellyBros-scr.jpg.}} 

  When you see a scene, all these feature detectors are doing their own thing, 
  sending back messages like ``there's Fred's face over there'', and ``a bird 
  just flew across the path''. If you are a bird expert, your detectors will 
  tell you what kind of bird, but most of us haven't trained our detectors up 
  well enough for that. The same goes for species of trees, or for violins: an 
  expert may recognise a particular instrument seen across the room, when most 
  of us just get the message ``violin''. 

  The high-level and low-level feature detectors allow you to look at a scene 
  in different ways, by ``directing your attention'' to one question or another 
  (whatever that really means --- but luckily we don't need to grapple with 
  philosophical issues of consciousness here). A familiar scene will be full of 
  high-level alerts from recognised objects, but if you are painting a picture 
  of the scene, you can also concentrate on, for example, the exact range of 
  colours in a patch of sunset cloud --- low-level information. 

  Hearing is similar. There are low-level detectors, which are using 
  information about things like loudness and frequency content (to be discussed 
  more carefully in the next chapter), but at the same time there are 
  high-level detectors listening out for particular things: a sound with a 
  musical pitch, or the speech patterns of someone with a particular regional 
  accent, and so on. Just using the low-level information means you have only 
  coarse and generic things to go on, disregarding all the subtle and 
  particular things you have learnt to recognise. Think of listening to your 
  native language, compared to taking a dictation test in a foreign language 
  that you are learning. 

  Skilled musicians, by long hours of practice, have trained up a set of 
  high-level feature detectors; not only for the sound of their instrument but 
  also for details of what it feels like to play a martelé bow stroke, or to 
  control a difficult high note on a trumpet. These musicians may be 
  exquisitely finely tuned to differences between instruments, especially when 
  they are playing them rather than just listening to them. Whether they are 
  explicitly conscious of these distinctions, or can talk about them in a way 
  that makes sense to anyone else, is another matter. But unconscious 
  perceptions can still be important: as an extreme example there is a 
  well-studied family of mental conditions related to ``blindsight'', in which 
  a person with brain damage in the visual cortex can sometimes point 
  accurately at an object, or reach out and pick it up, while maintaining that 
  they can't see it. People can act on sensory input that they do not think 
  they have. \tt{}Read more about it on Wikipedia\rm{} if you want to explore. 

  The low-level/high-level distinction is one reason that it is so hard to 
  quantify musical judgements by scientific measurement. It is relatively 
  simple to match, and indeed exceed, a lot of low-level human abilities using 
  computer processing, but it is another matter entirely to match the 
  high-level perception abilities of a normal human brain. That is true for 
  vision, or speech recognition, or natural language understanding, or for 
  assessing musical instruments. A lot of research is being done on these 
  things, and the ability of computers is improving, but they are still a long 
  way from matching what people can do. 

  An expert may be able to recognise the difference between two violins when 
  they listen to a recording of them, so the information about that difference 
  must be present in the recording. But to recognise this difference by 
  computer processing would require you to find out what specific aspects in 
  that information are being used by the high-level feature detectors of the 
  discriminating listener. There is a big difference between things that are 
  easy to measure and things that matter for musical decisions. It would 
  probably be simple to use a computer to show that the two recordings were 
  different. But that would not mean that the right differences had been found. 
  Think of photographs of faces again: it is one thing to know ``these are not 
  the same photograph'', it is another entirely to be able to say ``these are 
  two pictures of the same person, but he has shaved his moustache off in the 
  second one''. 

  One particular kind of low-level analysis of sound is called frequency 
  analysis or ``Fourier analysis''. We will look properly at this in the next 
  chapter, but a preliminary comment is in order. There is a persistent folk 
  belief, even (in fact especially) among scientists, that everything about 
  musical sound quality can be revealed straightforwardly by this approach, 
  leading to a common question about musical acoustics ``Is that a real 
  subject? Surely its all just frequency analysis?'' But when a skilled 
  listener is able to distinguish between the sound of two violins sufficiently 
  acutely to think that it might be worth paying a million pounds for one of 
  them, are they ``just'' doing Fourier analysis? Simply listening to the 
  relative loudness of the different harmonics in the sound? If it were that 
  simple, everyone would be able to do it because low-level information of this 
  kind is available to us all. It seems intuitively clear that there must be 
  much more to it than this: the expert has learned to assemble a variety of 
  information into a high level judgement not possible to most people. So we 
  have to guard against the hubris of scientists, and be prepared to listen to 
  musicians and expert instrument makers, even when the language they use to 
  express themselves may not have the sharpness of scientific terminology. 

  Having said all that, for most of this book these issues arising from the 
  slipperiness of human perception will be in the background. We will try not 
  to lose sight of them entirely, so that we address questions about the 
  physics of music and musical instruments which have real relevance to 
  musicians and instrument makers. But a cardinal rule of scientific research 
  is to do the easy things first, to give a secure base from which to reach out 
  towards the harder problems. ``Easy'' in this context really means ``well 
  understood by physicists'': how closely that corresponds to your notion of 
  ``easy'' may be a matter of opinion. We need to introduce some key concepts 
  about acoustics and vibration. Our aim is to apply these to musical 
  questions, but the same ideas are needed to understand and improve the 
  soundproofing between rooms in a building, or the vibration of suspension 
  bridges caused by high winds, or the way that offshore oil platforms or 
  railway lines can fail by ``fatigue'' as a result of long exposure to 
  vibration. 

  Virtually all ideas and techniques developed in modern acoustics and dynamics 
  turn out to have musical applications. A simple example is the ``tuned mass 
  damper''. When the famous ``wobbly'' Millennium Bridge in London had to be 
  closed because it vibrated too vigorously as people walked over it, a major 
  part of the ``fix'' was to add structures underneath the bridge that are 
  designed to vibrate at the troublesome resonance frequency of the bridge, and 
  suck the energy out of that vibration. These are tuned mass dampers, and they 
  are used in many other contexts: to protect buildings from excessive 
  vibration in earthquakes, and to reduce the vibration of hand-held power 
  tools, for example. They are also used as ``wolf note eliminators'' in 
  cellos: we will talk about this in section 9.4. 

  There are many other examples of musical applications of acoustical ideas and 
  of vibration engineering methods. Shock waves are responsible for sonic booms 
  from fighter aircraft or flying bullets, and also for the characteristic 
  ``brassy'' sound of a trumpet when it is played loudly (see Chapter ?). The 
  design of loudspeakers for good sound radiation at low frequencies is 
  mirrored by the design of the bodies and soundholes of violins and guitars 
  (see section 4.2). Sophisticated techniques developed to predict the 
  vibration of complex structures like cars and satellites also explain the 
  acoustical significance of the apparently decorative cut-outs in a violin 
  bridge (see section 5.3). 

  Not all the science we need to consider is physics, though. There are also 
  established scientific techniques that have been developed to give 
  quantitative information about human perception of sound: the general subject 
  is called psychoacoustics. Again, these methods can be applied to musical 
  questions, as we will explore in Chapter 6. Most psychoacoustical research 
  has concentrated on low-level aspects of perception, such as the minimum 
  detectable change in the loudness or pitch of sounds of various kinds, or the 
  way that one sound can mask the perception of another. The biggest problem we 
  will come up against when trying to apply these methods to music is that the 
  most interesting questions concern higher-level perceptual processing, and so 
  far it has proved very hard to design experiments on these questions that 
  reconcile the conflicting requirements of scientific respectability (enough 
  data for reliable statistics) and musical relevance (retaining some sense of 
  musical content and context for the listening subjects). 

  There are also things to be learned from the world of music technology and 
  computer synthesisers. Computers can be used to make many kinds of sound, and 
  naturally composers have been very keen to explore ways to harness these to 
  interesting musical effect. In terms of understanding the perceptual 
  ingredients of the sound of conventional instruments, this work on computer 
  synthesis poses a new kind of hazard. Composers are only interested in the 
  effect of a sound: they don't really care about the exact way that a 
  promising sound is achieved. If they want a computer sound that is a bit like 
  a clarinet, say, they don't in the least mind whether it is in fact created 
  in a way that mirrors how a conventional clarinet works. But if we want to do 
  synthesis for the purpose of understanding the clarinet, we do care about 
  this, and we may regard the tricks of computer synthesis as ``cheating''. 
  Just because it sounds like a clarinet, it doesn't necessarily tell you 
  anything about how you might adjust the reed or the tone-holes of a normal 
  clarinet to modify the sound. But still, if it really does sound like a 
  clarinet then it is surely telling us something important about the 
  perception of ``clarinet-ness''. We will look at some issues relating to 
  computer synthesis of sound in Chapter ?. 

  The main text of this book will be fairly chatty (or in scientific jargon, 
  ``hand-waving''). However, each section has links that give more detail for 
  the technically curious. These will open in a separate tab, so you needn't 
  lose your place in the main text: so this site works best on devices that 
  support multiple tabs. Crudely, the main text is aimed at musicians who would 
  like to know a bit about science, while the extra links are aimed at 
  scientists who would like to know a bit about music. 

  Both ``musicians'' and ``scientists'' here should be interpreted very 
  broadly: anyone interested in music, and anyone with the mindset and at least 
  a little of the training of a scientist. I make no apology for including some 
  of these technical details in an account of ``popular science''. To get an 
  impression of how a scientist, particularly a physical scientist, views the 
  world it is important to grasp that mathematics is the true language of 
  science. It provides the means to distill ideas into a precise and 
  unambiguous form, in a way that mere words can never quite do. The ideas can 
  then be tested rigorously, usually by other scientists who are not convinced 
  by something which seems so very compelling to its originator. 

  Even if you do not feel the need to face the details and are happy to leave 
  the mathematical material tucked away, there is an underlying message that is 
  important. This concerns the power of that elusive concept, ``theory''. 
  Because we use phrases like ``I know that's true in theory but....'', there 
  can be a sense that ``theory'' automatically implies something tentative, 
  likely to be wrong if examined closely. Well, it is true that all theories 
  are approximate, but some run very deep in a way which is hard to convey to 
  the non-practitioner. Others are indeed quite limited. The power of deep 
  theory is often not grasped by many people, including some professional 
  scientists, especially experimentalists. 

  There are some kinds of thing for which there is believed to be a good 
  theoretical understanding, at least in general terms. If a measurement seems 
  to show something which contradicts expectation based on one of these deep 
  theories (such as Newton's laws of motion, or the conservation of energy), it 
  either means that the experimentalist is on track for a Nobel prize or, more 
  likely, that there is something wrong with the measurement or its 
  interpretation. Even when someone is in fact on track for the Nobel prize 
  with some remarkable discovery that flies in the face of what everyone 
  believes, they still have to convince people that their results are right. 
  There are famous examples of radical overturning of conventional ideas: 
  Alfred Wegener was initially ridiculed for the idea of continental drift 
  whereas now it has become established orthodoxy in the theory of plate 
  tectonics. However, it is very much more common for the opposite to happen: a 
  dramatic announcement may be made of some great new discovery, only to be 
  revealed as an error or a misinterpretation after careful checking. 

  But these ``errors'' can still be the germ of creative developments: that is 
  how science works, by a continual competitive race to find and prove new 
  ideas that stand up to the most determined critical scrutiny. We will 
  repeatedly come up against frontiers of scientific research in the course of 
  this book. Not, on the whole, the glamorous kind of research that wins Nobel 
  prizes, but the everyday research of scientists as they work away at the 
  giant jigsaw puzzle that makes up currently accepted scientific knowledge. 
  The cumulative effect of many scientists putting pieces into this puzzle is 
  ultimately what gives science its power: big pictures gradually emerge from 
  patient and persistent work in specialised corners. Most individual pieces of 
  research can seem to the non-specialist uninteresting, even trivial, because 
  they are so tightly focussed on one small question. Only by stepping back 
  occasionally to look at the bigger picture does it become clear that progress 
  is gradually being made. 

  \fig{figs/fig-8aa763f1.png}{} 

