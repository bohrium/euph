  Figure 1 shows the geometry of an idealised Helmholtz resonator. A vessel 
  with rigid walls encloses a volume $V$, and is connected to the outside world 
  via a neck of length $L$ and cross-sectional area $S$. As usual, we assume 
  that the pressure and density take the forms $p(t)=p_0+p'(t)$ and 
  $\rho(t)=\rho_0+\rho'(t)$, where subscripts 0 denote the steady atmospheric 
  pressure and density, and primes denote the acoustic perturbations, assumed 
  to be small in comparison. We consider the case where the acoustic pressure 
  is $p'_1(t)$ just outside the neck of the vessel, and we aim to calculate the 
  corresponding pressure $p'_2(t)$ inside the volume. We define the mass flow 
  rate inwards through the neck to be $q(t)$. 

  \fig{figs/fig-6e31fb86.png}{Figure 1. An idealised Helmholtz resonator: a 
  rigid bottle of volume V with a neck of length L and cross-sectional area S} 

  Conservation of mass requires that 

  $$q=V\dfrac{\partial \rho'}{\partial t} \tag{1}$$ 

  so for a harmonic disturbance at angular frequency $\omega$, 

  $$q=V i \omega \rho' . \tag{2}$$ 

  This density change produces a proportional change in pressure, and we 
  already know from section 4.1.1 that the constant of proportionality is the 
  square of the speed of sound $c$: 

  $$p'_2=c^2 \rho'_2 = \dfrac{c^2 q}{i \omega V} . \tag{3}$$ 

  Now if the ``plug'' of air in the neck moves as a rigid body, Newton's law 
  requires 

  $$\left( p'_1 -- p'_2 \right) S = \rho_0 L S \dfrac{\partial u}{\partial t} 
  \tag{4}$$ 

  where $u$ is the velocity of the plug into the neck. But $q=\rho_o S u$, so 

  $$p'_1 = \dfrac{c^2 q}{i \omega V} + \dfrac{L}{S} i \omega q . \tag{5}$$ 

  Thus 

  $$q= -\dfrac{i \omega S}{L} \dfrac{p'_1}{\left[\omega^2 -\frac{c^2 
  S}{VL}\right]} \tag{6}$$ 

  and so finally 

  $$p'_2= -\dfrac{c^2 S}{VL} \dfrac{p'_1}{\left[\omega^2 -\frac{c^2 
  S}{VL}\right]} . \tag{7}$$ 

  Resonance obviously occurs when 

  $$\omega = c \sqrt{\frac{S}{VL}} .\tag{8}$$ 

  In this very simple theoretical model the response is infinite at resonance. 
  As usual, the amplitude is limited in practice by damping, which in this case 
  arises from viscous losses and from sound radiation. 

  To calculate a Helmholtz resonance frequency more accurately we would need to 
  include an end correction: the effective mass of air moving in and around the 
  neck is a little larger than the geometric dimensions of the neck itself, 
  because of the kinetic energy involved in the air flow near the ends of the 
  neck. This is usually allowed for by adding a small extra length to the real 
  length. The magnitude of the required correction depends on the geometry 
  around the opening, and requires numerical calculation. Two simple idealised 
  cases give an idea of the magnitude: they involve a circular opening of 
  radius $a$, and a neck which is either ``flanged'' or ``unflanged'', as 
  sketched in Fig.\ 2. 

  \fig{figs/fig-f01902af.png}{Figure 2. End corrections for a neck without 
  (left) and with (right) a flange} 

  One consequence is that even a thin-walled vessel with an opening can show a 
  Helmholtz resonance: an example is sketched in Fig.\ 3. For this case there 
  is a ``flange'' on both sides of the opening, so the effective neck length 
  for a circular opening of radius $a$ will be $L \approx 2 \times 0.85 a=1.7 
  a$. 

  \fig{figs/fig-9dd59bbd.png}{Figure 3. A thin-walled Helmholtz resonator, 
  where the effective neck length consists entirely of the end corrections.} 