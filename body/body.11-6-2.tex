  An enhanced model for free-reed behaviour was proposed by Millot and Baumann 
  [1]. It adds one extra ingredient to Fletcher's model (discussed in detail in 
  section 11.6.1), as sketched in Fig.\ 1. A tube of length $L_2$ and 
  cross-sectional area $S_2$ is placed in between the volume $V$ and the reed. 
  The pressure in the main volume, $p_1(t)$, might now be different from the 
  pressure at the inner face of the reed, $p_2(t)$. The other variables remain 
  the same as in the discussion of section 11.6.1: flow rate $u(t)$ past the 
  reed, reed tip displacement $x(t)$, reed properties $\omega_r$, $Q_r$ and 
  $m$, and the reed area function $F(x)$. 

  \fig{figs/fig-bf3fa852.png}{\caption{Figure 1. Millot's model, with the reed 
  connected to a chamber of volume $V$ by a tube of length $L\_2$ and 
  cross-sectional area $S\_2$.}} 

  Most of the equations are virtually unchanged. The Bernoulli expression 
  relating to the flow past the reed involves the pressure $p_2$: 

  \begin{equation*}p_2 \approx \dfrac{\rho_0 u^2}{2 C^2 F(x)^2} . 
  \tag{1}\end{equation*} 

  The same is true for the equation describing the reed dynamics: 

  \begin{equation*}\ddot{x}+\dfrac{\omega_r}{Q_r}\dot{x}+\omega_r^2 x = K_p p_2 
  . \tag{2}\end{equation*} 

  The equation relating the rate of change of pressure inside the chamber to 
  the net volume flow rate of air into it obviously involves the pressure 
  $p_1$: 

  \begin{equation*}\dot{p_1}=\dfrac{\rho_o c^2}{V}[U_0 -u -- K_x \dot{x}] . 
  \tag{3}\end{equation*} 

  Finally, we use the ``Helmholtz resonator'' approximation to treat the 
  ``plug'' of air in the new tube as if it was a rigid mass: Newton's law then 
  states 

  \begin{equation*}S_2(p_1-p_2)=\rho_0 S_2 L_2\dfrac{d}{dt}\left(\dfrac{u+K_x 
  \dot{x}}{S_2} \right) . \tag{4}\end{equation*} 

  This gives an equation linking $p_1$ and $p_2$. We can re-write it slightly 
  by making use of the formula for the Helmholtz resonance frequency of the 
  chamber and tube, from section 4.2.1: 

  \begin{equation*}\omega_h^2=\dfrac{c^2 S_2}{VL_2} . \tag{5}\end{equation*} 

  Equation (4) then becomes 

  \begin{equation*}p_1-p_2=\dfrac{\rho_0 c^2}{\omega_h^2 V} 
  \dfrac{d}{dt}\left(u+K_x \dot{x} \right) . \tag{6}\end{equation*} 

  We then follow the same procedure as in section 11.6.1, setting 
  $p_1=\bar{p_1}+ p'_1 e^{i \omega t}$ and so on. The only real change is the 
  new equation (6), which is linear and so is easy to deal with. After a little 
  algebra, we can reach a result that is almost identical to equation (16) from 
  section 11.6.1: 

  \begin{equation*}\left[ -\omega^2 + i \omega \dfrac{\omega_r}{Q_r} + 
  \omega_r^2 \right]x' = -K_p \dfrac{i \omega D + E}{i \omega A' + B} 
  x'\end{equation*} 

  \begin{equation*}=-K_p \dfrac{(i \omega D + E)(-i \omega A' + B)}{\omega^2 
  A'^2 + B^2} x' . \tag{7}\end{equation*} 

  The only difference is that the constant $A$ has been replaced by 

  \begin{equation*}A'=\dfrac{A}{1-\omega^2/\omega_h^2}=\dfrac{V}{\rho_0 c^2 
  (1-\omega^2/\omega_h^2)} . \tag{8}\end{equation*} 

  The new condition for instability is thus 

  \begin{equation*}\dfrac{\omega_r}{Q_r} < K_p\dfrac{EA'-DB}{\omega_r^2 A'^2 + 
  B^2} \tag{9}\end{equation*} 

  \noindent{}where $A'$ is evaluated at $\omega=\omega_r$, using the same 
  approximation as before that the reed generally vibrates close to its natural 
  frequency. The only terms in this equation which are not necessarily positive 
  are $A'$ and $E=U_0 \dfrac{\bar{F}'}{\bar{F}}$. But notice that these terms 
  appear as the product $A'E$, so if the signs of both are changed, the 
  stability threshold will be exactly the same. Instability requires this 
  product to be positive, so either an opening reed with $\omega_r < \omega_h$ 
  or a closing reed with $\omega_r > \omega_h$. 

  We can relate this pattern to Fletcher's impedance formulation of the 
  stability condition, described at the end of section 11.6.1. The input 
  impedance that would be measured at the reed, for the Millot model, can 
  easily be evaluated from the equations above: the result is 

  \begin{equation*}Z_M(\omega) = \dfrac{\rho_0 c^2}{i \omega 
  V}\left[1-\omega^2/\omega_h^2\right] .\tag{10}\end{equation*} 

  We saw in section 11.6.1 that for a closing reed, with $\bar{F}' < 0$, the 
  imaginary part of the impedance must be positive for instability, whereas for 
  an opening reed it must be negative. Equation (10) then shows that if 
  $\omega_h < \omega$, a closing reed could be unstable, whereas if $\omega_h > 
  \omega$ then the opening reed could be unstable. 

  \sectionreferences{}[1] L. Millot and Cl. Baumann, “A proposal for a minimal 
  model for free reeds”, Acta Acustica united with Acustica \textbf{93}, 
  122—144 (2007). 