  \tt{}Euphonics\rm{} 

  The science of musical instruments                           ISSN 2977-5612 

  The formulation of a ``digital waveguide'' simulation model for an idealised 
  bowed string is very similar to the equivalent model for the clarinet, 
  detailed in section 8.5.3. There are two main differences: first, the 
  excitation of the clarinet occurs at the end of the tube while a string is 
  bowed somewhere in the middle; and second, the nonlinear function associated 
  with the simplest friction model is more severely non-smooth than the 
  mouthpiece characteristic we assumed for the clarinet. The first issue makes 
  the formulation slightly more complicated, but makes no fundamental change in 
  the way the model operates. 

  The second issue makes a bigger difference: it means that stick-slip motion 
  of a bowed string always involves non-smooth behaviour. There is no 
  equivalent of the linearised argument leading to a threshold of blowing 
  pressure, as we found in the clarinet case. This fits with experience of 
  playing a violin: there is no possibility of quasi-sinusoidal motion when you 
  press lightly or bow slowly: if the string vibrates it is always stick-slip 
  motion of some kind with a rich spectrum of harmonics. 

  The starting point for modelling is similar to the clarinet case: we assume 
  that in the immediate vicinity of the single point where we want the bow to 
  act, the transverse vibration of the string can be approximated by that of an 
  ideal string (as analysed in section 3.1.1). First, we think about an 
  infinite string, extending away on both sides from the bowed point. If we 
  apply a force $f(t)$ to such an infinite string, the velocity response of the 
  string at the bowed point is $v(t)= f(t)/(2Z_0)$ where $Z_0=\sqrt{T/m}$ is 
  the wave impedance of the string, of tension $T$ and mass per unit length 
  $m$. In other words, our infinite string behaves like an ideal dashpot, with 
  velocity proportional to applied force. This motion at the bowed point 
  generates two outgoing waves of magnitude $f(t)/(2Z_0)$, sent off 
  symmetrically in the two directions. 

  Of course, our real string has a finite length. The situation is then as 
  sketched in Fig.\ 1. The force $f(t)$ generates outgoing waves as on the 
  infinite system (shown in red in the figure), but now after a delay reflected 
  waves (shown in blue) return to the bowed point from the two sides. Each 
  separate reflected wave can be computed by convolution of the outgoing wave 
  with a reflection function similar to the one used in the clarinet model. 
  This time we need two reflection functions, which we can call $r_1(t)$ and 
  $r_2(t)$, describing reflection behaviour from the bridge side and the finger 
  side respectively. They will have different delays, which will depend on 
  where the player chooses to place the bow on the string. This bowing position 
  will play an important role in later discussions: it is usually described via 
  a non-dimensional parameter $\beta$ which is the bow-bridge distance 
  expressed as a proportion of the string length. 

  \fig{figs/fig-76514fbc.png}{\caption{Figure 1. Schematic diagram showing 
  outgoing waves from the bowed point (in red) and reflected waves returning to 
  the bowed point (in blue).}} 

  At a given time $t$, we can denote the incoming and outgoing velocity waves 
  on the two sides of the bowed point by $v_1^{(i)}(t), v_1^{(o)}(t), 
  v_2^{(i)}(t)$ and $v_2^{(o)}(t)$. They are related in two different ways. The 
  outgoing wave on one side is given by the incoming wave from the other side, 
  plus the new contribution from the force $f(t)$, so that 

  $$v_1^{(o)} = v_2^{(i)} + \dfrac{f}{2Z_0} \mathrm{~~~ and~~~} v_2^{(o)} = 
  v_1^{(i)} + \dfrac{f}{2Z_0} . \tag{1}$$ 

  Second, the incoming waves on each side are generated from the corresponding 
  outgoing waves using the relevant reflection function: 

  $$v_1^{(i)} = v_1^{(o)} * r_1 \mathrm{~~~ and~~~} v_2^{(i)} = v_2^{(o)} * r_2 
  \tag{2}$$ 

  where * denotes the operation of convolution. We can call the combined 
  velocity resulting from these incoming waves 

  $$v_h(t)=v_1^{(i)}(t) + v_2^{(i)}(t) \tag{3}$$ 

  where the subscript `h' suggests ``history'', just as in the clarinet model 
  of section 8.5.3. 

  At a given time $t$ we can calculate $v_h(t)$ from our knowledge of the past 
  history of the motion. Now the actual velocity at the bowed point is the sum 
  of the infinite-string response and the returning reflected waves: 

  $$v(t)=\frac{f(t)}{2Z_0} +v_h(t) . \tag{4}$$ 

  This is the direct equivalent for the bowed-string problem of equation (9) 
  from section 8.5.3. It is the equation of a straight line in the $v-f$ plane, 
  with slope $2Z_0$ and horizontal position governed by the value of $v_h$. 

  The new values of $v(t)$ and $f(t)$ are given by the intersection of this 
  straight line with the friction force --- velocity curve. If the coefficient 
  of friction is denoted $\mu(v)$ and the normal bow force is $F_b$, then this 
  curve is given by $f=F_b \mu(v)$. For the particular plots and simulations 
  shown in section 9.2, the curve-fitted coefficient of friction takes the form 

  $$\mu(v)=(\mu_{ss}-\mu_d) e^{-v/v_{cc}} + (\mu_s-\mu_d) e^{-v/v_c} + \mu_d 
  \tag{5}$$ 

  with $\mu_{ss}=1.2$, $\mu_s=0.8$, $\mu_d=0.35$, $v_{cc}=0.01$ m/s and 
  $v_c=0.1$ m/s. 