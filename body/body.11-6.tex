

  We now turn to the next family of wind instruments, the ``free reeds'': this 
  family includes the harmonium, the accordion, the concertina and the 
  harmonica. We can illustrate the shared features of these instruments with a 
  harmonica. Figure 1 shows a typical 10-hole diatonic harmonica; first in its 
  complete state, and then after the cover plates have been removed. The 
  central “comb” creates a series of small chambers into which the player can 
  blow or suck air. Each of these chambers aligns with two thin brass reeds, 
  fixed over slots in the two reed plates. One set of reeds, shown in close-up 
  in the lower left-hand image, are fixed to the inner surface of the plate, 
  while the other set (shown on the right) are fixed to the outer surface. A 
  reed from the first set will vibrate when the player blows into the 
  corresponding hole: these are the “blow reeds”. The second set are called the 
  “draw reeds”, and are ordinarily set into vibration by sucking air out of the 
  hole. 

  A harmonium or reed organ has only “blow reeds”: air always flows in one 
  direction, pumped by hand, or foot pedals, or an electric pump. However, 
  instruments like the concertina, accordion or bandoneon have sets of 
  inward-facing and outward-facing reeds similar to the harmonica. The air flow 
  is created by the player expanding or compressing a bellows-like structure. 
  Buttons or keys are used to operate levers which open internal valves to 
  allow the air to flow into or out of each reed chamber. Figure 2 shows a 
  concertina, disassembled into some of its component parts. On the lower 
  right, the radial reed chambers can be seen. In each chamber one reed is 
  visible, while another is on the reverse side of the plate, hidden from view 
  by a leather flap which reduces air leakage when the first reed is being 
  played. 

  What all these instruments lack, in contrast to the reed and brass 
  instruments discussed in earlier sections, is any kind of resonating tube to 
  determine the playing pitch. Instead, the pitch is determined by the lowest 
  vibration resonance of the reed itself. But it is not immediately clear why 
  air-flow past a reed should set it into vibration, nor why vibration only 
  occurs when the air-flow is in one direction through the reed plate, and not 
  in the reverse direction. Why do blow reeds only blow, and draw reeds only 
  draw? 

  These are questions we want to address by modelling and simulation, but we 
  will discover that the answers are complicated, and not fully understood yet 
  --- we will come up against a frontier of research. In summary, it currently 
  looks as if there may be two different mechanisms of instability of free 
  reeds: one related to the kind of effects we have been talking about in the 
  previous sections on reed and brass instruments (and therefore amenable to a 
  similar style of modelling), the other involving more challenging fluid 
  dynamical effects that are harder to model in a simple way. Of course, for 
  something like the concertina shown in Fig.\ 2, part of the answer seems 
  simple: the leather flaps prevent significant air flow in the reverse 
  direction, so the question of which reed sounds for a given direction of 
  air-flow doesn't really arise. But there is more to the question than that, 
  as we shall see. 

  The first step towards addressing this question seems paradoxical. There is 
  very little scientific literature on the theory of free reed vibration, and 
  there has apparently been only one careful experiment designed to test the 
  predictions of a theory in quantitative detail (by Tarnopolsky, Fletcher and 
  Lai [1]). But that experiment makes use of a test system in which what looks 
  like a draw reed in fact chooses to vibrate as a blow reed! We will come to 
  the details of their model and experiment shortly, but it gives an immediate 
  hint that the answer to the question we just asked might be rather subtle and 
  complicated. 

  The key difference between the two reed configurations is sketched in Fig.\ 
  3. On the left, the reed is attached to the plate on the same side as the 
  source of air flow. That means that if the reed displaces slightly in the 
  direction of the air flow, it tends to reduce the area of gap between the 
  reed and the plate: a “closing reed”. On the right, the reed is attached on 
  the opposite side of the plate and if it moves in the direction of the air 
  flow the gap increases: an “opening reed”. We can recognise both these 
  tendencies from earlier sections: reed woodwind instruments like clarinets 
  and saxophones have closing reeds, while our model for a brass-player’s lip 
  buzzing was an opening reed. 

  \fig{figs/fig-eb253c51.png}{Figure 3. Sketch of the two reed configurations: 
  a ``closing reed'' on the left, and an ``opening reed'' on the right.} 

  We have already seen that the minus sign needed to turn a closing reed model 
  into an opening reed model had profound consequences for the behaviour of 
  those instruments. A similar profound difference applies to free reeds. All 
  the free-reed instruments we have mentioned are designed to play with closing 
  reeds, as we can see in the two pictures in the bottom row of Fig.\ 1: the 
  blow reeds, on the left, are inside the cavity, while the draw reeds on the 
  right are outside the cavity. A given reed cannot ordinarily be made to 
  vibrate by air flow in the ``opening reed'' direction, and this is vital to 
  the normal functioning of the instrument: you expect to get a different note 
  when blowing or drawing on a given reed cavity. 

  It is time to think about developing a theoretical model for free-reed 
  instruments. You might think that the simplest model would correspond to a 
  reed set in a wall between two large open spaces, with a different steady 
  pressure imposed in the two spaces. That would certainly create some air flow 
  past the reed, related to the pressure difference by the kind of model we 
  have already used successfully for woodwind and brass instruments. 

  But this configuration lacks a crucial ingredient if we want to build a model 
  like the earlier ones. In our models of woodwind and brass instruments, 
  spontaneous oscillation happened as a result of a feedback loop, in which the 
  air flow past the reed interacted with the acoustical response of the 
  instrument tube. But if both sides of the reed are essentially open to empty 
  space, no such strong acoustical feedback can occur. That is not to say that 
  a reed set in a wall between two large rooms might not vibrate with 
  sufficient pressure difference — at present, it appears to be an open 
  question whether this can occur. But if did occur, it would rely on a 
  different kind of instability, perhaps depending on the detailed pattern of 
  local fluid flow. We will side-step this possibility for the moment, but we 
  will return to it later, at least briefly. 

  The very simplest option to create some acoustic feedback is to have the reed 
  set in the wall of a closed chamber with a finite volume. Air flow in and out 
  of this chamber will automatically create changes of mean density, leading to 
  changes of pressure that can provide acoustical feedback on the reed 
  vibration. In terms of instruments, this chamber might correspond to the 
  bellows of a concertina or the mouth cavity of a harmonica player. The 
  resulting idealised geometry is sketched in Fig.\ 4, and this is precisely 
  the situation investigated by the experiment of Tarnopolsky, Fletcher and Lai 
  [1], designed to test a theoretical model developed earlier by Fletcher [2]. 

  \fig{figs/fig-81b6fb96.png}{Figure 4. Fletcher's model of a reed attached to 
  a closed volume of air, driven by a steady air flow. This sketch shows an 
  ``opening reed'' configuration, but the reed could equally well be attached 
  to the inside of the chamber wall to give a ``closing reed''.} 

  Fletcher’s analysis, as applied to this configuration, is described in the 
  next link. He used linearised theory to investigate the threshold of 
  instability in terms of the mean pressure inside the volume (which in turn is 
  determined by the rate of air-flow injected into the volume by the player). 
  Before we see some results of his predictions, we need to understand a 
  feature of free reeds which distinguishes them from clarinet reeds or brass 
  player’s lips. When the reed moves towards the slot in the reed plate, it 
  initially reduces the gap as we have already said. But the reed does not 
  close the gap completely: instead, the reed can pass right through the gap 
  and move out on the other side as sketched in Fig.\ 5. 

  \fig{figs/fig-fb72c86b.png}{Figure 5. Sketch showing a reed vibrating 
  vigorously, so that it passes all the way through the plate.} 

  Our model for air-flow through the reed (see section 11.3.1) involves the 
  area of the gap, as a function of the tip displacement of the reed. For the 
  opening-reed configuration shown in Fig.\ 4, that area function looks like 
  the red curve in Fig.\ 6. For a positive tip displacement (away from the wall 
  of the chamber, in the opening-reed direction), the area simply increases 
  with displacement. But for a negative displacement the area reduces 
  initially, then is roughly constant when the reed lies within the thickness 
  of the plate, then increases again once the reed comes out on the other side 
  so that the gap starts to grow. If the same reed were swapped around to the 
  closing-reed configuration, the result would be the blue curve in Fig.\ 6. 
  This is simply the reflection of the red curve in the vertical axis. 

  \fig{figs/fig-835ab317.png}{Figure 6. Plot of the area available for air flow 
  past a reed as a function of the tip displacement, for an opening reed (red) 
  and a closing reed (blue). This example is for a harmonica reed of length 
  14.5~mm and width 2~mm, stood off from a plate of thickness 0.8~mm by a 
  distance 0.5~mm. The clearance between reed and slot is 0.2~mm all round.} 

  Putting these ingredients together, Fletcher predicted that the reed would 
  vibrate most readily in the opening-reed orientation. He then predicted that 
  the threshold pressure would depend on the volume of the cavity, and also on 
  the Q-factor of the reed vibration. These are the predictions that were 
  tested in the experiment by Tarnopolsky, Fletcher and Lai. Their results are 
  summarised in Fig.\ 7. The four lines show the predicted pressure threshold 
  as a function of volume, for four different values of the Q-factor. The stars 
  in corresponding colours show the measured results. It is immediately clear 
  that the agreement is not perfect, but that the qualitative trends were 
  matched and the quantitative agreement is pretty good (they did not give 
  estimates of their experimental uncertainties). 

  \fig{figs/fig-da259f84.png}{Figure 7. Predicted threshold pressure according 
  to Fletcher's theory, compared to measurements replotted from Fig. 2 of 
  Tarnopolsky, Fletcher and Lai [1]. Colours indicate the Q-factor of the reed: 
  16 (red), 19 (blue), 29 (green) and 55 (black). Note that their original plot 
  contains an error: the labels for Q-factors were in the reverse order.} 

  This leaves us with the paradox mentioned earlier. Why does this 
  configuration of reed and chamber mean that the reed prefers to vibrate in 
  the opening-reed orientation, when all free-reed instruments do the opposite? 
  We can shed some light on this question by enhancing the model in a very 
  simple way suggested by Millot and Baumann [3] and sketched in Fig.\ 8. 
  Instead of connecting directly to the wall of the chamber, the reed is driven 
  via a small tube. As explained in the next link, this makes a small but 
  crucial difference to the stability analysis. One extra parameter now enters: 
  the Helmholtz resonance frequency of chamber and tube, when the reed is 
  absent and the tube is open at the right-hand end. 

  \fig{figs/fig-67af104a.png}{Figure 8. A modified version of Fig. 4, suggested 
  by Millot [3]. A tube has been added between the chamber and the reed.} 

  The behaviour then depends on whether that Helmholtz resonance frequency is 
  above or below the resonance frequency of the reed: a crucial sign is 
  reversed by this choice, and this interacts with another important sign 
  associated with the area function from Fig.\ 6. Figure 9 shows a version of 
  that area function relevant to the geometry of the Tarnopolsky, Fletcher and 
  Lai experiment, except that for reasons to be explained shortly I have set 
  the thickness of the reed plate to 1~mm rather than the actual 4~mm. The only 
  aspect of these curves that enters the stability analysis is the slope — but 
  this is not the slope near zero displacement, it is the slope at the 
  “operating point” of the reed. The static blowing pressure means that the 
  reed is displaced outwards. The four vertical dashed lines in the plot show 
  the displacements corresponding to zero pressure (on the left), and to 1, 2 
  and 3~kPa moving progressively rightwards. 

  \fig{figs/fig-8836c11a.png}{Figure 9. Plot of the area available for air flow 
  past a reed as a function of the tip displacement in the same format as Fig. 
  6, for an opening reed (red) and a closing reed (blue). This example is for 
  the reed geometry relevant to the experiment of Tarnopolsky, Fletcher and Lai 
  [1]: length 28~mm and width 52~mm, stood off from the base plate by a 
  distance 1~mm. The clearance between reed and slot is 0.5~mm all round. The 
  thickness of the plate is 1~mm, different from the actual value 4~mm in order 
  to illustrate a phenomenon explained in the text. The dashed lines mark the 
  static displacement of the reed under the influence of blowing pressure zero 
  (leftmost), 1~kPa, 2~kPa and 3~kPa (rightmost).} 

  Now look at the slopes of the red and blue curves where these dashed lines 
  cross them. For the opening reed (red curve), the slope stays positive 
  throughout, and indeed it hardly changes as the operating point shifts. But 
  the blue curve shows more complicated behaviour. Initially, the slope is 
  strongly negative. Once the pressure reaches about 1~kPa, the reed has moved 
  inwards far enough to compensate for the 1~mm stand-off from the reed plate, 
  so the reed enters the slot in the plate and the slope begins to level off. 
  Once the pressure gets above about 2~kPa, the reed tip starts to emerge on 
  the other side of the plate, and the slope reverses. By 3~kPa the reed has 
  moved well beyond the plate, and the slope is positive, not very much lower 
  than the slope of the red curve. It was in order to reach this final state 
  with a reasonable mouth pressure that I chose to reduce the plate thickness 
  to 1~mm. 

  Now we can look at the behaviour of the threshold blowing pressure, and Fig.\ 
  9 will help us to understand the results. Some cases are illustrated in 
  Figs.\ 10 and 11. The Q-factor of the reed is kept fixed at 55, while we vary 
  the frequency ratio of the reed resonance to the Helmholtz resonance. In 
  Fig.\ 10 the Helmholtz resonance is always higher than the reed resonance. 
  The solid red curve shows the threshold for the opening reed when the 
  Helmholtz resonance frequency is very high. This curve is essentially 
  identical to the black curve in Fig.\ 7 (it only looks different because the 
  plotting ranges are different in the two figures). 

  \fig{figs/fig-7ec0c143.png}{Figure 10. Predicted pressure thresholds in a 
  similar format to Fig. 7, but plotted over different ranges. The solid lines 
  relate to the opening-reed configuration, the dash-dot lines to the 
  closing-reed configuration. The Q-factor of the reed is 55 in all cases. The 
  four colours relate to different values for the ratio of reed resonance 
  frequency to Helmholtz resonance frequency. Here the ratios are 0.01 (red), 
  0.5 (blue), 0.8 (green) and 0.95(black).} 

  The explanation lies in the comparison of Figs.\ 4 and 8. The original 
  Fletcher case of Fig.\ 4 is a special case of Millot’s model in Fig.\ 8, in 
  which the length of the extra tube has been shrunk to zero. That has the 
  effect of sending the Helmholtz resonance frequency off towards infinity, so 
  that it is much higher than the reed resonance frequency. The other solid 
  curves in Fig.\ 10 show what happens as the Helmholtz resonance frequency 
  comes down, until it is only 5\% higher than the reed resonance (black 
  curve). The opening-reed configuration plays more and more easily! 

  The dash-dot curves in Fig.\ 10 show the corresponding pressure thresholds 
  for the closing-reed configuration. For all four curves, the closing reed can 
  indeed be induced to vibrate, with a pressure above 2~kPa. Figure 9 explains 
  what is happening: we already commented that above about 2~kPa, the slope of 
  the blue curve for the closing reed reverses, so that it has the same sign as 
  the red curve. That slope is the only thing that enters the stability 
  calculation, so we should not be too surprised to see Fig.\ 10 predicting 
  instability of the closing reed at this kind of pressure. 

  Figure 11 shows the corresponding plot for several cases in which the 
  Helmholtz resonance frequency is lower than the reed resonance frequency. The 
  behaviour has reversed: we only see dash-dot curves. The closing-reed 
  configuration is now the one that plays quite readily, and the opening-reed 
  case never plays. This is the pattern we expect from all the free-reed 
  instruments we looked at earlier. Notice that the pattern is also consistent 
  with what we found in earlier sections: the brass instruments, with a 
  opening-reed configuration, always have a playing frequency above the “reed” 
  resonance, whereas the reed woodwinds, with a closing-reed configuration, 
  typically play at frequencies well below the reed resonance. 

  \fig{figs/fig-9c047aef.png}{Figure 11. Predicted pressure thresholds in the 
  same format as Fig. 10, but with the values for the ratio of reed resonance 
  frequency to Helmholtz resonance frequency now greater than unity. Here the 
  ratios are 1.05 (red), 1.2 (blue), 1.5 (green) and 2 (black).} 

  Insofar as Millot’s model can be applied to real instruments, these results 
  suggest that with any free-reed instrument featuring pairs of opening and 
  closing reeds, we should always expect one reed to function as a draw reed 
  and the other as a blow reed. But they could be either way round: which is 
  the blow reed and which is the draw reed depends on the resonance frequencies 
  of the reeds relative to the Helmholtz resonance frequency of the instrument. 

  This raises an obvious question: why does it seem to be the case that all 
  normal free-reed instruments operate in the closing-reed configuration? We 
  can get some insight into that question from Fletcher’s original analysis, 
  which was couched in terms of impedance. Some details were given at the end 
  of the first side link above. The analysis shows that instability depends 
  crucially on the sign of the imaginary part of the impedance: that sign must 
  be positive at the oscillation frequency of a closing reed, or negative for 
  an opening reed. But this brings us to the frontier of current knowledge. Can 
  Millot's model really be applied to real instruments? And if so, why do they 
  all show a pattern with the imaginary part of the impedance positive? To go 
  further, what is needed is some measurements of input impedance in free-reed 
  instruments — a promising future topic for researchers in the field. 

  It is prudent, and also of interest in its own right, to check the 
  predictions of the linearised stability theory against the results of 
  simulations. For this purpose we will not use the rather unusual reed 
  geometry of the Tarnopolsky experiment, but instead we will apply Millot’s 
  model to a typical harmonica reed. We expect a harmonica reed to play in a 
  closing-reed configuration, so the Helmholtz resonance frequency was chosen 
  to be lower than the reed resonance by a factor 1.5. Running a grid of 
  simulations for various values of cavity volume and mouth pressure, we obtain 
  results as shown in Figs.\ 12 and 13. Figure 12 is colour-shaded to indicate 
  the spectral centroid of the waveform of pressure inside the reed chamber, 
  while Fig.\ 13 shows the peak-to-peak amplitude of that pressure waveform. 

  \fig{figs/fig-7fae24a7.png}{Figure 12. A set of synthesised results for a 
  harmonica reed with properties as listed in the caption of Fig. 6, tuned to 
  440 Hz. The ratio of reed resonance frequency to Helmholtz frequency has been 
  set to 1.5, so the reed plays in the closing-reed configuration as usual in a 
  harmonica. The colour shading indicates the spectral centroid frequency. The 
  blue line marks the threshold of instability according to Fletcher's 
  analysis.} 

  \fig{figs/fig-c2f98da6.png}{Figure 13. The same set of synthesised results as 
  in Fig. 12, now colour-shaded to indicate the peak-to-peak amplitude of the 
  internal pressure waveform in the final periodic portion of each note.} 

  It is immediately apparent from Fig.\ 12 that the simulation results agree 
  very well with the predicted stability threshold, shown in the blue curve. It 
  is not so easy to see this agreement in Fig.\ 13, however, because the 
  amplitude of the pressure signal fades away as the cavity volume increases. 
  To see what this means in terms of the pressure waveforms, Fig.\ 14 shows the 
  final periodic portion of the five cases marked by a green rectangle in 
  Figs.\ 12 and 13. You can listen to these five synthesised waveforms in Sound 
  1: it is obvious that the sound gets progressively quieter as the cavity 
  volume increases. We should not be surprised by this: we introduced the 
  finite cavity volume in order to provide acoustical feedback to the reed, but 
  the bigger the cavity the weaker that feedback will be because a given amount 
  of airflow induced by the reed motion will have less impact on the pressure 
  in a larger chamber. 

  \fig{figs/fig-f88e514f.png}{Figure 14. Simulated waveforms of internal 
  pressure for the same reed model as in Figs. 12 and 13. This plot shows 
  extracts from the final periodic waveform. The five cases correspond to the 
  five pixels marked with a green rectangle in Figs. 12 and 13, with the case 
  with the lowest cavity volume at the top.} 

  The pressure waveforms revealed in Fig.\ 14 show a pair of peaks in every 
  cycle. To see how these arise, Fig.\ 15 shows one of these waveforms 
  alongside other quantities that are computed in the course of the simulation. 
  The top curve shows the displacement of the reed tip. The dashed lines mark 
  the thickness of the reed plate: I have deliberately chosen a rather extreme 
  case, in which the reed vibrates through the plate and out on the other side. 
  The second panel of Fig.\ 15 shows the pressure waveform we have already seen 
  (it is the top one in Fig.\ 14); the third panel shows the rate of air flow 
  past the reed. Comparing these with the top panel, it is clear that each time 
  the reed tip passes through the plate, obstructing the gap through which air 
  can flow, there is strong dip in the air flow rate, and a big peak in the 
  pressure inside the reed cavity. The reed motion looks more or less 
  sinusoidal, but the ``peaky'' pressure waveform is very rich in higher 
  harmonics, a characteristic feature of the sound of free-reed instruments. 

  \fig{figs/fig-2ffac854.png}{Figure 15. Detailed waveforms for the simulation 
  shown in the top curve of Fig. 14. Top: reed tip displacement, with dashed 
  lines indicating the position of the reed plate; middle: the pressure 
  waveform as in Fig. 14; bottom: the air-flow rate around the reed.} 

  Figure 16 shows the initial transient from one of the simulations — 
  specifically, it corresponds to the second pixel from the left in the marked 
  rectangles in Figs.\ 12 and 13. The early part of the transient shows a 
  single pressure peak in each cycle, because the reed is not yet passing right 
  through the plate. But as the amplitude grows, the second peak builds up 
  until it is roughly half the height of the first peak, as we saw in the final 
  periodic waveform (second from the top in Fig.\ 14). 

  \fig{figs/fig-edd863d0.png}{Figure 16. Simulated initial transient of a 
  harmonica reed, corresponding to the second pixel from the left in the set 
  marked in Figs. 12 and 13, and to the final periodic waveform plotted in blue 
  in Fig. 14.} 

  I said we would return briefly to the question of whether there might be an 
  alternative source of instability of a free reed in a flow. An idealised 
  problem we mentioned at the beginning is that of a reed set in a partition 
  between two anechoic rooms. No model of the kind we have looked at can then 
  be formulated consistently: the pressure is simply constant (but different) 
  on the two faces of the reed and there is no predicted instability. However, 
  it is possible that there is some purely fluid-dynamical mechanism, not 
  relying on acoustic feedback, that can predict instability. 

  A suggestion along these lines has been advanced by Ricot, Caussé and 
  Misdariis [4]. They consider the flow pattern in the vicinity of a reed, but 
  their model involves non-trivial fluid dynamics and so in order to make 
  progress with the analysis they were forced to consider a highly idealised 
  geometry of the reed and its slot. When they solved their model numerically, 
  they did indeed find that it predicted oscillation of a closing reed. This 
  kind of model certainly deserves further research. 

  The success of Tarnopolsky, Fletcher and Lai [1] in matching experiment with 
  the acoustic feedback theory in an opening-reed configuration suggests that 
  when strong acoustic feedback is present, it can be dominant under at least 
  some circumstances. But it may well prove to be the case that the 
  fluid-dynamical mechanism produces a bias in favour of closing reeds, and 
  that this plays a significant role in the behaviour of real free-reed 
  instruments. 

  For our final topic in this section, we turn to a phenomenon that is surely 
  based on acoustic feedback: the ability of a harmonica player to “bend” a 
  note by manipulating their vocal tract. First, we need to describe the 
  effect. As we have already seen, each reed channel in a harmonica links to 
  two reeds, one opening inwards towards the channel, the other opening 
  outwards, away from it. In normal playing, the former reed is the blow note 
  and the latter the draw note: both operate in the closing-reed configuration. 

  But the player can do some surprising things, beyond this simple choice of 
  two notes depending on whether they blow or draw. Bending is the most basic 
  of these extended techniques. By doing something with their vocal tract 
  (tongue position, throat muscles and so on), a player can coax the higher of 
  the two notes in the channel to play progressively flat, with a limit that is 
  about a semitone above the pitch of the lower reed in the same channel. But 
  there is no corresponding possibility of making the lower note “bend” 
  upwards. Note-bending has been used by blues players for many years as a 
  regular and important part of their technical repertoire. More recently, some 
  virtuosic harmonica players (most famously Howard Levy) have refined their 
  control over the technique to such an extent that they can play all the 
  “missing” notes on a diatonic harmonica, to make it a fully chromatic 
  instrument. 

  There is some direct experimental evidence for what is going on during a 
  “bend” like this [5]. Unlikely as it may seem given the relatively high Q 
  factor of harmonica reeds, it turns out that both reeds in the channel are 
  involved. You start with a ``normal'' note based on the higher-tuned reed, 
  which is in a closing-reed configuration. The lower-tuned reed, which is in 
  an opening-reed configuration, starts to vibrate as the bend regime is 
  entered. As the note moves progressively flat, the vibration amplitude of 
  this opening reed grows, while the amplitude of the original closing reed 
  falls. 

  We can make a preliminary model of the bending effect using Millot’s model, 
  extended in a simple way to include the second reed in the channel. The 
  situation is sketched, very schematically, in Fig.\ 17. The reed channel is 
  shown as a rectangular box, with the two reeds attached. In the figure, the 
  upper reed opens outwards so it is the draw reed, while the lower one opens 
  inwards so it is the blow reed. The player shapes their lips around the 
  opening to the channel, and behind the lips is their mouth cavity. The player 
  can change the volume of this cavity by adjusting their tongue and other 
  aspects of the vocal tract. In reality the vocal tract is more complicated, 
  because there are other resonances associated with the shape of the duct, but 
  for now we ignore all that and see how far we can get with the simplest 
  model. The important parameter of this model will be the nominal Helmholtz 
  resonance frequency based on the mouth cavity as the closed volume, and the 
  reed channel as the “neck”: we derived the formula for that frequency back in 
  section 4.2.1. 

  \fig{figs/fig-6e7d7b84.png}{Figure 17. Schematic sketch of a harmonica reed 
  channel with its two reeds, and the attached mouth cavity of the player.} 

  Our simulations will be based on a typical harmonica reed pair: we have one 
  reed tuned to $A_4$ (440~Hz), while the other is a tone lower at $G_4$, 
  392~Hz. This exact combination can be found in the usual note arrangement of 
  a 10-hole diatonic harmonica in the key of G, in hole 4: $G_4$ is the blow 
  note and $A_4$ is the draw note. The next link gives some details of the 
  model, including the parameter values we will use. 

  We will shortly use the model to look at the pattern of “cold start” 
  transient responses in a plane parameterised by the blowing pressure and the 
  Helmholtz resonance frequency, in a similar vein to the “playability maps” we 
  have shown in earlier sections. But first, we can use the model to simulate a 
  typical action of a blues harmonica player. We are hoping to be able to 
  simulate a “draw bend” in which the higher note is started, and then a change 
  in the mouth cavity induces the pitch to drop. To simulate this, we can start 
  with a relatively large cavity volume, which gives a relatively low Helmholtz 
  resonance frequency, then progressively reduce the volume as the simulation 
  proceeds so that the Helmholtz resonance frequency passes through the 
  fundamental frequencies of the two reeds. You can listen to an example of 
  such a simulation in Sound 2. 

  The result doesn’t sound very much like a blues harmonica player, but it does 
  give at least a rough approximation to the phenomenon we are after. The sound 
  starts near the higher note, $A_4$. It then falls slowly but continuously for 
  a bit, then falls rather abruptly, then switches over to a steady lower note. 
  Figure 18 shows a portion of the spectrogram of this sound. It has been 
  zoomed in to show the frequency range around the 5th, 6th and 7th harmonics 
  of the played notes. The shape of the three ridges mirrors what has just been 
  said: a gradual fall in frequency (for which you have to look carefully in 
  the plot), then a more rapid drop, then a steady note. 

  \fig{figs/fig-ac94777d.png}{Figure 18. A portion of the spectrogram of Sound 
  2, showing harmonics 5, 6 and 7 of the synthesised harmonica pressure 
  waveform. It can be seen that the frequency moves downwards slowly at first, 
  then more rapidly, then finally settles into a steady pattern at a lower 
  frequency.} 

  Figures 19 and 20 show what the two reeds are doing. Figure 19 shows the tip 
  displacements of the two separate reeds (with the red plot shifted down for 
  clarity). Figure 20 shows the transition between the two reeds as a ratio: it 
  shows the amplitude of the $A_4$ reed (the draw reed) as a fraction of the 
  root-mean-square amplitude of the two reeds taken together. The curve starts 
  near the value 1, and falls continuously to end with values near 0.1. This is 
  very much the kind of behaviour found by Bahnson, Antaki and Beery [5]: a 
  gradual “handover” of activity between the two reeds as the bend proceeds. 

  \fig{figs/fig-bd9ffb21.png}{Figure 19. Tip displacements of the two reeds in 
  the synthesis of Sound 2. The result for the opening reed, in red, has been 
  shifted down by 10~mm for clarity.} 

  \fig{figs/fig-ae4b6fe1.png}{Figure 20. The tip displacement of reed 1 as a 
  fraction of the RMS displacement of the two reeds, showing the transition 
  during the course of the ``bend''.} 

  Zooming in on the waveforms of Fig.\ 19, it turns out that both reeds are 
  moving more or less sinusoidally, in opposite phase relative to the reed 
  channel. In other words, one moves in while the other moves out, so looking 
  at the layout in Fig.\ 17 the two reeds are moving in the same physical 
  direction at any given moment: both upwards or both downwards. Again, this 
  agrees with the observations of Bahnson, Antaki and Beery [5]. 

  Figure 21 shows the actual volume of the mouth cavity used in the simulation, 
  in order to access this range of Helmholtz resonance frequency. The values 
  seem a little on the small side for a human mouth cavity volume, but we 
  should probably not take the exact values too seriously, given the crudeness 
  of the “vocal tract” model we are using. We can take some comfort from the 
  fact that the values are of the same order of magnitude as those apparently 
  used by Bahnson, Antaki and Beery [5]: they converted a 60~cc syringe to make 
  a variable volume for their artificial blowing experiments to explore 
  bending. 

  \fig{figs/fig-803c1aa3.png}{Figure 21. The mouth cavity volume (expressed in 
  cubic centimetres, or ml) needed to produce the range of Helmholtz resonance 
  frequencies in the Sound 2 synthesis.} 

  Just as we have seen in earlier sections in the context of other wind 
  instruments and bowed strings, we can get a wider perspective on the 
  predictions of the harmonica model by using the simulation program to 
  generate a grid of examples covering a parameter plane of interest. We only 
  have two “player” parameters here, the blowing pressure and the Helmholtz 
  resonance frequency, so those define the plane we will study. 

  Figures 22 and 23 show this plane, for the two cases of blowing and drawing 
  at our single reed channel. These plots are colour-shaded to indicate the 
  playing frequency, normalised by the nominal frequency of the draw reed 
  (440~Hz). The bend illustrated in Sound 2 was generated by drawing, because 
  the draw reed was tuned to the higher pitch. Based on the experience of 
  harmonica players, we do not expect any corresponding bending behaviour to be 
  exhibited when blowing. That is exactly what the plots show. 

  \fig{figs/fig-5dd5ce88.png}{Figure 22. Map of the normalised playing 
  frequency in the plane of Helmholtz resonance frequency and mouth pressure, 
  for the case when the reed at lower frequency is a closing reed, and the one 
  at higher frequency is an opening reed. Frequency is normalised by the free 
  resonance frequency of the higher reed, 440~Hz.} 

  \fig{figs/fig-96eb5429.png}{Figure 23 Map corresponding to Fig. 22, for the 
  converse case when the reed at lower frequency is an opening reed, while the 
  one at higher frequency is a closing reed. The comparison of the two figures 
  would correspond to blowing and drawing on the same hole in the harmonica. 
  Frequency is normalised by the free resonance frequency of the higher reed, 
  440~Hz.} 

  Figure 22 shows red colour on the left, indicating a frequency close to the 
  blow-reed frequency. When the Helmholtz resonance frequency rises enough that 
  it comes within range of the blow-reed frequency (392~Hz), the note simply 
  stops. There is then black space, until at very high values of the Helmholtz 
  resonance frequency a note springs into life which is coloured yellow, 
  indicating that it is close to the draw reed frequency. The left-hand plot in 
  Fig.\ 24 shows a cross-section through this diagram at the blowing pressure 
  2~kPa, and that reveals that the new note is in fact significantly sharp 
  compared to the natural frequency of the draw reed. (The dotted semitone 
  lines indicate the nominal frequencies of an equal-tempered scale.) This is 
  probably illustrating a phenomenon known as an “over-blow” in the harmonica 
  world — a more advanced playing technique than note-bending, because it 
  requires more precise control from the player. 

  Figure 23 shows the corresponding plot to Fig.\ 22, but for drawing rather 
  than blowing. The colours are reversed: at low values of the Helmholtz 
  resonance frequency the played note is close in frequency to the draw reed, 
  as expected. After the transition, the colour in the right-hand half 
  indicates a frequency close to that of the blow reed. The right-hand panel of 
  Fig.\ 24 shows that actually the initial frequency was slightly below the 
  natural frequency of the draw reed, while the final frequency is slightly 
  above that of the blow reed (as you can see from the semitone lines). 

  The conspicuous difference between Figs.\ 22 and 23 is that Fig.\ 23 has no 
  black space in the centre. As you move across the diagram in a horizontal 
  line at most blowing pressures, there is a rather abrupt switch of pixel 
  colour, with no gap. The “bend” example in Sound 2 was more or less a 
  horizontal section across Fig.\ 23, at the blowing pressure 2~kPa. If you 
  look carefully at the colours in Fig.\ 23, you can just about see the 
  “bending” tendency: the yellow shades get darker as you approach the 
  transition to orange pixels. 

  Figure 25 shows a different view of the same sets of simulations as in Figs.\ 
  22 and 23: it is colour-shaded to indicate (roughly) the loudness of the 
  played note in each case. Colours indicate the peak-to-peak amplitude of the 
  internal pressure waveform at the end of each simulation run. The right-hand 
  panel of this figure highlights a feature you probably noticed in Sound 2: as 
  the “bend” went flat, it also got quieter, before jumping to a louder note 
  once it settled into the steady note based on the blow reed frequency. 

  In summary, this very simple model of a pair of harmonica reeds coupled to 
  the vocal tract of the player shows some promise. At least qualitatively, it 
  reproduces known features: the distinction between blowing and drawing on a 
  given reed channel, the bending phenomenon (occurring only in one direction — 
  in our example, drawing rather than blowing), and even the over-blowing 
  effect. To go further than this, what is probably needed is some measurements 
  of the input impedance at the reeds, as the player adjusts their vocal tract 
  in the way needed for bending. The actual impedance is no doubt more 
  complicated than the simple Helmholtz resonator model we have used here, and 
  players probably take advantage of other degrees of freedom when performing 
  advanced techniques. 

  As an aside, this study of note-bending in the harmonica has parallels in 
  other wind instruments: advanced players of many instruments have been shown 
  to make use of subtle adjustments to their vocal tract to improve their 
  control over pitch, dynamics and tone quality. For example, Fritz and Wolfe 
  have investigated these effects in the context of the clarinet [6], and there 
  are many other example of vocal tract interaction to be found on \tt{}Joe 
  Wolfe’s web site\rm{}. 



  \sectionreferences{}[1] A. Z. Tarnopolsky, N. H. Fletcher and J. C. S. Lai, 
  “Oscillating reed valves — an experimental study”, Journal of the Acoustical 
  Society of America \textbf{108}, 400—406 (2000). 

  [2] N. H. Fletcher: “Autonomous vibration of simple pressure-controlled 
  valves in gas flows”, Journal of the Acoustical Society of America 
  \textbf{93}, 2172—2180 (1993). 

  [3] L. Millot and Cl. Baumann, “A proposal for a minimal model for free 
  reeds”, Acta Acustica united with Acustica \textbf{93}, 122—144 (2007). 

  [4] Denis Ricot, René Caussé and Nicolas Misdariis, “Aerodynamic excitation 
  and sound production of blown-closed free reeds without acoustic coupling: 
  The example of the accordion reed”, Journal of the Acoustical Society of 
  America \textbf{117}, 2279—2290 (2005). 

  [5] Henry T. Bahnson, James F. Antaki and Quinter C. Beery, “Acoustical and 
  physical dynamics of the diatonic harmonica”, Journal of the Acoustical 
  Society of America \textbf{103}, 2134—2144 (1998) 

  [6] Claudia Fritz and Joe Wolfe, “How do clarinet players adjust the 
  resonances of their vocal tracts for different playing effects?”, Journal of 
  the Acoustical Society of America \textbf{118}, 3306—3315 (2005). 