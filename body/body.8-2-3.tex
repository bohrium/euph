  \tt{}Euphonics\rm{} 

  The science of musical instruments                           ISSN 2977-5612 

  Consider a mass-spring oscillator, whose motion is resisted by ideal Coulomb 
  friction. If the displacement of the mass is $x(t)$, there are three 
  different equations that might govern the next stage of motion, depending on 
  the value of the velocity $\dot{x}$. If $\dot{x}>0$, the mass is sliding 
  forwards so it is resisted by a friction force. But for Coulomb friction that 
  force is simply a constant, independent of the motion of the mass, so the 
  governing equation can be written in the form 

  $$\ddot{x}+p^2 x =-F\mathrm{~~~~~~~}(\dot{x}>0) \tag{1}$$ 

  where $p$ is the resonance frequency of the oscillator in the absence of the 
  friction force, and $F$ describes the constant friction force. The solution 
  of this equation must take the form 

  $$x=A \cos p(t-t_0) -- F/p^2\mathrm{~~~~~~~}(\dot{x}>0) \tag{2}$$ 

  for some values of the constants $A$ and $t_0$. If $\dot{x}<0$, the mass is 
  sliding the other way. The equation is the same, except that the friction 
  force is reversed: 

  $$\ddot{x}+p^2 x =F\mathrm{~~~~~~~}(\dot{x}<0). \tag{3}$$ 

  This time the solution must take the form 

  $$x=B \cos p(t-t_0) + F/p^2\mathrm{~~~~~~~}(\dot{x}<0) \tag{4}$$ 

  for some values of the constants $B$ and $t_0$. The third possibility is that 
  $\dot{x}=0$ so the mass is stationary. This is possible if the spring force 
  lies within the limits of sticking friction: $p^2 x$ must lie between $\pm 
  F$, so $x$ must lie between $\pm F/p^2$. 

  Suppose we start the motion with the mass at rest at $x=0$, and we apply a 
  positive initial velocity $V$ so that sliding commences with $\dot{x}>0$. The 
  solution follows eq. (2), with suitable values of $A$ and $t_0$ to satisfy 
  the initial conditions. The resulting sine wave is shown in the dashed line 
  in Fig.\ 1. Notice the asymmetric placement of the sine wave: it is centred 
  on $-F/p^2$, shown as the lower of the two dash-dot lines in the plot. But 
  the actual motion will only follow this while $\dot{x}>0$, so in fact the 
  only part that is relevant is the portion shown in red. 

  \fig{figs/fig-173474f6.png}{\caption{Figure 1. The first stage of vibration 
  with a friction damper}} 

  Now $\dot{x}<0$, so the motion of the mass now follows eq. (4) with suitable 
  values of the constants $B$ and $t_0$ so that it joins on to the red curve. 
  This time the sine wave, shown in Fig.\ 2, is centred on $F/p^2$, shown as 
  the upper of the two dash-dot lines. Again, this solution is only relevant 
  while the velocity remains negative, so the part we want from this sine wave 
  is the portion plotted in blue. 

  \fig{figs/fig-eeef1510.png}{\caption{Figure 2. The second stage of vibration 
  with a friction damper}} 

  After that, the pattern repeats. Each alternate half-cycle of sine wave is 
  centred around the upper or lower dash-dot line. Sooner or later, the point 
  of zero velocity at the end of one of these half-cycles of motion will lie 
  between those two lines, and at that point the motion stops. The mass simply 
  sits still thereafter, held in place by the friction force. The case plotted 
  in Figs.\ 1 and 2 is quite extreme, to make the pattern clear: the motion 
  will stop soon after the portion plotted in red and blue. The case shown in 
  Fig.\ 11 of section 8.2 had a lower friction force $F$, so that the motion 
  continued for more cycles and the linear envelope of decay was clearly 
  visible. 