  We now have a rather general description of the relation between the input 
  and output of a linear system. It is useful to put this into mathematical 
  form and to introduce two important concepts: we will need them later (in 
  section 6.5) when we start to talk about the sound made by a violin in 
  response to string motion created by the player. Think back to the ``black 
  box'' in Fig.\ 2 of section 2.1. We apply some input waveform $x(t)$, and we 
  obtain an output waveform $y(t)$. We can express both functions in terms of 
  their Fourier transforms, $X(\omega)$ and $Y(\omega)$. Now at each individual 
  frequency $\omega$ the input is converted into the output simply by 
  multiplication with the frequency response function of the box, which we can 
  call $G(\omega)$. In other words 

  $$Y(\omega) = G(\omega) X(\omega) . \tag{1}$$ 

  We would like to turn this into a direct relation between $x(t)$ and $y(t)$. 
  We will first do this mathematically, then we will see how to interpret the 
  result graphically and intuitively. 

  Before we begin, we need to introduce another important idea, the Dirac delta 
  function. Look at the family of pulse shapes shown in Fig.\ 1: they have a 
  width $h$ and a height $1/h$, so that they all have unit area underneath the 
  plot. As $h$ gets smaller, the pulse gets narrower and higher. The Dirac 
  delta function is the limiting case as $h\rightarrow 0$: it has an infinite 
  value for an infinitesimally short time, while still having unit area. This 
  is a mathematical idealisation of the kind of impulsive force we imagine 
  being applied to the toy drum by our drumstick, which had a very high value 
  for a very short time. 

  This delta function, written $\delta(t)$, has two important properties. The 
  first of them is this: 

  $$\int{f(t) \delta(t-T) dt} = f(T) \tag{2}$$ 

  where $f(t)$ is any function, and the limits of the integration must include 
  the value $t=T$, where the infinite peak of the delta function occurs. This 
  result is fairly obvious, if you just think about the sequence of shapes in 
  Fig.\ 1. The values of $f(t)$ do not matter except in the immediate vicinity 
  of $t=T$ because the delta function is zero everywhere else. So we might as 
  well replace $f(t)$ by the constant value $f(T)$. We could then take that 
  constant outside the integral, leaving just the delta function inside. But we 
  know the delta function integrates to the value 1 because it has been defined 
  that way. 

  The second important property concerns the Fourier transform of $\delta(t)$, 
  which we could call $D(\omega)$: 

  $$D(\omega)=\dfrac{1}{2 \pi} \int_{-\infty}^{\infty}{\delta(t) e^{-i \omega 
  t} dt} =\dfrac{1}{2 \pi} \tag{3}$$ 

  using eq. (2), and so, by the formula for the inverse Fourier transform, 

  $$2 \pi \delta(t)=\int_{\infty}^{\infty}{2 \pi D(\omega) e^{i \omega t} d 
  \omega} =\int_{-\infty}^{\infty}{e^{i \omega t} d \omega} . \tag{4}$$ 

  The delta function contains all frequencies, with equal amplitude. 

  Now the mathematical approach to eq. (1) is based on something called the 
  convolution theorem. Suppose we have any two functions $a(t)$ and $b(t)$, and 
  their corresponding Fourier transforms $A(\omega)$ and $B(\omega)$. Suppose a 
  third function $c(t)$ with Fourier transform $C(\omega)$ satisfies $C=AB$. 
  The theorem then states that 

  $$c(t)= \dfrac{1}{2 \pi} \int_{-\infty}^{\infty}{a(t-\tau) b(\tau) d \tau} 
  \tag{5}$$ 

  where the integral on the right-hand side is called a convolution integral. 
  To prove this theorem, we express this integral in terms of the Fourier 
  transforms $A$ and $B$: 

  $$\int_{-\infty}^{\infty}{a(t-\tau) b(\tau) d 
  \tau}=\int_{-\infty}^{\infty}{\int_{-\infty}^{\infty}{\int_{-\infty}^{\infty}{A(\alpha) 
  e^{i \alpha (t-\tau)} B(\beta) e^{i \beta \tau} d \beta} d \alpha } d \tau}$$ 

  $$=\int_{-\infty}^{\infty}{\int_{-\infty}^{\infty}{A(\alpha) B(\beta) e^{i 
  \alpha t} \int_{-\infty}^{\infty}{e^{i (\beta -- \alpha) \tau} d \tau} d 
  \beta} d \alpha }. \tag{6}$$ 

  But we can express the final integral in terms of a delta function using eq. 
  (4): 

  $$\int_{-\infty}^{\infty}{e^{i (\beta -- \alpha) \tau} d \tau} = 2 \pi \delta 
  (\beta -- \alpha) \tag{7}$$ 

  Substituting into eq. (6), we can now do the $\beta$ integration trivially 
  using eq. (2), and so 

  $$\int_{-\infty}^{\infty}{a(t-\tau) b(\tau) d \tau}=2 
  \pi\int_{-\infty}^{\infty}{A(\alpha) B(\alpha) e^{i \alpha t} d \alpha}$$ 

  $$=2 \pi \int_{-\infty}^{\infty}{C(\alpha) e^{i \alpha t} d \alpha} =2 \pi 
  c(t) \tag{8}$$ 

  which is the convolution theorem. 

  We can apply this theorem directly to eq. (1), in terms of the inverse 
  Fourier transform of $G(\omega)$. For reasons that will rapidly become 
  apparent, it is convenient to call this inverse transform $2 \pi g(t)$. The 
  result is then 

  $$y(t)= \int_{-\infty}^{\infty}{g(t-\tau) x(\tau) d \tau} . \tag{9}$$ 

  The function $g(t)$ has a simple physical interpretation in terms of our 
  input-output system. From eq. (1), if the input is a delta function then the 
  output would have $Y(\omega)=G(\omega)/2 \pi$ and the output waveform would 
  thus be $y(t)=g(t)$. Since a delta function is an idealised impulse, this is 
  called the impulse response of the system. Equation (9) thus says that in the 
  general case, the output is the convolution of the input $x(t)$ with the 
  impulse response of the system. 

  For a simple illustration, we can calculate the impulse response of the 
  mass-spring oscillator analysed in section 2.2.2. The mass is initially at 
  rest, then at time $t=0$ the delta function ``arrives''. It is a unit 
  impulse, so it gives the mass a unit change in momentum by Newton's law. So 
  immediately after $t=0$, the mass still has zero displacement but its 
  velocity has jumped to $1/m$. After that there is no applied force because 
  the delta function returns to zero, so the oscillator executes free vibration 
  at its natural frequency $\Omega=\sqrt{k/m}$, with a suitable amplitude and 
  phase to match these initial conditions. The displacement for $t \ge 0$, 
  which is the impulse response, is thus 

  $$g(t) = \dfrac{1}{m \Omega} \sin \Omega t= \dfrac{1}{\sqrt{k m}} \sin \Omega 
  t . \tag{10}$$ 

  There is a simple way to understand the meaning of the convolution integral 
  eq. (9). For the sake of graphical clarity, we will not illustrate using the 
  impulse response from eq. (10). Instead, we will suppose that our linear 
  system has an impulse response which is a simple exponential decay: 
  $g(t)=e^{-t}$. Now suppose that the input to the system is the function 
  $x(t)$ plotted at the top of Fig.\ 2. Slice this function up into thin strips 
  with width $\Delta T$, as indicated in the plot. Each strip, taken by itself, 
  gives a narrow pulse of input. The sum of all these pulses is the whole of 
  the function $x(t)$. 

  The strip near a time $t=T$ has an area $x(T) \Delta T$, so this strip 
  considered on its own is roughly the same as an impulse $x(T) \delta(t-T) 
  \Delta T$. This impulse, on its own, would produce output from the system 
  $x(T) \Delta T g(t-T)$ by definition of the impulse response. For our sample 
  problem these separate contributions are all plotted, one after another, down 
  the page. Choose a strip in the figure, and run you eye vertically downwards: 
  you will find one with a jump aligned with the centre of your strip, and it 
  will have a magnitude (positive or negative depending which strip you choose) 
  proportional to the height at the centre of your strip. 

  Since the input $x(t)$ is the sum of the strips, the total output $y(t)$ is 
  given, approximately, by the sum of all these scaled and delayed impulse 
  responses. In other words 

  $$y(t) \approx \sum_{slices}{x(T) g(t-T) \Delta T} . \tag{11}$$ 

  The result of adding up this sum for the finite strips in Fig.\ 2 is plotted 
  in Fig.\ 3. 

  Now we let the width of the strips get narrower and narrower, so that they 
  approximate more and more closely to the actual function $x(t)$. In the limit 
  the sum becomes an integral and the approximation becomes exact: 

  $$y(t)=\int_{-\infty}^{t}{x(T) g(t-T) dT} , \tag{12}$$ 

  which is exactly the convolution integral of eq. (9). The result for the 
  sample problem is plotted in Fig.\ 4. 