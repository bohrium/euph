  Material damping associated with small-amplitude vibration can be analysed by 
  making use of a result known, rather grandly, as the correspondence principle 
  of linear viscoelasticity (see for example the textbook by Bland [1]. This 
  principle can be described as follows. For any harmonic response problem, at 
  a frequency $\omega$, first solve the undamped problem. Now look to see where 
  elastic moduli enter the solution (such as Young's modulus $E$, or the plate 
  stiffnesses $D_1$--$D_4$ introduced in section 10.3.2). The corresponding 
  damped problem is solved simply by replacing them with suitable complex 
  values. So, for example, you would replace $E$ with $E(1+i \eta_E)$. 
  Naturally enough, such things are called ``complex moduli''. We have already 
  met this idea briefly, back in section 5.4.4 when we were talking about 
  damping in strings, but now we will use it more systematically. 

  The result is that the resonance frequencies predicted by the analysis become 
  complex numbers. So for example free vibration which for the undamped problem 
  varied like $e^{i \omega t}$ might now vary like 

  $$e^{i(\omega + i \Delta)t} = e^{i \omega t} e^{-\Delta t} \tag{1}$$ 

  It is clear that this has achieved the desired effect: the oscillation at 
  frequency $\omega$ now decays exponentially at a rate determined by the 
  imaginary part of the frequency, $\Delta$. 

  If material damping is small, as it usually is in vibration problems, we can 
  use Rayleigh's principle together with the viscoelastic correspondence 
  principle to find out how the damping varies from mode to mode of a system. 
  Suppose that we know the expressions for the kinetic and potential energies 
  of the system (without damping). Elastic moduli will usually come into the 
  potential energy, but not into the kinetic energy. Either analytically or, 
  more likely, numerically we calculate the first few modes of the undamped 
  system. Now: 

  (i)	The correspondence principle says that to solve the damped problem we 
  replace elastic moduli in the expression for potential energy with complex 
  moduli. For small damping these will only have small imaginary parts. 

  (ii)	Rayleigh's principle says that given an approximation to a mode shape we 
  can get a rather good approximation to its natural frequency by evaluating 
  the Rayleigh quotient. The modes of the damped system will be slightly 
  different from the modes of the undamped system, but the undamped mode shapes 
  will still give a good approximation. So we evaluate the Rayleigh quotient 
  using the true expression for the potential energy, with complex moduli, but 
  with the approximate expression for mode shape from the undamped calculation. 
  This gives a good approximation to the complex natural frequency, and hence 
  to the modal damping factor. 

  For a first example of applying this idea, think about beam vibration. We 
  already know from section 3.3.1 that for a bending beam with displacement 
  $w(x,t)$, bending stiffness $EI$ and mass per unit length $m$, the potential 
  energy is 

  $$ V = \dfrac{1}{2} \int{EI \left(\dfrac{\partial^2 w}{\partial x^2} 
  \right)^2 dx} \tag{2}$$ 

  and the kinetic energy is 

  $$T=\dfrac{1}{2} \int{ m \left(\dfrac{\partial w}{\partial t} \right)^2 dx } 
  . \tag{3}$$ 

  The only elastic modulus entering here is Young's modulus $E$. With material 
  damping we can replace this with $E(1+i \eta_E)$. The factor $\eta_E$ may 
  vary with frequency, but for the purposes of this approximate calculation we 
  can evaluate it at the undamped natural frequency $\omega_n$ corresponding to 
  mode shape $w_n(x)$. The Rayleigh quotient, from equations (2) and (3), gives 

  $$\omega^2 \approx \dfrac{E(1+i \eta_E) I \int{EI \left(\dfrac{\partial^2 
  w_n}{\partial x^2} \right)^2 dx}}{m \int{w_n^2 dx}}=(1+i \eta_E) \omega_n^2 
  \tag{4}$$ 

  because this expression apart from the factor $(1+i \eta_E)$ is the Rayleigh 
  quotient for the undamped problem, which is equal to $\omega_n^2$. 

  So the time dependence of a free vibration in this mode is 

  $$e^{i \omega_n t} e^{-\eta_E \omega_n t/2} . \tag{5}$$ 

  From the definition of Q-factor, the value for this mode is 

  $$Q_n \approx \dfrac{1}{\eta_E} . \tag{6}$$ 

  All modes of this beam will have the same Q-factor, except that the material 
  property $\eta_E$ may vary with frequency, but such variation is usually only 
  slow. If you measure damping as well as frequency in a beam test like the 
  ones described in section 10.3.1, that will immediately give the value of 
  $\eta_E$ for the direction aligned with your beam. 

  Now we can apply the approach to the damping of a wooden plate. We start from 
  the expressions for potential and kinetic energy, from section 10.3.2: 

  $$ V = \dfrac{1}{2} \int{\int{h^3 \left[ D_1 \left(\dfrac{\partial^2 
  w}{\partial x^2} \right)^2 + D_2 \dfrac{\partial^2 w}{\partial x^2} 
  \dfrac{\partial^2 w}{\partial y^2} \right. }} $$ 

  $$ \left. + D_3 \left(\dfrac{\partial^2 w}{\partial y^2} \right)^2 + D_4 
  \left(\dfrac{\partial^2 w}{\partial x \partial y} \right)^2 \right] dx dy 
  \tag{7}$$ 

  and 

  $$T=\dfrac{1}{2} \int{\int{ \rho h w^2 dx dy }} \tag{8}$$ 

  where $w(x,y)$ is the displacement of the plate, $h$ is the thickness, 
  $D_1$--$D_4$ are the four stiffness constants and $\rho$ is the density. 

  Once we introduce damping, all four of the stiffness constants become 
  complex: 

  $$D_j \rightarrow D_j(1+i \eta_j) \mathrm{,~~~~} j=1,2,3,4 . \tag{9}$$ 

  Now following through the same argument based on Rayleigh's principle, we 
  deduce that for mode $n$ the modal damping factor, the inverse of the modal 
  Q-factor, is a simple weighted sum of the four $\eta_j$: 

  $$\dfrac{1}{Q_n} \approx J_1 \eta_1 + J_2 \eta_2 + J_3 \eta_3 + J_4 \eta_4 
  \tag{10}$$ 

  where the dimensionless constants $J_1$--$J_4$ are defined as follows: 

  $$J_1 = \dfrac{D_1 \int{\int{h^3 \left(\dfrac{\partial^2 w_n}{\partial x^2} 
  \right)^2 dx dy }}}{\omega_n^2 \int{ \int{ \rho h w_n^2 dx dy }}} \tag{11}$$ 

  $$J_2 = \dfrac{D_1 \int{\int{h^3 \dfrac{\partial^2 w_n}{\partial x^2} 
  \dfrac{\partial^2 w_n}{\partial y^2} dx dy }}}{\omega_n^2 \int{ \int{ \rho h 
  w_n^2 dx dy }}} \tag{12}$$ 

  $$J_3 = \dfrac{D_1 \int{\int{h^3 \left(\dfrac{\partial^2 w_n}{\partial y^2} 
  \right)^2 dx dy }}}{\omega_n^2 \int{ \int{ \rho h w_n^2 dx dy }}} \tag{13}$$ 

  $$J_4 = \dfrac{D_1 \int{\int{h^3 \left(\dfrac{\partial^2 w_n}{\partial x 
  \partial y} \right)^2 dx dy }}}{\omega_n^2 \int{ \int{ \rho h w_n^2 dx dy }}} 
  . \tag{14}$$ 

  It follows immediately from the Rayleigh quotient for the original, undamped 
  plate that 

  $$J_1 + J_2 + J_3 + J_4 = 1 . \tag{15}$$ 

  These constants capture the partitioning of potential energy, and hence 
  energy dissipation rate, between the four terms associated with $D_1$, $D_2$, 
  $D_3$ and $D_4$ in equation (7). 

  In terms of measurement, the position is very similar to the previous 
  discussion of stiffness. With beam samples cut along and across the grain you 
  can determined the loss factors associated with the two Young's moduli. With 
  plate samples, the same three modes which gave simple estimates of $D_1$, 
  $D_3$ and $D_4$ now give direct estimates of $\eta_1$, $\eta_3$ and $\eta_4$. 
  The reason is that for each of those modes, one of the $J_j$ constants is 
  approximately equal to 1, while the other three are approximately zero. 
  Equation (10) then says that the measured loss factor (or inverse Q-factor) 
  is essentially equal to the corresponding $\eta_j$. We found before that 
  $D_2$ didn't play a very strong role in determining frequencies: the 
  equivalent result is that $\eta_2$ seems to have very little influence on 
  modal Q-factors, to the extent that it is usually not possible to determine a 
  convincing value from measurements. 

  It is useful to see some measured values of $\eta_1$, $\eta_3$ and $\eta_4$. 
  We can show results for the same two spruce plates whose stiffness values 
  were quoted in section 10.3.2. For the quarter-cut plate, the values were 
  $\eta_1 = 0.0051$, $\eta_3 = 0.0216$ and $\eta_4 = 0.0164$. The corresponding 
  results for the plate with a ring angle close to $45^\circ$ were $\eta_1 = 
  0.0074$, $\eta_3 = 0.0212$ and $\eta_4 = 0.0139$. Note that for both plates 
  the long-grain loss factor $\eta_1$ was significantly lower than the 
  cross-grain loss factor $\eta_3$: by about a factor of 4 for the quarter-cut 
  plate. This is in line with a very general trend across a wide range of 
  materials: there is a negative correlation of loss factor with stiffness. 
  Stiff materials like ceramics or hardened steel tend to have low damping, 
  soft materials like lead or rubber tend to have high damping. 

  Because $\eta_1$, $\eta_3$ and $\eta_4$ are significantly different, we can 
  expect mode-to-mode variations in modal damping factors, governed by the 
  values of $J_1$, $J_3$ and $J_4$. As an example, the values given above for 
  the quarter-cut spruce plate were based on measuring the Q-factors of 8 
  modes. These varied from 40 to 140, a range big enough to make very 
  significant differences in sound. Only 3 modes were needed to determine the 
  values of $\eta_1$, $\eta_3$ and $\eta_4$, and remaining 5 measured Q-factors 
  could be used to cross-check the theory on which this method is based. The 
  detailed variation from mode to mode was reproduced to very satisfactory 
  accuracy: see [2] for details. 

  Measuring values like the ones just quoted, for lightweight materials like 
  spruce, is quite challenging. The reasons were discussed in section 10.3: it 
  is difficult to support the test sample with adding significant extra 
  damping, and it is also hard to fix any kind of sensor to measure the 
  response without also adding extra damping. Furthermore, a sensor of some 
  kind is needed here: we cannot use Chladni patterns to measure damping, we 
  need some kind of quantitative response measurement in order to determine 
  decay rates or half-power bandwidths in FFT results. A description of how the 
  reported measurements were done is given in [2]. We will return to these 
  issues in section 10.4, when we talk about measuring frequency response 
  functions. 

  \sectionreferences{}[1] D. R. Bland, “The theory of linear viscoelasticity”, 
  Pergamon Press (1960). 

  [2] M. E McIntyre and J. Woodhouse, “On measuring the elastic and damping 
  constants of orthotropic sheet materials”, Acta Metallurgica \textbf{36}, 
  1397—1416 (1988). 