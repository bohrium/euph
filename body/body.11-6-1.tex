  Fletcher gave a rather general analysis of the linearised instability 
  threshold of a nonlinear valve subjected to a mean flow, with acoustical 
  feedback on one or both of its faces [1]. In a later paper [2], he then 
  applied this approach to the specific case of a free reed attached to a rigid 
  chamber of volume $V$, as sketched in Fig.\ 1. The predictions were compared 
  with the results of careful experiments, giving generally quite good 
  agreement --- some examples were shown in section 11.6. 

  \fig{figs/fig-82449765.png}{\caption{Figure 1. Sketch of the reed and its 
  chamber. An opening-reed configuration is depicted here, as in the 
  experimental results of Tarnopolsky, Fletcher and Lai [2], but the theory can 
  be applied equally well to a closing-reed configuration.}} 

  Fletcher's theoretical model is related closely to the models we have used 
  earlier for reed woodwind instruments and for brass instruments. It is based 
  on three equations. We first need to define some notation. There is a steady 
  inflow of air to the chamber, with volume flow rate $U_0$. The pressure 
  inside the chamber is $p_0+p(t)$, where $p_0$ is the steady atmospheric 
  pressure. It is assumed to be uniform throughout the volume, in a similar way 
  to our discussion of the Helmholtz resonator back in section 4.2.1. The 
  volume flow rate of air past the reed is $u(t)$, and it is assumed that the 
  pressure outside is simply atmospheric. 

  Exactly as we did before, we model the reed as a damped harmonic oscillator 
  representing the fundamental bending mode. It is assumed to have a natural 
  frequency $\omega_r$ and a Q-factor $Q_r$, and the tip displacement of the 
  reed is denoted $x(t)$. The reed has length $L$ and width $w$, and it has 
  effective modal mass $m$ per unit area. The area through which the flow 
  $u(t)$ is squeezed for a given position of the vibrating reed is denoted 
  $F(x)$ --- we will say something about how this area function can be 
  calculated a little later. 

  The first of our model equations is the Bernoulli expression relating to the 
  flow past the reed: 

  \begin{equation*}p \approx \dfrac{\rho_0 u^2}{2 C^2 F(x)^2} . 
  \tag{1}\end{equation*} 

  This is identical to the expression we have used before, except for the 
  inclusion of a vena contracta coefficient $C$, for which Tarnolpolsky et al. 
  suggest the value 0.61 for the jet geometry of interest here. The second 
  equation describes the reed dynamics: 

  \begin{equation*}\ddot{x}+\dfrac{\omega_r}{Q_r}\dot{x}+\omega_r^2 x = K_p p 
  \tag{2}\end{equation*} 

  \noindent{}where dots denote time derivatives and where $K_p =1.5 w L/m$: the 
  value 1.5 here is suggested by Tarnopolsky et al. based on the mode shape of 
  the lowest cantilever mode shape of the reed. The third equation relates the 
  rate of change of pressure inside the chamber to the net volume flow rate of 
  air into it: 

  \begin{equation*}\dot{p}=\dfrac{\rho_0 c^2}{V}[U_0 -- u -- K_x \dot{x}] 
  \tag{3}\end{equation*} 

  \noindent{}where $\rho_0$ is the density of air and $c$ is the speed of 
  sound. This is the same equation we used for the Helmholtz resonator in 
  section 4.2.1, except for the final term in the square brackets. This 
  describes the component of volume flow generated directly by the motion of 
  the reed. It involves a constant $K_x=0.4wL$, where the suggested numerical 
  value 0.4 is again based on the cantilever mode shape. 

  For the purposes of a linearised stability analysis, we now assume that 

  \begin{equation*}u=\bar{u}+u' e^{i \omega t} \mathrm{,~~} p=\bar{p}+p' e^{i 
  \omega t} \mathrm{,~~} x=\bar{x}+x' e^{i \omega t} \tag{4}\end{equation*} 

  \noindent{}where for example $\bar{p}$ is the mean value and $p'$ is the 
  (complex) amplitude of a small harmonic fluctuation at frequency $\omega$. We 
  substitute these expressions into the three equations, linearise with respect 
  to all the primed quantities where necessary, and then equate separately the 
  steady terms and terms involving $e^{i \omega t}$. Equations (2) and (3) are 
  both linear, so this involves very little effort. The results from equation 
  (2) are 

  \begin{equation*}\omega_r^2 \bar{x}=K_p \bar{p} \tag{5}\end{equation*} 

  \noindent{}and 

  \begin{equation*}\left[ -\omega^2 + i \omega \dfrac{\omega_r}{Q_r} + 
  \omega_r^2 \right]x' = K_p p' . \tag{6}\end{equation*} 

  Corresponding results from equation (3) are 

  \begin{equation*}\bar{u}=U_0 \tag{7}\end{equation*} 

  \noindent{}and 

  \begin{equation*}i \omega p' = -- \dfrac{\rho_0 c^2}{V} [u'+i \omega K_x x'] 
  . \tag{8}\end{equation*} 

  For equation (1) we need to work a little harder. First, we take the first 
  two terms of a Taylor expansion of $F(x)$: 

  \begin{equation*}F(x) \approx F(\bar{x}) + x' e^{i \omega t} 
  \dfrac{dF}{dx}(\bar{x})=\bar{F}+\bar{F}' x' e^{i \omega 
  t}\tag{9}\end{equation*} 

  \noindent{}where for brevity we define $\bar{F}=F(\bar{x})$ and 
  $\bar{F}'=dF(\bar{x})/dx$. 

  So now equation (1) gives 

  \begin{equation*}\bar{p}+p'e^{i \omega t}\approx \dfrac{\rho_0}{2C^2}\left( 
  \bar{u}+u'e^{i \omega t} \right)^2\left(\bar{F}+\bar{F}' x' e^{i \omega t} 
  \right)^{-2}\end{equation*} 

  \begin{equation*}\approx \dfrac{\rho_0}{2C^2 \bar{F}^2}\left( U_0^2 + 2 U_0 
  u' e^{i \omega t} \right) \left(1-2\dfrac{\bar{F}'}{\bar{F}} x' e^{i \omega 
  t} \right)\end{equation*} 

  \begin{equation*}\approx \dfrac{\rho_0}{2C^2 \bar{F}^2}\left( U_0^2 +2 U_0 u' 
  e^{i \omega t} -- 2U_0^2 \dfrac{\bar{F}'}{\bar{F}} x' e^{i \omega t} 
  \right)\tag{10}\end{equation*} 

  \noindent{}and so 

  \begin{equation*}\bar{p} =\dfrac{\rho_0 U_0^2}{2C^2 
  \bar{F}^2}\tag{11}\end{equation*} 

  \noindent{}and 

  \begin{equation*}p'=\dfrac{\rho_0 U_0}{C^2 \bar{F}^2}\left(u'-U_0 
  \dfrac{\bar{F}'}{\bar{F}}x' \right). \tag{12}\end{equation*} 

  We are interested in thresholds for pressure, so $\bar{p}$ is the variable of 
  interest. Equation (5) gives the static displacement of the reed tip in 
  response to that mean pressure, then equation (11) determines the input flow 
  rate $U_0$ necessary to create the desired pressure. 

  We now eliminate $u'$ between equations (8) and (12) to obtain 

  \begin{equation*}-\dfrac{i \omega V}{\rho_0 c^2}p' -- i \omega K_x x' = U_0 
  \dfrac{\bar{F}'}{\bar{F}} x' + \dfrac{U_0}{2 \bar{p}}p' 
  \tag{13}\end{equation*} 

  \noindent{}so that 

  \begin{equation*}(i \omega A +B)p'=-(i \omega D + E)x' 
  \tag{14}\end{equation*} 

  \noindent{}with 

  \begin{equation*}A=\dfrac{V}{\rho_0 c^2} \mathrm{,~~} B=\dfrac{C^2 
  \bar{F}^2}{\rho_0 U_0} \mathrm{,~~} D=K_x \mathrm{,~~} E=U_0 
  \dfrac{\bar{F}'}{\bar{F}} . \tag{15}\end{equation*} 

  Now we can substitute in equation (6): 

  \begin{equation*}\left[ -\omega^2 + i \omega \dfrac{\omega_r}{Q_r} + 
  \omega_r^2 \right]x' = -K_p \dfrac{i \omega D + E}{i \omega A + B} 
  x'\end{equation*} 

  \begin{equation*}=-K_p \dfrac{(i \omega D + E)(-i \omega A + B)}{\omega^2 A^2 
  + B^2} x' . \tag{16}\end{equation*} 

  In order for an instability to occur, the imaginary part of the right-hand 
  side expression must at least compensate for the damping term on the 
  left-hand side, so the condition for instability is 

  \begin{equation*}\dfrac{\omega_r}{Q_r} < K_p\dfrac{EA-DB}{\omega_r^2 A^2 + 
  B^2} \tag{17}\end{equation*} 

  \noindent{}where we have substituted $\omega_r$ for $\omega$ in the 
  denominator because we always expect our free reed to vibrate close to its 
  natural frequency. The constants $A$, $B$ and $D$ are always positive, but 
  $E$ depends on the sign of $\bar{F}'$. So we certainly need $\bar{F}'>0$ to 
  have any chance of satisfying the instability condition --- in other words, 
  it must behave as an ``opening reed'' in the immediate vicinity of the 
  equilibrium displacement $\bar{x}$. 

  To calculate a reasonable approximation to the area function $F(x)$, we can 
  follow a similar approach to previous authors [2,3]. For a given displaced 
  position of the reed, we can think of connecting the edge of the reed, all 
  the way round, to the nearest point on the slot in the reed plate, with a 
  kind of ``curtain''. The area of that curtain is what we want: it is the 
  minimum area that can be used to block any flow past the reed. 

  This calculation could be done using the cantilever mode shape, but for the 
  work reported here a simpler approach has been used, in which the reed is 
  envisaged as being rigid and flat, hinged to the base plate along one edge. 
  Three cases have to be considered: the reed lying entirely outside the slot, 
  the reed tip lying within the thickness of the slot in the base plate, and 
  the reed tip having emerged on the other side. For each case it is 
  straightforward to compute the area function, armed with the reed dimensions 
  together with the thickness of the base plate, the stand-off distance of the 
  ``hinge'' above the plate and the clearance around the edges. Some examples 
  for different geometrical configurations were shown in section 11.6. 

  We should mention a small complicating factor. Fletcher suggests that another 
  term could be included in equation (1), to take some account of the fact that 
  the air flow past the reed is not really a steady flow. He argues [2] that a 
  simple application of the non-steady version of Bernoulli's equation would 
  lead to 

  \begin{equation*}p \approx \dfrac{\rho_0 u^2}{2 C^2 F(x)^2} + \dfrac{d}{dt} 
  \left[ \dfrac{\rho_0 u \delta}{C F(x)} \right] \tag{18}\end{equation*} 

  \noindent{}in place of equation (1), where $\delta$ is an estimate of the 
  length over which the jet past the reed remains laminar: we might expect it 
  to be of the order of a few millimetres. Following through the same 
  linearisation procedure, one extra complex factor appears. The simplest way 
  to summarise the result is to say that everything follows identically if we 
  replace $C^2$ by 

  \begin{equation*}\dfrac{C^2}{1+i \omega \delta C \bar{F}/U_0}. 
  \tag{19}\end{equation*} 

  This complicates the algebra somewhat, but when the effect of making this 
  change is tested numerically for the geometry of any of the reeds we will be 
  concerned with, the effect turns out to be very slight. Our earlier neglect 
  of the extra term seems to be justified. 

  The stability analysis described above is specific to the situation sketched 
  in Fig.\ 1, but we could note that Fletcher's original version [1] was 
  couched in more general terms. This involves keeping equations (6) and (12), 
  but replacing equation (8) with the more general version 

  \begin{equation*}p'=-Z(\omega) \left(u'+i \omega K_x x' \right) 
  \tag{20}\end{equation*} 

  \noindent{}where $Z(\omega)$ is the impedance presented to one face of the 
  reed. The negative sign arises because the impedance we would measure would 
  involve positive volume flow into the acoustic system, whereas we have 
  defined our flow rate positive into the reed. We can recover the previous 
  case by recalling that the impedance of a volume $V$ is 

  \begin{equation*}Z_{volume}=\dfrac{\rho_0 c^2}{i \omega V}. 
  \tag{21}\end{equation*} 

  In this more general framework, the condition for instability becomes 

  \begin{equation*}\dfrac{\omega_r^2}{Q_r} < \mathrm{Im}\left[\dfrac{K_p(U_0 
  \bar{F}'/\bar{F} + i \omega K_x)}{-Y(\omega)-C^2 \bar{F}^2/\rho_0 U_0}\right] 
  \tag{22}\end{equation*} 

  \noindent{}where $\mathrm{Im}$ denotes the imaginary part, and $Y=1/Z$ is the 
  admittance of the acoustic system. Again, the expression could all be 
  evaluated at $\omega=\omega_r$ for an explicit approximation. Using the 
  notation from equation (16), for instability we require 

  \begin{equation*}\dfrac{\omega_r^2}{Q_r} <-K_p \mathrm{Im} \left[ \dfrac{i 
  \omega D + E}{Y +B}\right]=-K_p\mathrm{Im} \left[\dfrac{(i \omega D + 
  E)(B+Y^*)}{|Y|^2+B^2}\right]\end{equation*} 

  \begin{equation*}=-K_p\left[\dfrac{(\omega D (B+\mathrm{Re} Y) -- E 
  \mathrm{Im} Y}{|Y|^2+B^2}\right] . \tag{23}\end{equation*} 

  Certainly for this condition to be satisfied, the final expression must be 
  positive. The constants $B$ and $D$ are positive, so the first term in the 
  numerator always acts against the requirement. Because $Y$ is a driving-point 
  admittance, $\mathrm{Re} Y > 0$ and so lossiness in the acoustical system 
  always tends to discourage instability. For instability we always require 
  $E\mathrm{Im} Y > 0$: so a closing reed with $E<0$ needs $\mathrm{Im} Y<0$ 
  while an opening reed with $E>0$ needs $\mathrm{Im} Y>0$. In terms of 
  impedance, these sign conditions are reversed: $Y=1/Z=Z^*/|Z|^2$ so 
  $\mathrm{Im} Y$ has the opposite sign to $\mathrm{Im} Z$. 

  \sectionreferences{}[1] N. H. Fletcher: “Autonomous vibration of simple 
  pressure-controlled valves in gas flows”, Journal of the Acoustical Society 
  of America \textbf{93}, 2172—2180 (1993). 

  [2] A. Z. Tarnopolsky, N. H. Fletcher and J. C. S. Lai, “Oscillating reed 
  valves — an experimental study”, Journal of the Acoustical Society of America 
  \textbf{108}, 400—406 (2000). 

  [3] L. Millot and Cl. Baumann, “A proposal for a minimal model for free 
  reeds”, Acta Acustica united with Acustica \textbf{93}, 122—144 (2007). 