  We have said a little about wind instruments in earlier chapters, but now it 
  is time to go into more detail. It is useful to start by comparing the wind 
  instruments with the bowed-string instruments. Both types of instrument are 
  capable of producing sustained tones, and this automatically means that they 
  must all involve nonlinearity in some essential way (see section 8.1). In the 
  case of a bowed string, the main source of nonlinearity was from the 
  stick-slip frictional interaction between bow-hair and string. In the wind 
  instruments, the main nonlinearity is associated with things going on near 
  the mouthpiece end, which we will discuss in detail shortly. 

  An obvious difference between stringed instruments and wind instruments is 
  that wind instruments present the player with quite different control 
  variables: things like blowing pressure and details of what to do with the 
  lips, teeth and mouth cavity (generically called ``embouchure''), rather than 
  bow speed, force and position. We will try to develop ``playability 
  diagrams'' for wind instruments involving these control variables, somewhat 
  analogous to the Schelleng and Guettler diagrams for a bowed string (see 
  sections 9.3 and 9.5). 

  Wind instruments usually have no equivalent of the stringed instrument body, 
  which influences the sound of the instrument by imposing an extra stage of 
  filtering between the string motion and the radiated sound. In a stringed 
  instrument, it is the vibrating soundboard which generates sound in the 
  surrounding air, because the strings themselves are too small to be good 
  sound sources. In most wind instruments, the sound is radiated directly by 
  air motion: pressure variations inside the tube push air in and out of open 
  tone-holes, or through the bell of a brass instrument. But in another sense 
  the tube is the analogue of a string: it provides a set of resonances, often 
  approximately harmonically related, which interact with the nonlinearity to 
  produce the pitch and waveform of the played note. 

  There are two other important differences between bowed strings and wind 
  instruments. First, the resonances of air in a tube are usually much more 
  highly damped than the resonances of a string. A violin string might have Q 
  factors of the order of thousands, whereas tube resonances are more likely to 
  have Q factors of the order of hundreds or less. The second thing is that the 
  nonlinearity associated with friction is (as we saw back in Chapter 9) a very 
  severe one, whereas wind instrument nonlinearities tend to be less vigorous. 

  There is an important distinction between “smooth” and “non-smooth” 
  nonlinearities (see section 8.2), which suggests that rather different 
  mathematical approaches might be called for in the two cases. The big 
  disparity in Q factors also pushes us in a similar direction. The result is 
  that the scientific literature relating to wind instruments places a lot of 
  emphasis on periodic regimes of oscillation, thresholds, and bifurcations 
  between regimes. With the bowed string, we found earlier (in Chapter 9) that 
  there was no useful concept of a threshold of vibration, where the motion is 
  “almost sinusoidal”: a bowed string goes into strongly-nonlinear stick-slip 
  vibration straight away, and theoretical treatment tends to rely more heavily 
  on ``brute force'' numerical simulation. But for many wind instruments there 
  is a clear-cut threshold of blowing pressure or air speed, when the note 
  ``springs to life''. 

  \samsection{A. Names and categories} 

  The next step is to sort the enormous diversity of wind instruments into a 
  few general categories. If you look at the wind instruments in a western 
  symphony orchestra or wind band (see Fig.\ 1 for an example), a three-way 
  classification seems self-evident. There are the reed instruments (clarinet, 
  oboe, bassoon), the flute-like instruments (flute, piccolo) and the brass 
  instruments (trumpet, trombone, French horn). But if we cast our net a little 
  wider than the symphony orchestra, we soon find out that things are more 
  complicated and less clear. 

  For a start, by focussing on orchestral instruments we are missing one 
  category of instrument entirely: the “free reed” instruments, like the 
  harmonium, accordion or harmonica. This gives a four-way classification, 
  which will turn out to allow us to pigeon-hole virtually all wind instruments 
  (including the human voice). However, the categories are not always as 
  distinct as you might think. It is important to be clear what the basis of 
  the classification is. One thing that definitely does not influence the 
  classification of instruments is the material they are made of. Despite the 
  names, some “woodwind” instruments are made of metal while some “brass” 
  instruments are made of wood. 

  Rather, the key observation is one we have already mentioned: all wind 
  instruments are capable of producing a sustained tone, so they must all 
  involve nonlinearity in some essential way. The four-way classification 
  separates instruments based mainly on the details of the nonlinearity that 
  allows them to work. Three of the four are related, to the extent that they 
  blur into each other at the boundaries. These are the reeds, the brass and 
  the free reeds. The fourth member really is different, though. Flutes and 
  their relatives are driven by an air jet interacting with a sharp edge. This 
  excitation mechanism involves no mechanical moving parts, unlike the reeds or 
  vibrating lips of the other three categories of wind instrument. 

  \samsection{B. The four types of wind instrument} 

  I will give a brief description of the characteristic features of the four 
  categories of instruments here, but more detail will be given in separate 
  sections on each type. Figure 2 shows a sketch of the mouthpiece end of a 
  clarinet-like reed instrument. The player’s lips seal round the mouthpiece 
  and reed, and then the player applies a suitable blowing pressure so that air 
  flows through the narrow gap. As we already discussed briefly in section 8.5, 
  a crucial feature of this kind of backward-facing reed is that the player’s 
  blowing pressure tends to close the reed. If the pressure difference between 
  the mouth and the inside of the mouthpiece gets too big, the reed closes 
  completely against the rigid part of the mouthpiece (the “lay”) — or at 
  least, almost completely. In reality there will be a bit of leakage because 
  the reed and the lay do not meet perfectly. 

  Of course, the reed has a resonance frequency of its own. Indeed, it has many 
  resonances if we look sufficiently high in frequency, but the lowest 
  resonance is the most important for the normal musical functioning of a 
  clarinet. Even this lowest resonance is usually placed higher than the 
  fundamental frequency of any played note on the instrument, so reed resonance 
  is a relatively minor contribution to the behaviour of the instrument. Much 
  more important are the resonance frequencies of the tube, which the player 
  manipulates by opening and closing tone-holes. The playing frequency of any 
  particular note is predominantly governed by these tube resonances. 

  All this is in sharp contrast to our second category of instrument, which we 
  will continue to call “brass” instruments despite the comment about materials 
  made above. Figure 3 shows a sketch of the mouthpiece end of a typical brass 
  instrument. The tube is terminated by a cup-shaped mouthpiece. The player 
  presses their mouth firmly against this cup, and they “buzz” their lips. This 
  makes the lips open and close at the frequency of the note being played, and 
  high-speed video recordings of brass players [1] show that, when air is 
  flowing out of the mouth into the instrument, the lips open outwards as 
  sketched in the figure. So the lips act in a somewhat similar way to a 
  clarinet reed insofar as they provide a kind of nonlinear valve controlling 
  air flow into the instrument, but we will see in section 11.5 that the 
  behaviour is crucially different because of the outward-opening motion: 
  higher pressure in the mouth tends to make the gap get bigger, not smaller as 
  in a clarinet. 

  The tube of the instrument still has resonances, of course, but they do not 
  dominate the playing pitch to the same extent as in a reed instrument. The 
  brass player has to adjust their lip tension (“embouchure”) in order to 
  achieve the correct buzzing frequency for the desired note. The tube 
  resonances help a lot: players describe the sensation of the pitch being 
  “slotted”. But a sufficiently skilful and forceful brass player can coax many 
  pitches out of an instrument, especially when the resonances do not provide 
  much support — we will explain more in section 11.5. This leads to some brass 
  players’ party tricks: playing tunes on unlikely “instruments” such as vacuum 
  cleaner hoses, or (particularly striking) holding a steady, unchanging note 
  on a trombone while moving the slide in and out! 

  Things are different again for the third category of instruments, the free 
  reeds. Figure 4 shows a sketch of a reed from a harmonium or accordion. A 
  thin cantilever beam, usually of brass, is fixed to a heavier metal plate, 
  over a slot which is just a little bigger than the reed. As the reed 
  vibrates, it can move through this slot. If the amplitude is large, it will 
  move away from the plate on both sides at different stages in the cycle. As 
  in the instruments we have already looked at, the player creates pressure on 
  one side of the reed plate, and the moving reed provides a nonlinear valve 
  between the two sides of the plate. But this time, that valve opens wider on 
  both sides. In both the other cases, the valve tends to open when air flows 
  one way, and to close when it flows the other way: the difference between the 
  two cases lies in which way round these two things happen. 

  The other big difference between a harmonium or accordion and any of the 
  brass or reed instruments is that it has no resonating tube. The playing 
  frequency is usually very close to the resonance frequency of the reed. But 
  now we can see an example of blurring between these categories of instrument: 
  many Asian “mouth organs”, like the ones seen in Fig.\ 5, do have resonating 
  tubes. These are still formally classified as free reed instruments, but the 
  playing pitch (and the sound spectrum) is influenced by these tubes. A 
  related effect may be more familiar to western audiences: a harmonica player 
  is able to “bend” notes, a particularly vital ingredient of performance 
  technique for blues players. As will be explored in section 11.6, this is 
  achieved by manipulating a different source of resonance: the player’s mouth 
  cavity and vocal tract. 

  Thinking of the vocal tract reminds us that there is another familiar “wind 
  instrument”: the human voice. Where does this fit into the scheme we have 
  been outlining? When we speak or sing, the origin of the sound is that air 
  flow from our lungs is modulated by vibration of the vocal cords located down 
  in the larynx at the top of the trachea. (See \tt{}this Wikipedia page\rm{} 
  for more anatomical detail.) Figure 6 shows an animation of how the vocal 
  cords vibrate. The motion, and the fact that the vocal cords are made of 
  throat tissue not very different from the lips, reminds us of the brass 
  player’s lip vibration sketched in Fig.\ 3. 

  So should the human voice be classified as a brass instrument? Perhaps, but 
  some of the other key ingredients of a brass instrument are lacking. Although 
  the sound from the vocal cords does connect to a tube with resonances (the 
  vocal tract), this does not serve to provide “slotted” pitches. The pitch is 
  governed directly by the “reed resonance”, determined by how tightly your 
  throat muscles stretch the vocal cords (and by other variables such as the 
  pressure). In that respect, the singing voice is more like a free reed 
  instrument. 

  The vocal tract resonances are important, but not to modify the pitch. 
  Instead, they provide spectral colouration of the sound by the time it 
  emerges from your mouth, by amplifying some harmonics relative to others. 
  These patterns of amplification around vocal tract resonances are the 
  “formants” which determine your perception of vowel sounds, as mentioned back 
  in section 5.3. This makes the voice a relatively rare example of a wind 
  instrument involving a functional component playing a somewhat similar role 
  to that of the body of a stringed instrument. 

  Our final category describes wind instruments that depend on an edge tone of 
  some kind. Figure 7 shows a schematic sketch of part of the mouthpiece of a 
  recorder, or a flue organ pipe, or a referee’s whistle. Some examples of flue 
  organ pipes are shown in Fig.\ 8: wooden pipes in the centre, flanked by 
  metal ones. Air is blown through a slot, and emerges as a jet. A short 
  distance away from the slot is a sharp edge, and the air jet interacts with 
  this edge in some complicated, fluid-dynamical way. 

  In the presence of acoustic resonances (tube resonances in the case of the 
  recorder or the organ pipe, a Helmholtz resonance for the whistle) this 
  interaction may settle into a regular, periodic pattern and thus produce a 
  musical note. Even without an acoustic resonance, an air jet can sometimes 
  interact with an edge to produce a periodic variation in the flow pattern. 
  Another possibility involves resonant behaviour not of an acoustic tube, but 
  of the structure providing the edge: this effect is used deliberately in an 
  \tt{}Aeolian harp\rm{}, in which stretched strings are excited by the wind; 
  it also happens accidentally when power cables “sing” or “gallop” in the 
  wind. 

  These air-jet instruments are probably the most complicated of the four 
  categories in terms of modelling the underlying physics, but we need to 
  understand a bit about fluid flow to make progress with all four types of 
  wind instrument. This is the task of the next section: it will give a 
  qualitative introduction to some key concepts and phenomena of fluid 
  dynamics. After that, the remaining sections of this chapter will look at the 
  four categories of instrument in more detail. 

  \samsection{C. Tube basics} 

  Before that, in the remainder of this section, we will deal briefly with some 
  features held in common by several categories. First, a reminder of what we 
  learned in section 4.2 about acoustic resonators, and especially about 
  resonances in tubes with various bore profiles and boundary conditions. 
  Figures 9--12 reproduce some plots from that section. 

  Figure 9 shows the case of a straight pipe, open at both ends. This could be 
  an idealised model for a flute, recorder, or flue organ pipe (as in Fig.\ 8). 
  The fundamental mode has a half-wavelength between the ends, so its frequency 
  corresponds to a sound wave with wavelength twice the length of the pipe. The 
  higher modes have resonance frequencies in ratios 2, 3, 4… to this 
  fundamental frequency, so they fill a complete harmonic series. In reality, 
  the frequencies will all be a little lower because the pipe will “feel” a 
  little longer than its physical length: as explained in section 4.2.1, there 
  is an “end correction” to be added to the physical length, whose value 
  depends on the detailed geometry around the open ends of the pipe. 

  Figure 10 shows the corresponding plot for a straight pipe that is open at 
  one end but closed at the other. This could be an idealised model for a 
  clarinet, with the closed end at the mouthpiece, or for a “bourdon” organ 
  pipe, which is open at the mouth but closed at the top end. This time, the 
  fundamental mode has a quarter-wavelength between the ends, so its frequency 
  corresponds to a sound wave with wavelength four times the length of the 
  pipe. The higher modes have resonance frequencies in ratios 3, 5, 7… to this 
  fundamental frequency, odd-numbered frequencies of a harmonic series. Again, 
  for accurate frequencies it would be necessary to allow for end corrections. 

  Figure 11 shows the corresponding plot for a conical tube, which could be an 
  idealised model for an oboe or saxophone. The pressure mode shapes look 
  superficially similar to the closed-open modes of Fig.\ 10, because the 
  pressure waveforms all have a horizontal tangent at the left-hand end. This 
  will happen for any closed tube, whatever the detailed shape: once the 
  distance from the end is much smaller than the wavelength of sound, the 
  pressure is bound to be approximately uniform in space, just as we observed 
  when discussing the Helmholtz resonator back in section 4.2.1. Despite this 
  resemblance to the closed-open case of a straight tube, the resonance 
  frequencies are in fact exactly the same as for a straight open-open tube 
  like the one seen in Fig.\ 9. One familiar consequence is that the lowest 
  note of an oboe is about an octave higher than that of a clarinet, although 
  both instruments have tubes of approximately the same length. 

  Figure 12 shows the corresponding plot for an approximate model of a brass 
  instrument like a trumpet. There are no simple mathematical solutions for 
  such shapes: these mode shapes are numerically computed, as described back in 
  section 4.2. The black vertical lines mark the positions of the “effective 
  end point” for each mode: the position where the wave in the flaring horn 
  switches from oscillatory to evanescent behaviour. The behaviour revealed by 
  these lines is that the mode shapes look a bit like the ones in Fig.\ 10 for 
  a closed-open straight tube, except that the flaring bore shape has 
  “squashed” the lower modes into a shorter portion of the tube, while the 
  higher modes reach progressively closer to the open end of the bell. The 
  frequency ratios are approximately 0.7, 2, 3, 4. This is approximately a 
  harmonic series, except that the fundamental frequency is way out of tune. 
  This accounts for the fact that an instrument like this offers 
  ``slotted''pitches that fill a complete harmonic series rather than just the 
  odd terms of the series — except that the fundamental of the series is 
  missing. 

  This last comment gives a good opportunity to mention an important fact about 
  the overtone frequencies of any wind instrument tube: they play two quite 
  distinct roles in the musical behaviour. On the one hand, they provide 
  resonances that can interact with the nonlinear excitation mechanism to 
  determine the pitch of a played note. On the other hand, they influence the 
  sound spectrum and, more subtly, the “playability” of the note. Imagine you 
  are playing the lowest note on an oboe, with a frequency matching the lowest 
  resonance in Fig.\ 11. The nonlinear excitation mechanism (from the reed in 
  this case) also generates exact harmonics of the played note. If the higher 
  resonances of the tube are close in frequency to these harmonics, as they 
  would be for the idealised case in Fig.\ 11, they will all be resonantly 
  excited by the nonlinear harmonics. The result will be a sound richer in 
  those harmonic components. However, this argument only works for the lower 
  notes of an instrument like the oboe: in the higher registers the resonance 
  frequencies of a real tube cease to conform sufficiently accurately to a 
  harmonic pattern. 

  But also, it seems a plausible guess that the note will be made a bit easier 
  to play. All these resonances are, as it were, in agreement about what exact 
  pitch should be played. On the other hand, if the tube resonances were a bit 
  out of tune, so that each was shifted a bit away from exact harmonic 
  relations with the fundamental, surely these different modes would have 
  slightly different views about what the played pitch “should” be. That might 
  lead to a more sluggish transient as a compromise was “negotiated” between 
  the different resonances. This idea, involving “cooperative regimes of 
  oscillation”, goes back to an early study of the mechanics of wind 
  instruments by Bouasse [2], and more recently it was persuasively argued by 
  Benade [3]. Benade constructed a demonstration instrument he called the 
  “tacit horn”: the resonances were deliberately mis-aligned, with the result 
  that it was virtually impossible to sound any note on the instrument! 

  These two roles of tube resonances operate differently in our different 
  categories of instrument. In the reed and air-jet instruments, players 
  normally use oscillation regimes based on the first three or four modes of 
  the tube, not the higher modes. But in most brass instruments, players will 
  use many of the resonances of the tube as the basis for varying the pitch. 
  Indeed, in instruments like the bugle or natural horn there is no way to 
  change the effective length of the tube, and these regimes are the only 
  resource a player can draw on. Instruments with valves or slides are more 
  versatile because the tube length can be changed, but players still make use 
  of many more tube modes than a woodwind player will do. 

  \samsection{D. A useful measurement: input impedance} 

  There are a few more topics for us to discuss here, because they are 
  important for wind instruments in more than one category. One thing we have 
  already mentioned is that the standard measurement used to characterise the 
  linear acoustics of a wind instrument tube is the input impedance. The input 
  impedance is a frequency response function that captures the resonance 
  behaviour of the tube. It relates two acoustic quantities, the pressure and 
  the volume flow rate into the tube. For sinusoidal variation of both these 
  quantities, at a given frequency, the input impedance is a ratio: the 
  amplitude of the pressure divided by the amplitude of the flow rate. 

  It is worth looking at a simple example of input impedance, corresponding to 
  the ideal cylindrical pipes seen in Figs.\ 9 and 10. The input impedance of a 
  cylindrical pipe, open at the far end, is easy to calculate. The details are 
  given in the next link, and Fig.\ 13 shows an example of the result. This 
  particular case shows a tube 1 metre long, with internal diameter 20~mm. 
  Realistic damping has been included, using formulae from the literature: 
  again, the details are in the side link. 

  This gives us a chance to make an important observation about the 
  interpretation of input impedance for instruments in different categories. 
  Where do we find the resonances of the tube, by looking at a plot like this? 
  The answer depends on the boundary condition at the end of the tube where the 
  impedance has been measured. If the tube is closed at this end, as in Fig.\ 
  10, there can be no air flow in and out of the stopped end, but the pressure 
  variation can be large at a resonance. This means that pressure divided by 
  the volume flow rate must be very large — so the resonances correspond to 
  peaks in the input impedance. Sure enough, the peaks in Fig.\ 13 show the 
  odd-harmonic pattern that we expect from Fig.\ 10. 

  But if the tube is open at the end, as in Fig.\ 9, we have the opposite 
  situation. Air can flow in and out of the end of the tube, but the pressure 
  variation must be more or less zero because the open end is exposed to the 
  outside world, with a fixed ambient pressure. So this time, at a resonance we 
  expect very small values of the impedance, or equivalently we expect very 
  large values of its inverse, called the input admittance. This admittance is 
  plotted in Fig.\ 14: because of the decibel scale used here, the image is 
  simply the same as Fig.\ 13, plotted upside down. Now the peaks fall half-way 
  between the peaks in Fig.\ 13, and they are equally spaced in a complete 
  harmonic series. This is exactly what we expect from Fig.\ 9. 

  Finally, we look at the role played by tone-holes in the wall of the 
  instrument tube. Most reed and air-jet wind instruments have tone-holes. Less 
  familiarly, some “brass” instrument are played by covering tone-holes with 
  the fingers: examples are found in older instruments such as the 
  \tt{}cornett\rm{} (or cornetto) and the \tt{}serpent\rm{}. The purpose of 
  tone-holes seems self-evident: making a hole in the wall of the tube is 
  rather like cutting the tube short at that point, so that the frequencies of 
  all the resonances go up. However, as with so many things in musical 
  acoustics, it is a bit more complicated than that. 

  \samsection{E. What does a single tone-hole do?} 

  As a first step we can imagine a single tone-hole cut into an infinitely long 
  cylindrical tube, as sketched in Fig.\ 15. We can send a sinusoidal sound 
  wave at a chosen frequency down the tube. When this wave reaches the hole, 
  some of its energy is reflected back, while some of it is transmitted past 
  the hole and continues to propagate in the direction it was already 
  travelling. This problem is analysed in the next link, by thinking about the 
  air flow and the pressure in the small volume near the hole, shown as a 
  dotted box. Simple expressions are found for the reflection coefficient $R$ 
  and the transmission coefficient $T$. These depend on the geometrical 
  dimensions of the hole and the tube bore, and they also vary with frequency. 

  Figure 16 shows three examples of how the energy reflection coefficient 
  $|R|^2$ behaves when the area of the hole is varied from zero up to the 
  cross-sectional area of the tube, at three different frequencies. For the 
  lowest frequency (red curve), when the hole is as big as the cross-sectional 
  area of the tube virtually all the energy is reflected. This is the effect we 
  were expecting: the sound wave is more or less confined to the left-hand side 
  of the hole, just as if the tube had been cut off at that position. As the 
  size of the hole is reduced, the reflection coefficient drops gradually, 
  reaching zero as the hole area vanishes. Well, that is no surprise: if the 
  hole is no longer there, of course no energy is reflected from it. 

  At the two higher frequencies shown in Fig.\ 16 the reflection coefficient is 
  systematically smaller. Even with a very large hole, a significant fraction 
  of the energy is not reflected, but continues along the tube. As we will 
  explore shortly, this means that what happens further down the tube from the 
  tone-hole can have a significant impact on the behaviour at these higher 
  frequencies. To get a different view of the frequency dependence, Fig.\ 17 
  shows the reflection coefficient plotted against frequency. Curves are shown 
  for flutes of three different kinds, using bore size and tone-hole dimensions 
  measured by Wolfe and Smith [4]. All three curves show a similar trend: 
  perfect reflection at very low frequency, falling to rather low values at 
  2~kHz. Notice also that the three types of flute are systematically 
  different, and they are arranged in order of age. There has obviously been a 
  systematic evolution of flute design: we’ll come back to that in a moment. 

  When our infinite tube is replaced with a finite tube, we would like to know 
  what happens to the resonance frequencies of this tube as the single hole is 
  gradually closed. The answer turns out to be what you might guess. With a 
  large hole, the resonances are those of a shorter tube, cut off near the 
  hole. When the hole is reduced to a pinhole and then vanishes altogether, the 
  resonances are those of the full length of tube. Well, in between the 
  resonance frequencies change smoothly between these two limits. With this 
  rather primitive wind instrument with just a single tone-hole, the player 
  would in principle be able to produce a pitch that can be varied over this 
  range by using a finger to part-close (or “shade”) the hole progressively. 

  The physics behind this gradual shift of resonance frequencies can be 
  visualised in terms of an ``end correction''. With a single hole of any size, 
  the reflection is never perfect. This means that the effective length of the 
  tube is always longer than the distance to the centre of the tone-hole. With 
  a large hole and a correspondingly large reflection coefficient, this extra 
  bit of effective length is quite small, but as the reflection coefficient 
  reduces with a reduction of hole size, the sound wave penetrates further past 
  the hole and the end correction determining the effective length (and thus 
  the frequency of resonance) grows. Eventually, as the hole becomes 
  vanishingly small, this ``end correction'' reaches all the way to the end of 
  the tube, where virtually all of the sound energy is reflected back into the 
  tube. Some details of the calculation lying behind this description are given 
  in the next link. 

  This variation of the end correction with hole size is something that 
  instrument makers have taken advantage of, especially in earlier instruments 
  that rely on fingertips to cover holes, rather than using key mechanisms. 
  There are limits to how human fingers can be spread along the tube of a wind 
  instrument. This sometimes means that a tone-hole in the ``logical'' place 
  would be too far away from the others to be reached by the player's finger, 
  especially in larger instruments like the dulcian (or curtal), a predecessor 
  of the bassoon: see Fig.\ 18. The solution was to use a smaller hole with a 
  correspondingly bigger end correction, so that it could be placed in a more 
  convenient position. The resulting odd-looking distribution of holes can be 
  seen very clearly in the image. 

  \samsection{F. Multiple tone-holes} 

  Of course, real wind instruments have more than one tone-hole. Figure 19 
  shows a schematic sketch of a “flute”, with a typical fingering: there are 
  several tone-holes, the first few being covered while the remaining ones are 
  open. This introduces several complications. The first open tone-hole is no 
  longer in splendid isolation: not very far away are other open holes, before 
  we reach the end of the tube. Also, the closed tone-holes affect the profile 
  of the tube. The fingers or key-pads do not produce the same effect as not 
  having the hole in the first place, because they add a small local volume to 
  the tube, which can perturb the resonance frequencies a bit. 

  To see what happens to the frequency response of a real instrument tube with 
  various combinations of open and closed tone-holes we can look at some 
  measured input impedances, taken from Joe Wolfe's comprehensive web site on 
  flute acoustics \tt{}here\rm{}. We can look first at some results for a 
  baroque flute, which is a relatively simple instrument not too far removed 
  from the sketch in Fig.\ 19. It has 6 finger-holes, plus a 7th hole which is 
  operated by a key: you can see a picture in Fig.\ 22 below. 

  Figure 20 shows the input impedance for four different fingerings. Because we 
  are looking at a flute, which is open at the mouthpiece end, we must keep in 
  mind that the resonances of the tube are indicated in these plots by the 
  minima, not the peaks. The first plot shows the result with all the holes 
  closed, in the fingering for the lowest note of the instrument 
  ($\mathrm{D}_4$, 294~Hz at the modern pitch standard, but a semitone lower on 
  this flute at 277~Hz because it is adjusted for a pitch standard with 
  $\mathrm{A}_4$ at 415~Hz). The plot shows an orderly sequence of peaks and 
  minima, very reminiscent of the idealised version seen in Fig.\ 13. 

  The top right plot in Fig.\ 20 shows the impedance for a fingering rather 
  like the sketch in Fig.\ 19. It has three closed holes, then three open 
  holes, but the very last hole is closed by the key. The pattern is much less 
  regular. The explanation is that a baroque flute like this has relatively 
  small tone-holes (for a reason we will come to in a moment), and so a sound 
  wave coming from the mouthpiece end would only be partially reflected at the 
  first open tone-hole. A significant fraction of the energy will continue down 
  the tube and interact with the other open tone-holes as well as with the open 
  end of the tube. This complicated interaction leads to disruption of the 
  pattern of resonances. 

  \samsection{G. Cross fingerings and key systems} 

  The lower left plot shows the fingering for $\mathrm{A}\flat_4$. This one is 
  a “fork fingering” or “cross fingering”, with closed holes coming after the 
  first open hole. The possibility of using such fingerings to achieve 
  intermediate chromatic notes is probably the main reason that flutes like 
  this have rather small tone-holes. If the holes are to be closed by the 
  player’s fingers, unaided by complicated key mechanisms, there can only be as 
  many holes as the player has available fingers. This is not enough to offer a 
  full chromatic scale by simply opening one more hole for each semitone step. 
  So instruments like this one are tuned to give a diatonic scale in a 
  particular key, and the player has to use techniques like fork fingering to 
  sound the missing notes. Fork fingerings can only work if the reflection 
  coefficient at the first open tone-hole is fairly small: the hole's end 
  correction is then large enough that the influence of later closed holes can 
  be felt. 

  The final plot in Fig.\ 20 shows the result with only one tone-hole closed. 
  This one highlights a feature of the pattern of resonances which started to 
  show up in the previous two plots, but less clearly. There are only two 
  strong resonances (deep dips in the plot, remember), and for all frequencies 
  above about 1~kHz the resonances are significantly attenuated. This feature 
  is called the “tone-hole cutoff”, and it arises from the fact that there is a 
  fairly regular series of open tone-holes following the last closed one. An 
  approximate formula for this cutoff frequency in terms of the tone-hole 
  geometry was first given by Benade [5], on the assumption that the “lattice” 
  of open tone-holes was infinitely long, and uniformly spaced. A neat 
  derivation of the result can be found in the Appendix of reference [4]. 

  It is interesting to contrast this behaviour with some corresponding plots 
  for a modern flute, taken from the same web site and shown in Fig.\ 21. 
  Photographs of typical baroque and modern flutes are shown in Fig.\ 22. The 
  sketches accompanying each impedance plot show that a flute like this has 
  many more holes than the player has fingers. A complicated and ingenious 
  system of keys is used to open and close these holes — it was invented by 
  \tt{}Theobald Boehm\rm{} in the mid-19th century. A key system like this 
  frees the designer from some of the constraints of human fingers. As well as 
  allowing more holes, the individual holes can be larger because the key-pad 
  can be shaped to close a hole of virtually any shape. This is the explanation 
  for the contrast in reflection coefficient seen back in Fig.\ 17: the modern 
  flute (black line) showed much stronger reflection at all frequencies than 
  the baroque flute (red line). 

  The results of Boehm’s ingenuity can be seen in two features of the plots in 
  Fig.\ 21. The first plot, with all the holes closed, looks very similar to 
  the corresponding plot in Fig.\ 20. But after that, the behaviour is very 
  different. The modern flute shows an orderly sequence of resonances for all 
  the cases shown — there is very little of the disruption we saw in Fig.\ 20. 
  The second thing we can see is that the tone-hole cutoff for this flute is 
  significantly higher in frequency, up around 2~kHz. Both these features are, 
  qualitatively at least, what we might have anticipated from the contrast seen 
  in Fig.\ 17. Except for very high notes or some advanced playing techniques 
  such as “multiphonics” (we’ll come to those in section 11.4), there is 
  (almost) no need for fork fingerings on this flute. Each semitone has its own 
  key and hole combination. 

  \samsection{H. Closed holes and bore perturbations} 

  Now we have dealt in some detail with open tone-holes, we should have a brief 
  look at the perturbing effect of closed tone-holes. As we noted earlier, 
  these create a small extra volume, which will have a similar effect to a 
  slight bulge in the bore profile. Any perturbation to the bore will influence 
  frequencies slightly. This might interfere with carefully-planned alignment 
  of resonances, to improve the sound quality and playability. An instrument 
  maker might try to counter such effects by making deliberate modifications to 
  the bore profile. 

  So for one reason and another, it is of interest to know the effect on 
  resonance frequencies of a small change in bore profile. As explained in the 
  next link, Rayleigh’s principle (see section 3.2.5, and the application to a 
  marimba bar in section 3.3.1) gives us an easy way to answer this question. 
  We can describe the result qualitatively. For any given mode of a cylindrical 
  tube, like the ones shown in Figs.\ 9 and 10, the effect of a local 
  enlargement in the bore is biggest near a nodal point of pressure, or near an 
  antinode (a position of maximum pressure). An enlargement near a node 
  (including near an open end of the tube) has the effect of increasing the 
  frequency, whereas an enlargement near an antinode has the effect of reducing 
  the frequency. Since different modes have their nodes and antinodes in 
  different places, it follows immediately that the effect of a single 
  enlargement, such as a closed tone-hole, will be different for each mode, and 
  so it will indeed tend to interfere with the harmonic relations between 
  frequencies. But the same theoretical expression can be used to guide an 
  instrument maker wishing to adjust the bore to bring resonance frequencies to 
  desired values. 

  \samsection{I. Register holes} 

  Our final topic for this section is to note that not all holes in the tube 
  wall of a wind instrument function as tone-holes, to control the pitch of the 
  note. Most instruments also have register holes, to aid the player in 
  switching between a playing pitch based on the lowest mode of the tube, and 
  one based on the second mode. Players describe these alternatives as 
  different registers. Sometimes, a player can switch between registers by 
  adjusting the details of how they blow. For example, blowing harder into a 
  recorder or tin whistle may cause the pitch to jump to the next register, 
  often called “overblowing”. But life is easier for a player if they have a 
  controlled way to switch registers, and this is where register holes come in. 

  Consider the example of a cylindrical tube, open at both ends, like a flute 
  or recorder. As Fig.\ 9 reminds us, the first register is based on a pressure 
  mode shape with one half-wavelength in the length of the tube, whereas the 
  second register is based on a mode shape with a nodal point at the centre, 
  and two half-wavelengths in the length of the tube. If a small hole is 
  drilled in the tube wall near the centre, this will not affect the second 
  mode but it will perturb the first mode (which has a pressure maximum at this 
  position). It may shift the frequency of the first mode, and it may add some 
  dissipation and thus reduce the height of the resonance peak. Both effects 
  will tend to make it more difficult to play a note based on this mode, and 
  thus to make it more likely that the player will get a note in the second 
  register. 

  This is the principle of a register hole. Why does the hole need to be small? 
  The answer is that we don’t want to affect just one note, we would like to 
  encourage the register switch in other nearby notes, based on different 
  effective tube lengths because tone-holes have been opened or closed. Each of 
  these notes will have the second-register nodal point in slightly different 
  positions, so the hole can’t be perfectly positioned to suit them all. But if 
  the hole is fairly small, this doesn’t matter very much. If carefully 
  designed, it can have the desired register-changing effect on several notes. 

  If we are thinking about a clarinet rather than a recorder or flute, we can 
  make a similar argument based on the mode shapes in Fig.\ 10. The same 
  argument applies, except that the place we need to drill our register hole is 
  approximately 1/3 of the way down the tube from the mouthpiece, rather than 
  in the centre — because that is where the nodal point of the second pressure 
  mode shape occurs. 



  \sectionreferences{}[1] Murray Campbell, Joël Gilbert and Arnold Myers, “The 
  science of brass instruments”, ASA Press/Springer (2021). See section 3.1.3 
  for examples of video recordings of a brass-player's lips. 

  [2] H. Bouasse, “Instruments à Vent”, 2 volumes, Librairie Delagrave, 
  (1929–30). 

  [3] Arthur H. Benade; “Fundamentals of Musical Acoustics”, Oxford University 
  Press (1976), reprinted by Dover (1990). 

  [4] Joe Wolfe and John Smith, “Cutoff frequencies and cross fingerings in 
  baroque, classical, and modern flutes”, Journal of the Acoustical Society of 
  America \textbf{114}, 2263—2272 (2003) 

  [5] Arthur H. Benade, “On the mathematical theory of woodwind finger holes”, 
  Journal of the Acoustical Society of America \textbf{32}, 1591—1608 (1960) 