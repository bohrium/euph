  We can easily extend Millot's model from section 11.6.2 to include two reeds, 
  and thus represent one reed channel of a harmonica, coupled to the player's 
  mouth cavity. Figure 1 shows the sketch of the situation, reproduced from 
  section 11.6. Figure 2 shows the same system represented in the form of 
  Millot's model. A volume $V$, representing the mouth cavity, is fed with air 
  from the player's lungs at a steady volume flow rate $U_0$. A tube of length 
  $L$ and cross-sectional area $S$ opens off this volume, and at its end it 
  carries two reeds side by side. One is configured as an opening reed, the 
  other as a closing reed. One reed has length $L_1$, width $w_1$ and thickness 
  $h_1$, the other has corresponding dimensions $L_2, w_2$ and $h_2$. Each reed 
  is stood off from the end-plate by a clearance $x_0$, and the gap between 
  each reed and the slot in the end-plate is $b$ all the way round. 

  \fig{figs/fig-0f52e64b.png}{Figure 1. Schematic sketch of one reed channel of 
  a harmonica coupled to the player's mouth cavity, reproduced from Fig. 17 of 
  section 11.6.} 

  \fig{figs/fig-221c1c67.png}{Figure 2. Extension of Millot's model to 
  represent the system of Fig. 1. One opening reed and one closing reed are 
  attached to the end-plate of a tube representing the reed channel. The 
  right-hand sketch shows a face-on view of that end-plate.} 

  The pressure in the volume $V$ is $p_1(t)$, and the pressure acting on the 
  two reeds at the end of the tube is $p_2(t)$. The total volume flow rate into 
  the tube is $u(t)$, made up of the sum of contributions from the two reeds. 
  The flow rate through the gap surrounding reed 1 is $u_1$, and that past reed 
  2 is $u_2$. There are also flow rate contributions associated with the motion 
  of the two reeds. If the tip displacements are $x_1$ and $x_2$ respectively, 
  these contributions are proportional to $\dot{x_1}$ and $\dot{x_2}$ 
  respectively, so that the total flow rate is 

  $$u=u_1+K_{x1} \dot{x_1} + u_2 +K_{x2} \dot{x_2} \tag{1}$$ 

  where 

  $$K_{xj}=0.4 w_j L_j \mathrm{~~~~~} j=1,2. \tag{2}$$ 

  The flow rates through the reed gaps are given by the same expression as in 
  earlier models, derived from Bernoulli's law: 

  $$u_1=CF_1 \sqrt{\dfrac{2p_2}{\rho_0}}, \mathrm{~~~~}u_2=CF_2 
  \sqrt{\dfrac{2p_2}{\rho_0}} \tag{3}$$ 

  where $C=0.61$ is the vena contracta coefficient, $\rho_0$ is the density of 
  air, and $F_1$ and $F_2$ are the respective gap areas around the two reeds. 
  These areas are computed from the tip displacement by exactly the same 
  calculation as described previously. Figure 3 shows the two area functions 
  for the opening-reed and closing-reed configurations, using the particular 
  parameters assumed for the harmonica model: reed 1 has dimensions $14.5 
  \times 2 \times 0.13 \mathrm{~mm}$ and is tuned to frequency 440~Hz, reed 2 
  has dimensions $15 \times 2 \times 0.13 \mathrm{~mm}$ and is tuned to 392~Hz. 
  The gap around the reeds is $b=0.2 \mathrm{~mm}$ and the stand-off distance 
  is $x_0=0.5 \mathrm{~mm}$. 

  \fig{figs/fig-b5e3be20.png}{Figure 3. Gap area as a function of reed tip 
  displacement for the reed geometry used in this model. The red curve is for 
  the opening reed of the pair, the blue curve is for the closing reed.} 

  The reed channel is assumed to have length $L=28 \mathrm{~mm}$ and 
  cross-sectional area $S=20 \mathrm{~mm}^2$. This value of $L$ is somewhat 
  larger than the physical length of a typical harmonica channel to make some 
  allowance for the further constriction given by the player's lips. 

  The remaining equations follow from the earlier derivations in sections 
  11.6.1 and 11.6.2. The equation of motion of reed $j$, regarded as a damped 
  harmonic oscillator representing the lowest cantilever bending mode of 
  vibration, is 

  $$\ddot{x_j}+\dfrac{\omega_j}{Q_j} \dot{x_j} +\omega_j^2 x_j = K_{pj} p_2 
  \mathrm{~~~~~} j=1,2 \tag{4}$$ 

  where 

  $$K_{pj} = 1.5 \dfrac{w_j L_j}{m_j} \mathrm{~~~~~} j=1,2 \tag{5}$$ 

  as before, where $m_j$ is the total mass of reed $j$, calculated from the 
  volume $L_j \times w_j \times h_j$ using the density of brass, 8553~kg/m$^3$. 
  Reed $j$ has (angular) natural frequency $\omega_j$ and Q-factor $Q_j$. Both 
  reeds were assigned the same Q-factor, with the value 95. 

  The equation governing the pressure $p_1$ is 

  $$\dot{p_1}=\dfrac{\rho_0 c^2}{V} (U_0 -- u) \tag{6}$$ 

  as usual, where $c$ is the speed of sound. 

  Finally, Newton's law for the ``plug'' of air in the tube is 

  $$S(p_1-p_2) = \rho_0 L S \dfrac{d}{dt}\left[\dfrac{u}{S}\right] \tag{7}$$ 

  so that 

  $$p_1-p_2=\dfrac{\rho_0 c^2}{V \omega_h^2} \dot{u} \tag{8}$$ 

  in terms of the Helmholtz resonance frequency 

  $$\omega_h=\sqrt{\dfrac{c^2 S}{VL}}. \tag{9}$$ 

  We can augment equation (8) to allow the Helmholtz resonance to have some 
  damping, with Q-factor $Q_h$: 

  $$p_1-p_2=\dfrac{\rho_0 c^2}{V \omega_h^2} \left[\dot{u}+ 
  \dfrac{\omega_h}{Q_h}u \right] . \tag{10}$$ 

  For the simulations shown in section 11.6 the value $Q_h=4$ was used, to 
  reflect the rather high damping expected from a vocal tract resonance. 

  The value of inflow rate $U_0$ needed to create a given static pressure is 
  calculated by using static versions of this set of equations, setting all the 
  time derivatives to zero. Denoting steady values with an over-bar as in 
  sections 11.6.1 and 11.6.1, equation (8) tells us that 
  $\bar{p}_1=\bar{p}_2=\bar{p}$ say. Equation (6) gives $\bar{u}=U_0$. Equation 
  (4) then leads to 

  $$\omega_j^2 \bar{x}_j=K_{pj} \bar{p} \tag{11}$$ 

  while equation (1) gives 

  $$U_0=\bar{u}_1+\bar{u}_2= C \sqrt{\dfrac{2 \bar{p}}{\rho_0}} 
  (\bar{F}_1+\bar{F}_2) \tag{12}$$ 

  where $\bar{F_j}$ is deduced from the relevant one of the area functions 
  plotted in Fig.\ 3, evaluated at $\bar{x_j}$. 