  If your wood happens to be available in the form of a rectangular panel with 
  constant thickness, then there is a straightforward way to determine the 
  three main stiffnesses of the wood, using vibration resonance frequencies. I 
  will describe the measurement procedure first, then deal with the 
  mathematical background later in the section. 

  The simplest way to do this test is based on the traditional approach of 
  Chladni patterns. The plate is set into vibration, and powder is sprinkled on 
  it. When you excite a resonance of the plate, the powder bounces vigorously 
  and collects at the nodal lines, thus revealing the mode shape. \tt{}Ernst 
  Chladni \rm{}was a German scientist and musician in the 18th and early 19th 
  centuries. He found that he could excite vibration modes of metal plates by 
  bowing the edge. But for a controllable measurement, bowing is not very 
  convenient: these days is better to generate sine waves and play them through 
  a loudspeaker. 

  Figure 1 shows a typical setup. The loudspeaker is fixed to the underside of 
  a flat bench-top. It is driven by the sine-wave oscillator you can see under 
  the bench. Next to that is a frequency counter, which tells you the frequency 
  you are currently generating (very much like an electronic tuner that you 
  might use to tune an instrument). In the picture, a square plate of plywood 
  is being tested. It is showing a mode shape with a cross-shaped pattern of 
  nodal lines: usually this is the lowest-frequency mode of a panel like this. 

  \fig{figs/fig-dc1da399.png}{\caption{Figure 1. A ``shaker table'' being used 
  to generate a Chladni pattern on a plywood plate.}} 

  Notice how the plate has been arranged. It is being supported by four small 
  pieces of foam, carefully placed underneath the nodal lines. These are not 
  easily visible in Fig.\ 1, but they are more clear in Fig.\ 2, which shows a 
  different mode of the same plywood plate. The plate is positioned differently 
  in the two pictures, relative to the loudspeaker. This is deliberate. You 
  need your loudspeaker to be under an antinodal region of the mode shape, and 
  you also want to cover the loudspeaker aperture fairly well so that your 
  sound doesn’t all escape round the edges. In Fig.\ 2, the loudspeaker is not 
  visible because it is under the centre of the plate. But this position would 
  be no good for the mode in Fig.\ 1, where the centre is a nodal region. 
  Instead, the speaker is under one corner of the plate, where the vibration is 
  most vigorous. 

  \fig{figs/fig-7c4f3b78.png}{\caption{Figure 2. Chladni pattern of another 
  mode of the plate seen in Fig. 1.}} 

  Figure 3 shows computed predictions of the nodal line patterns for a few 
  modes of a plate like this. The exact details of nodal line shape, and also 
  the frequency order of the modes, depends on the stiffnesses we are trying to 
  measure, but any orthotropic plate with reasonably uniform mechanical 
  properties will have modes looking recognisably like these — with an 
  important exception that we will come to shortly. 

  \fig{figs/fig-1a380290.png}{\caption{Figure 3. Predicted nodal line patterns 
  of a few modes of a rectangular orthotropic plate with particular values of 
  the plate stiffnesses. The frequencies of these modes do not necessarily 
  appear in the order shown, though: that depends on the values of the 
  stiffnesses we are trying to measure. The exact shape of the nodal lines also 
  varies with those stiffnesses, but you can usually find modes looking 
  recognisably like these patterns.}} 

  Figure 3 tells you something else interesting. With a bit of practice, you 
  can use the knowledge of these nodal line patterns to find at least some of 
  the resonance frequencies by ear. Hold the plate lightly between finger and 
  thumb on a nodal line, tap with your knuckle in a place where the vibration 
  amplitude is large, and place your ear close to the same point. You should 
  hear a “tap tone” that corresponds to the frequency of that mode. At least 
  for the simplest mode shapes, this method can work very well. The 
  complication is that the tap sound you hear contains contributions from all 
  the modes, especially any others with a nodal line close to your holding 
  position. You can probably hear the lowest frequency among these, but it is 
  more difficult to “hear out” the higher ones. 

  The modes we are most interested in are the three in the top row of Fig.\ 3. 
  Figures 4, 5 and 6 show Chladni patterns of these three modes for a square 
  spruce plate. For this particular plate, these were modes numbers 1, 2 and 5: 
  two of the modes from the bottom row of Fig.\ 3 had frequencies lower than 
  the mode in Fig.\ 6, involving bending along the grain. The node lines in 
  Fig.\ 6 are somewhat distorted. This is a result of variation of mechanical 
  properties across the plate: you can see that the spacing of the annual rings 
  varies quite a lot. The rings are closely-spaced in the lower third of the 
  plate, then they abruptly get wider apart. Perhaps a neighbour of this tree 
  was felled at that time, letting more light through? But the slight 
  distortion does not stop us identifying the mode we are after. 

  \fig{figs/fig-b5a5a281.png}{\caption{Figure 4. The lowest mode of a 
  particular spruce plate. This is a twisting mode.}} 

  \fig{figs/fig-afe8bd60.png}{\caption{Figure 5. The second mode of the spruce 
  plate from Fig. 4, involving predominantly bending across the grain.}} 

  \fig{figs/fig-84143751.png}{\caption{Figure 6. The 5th mode of the spruce 
  plate from Figs. 4 and 5, involving predominantly bending along the grain.}} 

  From the frequencies of these three modes, plus knowledge of the dimensions 
  and weight of the plate, we can calculate stiffnesses associated with 
  twisting, bending across the grain, and bending along the grain, 
  respectively. We will give formulae for this shortly, together with some 
  background theory to explain what these stiffnesses mean. But first we should 
  deal with the exception to the claim that mode shapes like these can usually 
  be found. 

  It is probably simplest to introduce the key idea by forgetting about wood 
  (temporarily) and thinking about a metal plate, which will have properties 
  that are the same in all directions. Such a plate is called “isotropic”. What 
  will happen if our metal plate is exactly square? Our first guess might be 
  that the two modes corresponding to Figs.\ 5 and 6 will fall at the same 
  frequency. But in fact something more interesting happens. We do indeed find 
  two modes, fairly close in frequency, but the shapes look quite different 
  from those pictures. Figure 7 shows computer predictions of the node line 
  patterns of these two modes, and Fig.\ 8 shows them in a 3D view. 

  \fig{figs/fig-906d6135.png}{\caption{Figure 7. Predicted nodal line patterns 
  of two modes of a square isotropic plate. For obvious reasons, these are 
  usually called the ``X-mode'' and the ``ring mode''.}} 

  \fig{figs/fig-1b23e951.png}{\caption{Figure 8. The two modes from Fig. 7, 
  shown in a 3D view.}} 

  These modes are often described as the “X-mode” and the “ring mode”, for 
  obvious reasons. The best way to think about these two shapes is as the 
  difference and the sum of the two modes we found before. These involved more 
  or less one-dimensional bending in the two directions. Subtracting one from 
  the other generates the diagonal node lines we see on the left in Fig.\ 7, 
  while adding them together produces the “hill” shape that we see on the 
  right. 

  These two modes have different frequencies: usually the ring mode is higher 
  in frequency than the X-mode. The reason is to do with something we mentioned 
  earlier, Poisson’s ratio. Suppose we did a beam vibration test with a thin 
  rectangular beam of metal. As it bends, the upper surface is stretched while 
  the lower surface is compressed (or vice versa if it bends in the other 
  direction). Think about a small volume of metal near the upper surface. It is 
  being stretched along the beam axis, so Poisson’s ratio requires that it 
  should shrink a little in the perpendicular directions. Conversely, a small 
  volume of metal near the bottom surface is being compressed along the beam 
  axis, so it is expanding a bit in the cross direction. 

  So the top surface of our beam is shrinking across the width, while the 
  bottom surface is expanding. This combination will make the beam want to bend 
  in the cross direction, with the opposite sign to the way the main beam is 
  bending. This effect is sometimes called “anticlastic curvature”. Now return 
  to our X-mode and ring mode of the plate. The X-mode shows strong anticlastic 
  curvature, but the ring mode has the opposite behaviour, “synclastic 
  curvature”. So the X-mode is “going with the flow”, doing what Poisson’s 
  ratio wants, whereas the ring mode is fighting against Poisson’s ratio. The 
  effect is that Poisson’s ratio provides a bit of extra stiffness in the ring 
  mode compared to the X-mode, and that in turn means that it has a higher 
  frequency. 

  Returning to our spruce plate, something very similar can happen. The only 
  difference is that the plate doesn’t need to be square, it needs to be longer 
  in the grain direction. Specifically, the aspect ratio of the spruce plate 
  must be adjusted to match the proportions you would find if you made 
  long-grain and cross-grain beams which had the same frequency. 

  Now it is time to introduce the actual stiffness parameters for our plate, 
  and relate them to the mode frequencies we have been talking about. We will 
  use an energy-based approach to define a suitable set of stiffnesses. Back in 
  section 3.3.1 we saw the expressions for energy in a vibrating beam, and we 
  also met Rayleigh's principle. We now make use of both those ideas. The 
  potential energy of a bending beam was proportional to the squared curvature, 
  and the curvature is the second derivative of the displacement. 

  We expect a similar expression for our plate: a quadratic expression 
  involving second derivatives in $x$ and $y$, the coordinates along the 
  symmetry axes of the material (and the edges of the plate). Furthermore, we 
  expect this expression to be symmetrical if you interchange $x$ and $y$. 
  There are just four possible expressions satisfying these conditions, so our 
  most general expression for the potential energy must take the form 

  \begin{equation*} V = \dfrac{1}{2} \int{\int{h^3 \left[ D_1 
  \left(\dfrac{\partial^2 w}{\partial x^2} \right)^2 + D_2 \dfrac{\partial^2 
  w}{\partial x^2} \dfrac{\partial^2 w}{\partial y^2} \right. }} 
  \end{equation*} 

  \begin{equation*} \left. + D_3 \left(\dfrac{\partial^2 w}{\partial y^2} 
  \right)^2 + D_4 \left(\dfrac{\partial^2 w}{\partial x \partial y} \right)^2 
  \right] dx dy \tag{1}\end{equation*} 

  \noindent{}where $w(x,y)$ is the displacement of the plate, $h$ is the 
  thickness, and $D_1$--$D_4$ are four constants with the dimensions of 
  stiffness. The corresponding expression for kinetic energy is much simpler: 

  \begin{equation*}T=\dfrac{1}{2} \int{\int{ \rho h \left( \dfrac{\partial 
  w}{\partial t} \right)^2 dx dy }} \tag{2}\end{equation*} 

  \noindent{}where $\rho$ is the density. 

  Now we can calculate three simple estimates using Rayleigh's principle, to 
  obtain approximate expressions for the frequencies of the three modes in the 
  top row of Fig.\ 3. The first of those three modes involves twisting motion. 
  We can call the frequency (in Hz) $f_t$, and obtain a Rayleigh estimate using 
  the approximate mode shape $w \approx xy .$ The result, turned round to give 
  an expression for stiffness, is 

  \begin{equation*}D_4 \approx 0.274 f_t^2 \rho a^2 b^2/h^2 
  \tag{3}\end{equation*} 

  \noindent{}where the plate has dimensions $a \times b$. The other two modes 
  in the top row of Fig.\ 3 involve approximately one-dimensional bending in 
  the $x$ and $y$ directions. If we call the corresponding frequencies $f_x$ 
  and $f_y$, we can obtain Rayleigh estimates for both by using the first mode 
  of a free-free beam as an approximate mode shape. This gives the expressions 

  \begin{equation*}D_1 \approx 0.0789 f_x^2 \rho a^4/h^2 \tag{4}\end{equation*} 

  \noindent{}and 

  \begin{equation*}D_3 \approx 0.0789 f_y^2 \rho b^4/h^2 . 
  \tag{5}\end{equation*} 

  There is a way to improve on these three simple estimates, but it involves 
  more elaborate computation (see [1] for the details). For workshop use by a 
  guitar maker, though, these first estimates are good enough to be useful in 
  quantifying the stiffness properties of their stock of soundboard blanks. 

  You will have noticed that we do not have an estimate for $D_2$. There is a 
  procedure for finding that, but it involves cutting the plate to the 
  ``magic'' proportions needed to make the X-mode and ring mode appear. Again, 
  the details are described in [1]. But a guitar maker is unlikely to want to 
  cut their soundboard blanks in this way, so it is fortunate that numerical 
  studies suggest that $D_2$ is not very important, except for the special case 
  of a plate cut with the magic proportions. 

  Finally, we would surely like to relate these stiffness parameters $D_j$ to 
  more standard things such as Young's moduli and shear moduli. The full story 
  here is rather complicated, though. The only case which gives clear-cut 
  answers is the ideal quarter-cut plate, in which the $x$ and $y$ axes are 
  exactly aligned with the L and R directions in the solid timber. For that 
  case, 

  \begin{equation*}D_1 = E_L/12 \mu, D_2 = \nu_{LR}E_R/6 \mu = \nu_{RL}E_L/6 
  \mu, \end{equation*} 

  \begin{equation*}D_3 = E_R/12 \mu, D_4 = G_{LR}/3 \tag{6}\end{equation*} 

  \noindent{}where $E_L$ and $E_R$ are the two Young's moduli, $G_{LR}$ is the 
  shear modulus in the LR plane, $\nu_{LR}$ and $\nu_{RL}$ are the two 
  Poisson's ratios, and $\mu=1 -- \nu_{LR} \nu_{RL} .$ The two versions of the 
  expression for $D_2$ make use of the reciprocal relation for Poisson's 
  ratios. For the full gory details of more general cases, see [1]. 

  Finally, as a guide what kind of numbers you might expect if you measure the 
  $D_j$ for soundboard material, here are some measured results for spruce. A 
  quarter-cut plate gave $D_1=1320~\mathrm{~Mpa}$, $D_2=77~\mathrm{~Mpa}$, 
  $D_3=82~\mathrm{~Mpa}$, $D_4=227~\mathrm{~Mpa}$. For contrast, the spruce 
  plate shown in Fig.\ 9 of section 10.3, with a ring angle close to 
  $45^\circ$, gave $D_1=880~\mathrm{~Mpa}$, $D_2=24~\mathrm{~Mpa}$, 
  $D_3=13~\mathrm{~Mpa}$, $D_4=229~\mathrm{~Mpa}$. 

  \sectionreferences{}[1] M. E McIntyre and J. Woodhouse, “On measuring the 
  elastic and damping constants of orthotropic sheet materials”, Acta 
  Metallurgica \textbf{36}, 1397—1416 (1988). 