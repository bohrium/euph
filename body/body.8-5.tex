

  Musicians do not normally want their instruments to exhibit chaotic 
  behaviour. Nevertheless the ideas of chaos and sensitive dependence do have 
  some important manifestations in musical performance, and we will meet 
  examples later. But our final topic for this chapter is far more obviously 
  central to the business of making music. Some nonlinear systems can produce 
  sustained vibration “out of nowhere”: self-excited vibration. The human voice 
  is an example, and all the musical wind instruments: woodwind, brass, organ 
  pipes and so on. All the bowed-string instruments are also examples, as are 
  other instruments relying on friction to cause vibration. \tt{}Friction 
  drums\rm{}, for example, have a rod attached to a drum membrane of some kind, 
  and the player rubs the rod with rosin-coated fingers to produce a sound. The 
  familiar singing effect produced by running a wet finger round the rim of a 
  wineglass relies on similar behaviour. 

  We will come to musical examples very shortly, but we will start this section 
  with a non-musical example which allows us to make some links with the 
  discussion in the previous sections, based around the phase plane 
  representation. We will look briefly at something called the Van der Pol 
  equation, originally proposed by a Dutch electrical engineer back in 1927 to 
  describe the behaviour of certain electrical circuits based on valves (vacuum 
  tubes). These circuits exhibited spontaneous oscillation: a phenomenon still 
  crucial in many devices, such as the internal ``clock'' providing the timing 
  signal inside every computer or mobile phone. Since the 1920s, Van der Pol’s 
  equation has been found useful for modelling many other phenomena in physics, 
  geophysics and biology. 

  You can see Van der Pol's equation, and some extra detail about its history 
  and behaviour, in the next link. Essentially, the equation describes a linear 
  oscillator modified by a nonlinear damping effect. The coefficient of this 
  nonlinear term determines the details of the behaviour. If it is zero, the 
  system is just a linear oscillator with a single singular point at the origin 
  in the phase plane, which is a centre as expected for an undamped oscillator. 
  But when this nonlinear coefficient is positive, that centre turns into an 
  unstable spiral so that any trajectory starting near the origin spirals 
  outwards. But it doesn't continue outwards: instead, it tends to towards a 
  closed loop in the phase plane, in other words a periodic solution or limit 
  cycle. Furthermore, any trajectory starting outside that limit cycle spirals 
  inwards towards it. The limit cycle is an attractor for (almost) every 
  trajectory. 

  Figure 1 shows some examples, for different values of the nonlinear 
  coefficient. In each case four trajectories are shown: two in red starting 
  inside the limit cycle, and two in black starting outside it. The same four 
  starting points are used in every case. The limit cycle and its ``attractor'' 
  behaviour is very clear. The shape of the limit cycle changes with the value 
  of the coefficient. What this means for the waveform of displacement of the 
  oscillator is shown in Fig.\ 2, for the same four cases. When the coefficient 
  is small, the limit cycle is almost an ellipse and the waveform is almost 
  sinusoidal (after a starting transient). As the coefficient grows, the 
  waveform changes shape. It gradually acquires sharp corners, which inevitably 
  means that the periodic solution has a significant mixture of higher 
  harmonics in its Fourier series. The frequency of the self-excited vibration 
  also changes, getting lower as the coefficient increases, as is clear in the 
  lower row of plots in Fig.\ 2. 

  \fig{figs/fig-522efe5d.png}{} 

  \fig{figs/fig-af4bcf5e.png}{} 

  \fig{figs/fig-fe8e0c27.png}{} 

  \fig{figs/fig-ff2c34e2.png}{} 

  \fig{figs/fig-97db4550.png}{} 

  \fig{figs/fig-a9f0d44f.png}{} 

  \fig{figs/fig-c74b9a72.png}{} 

  \fig{figs/fig-882868bd.png}{} 

  We can now look at an example relating more directly to musical instruments. 
  The discussion will be phrased in terms of a clarinet, but really the 
  description to be given here would apply in general terms to any reed 
  instrument: saxophone, oboe, reed organ pipe or whatever. Figure 3 shows a 
  sketch of the mouthpiece end of a clarinet. The main tube of the instrument 
  is an acoustic duct with internal resonances of the kind we met back in 
  Section 4.2. The tube has tone-holes bored through it: the player can cover 
  some or all of these using fingertips or key mechanisms, to modify the 
  frequencies of the internal acoustic resonances. In a clarinet the tube is 
  usually made of wood or plastic, in instruments like the saxophone it would 
  be made of metal. These different material choices make very little 
  difference to us here, because they all result in a tube that is essentially 
  rigid. All the important action takes place in the air inside the tube. 

  \fig{figs/fig-9476305d.png}{\caption{Figure 3. Sketch of the mouthpiece end 
  of a clarinet, showing the pressure $p(t)$ just inside the mouthpiece, and 
  the volume flow rate $v(t)$ of air from the player's mouth into the 
  instrument.}} 

  At the mouthpiece, a flexible reed is attached to the tube. The player puts 
  this part in their mouth, and blows. What happens then is a matter of common 
  empirical experience. If the player blows very gently, the only sound is a 
  bit of rushing wind noise associated with the turbulent air flow through the 
  mouthpiece into the tube. But if the player gradually increases the blowing 
  pressure, at a certain point the instrument will “light up” and start to 
  produce a musical note. If the blowing pressure is increased further, the 
  tone quality of the note changes (and perhaps the pitch changes a little as 
  well). The sound tends to get louder and brighter with higher blowing 
  pressure. But if the blowing pressure is increased too far, the instrument 
  “chokes up” and the sound stops. 

  This sequence of events can be understood in a simple way. We can suppose 
  that the player simply provides a constant pressure in their mouth, called 
  $p_0$ in Fig.\ 3. But just inside the mouthpiece there will be a time-varying 
  pressure which we can call $p(t)$: this is the quantity we would like to 
  understand, since it is responsible for the sound of the instrument. There is 
  a second time-varying quantity we need to think about: this is the flow rate 
  of air from the player’s mouth, through the mouthpiece and into the tube. We 
  will call this $v(t)$, to describe the volume flow rate (in cubic metres per 
  second if we wanted to put a number to it). 

  The two quantities $p(t)$ and $v(t)$ are related to each other in two quite 
  different ways, illustrated schematically as a block diagram in Fig.\ 4. 
  First, they are connected via the linear acoustical behaviour of the tube. We 
  can imagine a laboratory experiment in which the instrument was supplied with 
  a sinusoidal flow of air at its mouthpiece end by an actuator, and the 
  pressure response inside the mouthpiece was measured by a small microphone. 
  By varying the frequency of the sine wave, the frequency response function of 
  the tube could be measured. This particular frequency response, with volume 
  flow as input and pressure as output, is called the input impedance of the 
  clarinet. 

  \fig{figs/fig-ecc26eae.png}{\caption{Figure 4. Schematic flowchart of the 
  acoustical interactions in a clarinet}} 

  Such measurements of input impedance are indeed routinely made on wind 
  instruments of all kinds: Fig.\ 5 shows one being carried out on a saxophone. 
  The measurement is being done in an anechoic chamber, a special room with 
  sound-absorbing walls to avoid complications from room acoustics. As with 
  other linear response measurements we have seen earlier, the test need not 
  necessarily be done using a sinusoidal signal. Any input can be used, 
  provided it can be measured. The input and output signals can be converted 
  into the frequency domain using an FFT routine in the computer, just as is 
  done when structural measurements are made using an impulse hammer. 
  Unfortunately, measuring volume flow rate is not so easy. The actual design 
  of a measuring rig like the one seen in Fig.\ 5 involves some ingenuity: we 
  will come back to this later, in section 10.4.1. 

  \fig{figs/fig-69ebee70.png}{\caption{Figure 5. Measuring the input impedance 
  of a saxophone. Image copyright Jean-Pierre Dalmont, reproduced by 
  permission.}} 

  A typical measured example of the input impedance of a clarinet is shown in 
  Fig.\ 6. The peaks in the plot correspond to the resonances of the tube. The 
  pattern of these peaks shows them with approximately harmonic spacing, but 
  only for the odd harmonics 1,3,5... This is as we should expect from the 
  discussion in section 4.2: the tube of a clarinet is approximately a uniform 
  duct, and it is effectively open at the bell end but closed at the mouthpiece 
  end. As shown in Fig.\ 12 of section 4.2, those conditions lead naturally to 
  odd-harmonic resonance frequencies. If it had been a tapered duct, like an 
  oboe or the saxophone in the photograph, the impedance curve would have shown 
  peaks near all the harmonics, with approximate frequency ratios 1:2:3:4... 

  \fig{figs/fig-85a6f6c8.png}{\caption{Figure 6. Measured input impedance of a 
  clarinet}} 

  The second relation between $p(t)$ and $v(t)$ involves the mouthpiece and 
  reed, acting as a kind of nonlinear valve. For the purposes of this initial 
  discussion, we will use a severely simplified approximation. We will take no 
  account of the fact that the flexible reed has resonance frequencies its own, 
  we will simply treat it as behaving like a spring. As a first step, we won’t 
  even allow that much: we can think about how the air flow through the 
  mouthpiece would behave if the reed were rigid. A pressure difference could 
  be applied across this rigid mouthpiece, and the resulting air flow rate 
  could be measured. What we would expect to see, disregarding any subtleties 
  of fluid mechanics such as vortices, would be something like the dashed line 
  in Fig.\ 7. The bigger the pressure difference, the bigger the air flow. If 
  the air flow is dominated by viscous resistance through the small gap, the 
  behaviour would just be linear, as sketched. The dashed line is sloping 
  downwards rather than upwards because of the sign convention we have used: a 
  positive value of the pressure difference $p(t)-p_0$ corresponds to sucking 
  the mouthpiece, not blowing it, so we would expect $v(t)$ to be negative. 

  \fig{figs/fig-0971cdad.png}{\caption{Figure 7. Schematic variation of air 
  flow into a clarinet mouthpiece as a function of the difference of pressure 
  between the player's mouth and the inside of the mouthpiece. The dashed line 
  shows the tangent to the curve through the origin: this is the behaviour you 
  might expect if the reed had been rigid.}} 

  For very small values of the pressure difference, the actual behaviour of the 
  mouthpiece with a flexible reed should be rather like the dashed line. But as 
  the player tries to blow more air through, making $p(t)-p_0$ more and more 
  negative, the reed will be pressed inwards. As a result, the flow rate $v(t)$ 
  will be less than the dashed line would suggest. The result will be something 
  like the solid line in Fig.\ 7: the further we move towards the left, the 
  more the reed is closed and the more the air flow is restricted. Eventually, 
  in an idealised situation, the reed will close completely against the rigid 
  part of the mouthpiece (called the “lay”). There would be no air flow at all, 
  indicated by the horizontal portion of the curve. 

  If the player applies a low mouth pressure while the clarinet is not making 
  any sound, that would correspond to shifting to a position on the curve like 
  the red dot in Fig.\ 8. Now suppose there is a little bit of pressure 
  variation (i.e. sound) inside the tube. This will change the pressure 
  difference a little, in the vicinity of the red dot: the air flow will then 
  vary in a way that follows the tangent to the curve at that point, shown in 
  the red line. 

  \fig{figs/fig-c6229fcd.png}{\caption{Figure 8. Mouthpiece characteristic as 
  in Fig. 7, showing the tangent to the curve near a particular ``operating 
  point'', set by a fairly low value of the player's mouth pressure.}} 

  Such variations have a simple physical interpretation. The tangent line is 
  down-sloping, similar to the dashed line in Fig.\ 7. But we know what that 
  line describes: it is the response associated with a viscous resistance, and 
  it involves energy dissipation. But now suppose the player blows a little 
  harder, so that the operating point on the curve shifts to a position like 
  the one shown in Fig.\ 9. Because we have gone past the peak in the curve, 
  the tangent line now has an upward slope. That would correspond to a negative 
  viscous resistance, and small fluctuations in pressure inside the mouthpiece 
  will result in energy being gained, not lost. (Of course, energy has not been 
  created from nowhere: this energy gain is actually supplied by the player’s 
  lungs doing a little more work.) 

  \fig{figs/fig-7cda2d6b.png}{\caption{Figure 9. Mouthpiece characteristic as 
  in Figs. 7 and 8, showing the tangent to the curve near a different 
  ``operating point'', corresponding to a slightly higher value of the player's 
  mouth pressure.}} 

  We can now make an intuitive leap, and guess what might happen. The tube of 
  the clarinet is an acoustic duct, with resonances. Each of these resonances 
  will, of course, involve some energy dissipation. Energy can be lost in 
  several different ways: some is “lost” because it is carried away by sound 
  waves radiating from the instrument, while some is lost within the tube, 
  mainly by viscous and thermal interaction with the walls. It seems plausible 
  that if for some particular resonance these losses can be compensated by the 
  “negative resistance” effect associated with the mouthpiece, as indicated in 
  Fig.\ 9, that resonance might become unstable. 

  We saw just such an effect with the Van der Pol equation, in the plots of 
  Fig.\ 1. The equilibrium position became unstable, leading to growing 
  oscillations, settling down after a while into a periodic limit cycle. This 
  is exactly what happens with our simplified model of a clarinet, and the 
  result is the behaviour we described earlier, very familiar from real 
  clarinets. There is a threshold of blowing pressure, at which the instrument 
  “lights up” and starts to make sound. The phenomenon is completely dependent 
  on the nonlinear behaviour of the mouthpiece with its flexible reed. 

  There are two different approaches we can use to test whether we have guessed 
  correctly how the clarinet model will behave. The more mathematical of the 
  two involves applying the method of harmonic balance (introduced in section 
  8.2.2) to the situation of Fig.\ 9. The details are described in the next 
  link: the conclusion is that there is indeed a threshold of blowing pressure 
  when a self-excited periodic oscillation becomes possible. 

  This analysis leads to several predictions. First, the frequency will 
  correspond to the highest peak in the input impedance, and we can see in 
  Fig.\ 6 that this is the fundamental resonance of the clarinet tube, as we 
  might have expected. Second, the nonlinear behaviour of the mouthpiece 
  inevitably means that the waveform of pressure, and hence the sound of the 
  instrument, will involve some harmonics as well as the fundamental. As the 
  amplitude of the fundamental frequency grows, the relative amplitudes of the 
  higher harmonics grow faster, so that the sound become richer or brighter. 
  Finally, for the particular case of the clarinet the odd-numbered harmonics 
  are likely to be much stronger than the even-numbered ones. But for a conical 
  instrument like the saxophone, the even harmonics will be strongly present. 

  The alternative approach is to turn directly to numerical simulation, and use 
  the computer to explore how the model behaves. A particularly efficient way 
  to do this involves formulating the behaviour of the linear part of the 
  clarinet in a slightly different way: not based on a frequency response 
  function like the input impedance, but describing the acoustic response of 
  the duct in terms of travelling waves. Similar methods are used in other 
  fields, such as speech synthesis, but in the context of understanding the 
  physics of musical instruments, the approach was first developed in the 
  context of the vibration of a bowed string, and we will explore that 
  application in the next chapter. But the clarinet model gives a simple way to 
  introduce the method. A pleasing name has been given to the general approach 
  by Julius Smith: he calls it \tt{}the “digital waveguide” method\rm{}. 

  Suppose a short pulse of air is injected into the mouthpiece. This will 
  generate a pressure pulse just inside the mouthpiece, which will propagate 
  down the tube, reflect from the bell (or perhaps from the first open 
  tonehole), and after a delay it will arrive back at the mouthpiece. The pulse 
  will have been inverted by the reflection, and it will have been spread out a 
  bit by energy dissipation effects during the journey along the pipe and back. 
  The result will look a bit like Fig.\ 10. We can call this a “reflection 
  function”: it captures everything we need to know about the linear acoustical 
  behaviour of the clarinet tube. Earlier, we said the same about the input 
  impedance: the reflection function provides the same information in a 
  different form, which happens to be very convenient for a computer simulation 
  of the time-varying behaviour of our simple clarinet model. 

  \fig{figs/fig-9c80c3be.png}{\caption{Figure 10. Idealised version of a 
  ``reflection function'' for a clarinet tube. A short pulse of pressure has 
  been injected at the mouthpiece at time $t=0$, and this plot shows the 
  pressure pulse which returns after a trip along the tube and back. It has 
  been inverted by the process of reflection at the open end of the tube, at 
  the bell or an open tonehole, and it has been spread out because of energy 
  dissipation during the journey. This particular example shows a delay of 
  5~ms, which corresponds to a fundamental resonance of the tube at 100~Hz.}} 

  The outline of the computer synthesis algorithm is then easy to explain — the 
  details are given in the next link. The process of synthesising sound from 
  our “clarinet” works in “discrete time”, to produce a digital sound file 
  sampled at a chosen regular sampling frequency, such as the standard 44.1~kHz 
  of a normal CD. At each time step, the computer makes use of the reflection 
  function together with knowledge of the past history of the pressure inside 
  the mouthpiece, to calculate the value of the incoming reflected pressure 
  wave arriving at the mouthpiece at that moment. 

  This incoming wave can be combined with the nonlinear mouthpiece model from 
  Fig.\ 7, with some assumed value of the mouth pressure $p_0$ (which could be 
  varying in time if we wished). As shown in the link, there is a graphical 
  interpretation of what is needed: the incoming wave determines the position 
  of a straight line, and the intersection of that line with the nonlinear 
  curve determines the new values of the volume flow through the mouthpiece and 
  of the outgoing pressure wave being sent off down the tube. Figure 11 
  illustrates the idea, with three different possible positions of the straight 
  line. Time is then advanced by one step, and the process is repeated — for as 
  long as we wish to compute the sound. 

  \fig{figs/fig-3f38ec08.png}{\caption{Figure 11. The nonlinear mouthpiece 
  curve as in Fig. 7, with three possible positions of the straight line 
  determined by the incoming pressure wave. The new values of pressure and flow 
  rate are given by the intersection of the line and the curve.}} 

  We can illustrate this approach by some examples. The method just described 
  has been used to simulate waveforms from the idealised clarinet, with a 
  variety of values of the mouth pressure $p_0$. These range from a value just 
  above the threshold to a value just below the limit where the reed stays shut 
  all the time and the instrument ``chokes''. The nonlinear mouthpiece 
  characteristic is exactly as plotted in Figs.\ 7--9, and the reflection 
  function is the one plotted in Fig.\ 10. Pressure waveforms for 6 cases are 
  shown in Fig.\ 12, and the corresponding volume flow waveforms in Fig.\ 13. 
  The colours are graduated from red to blue, as the mouth pressure increases. 
  We can notice immediately that the pressure waveform starts out with a low 
  amplitude and a quasi-sinusoidal shape in the reddest curve, and as the mouth 
  pressure increases the amplitude grows and the waveform develops into a 
  rather square shape. 

  \fig{figs/fig-c69e2399.png}{\caption{Figure 12. Six simulated pressure 
  waveforms from the simplified clarinet model. The plot shows the periodic 
  waveforms that have developed after the transient has completed. The values 
  of mouth pressure $p\_0$ are 2.13 (reddest curve), 2.15, 2.2, 2.3, 3 and 4.8 
  (bluest curve).}} 

  \fig{figs/fig-ae50b89a.png}{\caption{Figure 13. Waveforms of volume flow, 
  from the same set of simulations as Fig. 12. The two bluest curves, with the 
  highest values of mouth pressure, are rather similar and not easy to 
  distinguish. Both involve the reed being entirely shut (volume flow zero) for 
  a significant portion of each cycle.}} 

  With the lowest value of mouth pressure, the waveforms of both pressure and 
  volume flow are more or less sinusoidal, as expected from the harmonic 
  balance calculation described in section 8.5.2 above. For a different view of 
  what is going on, Fig.\ 14 shows the region of the nonlinear characteristic 
  curve that is ``explored'' by these waveforms. This version of the curve 
  shows pressure, rather than pressure difference: the curve has been shifted 
  sideways by $p_0$, so that the operating point is marked by the position of 
  the vertical axis. As expected, for this threshold case that operating point 
  is just to the left of the peak of the curve. 

  \fig{figs/fig-521c4bd5.png}{\caption{Figure 14. The flow characteristic of 
  the mouthpiece, as in Figs. 7--9, showing the range over which the pressure 
  and volume flow vary during the simulation with $p\_0 = 2.13$ from Figs. 12 
  and 13 The curve has been shifted sideways by $p\_0$, so that the ``operating 
  point'' falls on the vertical axis. This case is very close to the threshold 
  for oscillation.}} 

  As the mouth pressure increases, Fig.\ 12 shows that the waveform becomes 
  progressively less sinusoidal, and its amplitude grows. In other words, the 
  sound becomes brighter and louder, as predicted by the harmonic balance 
  argument. Figure 15 shows a plot corresponding to Fig.\ 14 for the case with 
  $p_0=2.2$, the third case in Figs.\ 12 and 13. More of the nonlinear curve is 
  now involved, which is why the waveforms in Figs.\ 12 and 13 are less 
  sinusoidal. But the red points are still confined to the smooth part of the 
  curve, so that the harmonic balance argument should apply to this case. 

  \fig{figs/fig-b355518c.png}{\caption{Figure 15. A plot in the same format as 
  Fig. 14, for the case with $p\_0 = 2.2$ from Figs. 12 and 13.}} 

  Figure 16 shows the corresponding plot for the case with $p_0=4.8$. Now the 
  points occupy the full range of the curve, including the horizontal portion 
  where the reed is shut so that there is no air flow into the tube. The 
  harmonic balance argument will no longer be useful, because the non-smooth 
  part of of the curve is involved. Notice also that the ``operating point'' is 
  at the very end of the smooth part of the curve. A small increase in $p_0$ 
  would shift that operating point to the flat portion, and this is the signal 
  that the instrument will cease to sound. 

  \fig{figs/fig-7c6535ea.png}{\caption{Figure 16. A plot in the same format as 
  Figs. 14 and 15, for the case with $p\_0 = 4.8$ from Figs. 12 and 13. This 
  time a wide range of the curve is covered by the oscillation, including the 
  horizontal portion where the reed is shut.}} 

  Looking back at the corresponding waveform of flow rate in Fig.\ 13, we see 
  that for this extreme case the volume flow is approximately zero most of the 
  time. The system spends most of the time in the positions indicated by the 
  left-most and right-most of the three straight lines in Fig.\ 11. For the 
  left-most line, the reed is shut because we are out on the flat portion of 
  the curve. For the right-most position, there is no flow because the pressure 
  inside the mouthpiece has adjusted to be the same as the mouth pressure 
  $p_0$. The waveform of flow rate shows a pair of sharp pulses during every 
  cycle, as a puff of air is let through while the system shifts rapidly 
  between these two positions. The rapidity of that movement explain why the 
  red points in Fig.\ 16 are rather sparse in the middle of the plot: the 
  simulation is only calculating the response at discrete sampled times, so we 
  do not see the intermediate positions marked in the plot. 

  Finally, it is interesting to hear the sounds of these various simulations. 
  These are presented in Sounds 1--6, in the order of increasing mouth 
  pressure. Each of these examples give 2 seconds of sound, including a 
  transient: the starting conditions are the same in every case. Sound 1 is so 
  quiet that you have to listen carefully to hear it at all. This is the case 
  just above the threshold: not only is it quiet, but it also has a very long 
  transient as the sound builds up. The ``negative resistance'' effect from 
  Fig.\ 9 is only just strong enough to overcome the energy dissipation in the 
  linear model of the clarinet tube (described by the reflection function from 
  Fig.\ 10), so the unstable growth of almost-sinusoidal oscillation is very 
  slow. 

  \aud{auds/aud-6f675170-plot.png}{\caption{Sound 1. The pressure waveform from 
  the simulation in Fig. 12 with $p\_0=2.13$. The sound is very quiet, to allow 
  for the louder sounds in subsequent cases.}} 

  \aud{auds/aud-e52ca6d5-plot.png}{\caption{Sound 2. The pressure waveform from 
  the simulation in Fig. 12 with $p\_0=2.15$.}} 

  \aud{auds/aud-ac356013-plot.png}{\caption{Sound 3. The pressure waveform from 
  the simulation in Fig. 12 with $p\_0=2.2$.}} 

  \aud{auds/aud-41fd5452-plot.png}{\caption{Sound 4. The pressure waveform from 
  the simulation in Fig. 12 with $p\_0=2.3$.}} 

  \aud{auds/aud-90bf30a6-plot.png}{\caption{Sound 5. The pressure waveform from 
  the simulation in Fig. 12 with $p\_0=3$.}} 

  \aud{auds/aud-5f05ed16-plot.png}{\caption{Sound 6. The pressure waveform from 
  the simulation in Fig. 12 with $p\_0=4.8$.}} 

  Working through the other cases, you should hear the sound getting 
  progressively louder and brighter, as we expected from looking at the 
  waveforms. You can also hear the transient getting shorter. Do these really 
  sound like a clarinet? Well, to my ears the sound, and especially the 
  progression of sounds as pressure is increased, is quite reminiscent of how a 
  clarinet behaves. For a very first demonstration of a computer model of a 
  nonlinear instrument, it seems quite promising. 

  But the computer is not a very good clarinettist! The sounds all seem very 
  robotic or mechanical. That is not a surprise: no effort has been made to 
  apply subtle shaping of the mouth pressure, as a real player would do. These 
  transient sounds simply switch on a fixed value of mouth pressure, with a 
  rather arbitrary initial condition for the internal pressure in the tube. 
  Furthermore, the model is also highly simplified. It omits many aspects of 
  the physics of a real clarinet, such as the effect of the reed's own 
  resonance and the effect of the detailed bore shape of the instrument 
  (including any closed tone holes). Finally, all we have computed is the 
  pressure waveform just inside the mouthpiece. This will not be the same as 
  the sound radiated by the instrument to the outside world. But we must learn 
  to walk before we run, and we have made a promising start. We will look in 
  much more detail at wind instruments in Chapter 11, after we have looked at 
  bowed-string instruments in Chapter 9. 

