  Up to now, we have been concentrating on linear systems. In the previous 
  section we saw examples of musical instruments that were “almost linear”, but 
  with nonlinear effects beginning to show, and sometimes proving important for 
  the sound. Shortly, we will look at some seriously nonlinear instruments. But 
  before that it is useful to step back a little to give a brief survey of what 
  we might expect to see, and what methods we might bring to bear in order to 
  make sense of nonlinear phenomena. 

  This will not be an easy task, and you should not expect a complete, 
  organised account of everything nonlinear systems can do. “Nonlinear” is one 
  of those terms that is defined purely by its negative. “Linear” means 
  something definite: linear systems are amenable to systematic discussion, as 
  we illustrated in Chapter 2. But “nonlinear” simply means “not one of 
  those…”. It is a bit like the term “agnostic”, which only tells you something 
  about what a person does not believe: it tells you nothing about what they do 
  believe. 

  When nonlinearity becomes important, some of the concepts we have been using 
  regularly up to this point become useless; or at the least, questionable and 
  potentially misleading. The prime example is the idea of a frequency response 
  function. We have been showing various plots of frequency response, for 
  example using bridge admittance to characterise the behaviour of the soundbox 
  of a stringed instrument. Well, think right back to Chapter 2. That was where 
  we first met the idea of frequency response plots, which relied on the idea 
  of linearity in two critical ways. 

  First, we learned that sine waves are special waveforms. For any linear 
  system, if you feed a sine wave in you get another sine wave out, at the same 
  frequency. Its amplitude and phase may have been changed, and those changes 
  may vary with the frequency of the particular sine wave, but the waveform is 
  always sinusoidal. The second crucial property of linear systems is 
  superposition: if you can express the input as a sum of separate terms, and 
  if you know how each of those terms separately is affected by the system, 
  then you simply add the separate outputs together to find the total response. 
  We applied this idea to Fourier analysis: express the input as a combination 
  of sinusoidal terms, work out how each term is affected by knowing the 
  frequency response function, then add these output sinusoids together to find 
  the actual output. 

  Both these properties go out of the window with a nonlinear system. We can 
  give very simple illustrations to demonstrate that claim. Let’s start with 
  the “sine wave in, sine wave out” property. Think about an electrical 
  amplifier, like the ones in home hi-fi systems. An ideal amplifier simply 
  takes an input electrical signal, and boosts it by a certain factor so that 
  it can drive a loudspeaker. An amplifier that behaves like this certainly 
  satisfies the sine wave property: every input waveform, sine waves included, 
  should produce output with the identical waveform except for the amplitude 
  being increased. 

  Well, there is a problem. Any amplifier is likely to have a maximum output 
  voltage that it is capable of producing — often related directly to the 
  voltage of the power supply which provides it with electrical power. Now 
  suppose you feed in a sine wave, with an amplitude big enough that a 
  correctly amplified version would exceed this limit. What is likely to happen 
  is a type of nonlinear response known as clipping. The peaks and troughs of 
  the output sine wave will be chopped off at the amplifier’s maximum voltage. 
  The result, in idealised form, will be an output waveform like the red curve 
  in Fig.\ 1, instead of the dotted black curve which represents a correctly 
  amplified sine wave. 

  \fig{figs/fig-8ff9a099.png}{\caption{Figure 1. Nonlinear response of an 
  amplifier: the dotted sine wave is the desired output, but if clipping occurs 
  the actual output will be like the red curve.}} 

  What can we say about the clipped output waveform? Well, it certainly isn’t a 
  sine wave. But it is still a repetitive, or periodic, waveform. We know from 
  section 2.2.1 that any periodic waveform can be expressed as a Fourier 
  series, which will consist of a sine wave at the fundamental frequency (the 
  frequency of our input sine wave), plus a series of sine waves at exactly 
  harmonic frequencies. So here we meet one of the characteristic effects of a 
  nonlinear system: it might generate harmonics of a frequency you feed in. In 
  the case of the hi-fi amplifier, this would be a bad thing. You don’t want 
  your amplifier to change the sound of what you are listening to, just make it 
  louder. But the sound of the clipped waveform will be very different from the 
  sound of a sine wave: much brighter, because of all the high-frequency 
  components appearing in the Fourier series. You can hear in in Sound 1: first 
  the sine wave, then the clipped version. 

\audio{}

  Clipping isn’t always bad, though. A less harsh version of clipping, called 
  “soft clipping”, seems to be part of the reason for the enduring preference 
  of electric guitarists for valve (vacuum tube) amplifiers. Modern solid-state 
  amplifiers may be more likely to produce “hard” clipping of the kind 
  illustrated in Fig.\ 1 and demonstrated in Sound 1. The soft clipping effect 
  enters more gradually as the input level approaches the amplifier's limit: a 
  schematic example is shown in Fig.\ 2. Guitarists can make constructive use 
  of the effect, whereas hard clipping comes in more abruptly and simply sounds 
  harsh. This is a first, tiny, glimpse that a musician may make constructive 
  use of nonlinear effects. That will be a major theme of this chapter: we will 
  meet examples in many different guises. 

  \fig{figs/fig-f7eb68d7.png}{\caption{Figure 2. Schematic plot of an amplifier 
  showing ``soft clipping'', to be contrasted with the ``hard clipping'' in 
  Fig. 1.}} 

\audio{}

  The idea of superposition also goes out of the window with any nonlinear 
  system. We have already seen an example of this in the discussion of “phantom 
  partials” in the previous section. The nonlinearity we looked at in section 
  7.4.1 was the simple example of a square law. What we found was that if you 
  take a single sine wave and square it, the result is a sine wave at double 
  the frequency you started with. This is a rather extreme example of 
  ``harmonic generation'': the nonlinearity gives an output containing only 
  second harmonic, with none of the original fundamental frequency. 

  So what happens if we take two sine waves at different frequencies and put 
  them through the square-law nonlinearity? Taking each one separately and 
  squaring it would give two sine waves at the two doubled frequencies. But we 
  saw in section 7.4.1 that when you add the two sine waves together and then 
  square them, the result includes extra sinusoidal terms at the sum and 
  difference of the two original frequencies. This is definitely not a 
  superposition of the two separate outputs. 

  There is another idea that we have made much use of, but which becomes 
  questionable in the presence of nonlinearity: the idea of modes of vibration. 
  Modes give a very powerful and general tool for understanding the vibration 
  of linear structures. Lord Rayleigh, back in the 19th century, was intimately 
  familiar with that fact, and it has been a cornerstone of vibration theory 
  ever since. Well, people do sometimes talk about “nonlinear modes”, but this 
  is a somewhat contentious idea which is still a topic of current research. As 
  a rule of thumb we should not expect discussions of nonlinear effects to 
  centre around modes in the same way as we have been using for linear 
  vibration. At the very least, we will exercise caution in any such 
  discussions. 

  All this seems very negative. Should we regard nonlinearity simply as a 
  nuisance, to be avoided wherever possible? In the context of a lot of 
  industrial problems involving noise and vibration control, the answer would 
  be “yes”. But for musical problems, the answer is an emphatic “no!”. At the 
  risk of offending percussionists, isn’t there something a bit limiting about 
  a lot of percussion instruments? Of course you can make interesting music on 
  a marimba, but the range of possibilities is much less wide than with the 
  human voice, or a violin. Nonlinearity is the key to this difference. 

  To give an impression of why that might be, it is useful to list some of the 
  possible consequences of nonlinearity: we will go into these in more detail 
  in subsequent sections. So far, we have only met some rather basic nonlinear 
  phenomena: harmonic generation (as in the clipping example);combination 
  frequencies (such as phantom partials); the pitch glide effect in which 
  vibration amplitude affects the frequency; and (in the lute string example) 
  the fact that some kind of impact might change the bandwidth of excitation, 
  and thus change the brightness of the sound. 

  But these examples give little hint of the richness of behaviour that can 
  stem from nonlinearity. The next phenomenon to mention gives the key to 
  entire families of musical instruments: the possibility of self-excited 
  vibration. Up to now, we have been discussing vibration and sound caused by 
  some obvious source of excitation: a drumstick, the clapper of a bell, the 
  finger or plectrum of a guitarist, the hammer of a piano. But what about the 
  wind instruments? A player blows steadily into the mouthpiece of a recorder, 
  for example, and a musical note may or may not emerge depending on whether 
  they get the technique just right. In some way, the player’s lungs must be 
  providing the energy to sustain the vibration and sound. But where does the 
  vibration come from? The same question could be asked about a violin: the 
  player pushes the bow across a string, and vibration appears “from nowhere”. 

  Linear theory can give no more than a hint of the answer to that question. We 
  will see in section 8.3 that it is sometimes possible to predict the 
  threshold for self-excited vibration using a linearised theory. Reed woodwind 
  instruments can illustrate the effect. If you blow very gently into a 
  clarinet or saxophone, you hear nothing except a bit of “breath noise” from 
  your blowing. But if you gradually increase your blowing pressure, at some 
  stage a musical note will “light up” from nowhere. That critical blowing 
  pressure is an example of a threshold of instability: we will see how that 
  works in section ?. But the only thing linear theory can predict is that once 
  you are beyond the threshold, the volume of the note will continue to grow in 
  an exponential way, louder and louder without limit. The real note doesn’t do 
  that: it settles to a steady amplitude. That kind of sustained self-excited 
  vibration always requires nonlinearity, whether we are talking about wind 
  instruments or bowed string instruments, or about the disastrous wind-driven 
  vibration of the \tt{}Tacoma Narrows bridge\rm{}. 

  We can return to the recorder or clarinet examples to illustrate the next 
  nonlinear phenomenon. You have gradually increased blowing pressure to find 
  the threshold, and achieved a decent musical note. If you carry on blowing 
  gradually harder, you may find that at some stage the instrument 
  spontaneously switches to playing a different note: a musician would call 
  this “overblowing”. This tells us that the recorder or clarinet is capable of 
  more than one kind of self-excited note, or regime of vibration, and also 
  that when you vary a parameter like the blowing pressure you may hit critical 
  values where a switch occurs. In mathematical accounts of the underlying 
  theory, such switches are called bifurcations. Regime switches like this are 
  crucial to many musical instruments: indeed, with a valveless brass 
  instrument like a bugle or natural horn this is the only way that a player 
  can change the note they are playing. Of course, in practice the player will 
  not only vary the blowing pressure: they will also do things with their lips, 
  varying their “embouchure” to allow them to control the switches reliably. 

  Now think about doing a similar experiment with a violin. Place your bow on a 
  string, move it at a steady speed, and gradually increase how hard you press 
  down. You don't find a clear-cut threshold this time, it all seems more 
  complicated than the wind instruments. When you are pressing very lightly, 
  you may create a variety of whistles and squeaks. Then you will find a value 
  of the bow force when the violin starts to make the kind of sound you (or 
  your violin teacher) are aiming for. But if you carry on increasing the 
  force, this nice musical note will eventually give way to some kind of 
  raucous “crunch”. What has happened is that regular periodic motion of the 
  string, responsible for the acceptable musical note, ceases to be possible. 
  Instead, something non-periodic (and non-musical) occurs instead. This is an 
  example of another important nonlinear phenomenon you will probably have 
  heard of: “chaos”. 

  The hallmark of chaotic response is something called “sensitive dependence”. 
  If you repeat the experiment with a new bow-stroke, you will produce a 
  similar-sounding “crunch” at the end, but if the two notes are recorded and 
  then compared carefully, the details will be different. Even a very good 
  player will not be able to take a bowed string deep into the raucous regime, 
  and achieve exactly the same motion twice. Tiny differences in the two bow 
  gestures will be amplified in the chaotic regime to produce results that are 
  completely different. 

  Good violinists probably never stray into this raucous, chaotic regime, but 
  they regularly encounter another manifestation of “sensitive dependence”. 
  During the transient at the start of each note, it is extremely hard to 
  control the bow gesture well enough that two identical waveforms can be 
  produced. By dint of many hours of practice, a good violinist will learn to 
  control their transients well enough that most of the time they can produce 
  something that sounds pretty much the same. But for some kinds of bow gesture 
  it is hard to do even that: we will go into some details on all this in 
  Chapter?. Such behaviour can only occur in a nonlinear system: linear systems 
  simply do not allow anything resembling sensitive dependence or chaos. 

  This example of bowed-string transients raises the final issue to be 
  mentioned in this introductory discussion. Let me remind you of something I 
  said right back in Chapter 1. I ventured a definition: “A musical instrument 
  is a contrivance which allows a performer to use gestures they are physically 
  capable of performing to make sounds that they like”. Every musical 
  instrument represents a balance between two competing desires. On the one 
  hand, the player wants an instrument with a wide repertoire of possible 
  sounds to give a good “palette” for their performance. But they have to be 
  capable of exercising rather precise control over these sounds, so they want 
  an instrument with good “playability”. The richer the potential palette, the 
  trickier the control problem is likely to become. We will investigate all 
  this and try to pin down at least some aspects of ``playability'' in Chapter 
  ?. 

  In summary, in this chapter and the ones that follow we will explore some of 
  the wonderful, but often frustrating, possibilities that arise from different 
  aspects of nonlinear behaviour in musical instruments. Profoundly nonlinear 
  instruments like the violin require a lot of practice before you can even 
  produce a single note with satisfying musical qualities, but in the hands of 
  an expert a violin can produce a whole world of interesting musical sounds. 
  Its motto might be “no pain, no gain”. But none of this musical potential 
  would be possible without the nonlinearity: as the software engineers would 
  put it, nonlinearity is not a bug, it’s a feature. 

