  Imagine you are visiting a violin dealer, choosing a new instrument. Perhaps 
  you have your teacher with you, or a colleague from your orchestra. You play 
  musical fragments to each other on various instruments, and you both try to 
  put into words what you hear and feel. You might make comments like ``This 
  one is more rich, a bit quiet under the ear, but a little weak on the A 
  string''. Straight away, many questions are raised in a scientist's mind. 
  Does everyone really use words like ``rich'' to mean the same thing? Does the 
  listener experience the sound the same as the player? If you came back the 
  following day and tried the instruments again, perhaps with a different 
  friend or in a different mood, would you make the same judgements? Would you 
  even be confident of recognising which instrument was which, without being 
  told? 

  There is more. The range of instruments you have tried will cost different 
  amounts: possibly spectacularly different, as illustrated by Fig.\ 1 where we 
  see three violins with market values ranging over 3 or 4 orders of magnitude. 
  How is it possible for instruments looking so similar, made from similar 
  materials, to be different enough to justify such a remarkable range of 
  market values? In the extreme: ``everyone knows'' that the very best violins 
  were made by Antonio Stradivari in Italy in the decades around 1700. So what 
  is the secret of these magical instruments? 

  This is the wrong question. We should first ask things like: Can we really 
  tell old Italian violins apart from more recent instruments? Do players or 
  listeners really prefer them, even when they haven't been told which is 
  which? But then: What exactly do we mean by ``prefer'' here? Does every 
  player prefer the same instrument, so that there really might be a valid 
  concept of a ``best violin''? Is a beginner looking for the same thing as a 
  competent amateur player in a string quartet, or an international soloist? 
  Even for a single violinist, is it obvious that the same instrument is 
  ``best'' for different styles of music, or for performing in different 
  acoustical settings? 

  We would like to bring some kind of scientific method to bear on all this, so 
  that we could draw conclusions backed by solid evidence. This leads us into 
  the world of psychoacoustics, a rather different kind of science from what we 
  have been talking about so far. This discipline is not based primarily on 
  physics and laboratory measurements. A typical experiment involves many 
  volunteers, listening with headphones to many groups of barely 
  distinguishable sounds, and trying to discern which is the ``odd one out'' of 
  each group. The purpose is to build up enough data to map out statistically 
  reliable thresholds for discrimination, or correlations between physically 
  measurable quantities and judgements of such things as ``richness''. The 
  style and limitations of this kind of experiment will have important 
  implications for addressing musical questions. Furthermore, the questions 
  raised so far are not a good place to begin delving into the world of 
  psychoacoustics. We need to start with simpler questions: but ``simpler'' 
  will prove to be only relative. Everything to do with perception turns out to 
  be complicated! 

  The purpose of a psychoacoustical experiment is to reverse-engineer an aspect 
  of human perception of sound. In the usual way of scientists, the researcher 
  will try to break the complicated totality down into small pieces, then look 
  at each piece separately. But from the perspective of the brain of the test 
  subject, this is a profoundly unnatural thing. As we saw back in Chapter 1, 
  your brain is constantly trying to combine the output from a host of 
  low-level and high-level feature detectors, and interpret the result in terms 
  of what is going on in the world around you. The experimenter wants to focus 
  on a single low-level detector, such as one to decide the relative loudness 
  of two sine waves at different frequencies. But unintended features of the 
  detailed procedure may bring in the influence of other detectors. From the 
  experimenter's perspective, this is a source of bias. But in the bigger 
  picture, it is telling us something else about the processes of perception. 

