

  First, another technical result must be pulled out of the hat: the idea, 
  trailed in the previous chapter, of Fourier analysis. The idea, essentially, 
  is that any waveform whatsoever can be built up by adding together sine 
  waves. Fourier was a French mathematician who, among other things, travelled 
  with Napoleon's army and was for a while made governor of Lower Egypt. He 
  discovered the result we are interested in while studying the flow of heat, 
  but the mathematical result applies equally well to other waveforms. 

  It is easiest to see what is going on if we look first at a periodic (or 
  repeating) waveform: we will consider a sawtooth wave. Fourier showed that 
  the sawtooth wave can be built up from a recipe, by taking a certain amount 
  of sine wave at the same frequency as the sawtooth then adding, in an 
  appropriate phase, a certain amount of the second harmonic, a sine wave at 
  twice the frequency, plus a certain amount of the third harmonic at three 
  times the frequency, and so on. Figure 1 shows how it works: the left-hand 
  column shows the successive sine waves being added, while the right-hand 
  column shows the sawtooth waveform gradually building up. By adding more and 
  more terms, the result gets closer and closer to a perfect sawtooth wave. You 
  can hear the effect of adding these successive terms in Sound 1. 

  \fig{figs/fig-d2971ee6.png}{\caption{Figure 1: A sawtooth wave being built up 
  by adding sine waves. Left column: the successive sine waves; right column: 
  the cumulative waveform at each stage. Only the first six terms are shown, 
  but the process could continue indefinitely, making the waveform 
  progressively more accurate.}} 

\audio{}

  The same idea can be applied to any other repetitive waveform: there is 
  nothing special about the sawtooth wave. A little less intuitively, the 
  approach can also be applied to non-periodic waveforms like the force pulse 
  in the upper trace of Fig.\ 1 of section 2.1. The difference is that you have 
  to add a bit of sine wave at every frequency, not just at separate harmonic 
  frequencies. The mathematical procedure for calculating the ``sine wave 
  recipe'' of a periodic waveform is called a Fourier series, and the 
  equivalent procedure for a general non-periodic waveform is called a Fourier 
  transform. This leads to a piece of jargon which may be familiar: the usual 
  computer routine for calculating this important recipe is called the ``Fast 
  Fourier Transform'' or FFT. The process of turning a recorded sound into an 
  MP3 file, for example, makes heavy use of the FFT. 

  So back to the drum. If it behaves as a linear system, then we know what 
  happens to any sine wave which is put in: nothing at all, apart from scaling 
  by an amplitude factor and shifting in phase. We aren't really interested in 
  sine wave input, though: the input we want is the force pulse from the 
  beater. But Fourier says that we can express that pulse as a combination of 
  sine waves. We could apply each of those sine waves separately to the drum, 
  getting a scaled and shifted sine wave out in each case. Now the second 
  important property of linear systems comes into play, known as the 
  ``principle of superposition''. This is a grand name for a simple idea: if 
  the input to a linear system is the sum of several components, then the 
  output is simply the sum of the corresponding separate outputs. 

  We can take each separate sine wave from the Fourier recipe, find the output 
  of the drum to that sine wave, then add them all together and we should get 
  the actual output, the vibration waveform from the lower trace of Fig.\ 1 of 
  section 2.1. It follows that we can calculate the response of the drum to any 
  kind of force input provided we know what it does to sine waves. In other 
  words, everything we need to know about the drum's behaviour is coded 
  (somehow) into the amplitude scale factor and the phase shift. Now remember 
  that these might vary with the frequency of the sine wave, so we could plot a 
  graph against frequency of the scale factor and the phase shift: this 
  combined information is called the frequency response function of the drum. 
  It is the first step in obtaining an ``acoustic fingerprint'' of the drum. We 
  will see some examples soon. 

  For the next step we need to know a little about the theory of vibration. The 
  essence of vibration lies in the interaction between restoring force and 
  inertia. If a system is displaced a little from a position of stable 
  equilibrium, a restoring force of one kind or another will make it tend to 
  return. Inertia will guarantee that, left to its own devices, the system will 
  overshoot the equilibrium position. A restoring force with the opposite sign 
  will then act, and some kind of oscillatory behaviour will result, at a 
  frequency governed by the balance between the restoring force and the 
  inertia. Restoring forces can arise from many physical causes, for example 
  gravity (as in a pendulum), elastic internal stresses (as in a steel spring 
  or a wobbling jelly) or change in pressure of a fluid (as in sound waves in 
  an organ pipe, or air compressed in a bicycle pump with your thumb blocking 
  the outlet). 

  The archetypal vibrating system consists of a mass supported by a spring. The 
  sketch in Fig.\ 2 shows a schematic version. The mass provides the inertia, 
  and the spring provides the restoring force. If the spring behaves in a 
  linear fashion, that restoring force obeys Hooke's law: the force is 
  proportional to the distance the mass has moved away from the equilibrium 
  position. The constant of proportionality is the stiffness of the spring, 
  called $k$ here. The units of $k$ are force per unit displacement, so N/m or 
  N m$^{-1}$. Applying Newton's law of motion to this simple system, two things 
  are easily shown: the displacement of the mass during free vibration is a 
  sine wave, and the frequency of that sine wave is determined by the ratio 
  $k/m$. Specifically, the frequency in Hz is given by $\frac{\sqrt{k/m}}{2 
  \pi}$. More detail is given in the next link. 

  \fig{figs/fig-94beb439.png}{\caption{Figure 2: Schematic diagram of the 
  simplest vibrating system: the mass m can vibrate in a vertical direction, 
  restrained by the spring k.}} 

  We could note a small detail here. We saw $k$ and $m$ appearing above, and 
  this is the first time that mathematical variables like this have appeared in 
  the main text. They appear in italic, because this is a typographical 
  convention. They also appear in a different display font, which is simply a 
  side-effect of the method used to incorporate mathematical text throughout 
  this web site. The result may be that they look a little odd, and perhaps a 
  little intimidating if mathematics is not something really familiar to you. 
  The message is: ``Don't panic!'' Simply read them in your mind as like any 
  other kind of ``k'' or ``m''. We also had the Greek letter $\pi$, ``pi''. 
  Most people are probably familiar with this one, but we will occasionally see 
  other Greek letters: it is a habit of mathematicians to use Greek letters for 
  certain things. I will mention their names when they first appear: please 
  don't regard this as a patronising gesture on my part, I am simply trying to 
  cater to the widest audience. 

  Now to delve deeper into the theory of vibration. The particular aspects we 
  need are not new or controversial: they were all known to Lord Rayleigh in 
  the nineteenth century. Rayleigh is the great guru of acoustics. He made 
  other scientific contributions as well: he was the first to explain why the 
  sky is blue, and he won the Nobel prize for the discovery of argon. He had 
  serious health problems in his youth, and at the age of 30 he spent some 
  months sailing on the Nile. Not only did this improve his health, but he 
  started writing ``The Theory of Sound'', published a few years later. This 
  astonishing work is one of the few scientific books from the nineteenth 
  century which is still in active use as a reference text, rather than merely 
  as a historical curiosity. 

  The first question to answer is: how do we know that it is OK to treat the 
  drum as a linear system? The real answer to that question is mathematical, 
  and is summarised in the next link. The key conclusion is that vibration of 
  any object will usually be linear provided the amplitude is small enough. How 
  small is small enough? That is a matter for empirical checking by experiment, 
  but the result is that most ``musical vibration'' is small enough that we can 
  get away with assuming linearity nearly all the time. It is not always true: 
  we will meet a striking exception later (this is a pun: think J. Arthur 
  Rank). Also, of course, we should not forget the warning from the previous 
  chapter that small effects cannot necessarily be ignored in musical problems. 
  Nevertheless, on the principle of not running before we can walk, we will 
  take advantage of the simplifying insights from linear theory for as long as 
  we can get away with it. 

  Back to the drum again. After the beater has bounced off, the drum continues 
  to produce output (i.e. to vibrate) when no input is being applied to it. The 
  general theory of linear vibration reveals that this is only possible in a 
  particular way. The drum has certain natural frequencies, which are 
  characteristic of that particular drum. Each of these natural frequencies is 
  associated with a particular pattern of vibration of the drum skin, called a 
  vibration mode. Modes of a linear system like the drum obey the ``principle 
  of superposition'': they can vibrate simultaneously, each at its own natural 
  frequency, uninfluenced by each other. Remarkably, each separate mode behaves 
  exactly like the simple mass-spring oscillator (known as a ``simple harmonic 
  oscillator'') discussed above. 

  The first few mode shapes for a perfect ``textbook'' drum are illustrated in 
  Fig.\ 3. The lowest mode, in the top left of the figure, has the whole skin 
  vibrating up and down, but all the higher modes have nodal lines, which are 
  lines where there is no displacement of the skin. The second mode has a 
  single line along one diameter. Higher modes have a mixture of nodal 
  diameters and nodal circles, and the patterns get more and more complicated 
  as we go to higher frequencies: the wavelength of the skin motion gets 
  shorter. Mode shapes of any vibrating structure behave in a way generally 
  similar to this: the simplest thing you can imagine for the lowest mode, then 
  more and more complicated patterns with shorter and shorter wavelengths as 
  the frequency goes up. 

  \fig{figs/fig-6adeb7ce.png}{\caption{Figure 3: The vibration modes of an 
  ideal circular drum. All modes in a particular column have the same number of 
  nodal circles, while all modes in a given row have the same number of nodal 
  diameters. The first few are shown here: the pattern continues indefinitely 
  to the right and downwards. Above each mode shape is listed its natural 
  frequency, relative to the fundamental mode.}} 

  There is one slight complication in this problem: nearly all the drum modes 
  actually occur in pairs. The second mode, with a frequency 1.59 times that of 
  the fundamental, is an example. The perfectly circular drum has no preferred 
  direction, so how could it know where to put that single nodal diameter? The 
  answer is that it doesn't know: the diameter could be put anywhere. One 
  particular choice is shown in the figure, and its natural companion would 
  have a nodal line rotated at $90^\circ$ so that it passed through the points 
  of maximum displacement of this one. Now, recall that modes can vibrate 
  simultaneously, independent of each other. It turns out with a small 
  algebraic calculation described in the next link that if you have these two 
  particular modes vibrating simultaneously, they can combine to give the same 
  shape again, but with the nodal diameter rotated. By choosing the amplitudes 
  of the two modes suitably, the diameter can indeed be put anywhere at all. 

  The sound of the drum after the input forcing has stopped can only, ever, 
  consist of a mixture of these particular frequencies. The drummer can control 
  the relative loudness of the different frequencies in the mixture by doing 
  the things drummers are familiar with: choosing a hard or soft drumstick, and 
  choosing where to hit the drum: a hit in the centre will produce a different 
  sound from one near the rim. There is a useful formula that reveals exactly 
  how this effect of striking position works in terms of the mode shapes of the 
  drum. It is a good example of the kind of ``deep theory'' mentioned in the 
  first chapter: one would expect this formula to apply, to a good 
  approximation, to any structure vibrating with small amplitude. The details 
  are given in eq. (11) and the associated discussion in the next link. This 
  one equation captures a great deal of important information about vibration 
  behaviour, not only for drums but for just about every other instrument we 
  will look at. 

  What about the reason a drum sounds different with different choices of 
  drumstick? That all comes down to Fourier analysis again. We have already 
  said that the force pulse from the bouncing drumstick can be broken down into 
  a mixture of sine waves. When you change from a soft to a hard stick you 
  change the shape of the force pulse, and that changes the mixture. Just as 
  you would guess, a hard stick bounces quickly off the drum, producing a very 
  narrow pulse of force. A soft beater stays in contact much longer, giving a 
  broader pulse. Now there is a very simple relation between the width and the 
  frequency mixture: wide pulses consist mainly of low frequencies only, but 
  narrower pulses contain a lot more high frequency. As a rule of thumb, if the 
  length of the pulse is $T$ seconds, then it generates frequencies up to about 
  $1/T$ Hz. Rayleigh thought about this problem, and gave a simple approximate 
  formula: the next link gives some details. A good first guess for the shape 
  of the force pulse is a half-cycle of a sine wave, and the Fourier analysis 
  of that waveform can be calculated mathematically. The result, as far as the 
  drummer is concerned, is that a soft beater will only excite the first few 
  natural frequencies of the drum, but a hard stick will excite more of them. 

  The actual sound of the struck drum involves one extra ingredient: it does 
  indeed consist of a mixture of sine waves, each corresponding to a particular 
  vibration mode of the structure, but each of these sine waves decays in 
  amplitude as time goes on, because energy is gradually lost. To give a 
  complete ``vibration fingerprint'' of the sound we thus need to know three 
  things about each of these sine waves: the frequency, the amplitude and the 
  decay rate or damping. The decay rate determines whether the sound goes ping 
  (slow decay) or thud (fast decay). When a sound immediately strikes you as 
  ``metallic'' or ``wooden'', you are probably responding mainly to a 
  difference in typical decay rates between those two kinds of material. 

  So, the set of natural frequencies, mode shapes and decay rates can be 
  regarded as the vibration fingerprint of that particular drum. But just now, 
  we said the same about the frequency response function of the drum, so what 
  is the connection? The answer is simple. One way to apply force to a drum 
  which is approximately a sine wave is to sing or hum gently, close to the 
  drum-skin. Tympanists routinely do this when they are tweaking the tuning of 
  an instrument. They are listening for resonances, frequencies where the drum, 
  as it were, sings back at you. These resonances are nothing other than the 
  natural frequencies we have just been talking about. Now recall that the 
  frequency response function tells us the amplitude scaling factor, which 
  varies with frequency. Resonances are frequencies where you get a lot of 
  output for not very much input: in other words, they will show up as peaks in 
  the plot against frequency. 

  \fig{figs/fig-b1e0c63a.png}{} 

  \fig{figs/fig-f78fb884.png}{} 

  It is time to stop talking in abstract terms, and see the response of a real 
  drum. The small toy drum shown in Fig.\ 4(a) will be examined. The frequency 
  response function has been measured, by tapping the drum at one point with a 
  hammer equipped with a force-measuring sensor, and the motion at a nearby 
  point on the skin has been measured. The skin of the drum is very light, so 
  we do not want to fix any kind of heavy sensor to the surface because that 
  would change the vibration behaviour enough to annoy the drummer. Instead, 
  the motion has been measured using a non-contact sensor called a laser 
  Doppler vibrometer. A laser beam is shone at the point to be measured: the 
  red dot of the laser beam is visible in Fig.\ 4(b). A spot of reflective 
  paint is put on the drum at this point, so some of the beam is reflected 
  back. Now when the drum is vibrating, something happens to the reflected 
  beam: its frequency is changed a little by the \tt{}Doppler effect\rm{}, the 
  familiar phenomenon that makes the note of an ambulance warning horn change 
  as the ambulance comes towards you, passes and moves away. The vibrometer 
  detects this change, and turns it into an electrical signal proportional to 
  the vibration velocity of the drum-skin parallel to the laser beam. That 
  signal can be captured into a computer. 

  Now we follow the procedure described earlier to find the frequency response. 
  Both the hammer signal and the vibrometer signal are turned into combinations 
  of sine waves, using an FFT routine in the computer. For each frequency we 
  then take the ratio of amplitudes, output divided by input, and also the 
  phase difference between the two. For the moment we will just look at the 
  amplitude response: it is plotted as a function of frequency in Fig.\ 5. In 
  fact, two different versions are shown, found by tapping in two different 
  places while keeping the measuring point the same. 

  The amplitude varies over a huge range, so to make it easier to see what is 
  going on it has been plotted on a logarithmic vertical scale. The left-hand 
  axis shows the level, in terms of powers of ten. (The notation $10^2$ means 
  $10 \times 10$, $10^{-2}$ means $1/(10 \times 10)=0.01$ and similarly for 
  other positive and negative powers.) The right-hand axis shows the same 
  information on a decibel scale. Decibels tell us about ratios of amplitudes 
  rather than absolute amplitudes. The conventional scale for this purpose 
  means that every 20 dB on the vertical scale corresponds to a factor of 10 in 
  the amplitude, as can be seen by comparing the left and right scales. 
  Decibels are familiar from noise level meters. The ``loudness'' of sound is 
  usually measured in that way, because our ears work, more or less, on a 
  logarithmic scale: more about this in Chapter 6. 

  \fig{figs/fig-36d1584b.png}{\caption{Figure 5: Frequency response functions 
  of the toy drum from Fig. 2.6, for two different positions of the tapping 
  hammer. The numbers on the y-axis correspond to calibrated values: the 
  frequency response function at any given frequency is a velocity divided by a 
  force, so the units are meters per second per Newton, m s$^{-1}$ N$^{-1}$.}} 

  Before we leave this discussion of scientific background material, there is 
  one more thing to be mentioned about the response of linear systems. We have 
  seen that the frequency response function allows you to convert each 
  sine-wave component of input into the corresponding component of output. But 
  surely there should be a way to relate the output time waveform to the input 
  time waveform, without going via the counter-intuitive process of taking 
  Fourier transforms? Well, there is indeed such a procedure, called 
  convolution. The details are given in the next link. We won't need this 
  straight away, but it will become important in Chapter 6 when we want to talk 
  about how the sound of a violin is influenced by the body vibration. 

