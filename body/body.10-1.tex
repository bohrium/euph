  This “underpinnings” chapter is rather different in character from the 
  previous ones. It will not address a particular technical topic, like 
  acoustics or nonlinearity. Rather, it will give an overview of a very diverse 
  range of activities that come under the umbrella of “measurement and 
  experimentation”. This will give an excuse to include various interesting 
  odds and ends. For this particular chapter, don’t be afraid to check the side 
  links: some of them are picture collections or “how to…” guides aimed at 
  instrument makers wishing to do measurements themselves. 

  Some experiments involve expensive laboratory-based kit, like a laser 
  vibrometer or a scanning electron microscope. At the other extreme are the 
  kind of “what will happen if…” explorations that instrument makers often 
  pursue. These will be workshop-based, and may not involve any equipment 
  beyond the maker’s usual tools. But there is a wide range of things that fall 
  between these two extremes. Some “science” measurements need only simple 
  equipment. On the other hand, many instrument makers with a systematic 
  mindset want to keep detailed records of each instrument as they build it. 
  Such record-keeping would begin with dimensions and weights, but may extend 
  to more technical things like wood properties, vibration frequencies of 
  components such as violin plates, and acoustic response measurements of one 
  kind or another on the completed instrument. 

  Measurements of all these kinds can involve traps for the unwary, and to get 
  the best out of a measurement involves knowing some “tricks of the trade” for 
  avoiding them. In the course of this chapter we will meet some examples. But 
  before getting down to specific details, it is useful to explore the general 
  question of why people might want to do measurements at all. There is no 
  single answer to that: different people bring different agendas. This chapter 
  is about physical measurements, so I will not talk again about 
  psychoacoustical experiments: these are very important, as discussed back in 
  Chapter 6, but they involve a different mindset, and a different set of 
  skills, pitfalls and tricks of the trade. For our present purpose, there are 
  two main types of agenda, coming from instrument makers and physicists. 

  Instrument makers are likely to be motivated to do measurements or 
  experiments by practical concerns of making better instruments, or addressing 
  particular problems that have arisen. They may ask questions like “How do I 
  make another one just like this?”; “Can I replace traditional materials, 
  either with something better or with something more sustainable?”; “How 
  should I adjust constructional details to address a tonal particular 
  problem?” or simply “What will happen if…?”. For some instruments makers, 
  there are also forensic questions about accurate copying of famous old 
  instruments. There is an old joke: “How many violin makers does it take to 
  change a light bulb?” The answer: several — one to do the job, the rest to 
  debate how Strad would have done it. But it is not only the violin world that 
  has iconic makers of the past: Lloyd Loar mandolins and Hermann Hauser 
  guitars, for example. 

  The physicists’ agenda is rather different. They all want to know “how does 
  it work”, but they fall into three different camps on how to approach that 
  question (although some individual physicists have a foot in more than one of 
  these camps — and some must have three feet!). There are those who are 
  primarily interested in theory and mathematical models. There are those who 
  aim to understand behaviour by detailed computer modelling. And then there 
  are those who are dedicated experimentalists. 

  Measurement has a role for all three approaches, but it feels different to 
  committed members of each camp. For the theorists, the role of measurements 
  is to test, confirm and calibrate the theoretical models: but those models 
  really hold the key to understanding. The computational folk are somewhat 
  similar. Measurements are for “validation” of the computational model: once 
  it is well enough validated, they expect to use the computer for the real 
  business of “doing science”. The committed experimentalists take a quite 
  different view. They are likely to assert that “measurement is ground truth”. 
  Theory is only useful if it agrees with the measurements, otherwise the 
  theory must surely be wrong? In extreme form, these people may think “surely 
  we can sort the whole thing out if only we had enough data?” Indeed, this is 
  quite a fashionable idea these days, with “big data” all around us. 

  The committee that awards Nobel Prizes for physics seems to have a strong 
  bias in favour of experimentalists. Although they have, of course, awarded 
  prizes to theorists like Einstein, they sometimes delay for a very long time, 
  and only award a prize once there is experimental confirmation. Einstein’s 
  prize was not for his crowning achievement, the theory of relativity, but for 
  much less well-known work on the photoelectric effect, which had more 
  immediate experimental evidence. There have been controversial decisions…but 
  this isn’t the place to air those. No-one has yet been given a Nobel prize 
  for studying musical instruments --- although some Nobel laureates like 
  Rayleigh and Raman were also interested in musical problems. 

  Cutting across these two broad agendas, there are several specific 
  motivations for doing a measurement or experiment. The first is measurement 
  aimed purely at data gathering. This might be an instrument maker wanting to 
  measure wood properties for their records, or wanting detailed geometric 
  information on a famous old instrument, but it could also be a physicist in 
  the third camp, collecting extensive acoustical and vibration data on an 
  instrument with “big data” and “data-driven science” in mind. The first two 
  are uncontroversial, but the third one raises important doubts: I’ll say more 
  about that shortly. 

  A second motivation for measurement is particular to instrument makers: 
  monitoring for quality control. A potential client may say “I really liked 
  the instrument you made for so-and-so: could you make me one like that?” The 
  maker immediately has a challenge: they are working with natural materials, 
  and no two pieces of wood will be exactly the same. The client doesn’t just 
  want an instrument that looks like the one they admired: they want one that 
  sounds like it. So the maker needs to try to create a similar acoustical 
  performance in the new instrument, despite inevitable differences in the raw 
  materials. Can measurements help them in this task? 

  A third category of motivation is rather different: measurement for 
  hypothesis testing. You have an idea for how something works, or what will be 
  the effect of a particular modification, so you design an experiment 
  specifically to probe this idea. Crucially, such experiments need not make a 
  “better” instrument, and they need not be subtle. The aim is to test the 
  hypothesis, and that test may be more clear and convincing with a 
  deliberately crude modification, far larger than you would necessarily want 
  to use in a real instrument. If your hypothesis stands up to this test, you 
  can then use it to inform more subtle and graduated changes. 

  From the perspective of a physicist, such experiments illustrate the classic 
  “scientific method”. But an instrument maker might also perform such tests, 
  not in order to extend scientific knowledge but as part of the practical 
  business of developing their skill. As a simple example, they may have an 
  idea about what will happen if they change the weight of a violin bridge. So 
  they might deliberately fit a very heavy bridge and a very light one, to see 
  if these extreme changes give tonal effects that follow what they were 
  expecting — even if the light bridge was too fragile to stand up to the 
  rigours of normal violin playing. 

  The final category of motivation is measurement as exploration. Try some 
  changes, and see (or hear) what happens. Such experiments can blur into the 
  previous category: perhaps you have only a rather vague hypothesis, but you 
  do the test anyway. The example of changing bridge weight could fit here too: 
  perhaps you don’t have a clear idea of what the effect of weight might be, 
  but it is an easy experiment so let’s try it and see… 

  But there is a very important snag. My example of changing bridge weight 
  seems clear, but it is misleading. Violin makers and violinists already know 
  quite a lot about the effect of bridge weight, because players regularly make 
  temporary changes to the weight, by fitting a mute. Makers are indeed 
  interested in bridge adjustment, but not just in the effect of weight. There 
  are many options for tweaking a violin bridge: wood choice, spacing of the 
  feet, thickness, and all manner of details of the shaping of those elaborate 
  cut-outs. Figure 1 reminds you of the complicated shape. 

  \fig{figs/fig-48988678.png}{Figure 1. A violin bridge.} 

  Well, you could still go for the “try it and see” approach, but you would 
  need to make many, many bridges to cover all the different variations, and 
  also probe how they work in combination with each other. In practice you 
  would probably only do a small number, then try to guess the bigger picture 
  from these limited results. This is fraught with danger! Attempts to 
  generalise and spot patterns from inadequate data have sometimes led to 
  important insights, but more often they have led to misleading claims, about 
  musical instruments and in the wider world. 

  This simple example illustrates a fatal weakness in the notion that an 
  experimenter could “just collect enough data and then see the answer”. Of 
  course there have been triumphs of “big data” and Artificial Intelligence 
  systems to extract patterns — the targeted advertisements in your web browser 
  are a familiar example, although whether you regard those as a triumph may 
  depend on whether you are an advertiser or a consumer. In any case, such 
  systems can only be developed and deployed in a context where large resources 
  can be devoted to the problem. That is never an option in the world of 
  musical instruments. 

  The preferable alternative is to combine the data gathering with some kind of 
  modelling, with a view to finding an informed way to interpret the 
  measurements. This could take several different forms: theoretical models 
  could involve technical mathematical theory of some kind or less formal 
  logical argument about how different factors might interact; alternatively, 
  the model could be a computer program that attempted to capture the essential 
  physics of the system. 

  We have already seen an example of such modelling in a context relevant to my 
  bridge-adjustment example, back in section 5.3 when we talked about the 
  “bridge hill” of a violin. A simple idealisation of one aspect of bridge 
  behaviour was represented in Fig.\ 12 in that section, and then applied to a 
  crude model of the violin body in order to see how the “hill” feature could 
  arise as a result of coupling of the bridge and body behaviour. The model 
  predictions (Fig.\ 8) matched, at least qualitatively, the measurement of 
  violin bridge admittance shown in Fig.\ 7. That match can give us some 
  confidence in using insights from the super-simple bridge model to make 
  informed guesses about how different aspects of bridge adjustment might 
  influence the behaviour. This then opens the way to replacing “scattergun” 
  exploratory experiments (with many, many bridges) with targeted 
  hypothesis-testing experiments, involving only a limited number of modified 
  bridges. 

  There is one important tool for generating insights from modelling which 
  deserves a special mention. This is an approach called dimensional analysis. 
  It is a very common experience that you know (at least approximately) the 
  governing mathematical equations for something, but you can’t solve them. 
  Dimensional analysis sometimes allows you to learn useful things about the 
  solutions by a very simple argument. 

  The idea is based on looking to see which physical parameters enter into the 
  equations, and thinking about the units they are measured in. We are 
  interested in mechanical problems, so all units can be reduced to 
  combinations of mass, length and time. For example, density is mass per unit 
  volume, so it has units kilograms per cubic metre: mass divided by length 
  cubed. For another example, the units of force can be deduced from Newton’s 
  law: force is mass times acceleration, so its units are kilograms times 
  metres per second per second, in other words mass $\times$ length / time 
  squared. There is one very simple thing we can say about any equation 
  expressing some aspect of physics: all the terms in the equation must have 
  the same units. You can’t add a mass to a length, for example; it simply 
  doesn’t make sense: what would 1~kg plus 1~m mean? 

  This simple insight is the basis of the method of dimensional analysis. Some 
  details are explained in the next link. I will illustrate the power of the 
  method by examples. Think first about a bending beam, like a xylophone bar or 
  the tines of a tuning fork. We found the governing equation for vibration of 
  beams back in section 3.2.1. For any particular problem, there are only three 
  physical parameters involved: the bending stiffness, the mass per unit 
  length, and the total length of the beam. As the next link explains, we can 
  use dimensional analysis to deduce that the frequency of any particular mode 
  of such a beam must be expressed in a particular form. It must be 
  proportional to the square root of the bending stiffness, inversely 
  proportional to the square root of the mass per unit length, and inversely 
  proportional to the square of the length. That is quite a lot of information 
  to be able to deduce simply by thinking about the units everything is 
  measured in! 

  We can go a bit further. Suppose our beam has a rectangular cross-section. 
  The analysis shows that the frequency must be independent of the width, 
  proportional to the thickness, and inversely proportional to the square of 
  the length. These “scaling laws” immediately tell you how to modify the shape 
  to make a set of xylophone bars or tuning forks tuned to different 
  frequencies. You can make the frequency lower by making the beam longer or by 
  making it thinner, or by some combination of those. Any change which reduces 
  the ratio of thickness to length squared by 6% will lower the pitch by one 
  semitone, for example. Conversely, a change of thickness and length which 
  kept that ratio the same would result in two bars or forks with the same 
  pitch, even though they looked different. It might have taken you quite a 
  long time to deduce this fact if you simply did measurements on bars of many 
  different shapes. 

  As a final step, notice what happens if you scale both the length and the 
  thickness by the same factor. A bit of cancellation then happens, so that the 
  frequency scales inversely with that factor. This is an example of a rather 
  general property: if you make a scaled replica of a structure, with all 
  dimensions scaled by the same factor, then all the frequencies will scale by 
  the inverse of that factor. For example if you scale down a xylophone bar, or 
  a violin, by 6%, all the frequencies will rise by one semitone (assuming that 
  you use the same material). 

  This idea of making scaled structures gives a link to the second important 
  idea that is thrown up by dimensional analysis. Most physics problems are 
  more complicated than a vibrating beam, and they involve a larger number of 
  physical parameters. You can still follow the methodology explained in the 
  previous link, but dimensional analysis will no longer tell you the complete 
  answer to your problem. But it tells you something else which is also 
  interesting. If the number of parameters is bigger than the number of units 
  involved (for mechanical problems this number is 3: mass, length and time) 
  then the analysis tells you that there are combinations of the parameters 
  which have no units: these are called dimensionless parameters. 

  Because they have no units, you cannot say anything about how the values of 
  those parameters affects the solution (without doing a lot more work, to 
  solve the equations). But what you can say is that if you change the 
  structure in such a way that the values of the dimensionless parameters do 
  not change, then the answer will also not change. This idea lies at the heart 
  of the familiar process of doing laboratory tests on scale models of 
  structures, such as the model aeroplane in the wind tunnel shown in Fig.\ 2. 

  \fig{figs/fig-6f36ca5b.png}{Figure 2. A scale model of an aeroplane being 
  tested in the Markham wind tunnel, at the Cambridge University Engineering 
  Department.} 

  Once you have built your scale model, how do you choose the air speed in the 
  tunnel, in order to reproduce more or less the same flow patterns around the 
  model as you would expect in the full-scale aeroplane? Fluid dynamicists are 
  very keen on dimensionless parameters, and they have defined many of them, 
  useful in different contexts. The most important thing you must do in a wind 
  tunnel test is to make sure to keep the same value of a dimensionless 
  parameter called the Reynolds number — the previous link gives some details 
  about what this number is and why it is important. 

  Another example of scale-model testing relates to the performance of 
  infrastructure like buildings or dams, for example in an earthquake. The 
  problem with a scale model in this context is that the total mass of a model 
  of the building or the dam scales down with the cube of the linear 
  dimensions. The result is that small models are, in relative terms, too 
  light. The solution to that problem, via a suitable dimensionless number, is 
  to make gravity stronger. This can be done by using a geotechnical 
  centrifuge, which spins your sample of soil and your model structure at a 
  high rate: exactly the same idea as the centrifuges used to train astronauts 
  to withstand the acceleration of space missions. Figure 3 shows what a 
  geotechnical centrifuge looks like. 

  \fig{figs/fig-48b94474.png}{Figure 3. Researchers inside the geotechnical 
  centrifuge at the Schofield Centre of Cambridge University. This centrifuge 
  can spin a ``bucket'' of soil with a scale model of a structure, using 
  centrifugal force to simulate an increase in the force of gravity in order to 
  match the relevant dimensionless parameter. The two researchers are standing 
  near the sample bucket, which is hinged so that it can pivot up as the 
  centrifuge spins. On the right-hand side is a counterweight. The whole system 
  is in an underground space, for safety in the event of something going 
  wrong.} 

  Returning to our main theme, there is an aside that we might note. The kind 
  of interplay between experiment and “theory” that we have been talking about 
  has an interesting parallel in the context of debugging computer models. If 
  you have never tried to write a computer program to solve a problem, you may 
  not be aware of the tricky and time-consuming process of debugging. Your 
  first effort to code something up invariably has errors in it. If you have no 
  idea at all what you expect the answer to look like, you are in danger of 
  believing the results of your first version that seems to run successfully. 

  But almost certainly, these results are wrong. How do you find out? The 
  process relies on you thinking about the problem, in order to generate some 
  ideas about how the solution ought to behave. This might involve special 
  cases of the parameter values which should result in recognisably extreme 
  behaviour, or it might involve cross-checks with standard physics results 
  (“it must be consistent with Newton’s laws, or conservation of energy”). So 
  you check these things out with your program. Probably something doesn’t 
  work, and after a bit of head-scratching you realise what you have done wrong 
  to create the disparity. 

  So how do you know when to stop looking for further bugs? There is no answer 
  to that, except that after a while you run out of ideas for cases to check. 
  The result is a bit counter-intuitive: the time it takes to debug a program 
  goes up, in proportion to the number of prejudices you can come up with about 
  the answer. The less you know, the quicker the process is — but of course 
  that doesn’t mean you get correct code quicker, it simply means that you stop 
  looking, and run the risk that there is still something wrong. This kind of 
  interplay with “theory” is crucial if you want to end up with a program that 
  has a good chance of being correct. 

  Exactly the same is true for interpreting experiments. It is only too easy to 
  do something wrong in the way a measurement is carried out, or in the way the 
  results are interpreted — these are equivalent to the bugs in a computer 
  code. Eliminating such problems involves, first, recognising that something 
  is wrong. You do that by scratching your head to come up with exactly the 
  same kind of prejudices about how the results ought to behave, and then 
  checking those aspects carefully in your measurement. We will see some 
  examples in later sections, when we look at particular types of measurement. 

  It is particularly important to think about this kind of interplay with 
  “theory” if you have done an undirected, exploratory investigation, and you 
  are now trying to spot patterns in the results. How do you recognise a 
  “significant pattern” and decide what it means? That is the million-dollar 
  question. The crucial first stage is to ask the question, and to be on the 
  lookout for patterns all the time — patterns that might be some interesting 
  feature of the results of your measurements, or might be indicating that 
  there is something wrong. How do you recognise “wrong”? Quite often from a 
  “gut feeling” based on experience. Data not meeting an expectation is not 
  necessarily “wrong” but it may be the germ which will lead to new 
  understanding. 

  As we emphasised right back in Chapter 1, there are some things for which 
  there is believed to be a good theoretical understanding, at least in general 
  terms. If you see something which contradicts previous experimental and/or 
  theoretical expectation, it either means you are on track for your Nobel 
  prize, or more likely that there is something wrong with your measurement or 
  your interpretation of it. Even if you are in fact on track for the Nobel 
  prize with some remarkable discovery which flies in the face of what everyone 
  believes, you still have to convince people that your results are right. So 
  in either case, seriously unexpected results have to be checked, checked and 
  checked again. Your new material shows up with ten times the 
  stiffness-to-weight ratio of any other known material? Better check your rig 
  by measuring something familiar, like steel or aluminium. You see nonlinear 
  response when everyone expects linear response? Better check that the 
  nonlinearity is really coming from the test object, not from some artefact of 
  your rig and the way it is put together. 

  To end this discussion of general issues surrounding the business of 
  measurement and experimentation, we should note one more very important idea 
  that must be kept in mind when planning any experiment, including the kind of 
  relatively informal experiment that an instrument maker might do. Suppose you 
  want to investigate the influence of bracing pattern on the soundboard of a 
  classical guitar — something guitar makers worry about a lot. You might think 
  of making three guitars, the same except that they have three different 
  bracing patterns. This would be a mistake! It would tempt you into a trap. 

  If you build the three guitars and then play them, no doubt they will all 
  sound somewhat different from each other. Are you hearing the characteristic 
  sound of the three different bracing patterns? Well, you might be, but you 
  could also be seriously misled. The problem stems from the fact that no two 
  guitars can be made exactly identical. Every piece of wood is different, and 
  every aspect of the making process is only controlled within a certain 
  tolerance: thickness distribution, details of glue joints, details of the 
  bridge, details of the setup. All those things will come out a little 
  different every time, even when the maker aims to keep them the same. The 
  differences of sound you hear between your three guitars might be caused by 
  the thing you intended to be different (the bracing pattern), but they might 
  not: you may be hearing the accumulated effect of all those other small 
  variations. 

  The solution to this problem is to realise that your experiment, like every 
  other experiment, needs a control. At a minimum, you should make four 
  guitars: the three you already planned, plus a fourth one that has the same 
  bracing pattern as one of the others. In other words, you try to build two 
  identical guitars among your set. Now you listen to them all, and the 
  question to decide is not “Do the ones with different bracing sound 
  different?”, but the more subtle question “Do the ones with different bracing 
  sound more different than the two supposed identical ones?” 

