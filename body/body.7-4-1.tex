  Transverse vibration of a string inevitably causes an increase in length of 
  any small element of the string, leading to a change in local tension and to 
  excitation of longitudinal vibration. In this section we will derive a simple 
  approximate equation governing the process, then use it to explore the 
  consequences for the frequency content of a plucked note. As in previous 
  derivations of governing equations, we will look at the balance of forces on 
  a small element of the string, sketched in Fig.\ 1. The transverse 
  displacement is $w(x,t)$, and we now also allow longitudinal displacement 
  $\xi(x,t)$. Because of the longitudinal stretch, the tension may also vary: 

  $$T(x,t) = T_0 +EA\epsilon(x,t) \tag{1}$$ 

  where $E$ is the Young's modulus of the string, $A$ is the cross-sectional 
  area, $\epsilon$ is the longitudinal strain and $T_0$ is the original string 
  tension. 

  \fig{figs/fig-ceaadbaa.png}{\caption{Figure 1. Sketch of an element of the 
  string. Longitudinal displacement is indicated in blue, tension is indicated 
  in red.}} 

  This element of string originally had length $\delta x$. Denoting the 
  stretched length by $\delta s$, Pythagoras' theorem gives 

  $$\delta s^2 \approx (\delta x + \delta \xi)^2 + \delta w^2 \tag{2} $$ 

  where $\delta \xi=\xi(x + \delta x) -- \xi(x)$ and $\delta w=w(x + \delta x) 
  -- w(x)$, so that 

  $$\delta s \approx \delta x \left[ \left(1 + \dfrac{\partial \xi}{\partial 
  x}\right)^2 + \left(\dfrac{\partial w}{\partial x}\right)^2 \right]^{1/2} . 
  \tag{3}$$ 

  We are most interested in the case of very small longitudinal motion $\xi$ 
  driven by fairly small transverse displacement $w$. With those 
  approximations, we can expand using the binomial theorem to see that the 
  leading-order contribution to string stretching is given by 

  $$\delta s \approx \delta x \left[ 1 + \dfrac{\partial \xi}{\partial x} + 
  \dfrac{1}{2}\left(\dfrac{\partial w}{\partial x}\right)^2 \right] . \tag{4}$$ 

  It follows that the strain is 

  $$\epsilon \approx \dfrac{\delta s -- \delta x}{\delta x} \approx 
  \dfrac{\partial \xi}{\partial x} + \dfrac{1}{2}\left(\dfrac{\partial 
  w}{\partial x}\right)^2 \tag{5}$$ 

  and the tension at position $x$ is 

  $$T \approx T_0 + EA\left[\dfrac{\partial \xi}{\partial x} + 
  \dfrac{1}{2}\left(\dfrac{\partial w}{\partial x}\right)^2 \right] . \tag{6}$$ 

  Now we can apply Newton's law to the net horizontal motion of the small 
  element of string: 

  $$m \delta x \dfrac{\partial^2 \xi}{\partial t^2} \approx T(x+\delta x) -- 
  T(x) \tag{7}$$ 

  where $m$ is the mass per unit length of the string, and the angles 
  $\theta_1$ and $\theta_2$ are assumed to be small. It follows that 

  $$m \dfrac{\partial^2 \xi}{\partial t^2} \approx \dfrac{\partial T}{\partial 
  x} \approx EA\left[ \dfrac{\partial^2 \xi}{\partial x^2} + \dfrac{1}{2} 
  \dfrac{\partial}{\partial x} \left(\dfrac{\partial w}{\partial x}\right)^2 
  \right] \tag{8}$$ 

  so that finally the governing equation we want is 

  $$m \dfrac{\partial^2 \xi}{\partial t^2} -- EA \dfrac{\partial^2 
  \xi}{\partial x^2} \approx \dfrac{EA}{2} \dfrac{\partial}{\partial x} 
  \left(\dfrac{\partial w}{\partial x}\right)^2 . \tag{9}$$ 

  This is the usual linear wave equation for longitudinal motion, with a source 
  term on the right-hand side determined by a quadratic function of the 
  transverse motion of the string. We can immediately see what the frequency 
  content of longitudinal motion must consist of. In terms of the usual jargon 
  for solving differential equations, it could involve a mixture of a 
  ``complementary function'' and a ``particular integral''. The complementary 
  function consists of solutions of the wave equation without the forcing term: 
  in other words, a linear combination of the longitudinal resonances of the 
  string. 

  The particular integral is the forced part of the motion, arising directly 
  from the right-hand side forcing term. The quadratic nature of that term 
  tells us what possible frequencies will be present. The slope $\partial 
  w/\partial x$ is a linear function of the transverse motion, so it must 
  consist of a linear combination of the transverse resonance frequencies of 
  the string. This linear combination is then squared, but other than that only 
  linear operations are involved. So to find the possible frequencies, it is 
  sufficient to look at a squared combination of the form 

  $$\left[\sum_j{a_j \sin \omega_j t} \right]^2 \tag{10}$$ 

  where $\omega_j$ is the $j$th transverse resonance frequency. We can ignore 
  the effects of damping for this qualitative discussion. This squared sum will 
  involve terms like 

  $$\sin^2 \omega_j t = (1 -- \cos 2 \omega_j t)/2 \tag{11}$$ 

  and 

  $$\sin \omega_j t \sin \omega_k t = [\cos (\omega_j -\omega_k)t -- \cos 
  (\omega_j +\omega_k)t]/2 . \tag{12}$$ 

  So the particular integral must consist of a linear combination of sinusoidal 
  terms with frequencies which are either frequency-doubled transverse 
  frequencies ($2 \omega_j$) or else sums and differences of pairs of 
  transverse frequencies ($\omega_j \pm \omega_k$). 

  For an ideal flexible string the sum in Eq. (10) would involve 
  harmonically-related frequencies, so it would look like a Fourier series. All 
  possible frequency doublings or sum and difference frequencies would also be 
  harmonics of the fundamental frequency: they would automatically give results 
  that exactly matched other frequencies in the series. However, for the real 
  string the effect of bending stiffness produces non-harmonic frequencies. It 
  is useful to look at explicit results for the resulting combination 
  frequencies. We showed in section 5.4.3 that 

  $$\omega_j \approx j \omega_1 (1+\alpha j^2) \tag{13}$$ 

  where $\alpha=\frac{EI\pi^2}{2T_0 L^2}$ in terms of the string length $L$ and 
  the bending stiffness $EI$. So, for example, the frequency-doubled version of 
  $\omega_3$ will be $6 \omega_1 (1+9 \alpha)$. Unless $\alpha = 0$, this is a 
  different frequency from $\omega_6 = 6 \omega_1 (1+36 \alpha)$. Similarly, 
  the sum $\omega_3 + \omega_4$ will be $3 \omega_1 (1+9 \alpha) + 4 
  \omega_1(1+16 \alpha) = \omega_1 (7 + 91 \alpha)$, which is a different 
  frequency from $\omega_7 = 7 \omega_1 (1+49 \alpha)$. The ``phantom 
  partials'' generated by the square law inside Eq. (9) will not match any of 
  the original frequencies $\omega_j$. This accounts for the complicated 
  patterns of peaks revealed in Figs.\ 5 and 6 of section 7.4. 

  Finally, our interest is not directly in the longitudinal string motion as 
  such, but in the force exerted on the soundboard, which then produces sound 
  radiation and contributes to the sound of the plucked note. For that, we need 
  to look back at Eq. (6). Apart from the steady tension $T_0$, there are two 
  terms. One involves $\partial \xi/\partial x$, and will consist of the 
  frequency components we have just described. The final term involves the 
  squared slope of the transverse string shape, at the termination point on the 
  soundboard. This is another quadratic combination of the transverse motion, 
  so it will involve the same set of possible frequency components, but with a 
  different pattern of amplitudes. 

  This distinction turns out to be important for the harp, and also for the 
  piano. The term involving $\partial \xi/\partial x$ will always be present, 
  but the final term involving the transverse motion will depend on a detail we 
  have ignored up to now when discussing stretched strings. The stiff-string 
  equation involves a fourth derivative in $x$, so that it requires two 
  boundary conditions at each end of the string, rather than just one as was 
  the case for the ideal string. 

  The extra condition is familiar from thinking about bending beams (see 
  section 3.2.1). For a string, anchored at the ends, there are two standard 
  choices: a pinned boundary has no bending moment at the end, so that it 
  allows a non-zero slope $\partial w/\partial x$ at the end, and so the final 
  term in Eq. (6) can contribute. On the other hand, a clamped boundary imposes 
  a condition of zero rotation, so that this slope is zero and the final term 
  in Eq. (6) is also zero. The termination of a real string will probably be 
  somewhere in between these two textbook cases, depending on small details of 
  how the instrument maker has adjusted the bridge/string interaction at the 
  termination. 

  The two cases produce different patterns of phantom partials. Comparison with 
  measurements on both the piano and the harp gives a strong suggestion that 
  the real boundary condition is closer to clamped than pinned. This has an 
  audible consequence for the sound of both instruments. Some sound examples 
  for the harp are included in Section 7.4. 