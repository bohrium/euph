  The Lorenz system of equations, first proposed by Edward Lorenz in 1963 [1], 
  are a set of coupled first-order differential equations relating three 
  variables $x(t)$, $y(t)$ and $z(t)$: 

  \begin{equation*}\dot{x}=\sigma (y-x) \tag{1}\end{equation*} 

  \begin{equation*}\dot{y}=x(\rho-z)-y \tag{2}\end{equation*} 

  \begin{equation*}\dot{z}=xy-\beta z \tag{3}\end{equation*} 

  \noindent{}where $\sigma$, $\rho$ and $\beta$ are constants. 

  In Lorenz's original model, $x$ is proportional to the rate of convection in 
  the atmosphere when heated from below by a vertical temperature gradient 
  proportional to $z$, with horizontal temperature variation characterised by 
  $y$. This type of convection is known as ``Rayleigh-Bénard convection'': the 
  same Rayleigh we met earlier in ``Rayleigh's principle''. 

  The equations appear simple: linear in each separate variable, and including 
  nothing more extreme than the products $xy$ and $xz$. For some ranges of 
  values of the constants $\sigma$, $\rho$ and $\beta$, the predicted behaviour 
  is indeed quite simple. However, for certain ranges of values complicated 
  chaotic behaviour is found. For all the cases plotted in section 8.4, the 
  values $\sigma =10$, $\beta=8/3$ and $\rho=28$ are used: these are the 
  original set of values used by Lorenz. 

  \sectionreferences{}[1] \tt{}Lorenz, Edward Norton\rm{}: \tt{}``Deterministic 
  nonperiodic flow''\rm{}, Journal of the Atmospheric Sciences, \textbf{20}, 
  130–141 (1963). 