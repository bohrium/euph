  When we were looking at plucked strings, we highlighted several aspects of 
  their behaviour that proved to be important for the sound: the string’s 
  damping behaviour and its bending stiffness, and the fact that each string 
  mode can exist in two different polarisations which may have slightly 
  different frequencies and damping because of interactions with the instrument 
  body. These factors all seem likely to have some significance for bowed 
  strings, too, so we would like to be able to incorporate them into a 
  simulation model. 

  When we discussed the simulation of plucked string sounds, we used a method 
  based in the frequency domain. In consequence, damping and bending stiffness 
  were described in the frequency domain (see section 5.4 and its various side 
  links). However, for our bowed string synthesis approach we need to express 
  these effects in the time domain via the reflection functions $r_1(t)$ and 
  $r_2(t)$, which operate by convolution. Another way of saying this is that we 
  want to express the stiffness and damping effects in the two sections of the 
  string as digital filters. 

  Digital filters fall into two main classes. The ones which correspond 
  directly to the convolution integral with a reflection function are called 
  ``Finite impulse response'' or FIR filters. A discretised version of a 
  function like $r_1(t)$ is stored, and at each time step these values are 
  multiplied by the appropriate values of the stored outgoing wave, and the 
  results added up. Filters in the second class are called ``Infinite impulse 
  response'' or IIR filters. They use a recursive algorithm of some kind, so 
  that although a finite number of arithmetic operations are performed at each 
  time step, the result behaves as if we had performed a convolution with an 
  infinitely long vector. IIR filters can be very efficient, because they 
  involve fewer arithmetic operations at each time step than corresponding FIR 
  filters, as will become clear shortly. 

  We can show an example of an IIR filter to represent the effect of a body 
  resonance. As we saw in section 9.4.1, the simplest model of a wolf note 
  required a single body mode. We represented this mode by a 
  mass-spring-dashpot oscillator with mass $m$, stiffness $k$ and dashpot 
  strength $c$, and we deduced the required the impulse response to be 

  $$g(t) \approx \dfrac{1}{m} \cos (\omega_b t) e^{-\omega_b \eta_b t/2} 
  \tag{1}$$ 

  where the resonance frequency is $\omega_b = \sqrt{k/m}$ and the modal loss 
  factor is defined by $\omega_b \eta_b \approx c/m$. 

  The computation will be carried out in discrete time, so the convolution 
  integral with a function $f(t)$ will be approximated by a sum of the form 

  $$c_n=h\sum_{j=0}^{\infty}{g(jh) f_{n-j}} \tag{2}$$ 

  where $h$ is the chosen time step and $f_j=f(jh)$ are the discrete values of 
  the stored outgoing wave. The upper limit of this sum is indefinite, 
  depending on the length of past history of the function $f$ which is 
  available at this point in the simulation. 

  Now note that we can write $g(jh)$ as the real part of a complex number, in 
  the form 

  $$g(jh) = \Re{\left[ \dfrac{1}{m} e^{(i-2\eta_b)\omega_b jh/2} \right]} = 
  \Re{\dfrac{1}{m} \alpha^j} \tag{3}$$ 

  where the complex constant $\alpha$ is given by 

  $$\alpha=e^{(i- \eta_b)\omega_b h/2} . \tag{4}$$ 

  It follows that our convolution integral is approximated by 

  $$c_n=\dfrac{h}{m}\Re{\left[\sum_{j=0}^{\infty}{\alpha^j f_{n-j}} 
  \right]}=\dfrac{h}{m}\Re{C_n} \tag{5}$$ 

  where 

  $$C_n=\sum_{j=0}^{\infty}{\alpha^j f_{n-j}}. \tag{6}$$ 

  Now we see that 

  $$c_{n+1}=\dfrac{h}{m}\Re{C_{n+1}}=\dfrac{h}{m} \Re{\left[f_{n+1}+\alpha 
  C_{n} \right]} . \tag{7}$$ 

  The sum of indefinite length has disappeared. All we need to do is store the 
  current value of $C_n$, then at each time step we update it by multiplying by 
  $\alpha$ and adding the new value of $f$. Finally, we take the real part and 
  apply the scale factor $h/m$ to obtain the result we want. In practice, for 
  maximum speed this IIR filter would be coded in a slightly different way to 
  avoid going through the step of using complex arithmetic, but the form shown 
  here illustrates particularly clearly how a simple recursive algorithm can be 
  used to represent an impulse response with an infinitely long tail. 

  There are two possible approaches to designing digital filters to represent 
  particular physical effects in our string. Sometimes, we can use a 
  physics-based argument to give a time-domain representation of an effect, 
  which can be used directly as an FIR filter in a reflection function. We will 
  see an example shortly, in relation to the effect of bending stiffness. 

  But there is another approach. Digital filters are used very widely, and 
  there are well-established techniques for designing them. We won't go into 
  the details of digital filter design here: for a full account written from 
  the perspective of musical applications see \tt{}this web site\rm{} by Julius 
  Smith. There are some standard forms of filter, and the usual procedure is to 
  choose one of these forms, choose a value of the order (the number of 
  parameter values used to define the filter), and then run a design code to 
  determine the corresponding digital filter which gives the best least-squares 
  match to the desired frequency domain behaviour. 

  The simulation results shown in section 9.5 were obtained using filters to 
  represent the effects of damping and bending stiffness, designed by this kind 
  of least-squares fitting methodology: the details are described in section 
  2.1 of reference [1]. But it is of some physical interest to show a possible 
  alternative approach to bending stiffness, based directly on a physical 
  argument. The governing equation for a stiff string, without damping, is 
  simply a combination of the equations derived earlier for a flexible string 
  (section 3.1.1) and a bending beam (section 3.2.1): 

  $$m \dfrac{\partial^2 w}{\partial t^2} =T \dfrac{\partial^2 w}{\partial 
  x^2}-EI \dfrac{\partial^4 w}{\partial x^4} \tag{8}$$ 

  where $w(x,t)$ is the displacement of the string, $m$ is its mass per unit 
  length, $T$ is its tension and $EI$ is its bending rigidity. 

  It is possible to derive a closed-form expression for the response of such a 
  stiff string to an initial impulse. The expression is approximate, based on a 
  mathematical technique known as matched asymptotic expansion, but it has been 
  shown to agree quite well with measurements. The gory details can be seen in 
  the Appendix of reference [2]. Figure 1 shows an example of the result. It is 
  a rather extreme case based on a very long string, to show the key features 
  clearly. The nominal arrival time for the pulse in the absence of bending 
  stiffness would be at $t=10$~ms. The result of the bending stiffness term in 
  equation (8) is to make waves of higher frequency travel progressively 
  faster. This results in a ``precursor'' of the main pulse, and if you look 
  carefully at the plot you can see the frequency gradually changing. Equation 
  (8) becomes physically unrealistic at very high frequency, predicting a wave 
  speed that increases no limit. In reality, there is a maximum possible speed, 
  so the function has been filtered with a cutoff frequency to remove the very 
  high frequencies. 

  \fig{figs/fig-cfbdb37a.png}{\caption{Figure 1. Example of the approximate 
  time response of a stiff string, observed at a distant point on the string 
  following an initial impulse at time $t=0$.}} 

  The final topic to mention here concerns the second polarisation of string 
  vibration. Only one polarisation is directly excited by the friction force 
  from bowing, but the other can be excited by variations of normal bow force. 
  In any case the two polarisations are coupled together at the terminations of 
  the string, especially at the bridge, by the three-dimensional nature of the 
  body’s vibration response. With a little effort these effects can all be 
  incorporated in the simulation. A second set of incoming and outgoing waves 
  need to be introduced. They will have their own reflection functions: the 
  damping and stiffness effects will probably be the same for both 
  polarisations, but a more complicated treatment is needed for the reflection 
  at the body (and perhaps at the player's finger). For full realism, this 
  needs to be based on a measurement of the $2 \times 2$ matrix of admittances 
  at the bridge. 

  A similar approach can be used, if desired, to represent the effect of 
  sympathetic vibration of the other strings of the instrument. Any other 
  string that is not damped by the player's left hand will be driven to some 
  extent via vibration of the bridge, which is shared with the string being 
  bowed. We simply need to add more sets of incoming and outgoing waves and 
  more reflection functions, based measuring more admittances. 

  \sectionreferences{}[1] Hossein Mansour, Jim Woodhouse and Gary P. Scavone, 
  “Enhanced wave-based modelling of musical strings, Part 1 Plucked strings”; 
  Acta Acustica united with Acustica, \textbf{102}, 1082–1093 (2016). 

  [2] J. Woodhouse; “On the playability of violins, Part I reflection 
  functions”; Acustica \textbf{78}, 125–136 (1993). 