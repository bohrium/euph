  In this section I will summarise the simplest model for the behaviour of an 
  air-jet instrument like the recorder or flute. It is known as the ``jet-drive 
  model'', and my account will closely follow the descriptions by Auvray et al. 
  [1] and Fabre [2]. We can define the geometry of the problem via an idealised 
  recorder mouthpiece, sketched in Fig.\ 1. Air is blown from a slot of height 
  $h$. The open ``mouth'' of the recorder has length $w$, after which there is 
  a sharp edge, the ``labium''. The edge is a distance $y_0$ below the 
  centre-line of the slot. 

  \fig{figs/fig-7ad761c3.png}{\caption{Figure 1. Schematic sketch of a recorder 
  mouthpiece}} 

  The mouth is open to the atmosphere on its upper surface, while its lower 
  surface opens into the tube of the instrument. The pressure just inside the 
  mouth is $p(t)$, and there will be a net volume flow rate into the tube that 
  we will call $v(t)$ as we did in the case of the clarinet mouthpiece back in 
  section 11.3.1. This pressure and volume flow rate are linked via the linear 
  acoustics of the tube, characterised by its input admittance 
  $Y(\omega)=V(\omega)/P(\omega)$ where $P$ and $V$ are the Fourier transforms 
  of $p$ and $v$. (Recall that for an open-ended tube like a recorder, the 
  resonances are marked by peaks of admittance. This contrasts with a 
  closed-ended tube like the mouthpiece end of a clarinet, where resonances 
  appeared as peaks of impedance as we saw in earlier sections.) 

  Key ingredients of the model are illustrated in cartoon form in Fig.\ 2. The 
  air emerging from the slot forms a jet, outlined in orange. This jet is 
  assumed to remain laminar over the distance to the labium. The jet is 
  perturbed by the air flow associated with the acoustic response of the pipe, 
  creating a sinuous disturbance that is convected along the jet, growing as it 
  travels. We will denote the downward displacement of the centre-line of the 
  jet (shown as a dotted line) by $\eta(x,t)$ where $x$ is distance measured 
  horizontally from the end of the slot. 

  \fig{figs/fig-da9ae12c.png}{\caption{Figure 2. Schematic sketch of the 
  unstable air jet issuing from the slot in a recorder mouthpiece and moving 
  towards the labium. The circles indicate where boundary layer separation 
  occurs, at the sharp edges of the slot.}} 

  When the jet reaches the labium, the sharp edge acts as a ``splitter'' so 
  that part of the jet flow is diverted into the tube while the rest is 
  deflected outside. This splitting action exerts a force on the fluid, and 
  that force acts as a dipole sound source (as explained in section 11.7.1), 
  exciting the acoustic response of the tube. So the feedback loop is closed: 
  sound pressure creates volume flow via the tube admittance, the flow triggers 
  perturbations on the jet, and those perturbations interact with the labium to 
  create the sound pressure. We need to look carefully at each of these stages, 
  to approximate the physics by a set of equations that can be combined to give 
  a complete simulation model. This is not a simple exercise: each stage 
  involves teasing through tricky physics to find simple approximations that 
  capture enough of the essence to give simulations that are at least 
  qualitatively plausible. 

  The first step is to understand that if the air had no viscosity, the flow 
  pattern would be completely different. As was explained back in section 
  11.2.1, any flow without viscosity will be an example of ``potential flow'', 
  which among other things means that the streamline pattern is reversible: if 
  the flow went the other way, the streamlines would be exactly the same (with 
  the arrows reversed). Figure 3 gives an indication of how the streamlines 
  would behave as the air comes out of the slot. They fan out in all directions 
  at the end of the slot, a pattern that is easier to imagine if you think of 
  reversing it. If instead of blowing, you sucked air from the mouthpiece, flow 
  would be gathered from all directions into the slot (as happens at the nozzle 
  of your vacuum cleaner). 

  \fig{figs/fig-21e27a1f.png}{\caption{Figure 3. Sketch of the streamlines of 
  an inviscid potential flow resulting from blowing air through a slot.}} 

  The biggest difference between Figs.\ 2 and 3 occurs at the sharp corners at 
  the end of the slot. In the potential flow, there is a singularity at any 
  sharp edge like this: the fluid acceleration and velocity would tend to 
  infinity. Near such a point, even very low viscosity will have a strong 
  influence on flow. The result is boundary layer separation, indicated by the 
  two circles in Fig.\ 2. This is how the air jet forms. The resulting flow is 
  definitely not reversible: you cannot make a jet by sucking, only by blowing. 
  (The reversed ``vacuum cleaner'' flow would still involve boundary layer 
  separation at the sharp corners, but this would affect the flow inside the 
  slot rather than outside it.) 

  There is another effect of viscosity on the flow, illustrated in Fig.\ 4. 
  This shows how the profile of flow speed across the width of the jet evolves 
  as the jet travels. At the entry to the slot, the flow is driven by uniform 
  pressure in the player's mouth, so the flow speed is the same at all 
  positions in the jet. This gives a ``top hat'' profile as illustrated on the 
  left-hand side of the figure. 

  \fig{figs/fig-44a16335.png}{\caption{Figure 4. Sketch of how the flow profile 
  evolves in the air jet, from a ``top hat'' profile at the channel entrance, 
  through the development of viscous boundary layers at the walls to the 
  channel, to a smooth profile in the free jet. The dashed black lines indicate 
  the growth of the boundary layer thickness. If the channel had been longer or 
  the flow slower, the boundary layers from the two sides would have met and 
  merged, leading eventually to a parabolic profile known as Poiseuille flow. 
  Outside the slot, the profile spreads and entrains air from the surrounding 
  space.}} 

  As the jet travels through the slot, viscous drag on the walls generates a 
  boundary layer on each side of the jet. The boundary layer thickness 
  increases with distance through the slot, as indicated by the dashed lines. 
  By the time the jet reaches the exit of the slot, the speed profile has been 
  rounded off on both sides as shown. If the slot had been longer, or the flow 
  had been slower, the boundary layers would have increased further in 
  thickness, to the extent that they might meet in the middle. The flow profile 
  would then tend towards a parabolic shape known as ``Poiseuille flow''. 

  Once the jet passes the sharp corners so that boundary layer separation 
  occurs, the jet profile continues to evolve in a slightly different way 
  because it is no longer in contact with the hard walls. The influence of 
  viscosity means that it ``entrains'' some air flow from the surrounding space 
  as indicated in the sketch. By the time the jet reaches the labium (provided 
  it remains laminar) the profile might have turned into a smooth humped shape 
  like the sketch on the right-hand side of the figure. In order to obtain 
  simple model equations, we will shortly use a convenient mathematical 
  function to describe a humped profile of this general form. 

  One way to describe the profile evolution shown in Fig.\ 4 is in terms of 
  vorticity, making use of results from section 11.2.1. As soon as the flow 
  enters the slot, it has regions that exhibit shear: differential velocity at 
  different vertical heights in the jet. Shear motion automatically implies 
  local rotation of the fluid, which is described by the vorticity. In the 
  absence of viscosity, we learned in section 11.2.1 that vorticity is 
  conserved along streamlines. But when viscosity is present, the vorticity 
  diffuses, rather like heat. It is this diffusion of vorticity away from the 
  walls of the slot that leads to progressive thickening of the boundary 
  layers. Further diffusion in the free jet after separation leads to a 
  progressively broader and smoother humped profile. 

  Vorticity is important for two ingredients of the model we are trying to 
  build. The classical analysis of the instability of a single shear layer, or 
  of a pair of mirror-image shear layers defining the sides of a jet, is 
  phrased in terms of vorticity. (You can find an account of this analysis in 
  section 10.3.2 of reference [2].) A small vorticity perturbation tends to 
  travel along the shear layer or the jet, growing as it goes. We will need a 
  description of this travel speed and growth rate. 

  But we also need to consider how the vorticity perturbation is applied to the 
  jet in the first place. Remember that without viscosity, vorticity is 
  conserved. So if we want to inject a vorticity perturbation associated with 
  the acoustic flow (perpendicular to the jet), we are most likely to be able 
  to achieve this at the point where viscosity has its most significant 
  influence. We already know where this happens: at the sharp corners of the 
  slot exit, where the boundary layer separation occurs. 

  Putting these two things together, and adding some semi-empirical estimates 
  of functional forms and the values of certain constants, the first governing 
  equations for our model describe the motion of the jet centre-line: 

  \begin{equation*}\eta(x,t)=e^{\alpha x} \eta_0(t-x/c_p) 
  \tag{1}\end{equation*} 

  \noindent{}where the growth rate 

  \begin{equation*}\alpha \approx \dfrac{0.3}{h}, \tag{2}\end{equation*} 

  \noindent{}the propagation speed of disturbances on the jet is 

  \begin{equation*}c_p \approx 0.4 u_0 , \tag{3}\end{equation*} 

  \noindent{}and 

  \begin{equation*}\eta_0(t)=\dfrac{h}{u_0} u(t) \tag{4}\end{equation*} 

  \noindent{}where $u_0$ is the nominal speed of the injected air and $u(t)$ is 
  the acoustic velocity, related to the volume flow rate by 

  \begin{equation*}v=u w d \tag{5}\end{equation*} 

  \noindent{}where $d$ is the width of the mouth in the direction perpendicular 
  to the diagrams in Figs.\ 1--4 so that $wd$ is the mouth area. 

  To complete the model, we need a relation between the jet motion, the 
  acoustic velocity, and the pressure inside the mouthpiece. According to 
  Auvray et al. [1] we can approximate this relation in terms of two 
  components. First, there is the component arising from the acoustic source 
  created by the jet interacting with the labium. The suggested relation 
  involves the rate of change of the fraction of the jet volume flow that 
  passes underneath the labium, into the tube. Assuming an idealised expression 
  for the jet flow profile (known as a ``Bickley profile''), they give the 
  resulting contribution to the pressure jump across the mouth in the form 

  \begin{equation*}\Delta p_{source}=\dfrac{\rho_0 \delta_d u_0 b}{w} 
  \dfrac{d}{dt} \left[ \tanh \left( \dfrac{\eta(w,t)-y_0}{b} \right) \right] 
  \tag{6}\end{equation*} 

  \noindent{}where 

  \begin{equation*}b \approx \dfrac{2h}{5} \tag{7}\end{equation*} 

  \noindent{}describes the effective half-width of the jet profile, and 

  \begin{equation*}\delta_d \approx \dfrac{4}{\pi}\sqrt{2 h w} 
  \tag{8}\end{equation*} 

  \noindent{}describes the effective separation of a pair of acoustic sources 
  above and below the labium: they combine to constitute the dipole source. 

  The second contribution to the internal pressure comes from interaction of 
  the acoustic flow with the labium. Figure 5 shows in sketch form how the flow 
  through the mouth would behave in the absence of viscosity. This is a 
  potential flow pattern. The flow accelerates as it approaches the sharp edge, 
  then decelerates symmetrically on the other side. As a result, Bernoulli 
  tells us that the pressure decreases, then increases back to the original 
  level. 

  \fig{figs/fig-9bac9fc2.png}{\caption{Figure 5. Sketch of streamlines of the 
  acoustic flow, without viscosity so that it is a potential flow pattern.}} 

  Figure 6 shows, again in sketch form, what happens with non-zero viscosity. 
  Just as we saw with the flow from the slot, the sharp edge creates a 
  singularity in the potential flow, so that in reality boundary layer 
  separation occurs at this point. A jet is formed on the downstream side, 
  which then shrinks to a slightly narrower configuration due to the vena 
  contracta effect. Now, the pressure falls as the flow approaches the edge, 
  but it does not recover on the other side. The result is a pressure 
  difference across the mouth, associated with a loss of energy further 
  downstream as the jet degenerates into turbulence. We can estimate the 
  pressure drop using Bernoulli's principle, just as we did for the clarinet 
  mouthpiece. The result is 

  \begin{equation*}\Delta p_{loss} = -\dfrac{1}{2} \rho_0 \dfrac{u^2 
  \mathrm{sign}(u)}{C^2} \tag{9}\end{equation*} 

  \noindent{}where $C$ is a vena contracta coefficient with a value of the 
  order of 0.6 as before. 

  \fig{figs/fig-519ccf34.png}{\caption{Figure 6. Sketch of streamlines for the 
  same situation as in Fig. 5, but with viscosity so that boundary layer 
  separation occurs at the sharp edge, and a downstream jet is formed.}} 

  The outside of the mouth is open to the atmosphere at ambient pressure, and 
  as usual we neglect the small contribution from radiated sound and assume 
  that the fluctuating component of pressure is zero there. So the pressure 
  inside the mouthpiece is given by the sum of the two contributions: 

  \begin{equation*}p(t)=\Delta p_{source} + \Delta p_{loss} . 
  \tag{10}\end{equation*} 

  We can learn something interesting from these equations if we look at the 
  threshold case when the pressure and volume flow signals are small and 
  quasi-sinusoidal: 

  \begin{equation*}p \approx \bar{p}e^{i \omega t}, v \approx \bar{v}e^{i 
  \omega t} . \tag{11}\end{equation*} 

  The two complex amplitudes $\bar{p}$ and $\bar{v}$ are linked by 

  \begin{equation*}\dfrac{\bar{v}}{\bar{p}} = Y(\omega) . 
  \tag{12}\end{equation*} 

  We are only going to be interested in the phases of the various complex 
  quantities in this calculation, so we will ignore the effect of the growth 
  factor $\alpha$ and the labium offset $y_0$. The approximate version of 
  equation (1) then says 

  \begin{equation*}\eta \approx \eta_0(t-\tau) \tag{13}\end{equation*} 

  \noindent{}where 

  \begin{equation*}\eta_0 \approx \dfrac{h}{u_0 w d} \bar{v} e^{i \omega t} 
  \tag{14}\end{equation*} 

  \noindent{}and the delay 

  \begin{equation*}\tau=w/c_p, \tag{15}\end{equation*} 

  \noindent{}so that 

  \begin{equation*}\eta \approx \dfrac{h}{u_0 w d} \bar{v} e^{i \omega 
  (t-\tau)} . \tag{16}\end{equation*} 

  Now we turn to equation (6). We are assuming that the amplitude of $\eta$ is 
  small, so we can use the familiar approximation $\tanh \theta \approx \theta$ 
  to simplify the equation to the form 

  \begin{equation*}\Delta p_{source} \approx \dfrac{\rho_0 \delta_d u_0 b}{w} i 
  \omega \dfrac{\eta}{b} \approx \dfrac{\rho_0 \delta_d u_0 h}{w^2 u_0 d} i 
  \omega \bar{v} e^{i \omega (t-\tau)} . \tag{17}\end{equation*} 

  The term $\Delta p_{loss}$ is second order in $\bar{v}$ so we can neglect it 
  and deduce that $p \approx \Delta p_{source}$. Substituting into equation 
  (12) and cancelling the amplitude $\bar{v}$, we obtain something of the form 

  \begin{equation*}(\mathrm{constants})i Y(\omega) e^{i \omega (t-\tau)} 
  \approx e^{i \omega t} \tag{18}\end{equation*} 

  \noindent{}where $(\mathrm{constants})$ denotes a collection of real, 
  positive quantities. In order for the phases of these complex quantities to 
  balance, we must have 

  \begin{equation*}e^{[i \pi/2 +i \arg(Y) -i \omega \tau]} = 1 
  \tag{19}\end{equation*} 

  \noindent{}where we have used the identity $i = e^{i \pi/2}$, so finally we 
  require 

  \begin{equation*}\arg(Y) + \pi/2 -- \omega \tau = 2m \pi 
  \tag{20}\end{equation*} 

  \noindent{}where $m$ could be any positive or negative integer (including 
  zero). If we express this phase balance result in terms of the frequency 
  $f=\omega/2 \pi$, we can deduce an expression for the delay $\tau$ as a 
  fraction of the period length: 

  \begin{equation*}\tau f= \dfrac{\arg Y(\omega)}{2 \pi} + \dfrac{1}{4} -m . 
  \tag{21}\end{equation*} 

  This tells us something important. If our ``recorder'' is producing a note at 
  a frequency close to one of the peaks of $Y$, then $\arg (Y)$ will be 
  approximately zero: admittance is real and positive near each peak. So the 
  player must adjust their jet speed (and the length $w$ in the case of a 
  transverse flute) in order to achieve a delay which is approximately a 
  quarter-period for $m=0$, or approximately 5/4, 9/4, 13/4,... periods for 
  other values of $m$. 

  Section 11.8 showed some results of simulations using this model. In order to 
  check the coding, the first case studied was an attempt to duplicate the 
  results shown by Auvray et al. [1] in their Fig.\ 6(a) and (b). The parameter 
  values were: $w=4 \mathrm{~mm}$, $h=1 \mathrm{~mm}$, $y_0=0.1 \mathrm{~mm}$ 
  and $d=10 \mathrm{~mm}$. The jet speed $u_0$ was varied over the range 
  1--56~m/s. The input admittance was a three-mode approximation to 
  measurements on a particular note of a recorder: the parameter values were 
  given by Auvray et al., and a plot was shown in Fig.\ 12 of section 11.8. 

  In reference [1], the data was extracted from single long simulations in 
  which the jet speed was slowly ramped up or down. Instead, the results shown 
  here use a separate transient for each value of the jet speed, in steps of 
  0.1~m/s. Each transient was ``seeded'' with a small non-zero value of the IIR 
  filter representing the lowest mode, in a similar way to results in earlier 
  sections on reed and brass instruments. Each run gave a 1~s length of 
  simulated waveforms at a sampling rate of 100~kHz. These waveforms were 
  post-processed to extract amplitude and frequency information. 

  Figures 7 and 8 show the results, plotted in the same format as Fig.\ 6 of 
  Auvray et al. [1]. The horizontal axis is a normalised version of the jet 
  speed, the inverse of a standard dimensionless number known as the Strouhal 
  Number which is often used in flow-induced vibration studies. This inverse 
  Strouhal Number is equal to $u_0/\omega w$, where $\omega$ is the playing 
  frequency (expressed in radians/s). Both figures are gratifyingly similar to 
  the corresponding plots by Auvray et al. [1], although there are some 
  differences of detail which probably arise from the different computational 
  strategy adopted here. 

  \fig{figs/fig-afcd0e42.png}{\caption{Figure 7. Playing frequency normalised 
  by the nominal fundamental frequency (564.4~Hz) plotted, in the same format 
  as Fig. 6(a) of Auvray et al. [1], against the inverse of the Strouhal 
  Number, which is the combination $u\_0/\omega w$ where $\omega$ is the 
  playing frequency.}} 

  \fig{figs/fig-e840f7df.png}{\caption{Figure 8. Amplitude of the acoustic 
  velocity $u$ normalised by the jet speed $u\_0$, plotted against the inverse 
  of the Strouhal Number in the same format as Fig. 6(b) of Auvray et al. 
  [1].}} 

  In reference [1], the authors show a corresponding pair of plots computed 
  using the discrete-vortex method. This alternative model shows qualitatively 
  similar behaviour, but is different in a number of quantitative details. The 
  authors suggest that the jet-drive model is most suitable for values of the 
  inverse Strouhal Number above 5, whereas below that value the discrete-vortex 
  model is better. 

  \sectionreferences{}[1] Roman Auvray, Augustin Ernoult, Benoît Fabre and 
  Pierre-Yves Lagrée, “Time-domain simulation of flute-like instruments: 
  Comparison of jet-drive and discrete-vortex models”, Journal of the 
  Acoustical Society of America \textbf{136}, 389—400 (2014) 

  [2] Antoine Chaigne and Jean Kergomard; “Acoustics of musical instruments”, 
  Springer/ASA press (2013): see Chapter 10, ``Flute-like instruments'', by 
  Benoît Fabre. 