  The mathematics behind experimental modal analysis starts from an expression 
  for frequency response functions that we discussed back in Chapter 2. A 
  rather general expression for any mechanical frequency response function of a 
  vibrating structure can be written in the form 

  $$G(x,y,\omega) = (i \omega)^\alpha \sum_n \dfrac{u_n(x) 
  u_n(y)}{\omega_n^2+2i\omega \omega_n \zeta_n-\omega^2} \tag{1}$$ 

  where the sum is over modes labelled by $n$. 

  In this formula, the $n$th mode is assumed to have mode shape $u_n$, resonant 
  frequency $\omega_n$ and damping ratio $\zeta_n$. The damping ratio is 
  related to the modal loss factor by $\eta_n = 2 \zeta_n$, and to the modal 
  Q-factor by $Q_n=1/\eta_n$. The notation $u_n(x)$ is a shorthand to denote 
  ``the amplitude of the mode shape at the excitation position $x$, in the 
  relevant direction of the applied driving force''. In a similar way, $u_n(y)$ 
  denotes the corresponding modal amplitude at the position and in the 
  direction of the measuring sensor. The mode shapes must be normalised in a 
  particular way in order for this formula to work: see section 2.2.5 for the 
  details. The initial factor $(i \omega)^\alpha$ determines which particular 
  frequency response function we are dealing with: $\alpha=0$ corresponds to a 
  measurement of displacement (known as receptance), $\alpha=1$ to a 
  measurement of velocity (i.e. admittance or mobility), and $\alpha=2$ to a 
  measurement of acceleration (i.e. accelerance). 

  We will look first at the behaviour of a single term from this modal 
  summation: define 

  $$G_n = \dfrac{(i \omega)^\alpha K_n}{\omega_n^2+2i\omega \omega_n 
  \zeta_n-\omega^2} \tag{2}$$ 

  where 

  $$K_n= u_n(x) u_n(y) . \tag{3}$$ 

  We can simplify this term by factorising the quadratic in the denominator, 
  then expanding $G_n$ in partial fractions. We can write 

  $$\omega_n^2+2i\omega \omega_n \zeta_n-\omega^2 = -- (\omega \mathrm{~} -- 
  \mathrm{~} \bar{\omega}_n)(\omega + \bar{\omega}^*_n) \tag{4}$$ 

  where $.^*$ denotes the complex conjugate, and the complex root 
  $\bar{\omega}_n$ is given by 

  $$\bar{\omega}_n=\omega_n (\cos \phi_n + i \zeta_n) \approx \omega_n (1 + i 
  \zeta_n) \tag{5}$$ 

  where $\phi_n$ is defined by 

  $$\sin \phi_n = \zeta_n \tag{6}$$ 

  and the final approximate expression is for the case of small damping 
  $\zeta_n \ll 1$, so that $\phi_n \approx \zeta_n$ and $\cos \phi_n \approx 
  1$. 

  Now we look for an expansion in partial fractions. The details depend on the 
  chosen value of $\alpha$, although we will see that the most important aspect 
  of the subsequent behaviour is the same for all values of $\alpha$. For the 
  purpose of demonstrating the method, we will analyse the case $\alpha = 1$. 
  We then have 

  $$G_n = -\dfrac{i \omega K_n}{(\omega \mathrm{~} -- \mathrm{~} 
  \bar{\omega}_n)(\omega + \bar{\omega}^*_n)} \tag{7}$$ 

  and we look for an expression of the form 

  $$G_n = \dfrac{A}{\omega \mathrm{~} -- \mathrm{~} \bar{\omega}_n} + 
  \dfrac{B}{\omega + \bar{\omega}^*_n}$$ 

  $$=\dfrac{A(\omega + \bar{\omega}^*_n) + B(\omega \mathrm{~} -- \mathrm{~} 
  \bar{\omega}_n)}{(\omega \mathrm{~} -- \mathrm{~} \bar{\omega}_n)(\omega + 
  \bar{\omega}^*_n)} \tag{8}$$ 

  where $A$ and $B$ are two constants. To make this expression agree with with 
  equation (7), we need to satisfy the two conditions 

  $$-iK_n = A + B \tag{9}$$ 

  and 

  $$A \bar{\omega}^*_n -- B\bar{\omega}_n = 0 . \tag{10}$$ 

  Solving these simultaneous equations gives 

  $$A = -\dfrac{iK_n}{2}(1+ i\tan \phi_n) \tag{11}$$ 

  and 

  $$B = -\dfrac{iK_n}{2}(1- i\tan \phi_n) , \tag{12}$$ 

  so that 

  $$G_n = -\dfrac{iK_n}{2} \left[\dfrac{1+ i\tan \phi_n}{\omega \mathrm{~} -- 
  \mathrm{~} \bar{\omega}_n} + \dfrac{1- i\tan \phi_n}{\omega + 
  \bar{\omega}^*_n} \right] \tag{13}$$ 

  $$\approx -\dfrac{iK_n}{2} \left[\dfrac{1+ i\zeta_n}{\omega \mathrm{~} -- 
  \mathrm{~} \bar{\omega}_n} + \dfrac{1- i\zeta_n}{\omega + \bar{\omega}^*_n} 
  \right] \tag{14}$$ 

  for small damping. 

  Equation (13) or (14) tells us that our term $G_n(\omega)$ has two poles in 
  the complex $\omega$ plane, at 

  $$\omega = \bar{\omega}_n \approx \omega_n(1 + i \zeta_n) \tag{15}$$ 

  and 

  $$\omega = -\bar{\omega}_n^* \approx \omega_n(-1 + i \zeta_n) . \tag{16}$$ 

  A pole is a place where the complex function goes to infinity, and these are 
  both simple poles where the infinite behaviour takes the form of the inverse 
  of a linear factor like $(\omega \mathrm{~} -- \mathrm{~} z)$, for a pole at 
  $z$. 

  Exactly the same is true for the other possible values of $\alpha$. For 
  $\alpha = 0$, the result corresponding to equations (13) and (14) is 

  $$G_n = -\dfrac{K_n}{2 \omega_n \cos \phi_n} \left[\dfrac{1}{\omega -- 
  \bar{\omega}_n} -- \dfrac{1}{\omega + \bar{\omega}^*_n} \right] \tag{17}$$ 

  $$\approx -\dfrac{K_n}{2 \omega_n } \left[\dfrac{1}{\omega -- \bar{\omega}_n} 
  -- \dfrac{1}{\omega + \bar{\omega}^*_n} \right] . \tag{18}$$ 

  For $\alpha = 2$ the equivalent result is 

  $$G_n = K_n+\dfrac{K_n \omega_n}{2} \left[\dfrac{(1-2\zeta_n^2)/\cos \phi_n + 
  2 i \zeta_n}{\omega -- \bar{\omega}_n}\right.$$ 

  $$\left. -- \dfrac{(1-2\zeta_n^2)/\cos \phi_n -- 2 i \zeta_n}{\omega + 
  \bar{\omega}^*_n} \right] \tag{19}$$ 

  $$\approx K_n+\dfrac{K_n \omega_n}{2} \left[\dfrac{1 + 2 i \zeta_n}{\omega -- 
  \bar{\omega}_n} -- \dfrac{1 -- 2 i \zeta_n}{\omega + \bar{\omega}^*_n} 
  \right] . \tag{20}$$ 

  We can now deduce the pattern of poles in the complex $\omega$ plane for our 
  entire frequency response function $G(x,y,\omega)$: it must look something 
  like Fig.\ 1. For the particular value $n=3$, the pair of poles are indicated 
  by the red and blue stars, at positions that are symmetrical in the vertical 
  axis. All the other values of $n$ contribute similar pairs, shown as green 
  stars. Note that different scales are used for the horizontal and vertical 
  axes, because for small damping the poles are all very close to the real 
  (horizontal) axis. In this figure, the value $\zeta_n = 0.01$ has been used 
  for all modes. 

  \fig{figs/fig-b05b41f7.png}{Figure 1. Sketch of the positions in the complex 
  plane of poles of the frequency response function. The red and blue stars 
  show the pair of poles contributed by the term $G_3$, while the green stars 
  show the corresponding pairs from other terms $G_n$.} 

  Now suppose we are interested in the behaviour of the frequency response 
  function close to one of the resonance peaks, for example for frequencies 
  close to $\omega_3$. That means we are close to the red star in Fig.\ 1. If 
  the damping is low enough that the peaks are well separated, that suggests 
  that we might ignore the effect of the neighbouring poles marked by green 
  stars. But if we are to neglect those, we can surely also neglect the pole 
  marked by the blue star --- although it is associated with the same vibration 
  mode as the red star, it is more remote in the complex plane than other poles 
  we have already decided to neglect. 

  Based on this argument, we can hope to obtain a good first approximation to 
  the behaviour from a single term of the form 

  $$G(\omega) \approx \dfrac{A}{\omega -- \bar{\omega}_3} \tag{21}$$ 

  where $A$ is a (complex) constant. Assuming that the damping is small, we can 
  see from equations (14), (18) and (20) that $A$ is approximately a real 
  number for $\alpha = 0$ or $\alpha = 2$, whereas for $\alpha=1$ it is 
  approximately a pure imaginary number. 

  For simplicity, we look first at the case where $A$ is a positive real 
  number. We also know that $\bar{\omega}_3 \approx \omega_3(1+i \zeta_3)$ from 
  equation (5). The claim now is that if we plot the trajectory of the complex 
  value $G(\omega)$ in the complex plane, as $\omega$ varies, we will obtain a 
  circle. The simplest way to prove this is first to guess where the circle 
  would have to be, then substitute into the equation for that guessed circle 
  and verify that it is indeed satisfied. 

  We can guess where the circle must be by looking at the behaviour at some 
  particular frequencies. When $\omega=0$, $G=-A/\omega_3(1+i \zeta_3) \approx 
  -A/\omega_3$. In other words, $G$ is approximately a negative real number. At 
  the opposite extreme, as $\omega \rightarrow\infty$, $G \rightarrow 
  A/\omega$. It is a small real number, tending towards zero from the positive 
  side. Now we look at the peak value, which will occur for $\omega \approx 
  \omega_3$. The value there is $G=iA/\omega_3 \zeta_3$: this is a pure 
  imaginary value, and it is very large because $\zeta_3$ is small. 

  So when the frequency $\omega$ is well below or well above the resonance 
  frequency, $G$ is near the origin in the complex plane, near the real axis 
  and on one side or the other of the origin. But at the peak, $G$ is high up 
  along the imaginary axis. If there is a circle passing through all these 
  points, it must surely be the one which passes through the origin and the 
  peak value, with its centre on the imaginary axis to preserve symmetry on 
  either side of the resonance frequency. So we might be looking for a circle 
  of radius $R=A/2\omega_3 \zeta_3$, centred on the complex value $iR$, as 
  sketched in Fig.\ 2. The sketch highlights the various features just 
  mentioned: the peak, the trends as $\omega$ tends to zero and infinity, and 
  the circle centre and radius. 

  \fig{figs/fig-29e744eb.png}{Figure 2. A modal circle in the complex plane 
  (red), labelled with some key features} 

  If the real and imaginary parts of $G$ are $G=P+iQ$, the equation of that 
  circle would be 

  $$P^2 + (Q-R)^2 = R^2. \tag{22}$$ 

  Well, we can easily calculate $P$ and $Q$: 

  
  $$G=\dfrac{A}{\omega-\omega_3(1+i\zeta_3)}=\dfrac{A[\omega-\omega_3(1-i\zeta_3)]}{|\omega-\omega_3(1+i\zeta_3)|^2} 
  \tag{23}$$ 

  and so 

  $$P=\dfrac{A(\omega-\omega_3)}{(\omega-\omega_3)^2 + \omega_3^2 \zeta_3^2} 
  \tag{24}$$ 

  and 

  $$Q=\dfrac{A\omega_3\zeta_3}{(\omega-\omega_3)^2 + \omega_3^2 \zeta_3^2} . 
  \tag{25}$$ 

  Substituting: 

  $$P^2 + (Q-R)^2 = \dfrac{A^2(\omega-\omega_3)^2}{[(\omega-\omega_3)^2 + 
  \omega_3^2 \zeta_3^2]^2}$$ 

  $$+ \left\lbrace \dfrac{A\omega_3\zeta_3}{(\omega-\omega_3)^2 + \omega_3^2 
  \zeta_3^2} -- \dfrac{A}{2 \omega_3 \zeta_3} \right\rbrace^2 $$ 

  $$=\dfrac{A^2(\omega-\omega_3)^2 + A^2 \omega_3^2 \zeta_3^2-A^2 
  [(\omega-\omega_3)^2 + \omega_3^2 \zeta_3^2]}{[(\omega-\omega_3)^2 + 
  \omega_3^2 \zeta_3^2]^2}+R^2$$ 

  $$=R^2. \tag{26}$$ 

  So we see that the single-pole expression for $G(\omega)$ does indeed mark 
  out a circle in the complex plane, exactly as sketched in Fig.\ 2. That 
  sketch also reveals another interesting feature. We have previously said that 
  the usual way to characterise the bandwidth of a resonance peak in a 
  frequency response function is via the half-power points on either side of 
  the peak. We now see that this is a very natural choice for a characteristic 
  bandwidth: those half-power points fall at the ends of the horizontal circle 
  diameter, perpendicular to the diameter through the peak value. The two 
  dashed black lines are angled at $\pm45^\circ$ to the imaginary axis, and the 
  angle between them is a right angle. (This is a well-known theorem of 
  geometry: ``the angle in a semicircle is a right angle.'') Pythagoras' 
  theorem then tells us that the two dashed lines both have length $\sqrt{2}R$. 
  So the squared amplitude is indeed reduced from the peak value by a factor of 
  2 at these points, and also the phase (given by the angles of the dashed 
  lines) leads and lags the phase at the peak by $45^\circ$. 

  All this discussion has been based on the case where the constant $A$ was a 
  positive real number. It is easy to see what happens for other values of this 
  constant. If it is a negative real number, the circle would go downwards from 
  the origin rather than upwards. If $A$ is complex, the circle would be 
  rotated. If we write $A=|A| e^{i \theta}$, then everything is multiplied by 
  the factor $e^{i \theta}$, producing a rotation by angle $\theta$. In 
  particular, if $A$ is a pure imaginary number, as we expect to be 
  approximately for the case for any frequency response based on measuring 
  velocity ($\alpha = 1$ in terms of the earlier analysis), then $\theta 
  \approx \pm 90^\circ$ so that the circle will be aligned horizontally rather 
  than vertically in the complex plane. 

  Our approximation was based on ignoring all the other poles of the frequency 
  response. We are far away from the resonance frequencies of all the other 
  poles, so that their frequency response will have low amplitude, and also it 
  will change very little over the narrow range around the peak we are looking 
  at. The net effect, therefore, is to add a (complex) constant to the 
  approximate expression (21), and so the main effect of these neglected poles 
  is simply to give a slight shift in the position of the centre of the circle. 

  This gives the basis for a technique for extracting modal parameters from a 
  grid of measured frequency response functions, which was very important in 
  the early years of experimental modal analysis when computer processing power 
  was more limited. For any response peak that is well separated from its 
  neighbours, a suitable narrow frequency range can be chosen so that the plot 
  in the complex plane looks convincingly like part of a circle. The computer 
  can then be made to best-fit a circle, with the radius and the coordinates of 
  the centre as fitted variables. Once the circle has been found, the offset of 
  the centre can be ignored, and the single-pole theoretical expression (21) 
  can be best-fitted to the data points to determine the complex values $A$ and 
  $\bar{\omega}_n$. With a grid of measured responses, $\bar{\omega}_n$ should 
  be the same for all of them, so a single best fit using all available data 
  will give the best estimate. The constant $A$, however, will be different at 
  each grid point. That variation maps out the mode shape. 

  These days, modal analysis software normally uses a range of more 
  sophisticated signal processing techniques to allow the simultaneous fitting 
  of modal parameters to several modes within a chosen frequency band. To an 
  extent, this allows modal information to be extracted even when adjacent 
  resonance peaks start to overlap. However, it is still a useful check on data 
  quality to look for convincing circle plots near isolated response peaks. As 
  modal overlap increases, the reliability of multi-mode fitting procedures 
  decreases, and it requires experience and sophistication to judge when the 
  result become too unreliable to be trusted. We will not delve any further 
  into these techniques here: for more detail see for example the textbook by 
  Maia et al [1]. 

  \sectionreferences{}[1] Nuno M. M. Maia and Júlio M. M. Silva (editors), 
  “Theoretical and experimental modal analysis”, Research Studies Press/Wiley 
  (1997) 