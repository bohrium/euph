  Back in section 8.5 we discussed a severely simplified model for a clarinet. 
  Our model involved coupling the linear acoustics of the tube to a non-linear 
  model of the mouthpiece and reed. We will look at wind instruments properly 
  in Chapter 11, but we can anticipate the fact that almost all wind 
  instruments involve a somewhat similar division between something non-linear 
  happening in and around the mouthpiece, coupled to linear acoustics of the 
  tube of the instrument. 

  The natural frequency response function to characterise this linear acoustics 
  is the input impedance, defined as the ratio of pressure to volume flow rate 
  at a plane defining the junction of tube and mouthpiece. In terms of 
  measurement, you can supply the necessary input volume flow rate with some 
  kind of loudspeaker, and you can measure the resulting pressure with a 
  microphone. But the definition of input impedance presents a challenge: there 
  is no easy and reliable sensor that can measure volume flow rate at acoustic 
  frequencies. Over the years, considerable ingenuity has been devoted to the 
  design of measurement methods that get around this difficulty. 

  In the world of wind instruments, measurements using continuous excitation 
  are still the most usual, although we will come to an impulse-based method 
  later. The earliest approach is illustrated in schematic form in Fig.\ 1. A 
  loudspeaker is driven by a sine wave, and is connected to the instrument tube 
  via a very thin capillary tube. The loudspeaker is enclosed in a robust 
  container so that no sound escapes except into the instrument tube through 
  the capillary. Two microphones, mounted flush with the surface, measure the 
  pressure at the two ends of the capillary. 

  \fig{figs/fig-a04c2e01.png}{\caption{Figure 1. Schematic diagram of a device 
  to measure input impedance using a capillary tube in the drive system.}} 

  The significance of the capillary is that it presents a very high resistance 
  to the acoustic flow, and this has the result that the volume flow rate is 
  governed almost entirely by the pressure at the left-hand end, inside the 
  loudspeaker cavity. The pressure at the other end, in the instrument tube, 
  makes virtually no difference. This means that once the device has been 
  calibrated, the pressure measured by the internal microphone can be converted 
  directly into a volume flow rate entering the instrument, decoupled from the 
  acoustic behaviour of the particular instrument being measured. The second 
  microphone then measures the resulting pressure, and the combination of these 
  two gives the input impedance of the instrument. 

  Unfortunately, the high resistance of the capillary means that such devices 
  only inject rather small volume flow rates into the instrument. In theory 
  this doesn’t matter, since the input impedance is a linear quantity, 
  independent of the excitation amplitude. However, in practice low amplitude 
  gives a reduction in signal-to-noise ratio. So it would be better to dispense 
  with the capillary, and this leads to the next type of impedance-measuring 
  device, shown schematically in Fig.\ 2. This time, the loudspeaker cavity is 
  coupled directly to the instrument tube, with a microphone near the junction. 

  \fig{figs/fig-aeeb7bb6.png}{\caption{Figure 2. Corresponding schematic 
  diagram of a device to measure input impedance, taking advantage of a 
  measurement of pressure in the cavity behind the loudspeaker piston.}} 

  A second microphone is in the cavity behind the loudspeaker piston. Provided 
  the walls of the device are sufficiently rigid, the volume flow rate into the 
  instrument must be exactly equal and opposite to the volume flow rate into 
  the rear cavity. This rate can be inferred from the pressure measured by the 
  second microphone — provided a sufficiently elaborate and careful set of 
  calibration tests has been carried out because the rear cavity will have 
  resonances of its own which must be compensated for. It is a device of this 
  type which is seen in use in Fig.\ 7 of section 10.4. 

  Another type of measurement strategy relies on making the connecting tube 
  between the loudspeaker and the instrument very long. This long, parallel 
  tube can only support plane acoustic waves, travelling in one direction or 
  the other. By placing two (or more) microphones in the wall of this tube and 
  then doing some computer processing, it is possible to separate the 
  contributions from left-travelling and right-travelling waves. By comparing 
  their amplitudes and phases, it is then possible to deduce the reflection 
  coefficient of the instrument tube. There is a standard formula linking this 
  reflection coefficient to the input impedance, and so the impedance can be 
  determined. 

  Finally, there is a variant of this approach which we have already seen, back 
  in section 10.2. If we drive our long connecting tube with a pulse, we can 
  deduce the shape of the reflected pulse from the instrument. This pulse 
  reflectometry information can be used for bore reconstruction, as we saw 
  earlier. But it can also be processed to yield the input impedance, via a 
  Fourier transform. 

  For more detail on all these methods, see Section 4.2 of the book by 
  Campbell, Gilbert and Myers [1], and the extensive references contained 
  therein. 

  \sectionreferences{}[1] Murray Campbell, Joël Gilbert and Arnold Myers, “The 
  science of brass instruments”, ASA Press/Springer (2021) 