  Now to get down to business. To make sense of the acoustics of almost any 
  musical instrument, whether a violin, a trumpet or a Caribbean steel drum, we 
  need to understand the key concepts of resonant frequencies and vibration 
  modes. I will give an informal but approximately correct account of the 
  underlying theory in this chapter, together with a similar introduction to 
  the related concept of frequency analysis. Although some of the descriptions 
  may seem a little abstract (especially if you read the details in the 
  additional links), everything will be illustrated by reference to our first 
  actual musical instrument, a toy drum. 

  The first step in looking at how musical instruments work is to divide them 
  into two groups. The crucial division, from the perspective of a physicist, 
  is not at all obvious. It does not follow the traditional 
  strings/wind/percussion groupings: the guitar lives with the xylophone, while 
  the violin lives with the clarinet. The distinction is between instruments 
  that can produce sustained sound, and those that can't. If your bow is long 
  enough, a violin note can go on as long as you like. Given enough breath, the 
  same is true of a note on a flute. On the other hand, a guitar, a piano or a 
  drum is set into vibration by hitting or plucking, but then left alone for 
  the sound to die away in its own good time. Playing a violin pizzicato shifts 
  it across the divide, as would bowing a guitar. 

  This distinction relates to a technical concept which is unavoidable in this 
  discussion, so technophobic readers must grit their teeth for a bit. The 
  instruments which die away have (usually) a property called linearity, 
  whereas the sustained instruments must be non-linear. So the wind 
  instruments, including the organ and the human voice, are non-linear, along 
  with the bowed-string instruments like the violin and oddities like the 
  musical saw, the hurdy-gurdy and the wet finger round the rim of a wine 
  glass. Percussion instruments might be linear, along with the guitar, the 
  piano and the humble tuning fork. 

  ``Linear'' is essentially a piece of mathematical jargon implying ``easy'', 
  while ``non-linear'' suggests that analysis is going to be more difficult. We 
  can tell that something curious must be going on in sustained instruments by 
  thinking about energy. All vibration involves energy in the vibrating system, 
  and some of that energy is constantly being converted into other forms: for 
  example into heat in the wood of a violin body, and into sound waves in the 
  air which carry some of the energy away with them. If the note continues at a 
  steady level, then something is putting just the right amount of energy back 
  into the vibration. This energy is coming from the violinist's bow arm, or 
  the clarinettist's lungs. But how does something steady and non-oscillatory, 
  like the motion of a bow arm, get converted into vibrational energy, and 
  furthermore do it at just the right rate to keep the note steady? That 
  discussion must wait until later. We will look at the ``linear'' instruments 
  first: they pose no dilemma about energy, since the gradual loss of energy is 
  precisely the reason that the sound of a note on one of these instruments 
  dies away with time. 

  This chapter introduces the main acoustical concepts needed. Some technical 
  details are given in the extra links, but the key properties of a linear 
  instrument can be explained, roughly at least, in plain words. To begin at 
  the beginning, suppose you hit a drum with a soft beater. You could imagine a 
  force-measuring sensor embedded in the beater, which could show the waveform 
  of force applied to the drum-head by the beater. It will look something like 
  the upper trace in Fig.\ 1: no force until the beater makes contact, then a 
  short pulse of force rising to a peak and then reducing as the beater starts 
  to rebound. As soon as the beater bounces off the drum there is no further 
  force. 

  In response to this force pulse, the drum skin vibrates. This vibration 
  continues after the beater has bounced off. The waveform, for example 
  measured by a vibration sensor on the drum or by a microphone measuring the 
  sound produced by the vibrating drum, might look like the lower trace in 
  Fig.\ 1. The whole thing could be represented schematically by Fig.\ 2: the 
  input force waveform goes into the ``black box'' representing the mechanical 
  behaviour of the drum, and out comes an output waveform. However, this 
  schematic is by no means limited to drums: many things have this 
  input-box-output pattern. A similar input force pulse could be applied to any 
  other vibrating object, like a saucepan lid or a violin body, and some kind 
  of vibration response generically similar to the lower trace in Fig.\ 1 would 
  be produced. The input need not be force: for example the same schematic 
  could represent a loudspeaker, in which an input waveform of electrical 
  voltage is turned into an output waveform of sound. 

  If the ``box'' describes a linear system, then it has an unexpected and 
  useful property when driven with a special kind of input waveform. Enter the 
  sine wave. A sine wave is a particular shape of repeating waveform which is 
  easy to visualise (and which has important mathematical properties, as we 
  will see shortly). Take a glass disc and put a spot of paint on the rim. Spin 
  the disc at a rate which you choose. Look at it edgewise on and watch the 
  spot of paint. It goes up and down, and if you plot a graph of the height 
  against time you get a sine wave, as illustrated in Fig.\ 3 The frequency of 
  the sine wave, in cycles per second or Hertz (Hz), is the same as the rate of 
  rotation of the disc. (Heinrich Hertz was a physicist in the nineteenth 
  century who, among other things, discovered radio waves.) 

  As sketched in Fig.\ 1, you ordinarily expect the output waveform from the 
  vibrating drum, or whatever, to be different from the input waveform. But if 
  the input waveform is a sine wave, then that isn't the case. For any linear 
  system, if you put a sine wave in you get a sine wave out. It has the same 
  frequency as the input wave, and the only things that can be changed are the 
  amplitude and the phase, as illustrated in Fig.\ 3. The change in amplitude 
  and phase might be different for different frequencies of sine wave --- an 
  important idea to which we will return shortly. 

