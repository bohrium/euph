  There is a standard procedure for examining the stability of an equilibrium 
  position, and the pendulum gives a good illustration of it. We already know 
  the governing equation, from section 8.2.1: 

  $$\ddot{\theta} = -\dfrac{g}{L} \sin \theta . \tag{1}$$ 

  For equilibrium, $\ddot{\theta}=0$ and so we need $\sin \theta =0$. The two 
  relevant solutions are $\theta=0,\pi$. Now we take those in turn, and make 
  use of a Taylor series expansion to linearise the governing equation very 
  close to the equilibrium. Near $\theta =0$ we simply use the familiar result 
  $\sin \theta \approx \theta$ to obtain 

  $$\ddot{\theta} \approx -\dfrac{g}{L} \theta . \tag{2}$$ 

  This is the simple harmonic equation, and its general solution takes the form 

  $$\theta = A \cos \Omega t + B \sin \Omega t \tag{3}$$ 

  where $\Omega^2=g/L$ and $A$ and $B$ are arbitrary constants. Whatever small 
  perturbation we make around the equilibrium position, and thus whatever the 
  values of $A$ and $B$, $\theta$ remains small because sine and cosine never 
  have a magnitude greater than unity. So the equilibrium is stable: any small 
  perturbation results in the pendulum remaining close to the equilibrium for 
  all later times. 

  For the case $\theta = \pi$, we first change variable so that we have a small 
  parameter available for the purposes of expansion: so let $\theta = \pi + 
  \alpha$. Now we need to remember, from the properties of the sine function, 
  that $\sin (\pi + \alpha) = -\sin \alpha$. So the approximate governing 
  equation this time is 

  $$\ddot{\alpha} = \dfrac{g}{L} \sin \alpha \approx \dfrac{g}{L} \alpha = 
  \Omega^2 \alpha. \tag{4}$$ 

  This time the general solution takes the form 

  $$\alpha = A e^{\Omega t} + B e^{-\Omega t} \tag{5}$$ 

  where $A$ and $B$ are again arbitrary constants. This time, virtually any 
  small perturbation away from the equilibrium position will produce a non-zero 
  value of $A$. However small that value may be, eventually the growing 
  exponential will make that term large. We conclude that the equilibrium is 
  unstable: a small perturbation does not remain small. This analysis does not 
  mean that the actual motion of the pendulum grows exponentially for all 
  subsequent time: remember that we have obtained this solution by assuming 
  that $\alpha$ is small. All we can deduce is that it does not remain small, 
  so that eventually the approximation must break down. 

  For the undamped pendulum case, there is an alternative approach based on an 
  energy argument. The potential energy, relative to the position $\theta =0$, 
  is 

  $$V=mgL(1- \cos \theta) \tag{6}$$ 

  and the kinetic energy is 

  $$T=\dfrac{1}{2} m L^2 \dot{\theta}^2 \tag{7}$$. 

  A position of stable equilibrium corresponds to a minimum of potential 
  energy, because any perturbation away from that position requires energy, and 
  thus reduces the kinetic energy. But near a maximum of potential energy, any 
  perturbation releases energy and increases kinetic energy, leading to an 
  unstable runaway. It is obvious from visualising the graph of $1-\cos \theta$ 
  that $\theta = 0$ is a minimum of energy, while $\theta = \pi$ is a maximum, 
  leading to the same result obtained above. 

  As a footnote, energy also provides an easy way to graph the phase portrait 
  of an undamped system. Any possible motion must conserve total energy 
  $E=T+V$, or in other words trajectories must be contour lines of constant 
  energy. So all we need to do is compute $E$ at a grid of points in the phase 
  plane and then draw a contour map: this is how Fig.\ 3 of section 8.3 was in 
  fact computed. 