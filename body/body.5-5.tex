

  Having introduced the key ideas and heard some examples of synthesised 
  plucked instruments, we are now equipped to do a serious case study. We will 
  examine the particular case of the banjo: the target is to find out what 
  makes the sound of a banjo distinctively different from other plucked-string 
  instruments like the guitar. Even when strings, scale lengths, and pitches 
  are chosen to be virtually identical, most listeners would agree that the 
  sounds of banjo and guitar can be distinguished from just a few plucked 
  notes. 

  The challenge is first to check whether the computer synthesis model captures 
  this characteristic banjo sound, and then to use the model to do a range of 
  `virtual adjustments' to the banjo so that we can hear the effects directly. 
  The aim, ultimately, is not simply to explore the banjo as such, but to find 
  how far we can get with the use of models like this to understand what gives 
  different plucked-string instruments their characteristic ``voices''. The 
  banjo represents an extreme: unless we can make good progress with this case, 
  it would seem unlikely that the approach will be good enough to explore more 
  subtle distinctions. 

  We have already seen, in sections 5.1--5.3, that there are large differences 
  of vibrational behaviour between a banjo and an acoustical guitar: far bigger 
  than the differences between guitars of a similar type, or between different 
  banjos. We have begun to explore the physics lying behind these differences. 
  More detail (a lot more detail) is given in two scientific journal papers 
  [1,2]: one purpose of this section is to provide supporting sound 
  demonstrations and discussion for these two papers. The two papers reveal 
  that the key physical differences are directly associated with the use of a 
  stretched membrane rather than a wooden plate as the ``soundboard''. 
  Membranes differ from plates in their mass, modal density and sound radiation 
  properties. These factors, and more, play a role in understanding the 
  measured vibration response. 

  This section will be quite detailed. If the technical material is 
  off-putting, skip ahead to the sound examples! 

  \textbf{A:} \textbf{Distribution of loss factors} 

  Since the aim is to understand the characteristic sound of a played banjo, it 
  is useful to look first at some experimental comparisons between notes on a 
  banjo and a guitar, based on normal playing. The particular 5-string banjo 
  used here is a \tt{}Deering Eagle II\rm{}, and the steel-string guitar is one 
  made by \tt{}Martin Woodhouse\rm{}. This guitar, somewhat unusually for a 
  steel-strung instrument, embodies a version of Torres-like fan bracing. 
  Conveniently, the top strings of this banjo and guitar are extremely similar: 
  both are plain steel strings with the same diameter, and they are under very 
  similar tensions. This allows a rather clean comparison between plucked notes 
  on the two instruments. 

  Plectrum-played notes were recorded at every semitone from the open string up 
  to the 12th fret, on both top strings. Each of these notes was analysed using 
  a spectrogram, as was described in section 2.4. A typical example is plotted 
  in Fig.\ 1. The plot shows a set of narrow vertical bands, associated with 
  the near-harmonic overtones of the string. Within the first 0.2 s or so after 
  the pluck, the spectrogram shows a bright patch indicating a significant 
  broad-band spread of radiated sound between the string overtones, extending 
  up into the kHz region. These broad-band signals arise from transient 
  excitation of modes of the coupled string-body system which have energy 
  mainly in the body rather than in the string: we will see later in this 
  section that these contribute directly to the characteristic sound of a banjo 
  (see subsection E). 

  \fig{figs/fig-fb90a851.png}{\caption{Figure 1. Spectrogram of the note $G\_4$ 
  (392 Hz), played with a plectrum on the top string of a banjo.}} 

  From a spectrogram like Fig.\ 1, it is straightforward to detect each 
  vertical band representing a decaying mode, and to analyse the variation of 
  level and phase with time in order to determine a best estimate of the 
  frequency and decay rate. The results can be plotted as a ``cloud'' of points 
  to reveal patterns in the distribution of loss factors with frequency. These 
  clouds of measured points are plotted below: the guitar in Fig.\ 2, the banjo 
  in Fig.\ 3. Two sets of plucked notes were recorded for each instrument, and 
  processed independently to give an indication of consistency of the 
  measurements: the two sets are plotted as red circles and stars, and a 
  reassuring correspondence can be seen between the two over the important 
  parts of both plots. 

  \fig{figs/fig-32b58107.png}{\caption{Figure 2. Loss factor versus frequency 
  for modes excited by plucking the top string of a guitar. Red circles and 
  stars show measured values; black dots show the predicted loss factor for 
  energy flow into the body alone, calculated from the measured bridge 
  admittance. The analysis method cannot detect modes significantly above the 
  magenta line. Green dashed lines indicate decay time constants $\tau$ = 50 ms 
  (top line), 100 ms, 200 ms, 300 ms and 400 ms (lowest line).}} 

  \fig{figs/fig-a5bccc69.png}{\caption{Figure 3. Loss factor versus frequency 
  for modes excited by plucking the top string of a banjo, in the same format 
  as Fig. 2.}} 

  In each plot, the magenta line indicates the limit of applicability of the 
  analysis method. Above this line the decay time becomes too short to be 
  resolved by the spectrogram approach, and absence of points in this region 
  does not mean that no such combinations of frequency and loss factor exist in 
  the real instruments. Lines to indicate the decay time constant are plotted 
  in green dashes: details are given in the caption of Fig.\ 2. This time 
  constant $\tau$ is associated with an exponential decay of the sound 
  proportional to $e^{-t/\tau}$, and it is related to the frequency and Q 
  factor by 

  \begin{equation*}\tau=\frac{Q}{2 \pi f} . \tag{1}\end{equation*} 

  (Greek letter $\tau$, pronounced ``tau''.) The black points in these plots 
  show estimates of the contribution to the loss factor arising only from 
  energy flow from the string into the instrument body, calculated from the 
  measured bridge admittance. This loss factor was derived and discussed in 
  section 5.1.2, via a calculation of the reflection coefficient for waves on 
  the string, hitting the bridge. The formula for the loss factor was given in 
  eq. (8) of that section. 

  The plots tell an interesting story. It is simplest to explain the guitar 
  case first, Fig.\ 2. The majority of the red points mark out two slightly 
  fuzzy lines, one with loss factors of the order of $10^{-2}$ and the other of 
  the order of $10^{-3}$. These indicate ``body modes'' and ``string modes'' 
  respectively, as mentioned above in connection with Fig.\ 1. The body modes 
  often show as clusters of many points, because in principle these modes are 
  excited by the transient nature of every plucked note, regardless of the 
  played pitch, so that many estimates of these modes are obtained from the 
  chromatic scale. Strictly, each body mode is not exactly the same for every 
  played note, because it is perturbed by coupling to the string. However, 
  except for special cases where a string overtone falls very close to an 
  unperturbed body mode, the shift is small. Probably the line of points for 
  these body modes would continue approximately horizontal beyond the magenta 
  line but for the limitations of the analysis technique. 

  The ``string modes'' consist of an approximately harmonic series based on the 
  fundamental of each successive note, so that the plotted points are spread 
  out along the frequency axis. The body modes have Q-factors around 100 or 
  lower, while the string modes have Q-factors of a few thousand and their 
  decay times determine the duration of each played note. It is very striking 
  in Fig.\ 2 that the line of the string modes is fairly featureless, and 
  mostly lies significantly above the black points. Rather unexpectedly, the 
  decay rate of string modes in the guitar is dominated by the damping of the 
  string itself, and loss into the body of the instrument is usually only a 
  small perturbation. There are exceptions where the guitar body has a strong 
  resonance, but remember that that this particular plot is confined to the 
  frequency range relevant to the top string, and the strongest body resonances 
  of the guitar lie lower in frequency (see section 5.3). 

  The plot for the banjo, Fig.\ 3, is strikingly different. The black points 
  lie considerably higher over much of the frequency range, as a direct result 
  of the higher input admittance of the banjo. There is still a trace of two 
  lines of red points showing string modes and body modes, but whenever the 
  black curve crosses above the position where the line of string modes 
  occurred for the guitar, it carries the actual loss factors up with it. 
  Energy dissipation arising from different physical mechanisms is additive, so 
  in theory the total loss factor cannot be lower than the black curve. 

  In the main this expectation is borne out by the data. There are a few red 
  points lying below the black curve, especially in the frequency range around 
  1 kHz, but these are probably associated with an aspect of the physics not 
  taken into account in this simple description: each string mode can occur in 
  two different polarisations. The description here, and the basis for the 
  calculation of the black points, considers only the string polarisation 
  perpendicular to the banjo membrane. Vibration in the plane parallel to the 
  membrane is likely to couple much less strongly, and thus exhibit lower loss 
  factors. The real plucks during the test procedure will have involved a 
  mixture of both polarisations. Probably a few peaks associated with the 
  second polarisation have been caught by the analysis. This would be expected 
  even if the second polarisation is associated with initially quieter sound. 
  It has a slower decay rate, so after a while it will dominate and be picked 
  up by the computer analysis. 

  The contrast between Figs.\ 2 and 3 has several consequences for the 
  behaviour of notes played on the banjo. For frequencies up to about 1 kHz, 
  the string modes often have significantly higher damping than for the same 
  string attached to a guitar. The decay time will be faster, and at some 
  frequencies it will be so fast that the distinction between string modes and 
  body modes is lost: this is flagged in the plot by the black points reaching 
  levels comparable with the line of body modes. Any note with a fundamental or 
  low overtone lying in one of these frequency ranges might be perceived by a 
  player as ``falling flat'': it will not ring on as much as usual. Secondly, 
  over most of the frequency range plotted here the string modes follow the 
  black points quite closely. This means that most of the energy put into the 
  string is lost by being transferred to the body, whereas in a guitar most of 
  it is dissipated by other loss mechanisms. 

  The result is that a note played on the banjo with the same player gesture as 
  a note on the guitar is likely to sound louder, and decay faster. No player 
  is likely to quarrel with that description. This makes it seem likely that at 
  least part of the essence of ``banjoness'' might be captured by a model based 
  on the effects revealed in these plots. However, in musical acoustics it is 
  often found that perceptual characteristics do not correlate in a 
  straightforward fashion with features that seem obvious in physical 
  measurements. To address this issue, it is necessary to listen to sounds from 
  the synthesis model and find out if they do in fact strike listeners as 
  convincingly banjo-like. 

  \textbf{B: }\textbf{Models and datum cases} 

  In the following subsections, a collection of synthesised banjo sounds will 
  be presented. The first step is to check whether the synthesis models can 
  succeed in ``sounding like a banjo''. After that, by varying parameters 
  within the models we can explore the predicted perceptual effect of some 
  variations, and see whether these match the expectations of banjo players and 
  makers. The same short musical extract is used for all examples: the first 
  few measures of a banjo arrangement of the tune ``The Arkansas traveler''. 

  Any synthesis model requires information about the strings, and the position 
  of the plucking point: information relating to the actual strings with which 
  the Deering banjo is fitted is given in Table 1. The default plucking 
  position is 120 mm from the bridge. 

  The sound files are organised in groups, each of which illustrates a 
  particular type of variation. Within each group of sounds, the scale factor 
  used to generate the sound files is kept constant, so that any variations in 
  loudness are preserved. However, between groups the scale factor may be 
  different. The files fall into two main classes: those based on the use of 
  measured admittance, and those based on what we will call the ``square 
  banjo'' model: it will be described shortly. For each class, a datum case has 
  been chosen as the baseline for comparisons. For measured admittance cases, 
  the datum is based on the admittance for the Deering banjo measured near the 
  first string position. This is the banjo admittance that has been seen 
  repeatedly in sections 5.1--5.3. The same admittance is applied to all five 
  strings. Here is the resulting sound: it is encouragingly banjo-like. 

  \aud{auds/aud-ffeef026-plot.png}{\caption{Sound B.1. The datum sound, based 
  on the measured bridge admittance of the Deering banjo near the first 
  string.}} 

  As will be seen and heard in the next few subsections, a number of 
  interesting questions can be explored using measured admittances. However, 
  there are other questions that cannot be addressed that way, at least without 
  making a lot of banjos with variations in their physical details (such as the 
  diameter of the banjo head). For that purpose, a model is needed that is 
  accurate enough that it still sounds realistically banjo-like, but which 
  allows physical parameters to be changed. This is where the ``square banjo'' 
  model comes in. It takes advantage of some pre-existing mathematical results 
  for the vibration and sound radiation efficiency of rectangular membranes 
  (there are no corresponding analytical results for sound radiation by a 
  circular membrane). Details of this model are given in the next link. 

  The bridge admittance for this datum case is compared with measured 
  admittances in Fig.\ 4. Measured admittances are shown for two cases: with 
  and without the resonator back fitted to the banjo. The case with the 
  resonator back (solid red line) is the datum case, used for Sound B.1. The 
  blue curve shows the datum case for the square banjo model: but this model 
  does not include any allowance for the effect of the resonator, so it is more 
  logical to compare it with the dotted red curve, measured on the real banjo 
  without the resonator. As expected from the discussion in section 4.2.2, the 
  main effect of fitting the resonator is to turn the lowest strong peak in the 
  dotted red curve (around 300 Hz) into a pair or peaks in the solid red curve, 
  lying on either side of the original peak frequency. Apart from this, the two 
  red curves look very similar over the entire frequency range. 

  \fig{figs/fig-17fa1cab.png}{\caption{Figure 4. Bridge admittances used in 
  synthesis. The solid red line is the datum admittance corresponding to Sound 
  B.1, measured on the bridge of the Deering banjo near the first string, with 
  the resonator back fitted to the banjo. The dotted red line shows the 
  corresponding admittance measured with the resonator back removed. The blue 
  line is the datum case of the square banjo model, used for Sound B.2.}} 

  The blue curve follows the trends of these two red curves fairly well. At low 
  frequency, it looks recognisably similar to the dotted red curve, except that 
  the first peak frequency is a little higher in frequency. This is an expected 
  deviation: a circular membrane has a lower fundamental frequency than a 
  membrane of any other shape with the same area, mass and tension. The formant 
  centred around 800 Hz is followed quite convincingly by the square banjo 
  model. The most obvious deviation between the blue and red curves comes at 
  frequencies above 3 kHz, because the real banjo has a ``bridge hill'' which 
  is missing from the square banjo model. This was discussed in section 5.3, 
  and sound examples to illustrate its influence will be given in subsection H 
  below. 

  The sound of the datum ``square banjo'' model appears below. While not 
  identical to the sound based on the measured admittance, it seems 
  sufficiently similar that we can hope the model will capture the perceptual 
  effect of parameter variations well enough to be useful. 

  \aud{auds/aud-dd0b57c3-plot.png}{\caption{Sound B.2. Datum sound for the 
  square banjo model, using the bridge admittance plotted in blue in Fig. 4.}} 

  For both types of synthesis model, there are some additional parameter 
  choices relating to internal ``housekeeping'' issues of various kinds: 
  filtering the output signal to be more like radiated sound, details of 
  damping models, and so on. These are somewhat secondary to our main agenda, 
  but for completeness some discussion and sound examples relating to all these 
  choices is given in the next link. 

  \textbf{C: Different measured admittances} 

  The most obvious thing to explore using synthesis based directly on measured 
  admittance is simply to use bridge admittances from different instruments, or 
  with different setup details on the same instrument, and synthesise the sound 
  assuming the same string behaviour. The first set of sounds relate to 
  admittance measured at different points on the banjo bridge. Sound C.1 is 
  simply an alternative measurement of the datum admittance, on a different 
  occasion. Sound C.2 uses the admittance at the bridge centre, at the position 
  of the third string. 

  \aud{auds/aud-b4fb9621-plot.png}{\caption{Sound C.1. Synthesis based on the 
  measured bridge admittance of the Deering banjo near the first string (red 
  curve in Fig. 4). This should be identical to the datum Sound B.1, except for 
  effects of variability in the experimental setup, and any small changes in 
  the banjo behaviour between the two measurement occasions, for example from 
  the membrane tension changing with temperature or relaxing over time.}} 

  \aud{auds/aud-4ad116eb-plot.png}{\caption{Sound C.2. Synthesis based on the 
  measured bridge admittance of the Deering banjo near the third string (black 
  curve in Fig. 4).}} 

  \aud{auds/aud-093e6eb6-plot.png}{\caption{Sound C.3. Synthesis based on the 
  measured bridge admittances of the Deering banjo, using a different 
  admittance for each string.}} 

  This admittance was plotted in section 5.3, reproduced here as Fig.\ 5. The 
  comparison of the red and black curves in this plot makes it hardly 
  surprising that a very different sound is produced. Sound C.3 uses a 
  different admittance for each of the five strings, and so is closer to the 
  real instrument than any of the cases using the same admittance for all five 
  strings. 

  \fig{figs/fig-c7145ef4.png}{\caption{Figure 5. Admittance at three positions 
  on the banjo bridge, reproduced from Fig. 17 of section 5.3.}} 

  The next group of sounds gives a similar comparison, but using a set of 
  admittances measured with the resonator back removed from the banjo. Sound 
  C.4 corresponds to the measurement position of the datum case, plotted as the 
  dotted red line in Fig.\ 4. The sound is somewhat different from the datum 
  case: compare with either Sound C.1 or Sound B.1. Part of this difference is 
  presumably the direct result of the change in the low-frequency modes, but it 
  also has an element of undesirable ``zinginess'' about it, to which we will 
  return in subsection I. Sounds C.5 and C.6 correspond to Sounds C.2 and C.3 
  respectively, using admittance at the bridge centre, and different 
  admittances for the five strings. Sound C.5 has even more of the 
  ``zinginess'' effect. 

  \aud{auds/aud-812de74a-plot.png}{\caption{Sound C.4. As Sound C.1 but with 
  the resonator back of the banjo removed.}} 

  \aud{auds/aud-f1e35923-plot.png}{\caption{Sound C.5. As Sound C.2 but with 
  the resonator back of the banjo removed.}} 

  \aud{auds/aud-0ba36305-plot.png}{\caption{Sound C.6. As Sound C.3 but with 
  the resonator back of the banjo removed.}} 

  The next two sounds relate to a study carried out in reference [2]. In the 
  process of refining models of the banjo behaviour, some measurements were 
  made with the bridge replaced by a solid circular bridge, placed either at 
  the centre of the head membrane (Sound C.7) or offset to a position similar 
  to that of the regular bridge (Sound C.8). Figure 6 shows a test in progress 
  with this bridge. Only a single string is carried by this ``bridge'', and the 
  tests were done with the resonator back removed. 

  \fig{figs/fig-ec0d7d73.png}{\caption{Figure 6. The banjo being tested with 
  the rigid circular bridge, carrying only one string.}} 

  \aud{auds/aud-75d194e8-plot.png}{\caption{Sound C.7. Synthesis based on the 
  measured bridge admittance of the Deering banjo when fitted with a rigid 
  circular ``bridge'' positioned at the centre of the head membrane.}} 

  \aud{auds/aud-427651a4-plot.png}{\caption{Sound C.8. Synthesis based on the 
  measured bridge admittance of the Deering banjo when fitted with a rigid 
  circular ``bridge'' positioned on the head membrane at a similar position to 
  the datum measurement.}} 

  The corresponding admittances are plotted in Fig.\ 7, in comparison with the 
  result with the normal bridge (also measured without the resonator), the same 
  as the dotted line in Fig.\ 4. It is perhaps not very surprising to discover 
  that these admittances produce distinctive sounds, but reassuring that both 
  sound (to my ears, at least) banjo-like. As would have been expected, the 
  ``bridge hill'' around 3 kHz has disappeared with the rigid bridge. 

  \fig{figs/fig-0d6af2c8.png}{\caption{Figure 7. Measured bridge admittances of 
  the Deering banjo without its resonator back. Blue curve: circular bridge at 
  the centre of the membrane; black curve: circular bridge offset from the 
  centre; red dashed curve: regular bridge, measured near the first string as 
  in Fig. 4.}} 

  Notice one feature of Fig.\ 7, which will be significant when we come to 
  subsection G. At the lowest frequencies, both cases with the circular bridge 
  show significantly higher admittance than the curve for the regular bridge, 
  and the formant has shifted to somewhat lower frequency (peaking around 500 
  Hz rather than 700 Hz). As was explained in section 5.3, this low-frequency 
  behaviour is influenced by a stiffness acting at the bridge, arising from 
  several effects all associated with the strings and their break angle over 
  the bridge. Care was taken to keep that angle the same with the circular 
  bridge. The difference, as can be seen in Fig.\ 6, is that only one string 
  was carried by the circular bridge, instead of five. This gives a large 
  reduction in the stiffness, raising the low-frequency admittance and lowering 
  the formant frequency; despite the fact that the mass of the circular bridge 
  is in fact lower than that of the normal bridge, which would tend to raise 
  the formant frequency. 

  The final set of sounds makes use of bridge admittances from entirely 
  different stringed instruments: two different guitars, and a violin. Sound 
  C.9 is the Woodhouse steel-string guitar used to generate the results in 
  Fig.\ 2, while Sound C.10 uses the admittance of a flamenco guitar by the 
  same maker. Both guitar syntheses give sounds that are much quieter than the 
  banjo cases, and the strings ring on for a lot longer as a result of the 
  lower bridge admittance. These findings are entirely in keeping with the 
  results plotted in Figs.\ 2 and 3. The start of each note tends to have an 
  audible ``thump'': this slightly unnatural effect is probably a consequence 
  of listening to what is essentially the body motion, not the radiated sound. 
  All the modal responses to the pluck are coherent in the body motion, so they 
  add up constructively at the initial instant after the pluck release. 

  \fig{figs/fig-c4454b0c.png}{\caption{Figure 8. Bridge admittances of a 
  steel-string guitar (black solid), a flamenco guitar (black dashed) and a 
  violin (blue), compared to the datum admittance of the banjo (red).}} 

  \aud{auds/aud-c7869b23-plot.png}{\caption{Sound C.9. Synthesis based on the 
  measured bridge admittance of a steel-string guitar, using the same strings 
  as the banjo. This admittance was plotted as the solid black curve in Fig. 
  8.}} 

  \aud{auds/aud-bafb44eb-plot.png}{\caption{Sound C.10. Synthesis based on the 
  measured bridge admittance of a flamenco guitar, using the same strings as 
  the banjo. This admittance was plotted as the dashed black curve in Fig. 8.}} 

  Finally, Sound C.11 gives an indication of what a violin might sound like 
  with a banjo neck and strings fitted to it. In some ways the sound is 
  intermediate between that of the banjo and the guitars. This is consistent 
  with the pattern of the bridge admittances: the admittance of the violin is 
  significantly higher than the guitars in the low kHz range, as a result of 
  the ``bridge hill''. The result is a sound which strikes some listeners as 
  reminiscent of a harpsichord. 

  \aud{auds/aud-6aba2218-plot.png}{\caption{Sound C.11. Synthesis based on the 
  measured bridge admittance of a violin, using the same strings as the banjo. 
  This admittance was plotted as the blue curve in Fig. 8.}} 

  \textbf{D: Strings and plucking} 

  The next set of sounds are all based on the datum measured banjo admittance, 
  and they explore variations in strings and playing details. 

  One easy comparison is to make otherwise identical simulations using 
  different gauges or materials for the strings. The original banjo has plain 
  steel strings for all except the 4th string, which is wrapped. Other 
  materials commonly used for musical instrument strings are nylon, 
  fluorocarbon and natural gut. These materials have all been well 
  characterised in an earlier study [3,4]. The only choice to be made is a set 
  of string gauges for each material. From a scientific standpoint, there is a 
  very straightforward approach to that issue. Since each string has a known 
  length and tuned frequency, the transverse wave speed is fixed. So a natural 
  choice is to keep both the tension and the mass per unit length the same in 
  every material. As a consequence, the characteristic impedance will also be 
  the same. To achieve this, the string diameter $d$ must be adjusted in 
  inverse proportion to the square root of the density ratio of the two 
  materials. The resulting set of gauges for the four materials is shown in 
  Table 2. 

  Sounds D.1, D.2 and D.3 respectively relate to nylon, fluorocarbon and gut 
  strings, and they should all be compared to the datum case Sound B.1 
  (reproduced below for convenience). All the polymeric strings have higher 
  intrinsic damping than the steel strings, and this comes out in a difference 
  of brightness in the sounds. Even so, the brightness may be over-estimated in 
  these sound examples because we have not taken account of the fact that these 
  polymeric materials all have far lower Young's modulus than steel, which 
  would affect the added stiffness effect: see the discussion around Sounds 
  G.8--G.13. 

  \aud{auds/aud-92fdf1d6-plot.png}{\caption{Copy of Sound B.1, with steel 
  strings, for convenience of comparison.}} 

  \aud{auds/aud-5019929e-plot.png}{\caption{Sound D.1. Synthesis based on the 
  datum admittance of the banjo, but with the strings replaced by nylon strings 
  with the set of gauges listed in Table 2.}} 

  \aud{auds/aud-f0af79de-plot.png}{\caption{Sound D.2. Synthesis based on the 
  datum admittance of the banjo, but with the strings replaced by fluorocarbon 
  strings with the set of gauges listed in Table 2.}} 

  \aud{auds/aud-89936760-plot.png}{\caption{Sound D.3. Synthesis based on the 
  datum admittance of the banjo, but with the strings replaced by gut strings 
  with the set of gauges listed in Table 2.}} 

  It should be explained that some banjos do indeed use strings of these other 
  materials, but usually in a context of seeking an ``old-time'' sound, often 
  with a natural skin head with somewhat lower tension than a typical Mylar 
  head. In that context, the actual choice of gauges would typically be lighter 
  than the values given here, and these strings would often be combined with a 
  lighter bridge (to be discussed in subsection G) to increase the brightness 
  of the sound. 

  A different variation in string choice would be to keep steel strings, but 
  choose heavier or lighter gauges than the datum case. Examples of the 
  synthesised result of such a change are given below. The realistic range of 
  gauges is fairly limited, so the cases explored here range from 20\% thinner 
  (Sound D.4) to 20\% thicker (Sound D.8) than the values given in Table 1. The 
  main difference in sound is in the decay rates: lighter gauges ring on a 
  little longer than the datum case, while heavier gauges give a fast-decaying, 
  more ``plunky'' sound. Of course, using a lighter or a heavier string gauge 
  may also affect the feel of the instrument in the hands of the player, but 
  that question would take us into different territory and we ignore it for 
  now. 

  \aud{auds/aud-948ce76d-plot.png}{\caption{Sound D.4. Synthesis based on the 
  datum admittance of the banjo, but with the strings replaced by steel strings 
  with gauges decreased by 20\%.}} 

  \aud{auds/aud-1d5ba040-plot.png}{\caption{Sound D.5. Synthesis based on the 
  datum admittance of the banjo, but with the strings replaced by steel strings 
  with gauges decreased by 10\%.}} 

  \aud{auds/aud-b8d212f7-plot.png}{\caption{Sound D.6. Synthesis based on the 
  datum admittance of the banjo, with the original string gauges.}} 

  \aud{auds/aud-797392ca-plot.png}{\caption{Sound D.7. Synthesis based on the 
  datum admittance of the banjo, but with the strings replaced by steel strings 
  with gauges increased by 10\%.}} 

  \aud{auds/aud-a3dc68d3-plot.png}{\caption{Sound D.8. Synthesis based on the 
  datum admittance of the banjo, but with the strings replaced by steel strings 
  with gauges increased by 20\%.}} 

  Another thing that is easy to explore is the influence of two variables under 
  the player's control: the plucking point and the nature of the plectrum or 
  fingertip. We have already heard a synthesised example of moving the plucking 
  point on a nylon-strung guitar, in section 5.4. Sounds D.9 and D.10 give two 
  similar examples for the banjo. The familiar change of sound associated with 
  this shift is captured well by these sounds. 

  \aud{auds/aud-55049cd8-plot.png}{\caption{Sound D.9. Synthesis based on the 
  datum banjo admittance and strings, with the plucking point moved to 200 mm 
  from the bridge.}} 

  \aud{auds/aud-11ad95ba-plot.png}{\caption{Sound D.10. Synthesis based on the 
  datum banjo admittance and strings, with the plucking point moved to 30 mm 
  from the bridge.}} 

  The effective width of the plectrum or fingertip can also be varied: this 
  models the range of sounds produced by changing from a hard plectrum, to a 
  fingernail, to the flesh of a fingertip. This effect is achieved by a rather 
  crude filter in the model, but the familiar change of sound is quite well 
  captured by Sounds D.11, D.12 and D.13. 

  \aud{auds/aud-2020cc9d-plot.png}{\caption{Sound D.11. Synthesis based on the 
  datum banjo admittance and strings, with a nominal plectrum width 5 mm.}} 

  \aud{auds/aud-43a2655a-plot.png}{\caption{Sound D.12. Synthesis based on the 
  datum banjo admittance and strings, with a nominal plectrum width 20 mm. This 
  is the value used for the datum cases.}} 

  \aud{auds/aud-a00ae93f-plot.png}{\caption{Sound D.13. Synthesis based on the 
  datum banjo admittance and strings, with a nominal plectrum width 50 mm.}} 

  \textbf{E. Alternative synthesis methods} 

  There are two variations in synthesis method that reveal useful information 
  about what contributes to ``characteristic banjo sound''. So far, all 
  examples have considered a single polarisation of string motion. But by 
  measuring the $2 \times 2$ matrix of admittance at the bridge, the frequency 
  domain approach can be extended to include the second polarisation [5]. 
  Allowing for both polarisations brings a new variable into play: the initial 
  angle $\theta$ of the pluck (remember, ``theta''), defined so that $\theta = 
  0$ corresponds to a pluck normal to the banjo head, while $\theta = 90^\circ$ 
  is the opposite extreme, plucking parallel to the head. The output variable 
  is still the motion at the bridge perpendicular to the membrane, since this 
  is the component of head motion mainly responsible for the radiation of 
  sound. 

  Sound E.1 uses only a single polarisation, as in the previous sounds, but it 
  uses the bridge admittance taken from this matrix set, measured some weeks 
  after the admittance used so far. Comparing this sound with the datum (Sound 
  B.1) gives a direct check on the effect of variability, either arising from 
  experimental details not perfectly repeated, or from physical changes in the 
  banjo head over time, for example as a result of changes in temperature or 
  from creep of the tensioned Mylar head. 

  \aud{auds/aud-41d47f87-plot.png}{\caption{Sound E.1. Synthesis based on the 
  measured banjo admittance near the first string, nominally the same as the 
  datum case but measured on a different occasion.}} 

  The remaining three sounds are all calculated using a two-polarisation model, 
  with three different values of $\theta$: Sound E.2 has $\theta = 0^\circ$, 
  Sound E.3 has $\theta = 45^\circ$, and Sound E.4 has $\theta = 90^\circ$. It 
  is hard to know what is the typical angle $\theta$ for normal banjo plucks, 
  but $45^\circ$ seems a good guess. To my own ears, Sounds E.1, E.2 and E.3 
  sound rather similar, suggesting that the precise pluck angle may not make 
  all that much difference. Sound E.4, though, is clearly different. This case 
  is quieter, as would be expected from weaker coupling of string motion to 
  head motion, but it also sounds rather unrealistic. There is a sense of 
  something ringing on in the background. This ringing seems to be an artefact 
  of the synthesis process when damping is too low. We will come back to this 
  issue in subsection I. 

  \aud{auds/aud-dcbe55eb-plot.png}{\caption{Sound E.2. Two-polarisation 
  synthesis based on the measured $2 \times 2$ bridge admittance matrix of the 
  banjo, with a pluck angle $\theta=0^\circ$.}} 

  \aud{auds/aud-8ed410ba-plot.png}{\caption{Sound E.3. Two-polarisation 
  synthesis based on the measured $2 \times 2$ bridge admittance matrix of the 
  banjo, with a pluck angle $\theta=45^\circ$.}} 

  \aud{auds/aud-729184b6-plot.png}{\caption{Sound E.4. Two-polarisation 
  synthesis based on the measured $2 \times 2$ bridge admittance matrix of the 
  banjo, with a pluck angle $\theta=90^\circ$.}} 

  All the synthesised sounds so far have been produced by the frequency domain 
  approach. We now compare the results with some obtained by a modal approach, 
  based on approximating the measured admittance by a modal summation, then 
  using that to calculate the couple modes of string plus body. Sound E.5 gives 
  the modal version of the datum Sound B.1 (given again here for convenience of 
  comparison). The sounds are extremely similar. This confirms that the modal 
  decomposition is sufficiently accurate for the present purpose, and also 
  provides an internal check on the synthesis coding since these two methods 
  are independent. 

  \aud{auds/aud-d22962ef-plot.png}{\caption{Copy of Sound B.1, for convenience 
  of comparison.}} 

  \aud{auds/aud-651b771c-plot.png}{\caption{Sound E.5. Modal-based synthesis, 
  based on a modal decomposition of the datum admittance.}} 

  The modal approach provides an opportunity to do something interesting, which 
  is not possible by the frequency domain method. Each coupled body/string mode 
  can be classified as either a string mode or a body mode, by computing how 
  the energy in the mode is partitioned between string and body. Having 
  separated the modes into these two groups, it is easy to do the synthesis 
  separately for each group. This gives a ``string only'' sound (Sound E.6) and 
  a ``body only'' sound (Sound E.7). Adding these together gives Sound E.5 that 
  we have already heard. On a casual listening, the string-only sound is quite 
  similar to the full synthesis with all modes. However, the difference is 
  clearly audible (provided your audio playback quality is good enough), but it 
  is hard to put into words. 

  \aud{auds/aud-9e4b217c-plot.png}{\caption{Sound E.6. Modal-based synthesis as 
  in Sound E.5, but including only ``string modes''.}} 

  \aud{auds/aud-0b25ef9a-plot.png}{\caption{Sound E.7. Modal-based synthesis as 
  in Sound E.5, but including only ``body modes''.}} 

  Corresponding spectrograms are shown in Figs.\ 9 and 10. The comparison of 
  these spectrograms reveals that it is hardly surprising if the body modes 
  make some audible difference to the sound. The body modes are strongly 
  excited, and although they generally have much faster decay times than the 
  string modes, typical banjo music like the passage used here has notes that 
  come thick and fast, and the body modes ring on for long enough to bridge the 
  gap to the next played note. They surely make a significant contribution to 
  the ``punctuation'' at the start of each note, and it seems likely that this 
  effect forms a significant part of characteristic banjo sound. Note that this 
  body mode contribution to the sound gives a possible mechanism for 
  recognising a particular instrument, to some extent independent of what music 
  is played. The mix of body modes is similar for every note, and constitutes a 
  kind of acoustical fingerprint of the instrument. 

  \fig{figs/fig-45232de7.png}{\caption{Figure 9. Spectrogram of the sample 
  music fragment, synthesised using ``string modes'' only.}} 

  \fig{figs/fig-4707ec3a.png}{\caption{Figure 10. Spectrogram of the sample 
  music fragment, synthesised using ``body modes'' only.}} 

  \textbf{F. Tension and size of head membrane} 

  For the remaining sound examples we turn to the square banjo model, in order 
  to have access to parametric variations. The first set of sounds illustrate 
  the effect of scaling the plan dimensions of the rectangular membrane by 
  various factors, from a half (Sound F.1) to double (Sound F.5) the nominal 
  size. This seems quite a drastic change (the area changes from four times the 
  real banjo down to a quarter of that area), but the effects on sound are 
  surprisingly small. 

  \aud{auds/aud-c68adbc0-plot.png}{\caption{Sound F.1. Synthesised example 
  using the square banjo model, with the linear dimensions of the head scaled 
  down by a factor 0.5 so that the area is only a quarter of the original.}} 

  \aud{auds/aud-110944f4-plot.png}{\caption{Sound F.2. Synthesised example 
  using the square banjo model, with the linear dimensions of the head scaled 
  down by a factor 0.75.}} 

  \aud{auds/aud-303cd735-plot.png}{\caption{Sound F.3. Synthesised example 
  using the square banjo model, with the original head size: this is the same 
  as the datum case, Sound B.2.}} 

  \aud{auds/aud-d8c824e4-plot.png}{\caption{Sound F.4. Synthesised example 
  using the square banjo model, with the linear dimensions of the head scaled 
  up by a factor 1.5.}} 

  \aud{auds/aud-236c7d26-plot.png}{\caption{Sound F.5. Synthesised example 
  using the square banjo model, with the linear dimensions of the head scaled 
  up by a factor 2 so that the area is four times that of the original.}} 

  To see a possible explanation, Fig.\ 11 shows three of the admittances 
  associated with this set. It is clear that all the individual resonances are 
  shifted when the size changes, as expected.But the overall envelope of the 
  admittance, governed by the formant discussed in section 5.3, does not 
  change: it is centred at around 700 Hz in this case. As explained in the 
  earlier discussion, the formant frequency is mainly determined by the mass of 
  the bridge, and by a combination of several sources of stiffness associated 
  with the presence of the strings on the instrument. None of these factors is 
  significantly influenced by changing the size of the head, so the formant 
  does not move. 

  \fig{figs/fig-02967916.png}{\caption{Figure 11. The bridge admittances 
  associated with Sounds F.2 (blue curve), F.3 (the datum case, red curve) and 
  F.5 (black curve).}} 

  Slightly different behaviour is seen in the next case, when the effect of 
  changing the head tension is investigated. Values range from a half (Sound 
  F.6) to double (Sound F.10) the nominal tension. Figure 12 shows a selection 
  of the resulting admittances. In the judgement of the authors, the sounds 
  differ rather more than in the case of changing the size of the head. The 
  formant frequency is affected by the change of tension, as expected and as 
  the plot confirms. This set of sounds, compared with the previous set, 
  suggests that changes in the formant might have larger perceptual 
  significance than changes in the individual resonances within a fixed formant 
  structure. We will explore this idea further with other parametric variations 
  in the next subsection. 

  \aud{auds/aud-7af1fa22-plot.png}{\caption{Sound F.6. Synthesised example 
  using the square banjo model, with the tension of the head scaled down by a 
  factor 0.5.}} 

  \aud{auds/aud-6e5bc805-plot.png}{\caption{Sound F.7. Synthesised example 
  using the square banjo model, with the tension of the head scaled down by a 
  factor 0.75.}} 

  \aud{auds/aud-e39675ff-plot.png}{\caption{Sound F.8. Synthesised example 
  using the square banjo model, with the original tension: this is the same as 
  the datum case, Sound B.2.}} 

  \aud{auds/aud-b42d7f83-plot.png}{\caption{Sound F.9. Synthesised example 
  using the square banjo model, with the tension of the head scaled up by a 
  factor 1.5.}} 

  \aud{auds/aud-6b3ac822-plot.png}{\caption{Sound F.10. Synthesised example 
  using the square banjo model, with the tension of the head scaled up by a 
  factor 2.}} 

  \fig{figs/fig-b1ac0df4.png}{\caption{Figure 12. The bridge admittances 
  associated with Sounds F.6 (blue curve), F.8 (the datum case, red curve) and 
  F.10 (black curve).}} 

  The effects of head size and head tension are only incorporated in this study 
  through their influence on the bridge admittance. On a real banjo these 
  changes would also affect radiation from the instrument, and coupling to the 
  internal air cavity. Both effects were discussed in references [1,2], but the 
  model we are using here is not sophisticated enough to capture the perceptual 
  consequences. 

  \textbf{G. Bridge mass and added stiffness} 

  The next set of sound examples concern effects that are concentrated around 
  the bridge of the banjo: the mass of the bridge, and the additional stiffness 
  from effects of the strings and break angle. Adjusting the bridge mass is 
  something commonly done by banjo players. A rather extreme range is explored 
  in the sound examples below, from an unfeasibly small value of 0.1 g (Sound 
  G.1) up to a very heavy 10 g bridge which would be considered by a player to 
  be a mute (Sound G.7). The sound changes are very significant, and broadly in 
  line with expectations. 

  \aud{auds/aud-d8a0c725-plot.png}{\caption{Sound G.1. Synthesis from the 
  square banjo model, with bridge mass 0.1 g.}} 

  \aud{auds/aud-42f7b9e5-plot.png}{\caption{Sound G.2. Synthesis from the 
  square banjo model, with bridge mass 0.5 g.}} 

  \aud{auds/aud-f84c8c59-plot.png}{\caption{Sound G.3. Synthesis from the 
  square banjo model, with bridge mass 1 g.}} 

  \aud{auds/aud-0780d85d-plot.png}{\caption{Sound G.4. Synthesis from the 
  square banjo model, with bridge mass 1.5 g (the default mass for this 
  model).}} 

  \aud{auds/aud-73663156-plot.png}{\caption{Sound G.5. Synthesis from the 
  square banjo model, with bridge mass 2.2 g (the actual mass of the Deering 
  banjo bridge).}} 

  \aud{auds/aud-e69ab2dc-plot.png}{\caption{Sound G.6. Synthesis from the 
  square banjo model, with bridge mass 5 g.}} 

  \aud{auds/aud-8cb13b0b-plot.png}{\caption{Sound G.7. Synthesis from the 
  square banjo model, with bridge mass 10 g.}} 

  Figure 13 shows examples of the associated admittances. The formant is 
  shifted drastically, as would be expected with these large changes of mass. 
  The choice of a bridge mass of 1.5 g for the datum case was motivated by 
  examining these admittances: it gives a formant frequency and bandwidth which 
  is a fairly close match to the measured admittance. By contrast, using the 
  actual bridge mass of 2.2 g shifts the formant significantly too low in 
  frequency. 

  \fig{figs/fig-a020d056.png}{\caption{Figure 13. The bridge admittances 
  associated with Sounds G.1 (blue curve), G.3 (red curve) and G.6 (black 
  curve).}} 

  Figure 14, and the next set of sound examples, illustrate the effect of 
  changing the additional stiffness. Values vary from essentially zero (Sound 
  G.8) to double the value used in the datum case (Sound G.13). Not 
  surprisingly, increasing the stiffness shifts the formant to higher 
  frequency. The effect is clearly audible. 

  \aud{auds/aud-c8cd04a3-plot.png}{\caption{Sound G.8. Synthesis from the 
  square banjo model, with added stiffness 0 kN/m.}} 

  \aud{auds/aud-763455e5-plot.png}{\caption{Sound G.9. Synthesis from the 
  square banjo model, with added stiffness 5 kN/m.}} 

  \aud{auds/aud-90cca687-plot.png}{\caption{Sound G.10. Synthesis from the 
  square banjo model, with added stiffness 10 kN/m.}} 

  \aud{auds/aud-6869bdbc-plot.png}{\caption{Sound G.11. Synthesis from the 
  square banjo model, with added stiffness 20 kN/m (the default value of this 
  model).}} 

  \aud{auds/aud-32e39aa6-plot.png}{\caption{Sound G.12. Synthesis from the 
  square banjo model, with added stiffness 30 kN/m.}} 

  \aud{auds/aud-b60bc2fd-plot.png}{\caption{Sound G.13. Synthesis from the 
  square banjo model, with added stiffness 40 kN/m.}} 

  \fig{figs/fig-8420596f.png}{\caption{Figure 14. The bridge admittances 
  associated with Sounds G.8 (blue curve), G.10 (red curve) and G.13 (black 
  curve).}} 

  One way to change this additional stiffness has already been seen in Fig.\ 7: 
  a bridge carrying only one string rather than the full set of 5 clearly 
  resulted in a similar change to what is seen in Fig.\ 14. More relevant to 
  normal banjo experience is another parameter change that would lead to an 
  increase in this stiffening effect: it would arise from an increase in the 
  break angle at the bridge; and the effect would disappear with a zero break 
  angle. A prediction would thus be that increasing the break angle should make 
  the sound brighter. Such a change is in accordance with banjo lore. 

  \textbf{H. The effect of the 3 kHz bridge hill} 

  The final change to be illustrated in this discussion is the effect of the 
  ``bridge hill'' occurring at about 3 kHz in the datum admittance. As was 
  explained in section 5.3, this feature is associated with dynamic response of 
  the banjo bridge, in conjunction with the underlying membrane. This feature 
  is not included in the square banjo model. It is of interest to hear the 
  effect of adding the feature back in. 

  The approach that has been used is to take weighted mixtures of the two datum 
  admittances, one measured and one from the square banjo model, using a scaled 
  version of the error function $\mathrm{erf}(\omega)$ to give a smooth 
  transition over a bandwidth of 200 Hz. The different sound examples below 
  correspond to different choices of the transition frequency. Sound H.1, with 
  a transition at 400 Hz, is more or less the original measured admittance. 
  Sounds H.1 and H.3 have transitions at 1.5 kHz and 2 kHz, both low enough 
  that the bridge hill feature is still included in the combined admittance. 
  Sounds H.4 and H.5, with transitions at 5.5 kHz and 9 kHz, do not result in 
  the 3 kHz hill feature being included. To my ears, the biggest change in 
  sound quality comes between Sound H.3 and Sound H.4, when the bridge hill is 
  first removed. 

  \aud{auds/aud-251465c7-plot.png}{\caption{Sound H.1. Combination of 
  admittance from the square banjo model with measured admittance, with a 
  turnover frequency 400 Hz so that the bridge hill is included.}} 

  \aud{auds/aud-37ea2654-plot.png}{\caption{Sound H.2. Combination of 
  admittance from the square banjo model with measured admittance, with a 
  turnover frequency 1.5 kHz so that the bridge hill is included.}} 

  \aud{auds/aud-94b0aa22-plot.png}{\caption{Sound H.3. Combination of 
  admittance from the square banjo model with measured admittance, with a 
  turnover frequency 2 kHz so that the bridge hill is included.}} 

  \aud{auds/aud-e6ee51f6-plot.png}{\caption{Sound H.4. Combination of 
  admittance from the square banjo model with measured admittance, with a 
  turnover frequency 5.5 kHz so that the bridge hill is excluded.}} 

  \aud{auds/aud-d3a36053-plot.png}{\caption{Sound H.5. Combination of 
  admittance from the square banjo model with measured admittance, with a 
  turnover frequency 9 kHz so that the bridge hill is excluded.}} 

  Selected examples of these admittances are plotted in Fig.\ 15. The plot is 
  a little hard to interpret at first glance, because large sections of these 
  curves are identical. For example, the appearance of a blue curve turning 
  into a red curve around 5 kHz is misleading: The red curve is obscured by an 
  identical black curve below that frequency, whereas above that frequency the 
  red curve obscures the blue curve until 9 kHz, where it in turn is obscured 
  by the black curve. Comparing the black and blue curves in this figure gives 
  an illustration of the level of agreement in the formant frequency and 
  bandwidth resulting from the chosen bridge mass and stiffness. This good 
  level of agreement is presumably the reason that Sounds H.1, H.2 and H.3 
  sound rather similar: swapping between the two datum admittances in 
  frequencies up to about 2 kHz makes relatively little difference. 

  \fig{figs/fig-a19cba6a.png}{\caption{Figure 15. The bridge admittances 
  associated with Sounds H.1 (blue curve), H.4 (red curve) and H.5 (black 
  curve).}} 

  These examples are now repeated, this time using the square banjo admittance 
  without the phase compensation fudge that was described in section 5.5.2. The 
  final sounds in this set illustrate the ``zinginess'' problem mentioned 
  earlier, and discussed in the next subsection. 

  \aud{auds/aud-353336e0-plot.png}{\caption{Sound H.6. Combination as in Sound 
  H.1, but using the uncompensated version of the square banjo admittance (see 
  section 5.5.2).}} 

  \aud{auds/aud-376d3c12-plot.png}{\caption{Sound H.7. Combination as in Sound 
  H.2, but using the uncompensated version of the square banjo admittance (see 
  section 5.5.2).}} 

  \aud{auds/aud-acc5d3ec-plot.png}{\caption{Sound H.8. Combination as in Sound 
  H.3, but using the uncompensated version of the square banjo admittance (see 
  section 5.5.2).}} 

  \aud{auds/aud-a71f5820-plot.png}{\caption{Sound H.9. Combination as in Sound 
  H.4, but using the uncompensated version of the square banjo admittance (see 
  section 5.5.2).}} 

  \aud{auds/aud-c6aa94a6-plot.png}{\caption{Sound H.10. Combination as in Sound 
  H.5, but using the uncompensated version of the square banjo admittance (see 
  section 5.5.2).}} 

  \textbf{I. ``Zinginess'' and the damping question} 

  Several of the sound examples in this section have exhibited a phenomenon 
  that results in synthesised sounds that strike many listeners as unrealistic. 
  The effect was first noticed in the context of synthesis using the simplified 
  ``square banjo'' model. Initial efforts to use this model, after adjusting 
  details to give a reasonable-looking match to the measured bridge admittance, 
  suffered from an undesirable ``zingy'' sound. The source of this sound was 
  traced to the fact that the real part of the admittance, containing the 
  information about energy absorption from the string, was significantly lower 
  than the measured values in the frequency range above about 3 kHz. This 
  resulted in coupled string-body modes in that frequency range with damping 
  that was too low, perceived as the zingy sound. Sounds X.15--X.19 and the 
  associated discussion in section 5.5.2 illustrate the phenomenon and the 
  somewhat unsatisfactory ``fudge'' procedure that was used to control it. 

  It is not surprising that synthesised banjo sounds might be sensitive to 
  details of damping. Human perception of transient sounds is often sensitive 
  to decay rates, and hence to damping. The results shown in Figs.\ 2 and 3 
  suggest that decay rates may give a very important cue for identifying an 
  instrument as a banjo. In a wider context, modal decay rates or Q factors 
  provide the main cue for identifying the material of an object as 
  ``metallic'' or ``wooden'' based on the sounds it makes [6]: we heard 
  examples in Chapter 3, associated with various synthesised percussion 
  instruments. 

  Part of the problem with the original square banjo model, therefore, could 
  stem from inadequacy of the damping model used. Radiation damping is captured 
  quite well, but structural damping is difficult to model and predict. This 
  issue is by no means unique to the banjo. In most contexts of vibration 
  prediction and measurement, theoretical models can be expected to give an 
  accurate representation of effects of mass and stiffness, but not of damping. 
  Sophisticated commercial software packages frequently offer only very crude 
  models for damping. This is not evidence of laziness among software 
  engineers, but of a shortfall in understanding of the underlying physics of 
  damping. There is no universal theory of vibration damping on a par with the 
  general linear theory of undamped vibration, based on the Lagrangian approach 
  and leading to concepts like the mass and stiffness matrices (see section 
  2.2.5). 

  However, a poor damping model cannot be the whole story. Unrealistic sounds 
  associated with the effects of low damping are not confined to cases based on 
  theoretical models: some of the synthesised sounds based on measured 
  admittance exhibit related effects. The most striking example is Sound E.4, 
  resulting from a two-polarisation synthesis of a pluck parallel to the banjo 
  membrane. Other cases are Sounds C.4 and C.5. These do not sound identical to 
  the square banjo case, but they are all characterised by a sense of 
  ``something ringing on too long'' in an unrealistic way. The admittances 
  involved in all these examples share the characteristic that the real part 
  gets very low in mid-kHz range: the first example involves the admittance on 
  the bridge top parallel to the membrane, while the other two involve 
  admittance at the bridge centre. 

  There are two possible interpretations of this problem. First, it is possible 
  that there is something not quite accurate in the admittance measurements. 
  The obvious candidate would be the process of compensation of phase to allow 
  for the measurement delay, as described in section 5.5.2. However, that 
  process was no different for these admittances than for others which resulted 
  in synthesised sounds that do not exhibit this problem. 

  The second possibility is that this is a real effect, but one that is not 
  obvious from normal banjo playing. It should be recalled that what is 
  computed by the synthesis algorithm is not the radiated sound, but the motion 
  of the body at the bridge. Different modes of the banjo head have very 
  different levels of radiation damping (discussion and measurements can be 
  found in refs. [1,2]). Modes with low radiation damping, and thus high Q 
  values, presumably do not feature very strongly in the sound received at a 
  distance from the banjo, but they may be strongly present in the body motion. 
  In principle this effect could be allowed for in synthesis using the 
  theoretical model, but it would rely on using the modal approach to 
  synthesis. Synthesis direct from measured admittance can only be done in the 
  frequency domain, so the option to weight the modes differently according to 
  their radiation efficiency is not available. 

  This idea would suggest that a real banjo might sound rather harsh if some 
  kind of body pickup was used to allow amplified sound. It might also sound 
  harsh in a recording using a close microphone, which would pick up near-field 
  sound from modes with low radiation efficiency. Both predictions are 
  consistent with anecdotal evidence about amplifying and recording banjos. So 
  perhaps the banjo, with very low intrinsic damping in both strings and body, 
  really is on the edge of making unpleasant sounds like those heard in the 
  synthesised files. 

  \textbf{J. Summary} 

  This extended case study has given examples of how physically-based sound 
  synthesis can be used to shed light on questions of interest to makers and 
  players of instruments, as well as to scientists. The banjo has been used 
  because it represents an extreme case in more than one sense. The sound of 
  the banjo is extreme among familiar plucked-string instruments, for reasons 
  that have been illustrated here. The strings of a banjo are coupled more 
  strongly to the body of the instrument, leading to loud sound that decays 
  rapidly, and to a significant component of ``body sound'' accompanying each 
  plucked note (recall Sounds E.5--E.7). 

  But the banjo is also extreme in the sense that more of the key variables can 
  be adjusted by the player. The choice of strings is always available to the 
  player of any instrument, but in the banjo the head tension and the mass, 
  break angle and other design features of the bridge can also be changed. 
  Synthesised sound examples associated with all these changes have been given 
  here. In the main, the impression given by these sounds agrees quite well 
  with the generally accepted view among banjo players of the tonal effects of 
  changes to head tension and bridge details. 

  \textbf{Postscript: experiments with different bridges} 

  All the results we have been describing in this section were concerned with 
  measurements and modelling of a single banjo, and a single bridge. After the 
  study described in this section had been completed, we thought it would be 
  interesting to see if the predictions of the modelling would carry over to 
  bridges of different types: banjo bridges are available in a wide variety of 
  shapes, weights and timbers. Many different bridges were fitted to the banjo 
  used here, and the admittances measured at the 1st and 3rd string positions. 
  Those admittances were used to make another set of synthesised sound files, 
  to illustrate at least some aspects of the predicted differences of sound. To 
  see and hear these results, go to \tt{}<span class="has-inline-color 
  has-vivid-red-color">this side link</span>\rm{}. 



  \sectionreferences{}[1] Jim Woodhouse, David Politzer and Hossein Mansour. 
  ``Acoustics of the banjo: measurements and sound synthesis'', Acta Acustica 
  \textbf{5}, 15 (2021). The article is available here: 
  \tt{}https://doi.org/10.1051/aacus/2021009\rm{} 

  [3] N. Lynch-Aird and J. Woodhouse ``Comparison of mechanical properties of 
  natural gut and synthetic polymer harp strings''. \tt{}Materials \textbf{11}, 
  2160, (2018)\rm{}.~ 

  [4] J. Woodhouse and N. Lynch-Aird ``Choosing strings for plucked musical 
  instruments''. \tt{}Acta Acustica united with Acustica \textbf{105}, 516-529, 
  (2019)\rm{}.~ 

  [5] J. Woodhouse ``On the synthesis of guitar plucks''.~ Acta Acustica united 
  with Acustica \textbf{90}, 928–944 (2004). 

  [6] S. McAdams, A. Chaigne and V. Roussarie ``The psychomechanics of 
  simulated sound sources: Material properties of impacted bars''. Journal of 
  the Acoustical Society of America \textbf{115}, 1306--1320 (2004). 