  We can approximate the acoustical behaviour of the mouthpiece of a typical 
  brass instrument by the simplified Helmholtz resonator model sketched in 
  Fig.\ 1. The cup has volume $V$, and the neck and backbore are idealised as a 
  cylinder of length $L$ and cross-sectional area $S$. The player injects a 
  volume flow rate $qe^{i \omega t}$ via their lips. The corresponding flow 
  rate through the exit tube is $q_1e^{i \omega t}$. From now on, for clarity I 
  will omit the factors $e^{i \omega t}$ on everything. The pressure inside the 
  cup is $p_1$, the pressure just outside the neck, at the entry to the 
  instrument tube, is $p_2$, and the instrument tube presents an input 
  impedance 

  $$Z(\omega)=p_2/q_1 \tag{1}$$ 

  at that point. Our aim is to determine the modified input impedance 

  $$Z_{in}(\omega)=p_1/q \tag{2}$$ 

  at the player's lips, on the dotted line in the figure. 

  \fig{figs/fig-de7ce7a8.png}{\caption{Figure 1. Sketch of the Helmholtz 
  resonator mouthpiece model}} 

  The net mass flow into the volume $V$ is $\rho_0 (q-q_1)$, where $\rho_0$ is 
  the mean density of air. This is balanced by a density perturbation $\rho'$ 
  inside the cup. For our harmonic signal, this gives 

  $$\rho_0(q-q_1)=i \omega V \rho' = i \omega V \dfrac{p_1}{c^2} \tag{3}$$ 

  using the relation between density and pressure via the speed of sound $c$. 
  Now we apply Newton's law to the mass of air in the exit tube, treating it as 
  if it were rigid just as we did when we looked at the Helmholtz resonator in 
  section 4.2.1: 

  $$(p_1-p_2) S = L S \rho_0 \dfrac{\partial u}{\partial t} \tag{4}$$ 

  where $u$ is the flow speed. But $q_1 = Su$, so for harmonic response 

  $$L \rho_0 i \omega \dfrac{q_1}{S}=p_1-p_2 = p_1 -Zq_1 \tag{5}$$ 

  using equation (1). 

  Eliminating $q_1$ using equation (3) we obtain 

  $$p_1=\left[ Z+\dfrac{i \omega L \rho_0}{S} \right] \left[ q-\dfrac{i \omega 
  V p_1}{\rho_0 c^2} \right] \tag{6}$$ 

  and so 

  $$Z_{in} =\dfrac{p_1}{q}=\dfrac{Z+i \omega L \rho_0/S}{1+ \dfrac{i \omega V 
  Z}{\rho_0 c^2} -- \omega^2 \dfrac{LV}{Sc^2}} . \tag{7}$$ 

  Now we can recall from section 4.2.1 that the resonance frequency $\omega_m$ 
  of an isolated Helmholtz resonator is given by 

  $$\omega_m^2 =\dfrac{c^2S}{VL} . \tag{8}$$ 

  If we introduce non-dimensionalised versions of the impedances defined by 

  $$X=\dfrac{i \omega V}{\rho_0 c^2}Z \mathrm{~~~and~~~} X_{in}=\dfrac{i \omega 
  V}{\rho_0 c^2}Z_{in} \tag{9}$$ 

  we can simplify equation (7) into the form 

  $$X_{in}=\dfrac{X-\omega^2/\omega_m^2}{X+1-\omega^2/\omega_m^2}. \tag{10}$$ 

  Finally, recognising the term $1-\omega^2/\omega_m^2$ in the denominator as a 
  form of the usual resonant denominator for the response of a single degree of 
  freedom oscillator, we can insert a factor to take account of the damping of 
  the Helmholtz resonator. If the Q-factor of the mouthpiece alone is $Q_m$, we 
  can write 

  $$X_{in} \approx \dfrac{X-\omega^2/\omega_m^2}{X+1+i \omega/\omega_m 
  Q_m-\omega^2/\omega_m^2}. \tag{11}$$ 

  We can get an idea of what equation (10) or (11) predicts about resonance 
  frequencies if we think about the undamped version, for which $X(\omega)$ is 
  a purely real function. The original resonances of the tube are given by the 
  poles of $X$, while the new ones are given by the zeros of the denominator of 
  equation (10). In other words, they occur at frequencies satisfying 

  $$X(\omega)=\dfrac{\omega^2}{\omega_m^2}-1. \tag{12}$$ 

  We can represent this equation graphically, as indicated schematically in 
  Fig.\ 2. 

  \fig{figs/fig-67dfe1b4.png}{\caption{Figure 2. Sketch of the graphical 
  construction to visualise the new resonance frequencies predicted by equation 
  (10). The green dashed line marks the Helmholtz resonance frequency 
  $\omega\_m$, where the red curve crosses the horizontal axis.}} 

  The graph of $X$, shown in blue, looks a bit like an upside-down version of 
  the tangent function. The original resonance frequencies are marked by the 
  vertical dotted lines, where $X$ goes off to infinity. Now we superimpose a 
  graph of the function $\omega^2/\omega_m^2 -- 1$, in red. This curve crosses 
  the blue curve at the points circled in green. It is immediately obvious that 
  there must be one of these green circles in every gap between an adjacent 
  pair of tube resonances. 

  For the purposes of the simulation results shown in section 11.5, parameter 
  values for a particular trombone mouthpiece were used. According to Campbell 
  et al. [1], the Denis Wick 7AL mouthpiece has a cup volume $V=12 \times 
  10^3\mathrm{~mm^3}$, and a direct measurement of the mouthpiece in isolation 
  gives a Helmholtz resonance frequency 535~Hz and a Q-factor of 23. However, 
  when the mouthpiece is plugged into the trombone tube the end correction will 
  be bigger than in free space, and a better fit to the impedance was given by 
  assuming that the relevant Helmholtz resonator frequency was reduced to 
  460~Hz. 

  \sectionreferences{}[1] Murray Campbell, Joël Gilbert and Arnold Myers, “The 
  science of brass instruments”, ASA Press/Springer (2021) 