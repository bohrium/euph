  Now we have enough background material on vibration and acoustics, we use it 
  in this chapter to start looking at stringed instruments. As explained back 
  in Chapter 2, we want to stick as long as we can with problems that can be 
  tackled using linear theory. That means we won't be looking yet at what 
  happens when a string is bowed, but we already know enough to look at plucked 
  strings. In any case, the body vibration of all stringed instruments is 
  generally small enough that linear theory works well, so although the chapter 
  is mainly focussed on plucked strings, we can say some useful things already 
  about the body vibration of a violin or cello. 

  Section 5.1 gives a general overview of how stringed instruments work. The 
  behaviour of an instrument body is most easily captured by measuring a 
  frequency response function. Since the vibrating strings mainly drive the 
  body at the bridge, this also makes the best place to apply forcing in order 
  to do laboratory measurements. Throughout this chapter, examples of the 
  resulting bridge admittance functions from a guitar, a violin and a banjo 
  will be used to highlight the similarities and differences between these 
  instruments, to give a sense of the wide range of stringed instruments. 

  Section 5.2 will look at the question of what determines how loud a stringed 
  instrument can be. For many instruments, increasing loudness seems to have 
  formed part of the motivation for changing designs over the years: a clear 
  example is the modern piano, compared to its predecessors the harpsichord and 
  fortepiano. It also may have been true of the violin: we will look at a 
  number of design features of the violin, and suggest that these may have been 
  motivated, at least in part, by a desire to make a loud instrument. 

  Section 5.3 gives a systematic comparison between the guitar, violin and 
  banjo. This section will emphasise the importance of two aspects of design. 
  All these instruments have signature modes at low frequency, which can be 
  manipulated individually by the instrument maker, and these account for some 
  aspects of the sound quality of the instruments. But there is a second, and 
  less obvious aspect of the vibration behaviour which an instrument maker can 
  manipulate. Many instruments, although apparently not the guitar, show 
  formants in their frequency response. These are bands of frequency where all 
  the modes show enhanced amplitude. The violin, cello and banjo all exhibit 
  such formants, and understanding them can explain many things about ``tonal 
  adjustment'' of instruments. 

  Section 5.4 explores how we can put together things we have already met, in 
  order to build computer models for plucked-string instruments. These models 
  can be used to synthesise sounds. That in turn can allow ``virtual 
  adjustments'' to be carried out in the computer, to explore the effect on 
  sound of changes to the body vibration behaviour, or to things the player can 
  control (such as the plucking point along the string). 

  Finally, section 5.5 gives an extended case study of the banjo. The banjo, 
  with its drum-like stretched membrane instead of a wooden soundboard, 
  represents an extreme case among plucked-string instruments. In this section 
  we explore whether the models we have developed are good enough to capture 
  the key differences, and thus to ``sound like a banjo''. An extensive set of 
  virtual adjustments are illustrated. 

