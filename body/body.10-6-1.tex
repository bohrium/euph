  The full mathematical and computational details of how nearfield acoustic 
  holography works are explained in the textbook by Earl Williams [1], but I 
  will give an outline account here to bring out some of the key features. The 
  method fits within a very broad class of mathematical problems called 
  ``boundary value problems'': we have a governing differential equation which 
  we wish to solve within some region, plus enough conditions on the boundaries 
  of that region to guarantee that there is only one possible solution. 

  Mathematicians have devoted a lot of effort over the years to establishing 
  just what is needed in terms of boundary conditions to go with particular 
  differential equations, in order to ensure a unique solution. We will make 
  use of one of those results --- but we won't prove the underlying uniqueness 
  theorem here: for that, the curious reader will have to consult a textbook 
  such as Morse and Feshbach [2]. 

  An easily visualised example of a boundary value problem is to do with heat 
  flow. Suppose we take a block of metal, and we sit it in a bath of icy water 
  so that the bottom and the sides are all held at $0^\circ$ C. But the top 
  surface is left out of the water, so that it is at room temperature, 
  $20^\circ$ C. We have fixed the temperature on all the surfaces of our block, 
  and that is enough information to determine, uniquely, the distribution of 
  temperature inside the block after it has been left to sit for long enough 
  for the temperature to settle to its equilibrium pattern. 

  Essentially the same boundary conditions are appropriate for our acoustics 
  problem. If we have a region containing no sound sources, the sound field 
  within the region must satisfy the wave equation (see section 4.1.1). If the 
  sound pressure is known at every point on the boundaries of our region, that 
  is enough information to determine the complete internal sound field (these 
  are called ``Dirichlet boundary conditions'' in the mathematician's jargon). 

  Figure 1 shows the region of interest for NAH reconstruction. Measurements 
  are made on a plane, shown as the solid line and labelled as the hologram 
  plane. For the purposes of theory we will imagine that the sound pressure has 
  been measured over this entire plane. The sound pressure distribution over 
  this plane has been caused by our vibrating guitar top (or whatever other 
  object is being tested). We can idealise this as a source plane, shown as the 
  dashed line. 

  \fig{figs/fig-66d3c263.png}{Figure 1. The domain in which NAH reconstructions 
  are carried out} 

  We can see straight away that there is a potential ambiguity. Exactly the 
  same sound pressure distribution would be created at the hologram plane if 
  the source plane were swapped to be at the same distance on the opposite 
  side. For an unambiguous reconstruction, the reconstruction method needs to 
  know ahead of time which side the source plane is on. This knowledge is 
  ``hard-wired'' into the algorithm: the two sides of the hologram plane need 
  to be treated in very different ways. 

  If we were only interested in reconstructing the sound field above the 
  hologram plane in Fig.\ 1, things would be easy. The boundary value problem 
  to be solved has the known (measured) sound pressure over the hologram plane. 
  To define the rest of our region for the purposes of the uniqueness theorem 
  stated earlier, we augment the plane with a very large hemisphere, shown 
  dashed, which we imagine to be pushed out ``to infinity''. On that 
  hemisphere, we know two things: the sound pressure tends to zero, and it 
  consists entirely of outgoing waves. There are no sound sources out there, so 
  no waves can be coming inwards ``from infinity''. (In the jargon, this is 
  sometimes called the ``Sommerfeld radiation condition''.) 

  We have now satisfied the conditions of the theorem, and the sound field 
  within this region is uniquely determined. In fact, there is a 
  straightforward way to compute it, which is a close relative to something we 
  have already seen, back in section 4.3.2. Then, we were looking at the 
  radiation of sound by a vibrating plate. Although we didn't mention it at the 
  time, this was also a boundary value problem which was formally completed by 
  a hemisphere at infinity on which the Sommerfeld radiation condition was 
  satisfied. We showed that the resulting sound field pressure $p(\mathbf{x})$ 
  at position $\mathbf{x}$ was given by the Rayleigh integral 

  $$p(\mathbf{x}) = e^{i \omega t} \int_S{G(\mathbf{x},\mathbf{x'}) 
  v(\mathbf{x'}) d^2 \mathbf{x'}} \tag{1}$$ 

  where the integral is over the plane $S$ of the vibrating source plate with 
  velocity $v(\mathbf{x'})e^{i \omega t}$ at position $\mathbf{x'}$, $d^2 
  \mathbf{x'}$ is the element of area in that plane, and the function $G$ is 
  defined by 

  $$G(\mathbf{x},\mathbf{x'})= \dfrac{i \omega \rho_0}{2 \pi}\dfrac{e^{i k 
  |\mathbf{x}-\mathbf{x'}|}}{|\mathbf{x}-\mathbf{x'}|} . \tag{2}$$ 

  where $\rho_0$ is the density of air and $k=\omega/c$ is the wavenumber of 
  sound waves in air, where $c$ is the speed of sound. Equation (1) takes the 
  form of a convolution integral, and the function $G$ is an example of 
  something called called a Green's function. (These are named after 
  \tt{}George Green\rm{}, an English mathematician of the early 19th century.) 

  Our hologram reconstruction problem is very similar, and it can be solved by 
  a convolution integral of exactly the same form as equation (1), with 
  $v(\mathbf{x'})$ replaced by $p_m(\mathbf{x'})$, the measured pressure at 
  position $\mathbf{x'}$ in the hologram plane: 

  $$p(\mathbf{x}) = e^{i \omega t} \int_H{F(\mathbf{x},\mathbf{x'}) 
  p_m(\mathbf{x'}) d^2 \mathbf{x'}} \tag{3}$$ 

  where now the integral is over the hologram plane $H$. We need a different 
  Green's function, $F$, which is closely related to the ``free space Green's 
  function'' $G$ of equation (2): 

  $$F(\mathbf{x},\mathbf{x'})=- \dfrac{1}{2 \pi} \dfrac{\partial}{\partial 
  z'}\left[ \dfrac{e^{i k 
  |\mathbf{x}-\mathbf{x'}|}}{|\mathbf{x}-\mathbf{x'}|}\right] \tag{4}$$ 

  where $z'$ denotes the component of $\mathbf{x'}$ in the direction 
  perpendicular to the hologram plane. 

  To reconstruct the sound field on the other side of the hologram plane, 
  between there and the source plane, is more difficult. The source of sound 
  waves is now on the lower plane, so we can still apply equation (3) to 
  calculate the sound field at any position above that plane in terms of the 
  pressure $p_s$ on the source plane: 

  $$p(\mathbf{x}) = e^{i \omega t} \int_S{F(\mathbf{x},\mathbf{x'}) 
  p_s(\mathbf{x'}) d^2 \mathbf{x'}} \tag{5}$$ 

  where the integral is now over the source plane $S$. But we don't know $p_s$! 
  Our measurement is of $p_m$, which is inside the new region. They are, of 
  course, related by 

  $$p_m(\mathbf{x}) = \int_S{F(\mathbf{x},\mathbf{x'}) p_s(\mathbf{x'}) d^2 
  \mathbf{x'}} \tag{6}$$ 

  where both $p_s$ and $p_m$ are assumed to be sinusoidal at frequency 
  $\omega$. 

  We need to invert equation (6) to obtain $p_s$ in terms of $p_m$. The key to 
  that is to Fourier transform everything in sight, and take advantage of a 
  mathematical result called the convolution theorem (we proved the basic 
  version of this theorem back in section 2.2.8). That theorem states that the 
  Fourier transform of a convolution integral is the product of the separate 
  Fourier transforms of the input function ($p_s$ in our case) and the Green's 
  function $F$. There may be a numerical factor as well, depending on the 
  particular convention in use for Fourier transforms: for my convention (see 
  section 2.2.1) there is a factor $4 \pi^2$ for the case we are interested in, 
  which involves a two-dimensional spatial Fourier transform in the two 
  in-plane directions (denoted $x$ and $y$), without affecting the dependence 
  on the perpendicular direction $z$. If we denote Fourier transformed 
  functions by a hat, then for example 

  $$\hat{F}(k_1,k_2,z)=\dfrac{1}{4 
  \pi^2}\int_{-\infty}^\infty{\int_{-\infty}^\infty{F(x,y,z)e^{i k_1 x+ik_2 y} 
  dx}dy} \tag{7}$$ 

  where $k_1$ and $k_2$ are the wavenumbers in the $x$ and $y$ directions. 

  By virtue of the convolution theorem, equation (6) then becomes 

  $$\hat{p}_m=4 \pi^2 \hat{F} \hat{p}_s \tag{8}$$ 

  so that we can accomplish the inversion we need trivially: 

  $$\hat{p}_s=\dfrac{\hat{p}_m }{4 \pi^2 \hat{F}} . \tag{9}$$ 

  An inverse Fourier transform then gives the pressure distribution we need. 
  However, there is a snag to be considered. Because of the division by 
  $\hat{F}$, the result may become seriously inaccurate if that function 
  becomes small for some values of the wavenumbers $k_1$ and $k_2$. This is 
  indeed a danger, for reasons that relate to another topic we investigated 
  earlier. 

  The physical interpretation of $\hat{F}$ for particular values of $k_1$ and 
  $k_2$ is that it describes the amplitude and phase of sound pressure at a 
  distance from the source plane, in response to a plane wave of pressure in 
  that source plane. The plane wave has a resultant wavenumber 
  $\sqrt{k_1^2+k_2^2}$. This is a similar question to the one we looked at in 
  section 4.3, with details in section 4.3.3: the sound radiated by a plane 
  wave travelling in a plate in contact with the air. 

  What we found in section 4.3 was that when the resultant wavenumber on the 
  surface is smaller than the wavenumber $k$ for sound in air, propagating 
  waves are generated so that on average the amplitude of sound is the same at 
  all distances from the surface. But when the resultant wavenumber on the 
  surface is greater than $k$, evanescent waves are generated which decay 
  exponentially away from the surface. Figure 2 reminds us what these 
  evanescent waves look like: it is a repeat of Fig.\ 13 from section 4.3. 

\moobeginvid\begin{tabular}{ccc} \vidframe{ 0.30 }{ vids/vid-b4246201-00.png }&\vidframe{ 0.30 }{ vids/vid-b4246201-01.png }&\vidframe{ 0.30 }{ vids/vid-b4246201-02.png } \end{tabular}\caption{Figure 2. A repeat of Fig. 13 from section 4.3, illustrating an evanescent wave that results when the wavenumber on the surface is higher than the wavenumber for sound in air at the frequency of interest.}\mooendvideo

  We learn three important things from this. First, high wavenumbers in the 
  Fourier transform are indeed likely to lead to very small values of $\hat{F}$ 
  because of the exponential decay, which gets faster as the wavenumber gets 
  bigger. The consequence for practical implementations of NAH is that it is 
  necessary to impose a low-pass filter on the wavenumbers used in the Fourier 
  decomposition, before the inverse transform is performed. Second, we see why 
  it is useful to do the measurement with the hologram plane as close as 
  possible to the source plane. The shorter the distance over which $\hat{F}$ 
  operates, the less scope there is for the exponential decay to produce very 
  small values. Finally, we see why the word ``nearfield'' gets into the name 
  of this method. To obtain maximum resolution for features in the source 
  plane, the hologram plane needs to be deep in the nearfield of the radiated 
  sound, so that the evanescent field components can be sampled without being 
  overwhelmed by measurement noise. 

  \sectionreferences{}[1] Earl G. Williams, “Fourier Acoustics”, Academic Press 
  (1999). 

  [2] P. M. Morse and H. Feshbach, “Methods of Theoretical Physics”, Volume I, 
  McGraw-Hill (1953). 