  Energy loss from a vibrating string due to viscosity in the surrounding air 
  can be estimated using a classical analysis going back to Stokes. The 
  associated loss factor for a string mode at angular frequency $\omega_n$ is 
  given by Fletcher and Rossing [1] in the form 

  \begin{equation*}\eta_{air} \approx \dfrac{\rho_a}{\rho} \dfrac{2 \sqrt{2} 
  M+1}{M^2} \tag{1}\end{equation*} 

  \noindent{}where $\rho_a$ is the density of air, and 

  \begin{equation*}M=\dfrac{d}{4} \sqrt{\dfrac{\omega_n}{\eta_a}} 
  \tag{2}\end{equation*} 

  \noindent{}where $\eta_a$ is the kinematic viscosity of air. Textbook values 
  can be used: $\rho_a = $1.2 kg/m$^3$ and $\eta_a = 1.5\times 10^{-5}$ 
  m$^2$/s. In the light of tests on many musical strings covering a wide range 
  of string gauges, it has been found that this formula does not quite capture 
  the variation with diameter to best accuracy: it slightly underestimates the 
  damping of thick strings and overestimates that of thin strings. It has been 
  found that the measurements can all be approximated well enough for the 
  present purpose by applying an ad hoc correction factor $(d+0.2)$ to 
  $\eta_{air}$, with the string diameter $d$ expressed in mm. 

  To see more clearly how this damping contribution varies with frequency, it 
  may be noted that for most musical strings the value of $M$ is fairly large, 
  so that 

  \begin{equation*}\eta_{air} \approx (d+0.2) \dfrac{\rho_a}{\rho} \dfrac{2 
  \sqrt{2}}{M} = (d+0.2) \dfrac{8 \sqrt{2} \rho_a}{\rho d} 
  \sqrt{\dfrac{\eta_a}{2 \pi f_n}} . \tag{3}\end{equation*} 

  \sectionreferences{}[1] Neville H Fletcher and Thomas D Rossing; ``The 
  physics of musical instruments'', Springer-Verlag (Second edition 1998) 