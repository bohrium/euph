  To estimate the loss factor associated with bending deformation of the 
  string, we can use yet another argument based on Rayleigh's principle, on the 
  assumption that material damping is small, as it invariably is for the 
  materials that musical strings are made from. We already know the expressions 
  for the kinetic and potential energies of the system (without damping). We 
  also already have a good idea of the mode shapes of the undamped system: they 
  are the sinusoidal shapes from section 3.1.1. 

  Now we use an argument to which we will return in rather more detail in 
  section 10.3.3. To solve the damped problem we can replace $E$ in the 
  expression for potential energy with a complex version $E(1+i \eta_E)$. Now 
  Rayleigh’s principle says that given an approximation to a mode shape we can 
  get a rather good approximation to its natural frequency by evaluating the 
  Rayleigh quotient. The modes of the damped system will be slightly different 
  from the modes of the undamped system, but the undamped mode shapes will 
  still give a good approximation. So we evaluate the Rayleigh quotient using 
  the true expression for the potential energy, with the complex modulus, but 
  with the approximate expression for mode shape from the undamped calculation. 

  This gives a good approximation to the natural frequency, which will now be a 
  complex number: $\omega_n=a+ib$, say. So the time dependence of free 
  oscillation of this mode will be proportional to 

  $$e^{i \omega_n t}=e^{i(a+ib)t}=e^{iat} e^{-bt} .\tag{1}$$ 

  The real part of the complex frequency describes the oscillation frequency as 
  before, while the imaginary part describes an exponential decay, just as we 
  would expect for a damped system. If we define a loss factor $\eta_{bend}$ to 
  describe this energy loss mechanism, then from the definition of loss factor 

  $$b=a \eta_{bend}/2$$ 

  so that 

  $$\eta_{bend} = \frac{2b}{a} =\frac{2\Im (\omega_n)}{\Re (\omega_n)} \approx 
  \frac{\Im (\omega_n^2)}{\Re (\omega_n^2)} \tag{2}$$ 

  in terms of the real and imaginary parts of the complex frequency, where we 
  have assumed small damping so that $b \ll a$. 

  The Rayleigh quotient for the damped stiff string, from eq. (6) of section 
  5.4.3, is 

  $$\omega_n^2 \approx \dfrac{\int_0^L{P \left(\frac{\partial u_n}{\partial 
  x}\right)^2} dx + \int_0^L{E(1+i \eta_E) I \left(\frac{\partial^2 
  u_n}{\partial x^2}\right)^2} dx}{\int_0^L{m u_n^2} dx} \tag{3}$$ 

  so that 

  $$\omega_n^2 \approx \dfrac{P}{m} \left(\dfrac{n \pi}{L}\right)^2 + 
  \dfrac{E(1+i \eta_E)I}{m} \left(\dfrac{n \pi}{L}\right)^4 . \tag{4}$$ 

  Using eq. (2), we can deduce 

  $$\eta_{bend} \approx \dfrac{E \eta_E I \left(\dfrac{n \pi}{L}\right)^2}{P} = 
  2 \alpha n^2 \eta_E$$ 

  in terms of the inharmonicity parameter $\alpha$ introduced in section 5.4.3. 