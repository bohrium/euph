  The mass-spring oscillator is easy to analyse. If the displacement of the 
  mass away from equilibrium is denoted $x(t)$, then Newton's law requires 

  \begin{equation*}m \dfrac{d^2x}{dt^2} + kx=0 \tag{1}\end{equation*} 

  \noindent{}for free motion of the mass, with no external force applied. If a 
  force $f(t)$ were to be applied, it would simply replace the zero on the 
  right-hand side of this equation. The general solution of the equation for 
  free motion can be written 

  \begin{equation*}x(t)=A \cos(\Omega t) + B \sin(\Omega t) 
  \tag{2}\end{equation*} 

  \noindent{}where $A$ and $B$ are arbitrary constants, and 

  \begin{equation*}\Omega^2=\dfrac{k}{m}. \tag{3}\end{equation*} 

  $\Omega$ is the oscillation frequency in radians per second: it is related to 
  the frequency $f$ in Hz by $\Omega = 2 \pi f \tag{4}$. 

  This solution can be written in other forms. The sin and cos terms can be 
  combined into a single sine wave: 

  \begin{equation*}x(t) = R \cos(\Omega t + \phi) \tag{5}\end{equation*} 

  \noindent{}where the amplitude 

  \begin{equation*}R = \sqrt{A^2 + B^2} \tag{6}\end{equation*} 

  \noindent{}and the phase shift $\phi$ satisfies 

  \begin{equation*}\tan{\phi} = -B/A. \tag{7}\end{equation*} 

  The form (5) makes it particularly clear that the most general free motion of 
  the oscillator consists of sinusoidal oscillation at (radian) frequency 
  $\Omega$, with arbitrary amplitude and phase. 

  The solution can also be written compactly in terms of the complex 
  representation (look back at section 2.1.2 if you are a bit hazy about 
  complex numbers in this application): 

  \begin{equation*}x(t)= C e^{i \Omega t} \tag{8}\end{equation*} 

  \noindent{}where as usual we implicitly assume that the real part of this 
  complex expression is to be taken. The complex amplitude $C$ is given by 

  \begin{equation*}C=R e^{i \phi}. \tag{9}\end{equation*} 

  Another important aspect of the behaviour of the simple oscillator is the 
  response to a sinusoidal applied force. The easiest way to analyse this case 
  is to use the complex representation, and assume an input $Fe^{i \omega t}$ 
  at some chosen frequency $\omega$. From the ``sine wave in, sine wave out'' 
  property, we know that the response $x(t)$ will also vary sinusoidally at the 
  same frequency, so write it in the form $X e^{i \omega t}$. Equation (1) then 
  requires 

  \begin{equation*}m \dfrac{d^2}{dt^2}\left[X e^{i \omega t} \right] + kX e^{i 
  \omega t}=Fe^{i \omega t} . \tag{10}\end{equation*} 

  Thus 

  \begin{equation*}(- \omega^2 m+k)X=F \tag{11}\end{equation*} 

  \noindent{}so that the frequency response function of the oscillator, $X/F$, 
  is given by 

  \begin{equation*}\dfrac{X}{F} = \dfrac{1}{k-\omega^2 m}=\dfrac{1/m}{\Omega^2 
  -- \omega^2} \tag{12}. \end{equation*} 

  For this particular linear system, the frequency response function turns out 
  not to be complex, but it does vary with frequency $\omega$. For low 
  frequencies $\omega < \Omega$, the result is positive so that the response is 
  in phase with the force. For higher frequencies $\omega > \Omega$, the value 
  is negative so that the response in the opposite phase to the force. When 
  $\omega$ is close to the natural frequency $\Omega$, the response is very 
  large: this is the phenomenon of resonance. When the two frequencies are 
  exactly equal, this model predicts response which is infinite. That is 
  obviously non-physical: we will find out how to fix this problem shortly (see 
  sections 2.2.7 and 2.2.8). 