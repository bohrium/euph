  We will illustrate the idea of dimensional analysis by working through an 
  example. Back in section 3.2.1 we found that the vibration of a bending beam 
  is governed by the equation 

  \begin{equation*}m \dfrac{\partial^2 w}{\partial t^2}+EI \dfrac{\partial^4 
  w}{\partial x^4}=0 \tag{1}\end{equation*} 

  \noindent{}where $w(x,t)$ is the beam displacement at position $x$ and time 
  $t$, and $m$ is the mass per unit length. The term $EI$ is called the 
  ``bending rigidity'' or ``bending stiffness'', and is the product of the 
  Young's modulus $E$ with a quantity $I$ which is called the second moment of 
  area of the cross-section. For the particular case of a rectangular section 
  there is a simple formula in terms of the width $b$ and the thickness $h$: 

  \begin{equation*}I=\dfrac{1}{12} bh^3. \tag{2}\end{equation*} 

  For any particular problem, there will be boundary conditions at the ends of 
  the beam. These mean that the length $L$ of the beam enters the problem. So 
  in total we have just three physical parameters: $EI$, $m$ and $L$. 

  We are interested in how a particular resonance frequency $\Omega$ depends on 
  these three parameters. It must some function 

  \begin{equation*}\Omega = F(EI,m,L) . \tag{3}\end{equation*} 

  We will try to find an expression for this function as a product of powers of 
  the three parameters, because that is the only way that quantities with 
  different units, or dimensions, can be combined: so try 

  \begin{equation*}\Omega=A (EI)^\alpha m^\beta L^\gamma \tag{4}\end{equation*} 

  \noindent{}where $A$ is a dimensionless scale factor, and $\alpha$, $\beta$ 
  and $\gamma$ are the three powers we want to find. 

  Next, we need the dimensions of the three parameters. The length $L$ is 
  trivial: it is indeed a length. We will write this fact as $[L] = 
  \mathrm{L}$, where the square brackets denote ``the dimensions of...'', and 
  we write L, M and T for the dimensions length, mass and time. The mass per 
  unit length satisfies $[m] = \mathrm{M} \mathrm{L}^{-1}$. We have to work a 
  little harder to find the dimensions of $EI$. Young's modulus is measured in 
  Pa: it has the same dimensions as pressure, i.e. force per unit area. So $[E] 
  = \mathrm{M} \mathrm{L} \mathrm{T}^{-2} \times \mathrm{L}^{-2} = \mathrm{M} 
  \mathrm{L}^{-1} \mathrm{T}^{-2}$. The second moment of area satisfies $[I] = 
  \mathrm{L}^4$, as is obvious for equation (2). Combining these, $[EI] = 
  \mathrm{M} \mathrm{L}^3 \mathrm{T}^{-2}$. Finally, $\Omega$ is a frequency, 
  measured in radians per second, so $[\Omega] = \mathrm{T}^{-1}$. 

  Now we put all these into equation (4), and insist that the left- and 
  right-hand sides are dimensionally consistent. 

  For that, we require 

  \begin{equation*}\mathrm{T}^{-1} = (\mathrm{M} \mathrm{L}^3 
  \mathrm{T}^{-2})^\alpha (\mathrm{M} \mathrm{L}^{-1})^\beta \mathrm{L}^\gamma 
  \tag{5}\end{equation*} 

  Now we simply equate the powers of M, L and T on the two sides of the 
  equation. This gives the three equations 

  \begin{equation*}0=\alpha + \beta \tag{6}\end{equation*} 

  \begin{equation*}0 = 3 \alpha -- \beta + \gamma \tag{7}\end{equation*} 

  \noindent{}and 

  \begin{equation*}-1=-2 \alpha \tag{8}\end{equation*} 

  \noindent{}which can immediately be solved to give $\alpha = 1/2$, $\beta = 
  -1/2$ and $\gamma= -2$. So equation (4) now says 

  \begin{equation*}\Omega=A (EI)^{1/2} m^{-1/2} L^{-2} = \dfrac{A}{L^2} 
  \sqrt{\dfrac{EI}{m}} . \tag{9}\end{equation*} 

  For the particular case of a beam with a rectangular cross-section, we can 
  simplify further. If the density of the material is $\rho$, then $m=\rho b h$ 
  so that, using equation (2), 

  \begin{equation*}\Omega= \dfrac{Ah}{12 L^2} \sqrt{\dfrac{E}{\rho}} . 
  \tag{10}\end{equation*} 

  The constant $A$ will depend on the particular boundary conditions, and on 
  which modal frequency we are looking at. But in terms of the dependence on 
  the parameters with dimensions, equation (10) gives a completely explicit 
  description, valid for all modes and all boundary conditions. 

  It is easy to see how this argument would generalise to other problems. In 
  other areas of physics we might have to take account of extra dimensions, 
  such as temperature and electric charge, but for a mechanical problem in 
  which M, L and T are the only dimensions we need to take into account, we 
  will always get three equations analogous to equations (6), (7) and (8). If 
  the problem has no more than three parameters with dimensions, we will 
  usually get an explicit solution as we saw here. 

  But if there are more than three dimensioned parameters, there will never be 
  enough equations for the number of unknowns. We can extend the beam example 
  to give a simple example of what happens. Suppose that our beam of length $L$ 
  has an additional concentrated mass $M$ attached to it, at a position $x=a$. 
  The two parameters $M$ and $a$ must be added to the three we already had, and 
  of course they have dimensions $[M] = \mathrm{M}$ and $[a] = \mathrm{L}$. We 
  extend equation (4) to try 

  \begin{equation*}\Omega=A (EI)^\alpha m^\beta L^\gamma M^\delta a^\epsilon 
  \tag{11}\end{equation*} 

  \noindent{}with two extra constants $\delta$ and $\epsilon$. We substitute 
  the dimensions, enforce dimensional consistency, and obtain three equations 
  analogous to (6), (7) and (8) which read 

  \begin{equation*}0=\alpha + \beta + \delta \tag{12}\end{equation*} 

  \begin{equation*}0 = 3 \alpha -- \beta + \gamma + \epsilon 
  \tag{13}\end{equation*} 

  \noindent{}and 

  \begin{equation*}-1=-2 \alpha. \tag{14}\end{equation*} 

  We can solve these for $\alpha$, $\beta$ and $\gamma$, leaving $\delta$ and 
  $\epsilon$ undetermined: 

  \begin{equation*}\alpha = 1/2, \beta = -1/2 -\delta, \gamma= -2 -- \delta 
  -\epsilon . \tag{15}\end{equation*} 

  Equation (11) can then be written 

  \begin{equation*}\Omega= A \sqrt{\dfrac{EI}{m}} m^{-\delta} L^{-2 -\delta 
  -\epsilon} M^\delta a^\epsilon \end{equation*} 

  \begin{equation*}= \dfrac{A}{L^2} \sqrt{\dfrac{EI}{m}} \left(\dfrac{M}{mL} 
  \right)^\delta \left(\dfrac{a}{L} \right)^\epsilon . \tag{16}\end{equation*} 

  The result is the same expression we had before, with the addition of two 
  terms involving undetermined powers $\delta$ and $\epsilon$. Those terms only 
  make sense if what is inside the brackets in both cases is dimensionless --- 
  otherwise arbitrary powers would cause havoc with our dimensional 
  consistency. But in this simple example, it is obvious that both terms are 
  indeed dimensionless: $M/mL$ is the ratio of the added mass to the total mass 
  of the beam, and $a/L$ is the position of the extra mass as a fraction of the 
  beam length. 

  This example illustrates something known as the ``Buckingham Pi theorem''. If 
  you have $N$ dimensions and $P > N$ variables in the problem, you expect to 
  find $P-N$ dimensionless groups of parameters popping up, just as we have 
  seen here. There are exceptions, and the full version of the theorem has some 
  small print, but this simplified version is good enough for our needs here. 
  For a bit more detail, see \tt{}this Wikipedia page\rm{}. 

  In section 10.1, the example was given of scale models for wind-tunnel 
  testing. The design of such tests relies on controlling a particular 
  dimensionless parameter relevant to fluid flow, called the Reynolds Number. 
  We can give a brief description of the derivation and relevance of this 
  number. We will look at the case of flow of an incompressible fluid with 
  density $\rho$, viscosity $\mu$, pressure $p$. The velocity field of the flow 
  is written $\underline{u}$, and the gravitation acceleration is 
  $\underline{g}$. The flow is governed by the Navier-Stokes equation: 

  \begin{equation*}\rho \left(\dfrac{\partial \underline{u}}{\partial t} + 
  (\underline{u} \cdot \nabla) \underline{u} \right) = \mu \nabla^2 
  \underline{u} -- \nabla p + \rho \underline{g} . \tag{17}\end{equation*} 

  The Reynolds Number gives an order-of-magnitude estimate of the ratio of two 
  of the terms in this equation: the inertia term $\rho (\underline{u} \cdot 
  \nabla) \underline{u}$ and the viscosity term $\mu \nabla^2 \underline{u}$. 
  If the flow has a typical speed $U$ and a typical length scale $L$, then 

  \begin{equation*}\rho (\underline{u} \cdot \nabla) \underline{u} \sim \rho 
  U^2 / L \tag{18}\end{equation*} 

  \noindent{}and 

  \begin{equation*}\mu \nabla^2 \underline{u} \sim \mu U / L^2. 
  \tag{19}\end{equation*} 

  The ratio is then 

  \begin{equation*}R = \dfrac{\rho U L}{\mu}, \tag{20}\end{equation*} 

  \noindent{}which is the Reynolds Number. When $R$ is very large, inertia 
  effects dominate over viscous effects, and viscosity can be ignored. On the 
  other hand when $R$ is small, viscosity dominates and inertia can be 
  neglected. These two limits lead to flows with very different character. 

  To get an impression of the difference, think about swimming in water. When 
  humans swim, the Reynolds number is large, but for a swimming micro-organism 
  the length scale is much smaller and the Reynolds Number becomes very small. 
  Humans are aware of inertia effects in the water as they swim, but they are 
  hardly aware of viscous drag. If you tried to swim in treacle, you would 
  enter the regime of a micro-organism because the viscosity is much higher so 
  the Reynolds Number becomes small. In the treacle, the moment you stopped 
  moving your arms and legs you would stop moving: in order to ``coast'' 
  through the water you rely on inertia. 