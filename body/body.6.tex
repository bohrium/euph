  This chapter is the third in the ``Underpinnings'' sequence. This time, we 
  look at how we hear sounds: a bit about the mechanics of hearing, and an 
  introduction to the science of psychoacoustics which provides a way to obtain 
  quantitative information about human perceptions and judgements. 

  Section 6.2 describes, briefly, how our ears work. Your inner ear contains a 
  structure, called the basilar membrane, which acts as a kind of mechanical 
  frequency analyser: the frequency components of incoming sounds are spread 
  out along the membrane. This pattern is then encoded into the pattern of 
  nerve fibres activated by the sound, because these fibres are connected to 
  sensors at different places along the membrane. 

  Section 6.3 describes what is probably the most familiar psychoacoustical 
  finding: a diagram showing how our perception of loudness of sine waves is 
  influenced by their frequency and amplitude. This example gives an 
  opportunity to introduce the kind of experiments that have to be carried out 
  in order to address this kind of perceptual question. 

  Section 6.4 examines other basic elements of sound perception, including some 
  detail on the filtering action of the basilar membrane. Aspects of the 
  mechanical behaviour of the basilar membrane have a rather direct influence 
  on how we perceive sound: some examples are given. 

  In section 6.5, we look at one type of musical question that can be addressed 
  using psychoacoustical techniques. This is based on the idea of a threshold 
  of perception, the smallest change in some attribute of sound such you are 
  capable of ``hearing the difference''. Experiments have been carried out to 
  determine the threshold of perception for some things we are directly 
  interested in: for example, shifting the body resonance frequencies in the 
  body of a violin or a guitar. The results are of direct interest to 
  instrument makers, by determining how big a change they need to make to the 
  structure of an instrument before a player will know that something is 
  different. 

  Finally, in section 6.6 we look at a very challenging type of problem: how do 
  we quantify human judgements of quality and preference between instruments? 
  Two examples are discussed, for which careful experiments have been carried 
  out. The first concerns the choice of constructional material for a guitar 
  body: do guitar makers really need to use traditional tropical hardwoods, or 
  are more sustainable alternatives just as good? The second question addresses 
  perhaps the most famous (or notorious) question in musical acoustics: is 
  there really something special about certain old Italian violins, by makers 
  such as Antonio Stradivari, or are contemporary violin makers capable of 
  producing instruments that are just as good (or indeed better)? 

