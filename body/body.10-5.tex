

  One of the uses for measured frequency response functions is to obtain 
  measured versions of mode shapes — a procedure known as experimental modal 
  analysis. We have seen some examples earlier, but now it is time to find out 
  how the method works. It relies on a theoretical result that we discussed way 
  back in section 2.2, with mathematical details in section 2.2.5. 

  The result can be described, approximately at least, in words. Any frequency 
  response function of a vibrating structure can be written entirely in terms 
  of modal parameters. Each mode, on its own, behaves exactly like a simple 
  damped mass-spring oscillator. The total frequency response is given by 
  adding together these modal contributions. The frequency and the damping 
  factor of each mode is always the same, whichever particular frequency 
  response is calculated. But the amplitude of each modal term in the mixture 
  varies, depending on the positions of the driving point and the observation 
  point. Specifically, the amplitude involves a product: the mode shape 
  evaluated at the driving point, multiplied by the same mode shape evaluated 
  at the observation point. 

  We should note that everything to be said in this section relies on this 
  theoretical formula. That means that the process cannot be applied to 
  measurements of radiated sound using a microphone, because there simply isn't 
  a corresponding general formula for that case. So we need to concentrate on 
  mechanical measurements, for example using accelerometers. 

  Before we introduce the full procedure for finding mode shapes, it is useful 
  to be reminded of how the mass-spring oscillator works. Figure 1 shows the 
  motion of such an oscillator, set off by a hammer tap. The top curve, in red, 
  shows the behaviour with no damping: a sinusoidal vibration at the natural 
  frequency of the oscillator continues for ever, because the energy put in by 
  the hammer tap is never lost. But as soon as we have some damping, the motion 
  decays away. 

  \fig{figs/fig-7317127c.png}{\caption{Figure 1. Free vibration response of a 
  simple mass-spring oscillator. Red: undamped; black: damped with Q=100; blue: 
  Q=33; green: Q=10.}} 

  The decay rate depends on the level of damping, which we can characterise by 
  the damping factor, or its inverse the Q-factor. The remaining curves in 
  Fig.\ 1 show the behaviour with three different levels of damping, with 
  progressively decreasing Q-factors: 100 for the black curve, 33 for the blue 
  one, and 10 for the green one. The meaning of the Q-factor is particularly 
  easy to visualise here: it tells you the number of cycles needed for the 
  vibration amplitude to reduce by a certain factor (specifically, the factor 
  $e^\pi \approx 23$). 

  Converting this behaviour into a frequency response, we obtain Fig.\ 2. 
  Because the measurements we are about to look at were made with an 
  accelerometer, the specific frequency response plotted here is the 
  accelerance: acceleration per unit force, as a function of frequency. The 
  four curves have colours corresponding to Fig.\ 1. The red curve, for the 
  undamped system, goes off the top of the plot and continues (in theory) to 
  infinity. But all the others have peaks with a finite height. As the damping 
  increases (i.e. as the Q-factor goes down), the peak height reduces. 

  \fig{figs/fig-f236e974.png}{\caption{Figure 2. Accelerance for the four 
  oscillators from Fig. 1, with the same colour code. The red curve runs off 
  the top of the plot to infinity in the absence of any damping.}} 

  The “sharpness” of the peak also reduces. The usual way to characterise this 
  is via the “half-power bandwidth”. You find the points on either side of the 
  peak where the amplitude has gone down by a factor $\sqrt{2}$, in other words 
  where the squared amplitude, proportional to the energy in the oscillation, 
  has gone down by a factor of 2. On the decibel scale of these plots, this 
  corresponds to a reduction of level by 3~dB. Now the half-power bandwidth is 
  the difference of these two frequencies. If you divide it by the peak 
  frequency, that gives you the loss factor, which is a dimensionless number. 
  The inverse gives you the Q-factor, so the Q-factor can be visualised as the 
  number of half-power bandwidths you can fit along the frequency axis between 
  zero and the peak. 

  Now we can turn to some measurements. But we will not use something as 
  complicated as a musical instrument body for this first demonstration. 
  Instead, we will analyse measurements on the steel ruler shown in Fig.\ 3. 
  The ruler was supported on soft rubber bands, and was tapped with the 
  miniature impulse hammer at 10 points along the mid-line, equally spaced 
  between the two extreme ends. Response was recorded by a small accelerometer 
  stuck underneath, at the right-hand end (you can see the red cable leading 
  from it). 

  \fig{figs/fig-efc68e6d.png}{\caption{Figure 3. The metal ruler used for the 
  experimental data shown in this section. It was tapped with the miniature 
  impulse hammer shown. The red cap on the hammer tip is to soften and prolong 
  the impact, to concentrate energy into lower frequencies. Vibration was 
  observed with a small accelerometer fixed under the right-hand end.}} 

  Figure 4 shows a sample of the results. The three curves correspond to 
  tapping points numbers 1, 6 and 9, counting from the left-hand end. The 
  plotted frequency range covers the first four bending resonances of the 
  ruler, which show up as the obvious strong peaks. It is the mode shapes of 
  these bending resonances that we will try to visualise from the set of 10 
  measured frequency responses. 

  \fig{figs/fig-8b5fe126.png}{\caption{Figure 4. Measured accelerance of the 
  steel ruler, for tapping positions 1, 6 and 9 counting from the left in Fig. 
  3. The amplitude of each accelerance $Y$ is plotted here, on a decibel 
  scale.}} 

  As an aside, you may be wondering about the smaller peaks that can be seen in 
  this plot. They have two sources. Very sharp features at multiples of 50~Hz 
  are caused by electrical interference (50~Hz is the frequency of the 
  electricity supply in the UK). The broader feature around 260~Hz, 
  particularly clear in the green curve, is a torsional resonance of the ruler. 
  If I had placed my accelerometer and my hammer taps exactly on the centre 
  line of the ruler, this resonance would not have shown up because any 
  torsional mode has a nodal line down the centre. But in practice neither 
  placement will have been exact, so the torsional resonance shows through at a 
  low level. 

  Returning to the mode shapes underlying the main peaks in Fig.\ 4, it is now 
  easy to explain roughly how the method works. I will first do it by 
  “hand-waving”, then fill in the more technical details of the method. These 
  peaks are well separated from each other, so that for frequencies close to a 
  peak, the response will be dominated by a single modal contribution. That 
  should produce a frequency response very similar to Fig.\ 2. We can thus 
  deduce the modal frequency by looking at the peak frequency, and the modal 
  damping by finding the half-power bandwidth. We should get the same answer 
  from all 10 measurements, except that for any particular mode, certain 
  tapping positions will fall near nodal points and so the peak will not appear 
  very strongly. For example, the second peak hardly appears in the green curve 
  in Fig.\ 4. 

  Now we look at the peak height, and see how it varies across the 10 different 
  tapping positions. Remember what the formula tells us: the height should be 
  proportional to the mode shape evaluated at the tapping position. It is also 
  proportional to the mode shape evaluated at the measurement position, but 
  that stays fixed throughout the experiment so it does not influence anything 
  provided we have fixed our accelerometer in a position that is not near a 
  nodal line for any mode we are interested in. 

  So the variation of peak height should map out the mode shape. But there is a 
  snag if we only look at plots like Fig.\ 4, showing the magnitude of the 
  response. We would have no way to distinguish positive from negative values 
  of the modal amplitude, so we would only see the shape in a “rectified” form. 
  The measured frequency response functions do contain the information we want, 
  but we have to look at them a little harder. 

  The next part of the description will involve the mathematical idea of 
  “complex numbers”. Don’t worry if that is mysterious to you: just skip 
  lightly over the details until you get to Fig.\ 8, which is the payoff for 
  introducing the concept here. That figure, when we reach it, gives a very 
  immediate visualisation of the thing we are trying to do, and that is all you 
  really need to understand. But before we get there, I want to show something 
  else to give an inkling of how the technical computer processing might work. 

  The measured frequency responses contain information about phase as well as 
  amplitude, and that is what we need to resolve the issue of what is positive 
  and what is negative. The way the information is actually coded in the 
  computer is in terms of complex numbers, which are the sum of a “real part” 
  and an “imaginary part”. Figure 5 shows the same data as Fig.\ 4, but 
  plotting the real and imaginary parts separately. 

  \fig{figs/fig-7304dbaf.png}{} 

  \fig{figs/fig-9731581b.png}{} 

  Now we concentrate on a narrow frequency range near one of the strong peaks, 
  and instead of plotting the two parts against frequency, we plot the 
  imaginary part against the real part. Something unexpected and somewhat 
  magical then happens: an example, for a range of frequencies near the 4th 
  resonance, is shown in Fig.\ 6. It is immediately obvious that the points dot 
  out perfect circles. The three channels of data are indicated by the same 
  colour code as before, and the measured data points are shown as stars. The 
  lines connecting these stars are just a guide to the eye: we only have data 
  at particular frequencies governed by the resolution of our measurement — in 
  this case, that resolution is 0.5~Hz. 

  \fig{figs/fig-5cd9eaa4.png}{\caption{Figure 6. The data from Fig. 5 plotted 
  as real part versus imaginary part, for the frequency range 343--364~Hz, 
  around the fourth resonance of the ruler. The colours are the same as in 
  previous figures. The measured data points are shown as stars, connected by 
  lines purely as a guide to the eye. Dashed lines show best-fitted circles in 
  each case.}} 

  The dashed line shows the computer’s “best-fitted” circle in each case. The 
  reason behind the appearance of these circles is explained in the next link, 
  along with some other mathematical details. But notice where the circles are 
  located in the plot. The origin (0,0) does not lie in the centre of any of 
  the circles, as you might have guessed. Instead, each circle nearly passes 
  through the origin. In fact, if we had plotted the corresponding diagram for 
  the mass-spring oscillator resonance from Fig.\ 2, we would have obtained a 
  circle passing almost exactly through the origin. The reason the measured 
  circles don’t quite do this is because of the small but non-zero influence of 
  the other modes, at higher and lower frequencies. The link fills in some 
  detail on all this. 

  Now, the amplitude that we plotted in Fig.\ 4 is given, at each frequency, by 
  the distance of the corresponding star from the origin. So to identify the 
  peak of amplitude, we need to look for the star which is furthest from the 
  origin. This occurs more or less at the top of the red and green circles, and 
  at the bottom of the blue circle. The fact that the blue circle goes 
  downwards when the other two go upwards is precisely the distinction we are 
  looking for: the modal amplitude for this mode is positive for the red and 
  green data, but negative for the blue data. 

  I deliberately chose to show data for the 4th mode for this initial 
  discussion. But we can see why these theoretical circles might be important 
  when we plot the corresponding diagram for the lowest mode, in Fig.\ 7. The 
  stars are far more sparse here, because the corresponding resonance peak is 
  narrower so that the 0.5~Hz frequency resolution has a bigger impact. But the 
  computer has still been able to fit circles, because there is in fact enough 
  information in these sparse measured points. As it has turned out, none of 
  the circles show a star near their respective tops or bottoms: the actual 
  peak frequency seems to be about half-way between two stars. We can use these 
  fitted circles to predict the true height, and hence the true modal amplitude 
  factor: it is given by the diameter of the corresponding circle. 

  \fig{figs/fig-aa186ace.png}{\caption{Figure 7. A plot corresponding to Fig. 
  6, for a frequency range 35.4~Hz--44~Hz around the lowest mode of the 
  ruler.}} 

  Modal analysis software takes advantage of things like circle-fitting, and 
  more sophisticated processing to take account of the influence of other 
  modes. But we can close this preliminary account with a graphic which gives a 
  clear impression of how things work without needing very much clever 
  processing. What we have seen in Figs.\ 6 and 7 is that the important 
  features of each peak are contained in the variation on the vertical axis, in 
  other words in the imaginary part of each frequency response function. 

  Figure 8 shows a plot of that imaginary part, now for all 10 tapping 
  positions. They are laid out in a three-dimensional manner, for a frequency 
  range that covers the first three bending resonances of the ruler. The peak 
  corresponding to the lowest resonance is marked by a red dot on each curve, 
  and these dots are connected by dashed lines as a guide to the eye. Green 
  dots and lines mark the second mode, and yellow ones mark the third mode. (A 
  detail: in order to generate this plot while avoiding the issue highlighted 
  by Fig.\ 7, the frequency resolution was made finer by a factor of 4 using 
  FFT-based interpolation with zero-padding, mentioned in section 10.4.2.) 

  \fig{figs/fig-939baccf.png}{\caption{Figure 8. 3D plot of the imaginary parts 
  of the 10 measured frequency responses (blue curves), spread out in their 
  spatial order. Red circles mark the peaks corresponding to the lowest 
  resonance, green ones mark the second resonance, and yellow ones mark the 
  third resonance. These three colours map out the corresponding mode shapes. 
  Thanks to George Stoppani for the idea of this graphic.}} 

  The red, green and yellow points correspond rather convincingly to the mode 
  shapes we are expecting for our ruler, which is a free-free bending beam as 
  discussed back in section 3.2.1. To remind you, animations of those three 
  modes are shown in Fig.\ 9(a)—(c). The measured shapes are not quite 
  symmetrical, unlike the theoretical ones. This is not an error: it shows the 
  influence of the mass of the accelerometer attached at one end of the beam. 

\moobeginvid\begin{tabular}{ccc} \vidframe{ 0.30 }{ vids/vid-c18a3404-00.png }&\vidframe{ 0.30 }{ vids/vid-c18a3404-01.png }&\vidframe{ 0.30 }{ vids/vid-c18a3404-02.png } \end{tabular}\caption{Figure 9(a). The lowest mode of an ideal free-free beam.}\mooendvideo

\moobeginvid\begin{tabular}{ccc} \vidframe{ 0.30 }{ vids/vid-795942a7-00.png }&\vidframe{ 0.30 }{ vids/vid-795942a7-01.png }&\vidframe{ 0.30 }{ vids/vid-795942a7-02.png } \end{tabular}\caption{Figure 9(b). The second mode of a free-free beam.}\mooendvideo

\moobeginvid\begin{tabular}{ccc} \vidframe{ 0.30 }{ vids/vid-cb62f69a-00.png }&\vidframe{ 0.30 }{ vids/vid-cb62f69a-01.png }&\vidframe{ 0.30 }{ vids/vid-cb62f69a-02.png } \end{tabular}\caption{Figure 9(c). The third mode of a free-free beam.}\mooendvideo

  Having seen how experimental modal analysis works through that simple 
  example, we can look at some results for instrument bodies. Yet again, I will 
  show “signature modes” of a violin body. Modal measurements by two different 
  people will be shown: George Stoppani and George Bissinger. You can see 
  George Stoppani in Fig.\ 10, doing a modal scan on a cello assisted by Ailin 
  Zhang. The cello is suspended on rubber bands at the neck and the endpin, and 
  the strings are damped (including the afterlength strings between the bridge 
  and the tailpiece). He has a single accelerometer fixed to the cello, and he 
  is tapping with his impulse hammer over a grid of points covering the body. 

  \fig{figs/fig-abdcd53a.png}{\caption{Figure 10. George Stoppani doing a modal 
  test on a cello.}} 

  George Bissinger’s measurements are made by the reciprocal method: his hammer 
  always taps on the bridge, and he uses a scanning laser vibrometer to measure 
  the response at many points in turn. For both of them, the resolution of the 
  modal images is determined by how many measurement points they are prepared 
  to use. To get good images you have to be prepared to patient, and test a lot 
  of points! George Bissinger’s set of measurement points can be seen in Fig.\ 
  11, for a violin. 

\moobeginvid\begin{tabular}{ccc} \vidframe{ 0.30 }{ vids/vid-ec22bd2a-00.png }&\vidframe{ 0.30 }{ vids/vid-ec22bd2a-01.png }&\vidframe{ 0.30 }{ vids/vid-ec22bd2a-02.png } \end{tabular}\caption{Figure 11. The grid of measurement points used by George Bissinger for his modal tests on violins.}\mooendvideo

  I will first show three different representations of the signature mode A0, 
  the “air resonance”. Figure 12 shows a still image of the motion of the top 
  and back plates of the violin in this mode, as measured by George Stoppani. 
  Figure 13 shows a 3D animated view of the same mode, also measured by him. 
  Figure 14 shows a George Bissinger measurement of the same mode (on a 
  different violin). This video is reproduced from the \tt{}web site of the 
  Strad3D project\rm{}. 

  \fig{figs/fig-74e5f539.png}{\caption{Figure 12. The signature mode A0 of a 
  violin body, measured by George Stoppani. The top and the back are both 
  viewed from outside the instrument. Blue colours denote inward movement 
  relative to the body cavity, orange colours denote outward movement. Image 
  copyright George Stoppani, reproduced by permission.}} 

\moobeginvid\begin{tabular}{ccc} \vidframe{ 0.30 }{ vids/vid-397c209d-00.png }&\vidframe{ 0.30 }{ vids/vid-397c209d-01.png }&\vidframe{ 0.30 }{ vids/vid-397c209d-02.png } \end{tabular}\caption{Figure 13. An animation of the mode A0 of a violin, measured by George Stoppani. Image copyright George Stoppani, reproduced by permission.}\mooendvideo

\moobeginvid\begin{tabular}{ccc} \vidframe{ 0.30 }{ vids/vid-d6f7dff8-00.png }&\vidframe{ 0.30 }{ vids/vid-d6f7dff8-01.png }&\vidframe{ 0.30 }{ vids/vid-d6f7dff8-02.png } \end{tabular}\caption{Figure 14. The mode A0 of a violin as measured by George Bissinger. This plot contains an extra feature: the yellow shapes show the ``air pistons'' associated with flow through the f-holes. These were measured by a different technique, to be explained in section 10.6: ``near-field acoustic holography''. Movie reproduced from Strad3D, by permission of Sam Zygmuntowicz and George Bissinger.}\mooendvideo

  We can learn a number of things by comparing these images. The first thing 
  that strikes you, probably, is the different style of graphics. The Bissinger 
  plot is generated by general-purpose commercial software, in the form of a 
  “wire frame” representation. It shows the actual grid of measurements points, 
  simply connected by straight lines. It has some virtues: you see the complete 
  violin rather than the separated plates, and you can see through the top 
  plate to the other side. On the downside it looks a bit crude, and perhaps it 
  is not completely clear what is going on in the motion depicted. 

  The Stoppani images are quite different. George Stoppani is a violin maker, 
  and makers care a lot about design details like the plate outline, corner 
  shapes and f-holes. So when he wrote his rather impressive software package 
  for modal analysis, he incorporated options to design realistic violin 
  shapes. He also uses a more sophisticated representation of the vibrating 
  shape, using smooth interpolation between the measurement points. This has 
  the virtue of making the plots look nicer, with the possible associated 
  disadvantage that you lose sight of the resolution of the underlying 
  measurements — in fact his measurement grid has a similar resolution to the 
  Bissinger grid seen in Fig.\ 11. 

  The next obvious difference is that the Bissinger plot shows the complete 
  violin, with its neck, fingerboard, tailpiece and so on. The Stoppani plot 
  just shows the top and back plates of the body. This is simply a choice made 
  for display purposes — both styles have advantages. There is something else 
  that is not a matter of choice, though: the Bissinger plot is fully 
  three-dimensional, whereas the Stoppani plot shows only motion normal to the 
  plane of the body. For this particular set of measurements, George Bissinger 
  had access to an advanced (and very expensive) 3D laser vibrometer system. 

  The Bissinger image of the complete instrument reveals that this mode 
  involves significant vibration of the neck and, especially, the projecting 
  length of the fingerboard. The extent to which this happens depends on 
  exactly how the instrument maker has shaped the fingerboard: the strongest 
  coupling with the “air mode” A0 occurs when the resonance frequency of the 
  fingerboard cantilever is tuned to roughly the same frequency. This coupling 
  is liked by some players, probably because the instrument feels more “alive” 
  to them when they feel vibration of the neck through their left hand [1]. 

  The final feature that distinguishes the Stoppani and Bissinger plots is a 
  slight digression from the perspective of this section, but it is significant 
  in terms of the underlying physics. The Bissinger plot shows two yellow 
  f-hole shapes, that “float” up and down during the vibration. These are not 
  part of the modal analysis, but an extra feature added to the graphic from a 
  different measurement on the same violin. They show a visualisation of the 
  “Helmholtz air pistons”, breathing in and out through the f-holes, deduced by 
  a method called “near-field acoustic holography” [2]: we will say a bit more 
  about this approach in section 10.6. 

  The mode A0 is a modified Helmholtz resonance, and so we are not surprised to 
  see very vigorous motion of these invisible pistons, in phase for the two 
  f-holes. From the discussion in section 4.2, we expect that the volume change 
  associated with the plate motion will be in the opposite phase to volume 
  change associated with the f-hole flows. The sound radiation is dominated by 
  the f-holes, and somewhat reduced by the associated plate motion. 

  However, the plate motion is crucial: A0 can only be excited during normal 
  violin playing via string forces at the bridge, and for that to work there 
  has to be bridge motion associated with the mode. That bridge motion is 
  perhaps made most clear by the visualisation in Fig.\ 13: there is obvious 
  rocking motion in the “island” area between the f-holes, and this will carry 
  the bridge along with it. The main reason for this strong rocking motion is 
  the effect of the soundpost. You can see in Figs.\ 12 and 13 that A0 has a 
  node line (showing white in the plots) passing very near the soundpost 
  position. 

  Figures 15, 16 and 17 show the same three visualisations of the violin 
  signature mode CBR. In this mode, the violin body behaves a bit like a very 
  thick flat plate: the top and the back move in more or less the same shape, 
  as is particularly clear in Fig.\ 15. The nodal lines of this mode shape give 
  no clue about the soundpost position, because the post is carried up and down 
  by the top and back moving in synchrony. 

  Figure 17, showing the motion in 3D, reveals that there is significant 
  rotation/shearing of the ribs in the C-bouts of the violin. This is where the 
  mode gets its acronym: CBR stands for “C-bouts rhomboidal”. That figure also 
  shows that the two f-hole “air pistons” are moving in opposite directions. 
  Every aspect of this mode shape is rather symmetrical about the long axis of 
  the body, so it has virtually no net volume change, and therefore it is not a 
  good radiator of sound. 

  \fig{figs/fig-24b12361.png}{\caption{Figure 15. The signature mode CBR of a 
  violin body, measured by George Stoppani, in the same format as Fig. 12. 
  Image copyright George Stoppani, reproduced by permission.}} 

\moobeginvid\begin{tabular}{ccc} \vidframe{ 0.30 }{ vids/vid-bc8e51f3-00.png }&\vidframe{ 0.30 }{ vids/vid-bc8e51f3-01.png }&\vidframe{ 0.30 }{ vids/vid-bc8e51f3-02.png } \end{tabular}\caption{Figure 16. The signature mode CBR of a violin body, measured by George Stoppani, in the same format as Fig. 13.  Image copyright George Stoppani, reproduced by permission.}\mooendvideo

\moobeginvid\begin{tabular}{ccc} \vidframe{ 0.30 }{ vids/vid-70fd7c5a-00.png }&\vidframe{ 0.30 }{ vids/vid-70fd7c5a-01.png }&\vidframe{ 0.30 }{ vids/vid-70fd7c5a-02.png } \end{tabular}\caption{Figure 17. The mode CBR of a violin as measured by George Bissinger, in the same format as Fig. 14. Movie reproduced from Strad3D, by permission of Sam Zygmuntowicz and George Bissinger.}\mooendvideo

  Figures 18, 19 and 20 show the same three visualisations of the mode B1-, and 
  Figures 21, 22 and 23 show the related mode B1+. These two modes are 
  sometimes called “baseball modes”: Figs.\ 18 and 21 reveal that in both cases 
  there is a single sinuous node line that snakes around the body in a pattern 
  like the seam of a baseball. The white nodal line travels off the edges of 
  the top plate and reappears at more or less the same position on the back 
  plate, then travels across that and skips back to the top plate, and so on. 
  In B1-, the node line is aligned roughly lengthways on the top, and crossways 
  on the back; in B1+ the pattern is reversed. 

  Both modes are rendered significantly asymmetric by the effect of the 
  soundpost (and to a lesser extent the bass bar). In consequence, both have 
  significant volume change. This leads to symmetrical motion of the two f-hole 
  “pistons”, as is clear in Figs.\ 20 and 23. Both modes are efficient 
  radiators of sound, and they play a very important role in the low-frequency 
  sound of a violin. Both also involve significant rocking motion of the 
  bridge, which allows them to be driven effectively by forces from the bowed 
  string. Indeed, especially in the case of B1+, the bridge motion can be so 
  strong that a wolf note is caused (see section 9.4). 

  \fig{figs/fig-f68aa741.png}{\caption{Figure 18. The signature mode B1- of a 
  violin body, measured by George Stoppani, in the same format as Fig. 12. 
  Image copyright George Stoppani, reproduced by permission.}} 

\moobeginvid\begin{tabular}{ccc} \vidframe{ 0.30 }{ vids/vid-e5ffb37c-00.png }&\vidframe{ 0.30 }{ vids/vid-e5ffb37c-01.png }&\vidframe{ 0.30 }{ vids/vid-e5ffb37c-02.png } \end{tabular}\caption{Figure 19. The signature mode B1- of a violin body, measured by George Stoppani, in the same format as Fig. 13.  Image copyright George Stoppani, reproduced by permission.}\mooendvideo

\moobeginvid\begin{tabular}{ccc} \vidframe{ 0.30 }{ vids/vid-04bdc53f-00.png }&\vidframe{ 0.30 }{ vids/vid-04bdc53f-01.png }&\vidframe{ 0.30 }{ vids/vid-04bdc53f-02.png } \end{tabular}\caption{Figure 20. The mode B1- of a violin as measured by George Bissinger, in the same format as Fig. 14. Movie reproduced from Strad3D, by permission of Sam Zygmuntowicz and George Bissinger.}\mooendvideo

  \fig{figs/fig-b41602e8.png}{\caption{Figure 21. The signature mode B1+ of a 
  violin body, measured by George Stoppani, in the same format as Fig. 12. 
  Image copyright George Stoppani, reproduced by permission.}} 

\moobeginvid\begin{tabular}{ccc} \vidframe{ 0.30 }{ vids/vid-c4c513bb-00.png }&\vidframe{ 0.30 }{ vids/vid-c4c513bb-01.png }&\vidframe{ 0.30 }{ vids/vid-c4c513bb-02.png } \end{tabular}\caption{Figure 22. The signature mode B1+ of a violin body, measured by George Stoppani, in the same format as Fig. 13.  Image copyright George Stoppani, reproduced by permission.}\mooendvideo

\moobeginvid\begin{tabular}{ccc} \vidframe{ 0.30 }{ vids/vid-44aa8fba-00.png }&\vidframe{ 0.30 }{ vids/vid-44aa8fba-01.png }&\vidframe{ 0.30 }{ vids/vid-44aa8fba-02.png } \end{tabular}\caption{Figure 23. The mode B1+ of a violin as measured by George Bissinger, in the same format as Fig. 14. Movie reproduced from Strad3D, by permission of Sam Zygmuntowicz and George Bissinger.}\mooendvideo



  \sectionreferences{}[1] J. Woodhouse, “The acoustics of ‘A0-B0 mode matching’ 
  in the violin”, Acta Acustica united with Acustica \textbf{84}, 947–946 
  (1998) 

  [2] George Bissinger, Earl G. Williams and Nicolas Valdivia, “Violin f-hole 
  contribution to far-field radiation via patch near-field acoustical 
  holography”, Journal of the Acoustical Society of America \textbf{121}, 
  3899—3906 (2007) 