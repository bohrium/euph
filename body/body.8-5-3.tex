  The starting point for the time-domain simulation approach, as applied to the 
  simple clarinet model, is to assume that just inside the mouthpiece there is 
  at least a short length of uniform tube. We can then use a general expression 
  for acoustic pressure of plane waves in a parallel duct: 

  \begin{equation*}p(x,t) = p_R(t-x/c) + p_L(t+x/c) \tag{1}\end{equation*} 

  \noindent{}where $c$ is the speed of sound, and $p_L$ and $p_R$ denote 
  left-travelling and right-travelling waves, respectively. We derived this 
  expression in section 5.4.2 in the context of waves on an ideal string, but 
  we also showed in section 4.1.1 that plane acoustic waves obey the same 
  one-dimensional wave equation as waves on a string. 

  Using results from section 4.1.1, we can also write down the corresponding 
  contributions to the acoustic volume flow rate: these are 

  \begin{equation*}Z_t^{-1}p_R(t-x/c) \mathrm{~~~and~~~} -Z_t^{-1}p_L(t+x/c) 
  \tag{2}\end{equation*} 

  \noindent{}where $Z_t$ is a constant called the wave impedance (or 
  characteristic impedance) of the tube. It is equal to $c$ times the density 
  of air divided by the cross-sectional area of the tube. 

  The reed entrance is at $x=0$, so at that point we have 

  \begin{equation*}p(t)=p_R(t) + p_L(t) \tag{3}\end{equation*} 

  \noindent{}and 

  \begin{equation*}Z_t v(t)=p_L(t) -- p_R(t) . \tag{4}\end{equation*} 

  Now we introduce the linear acoustical behaviour of the clarinet tube. 
  Suppose a pressure pulse in the form of a Dirac delta function is sent out 
  from $x=0$ at time $t=0$. The returning pulse will be delayed, inverted and 
  somewhat rounded, so that it looks something like Fig.\ 1. Define this 
  ``reflection function'' to be $r(t)$. In order that there should be no 
  permanent, steady difference of mean pressure between the inside and the 
  outside of the tube, linear acoustics requires this function to satisfy 

  \begin{equation*}\int_0^\infty{r(t) dt} =-1 . \tag{5}\end{equation*} 

  \fig{figs/fig-e01b3e01.png}{\caption{Figure 1. A simple reflection function 
  $r(t)$}} 

  Next we make use of a key result shown back in section 2.2.8. We can use 
  $r(t)$ to calculate the reflected pressure wave caused by any outgoing 
  pressure wave by the process of convolution. The result is that 

  \begin{equation*}p_L(t)=\int_0^\infty{r(\tau) p_R(t-\tau) d \tau} 
  \tag{6}\end{equation*} 

  \noindent{}and for brevity we can denote this convolution expression by 
  $p_L=r*p_R$. 

  Taking the sum and difference of eqs. (3) and (4) gives 

  \begin{equation*}p+Z_t v=2 p_L=2r*p_R \tag{7}\end{equation*} 

  \noindent{}and 

  \begin{equation*}p-Z_t v=2p_R . \tag{8}\end{equation*} 

  These can be combined into the form 

  \begin{equation*}p=p_h + Z_t v \tag{9}\end{equation*} 

  \noindent{}where 

  \begin{equation*}p_h=r*(p+Z_t v). \tag{10}\end{equation*} 

  The subscript `h' here is to suggest ``history'': $p_h$ is something we can 
  easily calculate, by the convolution integral, from a knowledge of the past 
  history of the system. So at a given time step $t$, $p_h$ is a known 
  constant. Equation (9) is then simply the equation of a straight line with 
  slope $1/Z_t$, crossing the $p$-axis at position $p_h$. But we know that $p$ 
  and $v$ are also connected via the nonlinear function plotted in Fig.\ 2. So 
  the new values of $p$ and $v$ are found at the intersection between the 
  straight line and the curve: three examples of possible positions of the 
  straight line are indicated in the plot. (A detail: the curve is plotted here 
  as a function of the pressure difference, not of the absolute pressure $p$. 
  This means that each straight line in this plot needs to cross the $p$ axis 
  at value $p_h-p_m$ where $p_m$ is the pressure inside the player's mouth.) 

  \fig{figs/fig-2aeaf43e.png}{\caption{Figure 2. The nonlinear characteristic 
  of the mouthpiece (black curve), and three examples of the straight line from 
  eq. (9) with different values of $p\_h$.}} 

  For the particular curve plotted, and the chosen slope of the straight line, 
  there is always just a single intersection so that the computation is very 
  simple. But if the slope of the line had been lower, so that it was less than 
  the maximum positive slope of the curve, multiple intersections could occur 
  for certain values of $p_h$. We will return to that issue in section 9.2, in 
  the context of bowed strings, where such multiple intersections have an 
  interesting physical consequence. 

  The specific parameter values used for the numerical examples of this simple 
  clarinet model in section 8.5 are as follows. They are based on choices made 
  in the original paper describing this method [1]. Units are used in which the 
  characteristic impedance $Z_t=1$. The smooth portion of the nonlinear curve 
  then has the cubic form 

  \begin{equation*}v=0.03(p_0 -p)(p-p_0+5)(p-p_0+10) . \tag{11}\end{equation*} 

  The maximum positive slope of this curve occurs at the point $p-p_0=-5$, and 
  the value of the slope there is 0.75, so it is indeed lower than $1/Z_t$. 

  \sectionreferences{}[1] Michael E. McIntyre, Robert T. Schumacher and Jim 
  Woodhouse; ``On the oscillations of musical instruments'', Journal of the 
  Acoustical Society of America \textbf{74}, 1325--1345 (1983) 