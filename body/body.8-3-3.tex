  Back in section 8.2.2 we met Duffing's equation. For undamped, free motion 
  the equation can be written: 

  $$\ddot{x} +\alpha x +\mu x^3 = 0 \tag{1}$$ 

  where $\alpha$ represents the stiffness of the linear spring and $\mu$ is the 
  nonlinear coefficient. If $\mu > 0$ it is a hardening spring, if $\mu < 0$ it 
  is a softening spring. To put this into a suitable form for looking at the 
  phase space representation, we write it as the pair of equations 

  $$\dot{x}=y \mathrm{,~~~~}\dot{y}=-\alpha x -\mu x^3 . \tag{2}$$ 

  To find singular points we need to solve $\dot{x}=0$, $\dot{y}=0$ so that 

  $$\alpha x +\mu x^3 = 0 \mathrm{~~and~~} y=0 \tag{3}$$ 

  The solutions of the first of these equations are $x=0$, together with 

  $$x^2=-\alpha / \mu \tag{4}$$ 

  if the latter yields a real solution for $x$, in other words if $\alpha / \mu 
  \le 0$. So $(0,0)$ is always a singular point, and there are also singular 
  points at 

  $$x=\pm \sqrt{\dfrac{-\alpha}{\mu}} \mathrm{,~~~} y=0 \tag{5}$$ 

  if one of $\alpha$ or $\mu$ is positive while the other is negative. 

  To find out what kind of singular point these are, we need to perform the 
  linearised analysis described in section 8.3.2. For the point $(0,0)$, the 
  linearised matrix $A$ can be written down immediately by ignoring the cubic 
  term in eq. (2): it is 

  $$A=\begin{bmatrix}0 \& 1\\ -\alpha \& 0\end{bmatrix} \tag{6}$$ 

  so the trace $T=0$ as expected for an undamped oscillator, while the 
  determinant $D=\alpha$. The results from section 8.3.2 then tell us that if 
  $\alpha > 0$ the singular point is a centre, while if $\alpha < 0$ it is a 
  saddle point. 

  For the other singular point, we are only interested in the case $\alpha < 0$ 
  so write $\beta = -\alpha$ with $\beta > 0$. Now write 

  $$x=\pm \sqrt{\dfrac{\beta}{\mu}} + \epsilon \tag{7}$$ 

  so that $\epsilon$ is small close to one of the pair of singular points. 
  Equations (2) then read 

  $$\dot{\epsilon}=y \mathrm{,~~~~} \dot{y}=\beta \left[\pm 
  \sqrt{\dfrac{\beta}{\mu}} + \epsilon \right] -\mu \left[\pm 
  \sqrt{\dfrac{\beta}{\mu}} + \epsilon\right]^3 \tag{8}$$ 

  with the linearised approximation 

  $$\dot{\epsilon}=y \mathrm{,~~~~} \dot{y} \approx \beta \left[\pm 
  \sqrt{\dfrac{\beta}{\mu}} + \epsilon \right] -\mu \left[\pm 
  \left(\dfrac{\beta}{\mu}\right)^{3/2} + 3\epsilon\dfrac{\beta}{\mu} \right] 
  \tag{9}$$ 

  and so 

  $$\dot{\epsilon}=y \mathrm{,~~~~} \dot{y} \approx -2 \beta \epsilon . 
  \tag{10}$$ 

  The matrix $A$ is thus 

  $$A=\begin{bmatrix}0 \& 1\\ -2\beta \& 0\end{bmatrix}, \tag{11}$$ 

  which has trace $T=0$ and determinant $D=2\beta$, so since $\beta > 0$, both 
  these singular points are centres. 

  These results can be summarised in the bifurcation diagram shown in Fig.\ 1: 
  this shows the positions $x$ of the singular points as a function of 
  $\alpha$, for the case when $\mu > 0$. Stable points (centres) are indicated 
  by solid lines, unstable points (saddles) by a dashed line. At the value 
  $\alpha=0$ a bifurcation occurs. If you think of $\alpha$ decreasing, then at 
  this point the single stable solution turns into three solutions, the outer 
  pair stable and the middle one unstable. This pattern occurs in many systems: 
  it is called a pitchfork bifurcation. 

  The behaviour of Duffing's equation can also be viewed from the energy 
  perspective. Equation (1) can be obtained by using the kinetic energy 

  $$T=\dfrac{1}{2} \dot{x}^2 \tag{12}$$ 

  and the potential energy 

  $$V=\dfrac{\alpha}{2} x^2 + \dfrac{\mu}{4} x^4, \tag{13}$$ 

  then using energy conservation in the form $\frac{d}{dt}[T+V]=0$. 

  Figure 2 shows a plot of $V(x)$ for various values of $\alpha$. When $\alpha 
  > 0$, the curves show a single minimum at $x=0$, indicating the stable 
  centre. But when $\alpha < 0$ the curve changes shape so that it has a 
  maximum at $x=0$, indicating the unstable saddle point, flanked symmetrically 
  by two minima indicating the pair of centres. 