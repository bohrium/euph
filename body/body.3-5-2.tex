  The musical saw is an example of a waveguide: a system that is long in one 
  dimension, with a cross-section that is uniform or only slowly varying along 
  the length. Such systems may be capable of carrying waves along the length, 
  with behaviour in the cross-sectional plane corresponding to some kind of 
  resonance. The musical saw is an example: the behaviour in the cross-section 
  is like a free-free bending resonance across the width of the saw blade. To 
  give an idea of the mathematical approach to analysing such a system, we can 
  use a simpler waveguide. We will take a bending beam, like the one analysed 
  in section 3.2.1, but we will give it an elastic foundation, sometimes called 
  a Winkler foundation. Instead of the beam being free to vibrate as in the 
  earlier calculation, we assume that it is tied to fixed ground by springs 
  uniformly distributed along the length. This is a system often used to 
  analyse the effect of foundations under buildings or ballast underneath 
  railway track. 

  The system is sketched in Fig.\ 1. We can assume a beam just like the one 
  analysed in section 3.2.1, but with the addition of the foundation spring 
  with a stiffness $K$ per unit length: the units of $K$ are thus N/m/m. It is 
  easy to see that we need just one extra term in the governing equation 
  compared to eq. (7) from section 3.2.1: this equation becomes 

  $$\rho A \dfrac{\partial^2 w}{\partial t^2}+EI \dfrac{\partial^4 w}{\partial 
  x^4} + K w=0\tag{1}$$ 

  where the term $Kw$ represents the force from the foundation springs. We are 
  interested in solutions to this equation that are travelling waves, so try 
  substituting 

  $$w(x,t) = e^{i k x} e^{i \omega t} .\tag{2}$$ 

  Then eq. (1) requires that the wavenumber $k$ satisfies 

  $$k^4 = \dfrac{\rho A \omega^2 -K}{EI} . \tag{3}$$ 

  Without the presence of $K$, eq. (3) would mean that for any value of 
  $\omega$ we can find a real solution for $k$. In physical terms, that means 
  that on a beam without the foundation, waves can propagate at any frequency. 
  But with the foundation term, things are different. For frequencies 
  satisfying 

  $$\omega^2 > \frac{K}{\rho A} \tag{4}$$ 

  we can still find a real solution for $K$. But if 

  $$\omega^2 < \frac{K}{\rho A} \tag{5}$$ 

  this is no longer possible. The possible solutions take the form 

  $$k=\frac{\pm 1 \pm i}{\sqrt{2}} \left[\dfrac{K -- \rho A \omega^2 }{EI} 
  \right]^{1/4} \tag{6}$$ 

  so that all possible solutions decay exponentially with $x$. Such solutions 
  are called evanescent waves. The transition frequency $\omega = 
  \sqrt{\frac{K}{\rho A} }$, above which travelling waves first become 
  possible, is called the waveguide cut-on frequency (or cut-off frequency, if 
  you are thinking about scanning downwards in frequency when you encounter 
  it). 

  Now imagine that the value of foundation stiffness $K$ varies slowly along 
  the the length of the beam. We can obtain an analogue of the S-shaped saw by 
  supposing that $K \propto x^2$. The extra stiffness is zero at $x=0$, then 
  increases symmetrically as you move away from that point in either direction. 
  If we drive the beam into vibration at some chosen frequency $\omega$ at 
  $x=0$, waves will propagate outwards in both directions. But they cannot pass 
  beyond the pair of points where $K=\rho A \omega^2$, so the vibration must be 
  confined between these two points, just as we found in the musical saw. 

  Of course, the saw cannot resonate at every frequency. This mechanism of 
  reflection is part of the story, but there is another condition necessary to 
  create a mode like the one seen in the musical saw. This is the condition of 
  phase closure. We can track our wave as it reflects up and down the waveguide 
  and completes a full round trip. The phase will advance steadily as the wave 
  propagates, and the process of reflection will involve a phase change of some 
  kind. The condition for resonance, and the formation of a mode, is that the 
  total change of phase after a round trip must be a multiple of $2 \pi$, so 
  that the wave ``joins up with itself''. 