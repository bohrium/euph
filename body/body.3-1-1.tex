  The idealised model of a stretched string assumes that it is perfectly 
  flexible, and stretched between two fixed points. It is allowed to vibrate 
  transversely with a (small) displacement from equilibrium given by $w(x,t)$. 
  Suppose the string has tension $T$, mass per unit length $m$, and length $L$. 
  To obtain the equation of motion, consider a small element of the string 
  between positions $x$ and $x+ \delta x$ as sketched below. (Note that the 
  displacement of the string is hugely exaggerated in the plot, for clarity.) 

  \fig{figs/fig-6b1849ed.png}{\caption{Figure 1. A small element of a vibrating 
  string, shown with an exaggerated vertical scale.}} 

  Newton's law for this small element requires 

  \begin{equation*}m~\delta x \frac{\partial^2 w}{\partial t^2} =-T \sin 
  \theta_1 + T \sin\theta_2 \tag{1}\end{equation*} 

  But the angles $\theta_1$ and $\theta_2$ are both very small, so 

  \begin{equation*}\sin \theta_1 \approx \theta_1 \approx \tan \theta_1 = 
  \left[ \frac{\partial w}{\partial x} \right] _x \tag{2}\end{equation*} 

  \noindent{}and 

  \begin{equation*}\sin \theta_2 \approx \theta_2 \approx \tan \theta_2 = 
  \left[ \frac{\partial w}{\partial x} \right] _{x + \delta x}. 
  \tag{3}\end{equation*} 

  Thus 

  \begin{equation*}m \frac{\partial^2 w}{\partial t^2} \approx T \left[ 
  \dfrac{\left[ \frac{\partial w}{\partial x} \right] _{x + \delta x} -- \left[ 
  \frac{\partial w}{\partial x} \right] _{x} }{\delta x} \right] \rightarrow T 
  \dfrac{\partial^2 w}{\partial x^2} \tag{4}\end{equation*} 

  \noindent{}as $\delta x \rightarrow 0$. Thus the equation of motion is 

  \begin{equation*}m \frac{\partial^2 w}{\partial t^2} -- T \dfrac{\partial^2 
  w}{\partial x^2} =0 . \tag{5}\end{equation*} 

  If the motion were not free because a force $f(x,t)$ per unit length was 
  applied to the string, $f$ would replace the zero on the right-hand side of 
  this equation. 

  A vibration mode of the string is a free motion in which all points move 
  sinusoidally at some frequency $\omega$. So to find the natural frequencies 
  and vibration modes, we need to look for solutions of the form 

  \begin{equation*}w(x,t) = u(x) e^{i \omega t} \tag{6}\end{equation*} 

  (remembering that we really mean ``real part of...'' this complex 
  expression.) Substituting in eq. (5) then gives 

  \begin{equation*}T \dfrac{d^2u}{dx^2} + m \omega^2 u = 0. 
  \tag{7}\end{equation*} 

  This is the simple harmonic equation, so we already know that the general 
  solution is 

  \begin{equation*}u(x) = A \cos kx + B \sin kx \tag{8}\end{equation*} 

  \noindent{}where $A$ and $B$ are arbitrary constants, and 

  \begin{equation*}k^2 =\dfrac{m \omega^2}{T} = \dfrac{\omega^2}{c^2} 
  \tag{9}\end{equation*} 

  \noindent{}where $c=\sqrt{T/m}$. 

  The quantity $k$ is called wavenumber and is a kind of ``spatial frequency''. 
  It bears the same relation to wavelength $\lambda$ as a frequency $\omega$ 
  bears to the period of oscillation $\tau$: $\omega=2 \pi/\tau$, and $k=2 
  \pi/\lambda$. 

  One interpretation of this solution is that sinusoidal waves can propagate 
  along the string in either direction, and $c$ is the speed of these waves. To 
  see this, it is better to use the complex form of the general solution: 

  \begin{equation*}u(x) =A' e^{ikx} + B' e^{-ikx} \tag{10}\end{equation*} 

  \noindent{}where $A'$ and $B'$ are new arbitrary constants. The time-varying 
  solution then looks like 

  \begin{equation*}w(x,t) = A' e^{i(\omega t + kx)} + B' e^{i(\omega t -- kx)} 
  \tag{11}\end{equation*} 

  \begin{equation*}=A' e^{i\omega (t + x/c)} + B' e^{i\omega (t -- x/c))}. 
  \tag{12}\end{equation*} 

  Now to find the modes we have to satisfy the boundary conditions at the ends 
  of the string --- we are assuming that $u = 0$ at both ends, which we can 
  take to be at the positions $x = 0$ and $x = L$. From $u(0) = 0$ we can 
  deduce from eq. (8) that $A = 0$. Now to satisfy $u(L) = 0$ we require 

  \begin{equation*}B \sin kL = 0 \tag{12}\end{equation*} 

  \noindent{}so that the only allowed values of k are 

  \begin{equation*}k=n \pi/L \mathrm{~~for~~} n = 1, 2, 3, ... 
  \tag{13}\end{equation*} 

  From eq. (9) these correspond to allowed values of the frequency $\omega$. So 
  we have a sequence of mode shapes 

  \begin{equation*}u_n(x)=\sin (n \pi x/L) \tag{14}\end{equation*} 

  \noindent{}with corresponding natural frequencies 

  \begin{equation*}\omega_n = n \pi c/L \mathrm{~~for~~} n = 1, 2, 3, ... 
  \tag{15}\end{equation*} 