

  Up until now, we have been using a particular model to describe the 
  frictional behaviour between the bow and the string. This model is based on 
  two assumptions: that the friction force is proportional to the normal force 
  (i.e. the bow force), and that during sliding the friction force is a 
  nonlinear function of the sliding speed. The time has come to look carefully 
  at both these assumptions. What is the empirical evidence and/or theoretical 
  understanding on which they are based? 

  This turns out to be an interesting but tricky question, and it will lead us 
  right up to a research frontier. Most things we have discussed so far are 
  governed by physical principles like Newton’s laws, which are well 
  established and uncontroversial. But friction is different: there is no 
  single underlying theory that has similar status. The details can be quite 
  different, depending on the nature of the materials in contact, and in many 
  cases we still don’t know the full story. Most of the time engineering 
  calculations involving friction make use of grossly simplified 
  approximations, but these are not good enough if we want to understand 
  bowed-string transients in detail. 

  The first assumption, that friction force is proportional to bow force, we 
  will defer for the moment. The question has an interesting history, but a 
  discussion of it will fit more neatly into the next section when we talk 
  about how a real bow might give different behaviour from a rigid rod. For 
  now, we will concentrate on the second assumption, the “friction curve 
  model”. Our starting point is the measurement we saw earlier, repeated here 
  as Fig.\ 1. 

  \fig{figs/fig-4a3cc277.png}{\caption{Figure 1. Measurement of friction force 
  during steady sliding, showing variation with sliding speed, reproduced from 
  Fig. 6 of section 9.2.}} 

  Looking at this plot, it seems obvious that the friction force varies, rather 
  dramatically, with the sliding speed. But we have to be careful to avoid a 
  trap here. This plot shows the results of a particular type of measurement: 
  for each separate data point, steady sliding was imposed at a particular 
  speed, and the friction force was measured. Sliding speed is the only 
  variable, so of course we can plot the measured forces against the speeds and 
  we are bound to get something looking like a friction curve. 

  But now suppose we do a different measurement, in which the sliding speed 
  varies in time (as it does with the actual bowed string). Do the results from 
  Fig.\ 1 prove that the friction force is determined, moment by moment, by the 
  instantaneous value of the sliding speed? Absolutely not! Other factors might 
  come in, to do with the history of the motion. Figure 1 only says that if the 
  speed stays constant for long enough, the force should settle down to the 
  value plotted there. To resolve this question, we need to do measurements of 
  a different kind. 

  It is not easy to observe the friction force directly during stick-slip 
  vibration of a bowed string, or indeed of any other stick-slip system like a 
  squealing vehicle brake. It is very difficult to insert any kind of 
  force-measuring sensor right in the contact region, without drastically 
  changing the behaviour. Instead, ingenuity must be exercised to design an 
  experiment in which the friction force can be reliably inferred from 
  measurements of some kind elsewhere on the vibrating structure. 

  For the case we are interested in, when the friction is mediated by 
  violinist’s rosin, two different experiments of this kind have been carried 
  out. The first, by Jonathan Smith [1], made use of a cantilever-like 
  mass-spring oscillator. Once the effective mass, stiffness and damping of the 
  device have been found by calibration tests, the motion of the mass during a 
  stick-slip vibration can be measured and the friction force can be 
  reconstructed by substituting the displacement, velocity and acceleration 
  into the equation of motion of the oscillator. 

  The second experiment used an actual bowed string. Bob Schumacher designed a 
  rig in which a violin E string was bowed by a glass rod which had been coated 
  in rosin [2, 3]. The rosin was dissolved in a solvent, then a rod was slowly 
  drawn up out this solution, leaving a thin film of rosin after the solvent 
  had evaporated. At both ends of the string, force-measuring sensors of the 
  kind described in section 9.1.1 were inserted. With a bit of ingenious 
  processing in the computer, these two force signals can be combined to give 
  estimates of the string velocity and the friction force at the bowed point. 
  The next link gives a few details about how this can be done. 

  The Schumacher experiment gives the most illuminating information for our 
  purpose, so we will show some results from this rig. Figure 2 shows a few 
  cycles of the inferred waveforms of velocity (upper plot) and friction force 
  (lower plot). The string was executing Helmholtz motion in this example. The 
  pulses of slipping, with negative velocities, are clear. In between, the 
  episodes of sticking do not show a velocity that is exactly constant. Small 
  ripples in this waveform arise because the string can roll on the “bow”, as 
  described in sections 9.5 and 9.5.3. 

  \fig{figs/fig-fa503c95.png}{\caption{Figure 2. Example waveforms of string 
  velocity and friction force, for a case of Helmholtz motion in the ``glass 
  bow'' experiment.}} 

  Now if we take these waveforms of velocity and force and we plot one against 
  the other, we get the result seen in Fig.\ 3. There is a more-or-less 
  vertical stripe at the right-hand side, showing the intervals of sticking. 
  But during slipping, instead of a single friction curve we see a loop. The 
  path around this loop moves anti-clockwise. The friction force is highest at 
  the start of slipping (or in fact just after the start of slipping), then it 
  falls as the slipping speed increases. But on the way back towards the 
  sticking line, the force remains much lower than the maximum value on the way 
  out. The details of the loop shape can vary in different bow strokes, but the 
  qualitative behaviour is always very similar to this plot. Furthermore, 
  Jonathan’s Smith’s experiment also gave very similar results, despite using a 
  very different vibrating system. 

  \fig{figs/fig-38a85e4a.png}{\caption{Figure 3. The results from Fig. 2, 
  plotted in the force-velocity plane. Friction force has been normalised by 
  the bow force. The loops during slipping are traversed in an anti-clockwise 
  direction: friction force is high at the start of slipping, but much lower at 
  the end. The more-or-less vertical patch of ``scribble'' on the right-hand 
  side of the plot shows the behaviour during sticking.}} 

  The conclusion is that the friction force is not simply determined by the 
  sliding speed: you get a different answer depending on whether the speed is 
  increasing or decreasing. Rosin does not follow the kind of friction-curve 
  model we have been assuming up to now. So what is going on? We should step 
  back a little, and ask what is special about rosin. Why do violinists use it 
  in the first place? Indeed, it is not only violinists: there are many 
  bowed-string instruments around the world, and there are also other musical 
  instruments which make use of stick-slip vibration, for example \tt{}friction 
  drums\rm{}. Players of all these instruments coat their bowhair, or their 
  hurdy-gurdy wheel, or in the case of friction drums their fingers, with 
  rosin. 

  Rosin is a natural material, obtained from various species of pine tree. The 
  raw tree resin is separated using distillation or solvents into a liquid 
  component called “spirit of turpentine” and a solid component, which is 
  rosin. It is manufactured in industrial quantities, because it has many uses: 
  it is used in making soaps, printing inks, paper, adhesives, and many other 
  applications. It is also frequently used for its high friction: not only for 
  violin bows, but also to give dancers, weightlifters or baseball players a 
  firmer grip on things. 

  Violin rosin is a clear, brittle material at room temperature: if you drop a 
  block on the floor it is likely to shatter. But if you hold it in your 
  fingers for a minute or two, it soon becomes sticky. It is a type of material 
  known technically as a glass. As any glass-blower knows, if you heat ordinary 
  glass up it does not show a sharp melting point at which it suddenly becomes 
  a liquid (like melting ice). Instead, there is a broad range of temperatures 
  over which the glass softens progressively. By holding the temperature in 
  this range, the glass-blower can shape and mould it into laboratory glassware 
  or decorative drinking glasses. 

  Rosin behaves in a similar way, but the changes happen at a lower temperature 
  than with glass. The softening range of temperatures, usually characterised 
  by the glass transition temperature marking the middle of the range, is not 
  very far above room temperature. What this means is that the mechanical 
  properties of rosin change rather quickly with a relatively small rise in 
  temperature. As the temperature rises, the rosin becomes softer, so that the 
  force required to deform a layer of rosin goes down. This is probably the 
  origin of the behaviour seen in Fig.\ 1. When a rosin-coated bow or rod is 
  forced to slide across another object (such as a violin string), heat is 
  generated. The faster the sliding, the hotter the rosin becomes, and the 
  lower is the resulting friction force. So the friction force does vary with 
  the sliding speed, but only indirectly. The real cause of the variation is a 
  change of temperature. 

  The idea that the friction force is strongly influenced by temperature, 
  rather than sliding speed as such, gives an immediate qualitative explanation 
  for the loop seen in Fig.\ 3. Heat is generated during episodes of sliding, 
  but very little is generated during episodes of sticking. So think what will 
  happen during a cycle of Helmholtz motion. At the end of a slip, the rosin 
  will be relatively warm and so the friction will be low. But during the 
  relatively long sticking interval, the heat has a bit of time to diffuse away 
  into the body of the string and bow. So by the time the Helmholtz corner 
  arrives back to trigger the next slip, the rosin layer has cooled down a bit, 
  and the friction force is higher. As it slips, the rosin heats up rapidly and 
  the friction falls. The result is a loop in the velocity—force plane, just as 
  we saw in the measured results. 

  What this description amounts to is a claim that the rosin melts a little and 
  re-freezes during every cycle of the string’s vibration, several hundred 
  times a second. Is that really credible? Remarkably, we can get some direct 
  evidence for this from the Schumacher experiment. We can use one of his glass 
  rods for just a few bow strokes in the apparatus, rotating it a bit each time 
  so that a different part of the rosined surface is used. We can then take 
  that rod and look at it in the scanning electron microscope, and we see 
  visible tracks left by the string’s vibration. 

  Figure 4 shows an example. The featureless grey background is the very smooth 
  surface of the rosin coating on the glass rod. Running across this background 
  we can clearly see three tracks, made up of a fairly regular row of little 
  vertical lines. Each of those lines is the “footprint” of a single sticking 
  event in the string’s vibration. The orientation of the string was vertical 
  in this image, and the rod has been moved in the horizontal direction when 
  performing the bow stroke. The three tracks have been created by three 
  separate bow strokes. 

  \fig{figs/fig-c14cdcb9.png}{\caption{Figure 4. Three tracks left on the glass 
  rod ``bow'' by single bow strokes}} 

  Figure 5 shows a zoomed view of a portion of one of these tracks, and Fig.\ 6 
  shows an even closer view. Each of the vertical scars in the track shows 
  churned-up rosin. This is the result of the string rolling back and forth a 
  little during that sticking interval. But then the string slips rapidly 
  across the surface of the rod, and we can see very clear evidence that the 
  rosin has been partially melted: look at the streaks on the left-hand side of 
  Fig.\ 6. These are surely ``threads'' of hot rosin, drawn out by the string 
  as it slipped across the surface. 

  \fig{figs/fig-5a4e83a2.png}{\caption{Figure 5. Close-up of a typical track, 
  showing two scars from episodes of sticking, separated by regions where 
  slipping has melted the rosin}} 

  \fig{figs/fig-b6b08926.png}{\caption{Figure 6. Closer view of the terrain 
  near a stick-slip transition}} 

  Figure 7 shows a low-angle view, to give a different perspective on the 
  terrain revealed by these images. The churned-up texture created by the 
  rolling string during the sticking event which caused the scar in the middle 
  of the image is particularly clear here. 

  \fig{figs/fig-cbd3736d.png}{\caption{Figure 7. Low-angle view along a typical 
  track}} 

  There are standard laboratory measurements that can give information about 
  the mechanical properties of rosin as a function of temperature. As a first 
  step, the usual way to determine the glass transition temperature is with a 
  device called a “differential scanning calorimeter”. This measures the amount 
  of heat absorbed by a small sample of rosin (or whatever material is being 
  tested), as the temperature is slowly raised. Whenever a material undergoes a 
  phase transition from solid to liquid, or from liquid to gas, heat energy 
  must be supplied in the critical temperature range. The resulting peak in the 
  heat absorbtion identifies the temperature of the transition. In the case of 
  a glass transition the peak is quite broad, but its highest point can still 
  be used to put a number on the glass transition temperature. 

  But we are most interested in mechanical properties, rather than thermal 
  properties as such. For this, we can use a device called a “rheometer” which 
  measures the force needed to deform a layer of the material. The specific 
  deformation we are interested in is shear, because a sliding frictional 
  contact requires shearing between the two surfaces. Figure 8 shows some 
  results of measurement of the shear viscosity of two different types of 
  commercial rosin for bowed instruments, which represent the extremes of 
  available behaviour. The temperature has been varied slowly during the test. 
  The red points are for violin rosin, of the kind we have been talking about 
  so far. The blue points are for a brand of rosin aimed at double bass 
  players. This bass rosin is supplied in a pot: unlike violin rosin, which is 
  supplied in a solid block, the bass rosin will flow, slowly, at ordinary room 
  temperatures so it needs to be prevented from getting away. 

  \fig{figs/fig-1c3c28fd.png}{\caption{Figure 8. Measured shear viscosity of 
  two types of commercial rosin, as a function of temperature: red points are 
  for violin rosin, blue points for a type of double bass rosin. The glass 
  transition temperatures for the two types of rosin, measured with a 
  differential scanning calorimeter, are shown as vertical dashed lines in 
  corresponding colours.}} 

  The glass transition temperatures of the two types of rosin are shown by 
  vertical dashed lines in corresponding colours: $16^\circ$C for the bass 
  rosin, and $49^\circ$C for the violin rosin. The variation of viscosity with 
  temperature reflects this difference: the two curves have very similar 
  shapes, but they are separated along the temperature axis by approximately 
  the difference of these two temperatures. These plots make it rather clear 
  why the bass rosin needs to be kept in a pot: it is already above its glass 
  transition temperature at normal room temperature. 

  There are two things to notice about this viscosity plot. First, there is a 
  gap in the sequence of red points. This is simply because two different types 
  of rheometer were needed to cover temperature ranges where the material was 
  “solid” and “liquid”. But it is easy to see that the two would join up if the 
  intermediate range of temperature could have been tested. The second thing to 
  note is the vertical scale. A logarithmic scale is needed for the viscosity 
  because the values have changed by almost 7 orders of magnitude over this 
  range of temperature. When I said that the mechanical properties of rosin 
  changed sensitively with small changes in temperature, I was not joking! 

  The next step is to try to extend the computer simulation model to 
  incorporate temperature-dependent friction. First, we need to calculate the 
  temperature in parallel with simulating the string motion. The simplest 
  version of such a calculation is easy to describe. Figure 9 shows a schematic 
  close-up of the contact region of rod, string and rosin. The rod (being used 
  as a bow as in the Galluzzo or Schumacher experiments) carries a layer of 
  rosin. It is moved across the string, seen in cross-section in the sketch. 
  There will be a small “contact footprint” between rod and string, with a size 
  that depends on the diameters of rod and string, and also on the normal force 
  (as will be discussed in the next section). In that footprint region, the 
  rosin will be warmer than ambient temperature, indicated by the red patch. 

  \fig{figs/fig-3a9c51bf.png}{\caption{Figure 9. Sketch of the contact region 
  between rod and string. The layer of rosin is shown schematically in blue. 
  The region shaded in red is the contact footprint: it is the temperature in 
  this patch that needs to be computed}} 

  To work out the temperature in this contact patch, we can do a “heat balance” 
  calculation. There are four effects to consider. Heat is being generated by 
  friction, at a rate which is simply the product of the friction force and the 
  speed of relative sliding between the two surfaces. Heat is being lost by two 
  mechanisms. First, there is diffusion of heat by conduction into the bodies 
  of the rod and the string. The second mechanism is associated with the moving 
  rod: cold rosin is carried into the contact area, while warmer rosin moves 
  out at the other end, taking some heat away with it. Finally, there is a term 
  associated with changes in the heat stored in the red contact region: if the 
  temperature is rising, then heat is being added; if it is falling, heat is 
  being removed. Putting these four things together yields a governing equation 
  that we can solve in the computer, alongside the simulation of the string's 
  motion. In the unlikely event that you are keen to see the gory details, they 
  are explained in reference [1] (you can find a PDF at number 61 in the list 
  \tt{}here\rm{}). 

  Now for the difficult part: we need to choose a specific model for exactly 
  how the friction force is influenced by the temperature. So far, I have been 
  talking as if the force will only depend on temperature, but this is rather 
  misleading. One thing we definitely know about any friction force is that it 
  always opposes the sliding motion. It must change sign if the sliding 
  direction is reversed — but of course the temperature will be the same, 
  regardless of the direction of sliding. So the friction “law” we are looking 
  for must involve the sliding velocity as well as the temperature, in some 
  way. We don’t really have enough experimental data to determine the correct 
  answer, so we need to resort to a bit of guesswork and empirical exploration. 

  The simplest model, and the one that has been most extensively investigated, 
  is to assume that the coefficient of friction is a function of temperature 
  alone [4,5]. This coefficient would be multiplied by the sign of the sliding 
  speed ($+$ or $-$) to give the necessary reversal of friction force when 
  sliding reverses. We can use the measurement from Fig.\ 1 to infer what the 
  variation with temperature must be. The same computer code can be used to 
  simulate the steady sliding experiment, and thus obtain the contact 
  temperature as a function of sliding speed. This can be used to convert the 
  results of Fig.\ 1 into a function of temperature: the result is shown in 
  Fig.\ 10. These results are based on violin rosin similar to the red points 
  in Fig.\ 8, so it probably has a similar glass transition temperature, around 
  $49^\circ$C. It is reassuring to see that this temperature falls near the 
  middle of the downward slope in Fig.\ 10. 

  \fig{figs/fig-e0ea8d60.png}{\caption{Figure 10. Coefficient of friction as a 
  function of temperature, inferred from the data of Fig. 1 by the procedure 
  described in the text.}} 

  Armed with this function of temperature, we are ready to incorporate the 
  thermal friction model into our bowed-string simulation. Figure 11 shows an 
  example of a Guettler constant-acceleration transient, simulated in this way. 
  Specifically, it is the case corresponding to the pixel (10,14) in the 
  Guettler plots seen in section 9.5 (10th acceleration from the left, 14th 
  force from the bottom, bow position $\beta=0.0899$). The red curve shows the 
  new simulation, while the black curve in this figure shows the corresponding 
  measured bridge force. In this case, both of them lead to the Helmholtz 
  sawtooth, after a rather short transient. 

  \fig{figs/fig-3ec7cb11.png}{\caption{Figure 11. A Guettler transient, 
  measured (black) and simulated with the thermal friction model (red)}} 

  Figure 12 shows the corresponding computed temperature variation. The mean 
  temperature is predicted to rise to some $70^\circ$C, and in every cycle of 
  the final Helmholtz motion there is a fluctuation by some $30^\circ$C. This 
  certainly seems like a big enough change to account for the appearance of the 
  tracks on the glass rod seen in Figs.\ 5—7: the rosin looked ``melted'' 
  during slipping, but semi-solid during sticking. 

  \fig{figs/fig-4a6d0b4c.png}{\caption{Figure 12. The predicted contact 
  temperature from the simulation shown in red in Fig. 11.}} 

  However, we only learn a rather limited amount by looking at a single 
  transient like this. The next step is to use the new model to scan a family 
  of transients and construct a Guettler diagram. Figure 13 shows an example, 
  chosen to match the cases we saw in Fig.\ 7 of section 9.5, which is 
  reproduced here as Fig.\ 14 for easy comparison. It is immediately obvious 
  that the new model behaves quite differently to the friction-curve model. It 
  has produced a fairly solid patch of colour (marking cases that led to 
  Helmholtz motion), in contrast to the “spotty” texture of the friction-curve 
  case. 

  \fig{figs/fig-0414dd56.png}{\caption{Figure 13. An example of a Guettler 
  diagram computed using the thermal friction model for the bowing position 
  $\beta=0.0899$.}} 

  \fig{figs/fig-7a92a26b.png}{} 

  \fig{figs/fig-d2d5f2ed.png}{} 

  So far, so good: but the patch of colour in Fig.\ 13 is nowhere near as big 
  as the corresponding patch in the measured Guettler diagram. A player of the 
  friction-curve “cello” would find their job almost impossible: every pale 
  pixel marking a good, fast transient is not far from some black pixels. This 
  means that if the cellist tried to repeat a successful bow stroke, they are 
  likely to be frustrated by sensitivity to small details of the gesture. The 
  “thermal model cello” of Fig.\ 13 would be preferred: there is a region of 
  the Guettler plane where every transient gives a successful outcome, with 
  relatively small variations in transient length. However, the real cello 
  would surely win hands down in this playability comparison: it offers a far 
  larger range of transients that “work”. From the point of view of a beginner, 
  this means that it is easier to perform a bow gesture that is not a disaster. 
  From the perspective of an expert, the larger available area of the plane 
  will open up a wider “sound palette” of musical possibilities. 

  So can we do any better than this? In terms of improved models that are 
  firmly based in understanding of the underlying physics, there is no good 
  answer to that question at present. But to give a hint that better models are 
  out there waiting to be found, I will show a few examples from a slightly 
  enhanced version of our thermal friction model. Comparing the two waveforms 
  in Fig.\ 11, we can see one qualitative difference at the very start of the 
  string's vibration. Both waveforms begin with a similar rising curve. What is 
  happening here is that the string is sticking to the bow, so it is being 
  pulled to one side by the accelerating motion of the bow. The motion we are 
  interested in can only begin after the first moment of release, when slipping 
  starts. In the measured waveform (in black), that first release happens with 
  a jump. The simulated waveform in red shows no such jump. Furthermore, as 
  explained in the next link, this version of the thermal friction model can 
  never show a jump. 

  There is a simple (but rather ad hoc) way to enhance our friction model to 
  allow the possibility of jumps. The argument is explained in the previous 
  link. An example of this enhancement in action is shown in Fig.\ 15. The two 
  waveforms from Fig.\ 11 are now followed by a third, in blue. This new 
  waveform does indeed show an initial jump that looks broadly similar to the 
  measurement (in black). 

  \fig{figs/fig-f3e2fbaf.png}{\caption{Figure 15. The two waveforms from Fig. 
  11, plus a corresponding simulation (in blue) for the modified thermal 
  friction model described in the text.}} 

  Using this new model to scan a range of transients, we obtain the Guettler 
  diagram shown in Fig.\ 16. Looking back at Figs.\ 13 and 14, is this better 
  or worse than the original thermal model? Better, surely: coloured pixels are 
  found over a larger area of the Guettler plane than in Fig.\ 13. The terrain 
  in both plots is a bit spotty, with black pixels dotted around, but this is 
  not necessarily a bad thing: we have seen in section 9.5 that the real bowed 
  cello string showed clear evidence of ``sensitive dependence'' leading to 
  spottiness rather like this. 

  \fig{figs/fig-c21b5390.png}{\caption{Figure 16. Guettler diagram in the same 
  format as earlier figures, simulated using the modified thermal friction 
  model.}} 

  It is worth digging into the details a bit more, to understand what is going 
  on and get a sense of what a player might think about it. This particular 
  presentation of the Guettler results certainly does not tell the whole story. 
  It is based on an automated routine for detecting Helmholtz motion, and then 
  establishing the length of each transient. The problem with that kind of 
  routine is that it is drawing a hard line between “Helmholtz”and “not 
  Helmholtz”. Sometimes, this distinction is clear, but not always: there are 
  “nearly Helmholtz” waveforms, and placing the threshold is a matter of 
  judgement on the part of the programmer of the automated detection routine. 
  The result is that a gradual change from Helmholtz to nearly-Helmholtz, which 
  might not bother a player very much, is turned into an abrupt change from a 
  coloured pixel to a ``failed'' black one. 

  If we process the results in a different way, we can get a different view of 
  the relative merits of the various simulation models we have described. One 
  approach is to look at the final waveform, near the end of each simulation, 
  and simply ask how similar this is to the corresponding measured waveform: no 
  attempt is made to classify the waveforms. Specifically, we will look at two 
  period-lengths near the end of each measured Guettler transient. A 
  corresponding chunk can be extracted from the simulated results, adjusting 
  the phase of the two-period chunk to give the best match. We can then 
  calculate a numerical measure of the difference between the two: the 
  particular measure chosen to be plotted here is the root-mean-square (RMS) 
  difference, after each separate waveform has been scaled to have unit RMS 
  value. 

  If we plot how this difference measure varies over the Guettler plane, we get 
  the results shown in Figs.\ 17, 18 and 19. Figure 17 compares the 
  measurements with the original thermal model, Fig.\ 18 with the modified 
  thermal model, and Fig.\ 19 with the friction-curve model. The paler and 
  ``hotter'' the colour, the more similar the waveforms are. Of the three 
  figures, Fig.\ 18 has the hottest colours overall. Figure 17 shows a similar 
  pattern but with less white and more yellow. Figure 19, for the friction 
  curve model, performs the worst. Based on these plots, the modified thermal 
  model looks the most promising. 

  \fig{figs/fig-ed2c6522.png}{\caption{Figure 17. Difference plot, comparing 
  two period-lengths at the end of each Guettler measurement with the 
  corresponding results simulated with the original thermal model. The brighter 
  the pixel, the more similar are the waveforms.}} 

  \fig{figs/fig-1489904e.png}{\caption{Figure 18. Difference plot in the same 
  format as Fig. 17, comparing measurements with simulations using the modified 
  thermal friction model.}} 

  \fig{figs/fig-79069779.png}{\caption{Figure 19. Difference plot in the same 
  format as Figs. 17 and 18, comparing measurements with simulations using the 
  friction-curve model.}} 

  To see what lies behind these plots, the next set of figures shows the 
  detailed results for a single row of the Guettler diagram. Figure 20 shows 
  the 20 waveform comparisons which give the difference results shown in Fig.\ 
  21, based on the original thermal model. In each case, the measured waveform 
  is shown in black, and the phase-matched simulation is shown in red. The 
  first two cases of Fig.\ 20 show rather irregular string motion for both 
  measurement and simulation. The details are quite different, so it is no 
  surprise that the difference measure is quite big. The next 11 cases show 
  something looking like Helmholtz motion in both measurement and simulation, 
  and the first few of these show quite a good match. 

  Cases 9--13 are interesting: the simulations appear to show Helmholtz motion, 
  as does the measurement, but the shapes are distinctively different in the 
  simulations. The bottom “points” of the sawtooth waveform have been rounded 
  off, in a way that is not echoed in the measurement. These waveforms are 
  sufficiently different from the ideal Helmholtz sawtooth wave that the 
  automatic classification routine has rejected them: that is why the 
  right-hand half of Fig.\ 13 was solid black. This is a good example of the 
  difficulties of automatic classification --- but the routine has correctly 
  identified the fact that the simulation is not a good match to the 
  measurements, which is the question we are most interested in at the moment. 
  The remaining cases for Fig.\ 20 show versions of double-slipping or other 
  Raman “higher types” in both measurement and simulation. The simulations 
  still have the characteristic rounded points, but the qualitative comparison 
  is otherwise reasonably good. 

  \fig{figs/fig-a3de2c63.png}{\caption{Figure 20. The waveform comparisons from 
  which the difference measure was calculated, for one row of the results from 
  Fig. 17 relating to the original thermal friction model. Measured waveforms 
  are shown in black, phase-matched simulated waveforms in red.}} 

  \fig{figs/fig-6472e7eb.png}{\caption{Figure 21. The row of Fig. 17 relevant 
  to the waveforms shown in Fig. 20.}} 

  Figure 22 and 23 show corresponding results for the modified thermal model. A 
  first thing to notice is that none of the simulated waveforms here show the 
  rounded bottom points, as we noticed with the original thermal model. As in 
  Fig.\ 20, the first two cases show irregular motion in both measurement and 
  simulation. Cases 3, 4 and 5 show Helmholtz motion in the measurement, while 
  the simulations either show Helmholtz motion with large Schelleng ripples, or 
  perhaps they show S-motion. This is another example of the difficulty of 
  automatic classification: it is far from clear how these waveforms should be 
  labelled. 

  \fig{figs/fig-d89a24e5.png}{\caption{Figure 22. Waveform comparisons in the 
  same format as Fig. 20, for one row of results simulated by the modified 
  thermal friction model.}} 

  \fig{figs/fig-e17afac7.png}{\caption{Figure 23. The row of Fig. 18 relevant 
  to the waveforms shown in Fig. 22.}} 

  The middle two rows of Fig.\ 22 show Helmholtz motion in both measurement and 
  simulation, in most cases. Case 11 is an exception: it has Helmholtz motion 
  in the measurement but double slipping in the simulation. We need to be a bit 
  careful in how we interpret this. We have already commented that the bowed 
  string, both in reality and in simulation, shows some evidence of “sensitive 
  dependence”. Back in section 9.5 we saw some results of repeat measurements 
  of the Guettler plane, and although the qualitative picture remained the 
  same, individual pixels could change between takes. What we are seeing in 
  case 11 of Fig.\ 21 may be illustrating this sensitivity: perhaps the choice 
  between Helmholtz motion and double-slipping is rather finely balanced for 
  this case, and therefore subject to sensitivity. The disagreement between 
  measurement and simulation is not necessarily pointing to incorrectness in 
  the model. 

  The final row of Fig.\ 22 shows some variety of double-slipping in both 
  measurement and simulation, in all cases. There are differences of detail, 
  leading to variations in the RMS difference measure, but actually the 
  qualitative agreement seems rather good throughout. In some cases, the 
  difference simply reflects a small difference in the phasing of the two 
  slips: I would suspect that such a difference has very little influence on 
  the perceived sound. Overall, then, the modified thermal model performs 
  rather well in this comparison: slightly better than the original thermal 
  model. That is not to say, by any means, that this model is perfect, but it 
  is very encouraging. Is the agreement good enough that this model could be 
  used reliably to do systematic simulation studies on playability differences? 
  The answer to that question is not yet clear: I said we would reach a 
  research frontier in this section. 

  Finally, for completeness Figs.\ 24 and 25 show corresponding results for the 
  friction-curve model. We need not go through and describe the waveforms 
  individually. Some cases show reasonable agreement with the measurements, but 
  it is pretty obvious that this model gives unsatisfactory results for many 
  cases. 

  \fig{figs/fig-dae4caca.png}{\caption{Figure 24. Waveform comparisons in the 
  same format as Figs. 20 and 22, for one row of results simulated by the 
  friction-curve model.}} 

  \fig{figs/fig-dd45fa7a.png}{\caption{Figure 25. The row of Fig. 19 relevant 
  to the waveforms shown in Fig. 24.}} 



  \sectionreferences{}[1] J. H. Smith and J. Woodhouse,~ “The tribology of 
  rosin”;~ Journal of the Mechanics and Physics of Solids~ \textbf{48} 
  1633–1681 (2000). 

  [2] J. Woodhouse, R. T. Schumacher and S. Garoff, “Reconstruction of bowing 
  point friction force in a bowed string”;~ Journal of the Acoustical Society 
  of America~ \textbf{108} 357–368 (2000). 

  [3] R.T. Schumacher, S. Garoff and~J. Woodhouse, “Probing the physics of 
  slip-stick friction using a bowed string”; Journal of Adhesion \textbf{81}, 
  723–750 (2005). 

  [4] J. Woodhouse, “Bowed string simulation using a thermal friction model”; 
  Acta Acustica united with Acustica \textbf{89}~ 355–368 (2003). 

  [5] P. M. Galluzzo, J. Woodhouse and H. Mansour, “Assessing friction laws for 
  simulating bowed-string motion”;~ Acta Acustica united with Acustica 
  \textbf{103}, 1080-1099, (2017).~ DOI 10.3813/AAA.919136 