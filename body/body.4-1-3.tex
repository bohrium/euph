  This section will make extensive use of vector calculus. The next link gives 
  an introduction to the key concepts, if you don't have them at your 
  fingertips. 

  We start by rewriting eqs. (15) and (16) from section 4.1.1 in terms of the 
  particle velocity $\underline{v'}=\frac{\partial \underline{\xi}}{\partial 
  t}$, to obtain 

  \begin{equation*}-\rho_0 \nabla \cdot \underline{v'} = \frac{1}{c^2} 
  \frac{\partial p'}{\partial t} \tag{1}\end{equation*} 

  \noindent{}and 

  \begin{equation*}\rho_0 \frac{\partial \underline{v'}}{\partial t} = -- 
  \nabla p' . \tag{2}\end{equation*} 

  We can combine these to give an important energy equation. Multiply eq. (1) 
  by $p'/\rho_0$, take the scalar product of eq. (2) with $\underline{v'}$ and 
  add the two together: 

  \begin{equation*}\dfrac{1}{\rho_0 c^2} p' \dfrac{\partial p'}{\partial t} + 
  p' \nabla \cdot \underline{v'} + \rho_0 \underline{v'} \cdot \dfrac{\partial 
  \underline{v'}}{\partial t} + \underline{v'} \cdot \nabla p' = 0 
  \tag{3}\end{equation*} 

  \noindent{}which can be rewritten 

  \begin{equation*}\dfrac{\partial}{\partial t} \left( \dfrac{1}{2} \rho_0 v'^2 
  + \dfrac{1}{2 \rho_0 c^2} p'^2 \right) + \nabla \cdot (p' \underline{v'}) = 0 
  . \tag{4}\end{equation*} 

  The two terms in the brackets describe the kinetic energy density $E_k=\rho_0 
  v'^2/2$ and the potential energy density stored in compression of the air, 
  $E_p=p'^2/2 \rho_0 c^2$. (A derivation of this expression for potential 
  energy is given at the end of the section.) The final term relates to energy 
  flux. The vector $\underline{I}=p' \underline{v'}$ is called the intensity: 
  it gives the rate and direction at which acoustic energy crosses a unit area 
  of space. If eq. (4) is integrated over a volume $V$ bounded by a surface 
  $S$, we obtain 

  \begin{equation*}\int_V{\dfrac{\partial}{\partial t} (E_k+E_p) dV} + 
  \int_V{\nabla \cdot \underline{I} dV}=0 \tag{5}\end{equation*} 

  \noindent{}so, using the divergence theorem, 

  \begin{equation*}\int_V{\dfrac{\partial}{\partial t} (E_k+E_p) dV} + 
  \int_S{\underline{I} \cdot \underline{n} dS}=0 \tag{6}\end{equation*} 

  \noindent{}which implies that the rate of change of total energy inside $V$ 
  is balanced by energy transport across its surface $S$ which is governed by 
  the intensity. 

  For a plane wave in the $x$ direction, we can write $p'=\hat{p} e^{i \omega 
  (t-x/c)}$. The only non-zero component of velocity will be in the $x$ 
  direction; we can denote it $u'$. From the $x$ component of eq. (2), we then 
  have 

  \begin{equation*}-\rho_0 \dfrac{\partial u'}{\partial t} = \dfrac{\partial 
  p'}{\partial x} = -\dfrac{i \omega \hat{p}}{c} e^{i \omega (t-x/c)} 
  \tag{7}\end{equation*} 

  \noindent{}so that 

  \begin{equation*}u' = \dfrac{\hat{p}}{\rho_0 c} e^{i \omega (t-x/c)} . 
  \tag{8}\end{equation*} 

  Pressure and velocity are in phase, and their ratio, called impedance, is 

  \begin{equation*}\frac{p'}{u'} = \rho_o c =Z \tag{9}\end{equation*} 

  \noindent{}where $Z$ is a real value that is independent of frequency. It is 
  a material property of the air, called the characteristic impedance, or 
  sometimes wave impedance. 

  The concept of impedance is used in several fields of science, and the 
  general context is always the same. There are two variables, one force-like 
  and the other motion-like, whose product gives power. In an electrical system 
  we would have voltage and current, in a mechanical system we would have force 
  and velocity. For our sound wave we have force per unit area, i.e. pressure, 
  and velocity, so that their product gives power per unit area. In all these 
  cases the impedance is the ratio of the force-like variable to the 
  motion-like variable. The inverse ratio is called admittance. We will look 
  extensively at mechanical admittance of stringed instrument bodies in Chapter 
  5, and at impedance curves for wind instruments in Chapter 11. 

  Returning to our plane wave in the $x$ direction, the intensity vector is 
  also in the $x$ direction, with magnitude 

  \begin{equation*}I=p' u' = \dfrac{\hat{p}^2}{Z} \cos^2{\omega (t-x/c)} . 
  \tag{10}\end{equation*} 

  Note that we can't use the complex exponential representation directly here, 
  because intensity is a nonlinear quantity. At a position $x$, the intensity 
  varies with time with a $\cos^2$ dependence. The most interesting quantity is 
  the average intensity over one cycle, and since we know that the average 
  value of $\cos^2$ is 1/2, the result is 

  \begin{equation*}\left< I \right> =\frac{1}{2} \frac{\hat{p}^2}{Z} . 
  \tag{11}\end{equation*} 

  For the spherical wave from section 4.1.2 we have 

  \begin{equation*}p'=\frac{\hat{p}}{r} e^{i \omega t -- i k r} 
  \tag{12}\end{equation*} 

  \noindent{}from eq. (7), and 

  \begin{equation*}u' = \frac{\hat{p}}{i\omega \rho_0 r} \left( \frac{1}{r} 
  +\frac{i \omega}{c} \right) e^{i\omega t -- ikr} = \frac{\hat{p}}{\rho_0 r} 
  \left(\frac{1}{c} + \frac{1}{i \omega r}\right) e^{i\omega t -- 
  ikr}\tag{13}\end{equation*} 

  \noindent{}for the radial component of velocity. We want to multiply this by 
  $p'$ from eq. (12) and average over a cycle, as we did in the plane-wave 
  case. But this time we need to be a bit careful. The first term in the 
  brackets in eq. (13) describes a component of $u'$ that is in phase with 
  $p'$, so that we can average the product as before, gaining a factor 1/2 in 
  the process. But the second term has an extra factor of $i$ in the complex 
  representation: this component of velocity is in quadrature with the pressure 
  (i.e. $90^\circ$ out of phase), which means that the product of the two will 
  average to zero over one cycle. So we only need to keep the first term, with 
  the result 

  \begin{equation*}\left< I \right> =\frac{1}{2} \frac{\hat{p}^2}{Z} 
  \tag{14}\end{equation*} 

  \noindent{}exactly as in the case of the plane wave. 

  This calculation gives a clue for how we should interpret the near field and 
  far field in this example. The term in eq. (13) which contributed to energy 
  transport was the far field term, decaying like $1/r$. The near field term, 
  decaying like $1/r^2$, was the one in quadrature, involving no net energy 
  flow. This near field term has a simple physical interpretation: it describes 
  incompressible motion of the air around the pulsating sphere. You can picture 
  it by thinking of pumping up a spherical balloon under water: the water flow 
  will be essentially incompressible, and it gets pushed symmetrically outwards 
  by the growing sphere. For incompressible motion like this, the condition on 
  a spherical surface of radius $r$ is that the total volume flow across the 
  surface must be independent of $r$. The area of the surface is proportional 
  to $r^2$, so we see why the velocity needs to vary like $1/r^2$. This 
  interpretation applies to many problems involving sound sources with small 
  Helmholtz number. Within the near field, there is a component of 
  incompressible motion, which can dominate over the sound-radiating component 
  of motion at small distances. 

  \textbf{Footnote}: a derivation of the potential energy expression 

  In equation (4) we saw the energy equation relating to sound waves. It was 
  stated that $p'^2/2\rho_0c^2$ was the potential energy density, but the truth 
  of that statement is not immediately apparent. We will give a derivation 
  here. Energy in a sound wave is stored by compressing the air (or other 
  fluid), so consider a small volume, compressed by an acoustic pressure $p'$. 
  The energy storage is directly analogous to the mechanical energy stored in a 
  spring: to derive the familiar expression $(1/2)kx^2$ for that energy, it is 
  necessary to integrate the contributions as the spring is gradually 
  compressed, starting from the relaxed state. 

  The equivalent integration for our fluid element is $E=-\int{p' dV}$. But the 
  mass of the fluid element $\rho V$ is fixed, so for small changes $dV$ and $d 
  \rho$ we can deduce $\rho_0 dV + V_0 d\rho =0$ where $V_0$ is the initial 
  volume and $\rho_0$ is the ambient density of air. So 

  \begin{equation*}E=\dfrac{V_0}{\rho_0}\int{p' d\rho'}=\dfrac{V_0 
  c^2}{\rho_0}\int{\rho' d\rho'}=\dfrac{V_0 c^2}{\rho_0}\dfrac{1}{2}\rho'^2 
  \tag{15}\end{equation*} 

  \noindent{}where we have used the result $p'=c^2 \rho'$ from section 4.1.1. 

  So the potential energy density is 

  \begin{equation*}\dfrac{1}{2}\dfrac{c^2}{\rho_0}\rho'^2 = 
  \dfrac{1}{2}\dfrac{1}{\rho_0 c^2}p'^2 \tag{16}\end{equation*} 

  \noindent{}using the same relation again. The final expression is the form we 
  need for equation (4). 