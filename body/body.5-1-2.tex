  The result of connecting a stretched string to a non-rigid instrument body is 
  that some energy will be transferred from the string into body vibration. To 
  analyse this, we will suppose the string has tension $T$ and mass per unit 
  length $m$, so that it has a wave speed $c=\sqrt{T/m}$ and a wave impedance 
  $Z_0=\sqrt{Tm}$. One end of the string, at $x=0$, is connected to the 
  instrument body, with bridge admittance $Y(\omega)$. The other end, at $x=L$, 
  is assumed to be rigidly fixed. 

  It is easiest to work in terms of travelling waves on the string. Suppose 
  that a sine wave with frequency $\omega$ and unit amplitude travels in from 
  $x>0$, and impinges on the bridge. A reflected wave will be generated, with 
  the same frequency but travelling in the opposite direction. It will have a 
  reduced amplitude, and perhaps a phase shift. We can represent both effects 
  through a complex reflection coefficient $R(\omega)$. The incident wave can 
  be written as $e^{i \omega(t+x/c)}$, while the reflected wave is $R e^{i 
  \omega (t-x/c)}$: the situation is sketched in Fig.\ 1. 

  \fig{figs/fig-cdf9e49a.png}{\caption{Figure 1: sketch of an incident wave on 
  a string, impinging on the bridge and generating a reflected wave.}} 

  The total deflection of the string is given by the sum of the two waves: 

  $$w(x,t)=e^{i \omega(t+x/c)}+R e^{i \omega (t-x/c)} \tag{1}$$ 

  (remembering that, as usual, the physical solution will be given by the real 
  part of this complex expression). We can now write down expressions for the 
  motion of the string at the bridge, and also for the force exerted on the 
  bridge at that point. These two things must be related via the admittance 
  $Y$, and the resulting equation will allow us to solve for the reflection 
  coefficient $R$. 

  The string displacement at $x=0$ is given directly from eq. (1) as $(1+R)e^{i 
  \omega t}$. The force on the bridge is given by the resolved component of the 
  tension in the direction normal to the soundboard, which is approximately 
  equal to 

  $$T \left[ \dfrac{\partial w}{\partial x} \right]_{x=0} = T \left[ \dfrac{i 
  \omega}{c} -- R \dfrac{i \omega}{c} \right] e^{i \omega t} = i \omega Z_0 
  (1-R) e^{i \omega t}\tag{2}$$ 

  since $Z_0=T/c$. So the condition for the admittance $Y$ is that 

  $$i \omega (1+R) = Y(\omega) i \omega Z_0 (1-R) .\tag{3}$$ 

  The factor of $i \omega$ on the left-hand side is needed to convert the 
  string's displacement into its velocity. This equation can easily be solved, 
  to give 

  $$R=\dfrac{Y(\omega) Z_0 -- 1}{Y(\omega) Z_0 + 1} .\tag{4}$$ 

  For any realistic stringed instrument, the body motion at the bridge is 
  always small compared to the string motion. This translates into the 
  statement that $|Y(\omega) Z_0| \ll 1$. We can then make use of the binomial 
  theorem to obtain a simpler approximate expression for $R$: 

  $$R \approx [Y(\omega) Z_0 -- 1][1-Y(\omega) Z_0] \approx -1 +2Y(\omega) Z_0 
  .\tag{5}$$ 

  This makes physical sense: a rigid termination would give an inverted 
  reflection with $R=-1$, and this is slightly modified by the non-rigid 
  bridge. 

  The final step is to think about the rate of energy loss from the string into 
  the body, and find an expression for the resulting loss factor of a string 
  mode, when any other sources of energy dissipation are ignored. A mode of the 
  string consists of a standing wave, which is the resultant of the 
  forward-travelling and backward-travelling waves in Fig.\ 1. The wave comes 
  to the bridge, is reflected as we have just analysed, then travels to the 
  other end of the string where it is reflected with no further loss of energy 
  because we have assumed a rigid termination there. It then completes the 
  circuit by arriving back at the bridge. 

  During one round trip, the amplitude of the wave has been reduced by a factor 
  $|R|$. The energy carried by a wave is proportional to the square of its 
  amplitude, so the energy has been reduced by a factor $|R|^2$ in that time. 
  The proportional loss of energy is thus $(1-|R|^2)$. The definition of loss 
  factor is that it describes the proportional energy loss per radian of 
  vibration, so we need to divide by the number of radians in a round trip. For 
  the $n$th mode of the string, with approximate mode shape $\sin (n \pi x/L)$, 
  there are $n$ cycles per round trip, or $2 \pi n$ radians. So the loss factor 
  due to energy loss through the bridge is 

  $$\eta^{(n)}_{body} \approx \dfrac{1-|R|^2}{2 \pi n} . \tag{6}$$ 

  Making use of the approximation in eq. (5), the magnitude of $R$ can be 
  calculated via 

  $$|R|^2=R R^* \approx [-1+2Y(\omega) Z_0] [-1+2Y^*(\omega) Z_0]$$ 

  $$\approx 1 -- 2 Z_0 (Y + Y^*) \approx 1 -- 4 Z_0 \Re(Y)\tag{7}$$ 

  where $.^*$ denotes the complex conjugate, and $\Re(.)$ denotes the real part 
  of a complex quantity. So finally, 

  $$\eta^{(n)}_{body} \approx \dfrac{2 Z_0 \Re(Y)}{ \pi n} . \tag{8}$$ 