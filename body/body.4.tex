  This chapter is the second of the ``Underpinnings'' series. It deals with 
  sound waves and acoustics, in a similar way that Chapter 2 dealt with 
  mechanical vibration. Key concepts are introduced and illustrated, to prepare 
  the ground for discussing question like how sound is radiated from a 
  vibrating violin body, and what determines how loud the instrument seems to a 
  listener out in the concert hall. 

  Section 4.1 introduces the basic ideas of sound waves: how sound can be 
  generated by a vibrating body, how the waves then spread out in space, and 
  how waves from different sources can interfere to cause complicated patterns 
  in space. 

  Section 4.2 deals with acoustic resonators of various kinds. All the musical 
  wind instruments involve resonances of sound inside tubes of various shapes. 
  The frequencies and mode shapes of these resonances determine how each 
  instrument behaves: what notes it can play, how it will ``overblow'' when 
  played outside the lowest pitch range, and how the tuning of these resonant 
  frequencies is governed by the shape of the tube bore. But tubes are not the 
  only type of acoustic resonator relevant to music. The simplest of all 
  resonators, the Helmholtz resonator, is responsible for the popping noise 
  when a cork is pulled from a bottle, and also governs the sound of a violin 
  or guitar in its lowest register. Finally, rooms have acoustic resonances, 
  and these influence the sound and reverberation behaviour of different kinds 
  of performance space. 

  Section 4.3 looks at the spatial patterns of sound radiation from different 
  kinds of vibrating structure. Most sound sources, whether loudspeakers, 
  violins or trumpets, tend to radiate sound rather uniformly in all directions 
  at low frequency, but they all become increasingly directional as frequency 
  rises. 

