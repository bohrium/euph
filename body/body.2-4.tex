

  Now back to the frequency response function of the drum. Figure 1, reproduced 
  from section 2.2, shows a sequence of very clear peaks, which correspond to 
  the resonances or natural frequencies of the drum. Comparing the two 
  different measurements, it is clear that they both have peaks at the same 
  frequencies, but that the pattern of heights of those peaks is very different 
  for the two tapping positions. This illustrates something said earlier: a 
  drummer excites a different mixture of the vibration modes by hitting in 
  different places. The peak heights in the measurement reflect the relative 
  amplitudes in the mixture, and the sound of the drum will be different in the 
  two cases. 

  \fig{figs/fig-ec199e5c.png}{\caption{Figure 1: Frequency response functions 
  of the toy drum, reproduced from section 2.2, for two different positions of 
  the tapping hammer. The numbers on the y-axis correspond to calibrated 
  values: the frequency response function at any given frequency is a velocity 
  divided by a force, so the units are meters per second per Newton, m s$^{-1}$ 
  N$^{-1}$.}} 

  Something else can be noticed in this plot --- or rather, not noticed where 
  it might perhaps have been expected. The peaks, showing the resonances or 
  natural frequencies, are spaced in a rather irregular way. They are 
  definitely not in a regular pattern in which the higher frequencies are exact 
  multiples of the first one, called the fundamental frequency. In other words, 
  the natural frequencies of this drum are not harmonically spaced. Now, as we 
  will see in chapter 3, some musically-important vibrating systems do have 
  natural frequencies which are harmonically spaced, at least approximately. 
  But most things don't do this: our toy drum is quite typical in this respect. 
  For some reason, there is a widespread belief that all sequences of resonant 
  frequencies are ``harmonics'', to the extent that people sometimes use the 
  word ``harmonic'' interchangeably with ``resonance''. If you had this idea 
  before you started reading this paragraph, then stop it right away! It is 
  going to be very important in the next chapter to examine whether or not some 
  of the natural frequencies of certain objects are harmonically spaced. If we 
  were to call the frequencies ``harmonics'', we would find ourselves saying 
  things like ``In this case the harmonics are not exactly harmonic'', and 
  confusion would reign. Call them natural frequencies, or resonances, or 
  overtones, or partials, but reserve the word ``harmonic'' for the theoretical 
  notion of frequencies that are exact multiples of a fundamental. 

  We can show some of the mode shapes corresponding to the resonances of the 
  toy drum using a time-honoured method called Chladni patterns. Ernst Chladni 
  was an eighteenth century scientist who found he could visualise the 
  vibration of a glass disc by holding it lightly on his fingertips, sprinkling 
  sand on the surface, then making it vibrate using a violin bow against the 
  edge. The sand bounces around on the vibrating disc, and tends to collect 
  along the nodal lines. Nowadays, rather than using a violin bow it is easier 
  and more controlled to use an electronic sine-wave generator and a small 
  loudspeaker to ``sing'' at the disc or, in our case, the drum. We sprinkle 
  powder on the surface: tea leaves are a good choice, because they are less 
  heavy than sand and bounce more easily, and also the black colour shows up 
  nicely on the white skin. The sine wave oscillator is gently adjusted through 
  the frequency range to seek out resonances. The tea leaves then reveal the 
  nodal line patterns. 

  Some examples for our drum are shown in Fig.\ 2. The first mode is not 
  illustrated, because there is nothing to see by this method: the only nodal 
  line is around the rim of the drum. The first strong peak in Fig.\ 1 shows 
  that this lowest mode occurs at about 130 Hz. The next two, appearing as a 
  close pair of peaks around 230 Hz, correspond to modes with a single nodal 
  diameter: one of these is shown in the first Chladni pattern. The companion 
  mode would have the nodal diameter rotated by $90^\circ$. For the perfect 
  textbook drum in Fig.\ 3 of section 2.2, these two modes occurred at exactly 
  the same frequency, but the real drum is not perfect, and the two modes have 
  slightly different frequencies. This might be caused by non-uniformity in the 
  mass distribution of the membrane, or by a slightly irregular distribution of 
  tension: something always prevents real objects from having perfect symmetry, 
  it only occurs in textbooks. For related reasons, the frequency of these 
  modes relative to the lowest mode is similar to the textbook value, but not 
  exactly equal to it. 

  \fig{figs/fig-f1c9d232.png}{} 

  \fig{figs/fig-7837e3fc.png}{} 

  \fig{figs/fig-8321bac6.png}{} 

  \fig{figs/fig-8997210f.png}{} 

  Representative Chladni patterns for three more modes of the drum are also 
  shown: one with one nodal circle, one with three nodal diameters, and one 
  mode at a higher frequency with a more complicated shape not matching any of 
  the patterns of the idealised theory. This is another universal phenomenon: 
  as you go up the sequence of modes for any structure, the shapes become 
  increasingly sensitive to small mechanical details, and sooner or later they 
  cease to be recognisable in comparison to theoretical estimates. This remark 
  does not only apply to our drum, it is equally true of large computer models 
  of cars, buildings or aeroplanes. 

  Chladni patterns are not the only approach to visualising vibration patterns: 
  more hi-tech options are also available. One way is the use a laser-Doppler 
  vibrometer, mentioned earlier, to scan over a grid of points on the surface 
  of the object, and then assemble animated views of the response at each peak 
  frequency. Two examples are shown below, obtained from the stretched membrane 
  forming the head of a banjo. The vertical scale is hugely exaggerated 
  compared to the actual vibration when the banjo is in use. The banjo head 
  exhibits similar shapes to the toy drum. The first one illustrated is similar 
  to the Chladni figure with three nodal diameters, while the second one shows 
  motion with two nodal circles. 

\moobeginvid\begin{tabular}{ccc} \vidframe{ 0.30 }{ vids/vid-7b38791f-00.png }&\vidframe{ 0.30 }{ vids/vid-7b38791f-01.png }&\vidframe{ 0.30 }{ vids/vid-7b38791f-02.png } \end{tabular}\mooendvideo

\moobeginvid\begin{tabular}{ccc} \vidframe{ 0.30 }{ vids/vid-de1e9eee-00.png }&\vidframe{ 0.30 }{ vids/vid-de1e9eee-01.png }&\vidframe{ 0.30 }{ vids/vid-de1e9eee-02.png } \end{tabular}\caption{Figure 3.  Animations of the motion of the head of a banjo, driven at two different frequencies.}\mooendvideo

  Strictly, neither this approach nor the Chladni method really shows modes: 
  these patterns are technically known as Operating Deflection Shapes, often 
  abbreviated to ODS. They are the response to driving at a single frequency 
  close to a response peak. The pattern will be dominated by the closest mode 
  shape, but the motion will also contain some contribution from other modes. 
  Evidence of this effect can be seen in the first of the animations: the 
  pattern appears to rotate slightly during the motion. This is caused by a 
  combination of the two modes which each have three nodal diameters: see the 
  discussion in section 2.2.4. On the real banjo head these two modes do not 
  have exactly the same frequency, so they contribute to this ODS with slightly 
  different phases. The combination produces the effect of rotation. 

  There is an experimental technique which can, at least in principle, reveal 
  true mode shapes rather than ODS patterns. It is called ``experimental modal 
  analysis''. A discussion of how this works will be deferred to section 10.5, 
  but as a taster, here is an example of a mode shape of the body of a violin 
  obtained by this method. 

\moobeginvid\begin{tabular}{ccc} \vidframe{ 0.30 }{ vids/vid-37441170-00.png }&\vidframe{ 0.30 }{ vids/vid-37441170-01.png }&\vidframe{ 0.30 }{ vids/vid-37441170-02.png } \end{tabular}\caption{Figure 4. A mode shape of a violin body. Data and animation copyright George Stoppani, reproduced by permission.}\mooendvideo

  Returning to the toy drum, we can use it to illustrate a very different way 
  to represent vibration or sound in graphical form. This way is known as 
  ``time-frequency analysis'', or the ``spectrogram'', and it has a lot in 
  common with the way our own ears process sound. In the description up to now, 
  plots with time on the axis were turned into plots with frequency on the 
  axis, using Fourier analysis and the FFT. But we don't hear those as 
  alternatives, we hear them somehow mixed up. Play a scale on a piano: how do 
  you describe the sound? Surely, as a sequence of pitches, in other words as 
  frequencies arranged in a particular pattern in time. 

  There are several ways to generate such a time-frequency description in the 
  computer. The simplest is good enough to show the idea. We first get the 
  sound or vibration waveform of interest into the computer by sampling it in 
  the normal way for any digital recording. But instead of taking the entire 
  chunk of sound and performing a huge FFT to turn the whole thing into a 
  Fourier sine-wave recipe, we chop it into smaller segments and FFT each one 
  separately. Stack the results next to each other in the order of the segments 
  and, lo, we have a time-varying frequency spectrum. The pictures look nicer 
  if we use a few processing tricks, such as overlapping the segments, but 
  essentially that is all there is to it. 

  An example for the drum is shown in Fig.\ 5. The main plot shows the 
  spectrogram, with frequency along the horizontal axis and time running 
  vertically upwards. The colours in the plot show the distribution of energy 
  in the sound, ``hotter'' for louder. To help relate this picture to the 
  earlier ones, the original time waveform is plotted vertically on the left, 
  synchronised with the time axis of the spectrogram. Similarly, the full FFT 
  is plotted along the bottom, aligned with the spectrogram frequency axis. 
  What the spectrogram shows is a set of vertical stripes, getting cooler as 
  they go up. These are the individual modes of the drum, each vibrating at its 
  own natural frequency, and dying away with time as the energy is dissipated 
  into sound radiation and heat. You can listen to the sound this spectrogram 
  is calculated from in Sound 1. 

  \fig{figs/fig-b59cd169.png}{\caption{Figure 5: The main colour-shaded image 
  shows a spectrogram of the toy drum. Along the bottom is the frequency 
  spectrum obtained by a full FFT of the whole time history. On the left-hand 
  side is the original time history. These subsidiary time and FFT traces are 
  correctly aligned with the spectrogram axes.}} 

\audio{}

  Looking closely at the bottom plot, it can be seen that the frequencies of 
  the resonances are more sharply defined in the full FFT than in the 
  spectrogram. Indeed at higher frequencies the separate resonant frequencies 
  begin to overlap and blur together in the spectrogram. This is the price of 
  trying to see time and frequency variation simultaneously. They are both 
  aspects of the same thing, and the more you try to see details of variation 
  in time, the more you blur the details in frequency; and vice versa. This is 
  in fact an acoustical version of a famous result in quantum mechanics called 
  the ``Heisenberg uncertainty principle'' --- but it is easier to understand 
  in acoustics! 

  The uncertainty effect does not only happen in the computer: your own hearing 
  system also has to cope with it. Listen to the sound of a piece of cardboard 
  whirring against the spokes of a bicycle wheel. If the wheel is going slowly, 
  you hear individual little thud noises as each spoke passes. But when the 
  wheel goes fast enough you hear a kind of note, with a pitch that goes up as 
  the wheel speeds up. Somewhere between these two extremes you have switched 
  from hearing ``time stuff'' to ``frequency stuff''. Where exactly is the 
  switchover? There is no precise answer, but it generally falls in the 
  vicinity of 20 thuds per second. A cat's purr happens at around about this 
  rate: do you hear it as a low note or a fast-repeating sequence in time? This 
  defines, a little vaguely, the lowest note you can perceive as a pitch (about 
  20 Hz), and at the same time the fastest temporal events you can resolve 
  (around 1/20 second, or 50 milliseconds). Needless to say there is more to it 
  than this: perception is endlessly subtle, but this will do us for the 
  moment. 

  A few spectrogram plots of other sounds show the kind of things they can 
  reveal. Figure 6 shows a single note on a guitar: you can hear it in Sound 2. 
  It is broadly similar to the drum picture, with vertical stripes gradually 
  cooling as they go up. But notice the regular pattern of the frequencies of 
  these peaks. A stretched string has natural frequencies which fall (almost) 
  in a harmonic pattern, and we will examine this more closely in the next 
  chapter. Figure 7 shows a single note played on a violin, with vibrato: you 
  can hear it in Sound 3. If you screw up your eyes a bit, or take your glasses 
  off, this spectrogram might look quite similar to the previous picture, with 
  a set of regularly-spaced vertical stripes. But the player's vibrato leads to 
  a regular ripple pattern, particular obvious at the higher frequencies. The 
  fundamental frequency is being modulated by a few percent. Each harmonic 
  inherits the same percentage variation, so as the frequency goes up the 
  absolute variation gets bigger and more obvious. Notice that the FFT plot 
  along the bottom shows only blurring of the peaks: the spectrogram analysis 
  is far better at bringing out the structure of the ``wobble''. 

  \fig{figs/fig-add26cf8.png}{\caption{Figure 6: A spectrogram of a plucked 
  note on a guitar}} 

\audio{}

  \fig{figs/fig-8a6ceb14.png}{\caption{Figure 7: A spectrogram of a note on a 
  violin, played with vibrato}} 

\audio{}

  And now, in the famous phrase, for something completely different. Figure 8 
  shows a spectrogram of the sound of a Chinese orchestral gong or tamtam: you 
  can hear it in Sound 4. The spectrogram has been smoothed to bring out the 
  most important aspect clearly. Look at the very bottom of the spectrogram 
  plot. This is the moment when the beater hit the tamtam. A soft beater was 
  used, and it only excites vibration up to about 500 Hz. But as time goes on, 
  something magical happens. There is a slanting ``front'' running up from 
  about 500 Hz on the bottom line of the plot, above which hotter colours are 
  seen. This is a striking demonstration of a nonlinear effect: over the first 
  half second or so, energy which was initially concentrated at low frequencies 
  makes its way, somehow, up to far higher frequencies. This is a visualisation 
  of the characteristic ``blooming'' sound of the tamtam. No linear system can 
  behave like this, and having seen this tantalising glimpse we will hurriedly 
  return to simpler questions for the time being. 

  \fig{figs/fig-01c4bad3.png}{\caption{Figure 8: A spectrogram of the sound of 
  a tamtam (Chinese gong).}} 

\audio{}

