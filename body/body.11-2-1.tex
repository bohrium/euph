  This section will make extensive use of vector calculus concepts and 
  notation: if you are a bit rusty on that, check \tt{}section 4.1.4\rm{} 
  first. 

  The governing equations for fluid flow involve two rather different 
  ingredients. First, we will use universal laws of physics: Newton's law and 
  the conservation of mass. The second ingredient is more empirical, and the 
  details could vary depend on which fluid we are interested in, and what kind 
  of flow regime we wish to study. This ingredient is a constitutive law, 
  relating the internal forces (described via a stress tensor) to the fluid 
  motion (described by a rate of strain tensor, provided our fluid can be 
  approximated by a classical ``Newtonian fluid'' with no ``memory''). If our 
  fluid is compressible, we also need a thermodynamic relation of some kind to 
  relate density to pressure. 

  Suppose that our fluid has velocity $\underline{u}(\underline{r},t)$, density 
  $\rho(\underline{r},t)$ and pressure $p(\underline{r},t)$. First, we will 
  look at the requirement for conservation of mass. Consider the small 
  brick-shaped element of volume sketched in Fig.\ 1, with one corner at 
  position $\underline{r}=(x,y,z)$ and sides of length $\delta x$, $\delta y$ 
  and $\delta z$ aligned with the $x,y,z$ axes of a Cartesian coordinate 
  system. This element has volume $\delta V= \delta x \delta y \delta z$. 

  \fig{figs/fig-30d0c10a.png}{\caption{Figure 1. A small element of volume}} 

  The rate of change of mass inside the volume is approximately 

  $$\dfrac{\partial \rho}{\partial t} \delta V \tag{1}$$ 

  where the derivative of density is evaluated at the centre of the element. 
  Now we need to look at the rate at which mass is flowing into or out of the 
  volume, through the various faces. For the two vertical faces perpendicular 
  to the $x$ axis, the net mass flow into the volume is 

  $$\left[\rho(x,y,z,t) u_1(x,y,z,t) -- \rho (x + \delta x,y,z,t) u_1(x + 
  \delta x,y,z,t) \right] \delta y \delta z $$ 

  $$\approx -\delta V \dfrac{\partial (\rho u_1)}{\partial x} \tag{2}$$ 

  where $(u_1,u_2,u_3)$ are the components of $\underline{u}$ so that $u_1$ is 
  the velocity perpendicular to the faces we are thinking about. The other two 
  pairs of faces contribute similar terms, so the net mass flow into the volume 
  is 

  $$-\delta V \dfrac{\partial (\rho u_1)}{\partial x} -\delta V \dfrac{\partial 
  (\rho u_2)}{\partial y} -\delta V \dfrac{\partial (\rho u_3)}{\partial z} = 
  -\delta V \nabla \cdot (\rho \underline{u}) . \tag{3}$$ 

  Equating this to the rate of increase of mass inside the volume from (1), we 
  obtain the equation 

  $$\dfrac{\partial \rho}{\partial t} = -- \nabla \cdot (\rho \underline{u}) . 
  \tag{4}$$ 

  Now we can use a standard vector identity 

  $$\nabla (\rho \underline{u}) = \underline{u} \cdot \nabla \rho + \rho \nabla 
  \cdot \underline{u}\tag{5}$$ 

  to rewrite equation (4) in the form 

  $$\dfrac{\partial \rho}{\partial t} + \underline{u} \cdot \nabla \rho = -- 
  \rho \nabla \cdot \underline{u}\tag{6}$$ 

  or 

  $$\dfrac{D \rho}{D t} = -- \rho \nabla \cdot \underline{u}\tag{7}$$ 

  where $\frac{D\rho}{Dt} = \frac{\partial \rho}{\partial t} + \underline{u} 
  \cdot \nabla \rho$ is the rate of change of density following the flow, in 
  other words the rate of change of density of a fluid particle. 

  Equation (7) immediately tells us that an incompressible flow, for which 
  $\frac{D\rho}{Dt} =0$, must satisfy 

  $$\nabla \cdot \underline{u} = 0 \mathrm{~~~~(incompressible~flow)} . 
  \tag{8}$$ 

  Equation (7) also gives us an example of a physical interpretation of the 
  divergence: $\nabla \cdot \underline{u}$ must be the rate of increase of 
  volume of a fluid element. 

  Fluid flow can also give us a physical example of the curl of a vector field. 
  To see this, we will look at the local flow field in the immediate 
  neighbourhood of a chosen point, which we can define to be $\underline{r}=0$. 
  The first two terms of a Taylor expansion of the velocity field then give 

  $$u_i \approx u_i^{(0)} + \sum_j{c_{ij} r_j} \mathrm{~~~for~~}i=1,2,3 
  \tag{9}$$ 

  where it is convenient to work in terms of components of the vectors for this 
  particular analysis, so $\underline{u} = (u_1,u_2,u_3)$ and 
  $\underline{r}=(r_1,r_2,r_3)$ (which are just another name for $x,y,z$). In 
  equation (9), $u_i^{(0)}$ are the components of the velocity at the reference 
  position $\underline{r}=0$, and the $3 \times 3$ matrix (or more technically 
  the second-rank tensor) 

  $$c_{ij} = \dfrac{\partial u_i}{\partial r_j} \mathrm{~~~for~~}i,j=1,2,3 . 
  \tag{10}$$ 

  Now we separate $c_{ij}$ into the sum of its symmetric part and its 
  antisymmetric part: 

  $$c_{ij} = \frac{1}{2}\left[c_{ij} + c_{ji} \right]+\frac{1}{2}\left[c_{ij} 
  -- c_{ji} \right] = e_{ij} + \omega_{ij} . \tag{11}$$ 

  The antisymmetric term has a simple interpretation. By writing everything out 
  in full and then comparing, it can be easily verified that 

  $$\sum_j{\omega_{ij} r_j}=\frac{1}{2}[\underline{\omega} \times 
  \underline{r}]_i \mathrm{~~~for~~}i=1,2,3 \tag{12}$$ 

  where the vector 

  $$\underline{\omega} = \nabla \times \underline{u} .\tag{13}$$ 

  This vector is called the vorticity. Equation (12) describes rigid rotation, 
  at angular velocity $\underline{\omega}/2$. 

  The effect of the symmetric term $e_{ij}$ (known technically as the rate of 
  strain tensor) can also be described quite easily. This is a symmetric $3 
  \times 3$ matrix, so by a standard result from linear algebra it has three 
  mutually perpendicular eigenvectors, and if we rotate our coordinate axes to 
  those three directions, the matrix is diagonalised. This means that the 
  associated term in equation (9) describes a flow which would turn a sphere 
  into an ellipsoid, with axes aligned with those three eigenvectors. 

  We can summarise all this with a geometric description. If you were to inject 
  a blob of dye to make a small sphere around the chosen reference point, the 
  subsequent flow can be described as the sum of three components. The blob 
  will rotate with an angular velocity $(1/2)\nabla \times \underline{u}$, its 
  volume will grow at rate $\nabla \cdot \underline{u}$, and the remaining part 
  of the flow will cause the sphere to distort into an ellipsoid. 

  We are nearly ready to apply Newton's law to our flowing fluid, and deduce 
  the governing equation. But first we need to relate the rate of strain tensor 
  $e_{ij}$ to the internal forces, which are described by another $3 \times 3$ 
  matrix, the stress tensor $\sigma_{ij}$. The definition of this stress tensor 
  is based on a hypothetical measurement. Suppose we imagine a small element of 
  area somewhere within the flow, described by a vector area $\underline{dA}$. 
  If we could somehow embed a force-measuring sensor in this little piece of 
  area, it would find that there was a force acting across the two faces. That 
  force could in principle be in any direction: for example the effect of 
  pressure in the fluid would give a force normal to the surface, parallel to 
  the vector $\underline{dA}$, whereas a shearing flow in the presence of 
  viscosity would give a shear force, acting parallel to the surface and thus 
  perpendicular to $\underline{dA}$. If this force has components 
  $(p_1,p_2,p_3)$ per unit area, the stress tensor is defined so that 

  $$p_i=\sum_j{\sigma_{ij} dA_j}\mathrm{~~~for~~}i=1,2,3 . \tag{14}$$ 

  If the fluid is at rest, the only internal force acting is pressure. At any 
  given point, the pressure always acts normal to any surface element 
  $\underline{dA}$, and the value of that force is the same, whatever the 
  orientation of the surface. This means that the stress tensor takes a very 
  simple form: 

  $$e_{ij}=-p \delta_{ij} \tag{15}$$ 

  where $\delta_{ij}$ is the mathematician's standard name for the identity 
  matrix: 

  $$\delta_{ij} = \left\lbrace \begin{matrix} {1 \mathrm{~~if~~} i=j}\\{0 
  \mathrm{~~if~~} i\ne j}\end{matrix} \right. \tag{16}$$ 

  When the fluid is flowing, we can generalise this result and define the 
  pressure by 

  $$p=-\dfrac{1}{3} \sum_{i=1}^3{e_{ii}} \tag{17}$$ 

  Now we can think about Newton's law. The acceleration of a fluid particle is 
  given by $\frac{D\underline{u}}{Dt}$: obviously we need the rate of change of 
  velocity following the particle. To combine this with what we know about the 
  internal forces, it is clearest to apply Newton's law to a region of the flow 
  field, rather than to a small element. So suppose we have a volume $V$ within 
  the flow, with a closed surface $S$ forming its outside ``skin''. The result, 
  for the $i\mathrm{th}$ component, is 

  $$\int{\int{\int_V{\rho \dfrac{Du_i}{Dt} dV}}}=\int{\int{\int_V{\rho f_i 
  dV}}}+\int{\int_S{\sum_j{\sigma_{ij} dA_j }}} \tag{18}$$ 

  where $f_i$ is the $i\mathrm{th}$ component of any external force (such as 
  gravity) that may be acting throughout the body of our fluid. It is expressed 
  here as a force per unit mass, so that if it is in fact the gravitational 
  force we would have $\underline{f}=\underline{g}$ in terms of the usual 
  gravitational acceleration $\underline{g}$. Now we can apply a version of the 
  divergence theorem to the final term, to convert it from a surface integral 
  to a volume integral: 

  $$\int{\int{\int_V{\rho \dfrac{Du_i}{Dt} dV}}}=\int{\int{\int_V{\rho f_i 
  dV}}}+\int{\int{\int_V{\sum_j{\dfrac{\partial \sigma_{ij}}{\partial r_j} dV 
  }}}} . \tag{19}$$ 

  Now we have a sum of three volume integrals equal to zero, but the volume 
  over which we have integrated is entirely arbitrary. The only way this result 
  could be true for every choice of volume $V$ would be if 

  $$\rho \dfrac{Du_i}{Dt}=\rho f_i+\sum_j{\dfrac{\partial \sigma_{ij}}{\partial 
  r_j}}\mathrm{~~~for~~}i=1,2,3 \tag{20}$$ 

  everywhere. This is the governing equation we want, but it still remains to 
  express $\sigma_{ij}$ in terms of $e_{ij}$, which is turn depends on 
  $\underline{u}$. 

  From here on I will only present the simplest case, when the flow is 
  incompressible so that equation (8) is satisfied. We want a relation between 
  $e_{ij}$ and $\sigma_{ij}$. If we make the assumptions (i) that this relation 
  is linear; and (ii) that the physical properties of the fluid are isotropic 
  (in other words without any preferred directions), then a standard result 
  from tensor algebra tells us what the form of the relationship must be: 

  $$\sigma_{ij} = -p \delta_{ij} + 2 \mu e_{ij} \tag{21}$$ 

  where $\mu$ is a constant, the viscosity. Substituting and tidying up leads 
  to the equation 

  $$\rho \dfrac{D \underline{u}}{Dt}= -\nabla p \mathrm{~} -- \rho 
  \underline{f}+\mu \nabla^2 \underline{u}. \tag{22}$$ 

  This is the Navier-Stokes equation for the case of incompressible flow. 
  Notice that the only nonlinearity here is the term $\underline{u} \cdot 
  \nabla \underline{u}$ hidden within the $D/Dt$ term. 

  The next interesting thing we can deduce is called Bernoulli's principle. 
  Again, I will just present the very simplest version of this, which is enough 
  to give the flavour. Suppose we have a flow which is steady, incompressible 
  and inviscid ($\mu=0$), with density the same everywhere. Our Navier-Stokes 
  equation then reads 

  $$\rho \underline{u} \cdot \nabla \underline{u}= -\nabla p \mathrm{~} -- \rho 
  \underline{f} . \tag{23}$$ 

  If the body force is gravitational, then it can be written as the gradient of 
  the gravitational potential $gz$ where $g$ is the usual gravitational 
  acceleration and $z$ is the vertical coordinate, measured upwards. Now note 
  that 

  $$\underline{u} \cdot \nabla \underline{u} = \frac{1}{2} \nabla 
  (\underline{u} \cdot \underline{u}) . \tag{24}$$ 

  Every term in the equation is now a gradient, and it can be rearranged into 

  $$\nabla \left(\dfrac{q^2}{2}+\dfrac{p}{\rho}+gz\right) = 0 \tag{25}$$ 

  where $q^2=\underline{u} \cdot \underline{u}$ is the square of the flow 
  speed. It follows that 

  $$\dfrac{q^2}{2}+\dfrac{p}{\rho}+gz\ = \mathrm{~constant.} \tag{26}$$ 

  We can see explanations of more than one phenomenon in this simple equation. 
  First, it tells us about the increase in static pressure with depth, for 
  example under water. If the water is not flowing, so $q=0$, it tells that 
  $p=K \rho-\rho g z$. Recalling that we have defined $z$ to be positive 
  upwards, if a diver goes down into the sea they will encounter pressure 
  rising with depth, just as we expect. This is called the hydrostatic 
  pressure, equal to $\rho g |z|$. More directly relevant to our interest in 
  wind instruments, if the gravity term is insignificant the equation tells us 
  that as flow speed $q$ increases, so the pressure must drop by an amount 
  proportional to $q^2$. This will be significant when we look at the 
  excitation of reed and brass instruments. 

  For completeness, it is useful to summarise (without derivations) some 
  extensions of Bernoulli's principle with less restrictive assumptions. If the 
  fluid is not incompressible, the next simplest assumption is that it is 
  barotropic: this means that the density is an instantaneous function of the 
  pressure, $\rho(p)$. In that case, we define the enthalpy 

  $$I(p)=\int{\dfrac{dp'}{\rho(p')}} \tag{27}$$ 

  and then Bernoulli's principle for steady irrotational flow (i.e. $\nabla 
  \times \underline{u}=0$) says that 

  $$\dfrac{q^2}{2}+I(p)+gz = \mathrm{~constant.} \tag{28}$$ 

  Next, if we have steady flow with vorticity, we define the quantity 

  $$H=\dfrac{q^2}{2}+I(p)+gz , \tag{29}$$ 

  and then Bernoulli's principle states that $\nabla H$ is perpendicular to 
  both $\underline{u}$ and $\underline{\omega} = \nabla \times \underline{u}$, 
  in other words $H$ is constant along both streamlines and vortex lines. 

  Finally, we can say a little about vorticity and how it evolves. If we take 
  the curl of equation (22) and then make use of some vector identities, we can 
  manipulate it into the form 

  $$\dfrac{D \underline{\omega}}{Dt}=\underline{\omega} \cdot 
  \nabla\underline{u} + \nu \nabla^2 \underline{\omega} \tag{30}$$ 

  where $\nu=\mu/\rho$ is called the ``kinematic viscosity''. This equation 
  tells us about two rather different aspects of vorticity dynamics. First, in 
  the inviscid case ($\nu=0$), it reveals a powerful and unexpected result. If 
  we think about a small ``thread'' of material, on a line element 
  $\underline{\delta l}$, the analysis leading to equation (9) tells us that 
  this material element evolves according to 

  $$\dfrac{D \underline{\delta l}}{Dt}=\underline{\delta l} \cdot 
  \nabla\underline{u} . \tag{31}$$ 

  This is exactly the same equation that governs the evolution of vorticity 
  $\underline{\omega}$ in an inviscid flow field. So if we choose a material 
  line element that lies along a vortex line, we see that the vorticity must 
  evolve with the flow. As it is often stated, ``vortex lines are frozen in the 
  fluid''. So, for example, if the material line gets stretched by the flow, 
  the vorticity will also increase in the same proportion. This is part of the 
  mechanism by which tropical storms and tornadoes can ``spin up'': the rising 
  of hot air stretches the vortex lines, creating faster rotation. 

  This result tells us something important. For any flow started from rest, 
  there must initially be no vorticity anywhere. If the fluid is inviscid, it 
  appears that this must remain true in the subsequent flow. Indeed, there is a 
  result called the Kelvin circulation theorem for an inviscid fluid. If you 
  consider a closed material curve $C$, the integral of the velocity round $C$, 
  called the circulation $K$, is constant in time: 

  $$K=\oint_C{\underline{u} \cdot \underline{dl}} . \tag{32}$$ 

  So how does circulation or vorticity ever arise in a flow? We can see one 
  route for this to happen if we restore the viscosity term of equation (30), 
  and note that the equation then has the character of a diffusion equation. If 
  the first term on the right-hand side were omitted, it would have exactly the 
  same form as the equation governing heat diffusion in a solid body, for 
  example. 

  So viscosity causes vorticity to diffuse through the fluid. An important 
  example of this is illustrated in Fig.\ 2. As explained in section 11.2, when 
  fluid flows over a fixed surface the no-slip boundary condition on the 
  surface leads to the formation of a boundary layer, with a flow profile 
  similar to the one shown in the figure. The figure shows a small material 
  element in the boundary layer. It is subjected to shearing flow, which 
  automatically means that the element has some net rotation --- in other 
  words, it has non-zero vorticity. So boundary layers on solid surfaces 
  generate vorticity, and equation (30) indicates that vorticity may then 
  diffuse out into the flow pattern. 

  \fig{figs/fig-0eeccaa5.png}{\caption{Figure 2. Flow over a fixed surface, 
  with a boundary layer profile. A material element in the boundary layer, 
  shown in dark blue, experiences differential flow speed and hence has net 
  rotation.}} 

  There is another, more vigorous, way that non-zero circulation can appear in 
  a flow. It still involves boundary layers, but rather than diffusion, the 
  process involves vortex shedding. It will be relevant to several aspects of 
  wind instruments, but I will introduce the idea via a different application: 
  how do aeroplanes fly? I will illustrate, with a series of hand-drawn 
  sketches of streamline patterns (which should not be regarded as more than 
  qualitative). 

  Figure 3 shows a schematic airfoil cross-section. When the aircraft first 
  starts to taxi down the runway, the flow pattern in the initial moments may 
  look a little like this sketch. There are two stagnation points, marked by 
  red rings: the streamlines joined to these points are the ones that separate 
  flow that passes above the wing from flow that passes below it. 

  \fig{figs/fig-f78b9c18.png}{\caption{Figure 3. Sketch of flow round an 
  airfoil immediately after motion has started.}} 

  But there is a big problem with this flow. The trailing edge of the wing has 
  a sharp edge, marked by the green ring. The flow speed close to this sharp 
  corner will get very high. Bernoulli then tells us that the pressure must get 
  very low. If this was a propellor blade and the fluid was water, this would 
  be the mechanism of cavitation: when the pressure drops low enough to cancel 
  the hydrostatic pressure, bubbles can be seen streaming from the propellor 
  blades. 

  But for the aircraft wing, something different happens, which is indicated 
  very schematically in Fig.\ 4. The boundary layer separates from the surface 
  and forms a separation bubble. This bubble then separates entirely from the 
  wing, and becomes an independent vortex with a circulation direction that is 
  anticlockwise in the diagram. In order to preserve the total circulation, and 
  thus satisfy Kelvin's theorem, there is equal and opposite net circulation 
  around the wing. 

  \fig{figs/fig-aa65cca0.png}{\caption{Figure 4. The airfoil as in Fig. 3, 
  forming a ``separation bubble'' near the trailing edge.}} 

  The result is as shown (schematically again) in Fig.\ 5. The vortex which was 
  shed is ``left behind on the airfield''. The wing then has just the right 
  amount of clockwise circulation (indicated by the red arrow) that the rear 
  stagnation point is shifted exactly to the sharp trailing edge, as sketched. 
  This eliminates the need for very fast local flow, because (as the name 
  suggests) the flow speed goes to zero at a stagnation point. The effect of 
  the clockwise circulation round the wing is that the air flowing over the top 
  surface is speeded up, while the air flowing over the bottom surface is 
  slowed down. Bernoulli then tells us that the pressure is lower on the top 
  surface than on the bottom surface, and it is this difference of pressures 
  that lifts the aircraft off the ground. 

  \fig{figs/fig-cd4b7009.png}{\caption{Figure 5. The airfoil as in Figs. 3 and 
  4, after a vortex has been shed and left behind. The flow near the trailing 
  edge is now smooth, the there is net circulation round the wing as indicated 
  by the red arrow.}} 

  Any flow interacting with a sharp corner is likely to lead to vortex 
  shedding, and this is relevant to many problems, including some involving 
  musical instruments. Examples might be air flow in and out of tone-holes, or 
  flow in and out of the f-holes of a violin near the Helmholtz resonance 
  frequency. More immediately important for this chapter, the excitation 
  mechanism of flute-like instruments involves these ingredients, as we will 
  see in section 11.6. 