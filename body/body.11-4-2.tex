  We want to use a measured input impedance to determine values of modal 
  parameters, based on the modal decomposition formula derived in section 
  11.4.1: 

  $$Z(\omega) = i \omega \sum_m{\dfrac{A_m}{\omega_m^2+ i \omega \omega_m/Q_m 
  -- \omega^2}} \tag{1}$$ 

  where $\omega_m$ is the (angular) frequency and $Q_m$ the Q-factor of mode 
  $m$. The coefficient $A_m = p_m^2(0)$, where $p_m(x)$ is the pressure mode 
  shape for a tube closed at the mouthpiece end where $x=0$. (The mode shape 
  must be normalised in a particular way, explained in section 11.4.1.) 

  As we showed in section 10.5.1, each single term from the modal summation 
  (1), when plotted in the Nyquist plane (real part of $Z$ versus imaginary 
  part of $Z$), should give a circle. Figure 1 shows a typical measured input 
  impedance function — this particular example is for a trombone, which we will 
  study in section 11.5. A small section near one of the peaks has been 
  highlighted in red. A Nyquist plot of that section gives the red curve in 
  Fig.\ 2, and it does indeed look plausibly like an arc of a circle. 

  \fig{figs/fig-1af0bd4e.png}{Figure 1. The measured input impedance of a 
  trombone, showing in red a selected frequency range surrounding a single 
  peak.} 

  \fig{figs/fig-3c68e0e0.png}{Figure . Example of a circle, fitted to the 
  Nyquist plot of impedance over the small frequency range shown in Fig. 1. The 
  blue star shows the point on the circle that would lie at the origin if we 
  did not have a shift caused by the effect of modes outside the chosen 
  frequency range. The blue line indicates the phase of the fitted modal 
  coefficient $A_m$: this blue line would be horizontal if the coefficient were 
  real.} 

  The fitting procedure now proceeds in two stages, both making use of the 
  Matlab function “fminsearch”. The first stage is purely geometrical: the 
  circle most closely approximating the red curve is found. The outputs of this 
  stage are the $(x,y)$ coordinates of the centre, and the circle radius. The 
  result, for our example, is the black dashed circle shown in Fig.\ 2. A 
  graphical comparison like this allows the user to be sure that there really 
  is only a single major peak in the selected frequency range — otherwise the 
  circle would be distorted. 

  The second stage is to best-fit the parameters of a single term of equation 
  (1) to the measured complex data, with the constraint that the result must 
  lie on the circle found in the first stage. However, we need to make some 
  allowance for the effect of all the other modes. Provided their resonance 
  frequencies are well outside the frequency range we have chosen, the net 
  effect of those contributions should be approximately a (complex) constant. 
  The outputs of this stage are this complex constant, plus $\omega_m$, $Q_m$ 
  and $A_m$ — but $A_m$ is allowed to be complex. The blue line in Fig.\ 2 
  indicates the fitted phase angle of $A_m$, and the blue star shows the point 
  on the circle that would lie at the origin without the effect of the other 
  modes. 

  Once this process has been carried out for all sufficiently strong peaks, 
  equation (1) can be used to reconstruct $Z$ from the modal parameters. But at 
  this stage we ignore the phases of the coefficients: if the pressure mode 
  shapes of the tube are real, as we would expect provided the damping is 
  fairly light, then the coefficients $A_m$ should all be positive real 
  numbers, because each is the square of a modal amplitude at the measurement 
  position. By this means, any errors in the measured phase of $Z$ are 
  minimised. 

  What we actually need for our simulation algorithm is the impulse response 
  corresponding to $Z$. Once we have our modal decomposition, there is a simple 
  and explicit formula for that impulse response. Each term in equation (1) 
  describes the frequency response function of a single degree of freedom 
  oscillator (like a mass on a spring). We derived the impulse response for 
  such an oscillator back in section 2.2.8. So we know that our impulse 
  response $g(t)$ is given by 

  $$g(t) \approx \sum_m{A_m \cos(\omega_m t) e^{-\omega_m t/2Q_m}} \tag{2}$$ 

  where, again, we use values of $A_m$ which are real and positive, ignoring 
  the fitted phase. 

  Now we can describe the steps of a mode-based simulation. After a given 
  number of time steps, we have available the stored history of the volume flow 
  rate $v(t)$ into the reed mouthpiece. The resulting pressure inside the 
  mouthpiece is then given by a convolution integral of the flow rate with the 
  impulse response we have just found: 

  $$p(t) = \int_0^t{v(t-\tau) g(\tau) d \tau} . \tag{3}$$ 

  The lower limit of the integral is written as $\tau=0$ here, on the 
  assumption that the simulation started at time $t=0$. In any computer 
  simulation we are really working in discrete time, so this integral is 
  approximated by a sum. More important, we don’t actually need to do the 
  potentially time-consuming convolution. Instead, as we saw back in section 
  9.5.2, we can use a recursive IIR filter to represent each term in the 
  summation. This speeds things up, and means that we don’t in fact need to 
  store the history of $v(t)$. 

  In this way, at a given time step in the simulation we calculate a new value 
  for the pressure. If we are using the quasi-static reed model, all we have to 
  do now is read off the corresponding new value of the volume flow rate from 
  the reed characteristic, equation (9) from section 11.3.1. We use this new 
  value to update all the IIR filters, and move on to the next time step. If we 
  are taking account of the reed’s resonance, we have another IIR filter 
  representing the reed motion, so we update that at each time step and then 
  deduce the new value of volume flow rate as described at the end of section 
  11.3.1. 