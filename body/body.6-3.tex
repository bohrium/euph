

  Probably the most familiar result of psychoacoustic testing is the chart 
  shown in Fig.\ 1. This shows contours of equal perceived loudness, for 
  listening to sine waves. The vertical axis shows the amplitude of the sound 
  pressure, plotted on a decibel scale. The zero level is defined by an 
  international convention, and is roughly the level of the quietest sine wave 
  you can hear. Now notice the numbers on the vertical scale: it covers a range 
  of 140 dB, and if you remember that 20 dB corresponds to a change in 
  amplitude by a factor of 10, this means that your full range of hearing (for 
  sine waves, remember) covers 7 factors of 10, so a total range of a factor of 
  10,000,000. 

  \fig{figs/fig-2c8fcfa4.png}{Figure 1. Contours of equal loudness for 
  sinusoidal waveforms. Image: Lindosland, Public domain, via Wikimedia 
  Commons} 

  The lowest curve in the chart has a simple interpretation. It is labelled 
  ``threshold'', which is probably self-explanatory: it indicates the quietest 
  sine wave that can be detected at each frequency, by a typical healthy young 
  adult with normal hearing. ``Young'' because as we get older our hearing 
  deteriorates, most obviously at higher frequencies. At age 70, I can no 
  longer detect a sine wave above about 10 kHz at any amplitude. Your personal 
  version of this threshold curve is one of the things that is commonly 
  measured if you are given an audiometry test to assess the state of your 
  hearing. 

  The highest curve in the plot is only ``estimated'' at frequencies above 1 
  kHz: this is because exposure to sounds that loud can cause damage to the 
  listener's ears, so that it is not ethically acceptable to collect the data 
  to complete the measured curve or to map out curves at higher levels. 
  Somewhere around a sound pressure level of 120--140 dB, we would reach the 
  ``threshold of pain'': the level at which a sine wave becomes physically 
  painful to the listener. This does not mean that sounds this loud are never 
  heard, of course. Notoriously, loud sounds such as those from aircraft 
  engines, firearms or loud amplified music commonly cause hearing damage 
  unless precautions are taken. ``Ear defenders'' are now routinely worn by 
  people working in noisy environments --- but not usually by party-goers or 
  members of rock bands. 

  The curves are labelled in terms of a unit of subjective loudness called the 
  phon. The numerical value associated with each curve is defined to be the 
  physical sound pressure level in decibels where the curve passes through 1 
  kHz. The curious wiggles in the curves are caused by resonances in the ear 
  canal and the middle ear: as a result of these resonances, sine waves at some 
  frequencies are amplified before they reach the cochlea, while at other 
  frequencies they are somewhat suppressed. 

  The chart in Fig.\ 1 is often interpreted as saying that the human ear is 
  most sensitive for frequencies around 1--5 kHz. Well, this is true in one 
  sense, but it is the opposite of the truth in another sense. If you are 
  thinking about detecting very quiet sounds, then it is true that your ear is 
  at its most sensitive in this frequency range. But notice that this is also 
  the range where the curves are widest apart: so in this range your ear is at 
  its least sensitive to changes in loudness. In the context of listening to 
  music, sensitivity to changes can be just as important as absolute 
  sensitivity to whether there is a sound at all. For example, where the curves 
  are wide apart a performer would need to make a bigger physical change in 
  order that the listener hears a crescendo. 

  It is useful to have an idea of how the charts in Fig.\ 1 were determined. In 
  a typical experiment, a test subject would be sitting in a quiet place, with 
  headphones on. They would be presented with a ``reference'' tone, perhaps at 
  1 kHz, at a particular level. They would then hear a second tone at a 
  different frequency, which would start at a random level. Their task is 
  adjust the level of this second tone until the loudness is judged to match 
  the reference tone. This procedure is then repeated many, many times. The 
  frequency of the test tone will be varied to cover the entire audible range, 
  including repeat runs with the same frequency to test for consistency of 
  judgement. The level, and perhaps the frequency, of the reference tone will 
  be varied. Many different test subjects will take the same test --- but will 
  be presented with the individual loudness-matching tasks in a different 
  random sequence. All the results are stirred together in the computer, and 
  some statistical tests performed to tell the experimenter when they have 
  enough results that the answer is reliable within a specified limit. 

  This kind of testing is subject to many traps and pitfalls. We can see a 
  strong hint of this in Fig.\ 1. The red curves show the current international 
  standard, but the blue curve shows an alternative version of one of these 
  curves, which used to be the international standard before a new set of 
  trials were carried out. They differ by a remarkably large amount, up to 
  about 20 dB. Both sets of experimenters knew what they were doing, and 
  followed a very similar procedure, so how come they reached an answer that 
  was so different? 

  There are many factors that can disturb the results of an experiment like 
  this. We will give a few examples, but for a full discussion see Moore [1]. 
  Presenting the sounds via headphones is not the same as presenting them with 
  loudspeakers. Even different kinds of headphones will give different results. 
  The reasons are to do with diffraction of the sound field around the 
  subject's head and external ears. The position at which the reference sound 
  pressure level is measured can make a significant difference: just moving 
  from a few millimetres outside the ear canal to a few millimetres inside it 
  is important. Sensitivity to low-frequency sounds presented through 
  headphones can be affected by sounds generated by your own body, especially 
  your heartbeat and blood flow. 

  Even less obviously, there are subtle sources of bias that can arise from 
  details of the testing procedure: here is an example. For a given level of 
  the reference tone, the test tone is initially presented at a random level, 
  chosen from within a certain range using a random number generator. But it 
  has been discovered that if this range of random amplitudes is not 
  symmetrical above and below the eventual ``correct'' level which matches the 
  loudness of the reference tone, there will be a bias in the results of the 
  test. Of course, you don't know where the matching level will be when you 
  start the experiment, so the testing has to be done in an iterative way. 
  Choose a range, find out what the listeners judge to be the matching level, 
  then if this turns out not to lie at the centre of the random range, you have 
  to repeat the whole experiment with a new range. Then keep repeating this 
  until you have results based on a range which is in fact symmetric about the 
  final chosen level. 

  All of this complication, and we are still only talking about hearing sine 
  waves! Normal sounds, and real music, involve far more complicated sounds. 
  They will involve multiple frequency components, and they will be varying in 
  time in a complicated way. We will start to investigate how to deal with this 
  in the next section. 



  \sectionreferences{}[1] Brian C. J. Moore; ``An Introduction to the 
  Psychology of Hearing'', Academic Press (6th edition 2013). 