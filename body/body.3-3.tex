

  Now we know how ideal beams behave we are ready to look at their serious 
  musical relatives, the xylophone and marimba. Each note of these instruments 
  has a bar of wood or metal, supported in some way at points roughly a quarter 
  of the way in from each end. When you tap the bar, you hear a sound that 
  should be perceived as having a definite musical pitch. The bars for 
  different notes have different sizes: longer bars for the lower notes, 
  shorter ones for higher notes. 

  Toy xylophones use uniform metal bars very much like the idealised theory we 
  saw in section 3.2. Is a toy xylophone ``musical''? Well, up to a point. The 
  notes of these toys have rectangular metal bars, which will fall somewhere 
  between the idealised ``beam'' and ``plate''. The various sound examples in 
  the previous section showed that a plate can easily become extremely 
  unmusical, but a beam is better. The sound of both was improved by using a 
  softer hammer. The explanation of all this is that none of the overtone 
  frequencies are ``tuned'' to be in harmonic relations with each other, so 
  their sounds tend to detract from any sense of definite pitch based on the 
  fundamental. 

  Actually, a perceived pitch need not be based on a fundamental frequency 
  which is physically present in the sound. A mixture of frequencies with 
  ratios 2:3:4, for example, can often produce a sensation of a pitched note at 
  the frequency that would have been ``1'' in this series. This phenomenon is 
  known by several names: the ``missing fundamental'', or ``residue pitch'', 
  for example. We will meet examples later, starting in the next section when 
  we look at church bells. 

  Getting back to the sound of a vibrating beam, there are two things we can do 
  to encourage a listener to hear a definite pitch. The simple approach is to 
  arrange things so that one overtone, usually the fundamental, is much louder 
  than any of the overtones. This can be done by using a soft hammer, but the 
  grown-up xylophone and marimba enhance this effect by using resonators. As 
  can be seen in Fig.\ 1, a suitably tuned tube is placed under each bar, to 
  enhance the sound radiated at the fundamental frequency. We will see in 
  section 4.3 how such resonators work. You can directly hear the effect of 
  these resonators in the vibraphone, which has a set of rotating vanes in the 
  tubes so that the resonant effect is modulated in and out in a regular way. 

  \fig{figs/fig-96c03a4e.png}{} 

  \fig{figs/fig-59dac254.png}{} 

  There is a second approach to making the sound more ``musical'': the 
  instrument maker can do some simple vibration engineering to make marimba 
  bars just a little more harmonic. They do this by removing wood from the 
  underside of the bars so that the thickness varies along the length of the 
  bar: see some examples in Fig.\ 2. Actually, the maker has several reasons 
  for doing this ``undercutting'' of the bars. Uppermost in their mind, 
  probably, is simply that they need to tune each bar to the precise pitch 
  called for by each separate note. Second, especially for the lower notes on a 
  marimba, vigorous undercutting allows you to get away with a shorter bar than 
  you would otherwise need, with obvious advantages in weight, portability and 
  usage of expensive raw material. 

  \fig{figs/fig-cab72b0e.png}{\caption{Figure 2. A selection of marimba bars, 
  showing the undercutting for tuning.}} 

  But there is a third thing a skilled maker can do by careful undercutting: 
  they can bring the ratio of the first two natural frequencies of the bar 
  close to a whole number. Recall that in the ideal beam this ratio was 2.76. 
  We will see shortly that symmetrical undercutting around the middle of the 
  bar allows you make this ratio bigger, but not easily to make it smaller. So 
  well-tuned marimba and xylophone bars have this ratio adjusted to be close to 
  3 or, much more commonly, 4. The examples shown in Fig.\ 2 all have enough 
  undercutting to produce 4:1 tuning. 

  So how do the sounds compare, from these different options? Sound 1 below is 
  a repeat of Sound 5 from section 3.2, with the frequency ratios of an ideal 
  beam, and modal Q-factors of 100. Sound 2 is the result of rounding all these 
  ratios to the nearest whole number, so that all overtones are perfect 
  harmonics. Sound 3 is based on measured frequency ratios from a xylophone bar 
  in which the second mode was tuned to three times the fundamental, and Sound 
  4 is a similar thing but based on measured frequency ratios from a marimba 
  bar in which the second mode was tuned to four times the fundamental. The 
  ratios for the xylophone bar are 1.00, 3.00, 6.16, 10.29, 14.01, 19.66, 
  24.02. For the marimba bar, they are 1.00, 3.92, 9.24, 16.27, 24.22, 33.54, 
  42.97. In both cases, the measured frequency ratios do not seem to be 
  deliberately tuned beyond the second mode. 

  To my ear, the two cases based on measured frequencies sound better than 
  Sound 2 with the ``perfect'' harmonics. The non-harmonic higher overtones 
  seem to give an interesting percussive edge to the sound which is lacking in 
  Sound 2. But I hear something else interesting: when comparing Sound 3 with 
  Sound 4 in rapid succession, the actual pitches of the two scales can sound 
  different. But the fundamental frequencies are exactly the same in both 
  cases, and the second mode is tuned to a near-harmonic in both cases. My 
  brain seems to be responding, in some way, to the 4:3 ratio of the two second 
  frequencies, plus the fact that all the higher overtones tend to follow in a 
  similar ratio. Recall from section 2.3 that 4:3 is the frequency ratio for an 
  interval of a fourth in musical language, and this is indeed the ``pitch'' 
  difference that I sometimes hear when comparing these two examples. 

  \aud{auds/aud-6461f559-plot.png}{\caption{Sound 1. Synthesised scale based on 
  frequencies of an ideal beam, with Q=100;}} 

  \aud{auds/aud-be7d58f1-plot.png}{\caption{Sound 2. Synthesised scale based on 
  frequencies of an ideal beam rounded to the nearest whole number, with 
  Q=100;}} 

  \aud{auds/aud-784cec10-plot.png}{\caption{Sound 3. Synthesised scale based on 
  measured frequency ratios from a xylophone bar, with Q=100;}} 

  \aud{auds/aud-15cfc71d-plot.png}{\caption{Sound 4. Synthesised scale based on 
  measured frequency ratios from a marimba bar, with Q=100;}} 

  To begin to understand how the tuning process is done, we can use a method 
  based on Rayleigh's principle, introduced in the previous section, to give a 
  graphical representation of what a maker can expect to do by adjusting the 
  pattern of thickness. As explained in the next link, from the mode shapes (as 
  plotted in Fig.\ 1 in the previous section) we can compute a function $G(x)$ 
  which tells you the sensitivity of the natural frequency of that mode to any 
  small change in the thickness pattern. Examples for the first three modes of 
  a uniform beam are plotted in Fig.\ 3. If a little wood is removed in a 
  particular area of the beam, this function tells you how much that mode 
  frequency will change. Removing wood where $G$ is negative will make the 
  frequency go down; removing wood where $G$ is positive will make it go up. 

  \fig{figs/fig-6ade9083.png}{\caption{Figure 3. Sensitivity of modal frequency 
  to thickness adjustment, for the first three modes of a free-free beam. The 
  mode shape is shown as a blue dashed line, the sensitivity function $G(x)$ as 
  a red line.}} 

  For the lowest mode, the function $G$ looks rather like the mode shape 
  itself. It is easy to understand why the curve has to be roughly this shape. 
  Recall that the vibration frequency is determined by the balance between 
  bending stiffness and inertia. Removing wood near the ends, where $G$ is 
  positive, means that you reduce the inertia. You also reduce the bending 
  stiffness, but there is no bending moment near a free end of the beam, so the 
  bending stiffness doesn't matter. So the frequency goes up, because the 
  dominant effect is a reduction of inertia. This mode has its maximum bending 
  moment in the centre. Removing wood there will reduce the bending stiffness, 
  and this effect will outweigh the reduction of inertia (mainly because 
  bending stiffness is proportional to $h^3$ but inertia is only proportional 
  to $h$, so bending stiffness is more sensitively affected by a thickness 
  reduction). So the frequency goes down. 

  The other plots in Fig.\ 3 show that the function $G$ does not usually look 
  at all like the mode shape: the first mode is an exception. For the second 
  and third modes, the only place $G$ goes positive is near the ends. 
  Everywhere else $G \le 0$ so that wood removal will reduce the frequency. But 
  mode 2 has $G=0$ at the centre, so removing wood near that point will have 
  very little effect on this mode frequency, whereas we have already seen that 
  it will reduce the frequency of mode 1. That is the main reason that normal 
  undercutting will increase the frequency ratio between modes 1 and 2. 

  The sensitivity curves plotted here are only strictly relevant to a small 
  change in thickness, starting from a uniform beam. The amount of undercutting 
  on normal marimba bars is far too large for these plots to tell the whole 
  story of the tuning process. However, the same approach can be applied for 
  each step of the undercutting process: given the mode shapes at a given 
  stage, plots like these could be generated to guide the next stage. In 
  practice, although the total undercutting can be quite drastic, the general 
  character of the mode shapes does not change very much, and so the 
  sensitivity functions remain roughly the same shape throughout. Some examples 
  can be seen in Fletcher and Rossing [1]: see the chapter ``Mallet percussion 
  instruments''. 



  \sectionreferences{}[1] Neville H Fletcher and Thomas D Rossing; ``The 
  physics of musical instruments'', Springer-Verlag (Second edition 1998) 