

  A lot of the technical literature on nonlinear systems uses an approach that 
  allows some geometrical insight into the behaviour, at least for systems with 
  smooth nonlinearity. This approach is based on an idea called phase space. We 
  will give an introduction in this section, based on the simplest possible 
  example. Suppose we have a nonlinear oscillator with a single mass, or more 
  generally a single degree of freedom. For any such system, if you specify the 
  position and the velocity at the moment when you set things off, that is 
  sufficient information to determine the subsequent free motion. (The 
  technical reason is that the motion will obey some version of Newton’s law, 
  and so the governing equation will be of second order, requiring two initial 
  conditions to determine a unique solution.) 

  We can picture this in a two-dimensional plot, with position and velocity as 
  the two axes, as sketched in Fig.\ 1. Our initial position and velocity, at 
  time $t=0$ say, is represented by a dot in this plane. The subsequent motion 
  of the system, for $t>0$, can then be plotted as a trajectory starting from 
  that point. If we think of running time backwards, we could create the other 
  half of the trajectory, coming from times $t<0$ and reaching our point at 
  $t=0$. But we could have chosen a different position and velocity for our 
  initial spot, and we could have done that anywhere in this plane. So there is 
  a trajectory passing through every point, and there is only one. To put that 
  in different words, the diagram must be filled up with a pattern of 
  trajectories, and they cannot cross. This diagram is called the phase plane, 
  or more generally phase space. The pattern of trajectories is the phase 
  portrait of the particular system we choose to study. It captures in one 
  picture every possible motion of the system. 

  \fig{figs/fig-23fcd867.png}{\caption{Figure 1. Sketch of a trajectory in the 
  phase plane}} 

  A good example to illustrate the idea is the pendulum discussed in section 
  8.2.1 and shown again here in Fig.\ 2. Its angle of swing is described by an 
  angle $\theta$ (Greek ``theta'', remember). We will use this angle for the 
  horizontal axis of our phase portrait, and its rate of change $d\theta / dt$ 
  as the corresponding ``velocity'' on the vertical axis. 

  \fig{figs/fig-80063d6c.png}{\caption{Figure 2. A pendulum}} 

  The phase portrait for the pendulum appears in Fig.\ 3. This plot doesn't 
  show arrows on the trajectories, but it is easy to work out where the arrows 
  need to point. When we are above the middle of the plot, that means that the 
  velocity $d\theta / dt$ is positive. In other words, the angle $\theta$ is 
  increasing, so all the arrows would point towards the right. In the lower 
  half, $d\theta / dt$ is negative so $\theta$ is decreasing and all the arrows 
  would point to the left. 

  \fig{figs/fig-29844698.png}{\caption{Figure 3. Phase portrait of a pendulum}} 

  The values marked on the horizontal axis may be confusing if you aren't used 
  to thinking of angles in radians, but the code is very simple. One complete 
  revolution of the pendulum takes $\theta$ through an angle of $2\pi$ radians. 
  The value $\theta=0$ corresponds to the pendulum hanging vertically 
  downwards. The values $-2\pi$, $2\pi$ and so on are simply further 
  representations of the same position, after a whole number of rotations. The 
  values $\theta = -3\pi,-\pi,\pi,3\pi$ and so on all indicate the 
  vertical-upwards position. 

  Figure 3 shows trajectories of two different shapes. Along the middle of the 
  plot there are closed-loop trajectories, whereas in the upper and lower 
  portions there are undulating lines that cross the entire plot. These 
  correspond to the two types of motion you should expect for a pendulum. The 
  closed loops show oscillation, back and forth about the position at the 
  bottom of the swing. The undulating lines in the upper part (with ``arrows of 
  time'' pointing to the right, remember) show the pendulum continuously 
  rotating, in the direction of increasing $\theta$ and so anticlockwise in the 
  view of Fig.\ 2. The undulating lines in the lower half have $\theta$ 
  decreasing, so they describe clockwise rotation. These three types of motion 
  are summarised in Fig.\ 4. 

  \fig{figs/fig-70b6a0d6.png}{\caption{Figure 4. The three types of 
  qualitatively different motion of the pendulum, indicated in the phase plane. 
  They are separated by curves emanating from saddle points. Such a curve is 
  called a separatrix.}} 

  The curves that separate motion of different types are obviously important. 
  These curves are trajectories with a very special property, but to see what 
  that is we should first back off a little and address a more general issue. 
  There are a few points in Fig.\ 3 which are special, because they correspond 
  to possible equilibrium positions of the pendulum. The pendulum has two 
  equilibrium positions. The obvious one is at $\theta=0$, where the pendulum 
  can hang vertically downwards without any motion. But there is another, where 
  the pendulum sits vertically upwards. In principle the pendulum could balance 
  in that position without moving, but in practice you don’t see it doing so 
  because this equilibrium position is unstable: you only have to tweak the 
  position away from the perfect vertical by the smallest amount, and it will 
  topple and start to swing. The mathematics behind stability and instability 
  of the pendulum is explained in the next link. 

  These equilibrium positions each appear multiple times in Fig.\ 3, because we 
  have “unwrapped” the angle $\theta$ to allow for the possibility of rotating 
  motion. The stable equilibrium corresponds to the point at the centre of each 
  of the sets of closed loops: at $\theta=0,2\pi,4\pi$ and so on, on the 
  mid-line of the plot where $d\theta/dt=0$ because the pendulum is stationary. 
  The unstable equilibrium also occurs on that mid-line, at the positions 
  $\theta=-\pi,\pi$ and so on. These are the only points in the plot where we 
  seem to have trajectories that cross. But I said earlier that this couldn’t 
  happen, so what is going on? We will see in a moment that they don’t in fact 
  cross, but they are important for another reason. 

  The behaviour of the phase portrait in the vicinity of an equilibrium point 
  can be analysed — at least for a system like this one where the nonlinearity 
  is smooth. The details are explained in the next link, but the main result 
  can be shown in pictures. For an undamped system like our pendulum, there are 
  only two types of behaviour that can occur near an equilibrium point (also 
  called a singular point by the mathematicians). The two corresponding phase 
  portraits are shown, in a rather general form, in Fig.\ 5. On the left, we 
  see what is called a centre: it features a set of closed loops, and describes 
  oscillation around the equilibrium point. Behaviour roughly like this will 
  always occur near a stable equilibrium point of any undamped system. On the 
  right we see a saddle point, which is the generic behaviour near any unstable 
  equilibrium point. 

  \fig{figs/fig-aa44ba08.png}{} 

  \fig{figs/fig-c7305961.png}{} 

  We can immediately recognise these two patterns in Fig.\ 3. Now look closely 
  at the right-hand plot of Fig.\ 5, for the saddle point, and pay close 
  attention to the pattern of arrows. The plot does indeed seem to show a pair 
  of trajectories crossing in the middle, but the arrows tell a different 
  story. One of these lines has inward-pointing arrows either side of the 
  “crossing”, while the other has outward-pointing arrows. In the context of 
  our pendulum, we can describe what is happening. The two inward-point arrows 
  indicate the trajectories in which you propel the pendulum upwards with just 
  enough energy that it approaches the unstable equilibrium, but never quite 
  gets there. You could do that either from the left-hand side or from the 
  right-hand side: these are the two inward-pointing segments. The 
  outward-pointing pair show the converse behaviour: the pendulum toppling 
  either to the right or to the left, after an infinitesimally small nudge away 
  from the unstable equilibrium. So what we have is four trajectories, not two, 
  and there is no crossing. The equilibrium point at the centre is a trajectory 
  all by itself. 

  Now look back at Fig.\ 4: the curves dividing the different types of 
  behaviour are precisely the trajectories we have just been talking about, 
  coming into and out of the saddle points marking the unstable equilibrium of 
  the pendulum. Such a trajectory is known in the jargon of the subject as a 
  separatrix, precisely because they always have this property of separating 
  behaviour with different qualitative descriptions. We begin to see the power 
  of the phase portrait: it gives a way to encapsulate everything the system is 
  capable of doing, and it has a built-in method for sorting out regions with 
  interestingly different behaviour. 

  We can take our pendulum example a little further. What happens if we 
  introduce some damping to our pendulum, so that the motion will eventually 
  die away? The phase portrait must change to reflect this fact. Nothing much 
  happens near the saddle points: they are still unstable, and the qualitative 
  behaviour stays the same. But the centre that previously described the 
  behaviour near the stable equilibrium turns into a stable spiral. The 
  corresponding phase portrait is illustrated in general form in Fig.\ 6: it 
  should be pretty much what you were expecting, showing trajectories 
  spiralling in towards the equilibrium point. 

  \fig{figs/fig-c7e4f900.png}{\caption{Figure 6. Phase portrait of a stable 
  spiral}} 

  A portion of the phase portrait of the damped pendulum is shown in Fig.\ 7. 
  Only one ``cell'' of the repeating pattern is shown, because you need to be 
  able to see the pattern clearly enough to follow some of the trajectories. It 
  remains true that all arrows in the upper half point to the right, and all 
  arrows in the lower half point to the left. Now look carefully to see what 
  has happened to the separatrix coming out of the saddle point in the middle 
  of the left-hand side of the plot. Instead of heading towards the next saddle 
  point, as it did in Fig.\ 3, it spirals inwards. That is exactly what we 
  expect on physical grounds. Without damping, if you nudge the pendulum very 
  gently away from the unstable equilibrium, it has just enough energy to do 
  one complete revolution and come (more or less) back to the vertical 
  position. But with damping, it loses energy as it moves so that it doesn't 
  have enough to get back to the vertical. It swings back and forth with 
  gradually decreasing amplitude. 

  \fig{figs/fig-16320d6a.png}{\caption{Figure 7. Portion of the phase portrait 
  of a pendulum with some damping added}} 

  So what about the claim that the separatrices (that's the plural of 
  ``separatrix'') divide regions with different qualitative behaviour? Well, 
  you have to think rather carefully about what Fig.\ 7 would look like if we 
  extended it to include more ``cells'' of the repeating pattern in the phase 
  plane. There is a string of saddle points, and each one issues a separatrix 
  heading upwards and rightwards in the top half of the plot. But these no 
  longer look as if they join up as they did in Fig.\ 4: instead, they nest 
  inside each other. In the region of the phase plane between two successive 
  separatrices of this type, you might describe the motion as ``doing two 
  complete anticlockwise rotations before spiralling in''. The next region 
  above this would do three complete rotations, then four, and so on. There is 
  a matching set of nesting separatrices in the lower half of the phase plane, 
  and of course those mark out regions with two, or three, or four clockwise 
  rotations before spiralling in. 

  Finally, it is interesting to return to the Duffing equation, discussed in 
  section 8.2, and look at its phase portrait. The mathematical details are 
  given in the next link, but the key thing to remember is that this equation 
  describes an oscillator with a spring that has a cubic term added to the 
  linear term describing a linear Hooke's law spring. For the case with a 
  hardening spring, the phase portrait is rather boring: the only equilibrium 
  point is at the origin in the phase plane, and the whole plane is filled with 
  closed loops characteristic of a centre. Figure 8 shows a plot. 

  \fig{figs/fig-ae432bad.png}{\caption{Figure 8. Phase portrait of the Duffing 
  equation, with a positive linear spring and a hardening nonlinear term.}} 

  But if we allow the linear component of the spring to have negative 
  stiffness, something dramatic happens. There are now three equilibrium 
  points: a saddle point and two centres. The resulting phase portrait is 
  illustrated in Fig.\ 9. The two centres represent stable equilibrium, but the 
  saddle point is an unstable equilibrium. Now, you may think this is a silly 
  example because negative springs are non-physical. You would be right, if you 
  are thinking purely about mechanical systems. But if you allow the system to 
  include magnets, you can easily make something that behaves like this. 
  Imagine a pendulum with a magnet at the bottom, swinging over a table which 
  has a second, repelling, magnet fixed to it, right underneath the pivot of 
  the pendulum. The magnetic repulsion acts like a negative spring, and if it 
  is strong enough the pendulum can't hang vertically downwards. But it can sit 
  either to the left or to the right, and it could oscillate about either of 
  those positions: just what Fig.\ 9 suggests. 

  \fig{figs/fig-8ca64f88.png}{\caption{Figure 9. Phase portrait of the Duffing 
  equation with a negative linear spring and a hardening nonlinear term}} 

  We can learn something interesting about the behaviour in Figs.\ 8 and 9 if 
  we plot a graph of the position of the equilibrium points against the 
  stiffness of the linear component of the spring. The result is plotted in 
  Fig.\ 10. When the spring stiffness is positive, there is only one solution 
  as we saw in Fig.\ 8. When it is negative there are three, as we saw in Fig.\ 
  9. The outer pair are stable, shown in solid lines. The middle one, shown as 
  a dashed line, is the unstable equilibrium associated with the saddle point. 
  At the centre of the plot we see the transition between the two. This pattern 
  is called, reasonably enough, a pitchfork bifurcation. Plots like Fig.\ 10 
  are called bifurcation diagrams, and they are often useful to summarise the 
  different regimes that a nonlinear vibrating system can show. 

  \fig{figs/fig-4f33b713.png}{\caption{Figure 10. Bifurcation diagram for the 
  Duffing equation. Stable equilibria are shown as solid lines, unstable 
  equilibria by a dashed line.}} 

