

  We have looked in some detail at the vibration behaviour of stringed 
  instrument bodies: but how important are all these details for the sound of 
  the instrument? It is time to turn our models into something we can listen 
  to. For the violin, that will involve understanding what happens when you bow 
  a string, which will take us into the tricky territory of non-linear systems. 
  We will put this off, until chapter ?. But for plucked-string instruments 
  like the guitar and the banjo we can get a long way with linear ideas. 

  What happens when you pluck a string? The simplest description is that the 
  player pulls the string to one side (with a plectrum, fingernail or the flesh 
  of a finger), then lets go. The string is then left to its own devices, to 
  vibrate freely and to feed some of its energy into the instrument body. When 
  pulled aside, the string will take up a triangular shape as sketched in Fig.\ 
  1. The player can choose the plucking point along the string, and the 
  orientation of the pluck. Otherwise, the only thing likely to vary is the 
  sharpness of the corner: a small, sharp plectrum will make a very sharp 
  corner, the flesh of a finger will make a more rounded one. 

  \fig{figs/fig-d726d9ca.png}{\caption{Figure 1. Sketch of a string pluck: the 
  string is pulled to this triangular shape by a force (the black arrow), then 
  it is released to vibrate freely.}} 

  The very simplest model is to assume a perfectly sharp corner, and an ideal 
  string without damping. We already know the mode shapes and natural 
  frequencies for this ideal string from section 3.1.1. For this simple system 
  there are three ways we can find the string motion following the pluck. They 
  are all useful in their different ways so we will briefly summarise all 
  three. 

  First, we can say that after the string has been released, the ensuing free 
  motion must consist of a linear combination of the modes, each vibrating at 
  its own natural frequency. This statement is, after all, true of any free 
  motion of any linear system. All we need to do is find the amplitude and 
  phase of each modal contribution. This can be done by expressing the initial 
  triangular shape as a Fourier series: the slightly messy mathematical details 
  are given in the next link. 

  One virtue of this approach is that it does not rely on the particular 
  assumed shape of the pluck: it could be applied to any other initial 
  conditions, such as a rounded corner from a finger-flesh pluck. The method 
  would be followed in the same way, based on calculating the Fourier series 
  representation of the initial shape. The effect of rounding off the sharp 
  corner would be to reduce the amplitudes of the higher Fourier components. 
  You need short wavelengths, corresponding to high-frequency modes, to 
  represent a sharp corner, but a rounded corner will have a natural cutoff at 
  the wavelength matching the curvature of the corner, above which the Fourier 
  coefficients will become small. The result is a familiar one: the sound of a 
  finger-flesh pluck is less bright than a plectrum pluck because it contains 
  less energy at high frequency. 

  We can learn something interesting about the sound of plucked strings from 
  the result of the calculation in the previous link. We can't calculate 
  `sound' as such from this model string with rigidly fixed ends, but we can 
  calculate the waveform of force exerted by the vibrating string on the 
  bridge. This is the driving force for body vibration, and so is closely 
  related to the sound that an instrument would make. The amplitude of the 
  $n$th harmonic of this force signal is proportional to the mode shape at the 
  pluck point, divided by the harmonic number, and the mode shapes are simply 
  sinusoidal as plotted in Fig.\ 1 of section 3.1. This amplitude is plotted in 
  Fig.\ 2, against harmonic number $n$ and relative pluck position $a/L$ (where 
  $L$ is the length of the string and $a$ is the plucking distance). 

  \fig{figs/fig-8320cc8e.png}{\caption{Figure 2. The first 20 harmonic 
  amplitudes for the waveform of force applied to the instrument bridge by an 
  ideal plucked string, for different values of the plucking point. The 
  plucking point is expressed as a relative fraction of the string length, so 
  that a value 0.5 means that the string is plucked at its mid-point, while a 
  small value means that the string is plucked very close to the bridge.}} 

  This plot reveals several interesting features. For every plucking position, 
  the fundamental ($n=1$) has the largest amplitude. But the value of that 
  fundamental amplitude, for a fixed plucking force, is higher when plucking 
  far from the bridge, and gets progressively lower as the pluck point gets 
  nearer to the bridge. Every guitarist is familiar with the change in loudness 
  and tone quality that results from plucking closer to the bridge: a 
  synthesised example of the effect will be given at the end of this section. 

  There is another factor contributing to this change of sound when the 
  plucking point is varied. Figure 2 shows that the variation of amplitude with 
  $n$ is different for each plucking point, leading to the set of curving 
  ridges which dominates the plot. The origin of this effect is easy to 
  understand. Look first at the line of points on the far side of the plot, for 
  plucking at $a/L=0.5$, in other words at the mid-point of the string. The 
  line is a zigzag, with every second amplitude falling to zero. The reason is 
  that whenever $n$ is an even number, the corresponding mode shape has a nodal 
  point in the centre. It is impossible to excite a mode of any system by 
  driving at a nodal point. So a mid-point pluck can only excite the odd 
  harmonics, $n=1,3,5,...$. This argument extends in a simple way to other 
  plucking positions. If you pluck at the 1/3 point, every third harmonic will 
  be missing. At the 1/4 point, every fourth will be missing, and so on. This 
  leads directly to the pattern of curving ridges and valleys. 

  The second approach to calculating the motion of our ideal string following a 
  pluck is to notice that for this particular problem of a plucking force 
  applied at a single point, we can short-cut some of the algebraic 
  complication of the Fourier series approach by making use of a general 
  formula for the step response of any linear system. We are applying a force 
  at the chosen plucking point. That force has a constant value $F$ until the 
  player lets go, then it then jumps down to zero. The general formula for step 
  response can be used to find the subsequent motion. The details are given in 
  the previous link: reassuringly, it is found to agree with the result 
  obtained by the Fourier series method. 

  A virtue of this step response method is that it can easily be extended to 
  the case where the string is connected to the instrument body rather than 
  rigidly fixed at both ends. Although Fig.\ 1 didn't explicitly show any body 
  motion, the sideways pull by the player will move the bridge a little, as 
  well as moving the string. One way to solve the coupled problem of the 
  string-body motion after the pluck is first to find the coupled modes of this 
  combined system. Once we know those, the step response formula can be applied 
  directly. We will come back to this shortly. 

  The third approach to finding the pluck response is very different. Rather 
  than using modes, we can take advantage of a general solution to the 
  differential equation governing the motion of an ideal string, due to the 
  18th-century French mathematician Jean le Rond d'Alembert. This approach uses 
  travelling waves: d'Alembert showed that any possible free motion of a string 
  can be described by a combination of a wave travelling leftwards at speed 
  $c$, and another wave travelling rightwards at the same speed. These waves 
  can have any shape whatsoever, but to satisfy the particular starting 
  conditions of our plucked string we need both waves to have a triangular 
  shape. The next link gives details of d'Alembert's solution. 

  Figure 3 shows an animation of the result. At the top, the two separate waves 
  are seen, travelling in opposite directions. You could picture these waves as 
  travelling on a string which extends off to infinity in both directions 
  beyond the actual vibrating string. We are viewing that infinite string 
  through a window matching the length of our actual string, so that waves 
  appear from one side, travel across the window, and vanish off the other 
  side. The lower plot shows the actual string motion, which is the sum of the 
  two waves. Both ends of the string are stationary, as they are supposed to 
  be. If you watch carefully, you will see that the red and blue travelling 
  waves always have equal and opposite magnitudes at the two end of the string, 
  so that when you add them together they always cancel. 

\moobeginvid\begin{tabular}{ccc} \vidframe{ 0.30 }{ vids/vid-d1a992c0-00.png }&\vidframe{ 0.30 }{ vids/vid-d1a992c0-01.png }&\vidframe{ 0.30 }{ vids/vid-d1a992c0-02.png } \end{tabular}\caption{Figure 3.  An ideal plucked string. Upper plots: the left-travelling and right-travelling waves that make up the solution; lower plot: the combined waveform, showing the string response to the pluck.}\mooendvideo

  Now we have a good idea how an ideal string behaves following a perfect 
  pluck. However, this is not enough for the purpose of making an accurate 
  synthesis of the motion of a real string on a real instrument. Several 
  factors have been omitted in this simple analysis, and we need to put at 
  least some of them back. 

  First, real strings do not have natural frequencies that are precisely 
  harmonic multiples of the fundamental frequency. There are two reasons. One 
  is to do with coupling to the body of the instrument, which we deal with 
  shortly. The other arises because any real string has some bending stiffness. 
  This problem is easily dealt with, as described in the next link. We simply 
  need to combine the behaviour of two systems we already know about, the 
  string and a bending beam. Then we can use Rayleigh's principle to get a good 
  approximation for the natural frequencies allowing for bending stiffness. 

  The result is that the frequency of the $n$th overtone (we must now stop 
  calling them `harmonics') is no longer simply an integer multiple of the 
  fundamental. Instead, the overtones are spaced progressively wider apart than 
  the harmonic series of the ideal string. The stronger the influence of the 
  string's bending stiffness, the stronger is this inharmonicity effect. It has 
  an important perceptual consequence, which we can guess from the discussion 
  of tuned percussion instruments in chapter 3. The more accurately harmonic 
  the frequencies are, the more precise is the resulting sense of pitch. So, 
  for example, thin harpsichord strings will have lower bending stiffness and 
  so be more accurately harmonic that thicker piano strings, which is one 
  reason why harpsichordists pay so much more attention to tuning and to 
  tempering systems. 

  The effects of inharmonicity due to bending stiffness can be reduced by using 
  over-wrapped strings. To design a string of given length with a low pitch but 
  which does not become too slack, the string has to be heavy. The simplest way 
  to achieve that is to choose a thicker gauge of string, but this will 
  increase the bending stiffness. Wrapping metal wire round a core of metal or 
  nylon fibres allows a heavier string to be made without the penalty of higher 
  bending stiffness. This is the reason behind the familiar appearance of sets 
  of piano strings or guitar strings: the high strings are solid monofilaments 
  of one material or another, but the lower strings are over-wrapped. 

  The modified natural frequencies can be inserted directly into a modal-based 
  expression for the pluck response, such as the expression given by the step 
  response method. However, the effect of bending stiffness invalidates the 
  d'Alembert travelling-wave method. The wave speed is no longer a single 
  constant value; instead it varies with frequency. Waves with high frequency, 
  or equivalently with short wavelengths, travel faster than those with lower 
  frequency or longer wavelength. The result is that the shape of the 
  travelling triangular waves in Fig.\ 3 will change with time, because the 
  different frequency components making up the shape travel at different 
  speeds. The motion immediately after the pluck will still resemble Fig.\ 3, 
  but it will gradually diverge from it as the effects of bending stiffness 
  build up. 

  But there is a more important reason that the motion of a real string will 
  rapidly deviate from the simple animation in Fig.\ 3. In the plot the motion 
  went on for ever, but of course in a real string it will die away as energy 
  is lost by the string. Furthermore, the different modes of the string will 
  decay at different rates. Usually, the high-frequency components decay more 
  rapidly than the lower-frequency ones, so that the motion will look 
  increasingly smooth and rounded as time goes on. Eventually, it will become 
  more or less sinusoidal because only the fundamental mode has significant 
  amplitude. This frequency-dependent decay pattern is an important part of the 
  audio ``signature'' of different instruments, and of different strings or 
  different notes on a given instrument. Some sound examples to illustrate this 
  will be given in section 5.5. 

  There are three main mechanisms for energy loss from a vibrating string. 
  Energy loss is additive, so the total loss factor for a given mode of the 
  string is the sum of three separate loss factors associated with the three 
  mechanisms. One we have already seen, in section 5.1.2. Energy is lost 
  through the bridge to the body of the instrument, and the associated loss 
  factor is determined by the product of the string impedance and the real part 
  of the bridge admittance of the body. This source of energy loss will be 
  included automatically, when we couple the string to the body to perform a 
  synthesis. 

  The other two energy loss mechanisms are associated directly with the string 
  motion itself, regardless of coupling to the body, and they need to be 
  included explicitly in a model of the string dynamics. One source of energy 
  loss is within the material of the string. The cyclical deformation of the 
  string during the vibration will involve some energy loss. For a monofilament 
  string, this loss is linked to the bending stiffness just discussed. The 
  Young's modulus of the string is in reality a complex number: the real part 
  is the usual elastic modulus, while the imaginary part describes energy loss 
  during cyclical deformation. This complex modulus can be used to derive an 
  expression for the loss factor: see the next link for details. 

  For a metal string, as on a piano or steel-string guitar, the intrinsic 
  damping of the material is usually very small, but for a nylon or gut string 
  it will be much bigger. This is one important reason for the difference of 
  sound between strings of different types. For a multi-layer wrapped string, 
  additional energy loss can occur as a result of friction between the 
  windings. The energy loss in that case will not be represented very 
  accurately by the complex modulus model: friction is intrinsically a 
  nonlinear process, whereas the complex modulus representation is based on 
  linear theory. Furthermore, on an instrument like the guitar where the 
  strings are in direct contact with the player's fingers, this loss mechanism 
  is likely to vary with time. Contaminants from the fingers gradually build up 
  between the windings of the string, increasing the energy dissipation rate. 
  This seems to be the main reason for the difference in sound between new and 
  old guitar strings [1]. 

  The third mechanism for energy loss from a vibrating string is to do with the 
  air surrounding the string: as the string moves, the associated air flow over 
  its surface leads to losses from viscosity. A loss factor to account for this 
  can be estimated using a classical result for flow round a circular cylinder: 
  the details are given in the next link. 

  We now have all the main ingredients for a realistic model of a vibrating 
  string, and we are ready to couple the string to the instrument body and 
  synthesise a version of the sound of plucking the string. One possible method 
  of synthesis has already been mentioned: if we can describe the body motion 
  in terms of its modes, we could compute the coupled modes of the string/body 
  system, then use the general formula for step response to create a pluck 
  sound. This approach is certainly possible, and we will show examples later, 
  but there are some details that are rather fiddly to get right. We will not 
  discuss the method further here: see [2] for more discussion. 

  There is an alternative synthesis method that is faster and in many ways 
  superior. This involves working in the frequency domain, and using an inverse 
  FFT at the final stage to create the required time-varying transient 
  response. This method is attractive because the string and the body are 
  coupled together at a single point, and there is a very simple method which 
  can be applied to any such problem: if two systems have input admittances 
  $Y_1$ and $Y_2$ and they are then rigidly connected at the points at which 
  these admittances are defined, then the coupled system has an admittance 
  $Y_{coup}$ at that point satisfying 

  $$\dfrac{1}{Y_{coup}}=\dfrac{1}{Y_1}+\dfrac{1}{Y_2} . \tag{1}$$ 

  This result expresses the fact that at the coupling point, the two subsystems 
  have equal velocities while the total applied force is the sum of the forces 
  applied to the two separate subsystems. 

  The body is naturally described in terms of its admittance, as already 
  described. By calculating the corresponding admittance at the end of a 
  string, which is derived in the next link, eq. (1) can be applied immediately 
  to give the admittance of the strung instrument at the bridge. 

  To use this method to derive a pluck response, use can be made of a general 
  reciprocal theorem for linear vibration response. This states that if you 
  apply a force at some point on a system, and observe the motion at another 
  point, then you would get exactly the same response if you observed at the 
  first point, and applied the force at the second point. This reciprocal 
  theorem is often used when making measurements: it allows you to swap your 
  excitation point and your measurement point if that makes the experimental 
  details easier. The proof of this result has actually already been given: it 
  follows directly from eq. (11) of section 2.2.5. This formula for the 
  frequency response function is completely symmetric between the forcing point 
  (labelled $j$) and the measurement point (labelled $k$ ). 

  We wish to find the body vibration which results from a step function of 
  force applied at a given position on the string. Reciprocally, we can 
  consider applying the force at the bridge, and calculate the resulting motion 
  at the relevant point on the string. To solve this reciprocal problem, we 
  simply have to multiply two frequency response functions together. The first 
  is the coupled admittance $Y_{coup}$ from eq. (1), which gives the velocity 
  at the bridge when a force is applied there. The second is the dimensionless 
  transfer function between a given displacement applied at one end of a string 
  and the corresponding displacement at the point where the pluck is to be 
  applied. This second transfer function is also derived in the next link. 

  Performing these operations and then applying an inverse FFT, we can create a 
  time history representing the velocity waveform at the bridge following an 
  ideal pluck at the chosen point on the string. There are two final 
  ingredients to be considered, to make the model a little more realistic. 
  First, we might want to allow for a pluck that is somewhat rounded rather 
  than perfectly sharp. We saw earlier that the effect of a rounded pluck is to 
  reduce the high-frequency content of the string motion, and hence of the 
  sound of the instrument. We can represent this effect in a simple way by 
  applying a suitable low-pass filter to the computed frequency response, 
  before doing the inverse FFT. 

  The final ingredient involves thinking more carefully about the output of the 
  synthesis. Really, we would like to compute the sound radiated by the 
  vibrating instrument body. To do this accurately is a very complicated 
  matter, as we glimpsed in Chapter 4. We will want to come back to the 
  question of representing radiated sound, but for the moment we can try to 
  find a simple way to synthesise something that gives a reasonably convincing 
  impression of ``sounding like a guitar'' (or whichever instrument we are 
  considering). A pragmatic approach will be used: empirically, the sound of 
  bridge velocity is a little dull, while the sound of bridge acceleration is a 
  bit too bright. In the frequency domain, the difference between these two 
  things is a factor proportional to frequency. It is easy to move between 
  these two extremes by applying a factor of ``frequency raised to some 
  power'', choosing a power lying between 0 and 1 based on listening to some 
  sound examples. Examples of the effect of varying this power will be shown in 
  section 5.5, together with many other sound examples computed by this 
  approach. 

  As a taster for the usefulness of synthesis, we give a few preliminary 
  examples here. The three sounds below were all synthesised using the 
  frequency domain method just described. A brief snatch from an anonymous lute 
  piece called ``The English Huntsuppe'' is ``played'' on a classical guitar. 
  Parameter values for a typical set of nylon strings were used, together with 
  the measured bridge admittance of a guitar. All the details of the modelling 
  can be found in reference [1]. The middle example, Sound 2, uses the measured 
  body admittance directly. Sounds 1 and 3 are the result of doing some 
  ``virtual woodwork'' in the computer to modify the admittance in a controlled 
  manner. 

\audio{}

\audio{}

\audio{}

  First, do these sound fairly credibly like guitars? Then, how would you 
  describe the difference between the three sounds? Can you guess what has been 
  changed? The answer is that in Sound 1, all the natural frequencies of the 
  body have been reduced by one semitone, whereas in Sound 3 they have been 
  increased by one semitone. Nothing else was changed. If you want to think of 
  a physical change to a guitar that would have roughly the same effect, then 
  you can picture Sound 1 as coming from a guitar with a slightly larger or 
  thinner top plate, and Sound 3 from one with a slightly smaller or thicker 
  top plate. As we showed in section 3.2.4, the area and the thickness of a 
  plate both affect the modal density. 

  Earlier in this section, we saw in Fig.\ 2 that the plucking point on the 
  string has a strong influence on the frequency spectrum of an ideal plucked 
  string. The next pair of sound examples illustrate this effect via 
  synthesised fragments of another lute piece, ``Queen Elizabeth's Galliard'' 
  by John Dowland. All the model parameters are the same as for Sound 2, except 
  that the plucking point has been varied. In Sound 4, the strings are plucked 
  150 mm from the bridge, while in Sound 5 they are plucked 20 mm from the 
  bridge. The open string length for this ``guitar'' was 650 mm, but as in a 
  real instrument the actual vibrating length for each note varies depending on 
  which fret is used to play the note. The difference of sound between these 
  two examples is encouragingly familiar from real playing. 

\audio{}

\audio{}



  \sectionreferences{}[1] J. Woodhouse, E.K.Y. Manuel, L.A. Smith and C. Fritz. 
  ``Perceptual thresholds for acoustical guitar models''. Acta Acustica united 
  with Acustica \textbf{98}, 475-486, (2012).~ DOI 10.3813/AAA.918531 

  [2] J. Woodhouse ``On the synthesis of guitar plucks''.~ Acta Acustica united 
  with Acustica \textbf{90}, 928–944 (2004). 