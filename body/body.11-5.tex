

  We now turn our attention to brass instruments (bearing in mind that they are 
  by no means always made of brass — the name is simply a convenient label for 
  this family of instruments). You might imagine that we will need to develop 
  new modelling to describe the physics of these instruments: they seem to be 
  quite different from the reed instruments. But there is a surprise. When we 
  come to computer simulations of brass instruments, we will find that a simple 
  model can be made with only a tiny change to the program we have already used 
  for reed instruments: just a single minus sign has to be introduced. 

  \textbf{A. What do brass players do with their lips?} 

  But this minus sign makes a crucial difference, and we should begin our 
  discussion of brass instruments with this. We need to understand what a brass 
  player does with their lips, in order to make the instrument sound. Figure 1 
  shows a repeat of an earlier sketch, of the cup mouthpiece of a brass 
  instrument with the player’s lips pressed against it. Figure 2 is another 
  reminder, of the mouthpiece of a reed instrument like the clarinet, discussed 
  extensively in earlier sections. 

  \fig{figs/fig-bf635206.png}{\caption{Figure 1. Schematic sketch of a 
  brass-player's lips, pressed against the mouthpiece. The real thing doesn't 
  have the sharp edges suggested here!}} 

  \fig{figs/fig-637a291d.png}{\caption{Figure 2. Reminder of the earlier model 
  for a clarinet mouthpiece and reed. The pink blobs represent a combination of 
  the player's lips and teeth.}} 

  For the clarinet, we were never in any doubt about the fact that, because the 
  reed faces into the player’s mouth, the reed tends to blow shut when the 
  player blows hard. What is the corresponding description of the lips of a 
  brass player? The mechanical behaviour of lips, made of squashy flesh, seems 
  less clear-cut than the behaviour of a small cantilever made of cane. Indeed, 
  the scientific literature on brass instruments contains considerable 
  discussion on this matter. But the consensus is that (for a reasonable first 
  approximation, most reliable at lower frequencies) lips behave rather as 
  indicated in Fig.\ 1: when they vibrate as the player ``buzzes'' them, they 
  open into the mouthpiece, away from the player’s mouth, and if the player 
  blows harder, the lips tend to open further rather than closing like a 
  clarinet reed. This is where our crucial minus sign comes in. 

  Having made this decision, the simplest model for how the lips behave is 
  exactly the same as the clarinet model set out in section 11.3.1, apart from 
  this change of sign. Figure 3 shows the result. The dashed red curve is the 
  now-familiar nonlinear valve characteristic of a reed mouthpiece, and the 
  blue curve is what we need for a model of brass playing. It is exactly the 
  same curve, flipped over. I have shown the two curves as completely 
  identical, for didactic purposes, but of course the parameter values relevant 
  to lips (mass, stiffness and so on) will be somewhat different from those of 
  a reed mouthpiece. So the functional form of the curve will be the same as 
  for the reed, but the details will a little different. The specific lip 
  parameters used here are taken from Table II of Velut et al. [1]. 

  \fig{figs/fig-78f25789.png}{\caption{Figure 3. The nonlinear characteristic 
  of a reed (red dashed curve) and its inversion that is relevant to brass 
  players' lips (blue curve)}} 

  But now, if you can remember as far back as section 8.5 when we first 
  explored the clarinet, you should spot a major snag. Figure 4 reminds you of 
  three diagrams from that section. If a note is to get started on the clarinet 
  from a gentle initial transient, it must grow from small amplitude. To 
  visualise that, we thought about the tangent to the curve. As the player 
  increases blowing pressure, the “operating point” on the valve curve moves to 
  the left. All the time the tangent at this operating point has a negative 
  slope, as in the first two diagrams in Fig.\ 4, the reed curve simply 
  contributes extra energy dissipation. 

  \fig{figs/fig-c2be5e58.png}{} 

  \fig{figs/fig-df25fed0.png}{} 

  \fig{figs/fig-8e6c0ed0.png}{} 

  But when the operating point moves over the hump of the curve, as in the 
  third diagram, the tangent line then slopes the other way, and it behaves 
  like a source of energy rather than dissipating energy. When that energy 
  source is strong enough to overcome the physical energy dissipation inside 
  the clarinet tube, the note will start — this is the threshold condition, 
  forming the lower edge of the wedge-shaped regions in most of the 
  pressure-gap diagrams from sections 11.3 and 11.4. (The exception was where 
  we noted evidence of an ``inverse bifurcation'', allowing a note to be 
  sustained in a region where it would not be possible to start it from a 
  gentle transient.) 

  Returning to the blue curve in Fig.\ 3, we can see that this reversal of 
  slope will never happen! The tangent to the curve always slopes the same way, 
  so it will always be a source of energy dissipation. So how does a brass 
  player ever get a note to start? Well, I cheated in the description I just 
  gave of the clarinet, because I failed to mention an important assumption. 
  The argument relied on the fact that the reed would displace in the same 
  direction as the force acting on it from the pressure difference on the two 
  sides. That seems intuitively obvious, but in fact it is only true at 
  frequencies below the resonance frequency of the reed or lips. 

  We need to recall something we saw way back in section 2.2, when we looked at 
  the very simplest vibrating system: a mass on a spring, as sketched in Fig.\ 
  5. If we apply a sinusoidal force to the mass, indicated by the arrow, then 
  if the frequency of the force is below the resonance frequency, the effect of 
  the spring dominates over the effect of the mass, and the mass moves in phase 
  with the force. But the resonance frequency is determined by the condition 
  that the effects of the spring and the mass are exactly equal. For a forcing 
  frequency higher than that, the effect of the mass now dominates. The mass 
  then moves in the opposite direction to the force: the phase has reversed. 

  \fig{figs/fig-48e66ef7.png}{\caption{Figure 5. Sketch of a mass-spring 
  oscillator}} 

  This phase reversal is the key to understanding what brass players do with 
  their lips. We have seen that the slope of the tangent to the nonlinear valve 
  characteristic is always negative. But if the player can arrange that their 
  lip resonance is a little lower than the note they are trying to play, the 
  negative sign of the tangent slope is “cancelled out” by the phase reversal. 
  The negative slope then behaves like a source of energy, and the note can 
  grow. So the physics suggests that brass players probably constantly adjust 
  the tension in their lip muscles, to change the resonance frequency of the 
  lips to lie just below each note they are trying to play. (Of course, that 
  isn't how a player would describe the feeling of playing: they cannot be 
  directly aware of the lip resonance frequency, only of the ``tightness'' of 
  embouchure.) This, at least, is our working hypothesis. We will want to put 
  it to the test by seeing some simulated examples. 

  \textbf{B. Input impedance and the role of the mouthpiece} 

  But before we start to look at simulation results, there are some important 
  aspects of the acoustics of a typical brass instrument tube that we should 
  look at. We have already said a bit about the mode shapes and resonance 
  frequencies of such tubes, back in section 4.2. The shape of a typical brass 
  instrument is sufficiently complicated that there is no equivalent of the 
  super-idealised models we used when we started to look at the clarinet and 
  the saxophone. We need to resort to computation, and Fig.\ 15 of section 4.2 
  showed the result of a numerical solution of the “Webster horn equation”, 
  with a bore profile chosen to resemble a brass instrument like a trumpet or 
  trombone, with a cylindrical section followed by a gently flaring section, 
  and finally an abruptly flaring bell. 

  Those computed modes can be plugged directly into the formula derived in 
  section 11.4.1, to estimate the input impedance. Our first step should be to 
  check whether the result gives a reasonable match to a direct measurement of 
  input impedance. We will look at the example of a trombone (with the slide in 
  first position). The red curve in Fig.\ 6 shows the result of applying the 
  procedure just described, and the blue curve shows a measurement on a real 
  trombone (without its mouthpiece -- we come to that in a moment). The blue 
  curve has been displaced downwards by 40~dB for clarity, but Fig.\ 7 shows a 
  comparison of the same two impedances zoomed to the low-frequency range, 
  superimposed without a shift. 

  Figure 6 shows that both the real trombone and the simulated version have 
  many resonances: the peaks are still going strong at the limit of the 
  measured frequency range, 4~kHz. Figure 7 shows that the simulated version 
  does a pretty good job of matching the measurement: the general level is 
  accurately matched, the peaks line up reasonably well, and the peak-to-valley 
  excursions are matched well, apart from the first few modes. For these low 
  modes, the simulated version shows higher peaks and deeper valleys, compared 
  to the measurement. 

  \fig{figs/fig-6ce11ea5.png}{\caption{Figure 6. Input impedance of a trombone, 
  without mouthpiece, with the slide in first position. The red curve shows a 
  simulated version using the model described in section 4.2, the blue curve 
  (shifted down by 40~dB for clarity) shows a measurement of a Conn 8H tenor 
  trombone (data supplied by Murray Campbell).}} 

  \fig{figs/fig-46f0e045.png}{\caption{Figure 7. The two input impedances from 
  Fig. 6, showing a comparison of the low-frequency behaviour without the 
  offset imposed in Fig. 6. The measured impedance (blue curve) is noisy and 
  probably unreliable at very low frequency, due to limitations in the 
  measurement rig.}} 

  Part of the reason for this discrepancy may be connected to inaccuracies in 
  the impedance measurement at very low frequency, and also to the assumed 
  damping in the simulated version. The model from section 4.2 gave undamped 
  mode shapes and frequencies. In order to convert to a realistic impedance, 
  the damping model described in section 11.1.1 has been used. This model 
  represents energy losses associated with viscous and thermal losses along the 
  walls of the pipe. The specific formula is widely quoted in the literature: 
  as is shown in the next link it gives a good approximation to the correct 
  damping factors, but of course there are small deviations in the details. 

  Now we need to add a mouthpiece to the trombone. The shape of a typical brass 
  mouthpiece is sketched in Fig.\ 8, and we can see in Fig.\ 9 that such a 
  mouthpiece has a profound influence on the input impedance. Concentrate for 
  the moment on the blue curve, a direct measurement of the same trombone as 
  the blue curve in Fig.\ 6, now fitted with a mouthpiece. Instead of peaks 
  extending all the way to high frequencies, we now see only a dozen or so 
  strong peaks, after which the curve smooths out to a rather featureless 
  shape. 

  \fig{figs/fig-3998bc38.png}{\caption{Figure 8. Sketch of a typical mouthpiece 
  for a brass instrument, consisting of a cup, a narrow ``throat'', and a 
  tapering ``backbore'' leading into the tube of the instrument.}} 

  \fig{figs/fig-5b08956c.png}{\caption{Figure 9. Input impedances in the same 
  format as Fig. 6, for the case with the mouthpiece included. The red curve 
  shows the result of coupling the simulated impedance to the Helmholtz 
  resonator model described in the text. The blue curve is a direct measurement 
  of the Conn 8H trombone fitted with a Denis Wick 5AL mouthpiece (data 
  supplied by Murray Campbell).}} 

  To see why this has happened, we turn to our simulated trombone and add a 
  mouthpiece model to that. We can’t use the Webster equation directly for the 
  mouthpiece, by simply changing the bore profile near the end of the tube. The 
  reason is that the Webster equation is based on an approximation which 
  assumes that the bore only varies slowly along the length of the instrument. 
  But the mouthpiece bore profile, from Fig.\ 8, changes dramatically over just 
  a few centimetres. 

  So we use a different approach. The mouthpiece sketched in Fig.\ 8, once it 
  has been closed at the left-hand side by the player’s lips, should remind you 
  of something we saw earlier: the Helmholtz resonator, from section 4.2. The 
  cup traps a volume of air, and the backbore makes a narrow entrance to that 
  volume. The result will be a resonance frequency governed by the balance of 
  the stiffness of the enclosed air, and the mass of the air in the backbore. 
  For low frequencies, while the wavelength of sound is very long compared to 
  the dimensions of the mouthpiece, the acoustical behaviour should be well 
  approximated by a simple model based on this effective mass and stiffness. 
  The next link described how such a model can be coupled to the input 
  impedance of the tube without mouthpiece. The result of that calculation is 
  the red curve in Fig.\ 9, and it can be seen that it looks reassuringly 
  similar to the measured blue curve. 

  The Helmholtz resonance frequency in this case was set at 460~Hz, and it is 
  clear from the plot that it is above this frequency that the resonance peaks 
  in the admittance begin to fade away: the mouthpiece imposes a “cutoff 
  frequency”. What is happening is that above the Helmholtz resonance frequency 
  the mass associated with air in the backbore starts to dominate the 
  behaviour. That mass is less and less willing to move as frequency rises 
  above the resonance. The result is that sound waves in the tube are reflected 
  by this immobile mass, so that the pressure variation cannot reach the 
  player’s lips on the other side of the cup volume. This automatically reduces 
  the heights of the impedance peaks to create the cutoff effect. 

  The simple Helmholtz resonator model tells us how the resonance peaks should 
  be affected by adding the mouthpiece. The model predicts that below the 
  mouthpiece resonance frequency the frequencies are all reduced, while their 
  Q-factors are slightly increased. Above the resonance frequency, the 
  Q-factors are decreased. The frequencies continue to decrease, but the rate 
  of reduction slows down and each frequency tends towards the frequency of the 
  next-lowest of the original frequencies. Something of this pattern can be 
  seen directly in Fig.\ 10, which shows the measured impedance with and 
  without the mouthpiece, over the low-frequency range. Each peak in the blue 
  curve lies lower in frequency than the corresponding peak of the black curve, 
  and the first few peaks are a bit taller and narrower because of the 
  increased Q-factor. 

  \fig{figs/fig-a2e1663e.png}{\caption{Figure 10. At comparison of the 
  low-frequency behaviour of the two measured trombone impedances shown in 
  Figs. 6 and 9. The black curve, without the mouthpiece, matches the blue 
  curve from Fig. 6; the blue curve, with the mouthpiece, is the blue curve 
  from Fig. 9.}} 

  We can see this behaviour in action by processing the various impedance 
  functions to extract modal parameters. Figure 11 shows one aspect of the 
  behaviour that is revealed. This plot shows the “effective fundamental 
  frequency” of each mode, calculated by dividing each frequency by the mode 
  number. The two lines at the top, in black and red, show the results without 
  the mouthpiece, from the impedances in Fig.\ 6. The measured results (in 
  black) are, predictably, a little less regular than the idealised synthetic 
  values (in red). But the general pattern of both curves is the same: all 
  frequencies except the lowest lie close to a horizontal line, demonstrating 
  that the frequencies are very close to the pattern of an ideal harmonic 
  series. For reasons explained earlier (see section 4.2) the fundamental 
  frequency does not follow this pattern: it is always far too low to fit into 
  the harmonic pattern. 

  \fig{figs/fig-d8e14eab.png}{\caption{Figure 11. Fitted modal frequencies, 
  each divided by the mode number to give an ``effective fundamental 
  frequency''. Red: synthesised without mouthpiece, from the red impedance 
  curve in Fig. 6; black: measured without mouthpiece, from the blue curve in 
  Fig. 6; green: synthesised with mouthpiece model, from the red curve in Fig. 
  9; magenta: measured with mouthpiece, from the blue curve in Fig. 9; blue: 
  hybrid result of adding the Helmholtz resonator mouthpiece model to the 
  measured impedance without mouthpiece.}} 

  The other lines in Fig.\ 11, with points marked by circles rather than stars, 
  show results with the mouthpiece. The magenta line shows the pattern of 
  frequencies from the measured impedance in Fig.\ 9, and the green line shows 
  the results for the computed model with the mouthpiece, the red curve in 
  Fig.\ 9. The blue line shows a halfway house between these: it shows the 
  results of applying the Helmholtz resonator model to the measured admittance 
  without mouthpiece. The green and blue lines lie very close together, both 
  showing frequencies reduced relative to the no-mouthpiece results. The 
  magenta line shows a similar pattern but with a larger reduction of 
  frequency. 

  Figure 12 shows the frequency plotted against the Q-factor, for the 
  synthesised impedances with and without mouthpiece. The colours and symbols 
  match Fig.\ 11. Looking carefully at this plot, you can see the pattern 
  described above. The two curves cross near the mouthpiece resonance 
  frequency. Below that crossing, each green circle is a little higher and to 
  the left of the corresponding red star: the frequency has been reduced, while 
  the Q-factor has been increased. Above the crossing point, the Q-factors on 
  the green curve fall away to significantly lower levels. The frequencies 
  always fall in the gaps between the frequencies marked by the red stars: one 
  green circle per gap. 

  \fig{figs/fig-86600f90.png}{\caption{Figure 12. Fitted modal Q-factors 
  plotted against fitted resonance frequency, for the synthesised impedances 
  without (red) and with (green) mouthpiece.}} 

  Figure 13 shows the same comparison involving measured results — again with 
  colours and symbols matching Fig.\ 11. The black and blue curves show a very 
  similar pattern to the curves in Fig.\ 12. The magenta curve, showing the 
  measured results with mouthpiece, shows a similar pattern to the blue curve 
  but the drop-off of Q-factors above the mouthpiece resonance is even more 
  marked. The deviation between the blue and magenta curves here is probably a 
  result of the Helmholtz resonator approximation becoming inadequate as 
  frequency rises, and wavelength reduces. 

  \fig{figs/fig-8dfc43d9.png}{\caption{Figure 13. Fitted modal Q-factors 
  plotted against fitted resonance frequency, for the measured impedances 
  without (black) and with (magenta) mouthpiece. The blue curve shows hybrid 
  results, applying the Helmholtz resonator mouthpiece model to the measured 
  impedance without mouthpiece.}} 

  Before we move on to look at simulation results for the trombone, it is 
  interesting to compare the results of Fig.\ 11 with corresponding results for 
  other instruments. Figure 14 shows a plot in the same format as Fig.\ 11, but 
  now including measured frequencies from the clarinet and the saxophone from 
  section 11.4, and for a cornetto that we will come to later in this section. 
  (The cornetto is an early instrument which is played with a brass-type 
  mouthpiece but which is made of wood and has finger-holes. You can see a 
  picture of one a bit later, in Fig.\ 23.) For the clarinet, the “effective 
  fundamental frequency” has been calculated by dividing the frequencies by the 
  numbers 1,3,5,… rather than by 1,2,3,… based on the behaviour of an ideal 
  cylindrical tube. 

  \fig{figs/fig-6a84d7cb.png}{\caption{Figure 14. Fitted modal frequencies, 
  each converted to an ``effective fundamental frequency'' as in Fig. 11. Black 
  with circles: trombone without mouthpiece; black with stars: trombone with 
  mouthpiece; red: clarinet; blue: soprano saxophone; green: cornetto. In the 
  case of the clarinet, frequencies have been divided by 1,3,5... rather than 
  by 1,2,3... to reflect the pattern of resonance frequencies of an ideal 
  cylindrical tube. The clarinet and saxophone data is the same as in Fig. 5 of 
  section 11.4, but plotted in a different format here.}} 

  The trombone, with or without its mouthpiece, is the only one of these 
  instruments to produce an essentially horizontal line of points in Fig.\ 14, 
  showing that it has the most “harmonic” set of tube resonances. This seems 
  slightly paradoxical. All the other instruments are based on cylindrical or 
  conical pipes, which in their idealised “textbook” versions give harmonic 
  overtones naturally. In contrast, the flaring trombone only achieves the 
  effect by dint of careful acoustical engineering. But in the world of real 
  instruments, the effect seems to be the other way round. 

  In fact we should not be too surprised. The trombone is the only one of these 
  instruments that makes direct, musical use of at least 7 or 8 of the tube 
  resonances, to play notes which the player hopes will be in tune. All the 
  others have finger-holes, so that players have more options for changing 
  notes, by opening and closing holes. When they use a higher mode of the tube 
  as the basis of a note, by playing in a higher register, they normally use a 
  register hole, which will slightly change the tuning of the resonances. So 
  when an instrument maker is designing and voicing such an instrument so that 
  it plays in tune, they can adjust the position, size and detailed shape of 
  these finger-holes and register holes. But a brass instrument designer must 
  put all their effort into shaping the bore profile, because that is the only 
  thing that determines the relative frequency tuning of the resonances (apart 
  from the influence of the player's lips and vocal tract, which we are 
  ignoring in this preliminary discussion). 

  \textbf{C. Simulation results for the trombone} 

  We now have all the ingredients in place to simulate some notes on the 
  trombone. First, we can look at periodic waveforms predicted by the model, 
  once the transient has run its course: a selection is shown in Fig.\ 15. In 
  each case, the pressure inside the mouthpiece is shown by the top curve, in 
  red; the volume flow rate through the lips is shown in the middle, in blue; 
  and the motion of the “reed”, in other words of the player’s lips, is at the 
  bottom, in black. The figure shows examples of a note in the first, second, 
  third and fourth registers. 

  \fig{figs/fig-dd5a9ee4.png}{} 

  \fig{figs/fig-1d25e0a5.png}{} 

  \fig{figs/fig-125ab577.png}{} 

  \fig{figs/fig-d1dbe0e0.png}{} 

  The plots reveal an interesting parallel with the ``Helmholtz motion'' of the 
  clarinet, discussed back in section 11.3. The simulated clarinet waveforms 
  were all recognisably related to an ingenious argument by Raman, which 
  suggested that the pressure waveform should always be, approximately, a 
  symmetrical square wave. Once the amplitude is high enough that the reed 
  closes during part of the cycle, this square wave involves an alternation 
  between two states. Either the reed is shut, so there is no flow, or else the 
  pressure inside the mouthpiece is rather close to the mouth pressure, so that 
  again there is rather little flow. 

  There is no argument as simple as Raman’s that applies directly to the 
  trombone, but nevertheless Fig.\ 15 shows that the periodic motion adjusts 
  itself into a rather similar state. Each of the four cases plotted in that 
  figure shows a similar pattern involving the lips closing completely, once 
  per cycle of the oscillation. Almost all the time, either the lips are shut 
  (indicated by horizontal lines in the lower two waveforms), or else the 
  pressure is almost constant at a value close to the mouth pressure (indicated 
  by a near-horizontal line in the upper plot, lying near the mouth pressure 
  indicated by the dashed line). 

  The result is that the flow rate through the lips is always rather small — 
  small enough that you can see a bit of “digital noise” in the blue curve of 
  the top left set, arising from the finite resolution of the simulation model. 
  There is another indication of how small that air flow must be: during the 
  intervals when the lips are open, you can hardly discern the deviation of 
  pressure from the dashed line in the top plot for each note --- even though 
  it is this difference of pressure that causes the volume flow shown in the 
  middle plots. 

  We are ready for a more systematic use of simulation. For the reed 
  instruments, we selected mouth pressure and reed gap as two key parameters 
  for a player, and we used simulation to populate the pressure-gap diagram to 
  give an indication of “playability”. For the trombone, mouth pressure is 
  still an important parameter, but rather than reed gap we will choose lip 
  resonance frequency as the most natural second parameter. So we will generate 
  a “pressure-lip resonance” diagram. Figure 16 shows an example, based on 
  modal simulations using the measured input impedance shown in Fig.\ 9. 

  \fig{figs/fig-8d49bab6.png}{\caption{Figure 16. Pressure--lip resonance 
  diagram for the trombone model, based on the measured input impedance shown 
  in Fig. 11. Colour shading indicates the playing frequency, normalised by the 
  nominal frequency $B\flat\_1$ (58.3~Hz). The Q-factor of the lip resonance is 
  here taken to be 15. The horizontal blue lines mark the peaks of impedance, 
  from the blue curve in Fig. 9.}} 

  The nominal pitch of the tenor trombone with the slide in first position is 
  $B\flat_1$ (58.3~Hz), so the colour shading in Fig.\ 16 as based on this 
  nominal pitch. For cases that produced a note rather than silence, the colour 
  indicates the playing frequency normalised by 58.3~Hz. It is immediately 
  clear that there are horizontal bands of colour corresponding approximately 
  to the values 1,2,3 up to 8. Each band requires a certain threshold mouth 
  pressure in order to start, and this threshold gets progressively higher for 
  the higher registers. The green circles mark the cases shown in Fig.\ 15, 
  lying in the middle of the first four bands, at the highest mouth pressure 
  considered here. The horizontal blue lines mark the impedance peak 
  frequencies --- but note that for these, the left-hand scale should be read 
  as actual frequency rather than lip resonance frequency. 

  Figure 17 sheds a little more light on the playing frequency: it shows the 
  same set of simulations, this time colour-shaded to indicate the frequency 
  deviation in cents away from the relevant harmonic multiple of the nominal 
  pitch. The lowest band has turned largely black in this plot: those notes 
  play at least a semitone flat. But the higher bands all have a stripe of dark 
  red in their lower portion, connoting something close to the desired 
  frequency. 

  \fig{figs/fig-9827b2ed.png}{\caption{Figure 17. Pressure--lip resonance 
  diagram using the same results as Fig. 16, now colour-shaded to indicate 
  frequency deviation in cents from nominal. The horizontal blue lines mark the 
  peaks of impedance, from the blue curve in Fig. 9.}} 

  A different view of these variations of playing frequency is given by Fig.\ 
  18, which shows in graphical form the right-most column of Fig.\ 16. The 
  vertical axis shows frequency on a logarithmic scale, and the horizontal 
  dotted lines mark equal-tempered semitones. Once again, blue lines mark the 
  peak frequencies of the impedance (blue curve in Fig.\ 9). Examining this 
  graph closely, you can see that the second band produces a frequency very 
  close to the value 2, running along the corresponding semitone marker line 
  until it turns abruptly upwards. The next band repeats the pattern, starting 
  by following the semitone marker line at the value 3. All the higher bands 
  have a similar pattern: with a carefully chosen mouth pressure, this 
  simulated trombone seems to be capable of playing in-tune notes for harmonics 
  2—8 of the nominal frequency. 

  \fig{figs/fig-0d6d3e73.png}{\caption{Figure 18. Normalised playing frequency, 
  corresponding to the right-hand column of Fig. 16. Horizontal dotted lines 
  mark equal-tempered semitones. The horizontal blue lines mark the peaks of 
  impedance, from the blue curve in Fig. 9.}} 

  However, the lowest band is quite different. The playing frequency varies 
  from 2 semitones below the nominal pitch to some 5 semitones above it, before 
  the symbols jump up to the second register. This is a consequence of the fact 
  that the lowest resonance of the trombone tube lies nowhere near the nominal 
  fundamental frequency, as Fig.\ 11 showed. Indeed, the blue line that would 
  have marked that resonance frequency cannot be seen because it is off the 
  bottom of the plot. It is perfectly possible to sound a note down in this low 
  register, but the tube provides very little help in setting the correct 
  pitch: the player can “lip” this note up or down over a wide range. As an 
  aside, this regime is the basis for a trombone-player’s party trick: it is 
  possible for a skilled player to hold a fixed note in this register, while 
  sliding the slide in and out with apparently no effect! 

  In Sound 1 you can hear the four notes shown in Fig.\ 15 and marked by 
  circles in Fig.\ 16. The first note is indeed conspicuously flat compared to 
  the harmonic series of the other three sounds. 

\audio{}

  Figure 19 shows a different view of these playing frequencies. This time, 
  they are normalised by the lip resonance frequency, and the plot confirms 
  something we anticipated: the playing frequency always lies above the lip 
  resonance frequency, so that the values in the plot are all bigger than 1. 
  Comparing this plot with Fig.\ 18, we can understand the pattern. Concentrate 
  first on the second band of frequencies. In Fig.\ 18 we saw a horizontal 
  portion close to the desired frequency, followed by a sharp rise. Figure 19 
  shows the converse pattern: the playing frequency gets nearer and nearer to 
  the lip resonance frequency, but when it gets very close the curve flattens 
  out because the playing frequency has to remain above the lip frequency. So 
  the rising portion in Fig.\ 18 shows the playing frequency tracking upwards 
  with the lip frequency. 

  \fig{figs/fig-b9af4ab6.png}{\caption{Figure 19. The playing frequencies from 
  Fig. 18, now normalised by the lip resonance frequency to confirm that the 
  playing frequency always lies below the lip frequency.}} 

  Next, we can look at the transient behaviour of these simulations. Figure 20 
  shows a version of the same pressure—lip resonance diagram but now 
  colour-shaded to indicate transient length. For the value of lip resonance 
  frequency marked by the horizontal green line, Fig.\ 21 shows four examples 
  of waveforms. The first case is for the first coloured pixel on the line, and 
  shows a very long transient. The remaining cases correspond to alternate 
  pixels along that row. The transient gets progressively quicker, as Fig 20 
  indicates. But for all these cases of “cold start” transients, the model is 
  very slow to speak. Trombonists need to do articulatory tricks in order to 
  create more abrupt starts to notes. 

  \fig{figs/fig-9f93b4f1.png}{\caption{Figure 20. Pressure--lip resonance 
  diagram as in Figs. 16 and 17, now shaded to indicate the length of 
  transient. The horizontal green line marks the cases shown in Fig. 21.}} 

  \fig{figs/fig-a6fa91e1.png}{} 

  \fig{figs/fig-2e383f60.png}{} 

  \fig{figs/fig-04fe6273.png}{} 

  \fig{figs/fig-414d7758.png}{} 

  There is one more important comment to be made about the trombone. The 
  simulations in Sound 1 capture a vaguely brass-like sound, but you may have 
  thought that they do not entirely convey the impression of a trombone played 
  loudly — even though the simulations used a very high mouth pressure. There 
  is a characteristic sound of a trumpet or trombone playing a crescendo: the 
  sound is initially fairly mellow, but as the level grows there is a gradual 
  transition to what is often described as “brassy” sound. The sound gets 
  brighter, indicating a marked increase in high-frequency content. 

  The explanation of this effect does not lie in the pressure waveform inside 
  the mouthpiece, which is what we have been concentrating on so far. As first 
  demonstrated by Hirschberg et al. [2], the form of pressure wave changes 
  during its rather long journey along the tube, before it emerges from the 
  bell as audible sound. There is a nonlinear process which happens with 
  large-amplitude waves of various kinds, called “steepening”. 

  The most familiar example is what happens to surface waves on water as they 
  approach a shelving beach. There is a viscous drag force near the ground, 
  which has the effect of slowing down the troughs of the travelling wave. But 
  the wave crests are further away from the ground and do not feel this drag. 
  The result is that the wave crests travel a bit faster than the troughs. The 
  wave might have been initially almost sinusoidal, but the front face steepens 
  as the crest catches up with the trough. This is the effect that surfers take 
  advantage of. Eventually the crest overtakes the trough, and the wave breaks. 

  Something similar happens when a high-intensity sound wave travels down a 
  straight tube, like the section of a trombone up to the end of the slide. The 
  speed of sound is sensitive to temperature. The sound wave is approximately 
  adiabatic, so the air is warmer near the points of maximum compression, and 
  cooler near the troughs of the pressure waveform. The louder the sound, the 
  bigger these temperature changes. So, like the water wave, the crests travel 
  a bit faster than the troughs and the wave steepens. If the tube is long 
  enough, it reaches a critical point. But instead of the wave breaking, it 
  forms a shock front: an abrupt jump in pressure. (A similar shock front is 
  responsible for the “sonic boom” when a supersonic aircraft passes overhead.) 

  We have already seen the result, back in section 10.6: Fig.\ 12 from that 
  section is reproduced here as Fig.\ 22. It shows schlieren flow visualisation 
  of a shock wave emerging from the bell of a trumpet driven at high amplitude 
  by a loudspeaker. A similar thing happens in response to normal playing at 
  fortissimo level. The abrupt pressure jump across the shock front guarantees 
  that the sound spectrum contains a lot of high-frequency content, and this is 
  responsible for the “brassy” sound. 

  \fig{figs/fig-776aaeca.png}{\caption{Figure 22. Four successive schlieren 
  images showing a shock wave radiating from the bell of a trumpet being driven 
  with a loudspeaker at the mouthpiece end, reproduced from Fig. 12 of section 
  10.6. Image reproduced from López-Carromero et al. [3] by permission of the 
  authors.}} 

  This effect is mainly confined to instruments like the trumpet and trombone. 
  The reason is that the steepening effect needs a sufficiently long stretch of 
  cylindrical tube in order to build up enough to form a shock. If the tube is 
  flaring, spreading of the wave-front reduces the amplitude, and the nonlinear 
  steepening effect is reduced. So instruments like the euphonium have too much 
  flare, while instruments like the clarinet have a tube that is too short. 
  This idea can be captured by a formula: Campbell, Gilbert and Myers [4] in 
  section 6.1.4 define a “brassiness potential parameter”, and in Fig.\ 6.11 of 
  that section they plot values of this parameter for brass instruments of 
  different kinds. This plot indeed shows that trumpets and trombones are more 
  likely to show “brassiness” than instruments like euphoniums or saxhorns. 

  \textbf{D. Simulation results for the cornetto} 

  For a contrasting style of “brass” instrument, we will show a few results for 
  the cornetto. The cornetto, in various forms, was used throughout the 
  medieval, renaissance and baroque eras and nowadays it is usually encountered 
  in period ensembles. The typical instrument, like the one shown in Fig.\ 23, 
  has a slightly curved tube with a conical bore. It is usually made of wood 
  covered with thin leather, and it has a small brass-style cup mouthpiece and 
  is played using finger holes. 

  \fig{figs/fig-d8ebf9b0.png}{\caption{Figure 23. A cornetto, being played by 
  Murray Campbell. Image copyright Murray Campbell, reproduced by permission.}} 

  The measured input impedance of a cornetto with all finger-holes closed is 
  shown in Fig.\ 24. Comparing with Fig.\ 9, we can guess that the Helmholtz 
  resonance frequency for the mouthpiece must be a bit above 1~kHz: beyond that 
  frequency we see the same kind of high-frequency cutoff of peak heights that 
  we saw for the trombone. We have already seen the frequencies resulting from 
  a modal fit to this impedance, in the green curve in Fig.\ 14. 

  \fig{figs/fig-4d380f8d.png}{\caption{Figure 24. The measured input impedance 
  of a cornetto (data supplied by Murray Campbell).}} 

  In order to run simulations based on this impedance, we need parameter values 
  for the lip model. In the case of the trombone, we were able to use values 
  from the literature. However, there is virtually no acoustical literature 
  about the cornetto, so we are reduced to a bit of guesswork. Because the 
  mouthpiece is so small, it seems reasonable that the width of the opening and 
  the effective mass of the lips might both be significantly smaller than for 
  the trombone case. For the purpose of the simulations to be shown here, a 
  width 6~mm has been used. It then turned out that the simulation would not 
  play notes with reasonable mouth pressures until the effective mass per unit 
  area of the lips was reduced substantially below the trombone value 
  (9~kg/m$^2$). The results to be shown here use the value 0.6~kg/m$^2$. 

  Figure 25 shows some typical periodic waveforms given by the resulting model: 
  one in the first register, the other in the second register. The cornetto is 
  not normally played in higher registers than that, because most note changes 
  are effected using the finger-holes. Indeed, as we will see in a moment, it 
  may not be possible for human lips to achieve a resonance frequency high 
  enough to excite the third register — although the computer has no such 
  difficulty, of course. Comparing these waveforms with the ones in Fig.\ 15, 
  we see strong similarities. Again, the lips close completely for part of 
  every cycle, although perhaps for a smaller proportion of the cycle than was 
  the case with the trombone. Again, the volume flow rate through the lips is 
  either zero (because the lips are closed) or rather small (because the 
  pressure inside the mouthpiece is close to the mouth pressure). 

  \fig{figs/fig-2491be85.png}{} 

  \fig{figs/fig-baf78ea8.png}{} 

  Figure 26 shows the pressure—lip resonance diagram. It shows regions 
  corresponding to the first three registers, but there is much more black in 
  this picture than in Fig.\ 16 for the trombone. If this model is to be 
  believed, the cornetto player has to place their lip resonance frequency with 
  some precision in order to get a note to sound, especially in the first 
  register where the band of colour is quite narrow. 

  \fig{figs/fig-e6e79522.png}{\caption{Figure 26. Pressure--lip resonance 
  diagram for the simulated cornetto, colour shaded to show playing frequency 
  normalised by the nominal frequency of the note (220~Hz). The green circles 
  mark the waveforms shown in Fig. 27, and the lower of the two is also used in 
  Fig. 32. The simulations assume a cold start. The lip parameter values are: 
  width 6 mm, mass per unit area 0.6~kg/m$^2$, Q-factor 15. Horizontal blue 
  lines mark the impedance peaks from Fig. 26.}} 

  Figure 27 shows the frequency deviation from the nominal value, similar to 
  Fig.\ 17 for the trombone. Figure 28, similar to Fig.\ 18, shows the 
  normalised playing frequency along the right-hand column of Fig.\ 26, with 
  horizontal lines marking equal-tempered semitones. Between them, these two 
  plots show that in the first register, the cornetto is much more well behaved 
  than the trombone. The playing frequency only varies a little either side of 
  the nominal (220~Hz), so that the player should have relatively little 
  difficulty in playing the note in tune. The second register (and the third) 
  tends to play sharp, by a fraction of a semitone. But don’t forget that this 
  instrument has finger-holes, and we are only looking at a single note here. 
  The intonation of the instrument as a whole will be determined mainly by the 
  skill of the instrument maker in placing and shaping the finger-holes. The 
  blue lines in these two plots show that the playing frequencies fall rather 
  close to impedance peaks (Fig.\ 28), while the lip resonance frequency 
  generally needs to be rather well below those impedance peak frequencies 
  (Fig.\ 27). 

  \fig{figs/fig-08f188de.png}{\caption{Figure 27. Pressure--lip resonance 
  diagram corresponding to Fig. 26, now colour-shaded to indicate the deviation 
  in cents from the nominal frequency in each register. Horizontal blue lines 
  mark the impedance peaks from Fig. 24}} 

  \fig{figs/fig-1583cf9f.png}{\caption{Figure 28. Normalised playing frequency 
  from the right-most column of Fig. 26, plotted in the same format as Fig. 18 
  for the trombone. Horizontal dotted lines mark equal-tempered semitones. 
  Horizontal blue lines mark the impedance peaks from Fig. 24.}} 

  Figure 29 shows the playing frequency normalised by the lip frequency. As 
  with the trombone, the plotted values are always greater than 1 — but here 
  they don’t even come very close to 1. That is why we didn’t see the lines in 
  Fig.\ 28 turning upwards as they did for the trombone. As the lip resonance 
  frequency is increased, the note ceases to sound rather than entering a 
  regime where the playing frequency is “carried upwards” by the lip resonance. 

  \fig{figs/fig-b2eccaa4.png}{\caption{Figure 29. The playing frequencies from 
  Fig. 27, normalised by the lip resonance frequency as in Fig. 19.}} 

  Figure 30 shows a typical transient waveform from a simulated note in the 
  first register: in fact, it is exactly the same note that was shown in the 
  left-hand plot of Fig.\ 25, corresponding to the lower of the two green 
  circles in Fig.\ 26. The behaviour of the volume flow rate (blue curve) 
  should be noted. Early in the transient, the flow rate through the lips is 
  quite high, but once the point is reached where the lips begin to close at 
  some point in the oscillation cycle the flow rate reduces conspicuously. The 
  pattern is strikingly reminiscent of the clarinet transients shown in Figs.\ 
  23 and 25 of section 11.3: the Raman model prediction from Fig.\ 25 in that 
  section is reproduced here as Fig.\ 31, as a reminder. 

  \fig{figs/fig-46399163.png}{\caption{Figure 30. The full transient waveform 
  of the simulation whose final periodic part was shown in the left-hand plot 
  of Fig. 27.}} 

  \fig{figs/fig-db153e14.png}{\caption{Figure 31. A simulated clarinet 
  transient using Raman's simplified model, reproduced from Fig. 25 of section 
  11.3.}} 

  Finally, Fig.\ 32 shows the influence of two key parameters of the lip model. 
  In the left-hand plot, the effective mass per unit area has been increased to 
  1~kg/m$^2$. The notes still play in all registers, but the threshold blowing 
  pressures have increased. A similar effect, but significantly stronger, is 
  seen in the right-hand plot. This shows the influence of halving the Q-factor 
  of the lip resonance to 7. 

  \fig{figs/fig-648ec4a1.png}{} 

  \fig{figs/fig-4f82c955.png}{} 

  \textbf{E Back to the trombone} 

  In the last two subsections we have seen some simulation results for a 
  trombone and a cornetto, using a very simple model of the action of the 
  player’s lips. This model gave qualitatively plausible behaviour, but if we 
  want to get somewhere close to quantitative agreement with the behaviour of 
  real instruments, we need to look a bit harder at this model. We will explore 
  in the context of the trombone rather than the cornetto, because the trombone 
  has been far more extensively studied. 

  For a first step, we can look at the influence of a parameter we haven’t said 
  much about: the assumed damping of the lip resonance. The trombone results we 
  have shown so far have all assumed a value 15 for the Q-factor of the lip 
  resonance. Figure 33 shows what happens if that value is halved to 7 or 
  doubled to 30. With the lower value, the threshold mouth pressures for all 
  notes tend to rise, and the 7th and 8th notes do not play at all within the 
  range of mouth pressure explored here. When the Q-factor is increased, the 
  opposite trends are seen. The threshold pressures all reduce, and at the very 
  top of the plot you can see a hint that the 9th note starts to sound. 

  \fig{figs/fig-0dc197f7.png}{} 

  \fig{figs/fig-3b68c7ac.png}{} 

  Now, a Q-factor as high as 30 sounds totally implausible for a resonance of 
  squashy flesh in the lips --- surely a vibrating lip could not ``ring'' for 
  30 cycles or more? Even the value 15 seems too high for plausibility. It is 
  not at all easy to obtain direct measurements of this Q-factor by measuring 
  real lips. There is a published study by Doc, Vergez and Hannebicq [5] 
  estimating lip parameters for trumpet playing which suggests Q-factors of the 
  same order we been using — but they do it by comparing measured thresholds 
  from a test rig using artificial lips with simulations using essentially the 
  same model used here. Their results suggest that as players increase lip 
  tension to raise the resonance frequency in order to play higher notes, the 
  Q-factor also increases. This trend seems quite plausible, but there are as 
  yet very little in the way of direct physiological measurements to test 
  whether the actual values of Q-factor apply to real lips. 

  What evidence there is (see for example Figure 3.30 of Campbell, Gilbert and 
  Myers [4]) suggests that human lips have a significantly lower Q factor than 
  the artificial lips (which consist of sausage-shaped balloons filled with 
  water). So we have a challenge: can we tweak the simulation model to be 
  capable of predicting that notes could be played at realistic threshold 
  pressures using lips with a Q-factor as low as 1? 

  Before we start exploring enhancements to the model, it is useful to review 
  the (rather sparse) set of known facts about trombones played by human lips. 
  These provide the “ground truth” against which simulation results must be 
  tested. The first source of information comes from the experience of 
  trombonists. For a given slide position, a trombonist expects to be able to 
  elicit notes forming (more or less) a harmonic series. They will have to 
  adjust their lips and blowing pressure to achieve these different notes. So 
  far, so good: we have already captured that aspect of behaviour, 
  qualitatively at least, in our pressure-lip resonance diagrams. 

  But there is an important aspect of a trombonist’s experience that we have 
  not reproduced. Figure 18 showed a plot of the predicted playing frequency as 
  a function of the lip resonance frequency, for a particular set of 
  simulations. The playing frequency in every case was near to the 
  corresponding tube resonance, or above that frequency. In trombonist’s 
  language, the simulated trombone notes could be “lipped up” to a higher pitch 
  by adjusting embouchure, but they could not be “lipped down” to a lower 
  pitch. But in reality, trombonists report that notes can be lipped down just 
  as easily as they can be lipped up. 

  It would obviously be useful to have some quantitative information about 
  played trombone notes, but published information is very sparse. Our main 
  resource will be a set of measurements by Boutin, Smith and Wolfe [6]. A 
  professional trombonist played three versions of a single note: one normal, 
  one lipped down and one lipped up. Using a combination of sensors and 
  high-speed video recordings, the authors measured a variety of waveforms for 
  each of these notes, including the pressure inside the mouthpiece, the volume 
  flow rate into the tube, the trajectories of the tips of upper and lower 
  lips, and the area of the lip opening. For all three of their measured notes, 
  the player’s blowing pressure was around 1~kPa. 

  Conveniently, the note chosen for those measurements is one already included 
  in our simulation results: the second natural note with the trombone slide in 
  first position, $\mathrm{B}\flat_2$ (nominally 116.5~Hz). To see how our 
  simulation model is doing before we start enhancing it, we can make a first 
  comparison of their normally-played note with the simulation marked by a 
  green circle in Fig.\ 33. This is the correct note, played with the correct 
  pressure which puts it just above the threshold for the particular model used 
  for Fig.\ 33. 

  A comparison of waveforms is shown in Fig.\ 34. The level of agreement is 
  encouraging, but perhaps rather surprising given the crudeness of the 
  simulation model being used here. This case used a Q-factor of 7, surely too 
  high for human lips. But Fig.\ 33 suggests an immediate problem if we were to 
  reduce the Q-factor in the simulation model. It shows that threshold 
  pressures go up as the Q-factor is reduced, and we are already near the 
  threshold at this blowing pressure with $Q=7$. If we are to find playable 
  notes at a blowing pressure near 1~kPa with a far lower value of the 
  Q-factor, something will have to change in the model to reduce the threshold 
  pressure for this note. 

  \fig{figs/fig-61c6ea6e.png}{\caption{Figure 34. Trombone simulation results 
  from the case marked with the green circle in Fig. 33 (solid lines), compared 
  with measurements by Boutin, Smith and Wolfe [6] (dashed lines) of a trombone 
  note played normally. Top: pressure inside the mouthpiece; middle: volume 
  flow rate into the instrument; bottom: lip opening area. Measured data 
  provided by Henri Boutin and reproduced by permission of the authors of 
  [6].}} 

  We can draw inspiration from Boutin, Smith and Wolfe [6]: they highlighted 
  some effects not included in the current simulation model, and suggested that 
  these might have a big influence on the energy balance, and hence on 
  thresholds. First, as also observed in earlier work by Copley and Strong [7], 
  they found that the horizontal and vertical components of the lip motion are 
  not in phase: the tips of the upper and lower lip each trace out a loop in 
  space. Figure 35 shows these loops, for the same note shown in Fig.\ 34. They 
  also pointed out effects associated with an additional component of the flow 
  into the tube caused by the forwards-and-backwards motion of the lips. 

  \fig{figs/fig-017e8c79.png}{\caption{Figure 35. Trajectories of the upper lip 
  (red) and lower lip (blue) as measured by Boutin, Smith and Wolfe [6] for the 
  same note featured in Fig. 34. Both loops are traversed in the anti-clockwise 
  direction. Data provided by Henri Boutin and reproduced by permission of the 
  authors of [6].}} 

  In order to investigate these suggestions we need to incorporate them into an 
  extended simulation model, described in the next link. We will do this in the 
  simplest possible way, but inevitably the resulting model becomes more 
  complicated because it has more parameters to explore. We will show some 
  results in pictorial form here: the full details behind the pictures are 
  given in the side link, together with more plots of the predicted behaviour. 

  To allow phase differences somewhat like the results seen in Fig.\ 35, we can 
  use a trick. Recall that the simulation model works by stepping forwards in 
  time in small steps. In order to create a phase lag in the vertical lip 
  motion, relative to the horizontal motion, we can simply calculate that 
  vertical displacement using a delayed value of the horizontal displacement, 
  stored from a few time-steps previously. The value of this time delay will be 
  a key parameter in the model. An approximate physical interpretation of this 
  model is that we still represent the lip motion by a single resonance, but 
  now the mode shape corresponding to this resonance has a phase difference 
  between the different components of motion. The tip of the “lip reed”, which 
  is representing a combination of the two lips, moves round an elliptical path 
  in space rather than along a straight line as it did in our model for a 
  clarinet reed in earlier sections. (In mathematical terms, we are using a 
  “complex mode shape”.) 

  \fig{figs/fig-5260ad60.png}{} 

  \fig{figs/fig-d9abad8e.png}{} 

  \fig{figs/fig-4aad8484.png}{} 

  \fig{figs/fig-b5023925.png}{} 

  \fig{figs/fig-38b7b822.png}{} 

  \fig{figs/fig-59a43006.png}{} 

  \fig{figs/fig-fafaa502.png}{} 

  \fig{figs/fig-9cfd6bc0.png}{} 

  \fig{figs/fig-8dd88816.png}{} 

  \fig{figs/fig-2de6cde9.png}{} 

  \fig{figs/fig-287132de.png}{} 

  \fig{figs/fig-483c4dc0.png}{} 

  \fig{figs/fig-b9bd3b4d.png}{} 

  \fig{figs/fig-60d0e720.png}{} 



  \sectionreferences{}[1] Lionel Velut, Christophe Vergez, Joël Gilbert and 
  Mithra Djahanbani, “How well can linear stability analysis predict the 
  behaviour of an outward-striking valve brass instrument model?”, Acta 
  Acustica united with Acustica \textbf{103}, 132–148 (2017) 

  [2] A. Hirschberg, J. Gilbert, R. Msallam and A. P. J. Wijnands, “Shock waves 
  in trombones”, Journal of the Acoustical Society of America \textbf{99}, 
  1754–1758 (1996). 

  [3] A. López-Carromero, D. M. Campbell, J. Kemp and P.L. Rendon, “Validation 
  of brass wind instrument radiation models in relation to their physical 
  accuracy using an optical schlieren imaging setup”, Proceedings of Meetings 
  in Acoustics, \textbf{28}, 035003 (2016). 

  [4] Murray Campbell, Joël Gilbert and Arnold Myers, “The science of brass 
  instruments”, ASA Press/Springer (2021) 

  [5] J.-B. Doc, C. Vergez and J. Hannebicq, “Inverse problem to estimate lips 
  parameters values of outward-striking trumpet model for successive playing 
  registers”, Journal of the Acoustical Society of America \textbf{153}, 
  168—178 (2023). 

  [6] Henri Boutin, John Smith and Joe Wolfe, “Trombone lip mechanics with 
  inertive and compliant loads (‘lipping up and down’)”, Journal of the 
  Acoustical Society of America \textbf{147}, 4133—4144 (2020) 

  [7] D. C. Copley and W. J. Strong, “A stroboscopic study of lip vibrations in 
  a trombone”, Journal of the Acoustical Society of America \textbf{99}, 
  1219–1226, (1996). 