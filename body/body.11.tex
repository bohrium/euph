  This chapter will look in detail at the diverse family of wind instruments, 
  building on earlier material about acoustic resonators and and self-excited 
  oscillation of a simple clarinet model. The first section reviews the 
  different nonlinear excitation mechanisms that distinguish the major families 
  of wind instruments: reed woodwind, brass, free reed and flute-like 
  instruments relying on the interaction of an air-jet with a sharp edge. 

  The second section steps back a little, to give a simple overview of some key 
  phenomena involving fluid flow. These play a role in the later detailed 
  discussions of families of instruments. The list includes turbulence, 
  Bernoulli’s law, vortex shedding, and the mechanisms by which sound can be 
  generated by fluid flow. 

  After that, separate sections dig into the behaviour of the major families of 
  instruments: reed woodwind instruments, “brass” instruments (not all of which 
  are in fact made of brass), free-reed instruments like the accordion and the 
  harmonica, and finally air-jet instruments like the flute or flue organ 
  pipes. 

  I must emphasise that this subject is not my own speciality, and many friends 
  and colleagues have helped me get to grips with the material of this chapter. 
  I particularly thank Joe Wolfe, André Almeida, Henri Boutin, Augustin 
  Ernoult, Jean Kergomard, Jean-Pierre Dalmont, Mico Hirschberg, Murray 
  Campbell, David Sharp, Anurag Agarwal, Max Nussbaumer, and Jeremy Barlow. 

