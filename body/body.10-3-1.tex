  Given a rectangular beam made from the material to be tested, it is easy to 
  measure the lowest resonance frequency (and perhaps some of the higher 
  frequencies as well), and deduce the Young's modulus of the material in the 
  direction aligned with the long axis of the beam. The theory has already been 
  covered in section 3.2.1: we know from equation (12) there that for a beam of 
  length $L$, width $b$ and thickness $h$ made from a material with density 
  $\rho$ and Young's modulus $E$ the natural frequencies are given by 

  $$\omega^2 = \dfrac{EI}{\rho A} k^4 \tag{1}$$ 

  where $A=bh$ is the cross-sectional area, $I=b h^3/12$ is the second moment 
  of area, and for a free-free beam the values of the wavenumber $k$ are given 
  by the roots of the equation 

  $$\cos kL \cosh kL =1 . \tag{2}$$ 

  We also showed that these roots are given approximately by 

  $$k_n \approx (n+1/2) \pi/L \tag{3}$$ 

  where $n=1,2,3...$ labels the modes in order of resonance frequency. This 
  approximate expression is good enough for all the modes except the one we are 
  most interested in, with $n=1$: a more accurate expression for that mode is 

  $$k_1 = 4.73/L . \tag{4}$$ 

  Putting this all together, the frequency $f_1$ (in Hz) of the lowest mode is 
  given by 

  $$f_1 = \sqrt{\dfrac{E}{\rho}} \dfrac{h}{L^2} \dfrac{(4.73)^2}{2 \pi 
  \sqrt{12}} = 1.028 \sqrt{\dfrac{E}{\rho}} \dfrac{h}{L^2} . \tag{5}$$ 

  This can be turned round to give an expression for $E$: 

  $$E=0.946 \rho \dfrac{f_1^2 L^4}{h^2}=0.946 \dfrac{m f_1^2 L^3}{b h^3} 
  \tag{6}$$ 

  where the final expression is expressed in terms of the total mass $m$ of the 
  beam, rather than the density. This mass can be determined directly by 
  weighing. 

  When using this expression it is, of course, important to express all the 
  quantities in a consistent set of units. The standard choice would be to 
  express all lengths in metres, and the mass in kilograms: $E$ will then be in 
  Pascals (Pa). 

  So how should you set about measuring the frequency $f_1$ for your test 
  sample? Figure 1 shows one good way to do it. A cross-grain spruce beam is 
  being tested — in fact it is one of the ones shown in Fig.\ 3 of section 
  10.3. You need to support the beam in a way that doesn’t interfere with the 
  vibration mode you are interested in, and that means positioning the supports 
  on nodal lines. For the lowest mode of a free-free beam the two nodal lines 
  are approximately 1/4 of the way in from each end. So the beam has been 
  supported in those positions with two thin rubber bands, threaded over a 
  section of U-channel which allows the position of the bands to be adjusted 
  very easily. Using thin rubber bands means that the measurement will not be 
  affected very much if the positioning isn’t perfect, because the extra 
  stiffness and/or mass contributed by the bands is minimal. 

  \fig{figs/fig-d14cc355.png}{Figure 1. One of the spruce beams from Fig. 3 of 
  section 10.3 being used to measure stiffness by finding the lowest resonance 
  frequency. The beam is supported in thin rubber bands placed at the nodal 
  points of the mode. A small microphone is placed underneath, near the 
  centre.} 

  This mode has large amplitude at the centre of the beam, so it is being 
  tapped (with a pencil) at that position. A small microphone has been placed 
  underneath at the same position, close to the beam but not touching it. The 
  signal from the microphone goes off to a computer running some kind of FFT 
  app, which will allow you to find the resonance frequency by analysing the 
  tap sound. Alternatively, you could achieve the same effect with a phone app, 
  with the phone’s microphone in a similar position close to the beam. 

  If you would like to include a cross-check in your measurement, you can try 
  to measure the frequencies of a few more of the vibration modes. They should 
  all give more or less the same value of $E$, using the relevant values of 
  $k_n$ from equation (3) to derive equations corresponding to equation (6). 
  But you need to be careful doing these measurements. The nodal lines are 
  different for every mode, so you need to move the rubber bands a bit nearer 
  to the ends for successively higher modes. Also, for the even-numbered modes 
  $n=2,4,…$ it is no good to tap and listen at the centre, because these modes 
  all have a nodal line there. A safe place to tap is at the very end: every 
  mode has an antinode there. But you also need to use an asymmetric position 
  of the rubber bands so that the microphone isn’t exactly in the centre of the 
  beam. 

  Finally, as will be explained further in section 10.3.3, you can use the same 
  measurement to determine the damping factor associated with $E$. For that, 
  you need a bit more sophistication in your FFT app. You either need to find 
  the decay rate of the sound of each mode following the tap, or equivalently 
  you find the half-power bandwidth of the peak in the frequency spectrum given 
  by the FFT. Section 2.2.7 gives the details of how that bandwidth is affected 
  by the damping. 

  You also need to be more careful with the experimental details if you want to 
  get accurate values of damping. The rubber bands will contribute extra 
  damping, and you want to minimise that. So try the test a few times, moving 
  the bands slightly between tests. The lowest value of damping among these 
  measurements should give the best estimate of the true damping of the wood. 
  In any case, measuring damping is always tricky. You can expect to get 
  reliable results for resonance frequencies to an accuracy around 1%, but for 
  damping factors the accuracy is unlikely to be better than 10%. Luckily, this 
  is good enough: we do not hear differences in damping very acutely, and it 
  usually requires a change much bigger than 10% to be audible. 