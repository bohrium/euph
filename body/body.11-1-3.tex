  In this section we look at the effect of a single tone-hole, of the kind 
  examined in section 11.1.2, on the resonance frequencies of a cylindrical 
  pipe of finite length, open at both ends. The pipe has total length $L$ and 
  cross-sectional area $S$, and the tone-hole of area $C$ and effective length 
  $h$ is located a distance $X$ from the end of the pipe. Define $L_1=L-X$ as 
  the length of tube from the excitation end to the tone-hole. We expect that 
  for a sufficiently large tone-hole the effective length of the tube with the 
  open hole will be just a little larger than $L_1$, but if the hole size is 
  reduced the effective length should tend towards $L$. We will calculate an 
  end correction based on the length $L_1$. 

  The procedure will be to calculate the input impedance of the tube with the 
  open tone-hole, then locate resonances of an open-ended tube (like a flute) 
  by finding the zeros of this impedance. The first step is to repeat the 
  calculation from section 11.1.1, but allowing the tube to be terminated in a 
  general impedance $Z_L(\omega)$. The only change is to the boundary condition 
  in equation (6) of that section. With the general impedance, we require 

  $$Z_L = \dfrac{Z_0}{S}\dfrac{Ae^{-ikL_1} + B e^{ikL_1}}{Ae^{-ikL_1} -- B 
  e^{ikL_1}} \tag{1}$$ 

  with all notation the same as in section 11.1.1. Rearranging gives 

  $$B=A e^{-2ikL_1}\dfrac{Z_L -- Z_0/S}{Z_L + Z_0/S} \tag{2}$$ 

  in place of equation (7). The rest of the calculation is unchanged, and the 
  resulting input impedance is 

  $$Z(\omega) = \dfrac{Z_0}{S}\dfrac{Z_L \cos kL_1 + i(Z_0/S) \sin kL_1}{iZ_L 
  \sin kL_1 + (Z_0/S) \cos kL_1} . \tag{3}$$ 

  It is reassuring to see that the earlier result for an open tube can be 
  immediately recovered from this expression by setting $Z_L=0$. 

  The terminating impedance $Z_L(\omega)$ is the combined result of the 
  tone-hole and the remaining length $X$ of tube. These are both open-ended 
  tubes, so their separate input impedances at the position of the three-way 
  junction are given by the result from section 11.1.1: for the tone-hole we 
  have 

  $$Z_1=i\dfrac{Z_0}{C} \tan kh \tag{4}$$ 

  while for the tube we have 

  $$X_2=i\dfrac{Z_0}{S} \tan kX . \tag{5}$$ 

  To see how to combine the two into a single impedance $Z_L$, we simply need 
  to note that at the junction of the tubes, the pressure is the same in both, 
  but the total volume flow rate into the mouths of the two tubes is the sum of 
  the separate flow rates. It follows that the impedances add like electrical 
  resistors in parallel, so that 

  $$Z_L=\dfrac{Z_1 Z_2}{Z_1 + Z_2} = \dfrac{i Z_0 \tan kh \tan kX}{S \tan kh + 
  C \tan kX} . \tag{6}$$ 

  Once we find the solutions for $k$ of this equation, the corresponding 
  frequencies will be given by $\omega=ck$. Substituting, we require 

  $$\tan kL_1 = \dfrac{iZ_L S}{Z_0} = -\dfrac{S\tan kh \tan kX}{S \tan kh + C 
  \tan kX} . \tag{7}$$ 

  One way to see how the solutions of equation (7) behave is to take a 
  graphical approach: plot the left and right hand sides of the equation, and 
  see where the curves cross. An example is shown in Fig.\ 1. This plot assumes 
  a pipe of length 1~m with internal diameter 20~mm (the same as the example 
  from section 11.1.1). The tone-hole is placed at position $X=0.1\mathrm~m$. 
  Three different hole diameters are illustrated: the three blue curves, from 
  top to bottom, correspond to diameters 20~mm, 6~mm and 2~mm. The horizontal 
  axis has been scaled to show frequencies in Hz directly. 

  \fig{figs/fig-e092d147.png}{Figure 1. Plot of the left-hand side of equation 
  (7) in red, and the right-hand side in blue for three different values of the 
  tone-hole area $C$. The intersections of red and blue curves define the 
  resonance frequencies.} 

  The qualitative behaviour revealed by this figure is exactly as we expected. 
  The resonances of a tube cut off at the position of the tone-hole would be 
  given by the zeros of the red curve, at $kL_1=n \pi$ for $n=1,2,3...$, and so 
  $f_n=cn/2L_1$. With the extra tube and the open tone-hole, the frequencies 
  are all lower than this, corresponding to a longer tube or equivalently a 
  positive end correction. For the upper blue curve, with a tone-hole diameter 
  matching the pipe diameter, the effect is small. But as the tone-hole size is 
  reduced, the frequencies fall progressively lower. 

  We can obtain approximate results that reveal what is going on more directly. 
  For any tone-hole we expect $kh \ll 1$. If we also assume that the tone-hole 
  is close to the end of the tube so that $kX \ll 1$ we can obtain a simplified 
  version of equation (7): 

  $$\tan kL_1 \approx -\dfrac{khX}{h+ \lambda X} \tag{8}$$ 

  where $\lambda = C/S$ is the area ratio we introduced in section 11.1.2. 

  Now if we let $kL_1=n\pi + \epsilon_n$ where $\epsilon_n \ll 1$, we can 
  approximate the tan function by $\tan kL_1 \approx \epsilon_n$ because the 
  slope is the same near each of the zeros. Solving equation (8) for 
  $\epsilon_n$ then gives 

  $$\epsilon_n \approx -\dfrac{n \pi h X}{L_1(h+\lambda X) +hX}. \tag{9}$$ 

  Now if we want to express the result in terms of an end correction $\Delta_n$ 
  we must satisfy 

  $$\omega_n \approx \frac{c(n \pi +\epsilon_n)}{L_1} \approx \dfrac{n \pi 
  c}{L_1 + \Delta_n} \tag{10}$$ 

  so that 

  $$\Delta_n \approx \dfrac{L_1 h X}{L_1(h+ \lambda X) +hX} \tag{11}$$ 

  if we ignore the product $\epsilon_n \Delta_n$. Within this linearised 
  approximation, the end correction is independent of $n$, so the resonances 
  are predicted to remain harmonically spaced. But it is clear from Fig.\ 1 
  that the approximation breaks down for smaller tone-holes, and then different 
  resonances will have different effective end corrections and the harmonic 
  relations will be perturbed. 

  As a check, if we set $\lambda = 0$ in equation (11) so that the tone-hole 
  vanishes, we obtain 

  $$\Delta_n \approx \dfrac{L_1 X}{L_1 + X} \approx X \tag{12}$$ 

  because we are assuming $X \ll L_1$. This is exactly what we expected: 
  without the tone-hole, the tube has length $L_1+X$. 

  But surely that result should not rely on any approximations? Sure enough, if 
  we go back to equation (7) and set $C=0$, the equation reduces to 

  $$\tan k L_1 =-\tan kX . \tag{13}$$ 

  Recalling the trigonometric identity 

  $$\tan(A+B)=\dfrac{\tan A + \tan B}{1-\tan A \tan B} \tag{14}$$ 

  we can see that this is exactly the condition for the solutions to satisfy 
  $\tan k(L_1+X)=0$. 