

  The approach of representing the behaviour of a vibrating system as a 
  ``flow'' of trajectories in the phase plane is the starting point for the 
  mathematician’s approach to nonlinear systems which leads to ``chaos 
  theory''. Single-degree-of-freedom oscillators with smooth nonlinearity 
  cannot exhibit chaotic behaviour in their free motion (although their forced 
  motion can sometimes do so: see \tt{}the Wikipedia page on the Duffing 
  equation\rm{} for some examples). The reason is that non-intersecting 
  trajectories in the plane don't have enough room to manoeuvre: they cannot 
  make patterns that are sufficiently complicated to suggest the description 
  ``chaotic''. However, a system with more degrees of freedom can do so. 

  The idea of the phase plane can be generalised to a phase space, with more 
  dimensions. We saw in the previous section that an oscillator with a single 
  mass needs to be represented in two dimensions (i.e. in the plane) in order 
  to give the crucial result that trajectories cannot cross. A vibrating system 
  with two masses can be treated in a similar way, but the corresponding phase 
  space needs four dimensions: unfortunately that makes it virtually impossible 
  to visualise the complete phase portrait. 

  So we will cheat. I will illustrate chaotic behaviour with an example that 
  has nothing to do with music or vibration, but one that has a 
  three-dimensional phase space so that we can plot pictures and visualise at 
  least part of what is going on. Three dimensions is sufficient for more 
  complicated behaviour to be possible, compared to what can happen when 
  confined to a plane. Trajectories can intertwine, or be tied in 
  three-dimensional knots. In fact, that’s just the start of it: the 
  intertwining can become almost unbelievably complicated. 

  The example we will look at relates to something called the \tt{}Lorenz 
  equations\rm{}. Edward Lorenz was a meteorologist interested in the dynamics 
  of the Earth’s atmosphere. Way back in the 1960s, he formulated a very simple 
  mathematical model of atmospheric convection. Computers were in their infancy 
  back then, but he conducted computational studies based on his model, aided 
  by his colleague Ellen Fetter. What they discovered in those studies laid the 
  foundations for modern chaos theory. The equations seem quite innocuous (you 
  can see them in the next link), but for some values of the parameters 
  appearing in the equations they predict behaviour that is complicated beyond 
  what anyone was expecting at the time. Their results have spawned a whole 
  academic industry: hundreds of research papers and at least one book have 
  been written about the Lorenz system. 

  We can get an inkling of this complexity from a computed example. Figure 1 
  shows the first part of a particular trajectory. The starting point is the 
  obvious ``loose end'', low down in the plot. This trajectory curves around, 
  then seems to fall into a plane, where it circulates round and round, 
  spiralling slowly outwards. Nothing obviously “chaotic” here, surely? 

  \fig{figs/fig-042c7948.png}{Figure 1. The early part of a particular 
  trajectory of the Lorenz equations.} 

  Figure 2 shows the same trajectory, computed for twice as long. Now what we 
  see is that after spiralling outwards for a while, it took off into space, 
  did a loop somewhat similar to the initial one, then dropped back onto the 
  plane, in a different place from before, and started spiralling out again. 

  \fig{figs/fig-ed2c8c5c.png}{Figure 2. The same trajectory as in Fig. 1, 
  computed for twice as long.} 

  Finally, Fig.\ 3 shows the result for the same trajectory when we follow it 
  for 5 times as long as in Fig.\ 1. The pattern begins to look quite 
  complicated. It involves repeated episodes of the kind of behaviour we have 
  just seen: spiral out for a while, then leap off, do a loop or two in a 
  different place, then drop back on to the “spiralling plane”, but in a 
  different place every time. The pattern never repeats: this one of the 
  hallmarks of chaotic response. Figure 4 lets you see this same trajectory in 
  3D, with a rotating animation. This reveals that what we have called a 
  “plane” is not in fact quite flat. It also shows that the loops on the 
  right-hand side of the plot also map out a different “curvy plane”. 

  \fig{figs/fig-10bc05d4.png}{Figure 3. The same trajectory as in Figs. 1 and 
  2, computed for 5 times as long as Fig. 1.} 

\moobeginvid\begin{tabular}{ccc} \vidframe{ 0.30 }{ vids/vid-a85b30b5-00.png }&\vidframe{ 0.30 }{ vids/vid-a85b30b5-01.png }&\vidframe{ 0.30 }{ vids/vid-a85b30b5-02.png } \end{tabular}\caption{Figure 4. Animated version of the trajectory of Fig. 3, showing the 3D structure rotating it}\mooendvideo

  A crucial characteristic of a chaotic system is “sensitive dependence on 
  initial conditions”. We can illustrate this directly with our Lorenz example. 
  Figure 5 shows two trajectories. The one in blue is the same as in Fig.\ 3. 
  Its starting point had coordinates (1,~1,~1) in the (x,y,z) space of these 
  plots. The trajectory plotted in red started from the point (1,~1.001,~1): so 
  close to the initial value of the blue trajectory that they are 
  indistinguishable at this plot scale. In the early part of the evolution, 
  both trajectories remain close. Because the red one was plotted on top of the 
  blue one, that means that you can hardly see that the blue one is there. 
  However, it is obvious in the plot that you don't have to wait all that long 
  before the red and blue lines are in different places, and both clearly 
  visible. The tiny difference in starting points has been amplified, so that 
  after a while the two trajectories look quite different. This is ``sensitive 
  dependence'' in action. 

  \fig{figs/fig-5a7f26f8.png}{Figure 5. Two nearby trajectories of the Lorenz 
  equation. In blue is the trajectory from Fig. 3, in red is one with a 
  starting position differing only by one part in a thousand.} 

  Bearing in mind that Lorenz was a meteorologist, this kind of behaviour 
  points towards the famous ``butterfly effect''. The traditional version is 
  that a butterfly flaps its wings somewhere in the Amazon basin, and a few 
  days later the weather in Europe is quite different from what it would have 
  been without the butterfly. It is now fully accepted that the world's weather 
  does indeed show this kind of sensitive dependence. This has led to a 
  different strategy for computer-based weather forecasting: you must accept 
  that no single run of a computer model, however sophisticated, will correctly 
  capture the actual progress of the weather, because the observations on which 
  the models are based have finite accuracy so that sensitive dependence is 
  bound to make the predictions wrong in detail. Instead, they run the model 
  many times with slightly different starting points, then combine the results 
  using statistical ideas. That is why a modern weather forecast gives you a 
  probability of rain at a particular time tomorrow, rather than a definite 
  yes/no prediction. 

  So can we get at least an idea of where this sensitive dependence comes from? 
  We can see the beginnings of such sensitivity in the systems we looked at in 
  section 8.3 based on the 2D phase plane, even though they are not chaotic. 
  One key ingredient is the behaviour near a saddle point, which in mechanical 
  terms describes an unstable equilibrium. Figure 6 reminds you of what happens 
  near a saddle point. This plot illustrates the fact that two trajectories 
  lying close together but on opposite sides of the separatrix heading into a 
  saddle point will, after they have passed close to the saddle point, be 
  widely separated into different regions of the phase plane. 

  \fig{figs/fig-4c7c0b70.png}{Figure 6. A saddle point in the phase plane} 

  To see the relevance to our example of the Lorenz equations, Fig.\ 7 shows a 
  different view of the trajectory from Fig.\ 3. This time, we are looking at 
  it from vertically above, down the z-axis. If you look in the centre of this 
  plot, you can see the characteristic pattern of trajectories near a saddle 
  point. An outward-spiralling trajectory will sooner or later come close to 
  this saddle point, and that may result in it being diverted towards the 
  second ``curvy plane''. The details of that diversion will be sensitive to 
  exactly how close it comes to the saddle point. Now, this is a gross 
  over-simplification of the full complexity of this system's phase portrait, 
  but it gives an inkling. In a phase space with at least three dimensions, the 
  separatrix itself can make its way back to the vicinity of the saddle point. 
  In fact, it can do so repeatedly, and be distorted into very convoluted 
  shapes: that's exactly what happens with the Lorenz system. Following this 
  train of ideas is one route into the mathematics of chaos, but we won't try 
  to go any deeper for the purposes of this introduction. 

  \fig{figs/fig-b7caa381.png}{Figure 7. The trajectory of the Lorenz equations 
  as shown in Fig. 3, but viewed down the z-axis. Notice the characteristic 
  pattern of a saddle point at the centre of this pattern.} 

  Systems capable of showing chaotic behaviour often share a feature in their 
  phase space behaviour. You can think of the trajectories in phase space as if 
  they were streamlines of a “phase fluid”. If you start with a concentrated 
  “blob” of this fluid and then follow it forwards in time, you will often seen 
  that the blob gets repeatedly stretched and folded. Well, repeated stretching 
  and folding is a very good way to create intricate fine structure. A homely 
  example is the way you make puff pastry. You start with a block of dough, and 
  you spread butter on the top. You then fold it over, then roll it out thinner 
  again with your rolling pin. You do that several times over. The result, most 
  obvious once you cook the pastry, is the familiar texture of thin, crumbly 
  layers. What has happened is that the original single layer of butter has 
  been turned into a very large number of thin layers, separated by thin layers 
  of pastry dough. The more times the stretching and folding is repeated, the 
  finer the structure becomes. A cook will only do a small number of folding 
  operations, but the mathematical system of equations does it infinitely many 
  times. 

  A more technological example of a similar process comes from traditional 
  sword-making. The modern version of “\tt{}Damascus steel\rm{}” is made by 
  bonding together sheets of two different types of steel, and then repeatedly 
  heating it, folding it over and hammering out thinner. The result is a fine 
  layered structure of harder and softer forms of steel: the hard layers will 
  hold a very fine edge, while the softer layers make the sword blade less 
  brittle. In recent times, Japanese traditional sword-makers have turned to 
  making woodworking tools, so we have a tenuous musical instrument link: many 
  violin makers will have Japanese tools in their armoury. Figure 8 shows a 
  Japanese knife blade: the surface shows a characteristic wood-grain or 
  watered-silk pattern. If you look carefully at the bevelled edge in the 
  close-up, you can see layers within the thickness of the metal. 

  We can use the Lorenz equations to illustrate one more qualitative 
  phenomenon. In Fig.\ 5 we emphasised the fact that a small change in initial 
  position leads to a drastic divergence of trajectories. But now look at Fig.\ 
  9. This shows three trajectories (in different colours), starting from three 
  very different positions. What we see here is that although the three 
  trajectories remain different in detail, all three of them look remarkably 
  similar in terms of a qualitative description. After a rather short initial 
  phase, all three of them seem to settle in to the two “curvy planes” which we 
  saw in the earlier examples. 

  \fig{figs/fig-98c6f4f7.png}{Figure 9. Three trajectories of the Lorenz 
  equations, with very different starting points} 

  To understand this (at least in a very vague way), we need to introduce a new 
  concept. To start, let’s go back to the simple linear oscillator, with 
  damping. For that system, the entire phase plane is filled with the pattern 
  of a stable spiral: Fig.\ 10 reminds you of the kind of pattern that 
  involves. Every trajectory, whatever its starting point, spirals in towards 
  the centre of the plot. This is simply the graphical view of something that 
  is physically obvious: no matter how you start the oscillator off, if you 
  wait long enough the initial energy will all be dissipated by the damping, 
  and it will settle back towards it equilibrium position. The position (0,0) 
  in the phase plane is called an attractor, for obvious reasons. 

  \fig{figs/fig-94b59e4c.png}{Figure 10. Phase portrait of a stable spiral} 

  There are other possible types of attractor. Some nonlinear systems show 
  self-excited oscillation (we’ll be looking at that in the next section). A 
  steady periodic oscillation makes a closed loop in phase space, usually 
  called a limit cycle in this context. Such a limit cycle can also be an 
  attractor : for at least some range of initial conditions, the trajectories 
  may all move towards it, so that the system settles down into the periodic 
  vibration pattern. We will see some examples in Fig.\ 1 of section 8.5. 

  Before the work of Lorenz and Fetter, and the other pioneers of chaos theory, 
  equilibrium points and limit cycles were the only kinds of attractors people 
  were aware of. But Fig.\ 9 suggests that the Lorenz equations, with the 
  particular parameter values used here, have some kind of attractor, which is 
  neither an equilibrium point nor a limit cycle. You can sense something of 
  the puzzlement of the pioneers when I tell you that the name they gave this 
  kind of thing is a “strange attractor”. That piece of jargon has stuck. It is 
  indeed the case that all (or, strictly, almost all) trajectories of the 
  Lorenz equations with these particular parameter values tend towards a single 
  strange attractor. This attractor is a very complicated beast: it has what 
  the mathematicians call a fractal structure: the more you zoom in on it, the 
  more fine details are revealed, with no limit. If you remember seeing 
  graphics of the “\tt{}Mandelbrot set\rm{}” that would give you the right kind 
  of impression. 

  After that digression to the Lorenz equations, we can return to vibration and 
  look at a simple system that exhibits chaotic behaviour. Back in section 8.3 
  we looked at the behaviour of a pendulum: there was no trace of chaotic 
  behaviour there. But we only have to extend the idea of a pendulum in a 
  simple way to find a chaotic system. Figure 11 shows a double pendulum: two 
  identical rods, hinged together and hung from a pivot at the top. 

  \fig{figs/fig-726a2bba.png}{Figure 11. Sketch of a double pendulum} 

  The equations governing the motion of a double pendulum like this are derived 
  in the next link. Armed with those equations, it is easy to solve them in the 
  computer to illustrate how the system behaves: Fig.\ 12 shows an example. The 
  pendulum has been launched from a particular position and allowed to move 
  freely. It is easy to believe, surely, that this motion might be chaotic. 

  Sometimes one or other of the two component pendulums goes ``over the top'' 
  to perform a rotation. In doing so, the system passes near to a saddle point 
  describing the unstable equilibrium when one or other (or both together) 
  balances in the vertical-upwards position. These saddle points are embedded 
  in a four-dimensional phase space, so we can't easily visualise them, but the 
  generic behaviour is similar to what we saw earlier with the Lorenz system. 
  Provided the motion has enough initial energy that at least one of the 
  pendulums can occasionally do a loop, nearly every motion trajectory exhibits 
  chaotic behaviour. 

\moobeginvid\begin{tabular}{ccc} \vidframe{ 0.30 }{ vids/vid-6f3972c4-00.png }&\vidframe{ 0.30 }{ vids/vid-6f3972c4-01.png }&\vidframe{ 0.30 }{ vids/vid-6f3972c4-02.png } \end{tabular}\caption{Figure 12. Example of free motion of an undamped double pendulum, simulated from the governing equations}\mooendvideo

  You might be thinking ``But this is just a computer simulation: does a real 
  double pendulum behave like that?'' Figure 13 shows an example based on 
  measured motion of a double pendulum: this one is used in a laboratory 
  experiment for second-year Engineering undergraduates at Cambridge. Unlike 
  the simulated version in Fig.\ 12, the real pendulum has some energy 
  dissipation so that if you leave it for long enough it will eventually end up 
  hanging vertically downwards. But you can see in the figure that before that 
  happens, the motion looks similarly chaotic. 

\moobeginvid\begin{tabular}{ccc} \vidframe{ 0.30 }{ vids/vid-3a08e0da-00.png }&\vidframe{ 0.30 }{ vids/vid-3a08e0da-01.png }&\vidframe{ 0.30 }{ vids/vid-3a08e0da-02.png } \end{tabular}\caption{Figure 13. Measured motion of a double pendulum: data courtesy of Hugh Hunt}\mooendvideo

  Figure 14 shows a direct demonstration of ``sensitive dependence'' using this 
  laboratory pendulum system. The operator has launched the system three times, 
  from approximately the same initial position. All three subsequent motions 
  are combined in this plot. You can see that for a while the three cases move 
  in a rather similar way, but they diverge progressively, and by the end the 
  three cases are moving in quite different ways. 

\moobeginvid\begin{tabular}{ccc} \vidframe{ 0.30 }{ vids/vid-050f1f2f-00.png }&\vidframe{ 0.30 }{ vids/vid-050f1f2f-01.png }&\vidframe{ 0.30 }{ vids/vid-050f1f2f-02.png } \end{tabular}\caption{Figure 14. Three different measured examples of motion of a double pendulum, starting from very similar initial positions: data courtesy of Hugh Hunt}\mooendvideo

