  The governing equation for free motion of a pendulum with damping is 

  $$\ddot{\theta} = -\lambda \dot{\theta} -\dfrac{g}{L} \sin \theta . \tag{1}$$ 

  A class of equations which includes this case would be 

  $$\ddot{x}=g(x,\dot{x}) \tag{2}$$ 

  where $g$ is any function. In the spirit of the phase-plane view, we can 
  recast this as a pair of first-order equations by writing 

  $$\dot{x} = y,\mathrm{~~~~~}\dot{y}=g(x,y). \tag{3}$$ 

  Again we could generalise this: the approach to be followed here would apply 
  to any equations of the form 

  $$\dot{x} = f(x,y),\mathrm{~~~~~}\dot{y}=g(x,y). \tag{4}$$ 

  where $f$ is another function. Singular points (or equilibrium points) occur 
  where $\dot{x}=\dot{y}=0$, so they occur at positions $(x_0,y_0)$ which are 
  roots of 

  $$f(x_0,y_0)=0, \mathrm{~~~~~}g(x_0,y_0)=0. \tag{5}$$ 

  Provided $f$ and $g$ are both smooth, we can expand both functions by Taylor 
  series in the vicinity of each singular point, and keep only the linear 
  terms: 

  $$f \approx a_{11} (x-x_0) + a_{12} (y-y_0)$$ 

  $$g \approx a_{21} (x-x_0) + a_{22} (y-y_0) \tag{6}$$ 

  so that 

  $$\begin{bmatrix}\dot{x}\\ \dot{y}\end{bmatrix} \approx A 
  \begin{bmatrix}x-x_0\\ y-y_0\end{bmatrix} \tag{7}$$ 

  where 

  $$A=\begin{bmatrix} a_{11} \& a_{12}\\a_{21} \& a_{22} \end{bmatrix} . 
  \tag{8}$$ 

  To see the behaviour determined by these linearised equations, it is useful 
  to look first for solutions satisfying 

  $$\begin{bmatrix} \dot{x} \\ \dot{y} \end{bmatrix} = \lambda \begin{bmatrix} 
  x-x_0 \\ y-t_0 \end{bmatrix} \tag{9}$$ 

  where $\lambda$ is a constant. In other words, we seek the eigenvalues and 
  eigenvectors of the matrix $A$ such that 

  $$A \mathbf{u} = \lambda \mathbf{u} . \tag{10}$$ 

  The associated solution of eq. (9) is simply 

  $$\mathbf{u} = \mathbf{u_0} e^{\lambda t} . \tag{11}$$ 

  If the eigenvalues are real, this describes motion radially inwards or 
  outwards along the eigenvector direction. If the eigenvalues are complex, 
  they must form a complex conjugate pair, and the corresponding motion 
  involves circulation around the singular point. 

  From the usual determinant, the eigenvalues are the roots of 

  $$(a_{11}-\lambda)(a_{22}-\lambda)-a_{12} a_{21} =0 \tag{12}$$ 

  which leads to 

  $$\lambda^2 -- T \lambda +D =0 \tag{13}$$ 

  where $T=a_{11}+a_{22}$ is the trace of the matrix $A$ and $D$ is its 
  determinant. 

  We can recognise this as the same as the characteristic equation for the 
  damped harmonic oscillator (see section 2.2.7), where $D$ plays the role of 
  the squared frequency, and $T$ is the negative of the damping coefficient. So 
  we already know what behaviour to expect: if $D$ is negative, the system is 
  unstable regardless of the value of $T$. On the other hand if $D$ is 
  positive, oscillatory behaviour can occur. If $T=0$ the system is undamped 
  and will exhibit steady oscillation. If $T<0$ we have positive damping and 
  the motion decays, while if $T>0$ the motion grows in an unstable manner. If 
  $T^2 < 4D$, damping is below the critical value so oscillation will be seen. 
  But if $T^2 > 4D$, damping is greater than the critical value and decay 
  without oscillation will occur. 

  These conditions map directly onto six categories of behaviour, each with its 
  own phase portrait: 

  $D<0$ gives a saddle point 

  $D>0$ with $T=0$ gives a centre 

  $D>0$ with $0 < |T| < 2\sqrt{D}$ gives a spiral, either stable or unstable 
  depending on whether $T<0$ or $T>0$ 

  $D>0$ with $|T| > 2\sqrt{D}$ gives a node, either stable or unstable 
  depending on whether $T<0$ or $T>0$ 

  An undamped system always has $T=0$, so the only options are saddle points 
  and centres. 

  Typical examples of these cases are illustrated in Fig.\ 1. We have seen the 
  phase portraits for most of these in section 8.3, but they are repeated here 
  for completeness. The top row shows the saddle point and the centre. The 
  second row shows shows a spiral and a node: for stable versions of these, the 
  arrows will point inwards, whereas for unstable versions they point outwards. 