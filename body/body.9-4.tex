

  Many bowed-string instruments, particularly cellos, exhibit a phenomenon 
  called a “wolf note”. When a cellist tries to play a particular note they 
  find that it is unusually prone to ``surface sound'', and they may get a kind 
  of “stuttering” or “warbling” effect instead of the steady note they were 
  expecting. This effect is fascinating to a physicist, but it makes life 
  difficult for cellists and they may go to some lengths to control and 
  moderate the problem. You can hear an example in Sound 1, where the player 
  has tried to produce the effect, rather than trying to suppress it as they 
  would usually do. The bow speed is steady and the player's left-hand finger 
  is not moving, but you can hear the intermittent warbling sound of the wolf, 
  and you can also hear instability of the pitch. 

  \aud{auds/aud-28b30a0c-plot.png}{\caption{Sound 1. A wolf note, played high 
  on the C string of a cello.}} 

  Figure 1 shows an extract of a bridge force recording during a warbling wolf 
  like this, and Fig.\ 2 shows a zoomed view of the portion outlined in black 
  in Fig.\ 1. What these waveforms reveal is that the cycle of the wolf 
  involves an alternation of Helmholtz motion, with the familiar sawtooth 
  waveform, and double-slipping motion. 

  \fig{figs/fig-71fc78b5.png}{\caption{Figure 1. Measured waveform of bridge 
  force during a wolf note played on a cello C string.}} 

  \fig{figs/fig-4600f193.png}{\caption{Figure 2. Zoomed view of the portion 
  outlined in black of the bridge force waveform in Fig. 1.}} 

  The physics behind the wolf note is yet another thing that was first 
  explained correctly by Raman [1]. Testing Raman’s explanation was an early 
  target for the computer model described in section 9.2, and we can give a 
  clear description of his argument using a simple simulated example. The great 
  advantage of simulation for this purpose is that we have access to complete 
  information about what is going on, including some things that are hard to 
  see clearly by measurement. 

  The first clue about the explanation comes from the fact that a classic cello 
  wolf note is always associated with a particularly strong resonance of the 
  cello body. To incorporate body resonances into the synthesis model, we need 
  to modify the reflection function describing how a wave is modifying during 
  travel from the bow to the bridge and back. The details of how to do this are 
  described in the next link. For the purposes of demonstration, we need only 
  include a single body resonance, with a frequency very close to the 
  fundamental frequency of the bowed note. 

  Figure 3 shows the resulting simulated bridge force, from one cycle of the 
  wolf note's ``warble''. The alternation of Helmholtz motion and double 
  slipping can be seen, although the details are not exactly the same as Figs.\ 
  2 and 3. This simulation does not attempt to reproduce all the details of 
  that particular cello, string and performance, but it captures the essence of 
  the phenomenon: a cyclical warbling waveform, when bowing a note with a 
  fundamental frequency matching a strong resonance of the instrument body. 

  \fig{figs/fig-6e809913.png}{\caption{Figure 3. Bridge force from a 
  computer-simulated wolf note}} 

  We can use the simulated results to generate a plot corresponding to Raman's 
  original measurement. He made simultaneous observations of the string motion 
  and the motion of the cello body, and this gave the basis for his explanation 
  of the wolf phenomenon. Figure 4 shows a corresponding plot from our 
  simulated example: the red curve shows the string velocity, while the blue 
  curve shows the ``body'' motion. The alternation of Helmholtz motion and 
  double-slipping motion is very clear in the red curve. 

  \fig{figs/fig-6971fe65.png}{\caption{Figure 4. Further results from the same 
  simulated wolf note as in Fig. 3. The red line shows the string velocity at 
  the bowed point (shifted upwards for clarity). The blue trace shows the 
  corresponding velocity of the ``body'', consisting of a single resonator: it 
  has been scaled by a factor 20 so that it can be seen clearly.}} 

  The single resonator representing the ``body'' shows a cyclical growth and 
  decay during the ``warble'' cycle. The explanation of the behaviour we see in 
  Fig.\ 4 is intimately connected to the idea of minimum bow force, explored in 
  the previous section. We found that the minimum bow force depends upon the 
  level of energy dissipation into the body of the instrument: with no 
  dissipation, the minimum bow force would go to zero. For this energy loss, 
  the more the body moves, the greater is the energy dissipation rate, and the 
  higher the resulting minimum bow force. 

  Now think about what happens when you start to bow a note matching a strong 
  body resonance. Initially the body is not vibrating, and it may be possible 
  to start the Helmholtz motion with a relatively low bow force. The resonant 
  response of the body will then start to grow, but as we can see in Fig.\ 4 
  this growth takes several cycles: the timescale is determined by the damping 
  factor of the body mode. As the motion at the bridge gradually increases, the 
  effective minimum bow force rises with it. If the player maintains the low 
  initial bow force, it may happen that the minimum level overtakes the actual 
  bow force. Helmholtz motion must then give way to double-slipping motion. But 
  such motion, especially if the two slips are rather symmetrical, produces far 
  less excitation at the fundamental frequency, and this allows the body 
  vibration to die away again. The effective minimum bow force will fall back 
  until it is below the actual force, and Helmholtz motion may then be able to 
  re-establish. The cycle repeats, and the result is the “warbling wolf”. All 
  these stages can be followed in the two traces of Fig.\ 4. 

  Having seen that the wolf behaviour is intimately linked to minimum bow 
  force, this raises the question of whether we can extend Raman’s and 
  Schelleng’s argument to predict the note-by-note variation in minimum bow 
  force for a particular instrument. The original argument was explained in 
  detail in section 9.3.1. The essential steps went like this. First, we 
  assumed a perfect Helmholtz motion on the string, which tells us the waveform 
  of force at the bridge. That force was used to predict the motion of the 
  bridge. Then, assuming that the bowing point is close to the bridge, we 
  estimated the extra force at the bow caused by that bridge motion. Finally, 
  the minimum bow force criterion was obtained from the requirement that this 
  extra force must not cause a second slip during the supposed sticking 
  interval of the Helmholtz motion. 

  Raman and Schelleng found the bridge motion by approximating the “body” by a 
  dashpot. But we can do something better than that, if we have a measurement 
  of the bridge admittance of the instrument. It is straightforward to make use 
  of this admittance to compute the actual bridge motion in response to the 
  Helmholtz sawtooth wave of bridge force, at any chosen frequency. The rest of 
  the Raman—Schelleng argument then carries through without any change. The 
  details are explained in the next link. 

  We can show an example: it relates to the cello responsible for the wolf note 
  we heard in Sound 1. Figure 5 shows the measured bridge admittance of this 
  cello. Frequency scales are given in semitones (lower) and in Hz (upper). 
  (American readers might call semitones ``half-steps''.) Actual values of the 
  admittance are shown on the right-hand scale; the left-hand scale shows the 
  result scaled by the impedance of the C string of the cello, giving the 
  string-to-body impedance ratio for that string. You may recall that we used a 
  similar scaled admittance when we talked about plucked strings, back in 
  Chapter 5: see for example Fig.\ 8 of section 5.1. 

  \fig{figs/fig-e6640213.png}{\caption{Figure 5. The bridge admittance of a 
  cello}} 

  Figure 6 shows the predicted minimum bow force, computed by the procedure 
  described in the previous link. The frequency axis here denotes the 
  fundamental frequency of a played note. For each of the four strings, a range 
  of 3 octaves from the open string is shown. This is rather more than you can 
  in fact play on a cello with a fingerboard of conventional length, but it 
  ensures that we have covered all playable notes. Vertical lines indicate 
  equal-tempered semitones: Cs are shown in yellow, other notes in grey. The 
  absolute numbers on the vertical axis should not be taken too seriously: the 
  values depend on the particular chosen values of bow speed, bowing position 
  $\beta$, and the friction coefficient of rosin. But all these things only 
  contribute to an overall scale factor: the extent of variation from note to 
  note and between the four strings should be reliably represented in the plot. 

  \fig{figs/fig-91405380.png}{\caption{Figure 6. Calculated minimum bow force 
  for the cello from Fig. 5. The four strings are indicated in colours from red 
  to blue. On each string a range of 3 octaves from the open string is shown. 
  Vertical lines show semitones: Cs in yellow, other notes in grey.}} 

  The curves for the different strings are identical apart from a scaling 
  factor relating to the impedance of the string. It is immediately clear that 
  the minimum bow force for a given note played on the C string is 
  significantly higher than for the other strings. The highest peak occurs at 
  F$_3$ (174.6~Hz), and this is indeed the wolf note played (on the C string of 
  the cello) in Sound 1. A plot like Fig.\ 6 reveals the location of the wolf 
  note from a simple physical measurement (of the bridge admittance). It can do 
  more than that: there are other peaks in the plot, and these serve to 
  indicate to a luthier where other potential ``problem notes'' may occur on 
  the cello. 

  This brings us to a question you may have been wondering about: why do cellos 
  suffer from wolf notes so much more than violins do? Plots like Fig.\ 5 give 
  an immediate clue, because this scaled admittance is a direct measure of how 
  strongly the string is coupled to the instrument body. One view of the wolf 
  note is that it is what can happen if that coupling gets a bit too strong. 
  Figure 7 shows the same scaled cello admittance as in Fig.\ 5, and it also 
  shows the corresponding scaled admittance for two violins (in blue curves). 

  \fig{figs/fig-98cf61ac.png}{\caption{Figure 7. The cello admittance from Fig. 
  5 (red) compared with corresponding scaled admittances for two different 
  violins (in blue).}} 

  The frequency scale based on semitones above the lowest tuned note of the 
  instrument allows these different instruments to be compared directly. This 
  comparison shows that the violins also have strong resonances at about the 
  same place in their playing range as the peak that caused the wolf in the 
  cello. The “signature” mode shapes responsible for these peaks were shown 
  back in section 5.3 (see Figs.\ 5c and 5d in that section). But we see in 
  Fig.\ 7 here that the peak in the cello is higher than the highest peak for 
  either violin, by about 5~dB. That may not look a very big difference in the 
  plot, but it corresponds to nearly a factor of 2 increase in the 
  string-to-body impedance ratio. In case you think I have cheated by choosing 
  violins with rather low peaks, I should say that the dashed curve in Fig.\ 7 
  is for a violin that most players find loud to the point of “crudeness”. 

  There is a simple reason for this difference between the cello and the 
  violin. If you take a violin, and you scale everything in the right way to 
  reflect the difference of tuning of the two instruments, you obtain an 
  instrument that is significantly bigger than a conventional cello. This 
  causes ergonomic problems for the player: the string length is uncomfortably 
  long, and the body size is rather unwieldy. No doubt in response to comments 
  of this kind from players, instrument makers adjusted the design, to give the 
  smaller body and shorter string length that we are used to. 

  But there is a price for this. We want the “signature mode” resonances to 
  fall in roughly the same place in both instruments, in order that the musical 
  qualities of the cello are rather like a “big violin”. We have already seen 
  in Fig.\ 7 that makers have indeed evolved a design for the cello that comes 
  quite close to achieving this objective. But in order to do so, they have had 
  to change two things relative to the “scaled violin”. The smaller body would 
  tend to have higher resonance frequencies, so that body (especially the top 
  plate) is made thinner. On the other hand, the shorter strings still need to 
  be tuned to the correct scaled notes, so the strings need to be heavier than 
  the “scaled violin” would suggest. So we have heavier strings on a lighter 
  body: exactly the recipe we talked about in section 5.2 to increase the 
  loudness of an instrument by making the scaled bridge admittance bigger. But 
  we now see that there is a down-side: this design change increases the 
  likelihood of wolf notes. 

  The idea of the “scaled violin” is not purely hypothetical: such instruments 
  have been designed and made. The theory underlying the scaling procedure was 
  another thing explained by John Schelleng [2], and the driving force behind 
  building the first instruments was American luthier Carleen Hutchins, who we 
  saw in Fig.\ 4 of section 9.2. They didn’t just build a cello-sized 
  instrument: they made an octet of instruments, of which the conventional 
  violin is number 3 in terms of size. You can read more, and see pictures of 
  them, on \tt{}this web site\rm{}. 

  The choice of body sizes for these octet instruments was arrived at by a 
  compromise between complete geometric scaling and ergonomic considerations 
  for the player [3]. The instruments larger than the violin are bigger than 
  the usual viola, cello and bass, but a fully-scaled bass would be impossibly 
  large for a human performer. Having chosen these body sizes, the plate 
  thicknesses and rib heights were adjusted based on Schelleng’s scaling 
  theory, to place the “signature modes” at corresponding frequencies to those 
  of the violin, relative to the tuning of each instrument. The choice of 
  strings was then made with an eye to controlling the string-to-body impedance 
  ratio, to keep the wolf under control. 

  Returning to the wolf note, there is one more topic we need to address. What 
  can a player or an instrument maker do to reduce the impact of a wolf note? 
  Well, there are some simple things. Fitting lighter-gauge strings to the 
  cello will reduce the string impedance, and thus reduce the height of the 
  peak in the scaled admittance as in Fig.\ 5 or 7. That peak could also be 
  reduced in height by increasing the damping of the body mode: for example, 
  players sometimes wedge something like a sock under the tailpiece. The 
  trouble with both those “solutions” to the wolf problem is that they will 
  influence all the notes, not just the wolf note. Players will very often 
  regard this as too high a price to pay: skilled players often learn to live 
  with their wolf, by careful control of bowing (and probably choosing to play 
  that F on the G string, not high on the C string). 

  But there is one kind of treatment that can add damping selectively to the 
  “wolf” peak, without having much effect on the rest of the playing range of 
  the instrument. Somewhat unusually in the world of musical instrument 
  acoustics, this is a familiar approach to vibration problems in many other 
  areas of engineering: it is usually called a “\tt{}tuned mass damper\rm{}” or 
  a “tuned absorber”. 

  The trick is to fix another oscillator of some kind to the structure you want 
  to control, with a resonance frequency tuned so that it exactly matches the 
  resonance that is causing a problem, in our case the wolf note. When that 
  problem mode starts to vibrate, the additional oscillator will also vibrate 
  strongly. If you have designed your additional oscillator so that it has 
  quite high damping, the result will be that it will suck some energy out of 
  the original vibrating structure: exactly the result we were hoping to 
  achieve. The details of how such a tuned mass damper works are given in the 
  next link. 

  We will look at some non-musical examples first. Figure 8 shows the famous 
  “wobbly bridge” in London. The Millennium Bridge was opened with much fanfare 
  to celebrate the new millennium in 2000, but it soon had to be closed again 
  because pedestrians became alarmed by the way the bridge vibrated in response 
  to their walking rhythm. The problem was fixed, at considerable expense, and 
  the bridge is now a major tourist attraction in London. Part of the fix 
  involved installing tuned mass dampers underneath the bridge deck: you can 
  see one in Fig.\ 9, looking very obviously like a spring-mass oscillator. 

  \fig{figs/fig-1b59f74d.png}{\caption{Figure 8. The''wobbly'' Millennium 
  bridge in London. Image: Giorgio Galeotti, CC BY 4.0 
  https://creativecommons.org/licenses/by/4.0, via Wikimedia Commons.}} 

  \fig{figs/fig-fb234644.png}{\caption{Figure 9. A tuned mass damper underneath 
  the London Millennium bridge. Image: Tagishsimon at English Wikipedia., CC 
  BY-SA 3.0 http://creativecommons.org/licenses/by-sa/3.0/, via Wikimedia 
  Commons.}} 

  Another rather neat example of the use of tuned mass dampers can be seen in 
  hi-tech archery bows used in competition shooting. When an archer draws their 
  bow back and then releases the arrow, they will excite a lot of vibration in 
  the bow, particularly in the fundamental mode because that mode shape is very 
  similar to the static shape of the drawn bow. This residual vibration will be 
  felt through the archer’s hand and arm: it will contribute to fatigue and 
  potentially lead to less accurate shooting. 

  The archer would like to reduce this vibration, but they absolutely do not 
  want to slow down the initial response of the bow when it is released, 
  because that would slow down the flight of the arrow. Adding a tuned-mass 
  damper is a good solution. Resonant vibration of the additional oscillator 
  will take a little time to build up, so it will not change anything during 
  the critical early time before the arrow is released. But the damper will 
  suck energy out of the residual vibration, exactly what is wanted. 

  You can see two competition archery bows in Fig.\ 10. It may not be 
  immediately obvious where to look to see the tuned mass dampers. Each bow has 
  a pair of rods, projecting from just below the hand-grip. In the picture the 
  archer is holding the bows back-to-front: in normal use these rods would lie 
  at an angle on either side of the archer’s left hand and arm. The thing you 
  can't see in the picture is that each rod is attached to the bow by a 
  flexible rubber mount. This rubber serves as both the spring and the damping 
  of the oscillator. 

  \fig{figs/fig-700114cf.png}{\caption{Figure 10. Olympic gold medallist Viktor 
  Ruban holds two recurve archery bows while in competition at the 2012 Olympic 
  Games. Tuned mass dampers can be seen on both bows: they are the pair of rods 
  with silver ends, projecting towards us from just below the archer's hands. 
  Image: Ian Patterson, CC BY 2.0 https://creativecommons.org/licenses/by/2.0, 
  via Wikimedia Commons}} 

  How does this idea apply to the cello? Several types of commercial “wolf note 
  eliminator” are available. The most familiar is shown in Fig.\ 11. A brass 
  weight is attached to one of the “afterlengths” of string, between the bridge 
  and the tailpiece. By adjusting its position carefully, the first resonance 
  frequency of this weighted length of string can be tuned to the wolf 
  frequency. This tuning needs to be quite precise: a useful tip is to fit a 
  heavy practice mute to the cello bridge to suppress the body vibration, then 
  tap the afterlength with a pencil, and adjust the position so that the note 
  you hear is tuned exactly to the wolf pitch. The device will then work as a 
  tuned mass damper exactly as we described above, adding additional damping to 
  the body mode and thus reducing the height of the peak in the scaled 
  admittance without making significant difference to the body response at 
  other frequencies. 

  \fig{figs/fig-5bdd95d3.png}{\caption{Figure 11. A wolf note eliminator (the 
  brass cylinder) in position on the afterlength of the G string of a cello. 
  Image: no machine-readable author provided. Mdd4696 assumed (based on 
  copyright claims). Public domain, via Wikimedia Commons.}} 

  There are also devices that work on the same principle but are glued to the 
  underside of the cello soundboard to make a permanent installation. A virtue 
  of these is that the player does not have to adjust something carefully every 
  time they change a string. Alternatively, an instrument maker can sometimes 
  make use of something that is already there on the cello. The tailpiece 
  (which holds the strings, visible at the bottom of Fig.\ 11) has several 
  resonances. By choosing the right tailpiece and then adjusting details of its 
  installation carefully, one of those resonant frequencies can sometimes be 
  made to match the wolf note so that it can act as a tuned mass damper. 



  \sectionreferences{}[1] C.V. Raman; ``On the `Wolf-note' of the Violin and 
  `Cello'', Nature \textbf{2435}, 362--363 (1916) 

  [2] J. C. Schelleng; ``The violin as a circuit'', Journal of the Acoustical 
  Society of America \textbf{35}, 326--338 (1963). 

  [3] C. M. Hutchins; ``Founding a family of fiddles'', Physics Today, 
  \textbf{20}, 233--244 (1967). Available online \tt{}here\rm{}. The key 
  diagram is \tt{}here\rm{}. 