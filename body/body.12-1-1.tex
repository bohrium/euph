  In section 12.1, we highlighted four aspects of impacts which should be 
  included in a comprehensive simulation model. Some details of how to 
  implement these will be described here. The first was to allow for the 
  possibility of some loss of energy during the bounce. As is familiar from 
  many problems involving energy dissipation, we do not have a universal 
  physical model for the mechanism(s) of dissipation, but we can represent the 
  main effect via a simple trick suggested by Stronge [1]. 

  We have already introduced the idea of a contact spring to represent the 
  local forces in the immediate vicinity of the contact region between two 
  bodies. During a bounce, that contact spring will first be squashed during a 
  compression phase, and then it will give its stored energy back during a 
  rebound phase as the bodies start to separate. If we want to recognise the 
  fact that a relatively small proportion of that stored energy is not given 
  back because it has been dissipated in some way, we can represent that 
  directly. 

  If the contact spring is assumed to be linear, then the force would be given 
  by a spring constant $k_c$ multiplied by the distance by which the spring has 
  been compressed as the impacting bodies come together. In a simulation model, 
  we can detect the moment at which the spring force starts to decrease because 
  we are entering the rebound phase, and from that moment on we can replace 
  $k_c$ by a value $Rk_c$ where $R$ is a factor less than 1. This means that a 
  fraction $R$ of the stored energy from the compression phase is discarded, 
  exactly the effect we are trying to achieve. The factor $R$ is the 
  energy-based coefficient of restitution, introduced by Stronge [1] in order 
  to resolve some paradoxical behaviour arising from earlier definitions of 
  coefficient of restitution based on ratios of velocities before and after 
  impact. 

  The second model ingredient we need to mention also concerns the details of 
  the contact spring. A linear spring may not be a very accurate model: an 
  alternative is something called a Hertzian spring. If you think of pressing a 
  spherical rubber ball against a rigid surface like a table, it is obvious 
  that there will a circular ``footprint'' of contact as a patch of the ball is 
  squashed flat. If you press a little harder, that footprint will grow. Press 
  a little harder still, and because of the bigger footprint the ball will 
  deform rather less than the first time you increased the force. In other 
  words, the contact has the character of a ``hardening spring'' (see Fig.\ 2 
  of section 8.2, for example). 

  Specifically, mathematical analysis of this problem (see for example sections 
  4.1 and 4.2 of Johnson [2]) suggests that the force $F$ in the spring should 
  vary according to 

  $$F=k_H \delta^{3/2} \tag{1}$$ 

  where $\delta$ is the displacement of the rubber ball, and $k_H$ is a 
  constant which plays a similar role to the linear spring constant $k_c$ 
  mentioned earlier (but note that because of the power-law form of equation 
  (1), the two constants do not have the same dimensions). If we want to 
  introduce energy dissipation into a Hertzian spring, we can use Stronge's 
  trick again: reduce the value of $k_H$ during the rebound phase of an impact. 

  The third and fourth model ingredients are related, and we can deal with them 
  together. We imagine an impact between two bodies occurring more or less at a 
  single point, and we need to allow for the fact that both bodies (the 
  ``instrument'' and the ``striker'') will be set into vibration by the contact 
  force. Figure 1 shows a sketch of the situation during contact of the two 
  bodies. The spring exerts the same force $F(t)$ outwards on both bodies. The 
  ``striker'' responds to this force with a vibration displacement $z_1(t)$, 
  and the ``instrument'' responds with a corresponding displacement $z_2(t)$. 

  \fig{figs/fig-71d73b4c.png}{\caption{Figure 1. Sketch showing an impact 
  between a ``striker'' (circle) and an ``instrument'' (rectangle), through a 
  contact spring that might be either linear or Hertzian.}} 

  The spring compression is 

  $$\delta=-(z_1+z_2), \tag{2}$$ 

  where the two displacements are measured from the point where contact is 
  first made, so the bodies are only in contact if $\delta > 0$: otherwise, the 
  bodies are out of contact and $F=0$. When they are in contact, the spring 
  force is either 

  $$F=k_c\delta \tag{3}$$ 

  if the spring is linear, or 

  $$F=k_H\delta^{3/2} \tag{4}$$ 

  if it is a Hertzian spring. 

  The displacements $z_1$ and $z_2$ can be calculated from the linear vibration 
  properties of the two bodies. As we have done in earlier chapters when 
  modelling bowed strings and wind instruments, we can express each of these 
  calculations as a convolution with the appropriate impulse response, then 
  compute it efficiently using a set of IIR filters deduced from the modal 
  properties of the system. 

  If the striker has modes with natural frequencies $\omega_n^s$, Q-factors 
  $Q_n^s$ and effective modal masses at the contact point $M_n^s$, its impulse 
  response is 

  $$g_s(t)=\sum_n{\dfrac{1}{M_n^s \omega_n^s} \sin \omega_n^s t~e^{-\omega_n^s 
  t/2 Q_n^s}}. \tag{5}$$ 

  This can be expressed as the imaginary part of a sum of complex exponentials, 
  and the convolution (written as a sum over discrete time steps for simulation 
  purposes) can then be expressed as a set of recursive digital filters as 
  described in detail in section 9.5.2. An identical procedure can be followed 
  for the instrument, with natural frequencies $\omega_n^i$, Q-factors $Q_n^i$, 
  effective modal masses at the contact point $M_n^i$, and an impulse response 
  $g_i(t)$. 

  This description is quite general. For the immediate purposes of section 12.1 
  we apply it to a specific model system: the ``instrument'' is a rectangular 
  thin plate, pinned round its boundary, and the ``striker'' is a pinned-free 
  bending beam which hits the plate (through a contact spring) at its free end. 

  We suppose that the plate has total mass $M_p$ and dimensions $a \times b$, 
  occupying the region $0 \le x \le a$ and $0 \le y \le b$. The mode shapes are 
  then 

  $$u_{nm}=A_{nm} \sin \dfrac{n \pi x}{a} \sin \dfrac{n \pi y}{b} \tag{6}$$ 

  where the normalisation constant $A_{nm}$ is determined by requiring 

  $$\int_0^a{\int_0^b{ u^2_{nm} \dfrac{M_p}{ab}~dx~dy}}=1 . \tag{7}$$ 

  Substituting and evaluating the integrals gives 

  $$A^2_{nm}=\dfrac{4}{M_p} \tag{8}$$ 

  so that the effective modal mass at a striking position $(x_0,y_0)$ is 

  $$M_{nm}=\dfrac{1}{u^2_{nm}(x_0,y_0)}=\dfrac{M_p}{4 \sin^2 \frac{n \pi 
  x_0}{a} \sin^2 \frac{m \pi y_0}{b}} . \tag{9}$$ 

  The corresponding natural frequencies were given in equation (6) of section 
  3.2.3: they satisfy 

  $$\omega_{nm} = \lambda \left[\frac{n^2 \pi^2}{a^2}+\frac{m^2 \pi^2}{b^2} 
  \right] \tag{10}$$ 

  where $\lambda$ is a constant depending on the stiffness, mass and thickness 
  of the plate. It can be conveniently fixed by specifying the frequency of the 
  fundamental mode, with $n=m=1$. For the specific results to be shown, the 
  plate had aspect ratio $a/b = 1.55$ and the lowest frequency was fixed at 
  100~Hz. The striking point was at $(0.3a,0.3b)$. The total mass $M_p$ was 
  chosen differently for different cases. All modes are assigned a Q-factor 
  200, to give a vaguely metallic sound quality. 

  The beam problem for the model striker can be solved in a similar way to the 
  free-free beam analysed in section 3.2.1. For these pinned-free boundary 
  conditions, the condition corresponding to equation (17) of that section is 

  $$\tan kL = \tanh kL . \tag{11}$$ 

  This equation was solved numerically, and the corresponding natural 
  frequencies calculated from the allowed values of $kL$. All modes were 
  assigned a Q-factor 30, roughly the expected kind of value for a wooden 
  stick. 

  The mode shapes were found by substitution into equation (13) of that 
  section, with relations between the constants $K_1$--$K_4$ determined by the 
  particular boundary conditions of this problem. The mass normalisation was 
  then carried out numerically, to yield values for the effective modal masses 
  at the striking point, the free end of the beam. 

  But there is an important exception to that description. A pinned-free beam 
  has a single rigid-body mode, in which it rotates freely about the pinned end 
  with shape 

  $$u_0(x)=A x . \tag{12}$$ 

  The mass normalisation can be carried out trivially here, with the result 
  that the effective modal mass at the free end is $M_0=M_b/3$ where $M_b$ is 
  the total mass of the beam. For the purposes of simulation, this rigid-body 
  mode was given a very low but non-zero frequency so that the IIR filter trick 
  could still be used. This mode is used, of course, to simulate the action of 
  the drummer wielding this ``drumstick''. Unlike all the other modal filters, 
  this one was initialised with non-zero motion in order to produce the strike. 

  \sectionreferences{}[1] W. J. Stronge: Impact mechanics, Cambridge University 
  Press (2000) 

  [2] K. L. Johnson: Contact mechanics, Cambridge University Press (1985) 