  As in section 7.2.1, we are interested in a monofilament string of circular 
  cross-section of diameter $d$ and length $L$, under tension $T$ and made of 
  material with Young's modulus $E$ and density $\rho$. A selection criterion 
  related to damping can be formulated using results derived earlier. The 
  rising trend at high frequency seen in Fig.\ 1 (reproduced from Fig.\ 2 of 
  section 7.2) is caused by the effect of material damping, acting through the 
  bending stiffness of the string. For the two thinner strings shown in the 
  figure, which produced acceptable musical sounds, it can be seen that the 
  data points run out at roughly the same value of loss factor, around 
  $10^{-3}$. 

  \fig{figs/fig-27b5a0c0.png}{\caption{Figure 1. Measured damping of three 
  nylon strings, reproduced from Fig. 2 of section 7.2}} 

  This suggests that the useful bandwidth of a given plucked string might be 
  associated with a threshold value of loss factor, and specifically of the 
  component of loss factor associated with bending stiffness. A model for that 
  loss factor was derived in section 5.4.4. Allowing for damping, Young's 
  modulus becomes a complex value $E(1+i \eta_E)$, and the loss factor of the 
  $n$th string overtone is then given by 

  \begin{equation*}\eta_{bend}=\dfrac{EI}{T}~\left(\dfrac{n \pi}{L} \right)^2 
  \eta_E = \left[\dfrac{nd}{L} \right]^2 \dfrac{\pi^2 E}{64 \rho \gamma^2} 
  \eta_E \tag{1}\end{equation*} 

  \noindent{}after substituting the expressions for $T$ and $I$ from section 
  7.2.1, where $\gamma=Lf_1$ as before. The factor $n^2$ in this expression 
  describes the rising trend. The criterion we want takes the form of a 
  threshold value of $\eta_{bend}$, and for any particular string this will 
  result in a threshold value of $n$ because the other factors in eq. (1) will 
  be known. 

  Of course there is not a crisply-defined threshold for damping, but for the 
  purposes of a selection criterion with the right order of magnitude, a 
  threshold value $\eta_{bend} \approx 2 \times 10^{-3}$ will be used. 
  Furthermore, we will require that a ``musically acceptable'' string should 
  have at least 10 overtones with damping lying below this threshold. The 
  appropriateness of these choices will be confirmed shortly for the three 
  strings shown in Fig.\ 1, and they have been further supported by a wide 
  range of case studies described in reference [1]. The material loss factor 
  $\eta_E$ is fixed for a given string and tuning: measurements on a range of 
  nylon strings [1,2] suggest a value $\eta_E \approx 0.04$. 

  The criterion can be expressed in graphical form. From eq. (1), it can be 
  seen that the value of $\eta_{bend}$ for a given material depends on two 
  parameters relevant to string choice: $nd/L$ and $\gamma$. For material of a 
  given density, the value of $\gamma$ determines the stress: 

  \begin{equation*}\sigma = 4 \rho \gamma^2. \tag{2}\end{equation*} 

  This is important, because it was shown in reference [2] that the value of 
  $E$ for nylon strings varies significantly with stress through the effect of 
  strain stiffening, described in section 7.2. In other words, $E$ is a 
  function of $\gamma$. It is straightforward to draw a contour map of 
  $\eta_{bend}/\eta_E$ in the $(\gamma, nd/L)$ plane. The result is shown in 
  Fig.\ 2, for nylon strings. Contours of $\eta_{bend}/\eta_E$ have been 
  plotted at intervals of 0.01 up to the value 0.1. Beyond that value the 
  string overtones will surely be too highly damped to be of interest: the 
  suggested threshold value is 0.05, in the middle of the plotted range. 

  \fig{figs/fig-fff59237.png}{\caption{Figure 2. Contours of 
  $\eta\_{bend}/\eta\_E$ for nylon strings, plotted at intervals of 0.01 up to 
  0.1. Discrete symbols show results for strings from Fig. 1: for each column 
  of points the lowest symbol shows the value for $n=1$, then above it values 
  for $n=10,20,30...$. Squares: string of diameter 0.50~mm tuned to 327.5~Hz, 
  plotted in black in Fig. 1; circles: string of diameter 0.96~mm tuned to 
  327.5~Hz, plotted in red in Fig. 1; diamonds: string of diameter 1.68~mm 
  tuned to 131~Hz, plotted in green in Fig. 1}} 

  Points corresponding to particular strings can be calculated and added to the 
  plot, but because of the presence of $n$ in the quantity plotted on the 
  $y$-axis, a given string gives a point for every relevant overtone. These 
  overtones all have the same value of $\gamma$, so they make a regular 
  vertical column in the plot. The open symbols show the three strings from 
  Fig.\ 1. For each string, the lowest plotted symbol shows the fundamental 
  $n=1$, and then to indicate the pattern without cluttering the plot with too 
  many points, symbols are plotted above it for $n=10, 20, 30...$ 

  To interpret the plot, consider first the thinnest string of the set in Fig.\ 
  1, indicated by square symbols towards the right-hand side of Fig.\ 2. 
  Locating the contour corresponding to $\eta_{bend}/\eta_E=0.05$, it can be 
  seen that the closest square symbol to that contour marks the value $n=50,$ 
  so the prediction is that this string should have about $50$ overtones with 
  damping lower than the chosen threshold. The middle string from Fig.\ 1 is 
  indicated by circular symbols, and because this string had the same length 
  and the same tuning as the thinnest string, they appear at the same value of 
  $\alpha$. However, the circular symbols are wider apart, and the 
  $\lambda=0.05$ contour passes between the two symbols marking $n= 20$ and 
  $30$. So for this string, roughly $25$ overtones should have damping below 
  the threshold. Comparing the two, the prediction is that the bandwidth of 
  lightly-damped string modes should be roughly twice as big for the thinner 
  string. Looking at where the plotted points run out in Fig.\ 1, this 
  prediction matches the observations quite well. Both these strings exceed the 
  requirement of at least 10 overtones with light damping. 

  The thickest string from Fig.\ 1 is indicated in Fig.\ 2 by diamond symbols, 
  towards the left-hand side. For this string, even the symbol corresponding to 
  $n=10$ lies above the $\lambda=0.05$ contour, so the prediction is that this 
  string should have too few lightly-damped overtones to comply with the 
  suggested criterion for acceptability. Furthermore, remember that the 
  criterion underlying this plot captures only the damping due to 
  viscoelasticity: for a very thick string like this, the damping due to air 
  viscosity takes over at low frequency while the viscoelastic loss is still 
  quite high, so that in fact the model predicts that this string should have 
  no modes at all with low damping. That is exactly what the measurements in 
  Fig.\ 1 revealed. 

  Based on this information, a damping criterion can be plotted on the string 
  selection chart developed in section 7.2. Because the vertical axis depends 
  on $d$ rather than on $nd/L$ as in Fig.\ 2, the length $L$ will make a 
  difference. For a given value of $L$, it is easy to take each value of 
  $\gamma$ and use the expression for $\eta_{bend}/\eta_E$ to calculate the 
  threshold value of $d$ for which a string of that length would have at least 
  10 overtones with damping lower than the chosen value. This leads to the 
  rising curving lines in the selection chart (Fig.\ 6 of section 7.2). 

  \sectionreferences{}[1] J. Woodhouse and N. Lynch-Aird ``Choosing strings for 
  plucked musical instruments''. \tt{}Acta Acustica united with Acustica 
  \textbf{105}, 516-529, (2019)\rm{}.~ 

  [2] N. Lynch-Aird and J. Woodhouse ``Mechanical properties of nylon harp 
  strings''. \tt{}Materials \textbf{10}, 497, (2017)\rm{}.~ 