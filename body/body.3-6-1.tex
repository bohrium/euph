  The idealised model of a stretched membrane is closely related to the one we 
  saw in section 3.1.1 for a string. The membrane is allowed to vibrate 
  transversely with a (small) displacement from equilibrium given by 
  $w(x,y,t)$. Suppose it has tension per unit length $T$, and mass per unit 
  area $m$. To obtain the equation of motion, we can consider a small 
  rectangular element of the membrane between positions $x$ and $x+ \delta x$ 
  and $y$ and $y+ \delta y$. (Note that the displacement is exaggerated in the 
  plot, for clarity.) 

  \fig{figs/fig-b2fdb90a.png}{\caption{Figure 1. A small element of a stretched 
  membrane, undergoing transverse vibration}} 

  Using exactly the same argument as in section 3.1.1, Newton's law for this 
  small element requires 

  $$m~\delta x~\delta y \frac{\partial^2 w}{\partial t^2} \approx T~\delta y 
  \left[ \left( \frac{\partial w}{\partial x} \right) _{x + \delta x} -- \left( 
  \frac{\partial w}{\partial x} \right) _{x} \right] $$ 

  $$+ T~\delta x \left[ \left( \frac{\partial w}{\partial y} \right) _{y + 
  \delta y} -- \left( \frac{\partial w}{\partial y} \right) _{y} \right] 
  .\tag{1}$$ 

  Now as $\delta x \rightarrow 0$ and $\delta y \rightarrow 0$ the two terms in 
  square brackets tend towards second derivatives multiplied by $\delta x$ and 
  $\delta y$ respectively. Cancelling a factor $\delta x ~ \delta y$, the 
  equation of motion becomes 

  $$m \frac{\partial^2 w}{\partial t^2} -- T \left[ \dfrac{\partial^2 
  w}{\partial x^2} + \dfrac{\partial^2 w}{\partial y^2} \right] =0 .\tag{2}$$ 

  We want to use this equation to find the vibration modes and natural 
  frequencies of a circular drum. Obviously, we do not want to analyse that 
  problem using Cartesian coordinates: we first want to transform eq. (2) into 
  polar coordinates $(r, \theta)$. From the usual relationships $x=r \cos 
  \theta$ and $y = r \sin \theta$ we can deduce 

  $$\frac{\partial r}{\partial x} = \cos \theta, \frac{\partial r}{\partial y} 
  = \sin \theta \tag{3}$$ 

  and 

  $$\frac{\partial \theta}{\partial x} = -- \frac{\sin \theta}{r}, 
  \frac{\partial \theta}{\partial y} = \frac{\cos \theta}{r} . \tag{4}$$ 

  Now we use the chain rule: for example, 

  $$\frac{\partial w}{\partial x} =\frac{\partial w}{\partial r} \frac{\partial 
  r}{\partial x} + \frac{\partial w}{\partial \theta} \frac{\partial 
  \theta}{\partial x}=\frac{\partial w}{\partial r} \cos \theta -- 
  \frac{\partial w}{\partial \theta} \frac{\sin \theta}{r} . \tag{5}$$ 

  Using this approach repeatedly and then gathering up all the terms, we find 
  after a little algebra that 

  $$\dfrac{\partial^2 w}{\partial x^2} + \dfrac{\partial^2 w}{\partial 
  y^2}=\dfrac{\partial^2 w}{\partial r^2} + \dfrac{1}{r} \dfrac{\partial 
  w}{\partial r}+ \dfrac{1}{r^2} \dfrac{\partial^2 w}{\partial \theta^2} 
  \tag{6}$$ 

  so that the equation of motion (2) becomes 

  $$m \frac{\partial^2 w}{\partial t^2} -- T \left[ \dfrac{\partial^2 
  w}{\partial r^2} + \dfrac{1}{r} \dfrac{\partial w}{\partial r}+ 
  \dfrac{1}{r^2} \dfrac{\partial^2 w}{\partial \theta^2} \right] =0 . \tag{7}$$ 

  We can solve this for the vibration modes of a circular drum with a fixed 
  edge all the way round a circle of radius $a$, by the method of separation of 
  variables. We try to find a solution of the form $w(r,\theta,t) = f(r) 
  g(\theta) e^{i \omega t}$. Substituting into eq. (7), we then require 

  $$T \left[ f^{\prime \prime} g +\frac{1}{r} f^{\prime} g+\frac{1}{r^2} f 
  g^{\prime \prime} \right] = -m \omega^2 f g \tag{8}$$ 

  where the prime symbol denotes differentiation. We can rearrange the terms in 
  eq. (8) into the form 

  $$\frac{ r^2 f^{\prime \prime}}{f} + \frac{ r f^{\prime}}{f} + \frac{m 
  \omega^2}{T}r^2 = -- \frac{g^{\prime \prime}}{g} . \tag{9}$$ 

  The significance of this rearrangement may not be immediately apparent, but 
  the key feature is that everything on the left-hand side is a function of $r$ 
  only, while the right-hand side is a function of $\theta$ only. The only way 
  that a function of $r$ could be equal to a function of $\theta$ (for all 
  values of $r$ and $\theta$) is if both expressions are in fact NOT functions 
  of anything, but simply constant. So we put each side separately equal to a 
  constant, and for reasons that will rapidly become apparent, it is convenient 
  to call this constant $n^2$. Look first at the equation for $\theta$: the 
  function $g$ must satisfy 

  $$g^{\prime \prime} = -- n^2 g. \tag{10}$$ 

  The general solution is $g=A \cos n \theta+B \sin n \theta$ with constants 
  $A$ and $B$. Bearing in mind that $\theta$ is the polar angle, this solution 
  only makes physical sense if $n$ is a whole number so that the solution 
  ``joins up'' when you do one complete circle around the origin. So the 
  allowed solutions have $n=0, 1, 2, 3,...$. 

  Now we can go back to eq. (9) and look at the corresponding equation for the 
  radial variation $f(r)$. This must satisfy 

  $$r^2 f^{\prime \prime} + r f^{\prime} + \left(\frac{m \omega^2}{T} r^2 -- 
  n^2 \right) f = 0 . \tag{11}$$ 

  This equation is called Bessel’s equation, and it arises in many problems 
  with circular geometry. Being a second-order ordinary differential equation, 
  it has two independent solutions (for a given value of $n$). These solutions 
  are collectively called Bessel functions. One of these solutions always has a 
  singularity (i.e. infinite value) at $r~=~0$, and so is not relevant for 
  describing the vibration of a complete circular drum. The second solution is 
  the one we want. It is usually denoted $J_n(kr)$, where $k= \omega 
  \sqrt{m/T}$. Its mathematical properties have been extensively explored, but 
  we need not go into any details here. Bessel functions can be readily 
  computed: they are available as library functions in scientific computation 
  systems such as Matlab, Mathematica, Python and so on. The first few of these 
  Bessel functions are plotted in Fig.\ 2. 

  \fig{figs/fig-2d256ea8.png}{\caption{Figure 2. The first few Bessel function 
  $J\_n$.}} 

  To find vibration modes of a membrane we must now enforce the boundary 
  condition, which is simply 

  $$w(a,\theta,t) =0, \mathrm{~~or~~} f(a)=0 . \tag{12}$$ 

  The natural frequencies are thus governed by the condition 

  $$J_n(ka)=0 . \tag{13}$$ 

  It is clear from the graphs above that for any value of $n$ there will be a 
  series of possible values of $ka$ at the successive zeros of $J_n$. Each will 
  lead to a value of natural frequency $\omega=k \sqrt{T/m}$. The vibration 
  mode shapes are then 

  $$u(r,\theta)=J_n(kr) \cos n \theta \mathrm{~~~or~~~}J_n(kr) \sin n \theta 
  \tag{14}$$ 

  They can be arranged in a two-dimensional family, according to the value of 
  $n$ (which determines how many nodal diameters there are) and the number of 
  nodal circles, governed by which zero of the Bessel function is chosen. The 
  result is sketched in Fig.\ 3, and the corresponding value of $ka$ is shown 
  by each mode sketch. The frequencies can be calculated from the values of 
  $ka$. 

  \fig{figs/fig-258751c1.png}{\caption{Figure 3. Node line patterns for the 
  first few modes of an ideal circular drum. The value of $ka$ is indicated 
  above each plot. The pattern would continue indefinitely to the right and 
  downwards.}} 