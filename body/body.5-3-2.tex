  It is useful to find a way to focus on a formant ``hill'' without the 
  distracting details of the individual body modes. This can often be done by 
  using an approach known as “Skudrzyk’s mean value method” [1]. The starting 
  point is eq. (12) from section 2.2.7, expressing the admittance in terms of 
  modal parameters. The relevant version of that equation is: 

  $$Y(\omega) =\sum_n \dfrac{i \omega u_n^2}{\omega_n^2+2i\omega \omega_n 
  \zeta_n-\omega^2} \tag{1}$$ 

  where the $n$th mode has amplitude $u_n$ at the bridge, frequency $\omega_n$ 
  and modal damping ratio $\zeta_n$. 

  Skudrzyk was able to deduce from this equation that the height of an isolated 
  modal peak is proportional to $1/\zeta_n$, while the level at the bottom of 
  an antiresonance dip is proportional to $\zeta_n$. It follows that the mean 
  level of a logarithmic plot follows the geometric mean of these two, and is 
  thus independent of damping. If the damping were increased, the peaks and 
  dips would blur out and all admittance curves would tend towards smooth 
  “skeleton” curves representing the logarithmic mean of the original curves, 
  in other words the mean trend in a decibel plot. 

  There is a physically appealing way to visualise the effect of increasing the 
  damping. When a force is applied at a point on the structure, it generates a 
  “direct field” consisting of outward-travelling waves. In time these will 
  reflect from the various boundaries and return. Modal peaks will occur at 
  frequencies where the reflections combine in phase-coherent ways. 
  Antiresonances occur when the sum of reflected waves systematically cancels 
  the original direct field. But at an “average” frequency, where neither of 
  these coherent phase effects occurs, the reflected waves from the various 
  boundaries tend to arrive in random phases and to cancel each other out, 
  leaving the direct field to dominate the response. If damping is increased, 
  the influence of reflections decreases. In the limit of high damping, the 
  desired skeleton of the admittance is given by the direct field alone. 

  The effect is thus the same as if the boundaries had been pushed further away 
  until the system becomes infinitely large. This gives a simple recipe to find 
  skeleton curves for some idealised systems. For a plate-based system, we need 
  to consider an infinite plate with the same material properties and 
  thickness. The vibration of a point-driven infinite plate has a simple 
  closed-form solution. If the plate has thickness $h$, and is made of 
  isotropic material with Young's modulus $E$, Poisson's ratio $\nu$ and 
  density $\rho$, the result is 

  $$Y_\infty^{p} (\omega)=\frac{1}{4h^2} \sqrt{\frac{3(1-\nu^2)}{E \rho}} . 
  \tag{2}$$ 

  We will also look at the behaviour of the banjo, so it is useful to note the 
  corresponding result for a membrane. But this case has a snag: if a point 
  force is applied to an ideal membrane, the linear governing equation predicts 
  that the response will be infinite. To avoid this issue, the force must be 
  applied through a finite footprint. If an infinite membrane with tension $T$ 
  and mass per unit area $\sigma$ is driven through a circular region of radius 
  $a$, the admittance is 

  $$Y_\infty^{m} (a,\omega)=\frac{ic}{2 \pi a T} \frac{H^{(2)}_0 (\omega 
  a/c)}{H^{(2)}_1 (\omega a/c)} \tag{3}$$ 

  where $c=\sqrt{T/\sigma}$ is the wave speed and $H^{(2)}_N$ is the Hankel 
  function of order $N$, describing outgoing waves. 

  \sectionreferences{}[1] E. Skudrzyk: The mean-value method of predicting the 
  dynamic response of complex vibrators. Journal of the Acoustical Society of 
  America \textbf{67}, 1105–1135 (1980). 