  A simple model of the body of an instrument like a guitar, allowing for a 
  Helmholtz resonance and a single mode of the top plate, was described in the 
  main text of section 4.2. A labelled version of Fig.\ 6 from that section is 
  shown in Fig.\ 1. A piston of mass $M$ and area $A$ (representing the body 
  mode) is supported by a spring of stiffness $K$, and is enclosed by a rigid 
  box of volume $V$ which has a single opening in the form of a neck of area 
  $S$ and effective length $L$ (including any end corrections). The ``Helmholtz 
  piston'' has mass $m=\rho_0 SL$ and outward displacement $Y$, while the body 
  mode piston has outward displacement $X$. 

  \fig{figs/fig-f999f515.png}{\caption{Figure 1. The two-piston model}} 

  The internal volume changes by $\delta V = AZ + SY$, resulting in a change of 
  density 

  $$\rho' = -\rho_0 \dfrac{\delta V}{V}=-\dfrac{\rho_0}{V} (AX+SY) . \tag{1}$$ 

  This produces a change of pressure 

  $$p' = c^2 \rho' =- \dfrac{c^2 \rho_0}{V} (AX+SY)=- \dfrac{c^2 \rho_0}{V} 
  (x-y) \tag{12}$$ 

  where we have made a change of variables 

  $$x=AX,\mathrm{~~~~~~}y=-SY . \tag{3}$$ 

  The outward force on the body mode piston is $Ap'$, and the force on the 
  Helmholtz piston is $Sp'$. We can now write Newton's law for the two masses: 

  $$M \ddot{X}=-KX -- \dfrac{c^2 \rho_0}{V} (x-y)A \tag{4}$$ 

  and 

  $$m \ddot{Y}=- \dfrac{c^2 \rho_0}{V} (x-y)S \tag{5}$$ 

  so that 

  $$M \dfrac{\ddot{x}}{A^2}=-K\dfrac{x}{A^2} -- \dfrac{c^2 \rho_0}{V} (x-y) 
  \tag{6}$$ 

  and 

  $$m \dfrac{\ddot{y}}{S^2}=- \dfrac{c^2 \rho_0}{V} (x-y) . \tag{7}$$ 

  If we now define $k_a=K/A^2$, $k_b=c^2 \rho_0/V$, $m_a=M/A^2$ and $m_b=\rho 
  SL/S^2 =\rho L/S$, these equations become 

  $$m_a \ddot{x}=-k_a x -k_b(x-y) \tag{8}$$ 

  and 

  $$m_b \ddot{y}=k_b (x-y) \tag{9}$$ 

  which are the equations governing the equivalent mass-spring system shown in 
  Fig.\ 2. 

  \fig{figs/fig-feb17592.png}{\caption{Figure 2. Equivalent mass-spring version 
  of the model in Fig. 1.}} 

  For a vibration mode, we assume as usual that $x(t)$ and $y(t)$ vary 
  proportional to $e^{i \omega t}$. We can then rearrange eqs. (8,9) to read 

  $$\dfrac{y}{x}=\dfrac{k_a+k_b-m_a \omega^2}{k_b} = \dfrac{k_b}{k_b-m_b 
  \omega^2} . \tag{10}$$ 

  The natural frequencies are thus the solutions of 

  $$\left[ k_a+k_b-m_a \omega^2 \right] \left[ k_b-m_b \omega^2 \right] = k_b^2 
  . \tag{11}$$ 

  This is a quadratic equation in $\omega^2$, which can be solved by the usual 
  formula to obtain the two natural frequencies. For each of these frequencies 
  we can deduce the ratio of amplitudes $y/x$ from either half of eq. (10), 
  then use eq. (3) to obtain $Y/X$. 

  We can estimate rough numerical values of the various parameters, suitable 
  for the application to a classical guitar body. The soundhole has diameter 
  $a=80$ mm and the body cavity has a volume $V \approx 0.013 \mathrm{m}^2$. 
  The neck length $L$ consists almost entirely of ``end corrections'' in this 
  case, since the thickness of a guitar top plate is only of the order of 3 mm. 
  From section 4.2.1, we deduce $L \approx 1.7 a = 68$ mm. For the body mode, 
  Christensen and Vistisen [1] suggest a piston area $A \approx 0.055 
  \mathrm{m}^2$ and a mass $M \approx 0.08$ kg. We can deduce a value for the 
  stiffness $K$ by assuming a frequency of the plate mode without the air. If 
  we choose a value 150 Hz for this frequency, we find $K \approx 70$ kN/m. 
  This combination of parameters, together with standard values for $\rho_0$ 
  and $c$, lead to natural frequencies 97 Hz and 200 Hz, which are in the 
  typical range for classical guitars. The associated mode shapes give the 
  amplitude ratios used to plot the animation in Fig.\ 8 of section 4.2. The 
  formula from section 4.2.1 gives a rigid-wall Helmholtz resonance frequency 
  129 Hz, which is reasonably close to the range suggested by Christensen and 
  Vistisen [1]. 

  \sectionreferences{}[1] O. Christensen and B. B. Vistisen. Simple model for 
  low-frequency guitar function. Journal of the Acoustical Society of America 
  \textbf{68}, 758–766 (1980). 