  In order to use the digital waveguide simulation approach to study the wolf 
  note, we need a body reflection function that includes the influence of the 
  body vibration. We can deduce an appropriate form from the analysis described 
  in section 5.1.2. We derived the following approximate expression for the 
  reflection coefficient $R$ for an ideal string with impedance $Z_0$ attached 
  to a body with bridge admittance $Y(\omega)$: 

  \begin{equation*}R \approx -1 + 2 Z_0 Y(\omega) \tag{1}\end{equation*} 

  \noindent{}from equation (5) of section 5.1.2. 

  To turn this into an approximate expression for the reflection function, we 
  simply need the inverse Fourier transform. We already know the answer to 
  that: the admittance $Y(\omega)$ is the Fourier transform of the impulse 
  response $g(t)$ of the bridge; the velocity response to a unit impulse (delta 
  function of applied force). So the reflection function $r(t)$ is given by 

  \begin{equation*}r(t) \approx -\delta(t-2\beta L /c) + 2Z_0 g(t-2\beta L /c) 
  \tag{2}\end{equation*} 

  \noindent{}where $2\beta L/c$ is the time delay for a wave on the string to 
  travel to the bridge and back at the wave speed $c$, $L$ being the string 
  length and $\beta$ being the relative position of the bowed point. 

  For the simplest wolf model, we would include just one body mode. We can 
  represent the resonance responsible for the wolf as a mass-spring-dashpot 
  oscillator with mass $m$, stiffness $k$ and dashpot strength $c$. We can 
  deduce the impulse response of such an oscillator from the analysis in 
  section 2.2.7: provided the damping is small, it is given by 

  \begin{equation*}g(t) \approx \dfrac{1}{m} \cos (\omega_b t) e^{-\omega_b 
  \eta_b t/2} \tag{3}\end{equation*} 

  \noindent{}where the resonance frequency is $\omega_b = \sqrt{k/m}$ and the 
  modal loss factor is defined by $\omega_b \eta_b \approx c/m$. It is easy to 
  see that this has the expected form for the velocity response to an impulse: 
  at $t=0$ the impulse provides a unit jump in momentum, so that the velocity 
  immediately after the impulse must be $1/m$. But there is no residual 
  acceleration immediately after the impulse, so the free vibration of the 
  oscillator must be in the cosine phase. 

  The expression (3) relates to an ideal string, so it includes the infinitely 
  sharp delta function. In the spirit of the simple model discussed in section 
  9.2, we can spread this out into a narrow pulse with finite width. The 
  reflection function then looks like the example shown in Fig.\ 1. Now, it is 
  obvious that the tail of this reflection function goes on for ever. Does that 
  mean we have to do convolutions of indefinite length in order to represent 
  the effect? Fortunately, the answer is no: there is a computational trick 
  whereby this infinite tail can be represented exactly while only requiring a 
  few arithmetic operations for each successive time sample. This trick, known 
  as an ``IIR digital filter'', will be described in section 9.5.2. 

  \fig{figs/fig-d8419e6b.png}{\caption{Figure 1. Example of the reflection 
  function from equation (3), incorporating the decaying oscillation of a 
  single body mode.}} 