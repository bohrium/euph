

  We can now return, yet again, to the three stringed instruments whose bridge 
  admittances were plotted in Fig.\ 8 of section 5.1. Here is a repeat of that 
  figure: 

  \fig{figs/fig-d71e4d29.png}{\caption{Figure 1. A repeat of Fig. 8 from 
  section 5.1: scaled bridge admittance magnitude, for three different stringed 
  instruments. Red curve: a violin; purple curve: a classical guitar; blue 
  curve: a banjo. The frequency scale is expressed in semitones, starting from 
  the lowest tuned note of each instrument. Successive octaves are indicated by 
  ticks.}} 

  Before launching into a more detailed interpretation of these curves, we 
  should have a look at the responses of more than one instrument of each type, 
  to get an idea of which features are general, and which are special to a 
  particular instrument. Figures 2 and 3 show results plotted in the same 
  format as Fig.\ 1, for representative selections of violins and guitars, 
  respectively. Because these plots are not mixing different instruments, it is 
  possible here to show absolute scales of frequency and admittance, as well as 
  the non-dimensional scales of Fig.\ 1. 

  \fig{figs/fig-e086871a.png}{\caption{Figure 2. Measured bridge admittance, in 
  the same format as Fig. 1, for 7 violins: two by Stradivari, one by Guarneri 
  `del Gesu', and four by modern makers.}} 

  \fig{figs/fig-cef1796d.png}{\caption{Figure 3. Measured bridge admittance, in 
  the same format as Figs. 1 and 2, for eight acoustic guitars. The set 
  includes classical, flamenco and steel-strung guitars; makers include Fleta, 
  Smallman, Ramirez and Fylde.}} 

  In both cases, some famous makers are included in the mix. Stradivari and 
  Guarneri `del Gesu' are represented among the violins. No maker in the guitar 
  world is quite as famous (or expensive) as these two, but several leading 
  makers are represented. Furthermore, several major styles of guitar 
  construction are included. (For cognoscenti, these styles involve fan-braced, 
  lattice-braced and X-braced instruments.) But the first impression from both 
  plots is that `all violins' are really rather similar, and `all guitars' are 
  also similar. It is something of a relief, though, to see that violins and 
  guitars are significantly and consistently different from each other. 
  Furthermore, the banjo response shown in Fig.\ 1 is significantly different 
  from either. 

  Armed with this knowledge, we can try to describe and understand the features 
  of the three types of instrument response, and explore what this tells us 
  about how the instruments sound. We have already seen, in section 5.1, that 
  any admittance usually shows well-separated modes at low frequency, but that 
  the modal overlap factor progressively increases so that at higher 
  frequencies modes lose their separate identities. 

  This distinction has an important consequence. Each type of instrument has a 
  small number of low-frequency modes, sometimes called `signature modes', that 
  can usually be recognised in every individual instrument. To an extent, an 
  instrument maker can control the shapes and natural frequencies of these 
  modes by variations in material choice and constructional details. At 
  somewhat higher frequencies it is still possible with careful measurements to 
  determine the modes despite moderate overlap, but the shapes are likely to be 
  much less recognisable from one instrument to another. 

  At higher frequencies still, things are even less simple. The modes overlap 
  and blur together, and the variability of mode shapes between instruments is 
  much greater. Although, mathematically speaking, the modes still determine 
  the behaviour, the modal point of view becomes progressively less useful. A 
  maker cannot hope to control everything at higher frequencies, but this does 
  not mean that they cannot control anything. Some instruments show features 
  called formants, that can offer a useful degree of control to the maker. We 
  will return to that idea shortly, but first we will look at some signature 
  modes for guitars and violins. 

  We will start with the guitar. Figure 4 shows a few examples, in the form of 
  holographic images. In terms of the frequency scale of Fig.\ 3, these 
  signature modes are found in the frequency range up to about semitone 30. In 
  all four modes shown here, motion is largely confined to the top plate. The 
  sides of the guitar are sufficiently rigid that there is little movement at 
  the edges. Some modes do, however, involve motion in the back plate, not 
  shown here: acoustic response in the air inside the soundbox provides some 
  coupling to the back. 

  \fig{figs/fig-6681895e.png}{} 

  \fig{figs/fig-5d5cdf3c.png}{} 

  \fig{figs/fig-59e26eb9.png}{} 

  \fig{figs/fig-ea974ea0.png}{} 

  Response of the internal air also provides the explanation of the fact that 
  the first two modes shown here look remarkably similar. Now, there is a very 
  general mathematical property of mode shapes called `orthogonality' (it was 
  mentioned in section 2.2.5), which says that it is not possible for two modes 
  to have shapes that are too similar. When a measurement shows apparently 
  similar shapes like the images here, it always means that something else is 
  participating significantly in the vibration, which has not been captured in 
  the measurement. 

  In this case, we already know what this is from section 4.2. The guitar body 
  is designed to have strong coupling between the lowest mode of the top plate 
  and the Helmholtz resonance of the internal air, `breathing' through the 
  soundhole. The result of this coupling is a pair of modes, each involving 
  significant amounts of both plate motion and air motion. But one of them has 
  these components in the same phase, while the other has them in opposite 
  phase: this difference allows them to satisfy the requirement of 
  orthogonality. The modes were illustrated schematically in the animation in 
  Fig.\ 8 of section 4.2. In actual fact, there are three coupled modes in the 
  guitar body, because the lowest mode of the back plate also couples to the 
  other two, but we haven't shown the third one in Fig.\ 4. 

  The violin also has `signature modes', but they are more complicated. Some 
  examples are shown in Fig.\ 5a--d, and this time we see both the top and back 
  motion for each mode. In terms of the frequency scale of Fig.\ 2, these modes 
  are found at frequencies below about semitone 20: low enough that the 
  Helmholtz number is fairly small, and volume change is an important factor in 
  sound radiation efficiency. Compared to the guitar modes in Fig.\ 4, the 
  violin shows a lot of movement around the edges: the `rib garland' of a 
  violin is made of thin strips of wood (usually maple), and it is very 
  flexible. For all four modes shown here, the node lines (visible as white 
  stripes) run off the edges of the top plate, and reappear in more or less the 
  same places on the back plate. The motion is truly three-dimensional. 

\moobeginvid\begin{tabular}{ccc} \vidframe{ 0.30 }{ vids/vid-5623ae89-00.png }&\vidframe{ 0.30 }{ vids/vid-5623ae89-01.png }&\vidframe{ 0.30 }{ vids/vid-5623ae89-02.png } \end{tabular}\caption{Figure 5a. A measured signature mode of a violin, usually called A0, at 272 Hz. Image copyright George Stoppani, reproduced by permission.}\mooendvideo

\moobeginvid\begin{tabular}{ccc} \vidframe{ 0.30 }{ vids/vid-956e1f26-00.png }&\vidframe{ 0.30 }{ vids/vid-956e1f26-01.png }&\vidframe{ 0.30 }{ vids/vid-956e1f26-02.png } \end{tabular}\caption{Figure 5b. A measured signature mode of a violin, usually called CBR, at 407 Hz. Image copyright George Stoppani, reproduced by permission.}\mooendvideo

\moobeginvid\begin{tabular}{ccc} \vidframe{ 0.30 }{ vids/vid-91e8dbeb-00.png }&\vidframe{ 0.30 }{ vids/vid-91e8dbeb-01.png }&\vidframe{ 0.30 }{ vids/vid-91e8dbeb-02.png } \end{tabular}\caption{Figure 5c. A measured signature mode of a violin, usually called B1-, at 462 Hz. Image copyright George Stoppani, reproduced by permission.}\mooendvideo

\moobeginvid\begin{tabular}{ccc} \vidframe{ 0.30 }{ vids/vid-94a2e206-00.png }&\vidframe{ 0.30 }{ vids/vid-94a2e206-01.png }&\vidframe{ 0.30 }{ vids/vid-94a2e206-02.png } \end{tabular}\caption{Figure 5d. A measured signature mode of a violin, usually called B1+, at 551 Hz. Image copyright George Stoppani, reproduced by permission.}\mooendvideo

  There has been a long history of researchers trying to visualise and 
  understand these modes, and this has resulted in a rather non-intuitive set 
  of labels that are commonly used to describe the modes. The first mode, Fig.\ 
  5a, is usually called `A0'. It is essentially the Helmholtz resonance of the 
  internal air, modified by the flexibility of the plates. The body of the 
  violin is `breathing' in and out. You might immediately think that this must 
  make the mode an efficient radiator of sound, because the volume is changing 
  so the body can behave as a monopole sound source. 

  Well, that conclusion is correct, but not for the obvious reason. As was 
  shown in Fig.\ 8 of section 4.2, the result of coupling a Helmholtz resonance 
  to flexible plates is that this lowest resulting mode has volume variations 
  of the box structure which are in the opposite phase to the volume variations 
  resulting from air flow in and out of the f-holes. The air flow contribution 
  is the bigger of the two, so the mode is indeed a good radiator of sound, but 
  the structural motion seen in the plot serves to reduce the net radiation a 
  little. However, it is very important that there is some structural motion 
  associated with the mode, because otherwise it would not show up in the 
  bridge admittance, and the vibrating strings would not be able to excite it. 

  The second mode, shown in Fig.\ 5b, is usually called `CBR' for historical 
  reasons. For this mode, the top and back plates are moving in very similar 
  ways: the entire body of the violin is vibrating rather like a thick bending 
  plate. The result is that there is very little change in volume, and this 
  mode is not a good radiator of sound. 

  The two modes in Figs.\ 5c,d are `twins': the top plate motion in the first 
  of them is very much like the back plate motion in the second, and vice 
  versa. An evocative name for them both is the `baseball modes': each has a 
  single sinuous nodal line that snakes around the box in a pattern like the 
  seam of a baseball. However, they are more commonly labelled `B1-` and `B1+'. 
  Both modes have significant volume variation, and both are excellent 
  radiators of sound. Both are strongly influenced by the presence of the 
  soundpost inside the violin body, producing asymmetry and coupling the top 
  and back plates together: see the discussion in section 5.1. 

  Until recently, no very persuasive description had been given of why the low 
  modes of a violin body should take these particular forms, or how an 
  instrument maker might be able to influence the details (and in particular 
  the sound radiation from them). However, an extensive series of explorations 
  by Gough using finite element (FE) analysis has given some answers to these 
  questions [1]. He explored a series of models starting from very simple 
  assumptions, then gradually adding in the complications of violin design one 
  at a time so that the progressive evolution and emergence of the signature 
  modes can be charted, and the relative influence of various contributory 
  factors assessed. The full picture revealed by Gough’s work is too 
  complicated to include here: an abbreviated summary showing some key aspects 
  can be found in [2]. 

  We will move away from signature modes now, and look higher up the frequency 
  range. The banjo hasn't been forgotten, but a discussion of that instrument 
  is best left until we have introduced an important new idea. The concept of a 
  formant has come from the study of the human voice, whether speaking or 
  singing. 

  It is simplest to think about singing. If you sing a steady note, there are 
  two choices you can make: the pitch of the note, and which vowel sound you 
  want the listener to hear. You set the pitch by the frequency of oscillation 
  of your vocal chords, but how is one vowel distinguished from another? You 
  know what you have to do in practice: you have to position your tongue and 
  lips in a particular configuration, different for each vowel. The sound is 
  made by your vocal chords, but it has to pass through your vocal tract before 
  it can emerge from your mouth as external sound waves. The vocal tract has 
  resonances, like any other acoustical duct (see section 4.2). The effect of 
  changing the positions of your tongue and lips is to shift these resonances 
  around. Because the walls of the vocal tract are made of soft flesh, the 
  resonances have quite high damping, so their half-power bandwidth is quite 
  large. 

  The result is a frequency spectrum of the vocal tract filter which will look 
  a bit like the schematic version shown as a dashed line in Fig.\ 6. The broad 
  resonances produce formants. The animation in this figure shows what happens 
  when the singer performs a chromatic scale. The individual harmonics change 
  with every note, but the heights of the peaks always mark out (to a greater 
  or lesser extent) the positions of the two formants included in this plot. 
  Your brain makes use of that information to recognise that the same vowel is 
  being sung at all these different pitches. The full story of vowel sounds, 
  and speech perception in general, is too complicated to go into detail here; 
  to learn more, look at the \tt{}Wikipedia article\rm{}. 

\moobeginvid\begin{tabular}{ccc} \vidframe{ 0.30 }{ vids/vid-50ed3f9e-00.png }&\vidframe{ 0.30 }{ vids/vid-50ed3f9e-01.png }&\vidframe{ 0.30 }{ vids/vid-50ed3f9e-02.png } \end{tabular}\caption{Figure 6.  Schematic example of formants. The red lines mark the harmonics produced by a singer, performing a one-octave chromatic scale starting at $G\_3$ (196 Hz). The dashed line shows a schematic version of the frequency response of the vocal tract, in a configuration corresponding to a particular vowel. As the pitch of the note changes, the filter shape is still `dotted out' by the heights of the harmonic peaks, so that your brain is able to perceive which vowel is being sung.}\mooendvideo

  We now want to apply this idea of formants to musical instruments. The 
  interpretation is slightly different in that context. The example in Fig.\ 6 
  shows a series of peaks, which sketch out the shape of a larger-scale 
  structure associated with the formant filter. In the application to stringed 
  instruments, a somewhat similar structure is sometimes seen, but now it 
  involves many peaks from the resonant behaviour of the body, whose heights 
  are systematically modulated to mark out a larger-scale structure. 

  The example that has been most thoroughly studied relates to the violin. It 
  produces a feature usually called the bridge hill [3,4]. We have already seen 
  the effect of this hill, without really being aware of it. It is directly 
  responsible for the difference between violins and guitars seen in Figs.\ 2 
  and 3, in the frequency range around semitones 24--48. The feature we are 
  interested in becomes much more obvious if we plot the phase as well as the 
  magnitude of the bridge admittance. An example, for a typical violin, is 
  shown in Fig.\ 7. The upper plot shows a magnitude plot that is by now quite 
  familiar. The `hill' feature appears in the frequency range indicated by the 
  horizontal dashed line. The lower plot shows the phase response, and this 
  does something very dramatic in the same frequency range: the phase drops 
  systematically towards $-90^\circ$. 

  \fig{figs/fig-b57c853e.png}{\caption{Figure 7. The bridge admittance of a 
  typical violin, this time showing the phase (lower plot) as well as the 
  magnitude (upper plot).}} 

  To understand what is going on, it helps to look at a simpler system. Instead 
  of the complicated violin body, we can imagine a system which has regularly 
  spaced resonance frequencies, all with the same damping factor (or Q-factor), 
  and all having the same modal amplitude at the point where we choose to drive 
  the system. We can calculate the admittance of this simple system using the 
  formula in eq. (12) of section 2.2.7: for a particular choice of parameter 
  values the result is shown in the left-hand plot in Fig.\ 8. This 
  super-simple system has no eye-catching and distracting features, so that the 
  effect of the next stage will be very clear. 

  \fig{figs/fig-7da0b1c3.png}{} 

  \fig{figs/fig-cf8d0029.png}{} 

  We now choose to drive this simple system in a different way. Instead of 
  applying a force directly, we drive it through a mass-spring resonator system 
  as sketched in Fig.\ 9. As explained in the next link, the modified 
  admittance can be calculated in terms of that of the original system, plus 
  the assumed values of the stiffness $k$ and the mass $m$. An example of the 
  result is shown in the right-hand plots of Fig.\ 8. It should be immediately 
  clear that this shows a strong resemblance to the violin response in Fig.\ 7: 
  the amplitude shows a `hill' where the level has been lifted by some 20 dB, 
  while the phase plot shows the dramatic drop towards $-90^\circ$ that we 
  noted earlier. The formant-like nature of the amplitude plot should also be 
  obvious. 

  \fig{figs/fig-3473d4d3.png}{\caption{Figure 9. Schematic sketch of an 
  instrument body with admittance $Y\_v(\omega)$, being driven through a simple 
  mass-spring oscillator.}} 

  It is useful to find a way to see the shape of the underlying formant `hill' 
  without the distracting details of the individual body modes. This can be 
  done in a very simple way by using an approach known as Skudrzyk’s mean value 
  method [5]: the details are described in the next link. The essence of the 
  argument is to distinguish the direct field of vibration radiating out from 
  where the force is applied, and the reverberant field made up of all the 
  reflected waves returning from the boundaries, and then bouncing around until 
  they die away from the effects of damping. Skudrzyk showed that the mean 
  trend of a decibel plot like the ones we have been looking at corresponds to 
  the direct field only: all the detail of modal peaks is the result of adding 
  in the reverberant field. The dashed lines in Fig.\ 8 were calculated this 
  way, from the direct field. They follow the mean trend of the decibel plots 
  of amplitude very well, and they also catch the phase behaviour convincingly. 

  The argument behind Skudrzyk's method can also be used with measured data. 
  The phase of the reverberant response will vary in an irregular way with 
  frequency, as the interference effects between different wave paths change, 
  and it may therefore be reasonable to expect that the reverberant field will 
  average to zero over a frequency band of sufficient width to contain several 
  modes of the system [6]. Computing a suitable moving average of a measured 
  admittance should therefore remove the reverberant field and leave the direct 
  field, which is what we want to give the Skudrzyk estimate of the trend. 

  The result, applied to the admittance from Fig.\ 7, is shown as the blue 
  dashed line in Fig.\ 10. The agreement is excellent. Now, you might think 
  this is not surprising: we have used an averaging process, and got a result 
  which predicts an average trend. But we have computed a linear average of the 
  complex admittance, which will include a lot of phase cancellation, whereas 
  the trend we have successfully matched is the logarithmic mean of the 
  magnitude (because it is a decibel plot). Really, this agreement is quite 
  surprising, even if it was predicted by the theory! 

  \fig{figs/fig-18e7981f.png}{\caption{Figure 10. The violin admittance from 
  Fig. 7, with blue dashed lines added to show the result of a linear average 
  with bandwidth 1000 Hz and a Hanning window, to estimate the direct field 
  contribution.}} 

  So the `bridge hill' in the violin seems to be the result of driving the body 
  through some kind of resonant system. But when you look at a violin you see 
  no obvious sign of the mass and spring from Fig.\ 9. There is a kind of 
  resonator, but it is not immediately apparent: as you might guess from the 
  name, the violin bridge plays a central role. Figure 11 shows the `island' 
  area of a violin: the bridge, with its curious-shaped cut-outs, sits in the 
  rather flexible region of the top plate between the f-holes. The strings sit 
  on the top curve of the bridge, and when a string is bowed the force applied 
  to the bridge is predominantly in the side-to-side direction. 

  \fig{figs/fig-01223da7.png}{\caption{Figure 11. The `island' area of a 
  violin. Out of sight inside the body, the bassbar passes underneath the 
  left-hand bridge foot, and the sound post is close to the right-hand foot.}} 

  The bridge tends to rock in response to this force. Now, if the bridge was 
  removed from the violin and its feet clamped in a vice, a typical violin 
  bridge turns out to have a resonance somewhere around 3 kHz, in which it 
  bends at the waist in a rocking motion, as sketched in the left-hand plot of 
  Fig.\ 12. The right-hand plot shows a simple mass-spring model of a bridge, 
  representing this rocking motion in a form that makes it look a little more 
  like Fig.\ 9. This bridge resonance was first described by Reinicke, back in 
  the 1970s [7]. 

  \fig{figs/fig-9cb109f6.png}{} 

  \fig{figs/fig-bf5b778f.png}{} 

  As Reinicke was already aware, when the bridge is on the violin the 
  corresponding rocking resonance is usually a lot lower in frequency, because 
  the top plate of the instrument is more flexible than the bridge itself. This 
  is the resonance that is responsible for the bridge hill, and it is the 
  reason that the bridge admittance of a violin is some 20 dB higher than the 
  corresponding admittance of a guitar in the low kHz range. It is governed by 
  the mass of the bridge, particularly the top part above the waist, and by a 
  combination of two stiffnesses. One is associated with the flexibility of the 
  bridge itself, the other with the rotational flexibility of the `island', 
  when driven by the pair of forces through the bridge feet. These two 
  stiffnesses are connected `in series', so that whichever is the lower of the 
  two has the dominant effect: usually, this is the stiffness of the `island'. 

  Violin makers are very well aware that bridge adjustment is an important 
  resource for tonal tweaking of a violin. They will think carefully about 
  choosing a bridge blank with the right material and foot spacing, and then 
  about the final thickness and the details of the cut-outs. What they are 
  doing, mainly, is adjusting the bridge hill. The model outlined here can be 
  used to understand the influence on the hill of the different parameters --- 
  see reference [4] for more detail. 

  One particular `bridge adjustment' is very familiar to all violinists: when a 
  player fits a mute, they are simply increasing the mass $m$ in Fig.\ 12 and 
  thus lowering the frequency of the bridge hill. This results in what a 
  scientist would call a low-pass filtering effect: the roll-off of response 
  above the hill, seen in Figs.\ 2 and 7, starts sooner so that the 
  high-frequency content of the violin sound is reduced. The sound might then 
  be described as more mellow, less strident and penetrating. 

  So now, finally, we have enough information to return to the three-way 
  comparison of the violin, guitar and banjo. The reason for choosing these 
  three instruments will now become clear: they exhibit three contrasting 
  balances between the effects of signature modes and formants. Figure 13 shows 
  a final version of the comparison of the scaled admittances: this time, the 
  three averaged curves have been added. 

  \fig{figs/fig-a71f0d40.png}{\caption{Figure 13. A final comparison of the 
  admittances of the three instruments from Fig. 1, this time including the 
  result of averaging as in Fig. 10. Blue: banjo; red: violin; black: guitar.}} 

  \textbf{The violin (and cello)} 

  The violin (red curves) has just been discussed. At low frequency it has the 
  set of signature modes shown in Fig.\ 5. The air resonance A0 shows as a 
  small peak around semitone 5 (270 Hz), and the three modes CBR, B1- and B1+ 
  show as a cluster of high peaks in the semitone range 12--20 (corresponding 
  to a frequency range around 400--700 Hz). After that, modal overlap increases 
  steadily, and the only other clear thing visible in these curves is the 
  bridge hill peaking around semitone 40 (2.3 kHz), as just discussed. 

  This is not quite the end of the story, though. Although it is not visible in 
  the particular plots shown here, there is often a second `hill' in a typical 
  violin, in a higher frequency range. This one is associated with `bouncing' 
  motion of the bridge and island, rather than rocking: Fig.\ 14 shows a 
  sketch. 

  \fig{figs/fig-3389a198.png}{\caption{Figure 14. A violin bridge,with a sketch 
  of the `bouncing' resonance, occurring around 6 kHz when the feet are rigidly 
  clamped.}} 

  It is worth inserting a brief comment here about the behaviour of the cello. 
  The cello, with its very different bridge design, has been studied far less 
  than the violin, but Reinicke included it in his study, and found three 
  resonances. As well as `bending' and `bouncing', like the violin bridge, it 
  has a lower-frequency resonance involving `swaying' on the long legs. 
  Sketches of these modes are shown in Fig.\ 15. 

  \fig{figs/fig-b3014f0f.png}{} 

  \fig{figs/fig-26dbec61.png}{} 

  \fig{figs/fig-bd683661.png}{} 

  These resonances of a cello bridge also seem to be associated with formants 
  in the bridge admittance. An illustration is shown in Fig.\ 16, in a similar 
  format to Fig.\ 13. Results are shown for two cellos, and both show clear 
  evidence of two ``hills'' in this frequency range: one around 1 kHz and the 
  other around 2--2.3 kHz. It seems likely that these are associated with the 
  first two bridge resonances sketched in Fig.\ 15, modified by the behaviour 
  of the top plate in the island area as we saw for the violin. 

  \fig{figs/fig-1a4b33f9.png}{\caption{Figure 16. Bridge admittance of two 
  cellos, scaled by the impedance of the top string as in earlier figures. The 
  average curves for both are shown as dashed lines.}} 

  Whether or not they think in these terms, violin and cello makers manipulate 
  both the signature modes and the bridge hill formant(s). Both play important 
  roles in aspects of the sound quality and `playability' of an instrument. 

  \textbf{The guitar} 

  We now return to a discussion of Fig.\ 13. The guitar admittance shows two 
  large peaks corresponding to the first two signature modes shown in Fig.\ 4. 
  The first of these is the equivalent of mode A0 in the violin, but air motion 
  is coupled more strongly to plate motion in the guitar, which is why the peak 
  is so much more prominent in the bridge admittance. Relative to the tuning of 
  the instrument, this mode is placed lower than in the violin: around semitone 
  3 (i.e. around the note $G_2$ near 100 Hz). The next mode is roughly an 
  octave higher, at a very similar frequency (relative to the tuning) to the 
  mode CBR in the violin. Indeed, the guitar and the violin both show a cluster 
  of three strong modes in very much the same range on this semitone scale. 

  After that, the guitar diverges strongly from the violin. The average curve 
  is almost featureless: it simply trends gently downwards. This more-or-less 
  flat trend line is what would be expected for bending vibration of any flat 
  plate, based on Skudrzyk's method. Recall that the Skudrzyk prediction of the 
  mean trend is the admittance of an infinitely-extended system: in this case, 
  an infinite plate. That is a system with a simple mathematical solution: the 
  response of an infinite plate to a point force is exactly the same as an 
  ideal mechanical resistance (or dashpot): the admittance is not complex but 
  is a real number, and it is independent of frequency. So the Skudrzyk mean 
  prediction would have a constant amplitude, and a constant phase of 
  $0^\circ$: like the left-hand plot of Fig.\ 8, in fact. 

  The averaged line for the guitar in Fig.\ 13 may just possibly show a weak 
  hill-like feature peaking around semitone 52 (1.7 kHz), but it would need a 
  careful and systematic study to find out whether this is a consistent 
  feature, and whether it has audible consequences. 

  But in general terms the description of the guitar bridge admittance we have 
  just given is in accord with the received wisdom among guitar researchers. 
  Most research has been devoted to understanding individual modes at low 
  frequency, and their associated sound radiation characteristics. There is no 
  equivalent in the guitar world of the research literature devoted to the 
  bridge hill in the violin. 

  \textbf{The banjo} 

  Like the other instruments, the banjo has well-defined modes at low 
  frequency, of course. We already know what they look like: they are 
  essentially the same as the mode shapes of the drum, shown back in section 
  2.2 and examined in detail in section 3.6.1. Adding the bridge and strings to 
  the banjo membrane changes the frequencies, and the asymmetric position of 
  the bridge means that the degenerate pairs of modes in the drum are separated 
  to significantly different frequencies. But the mode shapes remain very 
  similar to those of the drum: measured versions of a few modes of a 
  fully-strung banjo are shown in Figs.\ 17--19. Note that these laser 
  vibrometer measurements were made with the resonator back of the banjo 
  removed in order to access both sides of the membrane. This changes the 
  frequencies a little, as discussed earlier. 

\moobeginvid\begin{tabular}{ccc} \vidframe{ 0.30 }{ vids/vid-294cff97-00.png }&\vidframe{ 0.30 }{ vids/vid-294cff97-01.png }&\vidframe{ 0.30 }{ vids/vid-294cff97-02.png } \end{tabular}\caption{Figure 17. A mode (or, strictly, an operating deflection shape) of the banjo, at 297 Hz}\mooendvideo

\moobeginvid\begin{tabular}{ccc} \vidframe{ 0.30 }{ vids/vid-5600e0ad-00.png }&\vidframe{ 0.30 }{ vids/vid-5600e0ad-01.png }&\vidframe{ 0.30 }{ vids/vid-5600e0ad-02.png } \end{tabular}\caption{Figure 18. A mode (or, strictly, an operating deflection shape) of the banjo, at 500 Hz}\mooendvideo

\moobeginvid\begin{tabular}{ccc} \vidframe{ 0.30 }{ vids/vid-25e51848-00.png }&\vidframe{ 0.30 }{ vids/vid-25e51848-01.png }&\vidframe{ 0.30 }{ vids/vid-25e51848-02.png } \end{tabular}\caption{Figure 19. A mode (or, strictly, an operating deflection shape) of the banjo, at 716 Hz}\mooendvideo

  But Fig.\ 13 reveals something else: the averaged curve for the banjo shows a 
  large formant-like feature, peaking around semitone 33 (1 kHz). It also 
  suggests another similar (although smaller) feature around semitone 55 (3.5 
  kHz): we will return to this in a moment. But first, we need to understand 
  the low-frequency formant, which dominates the bridge admittance of the 
  banjo: even the `signature modes' fall within the range of influence of this 
  formant. 

  The formant is determined by an effective mass and stiffness acting at the 
  bridge, just as in the violin bridge hill, and the idealised case in Fig.\ 9. 
  The mass is clear enough: it is the mass of the bridge. The bridge of this 
  particular banjo weighs only 2.2 g, but the Mylar membrane weighs only about 
  18 g so the added mass of the bridge is quite a significant fraction of that. 
  But it is not at all obvious where the stiffness comes from. 

  It turns out that there are two sources of stiffness that are relevant, both 
  of them requiring rather careful investigation to pin down (see [8,9] if you 
  really want to know). One is associated with the fact that a membrane has a 
  lot more modes than a plate, as was shown in section 4.2.4: the modal density 
  of a membrane, on average, increases proportional to frequency. For a plate, 
  we showed in section 3.2.4 that the modal density was on average constant. 
  The growing modal density of a membrane means that the combined effect of all 
  the high-frequency modes adds up to something that has a significant effect 
  on stiffness at low frequency. 

  The second source of stiffness is associated with the presence of the 
  strings, and comes partly from the stiffness of the strings along their 
  lengths. When the bridge moves in response to string vibration, the geometric 
  configuration of the strings and tailpiece, with a break angle of about 
  $13^\circ$ over the bridge, means that the strings must be stretched a 
  little. The banjo has metal strings, and the stiffness involved in this 
  stretching is enough to matter. It acts just like a spring attached to the 
  bridge. There is a related contribution to stiffness from the in-plane 
  stiffness of the membrane itself: the downbearing force of the strings means 
  that there is an effective `break angle' where the membrane deforms around 
  the bridge feet. 

  These effects, the bridge mass and the various sources of stiffness, combine 
  to produce something like the schematic system shown in Fig.\ 9. The result 
  is the formant we saw in Fig.\ 13, behaving much like the violin bridge hill 
  but acting at significantly lower frequency. In section 5.5 we will be able 
  to listen to a collection of synthesised sound examples based on a computer 
  model of a banjo. This will allow us to hear the effect of changing various 
  parameters, like the bridge mass or the head tension. Those sounds give a 
  strong suggestion that any parameter change which affects the formant has a 
  clearly audible consequence, whereas a change which shifts the individual 
  resonances but leaves the formant in the same place gives only a rather 
  subtle change of sound. 

  We won't be sure until some serious psychoacoustical studies have been done 
  (see chapter ? for some discussion of what that would involve), but if this 
  speculation turns out to be right, the banjo gives a rather extreme example 
  of the interplay between signature modes and formants. Individual body modes 
  seem to have rather little perceptual significance, and the sound of the 
  instrument is dominated by the effects of at least one formant. This is the 
  opposite extreme to the guitar, where there don't seem to be any interesting 
  formants, and signature modes are the only game in town. 

  I say ``at least one formant'' above, because we already noted the appearance 
  of a second, admittedly smaller, formant-like peak in the banjo admittance. 
  There is also a third, at even higher frequency. These additional formants 
  show more prominently if admittance measurements are made at different 
  positions on the banjo bridge. Figure 20 shows a comparison of three 
  different measurements on the same banjo, plotted over a wider frequency 
  range with one additional octave. The red curves show the admittance near the 
  top string, as in the case we have been looking at so far. The black curves 
  show the result of measuring at the bridge centre, near the 3rd string. The 
  blue curves show the admittance measured in the perpendicular direction: 
  tapping and measuring horizontally on the top corner of the bridge, to give 
  the admittance that would be relevant to string vibration in the plane 
  parallel to the banjo head. 

  \fig{figs/fig-fdacbb39.png}{\caption{Figure 20. Scaled admittance measured at 
  three different positions on the banjo bridge, with averaged curves shown in 
  dashes. Red: near the top string as in previous plots; black: near the middle 
  string; blue: horizontally on the side of the bridge.}} 

  The three measurements are very different from each other. The second peak we 
  noted in the previous measurement appears far more strongly in the blue 
  curves. The third formant shows up most prominently in the black curve. To 
  track down what has caused these new formants required a detailed computer 
  model of the banjo and its bridge: for details, see [9]. The bridge of this 
  particular banjo has a typical three-legged shape, shown in Fig.\ 21. Both 
  high-frequency formants turned out to be associated with resonances of this 
  bridge, modified by the behaviour of the banjo membrane that it is sitting 
  on. The lower of the two, around semitone 52 in Fig.\ 17, involves bending of 
  the three legs so that the feet rotate (see Fig.\ 22); the higher one, around 
  semitone 62, involves a bending resonance of the bridge rather like the first 
  mode of a free-free beam as seen in Fig.\ 1 of section 3.2 (see Fig.\ 23). 

  \fig{figs/fig-216aba9d.png}{\caption{Figure 21. A close-up of the banjo 
  bridge. The white patches are reflective tape, used for the laser vibrometer 
  measurements.}} 

  \fig{figs/fig-e4a96241.png}{\caption{Figure 22. Image from the computer model 
  showing the predicted motion of the banjo membrane and bridge at the peak of 
  the lower ``bridge hill'' around 3 kHz. Image copyright Hossein Mansour, 
  reproduced by permission.}} 

  \fig{figs/fig-eed5eee2.png}{\caption{Figure 23. Image from the computer model 
  showing the predicted motion of the banjo membrane and bridge at the peak of 
  the higher ``bridge hill'' around 5 kHz. Image copyright Hossein Mansour, 
  reproduced by permission.}} 

  In section 5.5, among the various sound examples will be some that illustrate 
  the audible effect of these high-frequency formants. But before we are ready 
  for that, we need to know how to add strings to the instrument body so that 
  we can synthesise sounds based on an accurate representation of the physics 
  of the instrument. 



  \sectionreferences{}[1] C. E. Gough. A violin shell model: vibrational modes 
  and acoustics. Journal of the Acoustical Society of America \textbf{137}, 
  1210–1225 (2015). 

  [2] J. Woodhouse. The acoustics of the violin: a review.~ Reports on Progress 
  in Physics \textbf{77}, 115901.~ DOI 10.1088/0034-4885/77/11/115901 (2014). 

  [3] E. V. Jansson and B. K. Niewczyk: On the acoustics of the violin: bridge 
  or body hill? Journal of the Catgut Acoustical Society, Series 2, \textbf{3 } 
  23–27 (1999). 

  [4] J. Woodhouse. On the ``bridge hill'' of the violin.~ Acta Acustica united 
  with Acustica \textbf{91}, 155–165 (2005). 

  [5] E. Skudrzyk: The mean-value method of predicting the dynamic response of 
  complex vibrators. Journal of the Acoustical Society of America \textbf{67}, 
  1105–1135 (1980). 

  [6] J. Woodhouse and R. S. Langley. Interpreting the input admittance of 
  violins and guitars.~ Acta Acustica united with Acustica \textbf{98}, 
  611-628.~ DOI 10.3813/AAA.918542 (2012). 

  [7] W. Reinicke. Die Uebertragungseigenschaften des Streichinstrumentenstegs. 
  Doctoral dissertation, Technical University of Berlin (1973). 

  [8] Jim Woodhouse, David Politzer and Hossein Mansour. ``Acoustics of the 
  banjo: measurements and sound synthesis'', Acta Acustica \textbf{5}, 15 
  (2021). The article is available here: 
  \tt{}https://doi.org/10.1051/aacus/2021009\rm{} 